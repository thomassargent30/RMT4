
\input grafinp3
%\input grafinput8
\input psfig
%\eqnotracetrue

%\showchaptIDtrue
%\def\@chaptID{18.}

%\hbox{}

\def\toone{{t+1}}
\def\ttwo{{t+2}}
\def\tthree{{t+3}}
\def\Tone{{T+1}}
\def\TTT{{T-1}}
\def\rtr{{\rm tr}}
\footnum=0
\chapter{Credit and Currency\label{townsend}}
\index{credit and currency}%
\section{Credit and currency with long-lived agents}

This chapter describes Townsend's (1980) turnpike model of money
and puts it to work.  The model uses a particular pattern of heterogeneity
of endowments and locations to create a demand for currency.
The model is more primitive than the shopping time model of chapter
\use{fiscalmonetary}.  As with the overlapping generations model,
the turnpike model starts from a setting in which diverse intertemporal
endowment patterns across agents prompt borrowing and lending.  If
something prevents loan markets from operating, it is possible that
an unbacked currency can play a role in helping agents smooth their
consumption over time.  Following Townsend, we shall eventually appeal
to locational heterogeneity as the force that causes loan markets to
fail in this way.

\auth{Bewley, Truman}  \auth{Townsend, Robert M.}
The turnpike  model can be viewed as a simplified version
of the stochastic model proposed by Truman Bewley (1980).   We
use the model to study a number of interrelated issues and theories,
including (1) a \idx{permanent income model} of consumption, (2) a
Ricardian doctrine \index{Ricardian proposition}%
that government borrowing and taxes have
equivalent economic effects, (3) some restrictions on the operation
of private loan markets needed in order that unbacked currency be
valued, (4) a theory of inflationary finance, (5) a theory of
the optimal inflation rate and the optimal behavior of the
currency stock over time, (6) a ``legal restrictions'' theory
of inflationary finance, and (7) a theory of exchange rate
indeterminacy.\NFootnote{Some of the analysis in this chapter
follows Manuelli and Sargent (2010).  Also see Chatterjee and
Corbae (1996) and Ireland (1994) for analyses of policies within
a turnpike environment.}
\auth{Manuelli, Rodolfo}
\auth{Sargent, Thomas J.}
\auth{Ireland, Peter N.}
\auth{Corbae, Dean}
\auth{Chatterjee, Satyajit}

%\vfill\eject
\section{Preferences and endowments}

There is one consumption good.  It cannot be produced or stored.
The total amount of goods available each period
is constant at $N$.  There are $2N$ households, divided
into equal numbers $N$ of two types, according to their endowment
sequences.
   The two types of households, dubbed {\it odd} and {\it even}, have endowment sequences
$$ \eqalign{ \{y_t^o\}_{t=0}^{\infty} &= \{ 1, 0 , 1, 0, \ldots \}, \cr
             \{y_t^e\}_{t=0}^{\infty} &= \{ 0, 1, 0, 1, \ldots \}. \cr} $$
Households of both types order consumption sequences
$\{c_t^h\}$ according to the common utility function
$$ U = \sum_{t=0}^\infty \beta^t u(c_t^h),$$
where $\beta \in (0,1)$, and $u(\cdot)$ is twice continuously
differentiable, increasing, and strictly
concave, and satisfies
$$ \lim_{c \downarrow 0} u'(c) = +\infty.             \EQN turn_inada
$$

\section{Complete markets}

As a benchmark, we study a version of the economy with complete markets.
Later, we shall more or less arbitrarily shut down many
of the markets  to make room for money.

\subsection{A Pareto problem}

Consider the following \idx{Pareto problem}:  Let $\theta \in [0,1]$
be a weight indexing how much a social planner likes
odd agents.  The problem is to choose consumption
sequences $\{c_t^o, c_t^e\}_{t=0}^\infty$ to maximize
$$ \theta \sum_{t=0}^\infty \beta^t u(c_t^o) +
  (1-\theta) \sum_{t=0}^\infty \beta^t u(c_t^e), \EQN od1$$
subject to
$$ c_t^e + c_t^o = 1, \quad t \geq 0.\EQN od2$$
The first-order conditions are
$$ \theta u'(c_t^o) - (1-\theta) u'(c_t^e) = 0.$$
Substituting the constraint \Ep{od2} into this first-order
condition and rearranging gives the condition
$$ { u'(c_t^o)\over u'(1-c_t^o)} = {1 - \theta \over \theta}.\EQN od3$$
Since the right side is independent of time, the left must be also, so that condition \Ep{od3} determines the one-parameter family
of
optimal allocations
$$ c_t^o = c^o (\theta), \quad c_t^e = 1-c^o(\theta).$$

\index{complete markets}
\subsection{A complete markets equilibrium}
A household takes the price sequence $\{ q_t^0\}$ as given
and chooses a consumption sequence to maximize
$\sum_{t=0}^\infty \beta^t u(c_t)$ subject to
the budget constraint
$$ \sum_{t=0}^\infty q_t^0 c_t \leq \sum_{t=0}^\infty q_t^0 y_t .$$
The household's Lagrangian is
$$ L = \sum_{t=0}^\infty \beta^t u(c_t)
+ \mu  \sum_{t=0}^\infty q_t^0 (y_t - c_t) ,$$
where $\mu$ is a nonnegative Lagrange multiplier.
The first-order conditions for the household's problem
are
$$ \beta^t u'(c_t) \leq \mu q_t^0 , \quad = \ {\rm if} \ c_t > 0 .$$
\medskip
\noindent{\sc Definition 1:}  A {\it competitive equilibrium}
is a price sequence $\{q_t^o\}_{t=0}^\infty$ and an allocation
$\{c_t^o, c_t^e\}_{t=0}^\infty$ that have the property that
(a) given the price sequence, the allocation solves
the optimum problem of households of both types; and (b) $c^o_t+c^e_t=1$
for all $t\geq 0$.
\index{competitive equilibrium}

\medskip
To find an equilibrium, we have to produce an allocation and a price system
for which we can verify that the first-order conditions of both households
are satisfied.  We start with a guess inspired by the constant-consumption
property of the \idx{Pareto optimal allocation}.  We guess that
$c_t^o = c^o, c_t^e = c^e \ \forall t,$
where $c^e+c^o =1$.  This guess and the first-order condition for the
odd agents imply
$$ q_t^0 = {\beta^t  u'(c^o) \over \mu^o},$$
or
$$ q_t^0 = q^0_0 \beta^t, \EQN od4$$
where we are free to normalize by setting $q^0_0 = 1.$
For odd agents, the right side of the budget constraint
evaluated at the prices given in equation \Ep{od4} is then
$$ {1 \over 1 - \beta^2},$$
and for even households it is
$$ {\beta \over 1 - \beta^2}. $$
The left side of the budget constraint evaluated
at these prices is
$$ {c^i \over 1 - \beta}, \quad i = o,e  .$$
For both of the budget constraints to be satisfied with
equality,  we evidently require
that
$$ \eqalign{ c^o & = {1 \over \beta+1}  \cr
             c^e &= {\beta \over \beta+1}. \cr }  \EQN od5 $$
The price system given by equation \Ep{od4} and the constant-over-time
allocations given by equations \Ep{od5} are a competitive equilibrium.

Notice that the competitive equilibrium allocation corresponds to a particular
Pareto optimal allocation.
  \index{Ricardian proposition}
\subsection{Ricardian proposition}

   We temporarily add a government to the model.  The
government levies lump-sum taxes on agents of type $i=o,e$ at
time $t$ of $\tau_t^i$.  The government uses the
proceeds to finance a constant level of government
purchases of $G \in (0,1)$ each period $t$. Consumer
$i$'s budget constraint is
$$ \sum_{t=0}^\infty q_t^0 c_t^i \leq \sum_{t=0}^\infty q_t^0
       (y_t^i - \tau_t^i).$$
The government's budget constraint is
$$ \sum_{t=0}^\infty q_t^0 G = \sum_{i=o,e}\ \sum_{t=0}^\infty
        q_t^0 \tau_t^i.$$
We modify Definition 1 as follows:

\medskip
\noindent{\sc Definition 2:}  A {\it competitive equilibrium} is
a price sequence $\{q_t^0\}_{t=0}^\infty$,
a tax system $\{\tau_t^o, \tau_t^e\}_{t=0}^\infty$,
and an allocation $\{c_t^o, c_t^e, G_t\}_{t=0}^\infty$
such that given the price system and the tax system the following conditions
hold: (a) the allocation solves each consumer's optimum problem;
(b) the government budget constraint is satisfied for all $t\geq 0$; and
(c) $N(c^o_t+c^e_t)+G_t=N$ for all $t \geq 0$.

\medskip
Let the present value of the taxes imposed on consumer $i$
be $\tau^i \equiv \sum_{t=0}^\infty q_t^0 \tau_t^i$.  Then
it is straightforward to verify that the equilibrium price
system is still equation \Ep{od4} and that equilibrium allocations
are
$$\eqalign{ c^o &= {1 \over \beta +1} - \tau^o (1 -\beta) \cr
\noalign{\smallskip}
            c^e & ={\beta \over \beta+1} - \tau^e (1 -\beta). \cr}$$

This equilibrium features a ``Ricardian proposition'':


\noindent{\sc Ricardian Proposition:}
 The equilibrium is invariant to changes in the {\it timing}
of tax collections that leave unaltered the present value of lump-sum
taxes assigned to each agent.

\subsection{Loan market interpretation}
Define total time $t$ tax collections
as $\tau_t = \sum_{i=o,e} \tau_t^i$,
and write the government's budget constraint as
$$  (G_0 - \tau_0) = \sum_{t=1}^\infty {q_t^0 \over q_0^0}
    (\tau_t - G_t) \equiv
     B_1,$$
where $B_1$ can be interpreted as government debt issued at
time $0$ and due at time $1$.  Notice that $B_1$ equals
the present value of the future (i.e., from time 1 onward) government
{\it surpluses} $(\tau_t - G_t)$. The government's budget constraint
can also be represented as
$$ {q_0^0 \over q_1^0}  (G_0 -\tau_0)  +(G_1 - \tau_1)
    = \sum_{t=2}^\infty {q_t^0 \over q_1^0}(\tau_t - G_t) \equiv B_2,$$
or
$$ R_1 B_1 +(G_1 - \tau_1) = B_2,$$
where $R_1 = {q_0^0 \over q_1^0}$ is the gross rate of
return between time $0$ and time $1$, measured in time $1$ consumption
goods
per unit of time $0$ consumption good.  More generally,
we can represent the government's budget constraint by
the sequence of budget constraints
$$ R_t B_t + (G_t - \tau_t) = B_{t+1} ,\quad t \geq 0,$$
subject to the boundary condition $B_0=0$.
In the equilibrium computed here,
$R_t = \beta^{-1}$ for all $t \geq 1$.


Similar  manipulations of consumers' budget constraints can
be used to express them in terms of sequences of one-period
budget constraints.  That no opportunities are lost to
the government or the consumers by representing the budget
sets in this way lies behind the following fact:
  The Arrow-Debreu allocation in this economy can be
implemented with a sequence of one-period loan markets.


In the following section, we shut down {\it all} loan markets,
and also set government expenditures $G=0$.


\section{A monetary economy}
   We keep preferences and endowment patterns as they
were in the preceding economy, but we rule out all intertemporal
trades achieved through borrowing and lending or
trading of future-dated consumptions.
We replace complete markets with
a fiat money mechanism.  At time $0$, the government
endows each of the $N$ even agents with $M/N$ units
of an unbacked or inconvertible currency.  Odd agents
are initially endowed with zero units of the currency.
Let $p_t$ be the time $t$ price level, denominated in dollars
per time $t$ consumption good.  We seek an equilibrium
in which currency is valued ($p_t < + \infty \ \forall t \geq 0$)
and in which each period agents not endowed with goods pass currency
to agents who are endowed with goods. Contemporaneous exchanges
of currency for goods are the only exchanges that
we, the model builders, permit.  (Later, Townsend will give us a defense
or reinterpretation of this high-handed shutting down of
markets.)

Given the sequence of prices $\{ p_t \}_{t=0}^\infty$, the
household's problem is to choose nonnegative sequences
$\{ c_t, m_t\}_{t=0}^\infty$ to maximize $\sum_{t=0}^\infty
\beta^t u(c_t)$ subject to
$$ m_t + p_t c_t \leq p_t y_t + m_{t-1} , \quad t \geq 0, \EQN od6$$
where $m_t$ is currency held from $t$ to $t+1$.
   Form the household's Lagrangian
$$L = \sum_{t=0}^\infty \beta^t \{ u(c_t) +
  \lambda_t ( p_t y_t + m_{t-1}  - m_t - p_t c_t ) \} ,$$
where $\{ \lambda_t \}$ is a sequence of nonnegative
Lagrange multipliers.
The household's first-order conditions  for $c_t$ and
$m_t$, respectively, are
$$ u'(c_t) \leq \lambda_t p_t, \quad = \ {\rm if  } \ c_t > 0 ,$$
$$ - \lambda_t + \beta \lambda_{t+1} \leq 0 , \quad = \ {\rm  if } \
    m_t > 0 .$$
Substituting the first condition at equality into the
second gives
$$ { \beta u'(c_{t+1}) \over p_{t+1}} \leq {u'(c_t) \over p_t } , \quad
   = \ {\rm if} \ m_t > 0 .  \EQN od7 $$

\noindent{\sc Definition 3:}  A {\it competitive equilibrium\/} is an allocation
$\{c_t^o, c_t^e\}_{t=0}^\infty$, nonnegative
money holdings $\{m_t^o, m_t^e\}_{t=-1}^\infty$,
and a nonnegative price level sequence $\{p_t\}_{t=0}^\infty$ such that
(a) given the price level sequence and $(m^o_{-1},m^e_{-1})$;
the allocation solves the optimum problems of both types of
households; and (b)  $c^o_t+c^e_t=1$, $m^o_{t-1}+m^e_{t-1}=M/N$,
for all $t \geq 0$.
\medskip

The periodic nature of the endowment sequences prompts us to
guess the following two-parameter form of stationary equilibrium:
$$\eqalign{\{c_t^o\}_{t=0}^\infty& = \{c_0, 1- c_0, c_0, 1-c_0, \ldots\},
\cr
\noalign{\smallskip}
\{c_t^e\}_{t=0}^\infty& = \{ 1- c_0, c_0, 1- c_0, c_0, \ldots \} ,\cr}
\EQN od8 $$
and
$p_t = p$ for all $t \geq 0$.  To determine the
two undetermined parameters $(c_0,p)$, we use the first-order
conditions and budget constraint of the odd agent at time $0$. His endowment
sequence for periods $0$ and $1$, $(y^o_0,y^o_1)=(1,0)$, and
the Inada condition \Ep{turn_inada} ensure that both of his
first-order conditions at time $0$ will hold with equality. That is,
his desire to set $c^o_0 > 0$ can be met by consuming some of the
endowment $y^o_0$, and the only way for him to secure consumption
in the following period $1$ is to hold strictly positive money
holdings $m^o_0 > 0$. From his first-order conditions at equality,
we obtain
$$ { \beta u'(1-c_0) \over p } = { u'(c_0) \over p} ,$$
which implies that $c_0$ is to be determined as the
root of
$$ \beta - { u'(c_0) \over u'(1-c_0) }  =0 .\EQN od9$$
Because $\beta < 1, $ it follows that $c_0 \in (\frac 1/2, 1)$.  To
determine the price level, we use the odd agent's budget constraint
at $t=0$, evaluated at $m_{-1}^o = 0$ and $m_0^o = M/N$, to get
$$ p c_0 + M/N = p \cdot 1 ,$$
or
$$ p = { M \over N (1 - c_0)}.      \EQN turn_p
$$
See Figure \Fg{dmt62f} %18.1
 for a graphical
determination of $c_0$.

%%%%%%%%%%%%
%\topinsert
%$$
%\grafone{dmt61.ps,height=3in}{{\bf Figure 1.}  The tradeoff between time
%$t$ and time $t+1$ consumption faced by agent endowed with one unit in
%period $t$.}
%$$ \endinsert
%%%%%%%%%%%%%%%%%%%%%%%%%%%

%\midfigure{dmt61f}}
%\centerline{\epsfxsize=3truein\epsffile{dmt61.eps}}
%\caption{The tradeoff between time %$t$ and time $t+1$ consumption faced by
%agent endowed with one unit in %period $t$.}
%\infiglist{dmt61f}
%\endfigure

From equation \Ep{od9}, it follows that for $\beta < 1$, $c_0 > .5$ and $1-c_0 < .5$.
Thus, both types of agents experience fluctuations in their consumption sequences
in this monetary equilibrium.  Because Pareto optimal allocations have constant
consumption sequences for each type of agent, this equilibrium allocation is
not Pareto optimal.

%%%%%%%%%%%%%%%%%%%%%%%%%
%\topinsert{
%$$\grafone{dmt62.ps,height=3in}{{\bf Figure 18.1}
% The tradeoff between time-$t$ and time -$(t+1)$ consumption faced
%by agent $o(e)$ in equilibrium for $t$ even (odd).  For $t$
%even, $c_t^o =c_0$, $c_{t+1}^o=1-c_0$, $m_t^o=p (1-c_0)$, and
%5$m_{t+1}^o=0$.  The slope of the indifference curve at $X$ is
%$-u'(c_t^h)/\be u'(c_{t+1}^h)=-u'(c_0)/\be u'(1-c_0)=-1$, and
%the slope of the indifference curve at $Y$ is $-u'(1-c_0)/\beta
%u'(c_0)=-1/\beta^2$.}
%$$  }\endinsert
%%%%%%%%%%%%%%%%%%%%%%%%%%%

\midfigure{dmt62f}
\centerline{\epsfxsize=3truein\epsffile{dmt62.ps}}
\caption{The trade-off between time $t$ and time $(t+1)$ consumption faced
by agent $o(e)$ in equilibrium for $t$ even (odd).  For $t$ even,
$c_t^o =c_0$, $c_{t+1}^o=1-c_0$, $m_t^o=p (1-c_0)$, and $m_{t+1}^o=0$.
The slope of the indifference curve at $X$ is
$-u'(c_t^h)/\be u'(c_{t+1}^h)=-u'(c_0)/\be u'(1-c_0)=-1$, and the slope of
the indifference curve at $Y$ is $-u'(1-c_0)/\beta u'(c_0)=-1/\beta^2$.}
\infiglist{dmt62f}
\endfigure
\index{turnpike!Townsend's}
\section{Townsend's ``turnpike'' interpretation}
The preceding analysis of currency is artificial in the sense that it depends
entirely on our having arbitrarily ruled out the existence of markets for
private loans.  The physical setup of the model itself provided no reason for
those loan markets not to exist, and indeed good reasons for them to exist.
In addition, for many questions that we want to analyze, we want a model in
which private loans and currency coexist, with currency being
valued.\NFootnote{In the United States today, for example, $M_1$ %
consists of the sum of demand deposits (a part of which is backed by %
commercial loans and another, smaller part of which is backed by reserves %
or currency) and currency held by the public.  Thus, $M_1$ is not %
interpretable as the $m$ in our model.}

\auth{Townsend, Robert M.}
Robert Townsend has proposed a model whose mathematical structure is
identical with the preceding model, but in which a global market in
private loans cannot emerge because agents are spatially separated.
Townsend's setup can accommodate local markets for private loans,
so that it meets the objections to the model that we have expressed.
But first we will focus on a version of Townsend's model where
local credit markets cannot emerge, which will be mathematically equivalent
to our model above.

%%%%%%%%%%%%%%%%%%%%%%%%
%$$\grafone{turn10.eps,height=.5in,width=3in}{{\bf Figure 18.2}  Endowment
%pattern along a Townsend turnpike.  The turnpike
%is of infinite extent in each direction, and
%has equidistant trading posts.  Each trading
%post has equal numbers of east-heading and west-heading
%agents.  At each trading post  (the
%black dots) each period, for  each east-heading agent there
%is a west-heading agent with whom he would like to borrow or lend.
%But itineraries rule out the possibility of repayment.}  $$
%%%%%%%%%%%%%%%%%%%%%%%%%%%

\midfigure{turn10f}
\centerline{\epsfxsize=3truein\epsffile{turn10.eps}}
\caption{Endowment pattern along a Townsend turnpike.  The turnpike
is of infinite extent in each direction, and has equidistant trading
posts.  Each trading post has equal numbers of east-heading and west-heading
agents.  At each trading post (the black dots) each period, for each
east-heading agent there is a west-heading agent with whom he would like
to borrow or lend.  But itineraries rule out the possibility of repayment.}
\infiglist{turn10f}
\endfigure

The economy starts at time $t=0$, with $N$ east-heading migrants and $N$
west-heading migrants physically located at each of the integers along a
``turnpike'' of infinite length extending in both directions.  Each of the
integers $n=0,\pm 1, \pm 2,\ldots$ is a trading post number.  Agents can
trade the one good only with agents at the trading post at which they find
themselves at a given date.  An east-heading agent at an even-numbered
trading post is endowed with one unit of the consumption good, and an
odd-numbered trading post has an endowment of zero units (see Figure \Fg{turn10f}). %18.2).
  A
west-heading agent is endowed with zero units at an even-numbered trading post
and with one unit of the consumption good at an odd-numbered trading post.
Finally, at the end of each period, each east-heading agent moves one trading
post to the east, whereas each west-heading agent moves one trading post to the
west.  The turnpike along which the trading posts are located is of infinite
length in each direction, implying that the east-heading and west-heading
agents who are paired at time $t$ will never meet again.  This feature means
that there can be no private debt between agents moving in opposite directions.
An IOU between agents moving in opposite directions can never be collected
because a potential lender never meets the potential borrower again, nor does
the lender meet anyone who ever meets the potential borrower, and so on, ad
infinitum.

%\input gayejnl.txt
%\input gayedef.txt

Let an agent who is endowed with one unit of the good $t=0$ be called an agent
of type $o$ and an agent who is endowed with zero units of the good at $t=0$ be
called an agent of type $e$.  Agents of type $h$ have preferences summarized by
$\sum_{t=0}^\infty \be^t u(c_t^h)$.  Finally, start the economy at time 0 by
having each agent of type $e$ endowed with $m_{-1}^e=m$ units of unbacked currency
and each agent of type $o$ endowed with $m_{-1}^o=0$ units of unbacked currency.

With the symbols thus reinterpreted, this model involves precisely the same
mathematics as that which was analyzed earlier.  Agents' spatial separation and
their movements along the turnpike have been set up to produce a physical
reason that a global market in private loans cannot exist.  The various
propositions about the equilibria of the model and their optimality that were
already proved apply equally to the turnpike version.\NFootnote{A
version of the model could be constructed in which local private markets for
loans coexist with valued unbacked currency.  To build such a model, one would
assume some heterogeneity in the time patterns of the endowment of agents who
are located at the same trading post and are headed in the same direction.  If
half of the east-headed agents located at trading post $i$ at time $t$ have
present and future endowment pattern $y_t^h=(\a,\ga,\a,\ga\ldots)$, for
example, whereas the other half of the east-headed agents have
$(\ga,\a,\ga,\a,\ldots)$ with $\ga\not=\a$, then there is room for local
private loans among this cohort of east-headed agents.  Whether or not there
exists an equilibrium with valued currency depends on how nearly Pareto optimal
the equilibrium with local loan markets
 is.}$^,$\NFootnote{Narayana Kocherlakota (1998)
 has analyzed the frictions in the Townsend turnpike and overlapping
generations model.  By permitting agents to use history-dependent
decision rules, he has been able to support optimal allocations with
the equilibrium of a gift-giving game.  Those equilibria leave no
room for valued fiat currency. Thus, Kocherlakota's view is that the
frictions that give valued currency in the Townsend turnpike must include
the restrictions on the strategy space that Townsend implicitly
imposed.}
Thus, in Townsend's version of the model, spatial
separation is the ``friction'' that provides a potential social
role for a valued unbacked currency.  The spatial separation
of agents and their endowment patterns give a setting in which private
loan markets are limited by the need for people who trade IOUs to be
linked together, if only indirectly, recurrently over time and space.
\index{gift-giving game}
\auth{Kocherlakota, Narayana R.}
\index{Friedman rule}
\auth{Friedman, Milton}
\section{The Friedman rule}
Friedman's proposal to pay interest on currency by engineering a deflation can
be used to solve for a Pareto optimal allocation in this economy.  Friedman's
proposal is to decrease the currency stock by means of lump-sum taxes at a
properly chosen rate.  Let the government's budget constraint be
$$ M_t = (1 + \tau) M_{t-1} .$$
There are $N$ households of each type.  At time $t$, the government transfers or
taxes nominal balances in amount $\tau M_{t-1} / (2 N)$ to each household of each
type.  The total transfer at time $t$ is thus $\tau M_{t-1}$, because there are
$2 N$ households receiving transfers.

The household's time $t$ budget constraint becomes
$$ p_t c_t + m_t \leq p_t y_t + {\tau \over 2} {M_{t-1} \over N} +m_{t-1}. $$

We guess an equilibrium allocation of the same periodic
pattern \Ep{od8}.  For the price level, we make
the ``quantity theory'' guess $M_t / p_t = k$,
where $k$ is a constant.  Substituting this guess into
the government's budget constraint gives
$$ { M_t \over p_t} = (1+ \tau) {M_{t-1} \over p_{t-1}}{p_{t-1} \over p_t}$$
or
$$ k = (1 + \tau) k { p_{t-1} \over p_t},$$
or
$$ p_t = (1+\tau) p_{t-1},                                 \EQN turn_rorm
$$
which is our guess for the price level.

Substituting the price level guess and the allocation guess
into the odd agent's first-order condition \Ep{od7} at $t=0$ and
rearranging shows that $c_0$ is now the root of
$$ {1 \over (1 + \tau)} - {u'(c_0) \over \beta u'(1-c_0)} = 0 .\EQN od10$$
The price level at time $t=0$ can be determined by evaluating
the odd agent's time $0$ budget constraint at
$m_{-1}^o = 0$ and $m_0^o = M_0/N=(1+\tau)M_{-1}/N$, with the result
that
$$(1 - c_0) p_0 = {M_{-1} \over N} \left(1+ {\tau \over 2}\right) .$$

Finally, the allocation guess must also satisfy the even agent's
first-order condition \Ep{od7} at $t=0$ but not necessarily with
equality, since the stationary equilibrium has $m_0^e = 0$.
After substituting $(c_0^e, c_1^e) =(1-c_0, c_0)$ and
\Ep{turn_rorm} into \Ep{od7}, we have
$$
{1 \over 1+\tau} \leq {u'(1-c_0) \over \beta u'(c_0)}.     \EQN turn_foc
$$
The substitution of \Ep{od10} into \Ep{turn_foc} yields a restriction
on the set of periodic allocations of type \Ep{od8} that can
be supported as one of our stationary monetary equilibria,
$$
\left[ {u'(c_0) \over u'(1-c_0)} \right]^2 \leq 1 \quad \Longrightarrow
\quad c_0 \geq 0.5.
$$
This restriction on $c_0$, together with \Ep{od10}, implies a
corresponding restriction on the set of permissible monetary/fiscal
policies, $1+\tau \geq \beta$.

\subsection{Welfare}
For allocations of the class \Ep{od8}, the utility functionals
of odd and even agents, respectively, take values that
are functions of the single parameter $c_0$, namely,
$$\eqalign{ U^o(c_0) & = {u(c_0) + \beta u(1-c_0) \over 1- \beta^2}, \cr
\noalign{\smallskip}
   U^e(c_0)& = { u(1-c_0) + \beta u(c_0) \over  1- \beta^2}. \cr }  $$
Both expressions are strictly concave in $c_0$, with derivatives
$$\eqalign{ {U^o}'(c_0)& = { u'(c_0) - \beta u'(1-c_0) \over 1- \beta^2}, \cr
\noalign{\smallskip}
            {U^e}'(c_0) &= { - u'(1-c_0) + \beta u'(c_0) \over
                1 - \beta^2}. \cr }$$
The Inada condition \Ep{turn_inada} ensures strictly interior maxima with
respect to $c_0$.
For the odd agents, the preferred $c_0$ satisfies ${U^o}'(c_0) = 0 $,
or
$${u'(c_0) \over \beta u'(1-c_0)} = 1 ,                  \EQN turn_odd
$$
which by \Ep{od10} is the zero-inflation equilibrium, $\tau=0$. For the
even agents, the preferred allocation given by ${U^e}'(c_0) = 0 $
implies $c_0 < 0.5$, and can therefore not be implemented as a monetary
equilibrium above. Hence, the even agents' preferred stationary monetary
equilibrium is the one with the smallest permissible $c_0$, i.e.,
$c_0=0.5$. According to \Ep{od10}, this allocation can be supported by
choosing money growth rate $1+\tau = \beta$, which
is then also the equilibrium gross rate of deflation.
Notice that
all agents, both odd and even, are in agreement that they prefer
no inflation to positive inflation, that is, they prefer $c_0$
determined by \Ep{turn_odd} to any higher value of $c_0$.

To abstract from the described conflict of interest between odd and
even agents, suppose that the agents must pick their preferred
monetary policy under a ``veil of ignorance,'' before knowing their
true identity. Since there are equal numbers of each type of agent, an
individual faces a fifty-fifty chance of her identity being an odd or
an even agent. Hence, prior to knowing one's identity, the expected
lifetime utility of an agent is
$$
\bar U(c_0) \equiv {1 \over 2} U^o(c_0) + {1 \over 2} U^e(c_0)
            = {u(c_0) + u(1-c_0) \over 2(1-\beta)}.
$$
The {\it ex ante\/} preferred allocation $c_0$ is determined by the
first-order condition $\bar U'(c_0)=0$, which has the solution $c_0=0.5$.
Collecting equations \Ep{turn_rorm}, \Ep{od10}, and \Ep{turn_foc},
this preferred policy is characterized by
$$  {p_t \over p_{t+1}} = {1 \over 1+\tau}
   ={u'(c^o_t) \over \beta u'(c^o_{t+1})}
   = {u'(c^e_t) \over \beta u'(c^e_{t+1})} = {1 \over \beta}, \qquad \forall t \geq 0,
$$
where $c^i_j=0.5$ for all $j \geq 0$ and $i\in\{o,e\}$. Thus, the
real return on money, $p_t/p_{t+1}$, equals a common marginal rate of
intertemporal substitution, $\beta^{-1}$, and this return would therefore
also constitute the real interest rate if there were a credit market.
Moreover, since the gross real return on money is the inverse of the gross
inflation rate, it follows that the gross real interest rate $\beta^{-1}$
multiplied by the gross rate of inflation is unity, or the net nominal
interest rate is zero. In other words, all agents are {\it ex ante\/} in favor
of Friedman's rule.   \index{Friedman rule}

%%%%%%%%%%%%%%%%%%%
%\topinsert{ $$
%\grafone{townsend.ps,height=3.5in}{{\bf Figure 18.3} Utility possibility
%frontier in Townsend turnpike.  The locus of points $ABC$ denotes
%allocations attainable in stationary monetary equilibria.  The
%point $B$ is the allocation associated with the zero-inflation
%monetary equilibrium.  Point $A$ is associated with Friedman's rule,
%while points between $B$ and $C$ correspond to stationary monetary
%equilibria with inflation.}
%$$ } \endinsert
%%%%%%%%%%%%%%%%%%%%%%

\midfigure{townsendf}
\centerline{\epsfxsize=3truein\epsffile{townsend.ps}}
\caption{Utility possibility frontier on the  Townsend turnpike.  The locus of
points $ABC$ denotes allocations attainable in stationary monetary
equilibria.  Point $B$ is the allocation associated with the
zero-inflation monetary equilibrium.  Point $A$ is associated with
Friedman's rule, while points between $B$ and $C$ correspond to
stationary monetary equilibria with inflation.}
\infiglist{townsendf}
\endfigure

Figure \Fg{townsendf} %18.3
shows the ``utility possibility frontier'' associated with
this economy.  Except for the allocation associated with Friedman's rule,
the allocations associated with stationary monetary equilibria lie inside
the utility possibility frontier.
\index{deficit finance!as cause of inflation}

\section{Inflationary finance}
The government prints new currency in total amount $M_t-M_{t-1}$ in period $t$
and uses it to purchase a constant amount $G$ of goods in period $t$.  The
government's time $t$ budget constraint is
$$ M_t - M_{t-1} = p_t G, \quad t \geq 0. \EQN od11$$
Preferences and endowment patterns of odd and even agents
are as specified previously.
We now use the following definition:

\medskip

\noindent{\sc Definition 4:} A \idx{competitive equilibrium} is a price
level sequence $\{p_t\}_{t=0}^\infty$, a money supply
process $\{M_t\}_{t=-1}^\infty$,
an allocation $\{c_t^o, c_t^e, G_t\}_{t=0}^\infty$ and nonnegative
money holdings $\{m_t^o, m_t^e\}_{t=-1}^\infty$
 such that
(a)  given the price sequence and $(m^o_{-1},m^e_{-1})$,
the allocation solves the optimum problems of households of both types;
(b) the government's budget constraint is satisfied
for all $t \geq 0$; and (c) $N(c^o_t+c^e_t)+G_t=N$, for all $t \geq 0$; and
$m^o_{t}+m^e_{t}=M_{t}/N$, for all $t \geq -1$.

\medskip

For $t \geq 1$, write the government's budget constraint as
$$ {M_t \over N p_t} = {p_{t-1} \over p_t} {M_{t-1} \over N p_{t-1} }
      + {G \over N}  ,$$
or
$$ \tilde m_t = R_{t-1} \tilde m_{t-1} + g, \EQN od12$$
where $g = G / N$,$\tilde m_t = M_t / (N p_t)$ is
per-odd-person real balances, and $R_{t-1} = p_{t-1}/ p_t$
is the rate of return on currency from $t-1$ to $t$.

To compute an equilibrium, we guess an allocation of
the periodic form
$$\eqalign{ \{c_t^o\}_{t=0}^\infty &
 = \{c_0, 1 - c_0 -g, c_0, 1- c_0 -g, \ldots \}, \cr
\noalign{\smallskip}
   \{c_t^e\}_{t=0}^\infty &=\{ 1-c_0-g, c_0, 1-c_0-g, c_0, \ldots\}. \cr }
  \EQN od13$$
We guess that $R_t = R$ for all
$t\geq 0$, and again guess a ``quantity theory'' outcome
$$ \tilde m_t = \tilde m \quad \forall t \geq 0.$$
Evaluating the odd household's time $0$ first-order condition
for currency at equality gives
$$ \beta R = {u'(c_0) \over u'(1-c_0-g)}  . \EQN od14$$
With our guess,  real balances held by each odd
agent at the end of period $0$, $m^o_0/p_0$, equal $1-c_0$, and
time $1$ consumption, which also is $R$ times the
value of these real balances held from $0$ to $1$,
is $1-c_0 -g$.  Thus,
$(1-c_0)R = (1-c_0-g)$, or
$$ R = {1-c_0 -g \over 1-c_0 } . \EQN od15$$
Equations \Ep{od14} and \Ep{od15} are two simultaneous equations that
we want to solve for $(c_0,R)$.

Use equation \Ep{od15} to eliminate $(1-c_0-g)$ from equation \Ep{od14} to get
$$ \beta R = {u'(c_0) \over u'[R(1-c_0)] } .$$
Recalling that $(1-c_0) = m_0$, this can be
written
$$ \beta R ={u'(1-  m_0) \over u'(R  m_0)}.\EQN od17$$
For the power utility function $u(c)={c^{1-\delta} \over 1-\delta}$,
this equation can be solved for $m_0$ to get the demand function
for currency
$$ m_0 = \tilde m(R) \equiv{(\beta R^{1-\delta})^{1/\delta} \over 1 +
       (\beta R^{1-\delta})^{1/\delta}} .\EQN od18$$
Substituting this into the government budget constraint \Ep{od12} gives
$$ \tilde m(R)(1-R) = g . \EQN od19 $$
This equation equates the revenue from the inflation
tax, namely, $\tilde m(R)(1-R)$ to the government deficit, $g$.
The revenue from the inflation tax is the product of
real balances and the inflation tax rate $1-R$.  The equilibrium value
of $R$ solves equation \Ep{od19}.

%%%%%%%%%%%%%%%%%%%%
%\topinsert
%$$
%\grafone{town7.ps,height=1.75in}{{\bf Figure 18.4} Revenue from
%inflation tax \hfil\break [$m(R)(1-R)$]
% and deficit for $\beta=.95, \delta=2, g=.2$.  The gross
%rate of return on currency is on the $x$-axis; the revenue from
%inflation and $g$ are on the $y$-axis.}$$
%%%%%%%%%%%%%%%%%%%%
%$$\grafone{town5.ps,height=1.75in}{{\bf Figure 18.5} Revenue from
%inflation tax \hfil\break [$m(R)(1-R)$]
%and deficit for $\beta=.95, \delta=.7, g=.2$.  The
%rate of return on currency is on the $x$-axis; the revenue from
%inflation and $g$ are on the $y$-axis. Here there is a Laffer curve.}
%$$ \endinsert
%%%%%%%%%%%%%%%%%%%%%%

\midfigure{town7f}
\centerline{\epsfxsize=3truein\epsffile{town7.ps}}
\caption{Revenue from inflation tax  $m(R)(1-R)$ and deficit
for $\beta=.95, \delta=2, g=.2$.  The gross rate of return on currency is
on the $x$-axis;  $g$ and the revenue from inflation  are on the $y$-axis.}
\infiglist{town7f}
\endfigure

\midfigure{town5f}
\centerline{\epsfxsize=3truein\epsffile{town5.ps}}
\caption{Revenue from inflation tax $m(R)(1-R)$ and deficit for
$\beta=.95, \delta=.7, g=.2$.  The rate of return on currency is on the
$x$-axis;   $g$ and the revenue from inflation are on the $y$-axis.  Here
there is a Laffer curve.}
\infiglist{town5f}
\endfigure

\index{Laffer curve}
Figures \Fg{town7f} and \Fg{town5f} %18.4 and 18.5
 depict the determination of the stationary
equilibrium value of $R$ for two sets of parameter values.  For
the case $\delta=2$, shown in Figure \Fg{town7f}, %18.4,
 there is a unique
equilibrium $R$; there is a unique equilibrium for every
$\delta \geq 1$.  For $\delta \geq 1$, the demand function for
currency slopes upward as a function of $R$, as for the example in
Figure \Fg{town6f}.  %18.6.
 For $\delta <1$, there can occur multiple stationary
equilibria, as for the example in Figure \Fg{town5f}. %  18.5.
  In such cases, there
is a Laffer curve in the revenue from the inflation tax.  Notice
that the demand for real balances is downward sloping as a
function of $R$ when $\delta < 1$.

%%%%%%%%%%%%%%%%%%%%%%%%%
%\midinsert
%$$\grafone{town6.ps,height=2in}{{\bf Figure 18.6} Demand
%for real balances on the $y$-axis as function of the gross
%rate of return on currency on $x$-axis when $\beta =.95, \delta =2.$}
%$$
%%%%%%%%%%%%%%%
%$$\grafone{town8.ps,height=2in}{{\bf Figure 18.7}  Demand for
%real balances on the $y$-axis as function of the
%gross rate of return on currency on $x$-axis when $\beta = .95,
%\delta = .7.$}
%$$ \endinsert
%%%%%%%%%%%%%%%%%%%%%%%


\midfigure{town6f}
\centerline{\epsfxsize=3truein\epsffile{town6.ps}}
\caption{Demand for real balances on the $y$-axis as a function of the gross
rate of return on currency on the $x$-axis when $\beta =.95, \delta =2.$}
\infiglist{town6f}
\endfigure

%\midfigure{town8f}
%\centerline{\epsfxsize=3truein\epsffile{town8.ps}}
%\caption{Demand for real balances on the $y$-axis as a function of the gross
%rate of return on currency on the $x$-axis when $\beta = .95, \delta = .7.$}
%\infiglist{town8f}
%\endfigure

The initial price level is determined by the time $0$ budget constraint of the
government, evaluated at equilibrium time $0$ real balances.  In particular,
the time $0$ government budget constraint can be written
$$ {M_0 \over N p_0} - {M_{-1} \over N p_0} = g,$$
or
$$ \tilde m - g = {M_{-1} \over N p_0} .$$
Equating $\tilde m$ to its equilibrium value $1-c_0$ and solving
for $p_0$ gives
$$ p_0 = {M_{-1} \over N (1 - c_0 - g)}.$$
 \index{legal restrictions}
\section{Legal restrictions}

This section adapts ideas of Bryant and Wallace (1984)  and Villamil (1988) to the turnpike environment.
Those authors analyzed situations in which the government
could make all savers better off by introducing a price discrimination scheme for
marketing its debt.  The analysis formalizes some ideas mentioned
by John Maynard Keynes (1940).
\auth{Keynes, John Maynard}
\auth{Villamil, Anne}
\auth{Wallace, Neil}
\auth{Bryant, John}

Figure \Fg{bryant4f} %18.8
 depicts the terms on which an odd agent at $t=0$ can transfer
consumption between $0$ and $1$ in an equilibrium with inflationary finance.
The agent is endowed at the point $(1,0)$.  The monetary mechanism allows
him to transfer consumption between periods on the terms $c_1 = R (1 - c_0)$,
depicted by the budget line connecting $1$ on the $c_t$-axis with the point $B$
on the $c_{t+1}$-axis.  The government insists on raising revenues in the
amount $g$ for each pair of an odd and an even agent, which means that $R$ must
be set so that the tangency between the agent's indifference curve and the budget
line $c_1=R(1-c_0)$ occurs at the intersection of the budget line and the straight
line connecting $1-g$ on the $c_t$-axis with the point $1-g$ on the $c_{t+1}$-axis.
At this point, the marginal rate of substitution for odd agents is
$${u'(c_0) \over \beta u'(1-c_0 -g)} = R,$$
(because currency holdings are positive).  For even agents, the
marginal rate of substitution is
$$ {u'(1-c_0-g) \over \beta u'(c_0)}= {1 \over \beta^2 R} > 1,$$
where the inequality follows from the fact that $R< 1$ under
inflationary finance.

%%%%%%%%%%%%%%%%%%%%%%%%%
%\midinsert{$$
%\grafone{bryant4.eps,height=3.0in}{{\bf Figure 18.8} The budget
%line starting at $(1,0)$ and ending at the point B describes
% an odd agent's time $0$ opportunities in an equilibrium with inflationary finance.
%Because this equilibrium has the ``private consumption feasibility
%menu'' intersecting the odd agent's indifference curve, a ``forced saving''
%legal restriction can be used to put the odd agent onto a higher
%indifference curve than I, while leaving even agents better off
%and the government with revenue $g$. If the individual is
%confronted with a minimum denomination $F$ at the
%rate of return associated with the budget line ending at H,
%he would choose to consume $1-F$.}
%$$   }\endinsert
%%%%%%%%%%%%%%%%%%%%%%

\midfigure{bryant4f}
\centerline{\epsfxsize=3truein\epsffile{bryant4.eps}}
\caption{The budget line starting at $(1,0)$ and ending at the point $B$
describes an odd agent's time $0$ opportunities in an equilibrium with
inflationary finance.  Because this equilibrium has the ``private
consumption feasibility menu'' intersecting the odd agent's indifference
curve, a ``forced saving'' legal restriction can be used to put the odd
agent onto a higher indifference curve than $I$, while leaving even agents
better off and the government with revenue $g$. If the individual is
confronted with a minimum denomination $F$ at the rate of return associated
with the budget line ending at $H$, he would choose to consume $1-F$.}
\infiglist{bryant4f}
\endfigure

The fact that the odd agent's indifference curve intersects the solid line
connecting $(1-g)$ on the two axes indicates that the government could improve
the welfare of the odd agent by offering him a higher rate of return subject
to a minimal real balance constraint. The higher rate of return is used to
send the line $c_1 = (1-R) c_0$ into the lens-shaped area in Figure \Fg{bryant4f}
onto a higher indifference curve. The minimal real balance constraint is
designed to force the agent onto the ``postgovernment share'' feasibility
line connecting the points $1-g$ on the two axes.

Thus, notice that in Figure \Fg{bryant4f}, %18.8,
 the government can raise the same revenue by
offering odd agents the {\it higher\/} rate of return associated with the line
connecting $1$ on the $c_t$ axis with the point $H$ on the $c_{t+1}$ axis, provided
that the agent is required to save at least $F$, if he saves at all.  This minimum
saving requirement would make the household's budget set the point $(1,0)$  together
with the heavy segment $DH$.  With the setting of $F,R$ associated with the line $DH$
in Figure \Fg{bryant4f}, %18.8,
odd households have the same two-period utility as without this
scheme. (Points $D$ and $A$ lie on the same indifference curve.)  However, it is
apparent that there is room to lower $F$ and lower $R$ a bit, and thereby move the
odd household into the lens-shaped area.  See Figure \Fg{bryant5f}. %18.9.

The marginal rates of substitution that we computed earlier indicate that
this scheme makes both odd and even agents better off relative to the
original equilibrium. The odd agents are better off because they move
into the lens-shaped area in Figure \Fg{bryant4f}.  The even agents are better
off because relative to the original equilibrium, they are being
permitted to ``borrow'' at a gross rate of interest of $1$.  Since their
marginal rate of substitution at the original equilibrium is
$1/(\beta^2 R) > 1$, this ability to borrow makes them better off.

%%%%%%%%%%%%%%%%%%%%%%%
%$$
%\grafone{townsend2.ps,height=3.5in}{{\bf Figure 18.8} The budget
%line starting at $(1,0)$ and ending at the point B describes
% an odd agent's
%time $0$ opportunities in an equilibrium with inflationary finance.
%Because this equilibrium has the ``private consumption feasibility
%menu'' intersecting the odd agent's indifference curve, a ``forced saving''
%legal restriction can be used to put the odd agent onto a higher
%indifference curve than II, while leaving even agents better off
%and the government with revenue $g$.}
%$$

%%%%%%%%%%%%%%%%%%%%%%%
%%%%%%%
%\midfigure{townsend2f}
%\centerline{\epsfxsize=3truein\epsffile{townsend2.eps}}
%\caption{The budget line starting at $(1,0)$ and ending at the point B
%describes an odd agent's time $0$ opportunities in an equilibrium with
%inflationary finance.  Because this equilibrium has the ``private consumption
%feasibility menu'' intersecting the odd agent's indifference curve, a
%``forced saving'' legal restriction can be used to put the odd agent onto
%a higher indifference curve than II, while leaving even agents better off
%and the government with revenue $g$.}
%\infiglist{townsend2f}
%\endfigure


%  bryant1.eps has the picture with legal restrictions

%%%%%%%%%%%%%%%%%%%%%%%
%\topinsert{
%$$\grafone{bryant5.eps,height=3.0in}{{\bf Figure 18.9}  The minimum
%denomination $F$ and the return on money
%can be lowered {\rm vis-$\grave a$-vis} their setting associated with line DH in
%Figure 18.8 to make the odd household better off, raise the
%same revenues for the government, and leave even households better off
%(as compared to no government intervention).  The
%lower value of $F$ puts the odd household at $E$, which leaves him
%at the higher indifference curve I$'$.  The minimum
%denomination $F$ and the return on money
%can be lowered {\rm vis-$\grave a$-vis} their setting associated with line DH in
%Figure 18.8 to make the odd household better off, raise the
%same revenues for the government, and leave even households better off
%(as compared to no government intervention).  The
%lower value of $F$ puts the odd household at $E$, which leaves him
%at the higher indifference curve I$'$.}
%$$  }\endinsert
%%%%%%%%%%%%%%%%%%%%%%

\midfigure{bryant5f}
\centerline{\epsfxsize=3truein\epsffile{bryant5.eps}}
\caption{The minimum denomination $F$ and the return on money can be lowered
{\it vis-$\grave a$-vis\/} their setting associated with line DH in Figure \Fg{bryant4f}
to make the odd household better off, raise the same revenues for the government,
and leave even households better off (as compared to no government intervention).
The lower value of $F$ puts the odd household at $E$, which leaves him at the
higher indifference curve $I'$.  The minimum denomination $F$ and the return on
money can be lowered {\it vis-$\grave a$-vis} their setting associated with line
$DH$ in Figure \Fg{bryant4f} to make the odd household better off, raise the same revenues
for the government, and leave even households better off (as compared to no government
intervention).  The lower value of $F$ puts the odd household at $E$, which leaves
him at the higher indifference curve $I'$.}
\infiglist{bryant5f}
\endfigure


%The minimum
%denomination $F$ can be lowered {\it vis a vis} its setting in
%Figure 6 to make the odd household better off, raise the
%same revenues for the government, and leave even households better off.  The
%lower value of $F$ puts the household at E, which leaves him
%at the higher indifference curve I$'$.}

%\vfill\eject
 \index{exchange rate}
\section{A two-money model}

There are two types of currency being issued, in amounts $M_{it}, i=1,2$, by each of
two countries. The currencies are issued according to the rules

%\beginleftbox  Note:  no original eqn. 20 \endleftbox

$$ M_{it} - M_{it-1} = p_{it} G_{it}, \quad i=1,2, \EQN od21$$
where $G_{it}$ is total purchases of time $t$ goods by
the government issuing currency $i$, and $p_{it}$ is the time $t$
price level denominated in units of currency $i$.  We
assume that currencies of both types are initially
equally distributed among the even agents at time $0$.  Odd
agents start out with no currency.

Household $h$'s optimum problem becomes to maximize
$\sum_{t=0}^\infty \beta^t u(c_t^h)$ subject to the
sequence of budget constraints
$$ c_t^h + {m_{1t}^h \over p_{1t}} + {m_{2t}^h \over p_{2t}}
  \leq y_t^h + {m_{1t-1}^h \over p_{1t}} +{m_{2t-1}^h \over p_{2t} },$$
where $m_{jt-1}^h$ are nominal holdings of
country $j$'s currency by household $h$. Currency holdings
of each type must be nonnegative.  The first-order conditions
for the household's problem with respect to $m_{jt}^h$ for
$j=1,2$ are
$$\eqalign{ {\beta u'(c_{t+1}^h) \over p_{1t+1}}& \leq {u'(c_t^h) \over p_{1t}}
 , \quad = \ {\rm if} \ m_{1t}^h > 0, \cr
            {\beta u'(c_{t+1}^h) \over p_{2t+1}}& \leq {u'(c_t^h) \over p_{2t}}
 , \quad = \ {\rm if} \ m_{2t}^h > 0. \cr   }$$
If agent $h$ chooses to hold both currencies from $t$ to $t+1$,
these first-order conditions imply that
$$\EQNalign{
{ p_{2t} \over p_{1t}} &= {p_{2t+1} \over p_{1t+1}},  \cr
\noalign{\hbox{\rm or}}
p_{1t} &= e p_{2t}, \quad \forall t \geq 0, \EQN od22 \cr}$$
for some constant $e > 0$.\NFootnote{Evaluate both of the first-order
conditions at equality, then divide one by the other to obtain this
result.}  This equation states that if in each period
there is some household that chooses
to hold positive amounts of both types of currency, the
rate of return from $t$ to $t+1$ must
be equal for the two types of currencies, meaning that
the exchange rate must be constant over time.\NFootnote{As long
as we restrict ourselves to nonstochastic equilibria.}

We use the following definition:

\medskip
\noindent{\sc Definition 5:} A {\it competitive equilibrium\/} with two valued
fiat currencies is an allocation $\{c_t^o, c_t^e, G_{1t},G_{2t}\}_{t=0}^\infty$,
nonnegative money holdings $\{m_{1t}^o, m_{1t}^e, m_{2t}^o, m_{2t}^e  \}_{t=-1}^\infty$,
a pair of finite
price level sequences $\{p_{1t},p_{2t}\}_{t=0}^\infty$ and
currency supply sequences $\{M_{1t}, M_{2t}\}_{t=-1}^\infty$
such that \index{equilibrium} %
(a)  given the price level sequences and
$(m^o_{1,-1},m^e_{1,-1}, \hfil\break m^o_{2,-1},m^e_{2,-1})$, the
allocation solves the households' problems;
(b) the budget constraints of the governments are satisfied for all $t \geq 0$; and
(c) $N(c^o_t+c^e_t)+G_{1t}+G_{2t}=N$, for all $t \geq 0$; and
$m^o_{jt}+m^e_{jt}=M_{jt}/N$, for $j=1,2$ and all $t \geq -1$.

\medskip

In the case of constant government expenditures $(G_{1t},G_{2t})=(N g_1, N g_2)$
for all $t\geq 0$,
we guess an equilibrium allocation of the  form
\Ep{od13}, where we reinterpret $g$ to be
$g=g_1+g_2$.
We also guess an equilibrium with a constant real value of the
``world money supply,'' that is,
$$ \tilde m = {M_{1t} \over N p_{1t}} + {M_{2t} \over N p_{2t}},$$
and a constant exchange rate, so that we impose condition \Ep{od22}.  We
let $R =p_{1t}/ p_{1t+1} = p_{2t}/ p_{2t+1}$
be the constant common value of the rate of return on the
two currencies.

   With these guesses, the sum of the two countries' budget
constraints  for $t \geq 1$ and the conjectured
form of the equilibrium allocation imply an
equation of the form \Ep{od19},
where now
$$ \tilde m(R) = {M_{1t} \over p_{1t} N} +{M_{2t} \over p_{2t} N }.$$
Equation \Ep{od19} can be solved for $R$ in the  fashion described
earlier.  Once $R$ has been determined,
so has the constant real value of the world
currency supply, $\tilde m$.
 To determine the time $t$ price levels, we add the
time $0$ budget constraints of the two governments to get
$${M_{10} \over N p_{10}} + {M_{20} \over N p_{20}}
     = {M_{1,-1} + e M_{2,-1} \over N p_{10}} + (g_1+g_2) ,$$
or
$$ \tilde m - g = {M_{1,-1}+eM_{2,-1} \over N p_{10} }.$$
In the conjectured allocation, $\tilde m = (1-c_0)$,
so this equation becomes
$$ {M_{1,-1} + e M_{2,-1} \over N p_{10}} = 1 - c_0 - g ,    \EQN turn_p10$$
which, given any $e > 0$,
has a positive solution for the initial country 1 price
level. Given the solution $p_{10}$ and any $e \in (0,\infty)$,
the price level sequences  for the two countries are determined
by the constant rate of return on currency $R$.
To determine the values of the nominal currency stocks
of the two countries, we use the government budget constraints
\Ep{od21}.

  Our findings are a special case of the following remarkable
proposition:
\medskip
\noindent{\sc Proposition (Exchange Rate Indeterminacy):} Given the initial stocks of
currencies $(M_{1,-1},M_{2,-1})$ that are equally distributed among
the even agents at time $0$, if there is an equilibrium
for one constant exchange rate $e \in (0,\infty)$, then there exists
an equilibrium for any $\hat e \in (0,\infty)$
with the same consumption allocation but different currency supply sequences.
\index{exchange rate!indeterminacy}

\medskip

\noindent{\sc Proof:} Let $p_{10}$ be the country $1$ price level at
time zero in the equilibrium that is assumed to exist with exchange
rate $e$.
For the conjectured equilibrium with exchange rate $\hat e$, we guess
that the corresponding price level is
$$
\hat p_{10} = p_{10}
     {M_{1,-1} + \hat e M_{2,-1} \over M_{1,-1} + e M_{2,-1}}.
$$
After substituting this expression into \Ep{turn_p10}, we can verify
that the real value at time $0$ of the initial ``world money supply'' is
the same across equilibria. Next, we guess that the conjectured equilibrium
shares the same rate of return on currency, $R$, and constant
end-of-period real value of the ``world money supply'', $\tilde m$, as the
the original equilibrium. By construction from the original equilibrium,
we know that this setting of the world money supply process guarantees
that the consolidated budget constraint of the two governments is satisfied
in each period. To determine the values of each country's
prices and nominal money supplies, we proceed as above.
That is, given $\hat p_{10}$ and $\hat e$,
the price level sequences  for the two countries are determined
by the constant rate of return on currency $R$.
The evolution of the nominal money stocks
of the two countries is governed by government budget constraints
\Ep{od21}. \qed

\medskip
 Versions of  this proposition were
stated by Kareken and Wallace (1980).
\auth{Wallace, Neil}
\auth{Kareken, John}
See chapter \use{fiscalmonetary} for a discussion of a possible way to
alter assumptions to make the exchange rate  determinate.

%\input pformat
%\title{Add on}
 \index{money!commodity}
 \auth{Smith, Bruce D.}
\auth{Velde, Fran\c cois} \auth{Wallace, Neil}
\section{A model of commodity money}
Consider the following ``small-country'' model.\NFootnote{See Sargent and Wallace (1983),
Sargent and Smith (1997), and Sargent and Velde (1999)
for alternative models of commodity money.}
\
 There are now two goods, the consumption good and a durable good, silver.
Silver has a gross physical rate of return of $1$: storing one unit of silver
this period yields one unit of silver next period.  Silver is not valued
domestically, but it can be exchanged abroad at a fixed price of $v$ units of
the consumption good per unit of silver; $v$ is constant over time and is
independent of the amount of silver imported or exported from this country.
There are equal numbers $N$ of odd and even households, endowed
with consumption good sequences
$$\eqalign{ \{y_t^o\}_{t=0}^{\infty} &= \{ 1, 0, 1, 0, \ldots \}, \cr
\noalign{\smallskip}
            \{y_t^e\}_{t=0}^{\infty} &= \{ 0, 1, 0 , 1, \ldots \}. \cr }  $$
Preferences continue to be  ordered by $\sum_{t=0}^\infty \beta^t
u(c_t^i)$ for each type of person, where $c_t$ is consumption
of the consumption good.

   Each {\it even\/} person is initially endowed with $S$ units of
silver at time $0$. Odd agents own no silver at $t=0$.

 Households are prohibited from borrowing or lending with each other, or with
foreigners.  However, they can exchange silver with each other and with
foreigners.  At time $t$, a household of type $i$ faces the budget constraint
$$ c_t^i + m_t^i v \leq y_t^i + m_{t-1}^i v, $$
subject to $m_t^i \geq 0$, where $m_t^i$ is the amount of
silver stored from time $t$ to time $t+1$ by agent $i$.

%%%%%%%%%%%%%%%%%%%%%
%\topinsert{
%$$\grafone{silver.eps,height=3.0in}{{\bf Figure 18.10} Determination
%of equilibrium when $u'(v S) < \beta u'(c_0)$.  For as long as it is
%feasible, the even agent sets $u'(c_{t+1}^e) /u'(c_t^e) = \beta$ by running
%down his silver holdings.  This implies that  $c_{t+1}^e < c_t^e$ during
%the run-down period.  Eventually, the even agent runs out of silver,
%so that the ``tail'' of his allocation is $\{c_0, 1-c_0, c_0, 1-c_0, \ldots\}$,
%determined as before.  The figure depicts how the spending of silver
%pushes the agent onto lower and lower two-period budget sets.}$$
%}\endinsert
%%%%%%%%%%%%%%%%%%%%%

\midfigure{silverf}
\centerline{\epsfxsize=3truein\epsffile{silver.eps}}
\caption{Determination of equilibrium when $u'(v S) < \beta u'(c_0)$.  For
as long as it is feasible, the even agent sets $u'(c_{t+1}^e) /u'(c_t^e) = \beta$
by running down his silver holdings.  This implies that  $c_{t+1}^e < c_t^e$
during the run-down period.  Eventually, the even agent runs out of silver,
so that the tail of his allocation is $\{c_0, 1-c_0, c_0, 1-c_0, \ldots\}$,
determined as before.  The figure depicts how the spending of silver pushes the
agent onto lower and lower two-period budget sets.}
\infiglist{silverf}
\endfigure

%\vfill\eject
 \index{equilibrium}
\subsection{Equilibrium}

\noindent{\sc Definition 6:}
   A {\it competitive equilibrium\/} is an allocation
$\{c_t^o, c_t^e\}_{t=0}^\infty$ and nonnegative
asset holdings $\{m_t^o, m_t^e\}_{t=-1}^\infty$
such that, given $(m^o_{-1},\hfil\break m^e_{-1})$,
the allocation solves each agent's
optimum problem.
\medskip
  Adding the budget constraints  of the two types of agents with
equality at time
$t$ gives
$$ c_t^o + c_t^e = 1 +v(S_{t-1} - S_t) , \EQN od23  $$
where   $S_t = m_t^o+ m_t^e$ is the total (per odd person) stock
of silver in the country at time $t$.
Equation \Ep{od23} asserts that total domestic consumption at
time $t$ is the sum of the country's endowment plus
its imports of goods, where the latter equals its exports of silver,
$v (S_{t-1} - S_t)$.

Given the opportunity to choose nonnegative asset holdings with a
gross rate of return equal to $1$,
the equilibrium allocation to the odd agent is
$\{c_t^o\}_{t=0}^{\infty} =\{c_0, 1-c_0, c_0, 1-c_0, \ldots, \}$,
where $c_0$ is the solution to equation \Ep{od9}. Thus, the odd
agent holds $(1-c_0)$ units of silver from time $0$ to time $1$.  He
gets this silver either from even agents or from abroad.

Concerning the allocation to even agents, two types of equilibria are
possible, depending on the value of $v S$ relative to the
value $c_0$ that solves equation \Ep{od9}.
If $u'(v S) \geq \beta u'(c_0)$, the equilibrium allocation
to the even agent is
$\{c_t^e\}_{t=0}^{\infty} =\{ c_0^e, c_0, 1-c_0, c_0, 1-c_0, \ldots \}$,
where $c_0^e = v S$.  In this equilibrium, the
even agent at time $0$ sells all of his silver to support
time $0$ consumption.    Net {\it exports\/} of silver for the
country at time $0$ are $S - (1-c_0)/v$, i.e., summing up the
transactions of an even and an odd agent.  For $t \geq 1$, the country's
allocation and trade pattern is exactly as in the original
model (with a stationary fiat money equilibrium).


  If the solution $c_0$ to equation \Ep{od9} and $v S$ are such that
$u'(v S) < \beta u'(c_0)$,
  the equilibrium allocation to the
odd agents remains the same, but the allocation to the
even agents is different.  The situation is  depicted in
Figure \Fg{silverf}. %18.10.
  Even agents have so much silver
at time $0$ that they want to carry over positive amounts of
silver into time $1$ and maybe beyond.  As long as they
are carrying over positive amounts of silver from $t-1$ to
$t$,
the allocation to even agents has to satisfy
$$ {u'(c_{t-1}^e) \over \beta  u'(c_{t}^e) } = 1, \EQN od24$$
which implies that $c_{t}^e < c_{t-1}^e$.
Also, as long as they
are carrying over positive amounts of silver, their
first $T$ budget constraints can be used
to deduce an intertemporal budget constraint
$$\sum_{t=0}^T c_t^e \leq
         \cases{vS + (T+1) /2, &  if  $T$ odd; \cr
                vS + T/2,      &  if $T$ even. \cr} \EQN od25 $$
%& \leq vS + (T+1) /2  \quad T \ {\rm odd} \cr
%                               & \leq vS + T/2 \quad T \ {\rm even} .\cr}
The even agent finds the largest  horizon $T$ over which he satisfies
both  \Ep{od24} and \Ep{od25} at equality
%, as well as
%$c > c_0$ or $c > (1-c_0)$
with nonnegative carryover of silver for each period.  This largest
horizon $T$ will occur on an even date.\NFootnote{Suppose to the contrary
that the largest horizon $T$ is an odd date. That is, up  until date $T$, both
 \Ep{od24} and \Ep{od25} are satisfied with nonnegative savings for each period.
Now, let us examine what happens if we add one additional period and the
horizon becomes $T+1$. Since that additional period is an even date,
the right side of budget constraint \Ep{od25} is unchanged.
Therefore, condition \Ep{od24} implies that the extra period induces the
agent to reduce consumption in all periods $t\leq T$,
in order to save for consumption in period $T+1$. Since the
initial horizon $T$ satisfied  \Ep{od24} and \Ep{od25} with
nonnegative savings, it follows that so must also horizon $T+1$.
Therefore, the largest horizon $T$ must occur on an even date.}
 The equilibrium allocation to the
even agents is determined by ``gluing'' this initial piece
with declining consumption onto a ``tail'' of the
allocation assigned to even agents in the original
model, starting on an odd date,
$\{c_t\}_{t=T+1}^\infty =\{ c_0, 1-c_0, c_0, 1-c_0, \ldots \}$.\NFootnote{Is
the equilibrium with $u'(v S) < \beta u'(c_0)$, a  stylized model of Spain in
the sixteenth century? At the beginning of the sixteenth century, Spain suddenly received
a large claim on silver and gold from the New World.  During the century,
Spain exported gold and silver to the rest of Europe to finance government
and private purchases.}

\subsection{Virtue of fiat money}

This is a model with an exogenous price level and an endogenous
stock of currency.  The model can be used to express a version
of  Friedman's and Keynes's condemnation of commodity money systems:
the equilibrium allocation can be Pareto dominated by the allocation
in a fiat money equilibrium in which, in addition to
the stock of silver at time $0$, the even agents are
endowed with $M$ units of an unbacked fiat currency.
We can then show that there exists a monetary
equilibrium with a constant price level $p$ satisfying
\Ep{turn_p},
$$ p = { M \over N (1 - c_0)}. $$
In effect, the time $0$ endowment of the even agents is increased by $1-c_0$
units of consumption good.  Fiat money creates wealth by removing commodity money
from circulation, which instead can be transformed into consumption.


\section{Concluding remarks}

The model of this chapter is basically a ``nonstochastic incomplete markets
model,'' a special case of the stochastic incomplete markets models of chapter
\use{incomplete}.  The virtue of the model is that we can work out many things by
hand.  The limitation on markets in private loans leaves room for a
consumption-smoothing role to be performed by a valued fiat currency.
The reader might note how some of the monetary doctrines worked out precisely
in this chapter have counterparts in the stochastic  incomplete markets
models of chapter \use{incomplete}.

%\section{Exercises}
 \showchaptIDfalse
\showsectIDfalse
\section{Exercises}
\showchaptIDtrue
\showsectIDtrue
\medskip
\noindent{\it Exercise \the\chapternum.1}\quad {\bf Arrow-Debreu}
\medskip\noindent
 Consider an environment with equal numbers $N$ of two types of agents, odd and even,
who have endowment sequences
 $$\eqalign{ \{y_t^o\}_{t=0}^\infty &= \{1, 1 ,0,1 , 1, 0, \ldots \} \cr
\noalign{\smallskip}
           \{y_t^e\}_{t=0}^\infty & = \{0, 0, 1, 0, 0 , 1,  \ldots \}. \cr}$$
Households of type $h$ order consumption sequences
by $\sum_{t=0}^\infty \beta^t u(c_t^h)$.  Compute the
Arrow-Debreu equilibrium for this economy.

\medskip

\noindent{\it Exercise \the\chapternum.2}\quad {\bf One-period consumption loans}
\medskip\noindent
  Consider an environment with equal
numbers $N$ of two types of agents, odd and even, who have endowment sequences
 $$\eqalign{ \{y_t^o\}_{t=0}^\infty &= \{1, 0 ,1, 0, \ldots \} \cr
\noalign{\smallskip}
           \{y_t^e\}_{t=0}^\infty & = \{0, 1, 0, 1, \ldots \}. \cr}$$
Households of type $h$ order consumption sequences
by $\sum_{t=0}^\infty \beta^t u(c_t^h)$. The only market
that exists is for one-period loans.  The budget constraints
of household $h$ are
$$ c_t^h + b_t^h \leq y_t^h + R_{t-1} b_{t-1}^h ,\quad t \geq 0,$$
where $b_{-1}^h = 0, h=o,e$.  Here $b_t^h$ is agent $h$'s
lending (if positive) or borrowing (if negative) from
$t$ to $t+1$, and $R_{t-1}$ is the gross real rate of interest
on consumption loans from $t-1$ to $t$.

\medskip
\noindent{\bf a.}  Define a competitive equilibrium with
one-period consumption loans.
\medskip
\noindent{\bf b.}  Compute a competitive equilibrium with
one-period consumption loans.
\medskip
\noindent{\bf c.}  Is the equilibrium
allocation Pareto optimal?  Compare the equilibrium
allocation with that for the corresponding Arrow-Debreu
equilibrium for an economy with identical endowment
and preference structure.


\medskip
\noindent{\it Exercise \the\chapternum.3}\quad {\bf Stock market}
\medskip\noindent
  Consider a ``stock market'' version of an
economy with endowment and preference structure identical
to the one in the previous economy.  Now odd and even
agents begin life owning one of two types of ``trees.''  Odd
agents own the ``odd'' tree, which is a perpetual
claim to a dividend sequence
$$ \{y_t^o\}_{t=0}^\infty = \{ 1, 0, 1, 0, \ldots \} ,$$
while even agents initially own the
``even'' tree, which entitles them to a perpetual claim on
dividend sequence
$$ \{y_t^e\}_{t=0}^\infty = \{0, 1, 0, 1, \ldots \}.$$
Each period, there is a stock market in which
people can trade the two types of trees.  These are the
only two markets open each period. The time $t$ price
of type $j$ trees is $a_t^j, j=o,e$.
The time $t$ budget constraint of agent $h$ is
$$ c_t^h + a_t^o s_t^{ho} + a_t^e s_t^{he} \leq
      ( a_t^o + y_t^o) s_{t-1}^{ho} + (a_t^e + y_t^e) s_{t-1}^{he} ,$$
where $s_t^{hj}$ is the number of shares of
stock in tree $j$ held by agent $h$ from $t$ to $t+1$.
We assume that $s_{-1}^{oo} = 1, s_{-1}^{ee} = 1,
s_{-1}^{jk} = 0 \ {\rm for} \ j \neq k$.

\medskip
\noindent{\bf a.}  Define an equilibrium of the stock market
economy.
\medskip\noindent{\bf b.}  Compute an equilibrium of the stock market
economy.
\medskip
\noindent{\bf c.}  Compare the allocation of the stock
market economy with that of the corresponding Arrow-Debreu
economy.
%\vfil\eject
\medskip
\noindent{\it Exercise  \the\chapternum.4} \quad {\bf Inflation}
\medskip\noindent
   Consider a Townsend turnpike model  in which there are
$N$ odd agents and $N$ even agents who have endowment
sequences, respectively, of
$$\eqalign{ \{y_t^o\}_{t=0}^\infty &= \{1, 0 ,1, 0, \ldots \} \cr
\noalign{\smallskip}
           \{y_t^e\}_{t=0}^\infty & = \{0, 1, 0, 1, \ldots \}. \cr}$$
Households of each type order consumption sequences
by $\sum_{t=0}^\infty \beta^t u(c_t)$.
  The government
makes the stock of currency move according to
$$ M_t = z M_{t-1},  \quad t \geq 0 .$$
At the beginning of period $t$,
the government hands out $(z-1) m_{t-1}^h$ to each
type $h$ agent who held $m_{t-1}^h$ units of currency from
$t-1$ to $t$.  Households of type $h=o,e$ have time $t$ budget constraint of
$$ p_t c^h_t + m_t^h \leq p_t y_t^h + m_{t-1}^h + (z-1) m_{t-1}^h .$$
\medskip

\noindent {\bf a.}
Guess that an equilibrium endowment sequence of the periodic
form \Ep{od8} exists.  Make a guess at an equilibrium price sequence
$\{p_t\}$ and compute the equilibrium values of
($c_0, \{p_t\}$).  ({\it Hint:} Make a ``quantity theory''
guess for the price level.)

\medskip
\noindent{\bf b.}  How does the allocation vary with
the rate of inflation?  Is inflation ``good'' or ``bad''? Describe
odd and even agents' attitudes toward living in economies
with different values of $z$.

\medskip

\noindent{\it Exercise \the\chapternum.5}\quad {\bf A Friedman-like scheme}
\medskip\noindent
  Consider Friedman's scheme to improve
welfare by  generating
a deflation.   Suppose that the
government tries to boost the rate
of return on currency above $\beta^{-1}$ by
setting $\beta > (1+\tau)$.  Show that there exists no equilibrium
with an allocation of the class \Ep{od8} and  a price-level
path satisfying $p_t = (1+\tau) p_{t-1}$, with odd agents
holding $m_0^o>0$.  [(That is, the piece of the ``restricted
Pareto optimality frontier'' does not extend above
the allocation (.5,.5) in Figure \Fg{townsendf}.)%18.3.]
\medskip
\noindent{\it Exercise \the\chapternum.6}\quad{\bf Distribution of currency}
\medskip\noindent
  Consider an economy consisting of large and equal
numbers of two types of infinitely lived agents. There
is one kind of consumption good, which is nonstorable.
``Odd'' agents have period 2 endowment pattern $\{y_t^o\}_{t=0}^\infty$,
while ``even'' agents have period 2 endowment pattern
$\{y_t^e\}_{t=0}^\infty$.
 Agents of both types have preferences that are ordered
by the utility functional
$$ \sum_{t=0}^\infty \beta^t \ln (c_t^i) , \quad i = o,e , \quad 0 <
  \beta < 1  ,  $$
where $c_t^i$ is the time $t$ consumption of the single good
by an agent of type $i$.

Assume the following
endowment pattern:
$$ y_t^o =\{1, 0, 1, 0, 1, 0, \ldots \} $$
$$ y_t^e = \{0, 1, 0, 1, 0 ,1, \ldots \} .$$


\medskip
\noindent
Now assume that all borrowing and lending is prohibited, either
{\rm ex cathedra\/} through legal restrictions or by virtue of traveling
and locational restrictions of the kind introduced by Robert
Townsend. At time $t=0$, all odd agents are endowed with
$\alpha H$ units of  an unbacked, inconvertible currency,
and all even units are endowed with $(1-\alpha) H$ units
of currency, where $\alpha \in [0,1]$.  The currency is
denominated in dollars and  is perfectly durable.
Currency is the only object that agents are permitted
to carry over from one period to the next.
Let $p_t$ be the price level at time $t$, denominated in units
of dollars per time $t$ consumption good.

\medskip
\noindent{\bf a.}  Define an {\it equilibrium with valued fiat currency}.

\medskip
\noindent{\bf b.}  Let an ``eventually stationary'' equilibrium with
valued fiat currency be one in which there exists a $\bar t$ such
that for $t \geq \bar t$, the equilibrium allocation to each
type of agent is of period 2 (i.e., for each type
of agent, the allocation is a periodic sequence that oscillates
between two values).  Show that for each value of $\alpha \in [0,1]$,
there exists such an equilibrium. Compute this equilibrium.

\medskip
\noindent {\it Exercise \the\chapternum.7}\quad {\bf Capital overaccumulation}
\medskip\noindent
  Consider an environment with
equal numbers $N$ of two types of agents, odd and even,
who have endowment sequences
$$\eqalign{ \{y_t^o\}_{t=0}^\infty &= \{ 1 - \varepsilon,
\varepsilon, 1-\varepsilon, \varepsilon, \ldots\} \cr
\noalign{\smallskip}
      \{y_t^e\}_{t=0}^\infty &= \{ \varepsilon, 1-\varepsilon, \varepsilon,
         1-\varepsilon, \ldots \} .\cr }$$
Here, $\varepsilon$ is a small positive number that is very close
to zero.  Households of each type $h$ order consumption sequences
by $\sum_{t=0}^\infty \beta^t \ln(c_t^h)$ where $\beta \in (0,1)$.
The one good in the model is storable.  If a nonnegative amount
$k_t$ of the good is stored at time $t$, the outcome
is that $\delta k_t$ of the good is carried into period $t+1$,
where $\delta \in (0,1)$.  Households are free to store
nonnegative amounts of the good.

\medskip
\noindent{\bf a.}  Assume that there are no markets.  Households
are on their own.  Find the autarkic consumption allocations
and storage sequences for the two types of agents.  What
is the total per-period storage in this economy?

\medskip
\noindent{\bf b.}  Now assume that there exists a fiat
currency, available in fixed supply of $M$, all of which
is initially equally distributed among the even agents.  Define
an equilibrium with valued fiat currency.  Compute
a stationary equilibrium with valued fiat currency.  Show
that the associated allocation Pareto dominates the one
you computed in part a.

\medskip
\noindent{\bf c.}  Suppose that in the storage technology
$\delta=1$ (no depreciation) and that there is a fixed supply
of fiat currency, initially distributed as in part b.  Define
an ``eventually stationary'' equilibrium.  Show that there
is a continuum of eventually stationary equilibrium price levels
and allocations.

\medskip
\noindent{\it Exercise \the\chapternum.8}\quad {\bf Altered endowments}
\medskip\noindent
Consider a Bewley model identical to the one in the text,
except that now the
 odd and even  agents are endowed with the sequences
$$ \eqalign{ y_t^0 &= \{1 - F, F, 1-F, F, \ldots \} \cr
             y_t^e & =\{F, 1- F, F, 1-F, \ldots \}, \cr }$$
where  $ 0 < F< (1-c^o)$, where $c^o$ is the solution of equation \Ep{od9}.

   Compute the equilibrium allocation and price level.  How
do these objects vary across economies with different
levels of $F$?  For what values of $F$ does a stationary
equilibrium with valued fiat currency exist?

%\vfill\eject
\medskip
\noindent{\it Exercise \the\chapternum.9}\quad {\bf Inside money}
\medskip\noindent
   Consider an environment with equal numbers $N$ of two types
of households, odd and even, who have endowment sequences
$$\eqalign{ \{y_t^o\}_{t=0}^\infty &= \{1,0,1, 0, \ldots\} \cr
\noalign{\smallskip}
     \{y_t^e\}_{t=0}^\infty &= \{0, 1,0,1, \ldots\}. \cr}$$
Households of type $h$ order consumption sequences by
$\sum_{t=0}^\infty \beta^t u(c_t^h) .$  At the beginning
of time $0$, each even agent is endowed with $M$ units
of an unbacked fiat currency and owes  $F$ units of consumption
goods; each odd agent is owed $F$ units of consumption
goods and owns $0$ units of currency.  At time $t \geq 0$,
a household of type $h$ chooses to carry over
$m_t^h \geq 0$ of currency from time $t$ to $t+1$.
(We start households out with these debts or assets at time $0$
to support a stationary equilibrium.)  Each period $t \geq 0$,
 households can issue indexed one-period debt in amount
$b_t$, promising to pay off $b_t R_t$ at $t+1$, subject to
the constraint that $b_t \geq -F/R_t$, where $F>0$ is a parameter
characterizing the borrowing constraint and $R_t$
is the rate of return on these loans between time
$t$ and $t+1$.  (When $F = 0$, we get the Bewley-Townsend model.)
A household's period $t$ budget constraint is
$$ c_t + m_t/p_t + b_t = y_t + m_{t-1}/p_t + b_{t-1}R_{t-1},$$
where $R_{t-1}$ is the gross real rate of return on indexed debt
between time $t-1$ and  $t$.  If $b_t<0$, the  household is borrowing
at $t$, and if $b_t>0$, the household is lending at $t$.
\medskip
\noindent{\bf a.}  Define a competitive equilibrium in which
valued fiat currency and private loans coexist.
\medskip
\noindent{\bf b.}  Argue that, in the equilibrium defined in part
a, the real rates of return on currency and indexed debt must be
equal.
\medskip
\noindent{\bf c.}  Assume that $0 < F< (1-c^o)/2$, where $c^o$ is
the solution of equation \Ep{od9}.  Show that there exists a
stationary equilibrium with a constant price level and that the
allocation equals that associated with the stationary equilibrium
of the $F=0$ version of the model. How does $F$ affect the price
level?  Explain.

\medskip
\noindent{\bf d.}  Suppose that $F=(1-c^o)/2$.  Show that there is
a stationary
 equilibrium with private loans but that fiat currency is valueless in
 that equilibrium.

\medskip
\noindent{\bf e.}  Suppose that $F={\beta \over 1+ \beta}$.  For a
stationary equilibrium, find an equilibrium  allocation and
interest rate.
\medskip
\noindent{\bf f.}  Suppose that $F \in [(1-c^o)/2, {\beta \over
1+\beta}]$. Argue that there is a stationary equilibrium (without
valued currency) in which the real rate of return on debt is $R
\in (1, \beta^{-1}).$

\medskip
\noindent{\it Exercise \the\chapternum.10}\quad {\bf Initial conditions and inside
money}

\medskip
\noindent
  Consider a version of the preceding model in which each
odd person is initially endowed with no currency and no IOUs,
and each even person is initially endowed with $M/N$ units of currency
but no IOUs.  At every time $t\geq 0$, each agent can issue one-period
IOUs promising to pay off $F/R_t$ units of consumption
in period $t+1$, where $R_t$ is the gross real rate of return on currency
or IOUs between periods $t$ and $t+1$.  The parameter
$F$ obeys the same restrictions imposed in exercise {\it \the\chapternum.9\/}.

\medskip
\noindent{\bf a.}  Find an equilibrium with valued fiat
currency in which the tail of the  allocation for $t \geq 1$ and
the tail of the price level sequence, respectively, are
identical with that found in exercise {\it \the\chapternum.9\/}.
\medskip
\noindent{\bf b.}  Find the price level, the allocation, and
the rate of return on currency and consumption loans at period
$0$.

\medskip
\index{real bills doctrine}

\noindent{\it Exercise \the\chapternum.11}\quad {\bf Real bills experiment}
\medskip
\noindent
  Consider a version of  exercise {\it \the\chapternum.9\/}.  The initial
conditions and restrictions on borrowing are as described
in exercise {\it \the\chapternum.9\/}.  However, now
the government augments the currency stock by an
``open market operation'' as follows:
In period $0$, the government issues $\bar M - M$ units
per each odd agent for the purpose of purchasing $\Delta$ units
of IOUs issued at time $0$ by the even agents.  Assume
that $0 < \Delta<  F$.  At each time $t\geq 1$, the
government uses any net real interest payments
from  its stock IOUs from the private
sector to decrease the outstanding stock of currency.  Thus,
the government's budget constraint sequence is
$$ \eqalign{ {\bar M - M \over p_0} & = \Delta,  \quad t=0, \cr
\noalign{\medskip}
              {\bar M_t - \bar M_{t-1}\over p_t}
                & = -(R_{t-1} -1) \Delta
                  \quad t \geq 1 .\cr}  $$
Here,$R_{t-1}$ is the gross rate of return on consumption
loans from $t-1$ to $t$, and $\bar M_t$ is the total stock of
currency outstanding at the end of time $t$.

\medskip
\noindent{\bf a.}  Verify that there exists a stationary equilibrium
with valued fiat currency in which the allocation has the
form \Ep{od8} where $c_0$ solves equation \Ep{od9}.

\medskip
\noindent{\bf b.}  Find a formula for the price level in this
stationary equilibrium.  Describe how the price level varies
with the value of $\Delta$.
\medskip
\noindent{\bf c.}  Does the ``quantity theory of money'' hold
in this example?
