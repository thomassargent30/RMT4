
\input grafinp3
%\input grafinput8
\input psfig
%\eqnotracetrue

%\showchaptIDtrue
%\def\@chaptID{8.}

%\eqnotracetrue

\hbox{}
\def\toone{{t+1}}
\def\ttwo{{t+2}}
\def\tthree{{t+3}}
\def\Tone{{T+1}}
\def\TTT{{T-1}}
\def\rtr{{\rm tr}}

%%%\chapter{Fiscal Policies in the Nonstochastic Growth Model\label{linappro}}
\chapter{Fiscal Policies in a Growth Model\label{linappro}}
\footnum=0

\section{Introduction}

This chapter studies  effects of technology and fiscal shocks
on equilibrium outcomes in a nonstochastic growth model. We
use the model to state some classic doctrines about the effects of various types of taxes and also
 as a laboratory to exhibit  numerical
techniques for approximating equilibria and to display the
structure of dynamic models in which decision makers have perfect
foresight about future government decisions.  Foresight imparts   effects on prices and allocations that precede
 government actions that cause them.

Following Hall (1971), we augment a nonstochastic version of the
standard growth model with a government that purchases a stream of
goods and that finances itself with an array of distorting flat-rate
taxes. We take government behavior as exogenous,\NFootnote{In
chapter
\use{optax}, we take up a version of the model in which the government
chooses taxes to maximize the utility of a representative consumer.}
which means that for us a {\it government\/} is
 simply a list of sequences for government purchases $\{g_t\}_{t=0}^\infty$
and   taxes $\{\tau_{ct},  \tau_{kt}, \tau_{nt},
\tau_{ht}\}_{t=0}^\infty$. Here $\tau_{ct}, \tau_{kt}, \tau_{nt}$
are, respectively, time-varying flat-rate rates on consumption,
earnings from capital, and labor earnings;  and $\tau_{ht}$ is a lump-sum tax (a ``head
tax'' or ``poll tax'').

Distorting taxes prevent a competitive equilibrium allocation
from solving a planning problem. Therefore, to compute an equilibrium allocation and price
system, we
solve a system of nonlinear difference equations consisting of the
first-order conditions for decision makers and the other
equilibrium conditions.  We first use a method called shooting. It  produces an accurate
approximation. Less accurate but in some ways more revealing
approximations can be found by following Hall (1971), who solved a
 linear approximation to the equilibrium conditions.
We  apply the lag operators described in appendix \use{appa1} of chapter \use{timeseries}  to find and represent the solution in a way that is
especially helpful in revealing the dynamic effects of perfectly
foreseen alterations in   taxes and expenditures and  how current values of
 endogenous
variables respond to paths of future exogenous variables.\NFootnote{See  Sargent
(1987a) for a more comprehensive account of lag operators.  By
using lag operators, we extend Hall's results to allow arbitrary
fiscal policy paths.}  \index{shooting algorithm}%

\section{Economy}

\subsection{Preferences, technology, information}

There is no uncertainty,  and decision makers have perfect foresight.
A representative household has preferences over nonnegative streams of
a single
consumption good $c_t$ and leisure $1-n_t$ that are ordered by
$$ \sum_{t=0}^\infty \beta^t U(c_t, 1-n_t), \quad \beta \in (0,1) \EQN ric1 $$
where $U$ is strictly increasing in $c_t  $ and $1-n_t $, twice
continuously differentiable, and strictly concave. We require that $c_t \geq 0 $
and $n_t \in [0,1]$.  We'll
typically assume that $U(c,1-n) = u(c)+v(1-n)$. Common alternative
specifications in the real business cycle literature are $U(c,1-n)
= \log c + \zeta \log (1-n)$ and $U(c,1-n) = \log c + \zeta
(1-n)$.\NFootnote{See Hansen (1985) for a comparison of these two specifications.  Both of these specifications
fulfill the necessary conditions for the existence of a balance growth path set forth by King, Plosser, and Rebelo (1988), which require that income and
substitution effects cancel in an appropriate way.}
\auth{King, Robert G.}%
\auth{Plosser, Charles I.}%
\auth{Rebelo, Sergio}%
 We shall also focus on
another frequently studied special case that
 has $v=0$ so that $U(c,1-n)=u(c)$.

The technology is
$$\EQNalign{ g_t + c_t + x_t &  \leq F(k_t, n_t) \EQN ric02;a \cr
   k_{t+1} & = (1-\delta) k_t + x_t \EQN ric02;b \cr }  $$
where $\delta \in (0,1)$ is a depreciation rate,  $k_t$ is the stock
of physical capital, $x_t$ is gross investment,
and $F(k,n)$ is a linearly homogeneous  production
function with positive and decreasing marginal products of capital and
labor.\NFootnote{In section \use{sec:growthKPR}, we modify the production function to admit
labor augmenting technical change, a form that respects the King, Plosser, and Rebelo (1988)
necessary conditions for the existence of a balance growth path.}
It is sometimes convenient to eliminate $x_t$ from \Ep{ric02}  and express
the technology as
$$ g_t + c_t + k_{t+1} \leq F(k_t, n_t) + (1-\delta) k_t. \EQN ric2 $$


\subsection{Components of a competitive equilibrium}
There is a competitive equilibrium with all trades occurring at
time $0$. The household owns  capital, makes investment decisions,
and rents capital and labor to a representative production firm.
The representative firm uses capital and labor to produce goods
with the production function $F(k_t, n_t)$. A {\it price system\/}
is a triple of sequences $\{q_t, \eta_t, w_t\}_{t=0}^\infty$, where
$q_t$ is the time $0$ pretax price of one unit of investment or
consumption at time $t$ ($x_t$ or $c_t$),  $\eta_t$ is the pretax
price at time $t$ that the household receives from the firm for
renting capital at time $t$, and $w_t$ is the pretax price at
time $t$ that the household receives for renting labor to the firm
at time $t$. The prices $w_t$ and $ \eta_t$ are  expressed in terms
 of
time $t$ goods, while $q_t$ is expressed in terms of the  numeraire at time $0$.





%\section{Budget constraints and competitive equilibrium}

%We study competitive equilibria with trading at time $0$.
We extend the chapter \use{recurge} definition of  a competitive
 equilibrium   to include activities of a  government.
We say that a government expenditure and tax plan that satisfy a
budget constraint is {\it budget feasible}.   A set of competitive
equilibria is   indexed by alternative budget-feasible government
policies.

The household faces the budget constraint:
$$\eqalign{ & \sum_{t=0}^\infty q_t \left\{(1+\tau_{ct}) c_t +
   [k_{t+1} - (1-\delta) k_t] \right\}  \cr
 &      \leq \sum_{t=0}^\infty  q_t  \left\{ \eta_t k_t -\tau_{kt}(\eta_t - \delta) k_t +
      (1-\tau_{nt})w_t n_t -  \tau_{ht} \right\}. \cr } \EQN ric3 $$
Here we have assumed that the government gives  a depreciation allowance $\delta k_t$
from the gross rentals on capital $\eta_t k_t$ and so collects taxes $\tau_{kt} (\eta_t - \delta) k_t$
on rentals from capital.
The government faces the budget constraint
$$\sum_{t=0}^\infty q_t g_t   \leq
        \sum_{t=0}^\infty  q_t \Bigl\{ \tau_{ct}  c_t
  +
\tau_{kt} (\eta_t -\delta) k_t +  \tau_{nt} w_t
      n_t +  \tau_{ht} \Bigr\} .  \EQN ricg $$
There is a sense in which we have given the government access to
too many kinds of taxes, because when lump-sum taxes are available,
the government  should not use any  distorting taxes. We include all of these
taxes because, like Hall (1971), we want a framework that is
sufficiently general to allow us to analyze  how the various taxes
distort production and consumption decisions.

\section{The term structure of interest rates}\label{sec:termstructure_growth}%
\index{term structure of interest rates}
The price system $\{q_t\}_{t=0}^\infty$ evidently embeds within it a term structure of interest rates.
It is convenient to represent $q_t$ as
$$ \EQNalign {q_t & = q_0 {\frac{q_1}{q_0}}{\frac{q_2}{q_1}} \cdots {\frac{q_t}{q_{t-1}}} \cr
   & = q_0 m_{0,1} m_{1,2} \cdots m_{t-1,t} \cr} $$
where $m_{t,t+1} = {\frac{q_{t+1}}{q_t}}$. \index{discount factor}%
We can represent the one-period {\it discount factor\/} $m_{t,t+1}$ as
$$ m_{t,t+1} = R_{t,t+1}^{-1} = {\frac{1}{1+r_{t,t+1}}} \approx \exp(- r_{t,t+1}) . \EQN mdefff $$
Here $R_{t,t+1}$ is the gross one-period rate of interest between $t$ and $t+1$ and
$r_{t,t+1}$ is the net one-period rate of interest between $t$ and $t+1$.
Notice  that $q_t$ can also be expressed as
$$\EQNalign{ q_t &= q_0 \exp(-r_{0,1}) \exp(-r_{1,2}) \cdots \exp(-r_{t-1,t}) \cr
              & = q_0 \exp \bigl( - (r_{0,1} + r_{1,2} + \cdots + r_{t-1,t} )\bigr) \cr
              & = q_0 \exp (-t r_{0,t} ) \cr } $$
where
$$ r_{0,t} = t^{-1} (r_{0,1} + r_{1,2} + \cdots +  r_{t-1,t} ). \EQN expectations_theory $$
Here $r_{0,t} $ is the net $t$-period rate of interest between $0$ and $t$. Since $q_t$ is the time $0$ price of one unit
of time $t$ consumption, $r_{0,t}$ is said to be the yield to maturity on a `zero coupon bond' that matures at time $t$.
A zero coupon bond promises no coupons before the date of maturity and  pays only  the principal due at the date of maturity.     Equation \Ep{expectations_theory}
expresses the expectations theory of the term structure of interest rates, according to which interest rates on $t$-period (long) loans are
averages of  rates on one period (short) loans expected to prevail over the horizon of the long loan. \index{term structure!of interest rates}%
More generally,
the $s$-period long rate at time $t$ is
$$r_{t,t+s} = {1 \over s} ( r_{t,t+1} + r_{t+1,t+2} + \cdots + r_{t+s-1,t+s}). \EQN expectations_theory_2 $$
A graph of $r_{t,t+s}$ against $s$ for $s=1, 2, \ldots, S$ is called the (real) yield curve at $t$.\index{expectations theory of term structure}%
\index{zero coupon bonds}%
%
%An insight about the expectations theory of the term structure of interest rates can be gleaned from computing one-period holding period
%yields on zero coupon bonds of maturities $1, 2, \ldots$.  Someone who at time $0$ purchases one unit of time $1$ consumption for  $q_1$ units
%of time $0$ consumption and  then sells it
%at time $1$ for $1 = {\frac{q_1}{q_1}}$ units of the time $1$ consumption good
%  earns a one-period return of ${\frac{1-q_1}{q_1}}$.  Someone who at time $0$ purchases one
%unit of time $t$ consumption at a price $q_t$ of the time $0$ consumption good and then sells it at time $1$ for a price of ${\frac{q_t}{q_1}}$ of the
% time $1$ consumption good
%earns  a one-period return of ${\frac{q_t/q_1 - q_t}{q_t}} = {\frac{1-q_1}{q_1}}$. Evidently,  at time $0$ the one-period yield is {\it identical\/} for pure discount
%bonds of {\it all\/} maturities.  More generally, at time $t$ the one-period holding period yield on zero coupon bonds of all maturities
%equals ${\frac{1 - q_{t+1}}{q_{t+1}}}$.  One way to characterize the expectations theory of the term structure of interest rates is by the requirement
%that the price vector $\{q_t\}_{t=0}^\infty$ of zero coupon bonds must be such that one-period holding period yields are equated across zero coupon bonds of
%all maturities.  Note also how the price system $\{q_t\}_{t=0}^\infty$ contains forecasts of one-period holding period yields on zero coupon bonds of all maturities
%at all dates $t \geq 0$. \index{holding period yields}%



An insight about the expectations theory of the term structure of interest rates can be gleaned from computing  gross one-period holding period
 returns on zero coupon bonds of maturities $1, 2, \ldots$.  Consider the gross return earned by someone who at time $0$ purchases one unit of time $t$ consumption for  $q_t$ units
of the numeraire and  then sells it
at time $1$.  The person pays ${\frac{q_t}{q_0}}$ units of time $0$ consumption goods to earn ${\frac{q_t}{q_1}}$ units of time $1$
 consumption goods.  The gross rate of return from this trade  measured in time $1$ consumption goods per unit of time $0$ consumption goods
 is ${\frac{q_0}{q_1}}$, which does not depend on the date $t$ of the good bought at time $0$ and then sold at time $1$.
   Evidently,  at time $0$ the one-period return is {\it identical\/} for pure discount
bonds of {\it all\/} maturities $t \geq 1$.  More generally, at time $t$ the one-period holding period gross return on zero coupon bonds of all maturities
equals ${\frac{q_{t}}{q_{t+1}}}$.

A  way to characterize the expectations theory of the term structure of interest rates is by the requirement
that the price vector $\{q_t\}_{t=0}^\infty$ of zero coupon bonds must be such that one-period holding period yields are equated across zero coupon bonds of
all maturities.  Note also how the price system $\{q_t\}_{t=0}^\infty$ contains forecasts of one-period holding period yields on zero coupon bonds of all maturities
at all dates $t \geq 0$. \index{holding period yields}%
\index{one-period returns}%


In subsequent sections, we'll indicate how the growth model with taxes and government expenditures  links the  term structure of interest
rates to aspects of government fiscal policy.

\section{Digression: sequential version of government budget constraint}

 We have used the time $0$ trading abstraction
 described in chapter \use{recurge}.  %That chapter describes another
% formulation with
% sequential trading of one-period Arrow securities that  supports
% the same equilibrium allocations. A formulation with
Sequential
 trading of one-period risk-free debt can also  support the equilibrium allocations that we
 shall study in this chapter. It  is especially useful explicitly to describe  the sequence of one-period
 government debt that is implicit in the equilibrium tax policies here.

We presume that the government enters period $0$ with no government debt.\NFootnote{Letting
$B_{-1}=0$ be the  government debt owed at time $-1$ allows us to apply equation
\Ep{govt_budget_sequence } to date $t=0$ too.}
Define total tax collections as $T_t = \tau_{ct} c_t +
    \tau_{kt} (\eta_t-\delta) k_t +  w_t      \tau_{nt}n_t +  \tau_{ht}$ and express the government budget constraint
      as
$$ \sum_{t=0}^\infty q_t (g_t - T_t) = 0 . \EQN govt_budget_AD $$
This can be written as
$$ g_0 - T_0 = \sum_{t=1}^\infty {\frac{q_t}{q_0}} (T_t - g_t), $$
which states that the government deficit $g_0 - T_0$ at time $0$  equals
the present value of future government surpluses.  Here
 $B_0 \equiv \sum_{t=1}^\infty {\frac{q_t}{q_0}} (T_t - g_t) $ is
the value of government debt issued at time $0$, denominated in units of time $0$ goods.
%Suppose that all of this debt is one-period in duration.
We can use this definition of $B_0$ to deduce
$$ B_0 {\frac{q_0}{q_1}} = T_1 - g_1 + \sum_{t=2}^\infty {\frac{q_t}{q_1} } (T_t - g_t ) $$
or, by recalling from the previous subsection that  $R_{0,1} \equiv {\frac{q_0}{q_1}}$ denotes the gross one-period  real interest rate
between time 0 and time 1,
$$ B_0 R_{0,1} = T_1 - g_1 + B_1 $$
where now
$$ B_1 \equiv \sum_{t=2}^\infty {\frac{q_t}{q_1} } (T_t - g_t ) $$
is the value of government debt issued in period 1 in units of time 1 consumption.
Iterating this construction forward  gives us a sequence of period-by-period government budget constraints
$$ g_t + R_{t-1,t} B_{t-1} = T_t + B_t \EQN govt_budget_sequence $$
for $t \geq 1$,  where $R_{t-1,t} = {\frac{q_{t-1}}{q_t}}$ and
$$ B_t \equiv \sum_{s=t+1}^\infty {\frac{q_s}{q_t} } (T_s - g_s ).\EQN govt_debt_sequence $$
The left side of equation \Ep{govt_budget_sequence} is time $t$ government expenditures including
interest and principal payments on its debt, while the right side is total revenues including those raised by
issuing new one-period debt in the amount $B_t$.



Thus, embedded in a government  policy that satisfies \Ep{ricg} is a sequence of one-period
government debts satisfying \Ep{govt_debt_sequence}. The value of government debt at $t$ is the present value
of government surpluses from date $t+1$ onward. Equation \Ep{govt_debt_sequence} states that government {\it debts\/}
at time $t$ signal future {\it surpluses\/}.

Equation \Ep{govt_budget_sequence} can be represented as
$$
B_t - B_{t-1} = g_t - T_t + r_{t-1,t} B_{t-1}. \EQN govt_budget_sequence_2
$$
Here $g_t - T_t$ is what is commonly called  either the {\it net-of-interest\/} government deficit
or the {\it operational\/} government deficit or the {\it primary\/} government deficit, while $r_{t-1,t} B_{t-1}$ are net interest payments on the government
debt and  $ g_t - T_t + r_{t-1,t} B_{t-1}$ is the gross-of-interest government deficit.
Equation \Ep{govt_budget_sequence_2} asserts that the change in government debt equals the
gross-of-interest government deficit.
\index{government deficit!net of interest}%
\index{government deficit!gross of interest}%
\index{government deficit!operational}%
\index{government deficit!primary}%

The Arrow-Debreu budget constraint \Ep{govt_budget_AD} automatically enforces a `no-Ponzi scheme' condition on the path of
government debt $\{B_t\}$.
To see this, first recall that ${\frac{q_s}{q_t}} =  R_{t,t+1}^{-1} \cdots R_{s-1,s}^{-1}$ and
write  \Ep{govt_debt_sequence}
as
$$ B_t = \sum_{s=t+1}^T {\frac{q_s}{q_t}} (T_s - g_s )+ \sum_{s=T+1}^\infty {\frac{q_s}{q_t} } (T_s - g_s ) $$
or
$$ B_t \equiv \sum_{s=t+1}^T {\frac{q_s}{q_t}} (T_s - g_s )+ {\frac{q_{T}}{q_t}}B_{T} $$
or
$$ B_t \equiv \sum_{s=t+1}^T {\frac{q_s}{q_t} } (T_s - g_s )+ R_{t,t+1}^{-1} \cdots R_{T-1,T}^{-1} B_{T}. $$
% The condition that $B_t$ is finite implies that $\lim_{T\rightarrow +\infty} R_{t,t+1}^{-1} \cdots R_{T-1,T}^{-1} B_{T} =0$.
An argument like that in subsection \use{sec:no_arbitrage100} can be applied to show that in an equilibrium $\lim_{T \rightarrow +\infty} q_T B_{T+1} = 0$.
\subsection{Irrelevance of maturity structure of government debt}

At time $t$, the government issues a list of {\it bonds} that in the aggregate  promise to pay a stream $\{\xi^t_s\}_{s=1}^\infty$  of goods at time $s > t$ satisfying
$$
B_t = \sum_{s=t+1}^\infty {\frac{q_s}{q_t}} \xi^t_s  .\EQN eqn_value_of_bonds
$$
The only restriction that our model puts on the term structure of payments $\{\xi^t_s\}_{s=1}^\infty$ is that it must satisfy
$$ \sum_{s=t+1}^\infty {\frac{q_s}{q_t}} \xi^t_s =  \sum_{s=t+1}^\infty {\frac{q_s}{q_t} } (T_s - g_s )  \equiv B_t \EQN term_structure_payments
$$
The model of this chapter asserts that one payment stream $\{\xi^t_s\}_{s=t+1}^\infty$  that satisfies \Ep{term_structure_payments} is as good as any other. The model pins down  the total value of the continuation government IOU stream $\{\xi^t_s\}_{s=t+1}^\infty$ at each $t$, but it leaves the
maturity structure of payments, whether early or late, for example, undetermined.\NFootnote{For models that  restrict the maturity structure of government
debt by imposing more imperfections than we does this chapter, see Lucas and Stokey (1983), Angeletos (2002),  Buera and Nicolini (2004), and Shin (2007). Lucas and Stokey show how to set the maturity
structure  of debt payments to
induce a sequence of authorities responsible for choosing flat rate taxes on labor to implement a Ramsey plan.   Angeletos (2002), Buera and Nicolini (2004),
and Shin (2007) use variations over time in the maturity  structure of risk-free government debt  to complete markets.}
\auth{Stokey, Nancy L.}%
\auth{Lucas, Robert E., Jr.}%
\auth{Buera, Francisco}%
\auth{Nicolini, Juan Pablo}%
\auth{Angeletos, George-Marios}%
\auth{Shin, Yongseok}%
Two polar examples of maturity structures of the government debt are:

\medskip
\item{1.}  All debt consists of  {\it one-period pure discount bonds\/} that are rolled over every period:
%$$ z_{jt} = \cases{ 0 & if $t \leq T-1$ \cr
%                    \overline z_j & if $ t \geq T$ \cr} \EQN path1 $$

$$ \xi^t_s = \cases{ \bar \xi^t & if $s=t+1 $\cr
                    0 & if $ s \geq t+2$ \cr}  $$
where $\bar \xi^t$ satisfies ${\frac{q_{t+1}}{q_t}} \bar \xi^t = B_t$.

\medskip
\item{2.}  All debt consists of {\it consols\/} that in the aggregate promise to pay a constant total coupon $\hat \xi^t$ for $s \geq t+1$, where
$\hat \xi^t$ satisfies
$$ \hat \xi^t\sum_{s=t+1}^\infty {\frac{q_s}{q_t}}  =  B_t.
$$
\index{consol}%

\noindent The sequence of period-by-period net returns on the government debt $\{ r_{t,t+1} B_t\}_{t=0}^\infty$ is independent of
the government's choice of sequences $\{\{\xi^t_s\}_{s=t+1}^\infty\}_{t=0}^\infty$.










\section{Competitive equilibria with distorting taxes}

 A representative household
chooses a sequence $\{c_t, n_t, k_{t+1}\}_{t=0}^\infty $  to maximize \Ep{ric1}
subject to \Ep{ric3}.  A representative firm chooses $\{k_t,
n_t\}_{t=0}^\infty$ to maximize $\sum_{t=0}^\infty q_t [ F(k_t,
n_t)    - \eta_t k_t - w_t n_t]$.\NFootnote{Note the contrast with
the setup in chapter
\use{growth1}, which has two types of firms.  Here we assign to the
household the physical investment decisions made by the type II
firms of chapter \use{growth1}.} A budget-feasible government
policy is  an expenditure plan $\{g_t\}_{t=0}^\infty $  and a   tax plan that
satisfy \Ep{ricg}. A feasible allocation is a sequence $\{c_t,
x_t,  n_t, k_t\}_{t=0}^\infty $ that satisfies \Ep{ric2}.

\medskip
\noindent{\sc Definition:} A {\it competitive equilibrium with distorting taxes\/}
 is a budget-feasible
government policy, a feasible allocation,
and a price system such that, given the price system and the government
policy, the allocation solves the household's problem and the firm's problem.
\medskip

\subsection{The household:
no-arbitrage and asset-pricing formulas}\label{sec:no_arbitrage100}%
A no-arbitrage argument implies  a restriction on prices
and tax rates across time from which there emerges a formula for
the ``user cost of capital'' (see Hall and Jorgenson, 1967).
Collect terms in similarly dated capital stocks and thereby
rewrite the household's budget constraint \Ep{ric3} as
%\offparens
$$\eqalign{ & \sum_{t=0}^\infty q_t\bigl[(1+\tau_{ct}) c_t\bigr]
\leq
 \sum_{t=0}^\infty q_t  (1-\tau_{nt})w_t
    n_t - \sum_{t=0}^\infty q_t \tau_{ht} \cr
 &+ \sum_{t=1}^\infty \bigl[ ((1-\tau_{kt})(\eta_t- \delta) + 1)q_t  -
   q_{t-1} \bigr] k_t  \cr &
     + \bigl[  \left(1-\tau_{k0}\right) (\eta_0 - \delta)
   + 1 \bigr] q_0k_0
  - \lim_{T \rightarrow \infty}  q_T k_{T+1}
 \cr} \EQN ric0a $$
 %\autoparens
The terms $\bigl[ \left(1-\tau_{k0}\right) (\eta_0 - \delta)
   + 1 \bigr] q_0 k_0$ and
  $- \lim_{T \rightarrow \infty}  q_T k_{T+1}$
remain after creating the weighted   sum in $k_t$'s for $t \geq
1$. \auth{Hall, Robert E.} \auth{Jorgenson, Dale}%

The household inherits a given $k_0$ that it takes as an
initial condition, and it is free to choose any sequence
$\{c_t, n_t, k_{t+1}\}_{t=0}^\infty$ that satisfies \Ep{ric0a}
where all prices and tax rates are taken as given.
The objective of the household is to maximize lifetime
utility \Ep{ric1}, which is increasing in consumption
$\{c_t\}_{t=0}^\infty$ and,
for one of our preference specifications below, also
increasing in leisure $\{1 -n _t\}_{t=0}^\infty$.
%Since the household can only consume nonnegative amounts
%of consumption and nonnegative amounts of leisure, we have
%the restrictions that $c_t\geq 0$ and $n_t\in[0,1]$ for all $t$.
%The household has no direct preferences over the
%sequence of capital
%$\{k_{t+1}\}_{t=0}^\infty$ that serves  as a vehicle
%for achieving the household's goal of utility maximization.

All else equal, the household would be happier with larger
values on the right side of \Ep{ric0a}, preferably plus infinity,
which would enable it to purchase unlimited amounts of consumption
goods. Because resources are finite, we know that the right side
of the household's budget constraint must be bounded in an
equilibrium. This fact leads to an important restriction on the
price and tax sequences.  If the right side of the household's
budget constraint is to be bounded, then the terms multiplying
$k_t $ for $t \geq 1$ must all  equal zero because if any  of them
were strictly positive (negative) for some date $t$, the household
could make the right side of \Ep{ric0a} an arbitrarily large
positive number by choosing an arbitrarily large positive (negative)
value of $k_t$. On the one hand, if one such term were strictly
positive for some date $t$, the household could purchase an
arbitrarily large capital stock $k_t$ assembled at time $t-1$ with
a present-value cost of $q_{t-1}  k_t$ and then
sell the rental services and the undepreciated part of that
capital stock to be delivered at time $t$, with a present-value
income of $[ (1-\tau_{kt})(\eta_t -\delta)  + 1]q_t k_t$.
If such a transaction were to  yield a strictly positive profit, it would
offer the consumer a pure arbitrage opportunity and the right side of \Ep{ric0a} would
become unbounded. On the other hand, if there is one term
multiplying $k_t$ that is strictly negative for some date $t$, the
household can make the right side of \Ep{ric0a} arbitrarily large
and positive by ``short selling'' capital by setting $k_t<0$. The
household could turn to purchasers of capital assembled at time
$t-1$ and sell ``synthetic'' units of capital to them. Such a
transaction need not involve any actual physical capital:  the
household could merely undertake trades  that would give the other
party to the transaction  the same costs and incomes as those
associated with purchasing capital assembled at time $t-1$. If
such \idx{short sales} of capital yield strictly positive profits,
it would provide the consumer with a  pure arbitrage opportunity and the right side of
\Ep{ric0a} would become unbounded. Therefore, the terms
multiplying $k_t$
 must {\it equal\/} zero for all $t\geq 1$, so that
%$$ q_t   = q_{t+1}  (1-\delta)
%  + \eta_{t+1} (1-\tau_{kt+1}) \EQN ric4euler  $$
$$ {\frac{q_t}{q_{t+1}}}    =  \left[ (1-\tau_{kt+1})( \eta_{t+1} -\delta) + 1 \right] \EQN ric4euler  $$
for all $t \geq 0$. These are
zero-profit or no-arbitrage conditions.
%Unless these no-arbitrage conditions hold, the household
%is not optimizing.
We have derived these conditions by using
only the weak property that $U(c,1-n)$ is
increasing in consumption (i.e.,  that the household
always prefers more to less).

It remains to be determined how the household sets the last term
on the right side of \Ep{ric0a}, $- \lim_{T \rightarrow \infty}
 q_T k_{T+1}$. According to our preceding argument,
the household would not purchase an amount of capital  that would
make this term  strictly negative in the limit because that would
reduce the right side of \Ep{ric0a} and hence diminish the
household's resources available for consumption. Instead, the
household would like to make this term strictly positive and
unbounded, so that the household could purchase unlimited amounts
of consumption goods. But the market would stop the household from
undertaking such a short sale in the limit, since no party would
like to be on the other side of the transaction. This is obvious
when considering a finite-horizon model where everyone would like
to short sell capital in the very last period because there would
then be no future period in which to fulfil the obligations of
those short sales. Therefore, in our infinite-horizon model, as a
condition of optimality, we  impose the
  terminal condition  that
  $- \lim_{T \rightarrow \infty}  q_T k_{T+1} =0 $.
Once we impose formula \Ep{ric4;a} below linking $q_t$ to $U_{1t}$,
this terminal condition puts the following restriction on the
equilibrium allocation:
$$
  - \lim_{T \rightarrow \infty}  \beta^T {U_{1T}
\over (1+\tau_{cT})} k_{T+1} =0. \EQN termk $$




%%%%%%%%%%%%%%%%%%%%%%%%%%%%%%%%%%%%%%%%%%%%%%%%%%%%%%%%%
%The household inherits a given $k_0$ that it takes as an
%initial condition. Under an Inada condition on $U$, the
%household's marginal condition \Ep{ric4;a} below implies that
%$q_t$ exceeds zero for all $t \geq 0$, and we require that the
%household's choice respect  $k_t \geq 0$. Therefore, as a
%condition of optimality, we  impose the
%  terminal condition  that
%  $- \lim_{T \rightarrow \infty} (1-\tau_{iT}) q_T k_{T+1} =0 $.
%If this condition did not hold,
%the right side of \Ep{ric0a} could be increased.
%%Using formula \Ep{ric4;a} below, we shall express
%Once we impose formula \Ep{ric4;a} that links $q_t$ to $U_{1t}$,
%this terminal condition puts the following restriction on the
%equilibrium allocation:
%$$
%  - \lim_{T \rightarrow \infty} (1-\tau_{iT}) \beta^T {U_{1T}
%\over (1+\tau_{cT})} k_{T+1} =0. \EQN termk $$
%
%Because resources are finite, we know that the right side of the household's
%budget constraint must be bounded in an equilibrium.  This fact leads
%to
%an important restriction on
%the price sequence.
%On the one hand, if the right side of the
%household's budget constraint is to be bounded,
%then
%the terms multiplying $k_t $ for $t \geq 1$ have to  be {\it less\/}
% than or equal
%to zero. On the other hand,
% if the household is ever  to set $k_t >0$ (which
%it will want to do in a competitive equilibrium),  then
%these same terms must be {\it greater\/}
% than or equal to zero for all $t \geq 1$.
%Therefore,  the terms multiplying $k_t$
% must {\it equal\/} zero for all $t\geq 1$:
%$$ q_t (1-\tau_{it})  = q_{t+1} (1-\tau_{it+1}) (1-\delta)
%  + \eta_{t+1} (1-\tau_{kt+1}) \EQN ric4euler  $$
%for all $t \geq 0$.
%These are
%zero-profit or no-arbitrage conditions.
%Unless these no-arbitrage conditions hold, the household
%is not optimizing.  We have derived these conditions by using
%only the weak property that $U(c,1-n)$ is
%increasing in both arguments (i.e.,  that the household
%always prefers more to less).
%%%%%%%%%%%%%%%%%%%%%%%%%%%%%%%%%%%%%%%%%%%%%%%%%%%%%%%%%%%%

The household's initial capital stock $k_0$ is given.  According
to \Ep{ric0a}, its value is
$ [(1-\tau_{k0})(\eta_0 - \delta)+ 1] q_0 k_0$.

\subsection{User cost of capital formula}

The no-arbitrage conditions  \Ep{ric4euler} can be rewritten as
the following expression for the ``user cost of capital'' $\eta_{t+1}$:
%$$ \eta_{t+1}=\left( {1 \over 1 -\tau_{kt+1}}\right) \left[
%q_t  -q_{t+1}(1-
%   \delta ) \right]. \EQN ric1a $$
$$ \eta_{t+1}=\delta + \left( {1 \over 1 -\tau_{kt+1}}\right) \left(
{\frac{q_t}{q_{t+1}}} - 1 \right). \EQN ric1a $$
Recalling from \Ep{mdefff} that $m_{t,t+1}^{-1}= R_{t,t+1} = (1 + r_{t,t+1} ) = {\frac{q_t}{q_{t+1}}}$,
equation \Ep{ric1a} can be expressed as
$$\eta_{t+1} = \delta + \left( {r_{t,t+1} \over 1 -\tau_{kt+1}}\right) . $$
The user cost of capital   takes into account the rate of taxation
of capital earnings, the capital gain or loss from $t$ to $t+1$,
and  a depreciation
cost.\NFootnote{This is a discrete-time version of a continuous-time
 formula derived by Hall and Jorgenson (1967).}


\subsection{Household first-order conditions}
 So long as
the no-arbitrage conditions
\Ep{ric4euler} prevail,
households are indifferent
about how much capital they hold. Recalling that the one-period utility function is
$U(c, 1-n)$, let $U_1 = {\partial U \over \partial c}$ and $U_2 = {\partial U \over \partial 1-n}$
so that ${\partial U \over \partial n } = - U_2$. Then we have that the household's
first-order conditions with respect to $c_t, n_t$ are:
$$ \EQNalign{ \beta^t U_{1t} & = \mu q_t (1+\tau_{ct}) \EQN ric4;a \cr
            \beta^t U_{2t} & \leq \mu q_t  w_t (1- \tau_{nt}),
  \ \ = {\rm if} \ n_t < 1,  \EQN ric4;b \cr}
$$
where $\mu$ is a nonnegative Lagrange multiplier on the  household's
budget constraint \Ep{ric3}.
Multiplying the price system by a positive scalar simply rescales
the multiplier $\mu$, so   we are free to choose a numeraire by setting
$\mu$ to an arbitrary positive number.

\subsection{A theory of the term structure of interest rates}

Equation \Ep{ric4;a} allows us to solve for $q_t$ as a function of consumption
$$ \mu q_t = \beta^t U_{1t} / (1+\tau_{ct} ) \EQN q_t_formula;a $$
or in the special case that $U(c_t, 1-n_t) = u(c_t)$
$$\mu q_t = \beta^t u'(c_t) / (1+\tau_{ct}).   \EQN q_t_formula;b $$
In conjunction with the observations made in subsection \use{sec:termstructure_growth}, these formulas link the term
structure of interest rates to the paths of $c_t, \tau_{ct}$.  The government policy  $\{g_t,  \tau_{ct}, \tau_{nt}, \tau_{kt}, \tau_{ht} \}_{t=0}^\infty$
affects the term structure of interest rates directly via $\tau_{ct}$ and indirectly via its impact on the path for $\{c_t\}_{t=0}^\infty$.


\subsection{Firm}
Zero-profit conditions for the representative firm impose additional
restrictions on equilibrium prices and quantities.
The present value  of the firm's profits
is
$$ \sum_{t=0}^\infty q_t\ \bigl[  F(k_t, n_t) - w_t n_t - \eta_t k_t\bigr]. $$
Applying Euler's theorem on linearly homogeneous functions to $F(k,n)$,
 the firm's
present value is:
$$ \sum_{t=0}^\infty q_t \left[
 (F_{kt}  - \eta_t) k_t  +( F_{nt} -w_t) n_t\right]. $$
No-arbitrage (or zero-profit) conditions are:
$$ \eqalign{ \eta_t & = F_{kt} \cr
              w_t & =  F_{nt} .  \cr }\EQN ric6 $$


\section{Computing equilibria}

The definition of a competitive equilibrium and the concavity
conditions that we have imposed on preferences imply that an
equilibrium is a price system $\{q_t, \eta_t, w_t\}$, a budget feasible
government policy $\{g_t,\tau_t\}\equiv \{g_t, \tau_{ct}, \tau_{nt}, \tau_{kt},
 \tau_{ht}\}$, and an allocation $\{c_t, n_t, k_{t+1}\}$
that solve the system of nonlinear difference equations consisting of \Ep{ric2}, \Ep{ric4euler}, \Ep{ric4}, and \Ep{ric6} subject to the
initial condition that $k_0$ is given and the terminal condition
\Ep{termk}. In this chapter, we shall simplify things by treating  $\{g_t,\tau_t\}\equiv \{g_t, \tau_{ct}, \tau_{nt}, \tau_{kt} \}$ as exogenous
and then use $\sum_{t=0}^\infty q_t \tau_{ht}$ as a slack variable that we choose to balance the government's budget.  We now attack this system of difference
equations.


%We now describe how to compute an equilibrium by
%solving or approximately solving
% a nonlinear difference equation in capital with
%government policy variables  as forcing functions.

\subsection{Inelastic labor supply}
We'll start with the following special case. (The general case
is just a little more complicated, and we'll describe it below.)
Set $U(c,1-n)= u(c)$, so that the household gets no utility from leisure,
and set $n=1$. %As in chapter \use{growth},
We
define $f(k)= F(k,1)$ and
express feasibility as
%$$ c_t = f(k_t) + (1-\delta)k_t - k_{t+1} - g_t. \EQN ric7  $$
$$ k_{t+1} = f(k_t) +(1-\delta) k_t -g_t -c_t. \EQN ric7 $$
Notice that $F_k(k,1)=f'(k)$ and
$F_n(k,1)=f(k)-f'(k)k$.
 Substitute \Ep{ric4;a}, \Ep{ric6}, and \Ep{ric7}
into \Ep{ric4euler} to get
$$
\eqalign{ & { u'\bigl(f(k_t) + (1-\delta)k_t - g_t - k_{t+1}\bigr) \over
    (1+\tau_{ct})}  \cr
& - \beta {u'(f(k_{t+1}) + (1-\delta)k_{t+1} - g_{t+1} - k_{t+2}) \over
    (1+\tau_{ct+1})} \times \cr
&  \left[(1-\tau_{kt+1})(f'(k_{t+1})-\delta) + 1 \right]= 0.
\cr} \EQN ric8 $$ Given the government policy sequences, \Ep{ric8}
is a second-order difference equation in capital. We can also express
\Ep{ric8} as

%%%% FFFFF Ask Francois for help on spacing
 {\ninepoint
$$ u'(c_t) = \beta u'(c_{t+1}) {(1+\tau_{ct}) \over(1+\tau_{ct+1})}
     \left[(1-\tau_{kt+1})(f'(k_{t+1})-\delta) + 1 \right] .\EQN tom100$$
}%endninepoint
%$$ k_{t+1} = f(k_t) +(1-\delta) k_t -g_t -c_t. \EQN tom101 $$

\noindent
 To compute an equilibrium,
we must find a solution of the difference equation \Ep{ric8}
that satisfies
two boundary conditions. As mentioned above, one boundary
condition is supplied by the given level of $k_0$ and the other by
\Ep{termk}.
%The other  boundary condition
%is given by $\lim_{T\rightarrow \infty}
%(1-\tau_{iT})q_T k_{T+1}=0 $,   which we express as
%$$ \lim_{T\rightarrow \infty} \beta^T {u'(c_T)\over (1+\tau_{cT})}
%  k_{T+1} = 0.  \EQN{boundary2}$$
%Equation \Ep{boundary2} plays the role of a terminal condition.
%We want to solve the pair of equations \Ep{tom100} and \Ep{ric7}
 % subject to our two boundary conditions.
%the given initial condition $k_0$ and the terminal
%condition \Ep{boundary2}.
%We'll use the shooting method.
To determine a particular terminal value $k_\infty$, we restrict
the path of government policy so that it converges, a way to impose \Ep{termk}.

\subsection{The equilibrium steady state}

Tax rates and government expenditures serve as  forcing
functions for the difference equations  \Ep{ric7} and \Ep{tom100}.
Let
$z_t =\left[\matrix{g_t &   \tau_{kt} & \tau_{ct} \cr}
\right]'$ and write \Ep{ric8} as
$$ H(k_t, k_{t+1}, k_{t+2}; z_t, z_{t+1}) = 0 . \EQN ric10 $$
To allow convergence to a steady state, we assume government
policies that are eventually constant, i.e., that satisfy
$$\lim_{t\rightarrow \infty}   z_t  = \overline z. \EQN limitz $$
When we actually solve our models, we'll set a date $T$ after
which all components of the forcing sequences that comprise $z_t$
are constant.
%We seek a steady state.
A terminal steady-state capital stock $\overline k$ evidently solves
$$H(\overline k, \overline k, \overline k, \overline z, \overline z)
=0. \EQN steadstk $$
For our model, we can solve \Ep{steadstk} by hand.
In a steady state, \Ep{tom100}
becomes
$$ 1 = \beta[ (1-\overline \tau_k) (f'(\overline k) -\delta) + 1]. $$
Notice that  an eventually constant consumption tax $\overline \tau_{c}$ does not distort
$\overline k$ {\it vis-a-vis\/} its  value in an economy without distorting
taxes.
Letting $\beta = {1 \over 1+\rho}$, we can express the preceding equation as
$$ \delta + {\rho \over 1- \bar \tau_k}
  = f'(\overline k)  . \EQN steadstk $$
 When $\tau_k =0$, equation \Ep{steadstk} becomes
$(\rho+\delta)=f'(\overline k)$, which is a celebrated formula for the
so-called ``augmented  Golden Rule'' capital-labor ratio.\index{Golden rule!augmented}%

When the exogenous sequence  $\{g_t\}_{t=0}^\infty$  converges, the steady state capital-labor ratio that solves $(\rho+\delta)=f'(\overline k)$ is  the
asymptotic value of the capital-labor  ratio that would be approached by
a benevolent planner who chooses $\{c_t, k_{t+1}\}_{t=0}^\infty$ to maximize $\sum_{t=0}^\infty \beta^t u(c_t) $
subject to $k_0$ given and the sequence of constraints $c_t + k_{t+1} + g_t \leq f(k_t) + (1-\delta) k_t$.

\subsection{Computing the equilibrium path with the shooting algorithm}

Having computed the terminal steady state, we are now in a
position to apply the {\it shooting algorithm\/} to compute an
equilibrium path that starts from an arbitrary initial condition
$k_0$, assuming a possibly time-varying path of government policy.\NFootnote{We recommend a suite of computer programs
called {\tt dynare\/}. We have used
{\tt dynare\/}  to execute the numerical experiments described in this chapter.  See
Barillas, Bhandari,
Bigio,  Colacito, Juillard, Kitao, Matthes, Sargent,
and Shin (2012) %Barillas, Bhandari, Colacito,  Kitao,  Matthes, Sargent, and  Shin (2010)
for dynare code that performs these
and other calculations. See $<$http://www.dynare.org$>$.}%
%\auth{Barillas, Francisco}%
%\auth{Bhandari, Anmol}%
%\auth{Colacito, Riccardo}%
%\auth{Kitao, Sagiri}%
%\auth{Matthes, Christian}%
%\auth{Sargent, Thomas J.}%
%\auth{Shin, Yongseok}%
\auth{Barillas, Francisco}%
\auth{Bhandari, Anmol}%
\auth{Colacito, Riccardo}%
\auth{Kitao, Sagiri}%
\auth{Matthes, Christian}%
\auth{Sargent, Thomas J.}%
\auth{Shin, Yongseok}%
\auth{Juillard, Michel}%
\auth{Bigio, Saki}%
The shooting algorithm solves the two-point
boundary value problem by  searching for an initial $c_0$ that
makes the Euler equation \Ep{ric8} and the feasibility condition
\Ep{ric2} imply that $k_S \approx \overline k$, where $S$ is a
finite but large time index meant to approximate infinity and
$\overline k$ is the terminal steady value associated with the
policy being analyzed. We let $T$ be the value of $t$ after which
all components of $z_t$ are constant.
 Here are the steps of the algorithm.\NFootnote{This algorithm proceeds in the spirit of the invariant-subspace
 method (implemented via a  Schur decomposition) for solving the first-order conditions associated with the optimal linear regulator that we described in
 section \use{lagrangianformulation} of chapter \use{dplinear}.}
\index{shooting algorithm}%
%Write \Ep{ric8} as
%$$(1-\tau_{it}) {u'(c_t) \over 1+\tau_{ct}} = \beta {u'(c_{t+1}) \over 1+\tau_{ct+1}}
%[ (1-\tau_{it+1})(1-\delta) + (1-\tau_{kt+1})f'(k_{t+1})] \EQN tom100 $$
%and feasibility as
%$$ k_{t+1} = f(k_t) +(1-\delta) k_t -g_t -c_t. \EQN tom101 $$

\medskip
\noindent{\bf 1.}  Solve \Ep{ric10} for the terminal steady-state $\overline k$
that is associated with the permanent policy vector $\overline z$
(i.e., find the solution of \Ep{steadstk}).
\medskip
\noindent{\bf 2.} Select a large time index $S > > T$ and guess an initial
consumption rate $c_0$.  (A good guess comes from
the linear approximation to be described in section \use{sec:linapproximation}.)
Compute $u'(c_0)$ and solve
\Ep{ric7} for $k_1$.
\medskip
\noindent{\bf 3.}  For $t=0$, use \Ep{tom100} to solve for $u'(c_{t+1})$.
Then invert $u'$ and compute $c_{t+1}$.   Use \Ep{ric7} to compute
$k_{t+2}$.

\medskip
\noindent{\bf 4.} Iterate on step 3 to compute candidate values
$\hat k_t, t=1, \ldots, S$.
\medskip
\noindent{\bf 5.}  Compute $  \hat k_S -\overline k$.
\medskip
\noindent{\bf 6.}  If $\hat k_S > \overline k$, raise $c_0$ and compute a new
$\hat k_t, t=1, \ldots, S$.
\medskip
\noindent{\bf 7.}  If $\hat k_S < \overline k$, lower $c_0$.
\medskip
\noindent{\bf 8.} In this way, search for a value of $c_0$ that makes
$\hat k_S \approx \overline k$.
\medskip
\noindent{\bf 9.} Compute $\sum_{t=0}^\infty q_t \tau_{ht}$ that satisfies the
government budget constraint at equality.


\subsection{Other equilibrium quantities}

After we solve  \Ep{ric8} for an equilibrium $\{k_t\}$  sequence,
we can recover other equilibrium quantities and prices from the following
equations:
\offparens
$$ \EQNalign{  c_t & = f(k_t) + (1-\delta) k_t - k_{t+1} - g_t
      \EQN formula1;a \cr
q_t & = \beta^t u'(c_t)/(1+\tau_{ct}) \EQN  formula1;b \cr \eta_t
& =  f'(k_t) \EQN formula1;c \cr w_t & = f(k_t) - k_t
f'(k_t) \EQN formula1;d \cr
 \bar R_{t+1} & = {(1+\tau_{ct}) \over(1+\tau_{ct+1})}
     \Biggl[
%  \, \;\
   (1-\tau_{kt+1})(
    f'(k_{t+1})-\delta) + 1 \Biggr]  \cr
    & = {(1+\tau_{ct}) \over(1+\tau_{ct+1})} R_{t,t+1} \EQN formula1;e \cr
    R_{t,t+1}^{-1} & = m_{t,t+1} = \beta {u'(c_{t+1}) \over u'(c_t) } {(1+\tau_{ct}) \over(1+\tau_{ct+1})} \EQN formula1;f \cr
    r_{t,t+1} & \equiv R_{t,t+1} -1 = (1-\tau_{k,t+1}) ( f'(k_{t+1}) - \delta) \EQN formula1;g \cr
%s_t/q_t & = [ (1-\tau_{kt})f'(k_t) +
%(1-\tau_{it})(1-\delta)] \EQN formula1;f \cr
} $$
\autoparens
%where $\bar R_{t+1}$ is the rate at which the market and the tax system allow households to substitute consumption at $t$
%for consumption at $t+1$measured in
%units of consumption goods at $t+1$ per consumption good at $t$.
It is convenient to express \Ep{tom100} as
%{\ninepoint
$$ u'(c_t) = \beta u'(c_{t+1}) \bar R_{t+1}
%{(1+\tau_{ct}) \over(1+\tau_{ct+1})}
%     \left[ {(1-\tau_{it+1})\over (1-\tau_{it})} (1-\delta)
%   +{(1-\tau_{kt+1})
%\over (1-\tau_{it})} f'(k_{t+1})\right] .
\EQN formula1;h$$
%}%endninepoint
or
$$ \bar R_{t+1}^{-1} = \beta u'(c_{t+1}) / u'(c_t) .$$
The left side of this equation is the rate which the market and the tax system allow the household to substitute consumption at $t$
for consumption at $t+1$. % measured in
%units of consumption goods at $t+1$ per consumption good at $t$.
 The right side is the rate at which the household is willing
to substitute consumption at $t$
for consumption at $t+1$.

An equilibrium satisfies equations \Ep{formula1}. In the case of
 constant relative risk aversion (CRRA) utility $u(c) = (1-\gamma)^{-1}c^{1-\gamma}, \gamma \geq 1$,
\Ep{formula1;h} implies
$$  \log({c_{t+1}\over c_t}) = \gamma^{-1}\log \beta + \gamma^{-1}\log \bar R_{t+1}, \EQN keyinsight $$
which shows that the log of consumption growth varies directly
with $\bar R_{t+1}$.
Variations in distorting taxes have effects on consumption and
investment that are intermediated through this equation, as
several  experiments below  highlight.

\subsection{Steady-state $\bar R$}

Using \Ep{steadstk} and formula \Ep{formula1;e}, we can determine that the steady
state value of $\bar R_{t+1}$
is\NFootnote{To compute steady states, we assume that all tax rates and government
expenditures are constant from some date $T$ forward.}
$$ \bar R_{t+1} =  (1+\rho). \EQN steadstR $$

\subsection{Lump-sum taxes available}

If the government can  impose lump-sum taxes,  we
can implement the shooting algorithm for a specified $g, \tau_k,
 \tau_c$, solve for equilibrium prices and quantities,  and
then find an associated value for $q\cdot \tau_h =
 \sum_{t=0}^\infty q_t \tau_{ht}$   that balances the government budget.
This calculation treats
 the present value of lump-sum taxes as a residual that
balances the government budget.
In calculations presented later in this chapter,
we shall assume that lump-sum taxes are available and so shall use this
procedure.

\subsection{No lump-sum taxes available}
If lump-sum taxes are not available, then an additional step
is required to compute an equilibrium.  In particular,  we
have to ensure that taxes and expenditures are such that the
government budget constraint \Ep{ricg} is satisfied at an
equilibrium price system with $\tau_{ht} = 0 $ for all $t\geq 0$.
Braun (1994) and McGrattan (1994b) accomplish this by employing an
iterative algorithm that alters a particular distorting tax until
\Ep{ricg} is satisfied.  The idea is first to compute a candidate
equilibrium for one arbitrary tax policy with possibly nonzero lump sum taxes,  then to check whether
the government budget constraint is satisfied.  Usually we will find that lump sum taxes must be
levied to balance the government budget in this candidate equilibrium.  To find an equilibrium
with zero lump sum taxes, we can proceed as follows.  If the government
budget would have have a deficit   in present value without lump sum taxes (i.e., if the present value
of lump sum taxes is positive  in the candidate equilibrium), then either decrease some
elements of the government expenditure sequence or increase some
elements of the tax sequence and try again.  Because
there exist so many equilibria, the class of tax and expenditure
processes has to be restricted drastically  to narrow the search
for an equilibrium.\NFootnote{See chapter \use{optax} for theories
about how to choose taxes in socially optimal ways.}




\section{A digression on back-solving}\label{sec:back-solving}%
The shooting algorithm takes sequences  for $g_t$  and the various
tax rates as  given and finds paths of the allocation $\{c_t,
k_{t+1}\}_{t=0}^\infty$ and the price system that solve the system
of difference equations formed by \Ep{tom100} and \Ep{formula1}.
Thus, the shooting algorithm
views government policy as exogenous and the price system
and allocation as endogenous.   Sims (1989) proposed another
way to solve  the growth model that exchanges the roles of
some  exogenous and endogenous variables. In particular,
his
  {\it back-solving\/} approach  takes a path  $\{c_t\}_{t=0}^\infty$
  as given, and
then proceeds as follows. \index{back-solving}

\medskip
\noindent{\it Step 1:} Given $k_0$ and sequences for the various
tax rates, solve \Ep{tom100} for a sequence $\{k_{t+1}\}$.
\medskip
\noindent{\it Step 2:} Given the sequences for $\{c_t, k_{t+1}\}$,
solve the feasibility condition \Ep{formula1;a}
for a sequence of government expenditures $\{g_t\}_{t=0}^\infty$.
\medskip
\noindent{\it Step 3:} Solve formulas \Ep{formula1;b}--\Ep{formula1;e}
for an equilibrium price system.
\medskip

The present model can be used to illustrate other applications of
back-solving. For example, we could start with a given
 process for
$\{q_t\}$,  use \Ep{formula1;b} to solve for $\{c_t\}$, and
proceed as in steps 1 and 2 above to determine processes for
$\{k_{t+1}\}$ and $\{g_t\}$, and then finally  compute the
remaining prices from the as yet unused equations in
\Ep{formula1}. \auth{Sims, Christopher A.}

  Sims recommended this method because it adopts a flexible or
``symmetric'' attitude toward exogenous and endogenous variables.
Diaz-Gim\' enez,  Prescott, Fitzgerald, and
 Alvarez (1992), Sargent and Smith (1997), and Sargent and Velde
(1999) have all used the method. We shall not use it in the
remainder of this chapter, but it is a useful method to have in
our toolkit.\NFootnote{Constantinides and Duffie (1996) used
back-solving to reverse engineer a cross-section of endowment
processes that, with incomplete markets, would prompt households
to  consume their endowments at a given stochastic process of
asset prices.} \auth{Duffie, Darrell} \auth{Constantinides, George
M.} \auth{Velde, Fran\c cois} \auth{Smith, Bruce D.}

\section{Effects of taxes on equilibrium allocations and prices}

   We use the model  to analyze the effects of
government expenditure and tax sequences.
The household can affect his payments of a {\it distorting\/}
by altering a decision. The household cannot affect his payments of a {\it nondistorting\/}
tax.
In the present model,  $\tau_k, \tau_c, \tau_n$ are distorting taxes
and the lump-sum tax $\tau_h$ is nondistorting.
\index{nondistorting tax}%
\index{distorting tax}%
We can deduce
 the following
outcomes
from \Ep{formula1} and \Ep{steadstk}.
\medskip
\noindent{\bf 1.}  {\bf Lump-sum taxes and Ricardian equivalence.}
Suppose that the distorting taxes are all zero and that
only lump-sum taxes are used to raise government revenues.  Then
the equilibrium allocation is identical with  one that solves a version
of  a
planning problem in which $g_t$ is taken as an exogenous
stream that is deducted from output.
To verify this claim, notice that lump-sum taxes appear nowhere
in formulas \Ep{formula1}, and that these equations are identical
with the first-order conditions and feasibility conditions for
a planning problem.
The timing of lump-sum taxes is irrelevant because only
the present value of taxes $\sum_{t=0}^\infty q_t \tau_{ht}$ appears in
the budget constraints of the government and the household.
\index{Ricardian proposition}
\medskip
\noindent{\bf 2.}  {\bf
When the labor supply is inelastic,
 constant $\tau_c$ and $\tau_n$ are not distorting.}
 When the labor supply is inelastic, $\tau_n$  is not
a distorting tax.    A  {\it constant\/} level of    $\tau_c$ is
not distorting.


\medskip
\noindent{\bf 3.}  {\bf Variations in $\tau_c$ over time are distorting.}
They affect the path of capital and consumption through equation
\Ep{formula1;g}.
\medskip
\noindent{\bf 4.}  {\bf Capital taxation is distorting.}
 Constant levels of  the capital tax $\tau_k$ are distorting (see \Ep{formula1;g} and
\Ep{steadstk}).

%\item{4.} An increase in the investment tax credit
%blah

%%%%%%%%%%%%%
%$$\grafone{experiment03.eps,height=3in}{{\bf Figure XXX.3.}
%Response to foreseen once-and-for all increase in $\tau_i$ at
%$t=10$. From left to right, top to bottom: $k, c, R, w/q, s/q,
%r/q$.} $$
%%%%%%%%%%%%%%%%
%
%\midfigure{experiment03f}
%\centerline{\epsfxsize=3truein\epsffile{experiment03.eps}}
%\caption{Response to foreseen once-and-for all increase in $\tau_i$ at
%$t=10$. From left to right, top to bottom: $k, c, R, w/q, s/q, r/q$.}
%\infiglist{experiment03f}
%\endfigure

%%%%%%%%%%%%%%%%
%$$\grafone{experiment06.eps,height=3in}{{\bf Figure XXXX.4.}
%Response to foreseen increase in $\tau_k$ at $t=10$.
%From left to right, top to bottom: $k, c, R, w/q, s/q, r/q$.}
%$$
%%%%%%%%%%%%%%%%



\section{Transition experiments with inelastic labor supply}\label{sec:transinelastic}
We continue to study the special case with $U(c,1-n)=u(c)$.
  Figures \Fg{fig_g} through \Fg{fig_tk} apply the  %\Fg{experiment01f}, \Fg{experiment02f}, %\Fg{experiment03f},
%  \Fg{experiment06f},
%  \Fg{experiment04f}, and \Fg{experiment05f} show the results of applying the
shooting algorithm to an economy with
$u(c) = (1-\gamma)^{-1} c^{1-\gamma}, f(k) = k^\alpha$
with parameter values $\alpha=.33, \delta = .2,  \beta=.95 $ and an initial
constant level of $g$ of $.2$.  All of the experiments except one to be described in figure \Fg{fig_g_gamma} set the critical utility curvature parameter
 $\gamma=2$.   We initially set all distorting taxes to
zero and consider perturbations of them that we describe in the experiments
below.  %% check value of \alpha from Riccardo's program.

  Figures \Fg{fig_g} to \Fg{fig_tk}  show responses to foreseen once-and-for-all increases
in $g$, $\tau_c$,  and $\tau_k$, that occur at time
$T=10$, where $t=0$ is the initial time period.  Prices induce
effects that precede the policy changes that cause them. We start
all of our experiments from an initial steady state that is
appropriate for the pre-jump settings of all government policy
variables. In each panel, a dashed line displays a value
associated with the steady state at the initial constant values of
the policy vector. A solid line depicts an equilibrium path under
the new policy. It starts from the value that was associated with
an initial steady state that prevailed before the policy change at
$T=10$ was announced.  {\it Before\/} date $t= T= 10$, the
response of each variable is entirely due to expectations about
future policy changes. {\it After\/} date $t=10$, the response of
each variable represents a purely transient response to a new
stationary level of the ``forcing function'' in the form of the
exogenous policy variables. That is, before  $t=T$, the forcing
function is changing as date $T$ approaches; after date $T$, the
policy vector has attained its new permanent level, so that the
only sources of dynamics are transient.

Discounted {\it future\/}  values of fiscal variables impinge on current outcomes, where the
 discount rate in question is endogenous, while departures of the capital stock from its terminal
 steady-state value set in place a force for it to decay toward its
steady state rate at a particular rate.
 These two forces, discounting of the future and transient decay back toward the terminal steady
 state, are evident in the experiments portrayed in Figures \Fg{fig_g}--\Fg{fig_tk}.
 In section \use{sec:lambda_2},
  we express the decay rate as a function of the key curvature parameter
 $\gamma$ in the one-period utility function $u(c)=(1-\gamma)^{-1}c^{1-\gamma}$, and we note that
 the endogenous rate at which future fiscal variables are discounted is tightly linked to that decay
 rate.



\midfigure{fig_g}
\centerline{\epsfxsize=3truein\epsffile{fig_g.eps}}
\caption{Response to foreseen once-and-for-all increase in $g$ at $t=10$.
From left to right, top to bottom: $k, c, \bar R, \eta, g$. The dashed line is the original
steady state.}
\infiglist{fig_g}
\endfigure


\midfigure{fig_g_gamma}
\centerline{\epsfxsize=3truein\epsffile{fig_g_gamma.eps}}
\caption{Response to foreseen once-and-for-all increase in $g$ at $t=10$.
From left to right, top to bottom: $k, c, \bar R, \eta, g$. The dashed lines show the original
steady state. The solid lines  are for $\gamma=2$, while the dashed-dotted lines are for $\gamma=.2$}
\infiglist{fig_g_gamma}
\endfigure


\midfigure{fig_term}
%\centerline{\epsfxsize=3truein\epsffile{fig_1163.eps}}
\centerline{\epsfxsize=3truein\epsffile{fig_term.eps}}
\caption{Response to foreseen once-and-for-all increase in $g$ at $t=10$.
From left to right, top to bottom: $c, q, r_{t,t+1}$ and yield curves $r_{t, t+s}$  for $t=0$ (solid line),
$t=10$ (dash-dotted line) and $t=60$ (dashed line); term to maturity $s$ is on the $x$ axis for the yield curve, time $t$
for the other panels.}
\infiglist{fig_term}
\endfigure
%%%%%%%%
%
\midfigure{fig_tc}
\centerline{\epsfxsize=3truein\epsffile{fig_tc.eps}}
\caption{Response to foreseen once-and-for-all  increase in $\tau_c$ at
$t=10$. From left to right, top to bottom: $k, c, \bar R, \eta, \tau_c$.}
\infiglist{fig_tc}
\endfigure



\midfigure{fig_tk}
\centerline{\epsfxsize=3truein\epsffile{fig_tauk_gamma.eps}}
%\centerline{\epsfxsize=3truein\epsffile{fig_1165.eps}}
\caption{Response to foreseen increase in $\tau_k$ at $t=10$.  From left to
right, top to bottom: $k, c, \bar R, \eta, \tau_k$. The solid lines depict equilibrium outcomes when $\gamma=2$,
the dashed-dotted lines when $\gamma = .2$.  }
\infiglist{fig_tk}
\endfigure

\medskip
\noindent{\bf Foreseen jump in $g_t$.} Figure \Fg{fig_g}
shows the effects of a foreseen permanent increase in $g$ at
$t=T=10$ that is financed by an increase in lump-sum taxes.
Although the steady-state value of the capital stock is unaffected
(this follows from the fact that $g$ disappears from the steady
state version of the Euler equation \Ep{ric8}),  consumers make the
capital stock
vary  over time.    If the
government consumes more, the household must consume less.  The competitive economy
sends a signal to consumers that they must consume less in the form of an increase in
the stream of lump sum taxes that the government uses to finance the increase in its expenditures.
 Because consumers care about the present value of
lump-sum taxes and are indifferent to their timing, an
adverse wealth effect on consumption precedes the actual rise in government
expenditures.  Consumers choose immediately  to increase their
saving in response to
the adverse wealth effect that they suffer
from the increase in lump-sum taxes that finances the
permanently higher level of government expenditures. Because the present value of lump-sum
taxes jumps immediately, consumption also falls immediately in anticipation of the increase in government
expenditures. This leads to a gradual build-up of capital in the
dates between $0$ and $T$, followed by a gradual fall after $T$.
Variation over time in the capital stock helps smooth
consumption over time, so that the main force at work is  the consumption-smoothing motive featured
in Milton Friedman's permanent income theory. The variation over time
in  $\bar R$ reconciles the consumer to a consumption
path that is not completely smooth.   According to
\Ep{keyinsight}, the gradual increase and then the decrease in
capital are inversely related to variations in the
gross interest rate that reconcile the household to a consumption path that varies over time.

Figure \Fg{fig_g_gamma} compares the responses to a foreseen increase in $g$ at $t=10$ for two economies,
our original economy with $\gamma=2$, shown in the solid line,  and an otherwise identical economy with $\gamma=.2$,
shown in the dashed-dotted line.  The utility curvature parameter $\gamma$ governs the household's willingness to substitute
consumption across time. Lowering $\gamma$  increases the household's willingness to substitute consumption across time.
This shows up in the equilibrium outcomes in figure \Fg{fig_g_gamma}. For $\gamma=.2$, consumption is much less smooth than when $\gamma=2$, and  is closer to being
a mirror image of the government expenditure path, staying high until government expenditures rise at $t=10$. There are  much smaller build ups and draw downs of capital,
and this leads to smaller fluctuations in $\bar R$ and $\eta$.  These two experiments reveal the   dependence of the strength of both the `feedforward' anticipation effect
and the `feedback' transient effect that wears off initial conditions on the magnitude of $\gamma$.  We discuss this more later in section \use{sec:lambda_2} with the aid of equation \Ep{lambda_gamma}.


For $\gamma = 2$ again, figure \Fg{fig_term} describes the response of $q_t$ and the term structure of interest rates to a foreseen increase in
$g_t$ at $t=10$.  The second panel on the top compares $q_t$ for the initial steady state with $q_t$ after the increase in $g$ is foreseen
at $t=0$, while the third panel compares the implied short rate $r_t$ computed via  the section  \use{sec:termstructure_growth} formula
$r_{t,t+1} = - \log(q_{t+1}/q_t) = - \log \bigl[ \beta {\frac{u'(c_{t+1})}{u'(c_t)}}{\frac{(1+\tau_{ct})}{(1+\tau_{c,t+1})}} \bigr]$ and the fourth panel reports the term structure of interest rates $r_{t,t+s}$ computed via formula \Ep{expectations_theory_2} for
$t=0, 10$ and $t=60$ in three separate yield curves for those three dates. In this panel, the term to maturity $s$ is on the $x$ axis, while in the other panels,
calendar time $t$ is on the $x$ axis. In this model, $q_t = \beta^t c_t^{-\gamma}$ and $r_{t,t+1} = - \log \beta \Bigl({\frac{c_{t+1}}{c_t}}\Bigr)^{-\gamma}$, so the term structure of interest rates
 reflects the equilibrium path for $\{c_t\}_{t=0}^\infty$. At $t=60$, the system has converged to the new steady state and the term structure of interest rates is flat.
At $t=10$, the term structure of interest rates is upward sloping because, as the top left panel showing consumption reveals, the {\it rate of growth} of
consumption is expected to increase over time.  At $t=0$, the term structure of interest rate is `U-shaped', declining until maturity 10, then increasing for
longer maturities.  This pattern reflects the pattern for consumption growth, which declines at an increasing rate until $t=10$, then at a decreasing rate
after that.

\medskip\noindent{\bf Foreseen jump in $\tau_c$.} Figure \Fg{fig_tc}
portrays the response to a foreseen increase in the consumption
tax.  As we have remarked, with an inelastic labor supply, the
Euler equation \Ep{ric8} and the other equilibrium conditions show
that {\it constant\/} consumption taxes do not distort decisions,
but that  anticipated {\it  changes\/} in them do. Indeed, \Ep{ric8}
or \Ep{tom100}  indicates that a foreseen increase in $\tau_{ct}$
(i.e., a  decrease in ${(1+\tau_{ct})\over(1+\tau_{ct+1})}$)
operates like an {\it increase\/} in $\tau_{kt}$.  Notice that
while  all variables in Figure \Fg{fig_tc} eventually
return to their initial steady-state values, the anticipated
increase in $\tau_{ct}$ leads to an immediate jump in consumption
at time $0$, followed by a consumption binge that sends the
capital stock downward until the date $t=T=10$, at which
$\tau_{ct}$ rises. The fall in capital causes $\bar R$ to rise over time, which via \Ep{keyinsight}
requires the growth rate of consumption to rise until $t=T$.  The
jump in $\tau_c$ at $t=T=10$ causes  $\bar R$ to be depressed below 1, which via \Ep{keyinsight}
accounts for the drastic fall in consumption at $t=10$.  From date
$t=T$ onward, the effects of the {\it anticipated\/} distortion stemming
from the fluctuation in $\tau_{ct}$ are over, and the economy is
governed by the transient dynamic response associated with a
capital stock that is now below the appropriate  terminal steady-state
 capital stock. From date $T$ onward, capital must rise. That
requires austerity: consumption plummets at date $t=T=10$. As the
interest rate gradually falls,
  consumption grows at a diminishing rate along the path
toward the terminal steady state.
% As predicted by the Euler
%equation \Ep{ric8} or \Ep{tom100}, the diminishing rate of
%consumption growth is associated with a declining interest rate.
%The same comovement of consumption growth and the interest rate
%characterizes the period before $t=T$, with the reverse sign. This
%comovement reflects equation \Ep{tom100}.

%\medskip
%\noindent{\bf Foreseen rise in investment tax credit $\tau_{it}$.}
%Figure \Fg{experiment03f} shows the consequences of a foreseen
%permanent jump in the investment tax credit $\tau_i$ at $t=T=10$.
%All distorting tax rates are initially zero.  As formula
%\Ep{steadstk} predicts,
% the eventual effect of the policy is to drive capital toward
%a higher steady state.   The increase in  capital is accomplished
%by an immediate reduction in consumption, followed by further
%declines (notice that the interest rate is falling) at an
%increasing absolute rate of decline until $t=T=10$.  At $t=9$ (see
%formula \Ep{formula1;e}), there is an abrupt decline in $\bar R_{t+1}$,
%followed by an abrupt increase  at $t=10$.  As equation \Ep{keyinsight}
%confirms, these changes in $\bar R_{t+1}$ that are induced by the jump
%in the investment tax credit at $t=10$ are associated with a large
%drop in $c$ at $t=9$, followed by a sharp increase in its rate of
%growth at $t=10$.    The jump in $R$ at $t=10$ is followed by a
%gradual decrease back to its steady-state level as capital rises
%toward its higher steady-state level.  Eventually consumption
%rises above its old steady-state value and approaches a new higher
%steady state.  This new steady state has too high a capital stock
%relative to what a planner would choose for this economy (``capital
%overaccumulation'' has been ignited by the investment tax credit).
%Because the household discounts the future, the reduction in
%consumption in the early periods is not adequately balanced  by
%the permanent increase in consumption later.
%\index{capital overaccumulation}%%
%Notice how $s/q$ starts falling at an increasing absolute rate
%prior to $t=10$.  This is due to the adverse effect of the cheaper
%new future capital (it is cheaper because it benefits from the
%investment tax credit) on the price of capital that was purchased
%before the investment tax credit is put in place at $t=10$.

\medskip
\noindent{\bf Foreseen jump in $\tau_{kt}$.} For the two $\gamma$ values $2$ and $.2$, Figure
\Fg{fig_tk} shows the response to a foreseen permanent jump
in $\tau_{kt}$ at $t=T=10$. Because the  path of government
expenditures is held fixed, the increase in $\tau_{kt}$ is
accompanied by a reduction in the present value of lump-sum taxes
that leaves the government budget balanced.  The increase in
$\tau_{kt}$ has effects that precede it.  Capital starts declining
immediately due to a rise in current consumption and a
growing flow of consumption. The after-tax gross rate of return on
capital starts rising at   $t=0$ and increases until $t=9$. It
falls precipitously at $t=10$ (see formula \Ep{formula1;e} because
of the foreseen jump in $\tau_k$. Thereafter, $\bar R$ rises, as
required by the transition dynamics that propel $k_t$ toward its
new lower steady state. Consumption is lower in the new steady
state because the new lower steady-state capital stock produces
less output.  Consumption is smoother when $\gamma =2$ than when $\gamma=.2$.  Alterations in $\bar R$ accompany  effects of the tax increase at $t=10$
on  consumption at earlier and later dates.
  %As revealed by formula \Ep{steadstsq}, the
%steady-state value of capital $s/q$  is not altered
%by the permanent jump in $\tau_k$, but volatility is put into
%its time path by the foreseen increase in $\tau_k$.   The rise
%in $s/q$ preceding the jump in $\tau_k$ is entirely due to the
%falling level of $k$.  The large drop in $s/q$ at $t=10$ is caused
%by the contemporaneous jump in the tax on capital (see formula
%\Ep{formula1;f}).


\medskip
So far we have explored consequences of foreseen once-and-for-all
changes in government policy.  Next we describe some experiments
in which there is a foreseen one-time  change in a policy variable
(a ``pulse'').
\medskip

\index{pulse}
\medskip
\noindent{\bf Foreseen one-time pulse in $g_{10}$.}  Figure \Fg{fig_g1}
shows the effects of a foreseen one-time increase in $g_t$ at
date $t=10$ that is financed entirely by alterations in lump sum
taxes.  Consumption drops immediately, then falls further over
time in anticipation of the one-time surge in $g$.  Capital is
accumulated before $t=10$. At $t=T=10$, capital jumps downward because the
government consumes it.  The reduction in capital is accompanied
by a jump in  $\bar R$ above its steady-state value.
The gross return $\bar R$ then falls toward its steady rate level and
consumption rises at a diminishing
rate toward its steady-state value. %The value of existing capital
%$s/q$ is depressed by the accumulation of capital that
%precedes the pulse in $g$ at $g=10$, then jumps dramatically due to the
%capital consumed by the government,  and falls back toward its initial
%steady-state value.
  This experiment
highlights what again looks like a  version of a permanent income theory
response to a foreseen decrease in the resources
available for the public to spend (that is what the increase in
$g$ is about), with effects that are modified by the general
equilibrium adjustments of the gross return $\bar R$.


\midfigure{fig_g1}
\centerline{\epsfxsize=3truein\epsffile{fig_g1.eps}}
\caption{Response to foreseen one-time pulse increase in $g$ at $t=10$. From
left to right, top to bottom: $k, c, \bar R, \eta, g$.}
\infiglist{fig_g1}
\endfigure
%
%\medskip
%\noindent{\bf Foreseen one time pulse in $\tau_{i10}$.} Figure
%\Fg{experiment05f}  shows the response to a foreseen one-time
%investment tax credit at $t=10$.  The most striking thing about
%the response is the dramatic increase in capital at $t=10$, as
%households take advantage of the temporary boost in the after-tax
%rate of return $R$ that is induced by the pulse in $\tau_i$.
%Consumption drops dramatically at $t=10$ as the rate of return on
%capital rises temporarily.  Consumers want to smooth out the drop
%in consumption by reducing consumption before $t=10$, but the
%equilibrium movements in the after-tax return $R$ attenuate their
%incentive to do so.  After $t=10$, consumption jumps in response
%to the jump in interest rates.  Thereafter, rising interest rates
%cause the (negative) rate of consumption growth to rise toward
%zero as the initial steady state is attained once
%more.\NFootnote{Steady-state values are unaffected by a
%one-time pulse.} Notice the negative effects on the value of
%capital that precede the pulse in $\tau_i$.    This experiment
%shows  why most economists frown upon temporary investment tax
%credits: they induce volatility in consumption that households
%dislike.


\section{Linear approximation}\label{sec:linapproximation}%
   The present model is simple enough that it is very easy
to apply the shooting algorithm. But for models with  larger state
spaces, it can be more difficult to apply the shooting algorithm. For those
models, a frequently used procedure is to obtain a linear or
log linear approximation
around a steady state of the difference equation for capital, then to solve it to get an approximation of
the dynamics in the vicinity of that steady state. The present model
is a good laboratory for illustrating how to construct linear approximations.
 In addition to providing an easy way to
approximate a solution, the method illuminates important features
of the solution by partitioning it into two parts:\NFootnote{Hall
(1971) employed linear approximations to exhibit  some of this
structure.} (1) a ``feedback'' part that portrays the transient
response of the system to an initial condition $k_0$ that deviates
from an  asymptotic steady state, and (2) a ``feedforward'' part
that shows the current effects of foreseen
tax rates and expenditures.\NFootnote{Vector
autoregressions embed the consequences of both backward-looking
(transient)  and forward-looking (foresight) responses to
government policies.}

%%%%%%%%%%%%
%$$\grafone{experiment04.eps,height=3in}{{\bf Figure XXX.5.}
%Response to foreseen one-time pulse increase in $g$ at $t=10$. From left to
%right, top to bottom: $k, c, R, w/q, s/q, r/q$.}
%$$
%%%%%%%%%%%%%%


%%%%%%%%%%%%
%$$\grafone{experiment05.eps,height=3in}{{\bf Figure XXX.6.}
%Response to foreseen one-time-pulse increase in $\tau_i$ at
%$t=10$. From left to right, top to bottom: $k, c, R, w/q, s/q,
%r/q$.} $$
%%%%%%%%%%%%%%%%%
%
%\topfigure{experiment05f}
%\centerline{\epsfxsize=3truein\epsffile{experiment05.eps}}
%\caption{Response to foreseen one-time-pulse increase in $\tau_i$
%at $t=10$. From left to right, top to bottom: $k, c, R, w/q, s/q, r/q$.}
%\infiglist{experiment05f}
%\endfigure


To obtain a linear approximation, perform the following
steps:\NFootnote{For an extensive treatment of lag operators
and their uses, see Sargent (1987a).}

\medskip
\noindent{\bf 1.}  Set the government policy $z_t = \overline z$, a constant level.
Solve $H(\overline k, \overline k, \overline k, \overline z,
\overline z) =0$ for a steady-state $\overline k$.
\medskip
\noindent{\bf 2.}  Obtain a  first-order Taylor series approximation around $(\overline k,
\overline z)$:
$$ \eqalign{ &H_{k_t}(k_t - \overline k) + H_{k_{t+1}} (k_{t+1} -\overline k)
   + H_{k_{t+2}}(k_{t+2} - \overline k) \cr & +H_{z_t} (z_t - \overline z)
  + H_{z_{t+1}}(z_{t+1} - \overline z) = 0 \cr} \EQN ric12 $$
\noindent{\bf 3.}  Write the resulting  system as
$$ \phi_0 k_{t+2} + \phi_1 k_{t+1} + \phi_2 k_t = A_0+ A_1 z_t + A_2 z_{t+1}
\EQN ric12 $$
or
$$ \phi(L) k_{t+2} = A_0 + A_1 z_t + A_2 z_{t+1} \EQN ric12a $$
where $L$ is the lag operator (also called the backward shift operator)
defined by $L x_t = x_{t-1}$.
Factor the characteristic polynomial on the left as
$$ \phi(L)= \phi_0 + \phi_1 L + \phi_2 L^2 =
   \phi_0 (1-\lambda_1L) (1-\lambda_2L ). \EQN keyfactor $$
For most of our problems, it will turn out that
one of the $\lambda_i$'s  exceeds unity and that  the other is less
than unity. We shall therefore adopt the convention that
 $|\lambda_1| > 1$ and $|\lambda_2| < 1$.   At this point, we ask the
 reader to accept that the values of $\lambda_i$ split in this
 way.
We   discuss why they do so in section \use{sec:exist_unique}. Notice that  equation
\Ep{keyfactor} implies that $\phi_2 =  \lambda_1 \lambda_2
\phi_0$. To obtain the factorization \Ep{keyfactor}, we proceed as
follows. Note that $(1-\lambda_i L) = -\lambda_i\left(L - {1 \over
\lambda_i}\right)$. Thus,
$$ \phi(L) = \lambda_1 \lambda_2 \phi_0 (L - {1 \over \lambda_1})
(L - {1 \over \lambda_2}) = \phi_2 (L - {1 \over \lambda_1})
(L - {1\over \lambda_2}) \EQN factor10 $$
because $\phi_2 =  \lambda_1 \lambda_2\phi_0$.
Equation \Ep{factor10} identifies ${1 \over \lambda_1}, {1 \over \lambda_2}$
as the zeros of the polynomial
$ \phi(\zeta)$,
%$\lambda_i, i=1,2$ are the reciprocals of the zeros $z_0$ of the
%polynomial $\phi(z)$,
i.e., $\lambda_i = \zeta_0^{-1}$ where  $\phi(\zeta_0) = 0$.\NFootnote{The Matlab roots
 command {\tt roots(phi)}
finds zeros of polynomials, but you must arrange  the polynomial
as $\phi= [\matrix{\phi_2 &  \phi_1 & \phi_0\cr}]$.}
 We
want to operate on both sides of \Ep{ric12a} with the inverse of
$(1-\lambda_1 L)$, but that inverse is unstable backward (i.e.,
the power series $\sum_{j=0}^\infty \lambda_1^j L^j$ has
coefficients that diverge in higher powers of $L$). Fortunately
$(1-\lambda_1 L)$  has a stable inverse in
the {\it forward\/} direction, i.e., in terms of the forward shift
operator $L^{-1}$.\NFootnote{See appendix \use{appa1} of chapter \use{timeseries}.} In particular, notice that
$(1-\lambda_1 L) = -\lambda_1 L( 1-\lambda_1^{-1} L^{-1})$.\NFootnote{Notice that  we can
express $(1-\lambda_1 L)^{-1} = - \lambda_1^{-1} L^{-1} (1-\lambda^{-1} L^{-1})^{-1}$ as
$ -\lambda_1^{-1} L^{-1} \sum_{j=0}^\infty \lambda_1^{-j} L^{-j}$.}
Using this result and $\phi_2 = \lambda_1 \lambda_2 \phi_0$, we can  rewrite $\phi(L)$
as\NFootnote{Justifications for these steps are described at
length in Sargent (1987a) and with rigor in Gabel and Roberts
(1973).}
$$\phi(L) = -{1\over \lambda_2} \phi_2  (1 -  \lambda_1^{-1}L^{-1}) (1
  - \lambda_2 L)L. $$
% Flip the direction of time by inverting the unstable root and
%using
% $$ (1-\lambda_1 L ) =
%-\lambda_1 L( 1 -\lambda_1^{-1} L^{-1}) . $$
Represent
%$$ \alpha(1-\lambda_1 L^{-1}) (1-\lambda_2 L)
%  = \phi_0 L^{-1} + \phi_1  +\phi_2 L \EQN ric13 $$
equation  \Ep{ric12} as
$$ -\lambda_2^{-1}  \phi_2 L (1-\lambda_1^{-1} L^{-1}) (1-\lambda_2 L) k_{t+2}
  =A_0+ A_1 z_t + A_2 z_{t+1}.  \EQN ric14 $$
Operate on both sides of \Ep{ric14} by $-(\phi_2/ \lambda_2)^{-1}
(1-\lambda_1^{-1} L^{-1})^{-1}$ to get
the following representation:\NFootnote{We have thus solved
 the stable root backward and the unstable root forward.}
$$ (1-\lambda_2 L) k_{t+1} = {-\lambda_2 \phi_2^{-1}  \over 1 -
\lambda_1^{-1} L^{-1}}
      \left[ A_0+ A_1 z_t + A_2 z_{t+1} \right].  \EQN ric14 $$
      This concludes the procedure.
      \medskip

Equation \Ep{ric14} is our linear approximation to the equilibrium
$k_t$ sequence.
It can be expressed as
$$ k_{t+1} =  \lambda_2 k_t  -\lambda_2 \phi_2^{-1}
      \sum_{j=0}^\infty (\lambda_1)^{-j}  \left[ A_0+ A_1 z_{t+j} +
A_2 z_{t+j+1} \right].  \EQN ric14app $$
We can summarize the process of obtaining this approximation as
solving stable roots backward and unstable roots forward.
Solving the unstable root forward imposes
the terminal condition \Ep{termk}.  This step corresponds
to the step in the shooting algorithm that adjusts the initial investment
rate to ensure that
the  capital stock eventually approaches the terminal steady-state capital
stock.\NFootnote{The invariant subspace methods described in chapter \use{dplinear}
are also all about solving stable roots backward and unstable roots forward.}



The term $\lambda_2 k_t$ is sometimes called
the ``feedback'' part.  The coefficient $\lambda_2$
measures the transient response rate, in particular, the rate at which
capital returns to a steady state when it starts away
from it.  The remaining terms on the right side of   \Ep{ric14app}
are sometimes called the ``feedforward'' parts. They depend on the
infinite {\it future\/} of the exogenous $z_t$ (which for us
contain  the components of government policy) and measure the
effect on the current capital stock $k_t$ of perfectly foreseen
paths of fiscal policy.
The decay parameter $\lambda_1^{-1}$ measures the rate at which
expectations of future fiscal policies are discounted in terms
of their effects on current investment decisions.
To a linear approximation, every rational expectations model has embedded within it both
feedforward and feedback parts.  The decay parameters
$\lambda_2$ and $\lambda_1^{-1}$ of the feedback and feedforward
parts are determined by the roots of the characteristic
polynomial.
  Equation \Ep{ric14app} thus neatly exhibits the mixture of the
pure foresight and  the pure transient  responses   that are
reflected in our examples in Figures  \Fg{fig_g} through \Fg{fig_tk}. The
feedback part captures the purely transient response and the
feedforward part captures the perfect foresight component.

\subsection{Relationship between the $\lambda_i$'s}

It is a remarkable
fact that if an equilibrium solves a planning problem,
then the roots are linked by  $\lambda_1 = { 1 \over \beta \lambda_2}$, where
$\beta \in (0,1)$ is the planner's discount factor.\NFootnote{See Sargent (1987a,
chap. XI) for a discussion.}  In this case, the feedforward decay rate
$\lambda_1^{-1} = \beta \lambda_2$.
  Therefore, when the equilibrium allocation
solves a planning problem, one of the $\lambda_i$'s is less than
${1\over \sqrt {\beta}}$ and the other exceeds ${1\over \sqrt {\beta}}$
(this follows because $\lambda_1 \lambda_2 = {1\over\beta}$).\NFootnote{Notice that this means
that the solution \Ep{ric14app} remains valid for those divergent $z_t$ processes, provided
that they satisfy $\sum_{t=0}^\infty \beta^t z_{jt}^2 < +\infty$.} From this it follows
that one of the $\lambda_i$'s, say $\lambda_1$, satisfies $\lambda_1 > {1 \over \sqrt{\beta}} > 1$
and  the other $\lambda_i$, say $\lambda_2$ satisfies $\lambda_2 < {1\over \sqrt{\beta}}$.
Thus, for $\beta$ close to $1$,
the condition $\lambda_1 \lambda_2 = {1\over \beta}$ almost implies our earlier assumption
that $\lambda_1 \lambda_2=1$, but not quite.
% O%ur earlier assumption that $\lambda_2$ is less than unity  is stronger than what is  true in general
%for planning problems, but for many problems this stronger assumption will hold.
Having $\lambda_2 < {1 \over \sqrt{\beta}}$ is sufficient to allow  our linear approximation for $k_t$
to satisfy $\sum_{t=0}^\infty \beta^t k_t^2 < +\infty$ for all $z_t$ sequences that satisfy
$\sum_{t=0}^\infty \beta^t z_t \cdot z_t < + \infty$.

 A relationship between the feedforward and feedback decay rates
appears evident in
the experiments depicted in Figure \Fg{fig_g_gamma}. In particular, when the utility curvature parameter $\gamma=2$, the rates at which future events are
discounted in influencing outcomes before $t=10$ and the rates of convergence back to steady state after $t=10$ are both lower than when $\gamma =.2$.

\subsection{Conditions for existence and uniqueness}\label{sec:exist_unique}%
For equilibrium allocations that do not solve planning problems, it ceases to be true
that $\lambda_1 \lambda_2 = {1\over \beta}$. In this case, the location of the zeros of the characteristic
polynomial can be used to assess the existence and uniqueness of an equilibrium up to a linear
approximation.  If both
$\lambda_i$'s exceed ${1\over \sqrt{\beta}}$, there exists no equilibrium allocation for which
$\sum_{t=0}^\infty \beta^t k_t^2 < \infty.$  If both $\lambda_i$'s are less than ${1 \over \sqrt{\beta}}$,
there exists a continuum of equilibria that satisfy that inequality.  If the $\lambda_i$'s split, with
one exceeding and the other being less than ${1 \over \sqrt{\beta}}$, there exists a unique
equilibrium.

\subsection{Once-and-for-all jumps}

 Next we specialize \Ep{ric14} to capture
some examples of foreseen policy changes that we have studied
above. Consider the special case treated by Hall (1971) in which
the $j$th component of $z_t$ follows the path
$$ z_{jt} = \cases{ 0 & if $t \leq T-1$ \cr
                    \overline z_j & if $ t \geq T$ \cr} \EQN path1 $$
We define
$$\eqalign{ v_t & \equiv \sum_{i=0}^\infty \lambda_1^{-i} z_{t+i,j} \cr
    & = \cases { {({1 \over \lambda_1})^{T-t} \overline z_j
   \over 1-({1 \over \lambda_1})}
     & if $t \leq T $ \cr
                 {1 \over 1 -({1 \over \lambda_1})}
      \overline z_j & if $ t \geq T$ \cr} \cr}
  \EQN ric16
$$

$$\eqalign{ h_t & \equiv \sum_{i=0}^\infty ({ 1\over \lambda_1})^i z_{t+i+1,j} \cr
    & = \cases { {({1\over \lambda_1})^{T-(t+1)} \overline z_j \over 1-({1\over \lambda_1})}
     & if $t \leq T-1 $ \cr
                 {1 \over 1 -({1 \over \lambda_1})} \overline z_j & if $ t \geq T-1$. \cr} \cr}
  \EQN ric17 $$
Using these formulas, let the vector  $z_t$ follow the path
$$ z_t = \cases{ 0 & if $t \leq T-1$ \cr
                 \overline z & if $ t \geq T$ \cr  }$$
where $\overline z$ is a vector of constants.
Then applying
\Ep{ric16} and \Ep{ric17} to \Ep{ric14} gives the formulas
$$ k_{t+1} = \cases{ \lambda_2 k_t
-{(\phi_0  \lambda_1)^{-1}A_0\over 1 -({1 \over \lambda_1})}
     - {(\phi_0 \lambda_1)^{-1} ({1\over \lambda_1})^{T-t} \over 1
 - ({1\over \lambda_1})}
     (A_1 + A_2 \lambda_1) \overline z &  if  $t \leq T-1 $ \cr
      \lambda_2 k_t
%-{(\phi_0  \lambda_1)^{-1}A_0\over 1 -({1 \over \lambda_1})}
 -{(\phi_0 \lambda_1) ^{-1} \over 1-({1 \over \lambda_1})}
 [A_0+ (A_1 + A_2)
     \overline z] & if $t \geq T$. \cr} $$


\subsection{Simplification of formulas}

These formulas can be simultaneously generalized and
simplified by using the following trick.
Let $z_t$ be governed by the state-space system
$$\EQNalign{ \bar x_{t+1} & = A_x \bar x_t \EQN newsys1;a \cr
           z_t & = G_z  \bar x_t, \EQN newsys1;b \cr} $$
with initial condition $\bar x_0$ given.  In chapter
\use{timeseries}, we saw that many finite-dimensional linear time
series models could
be represented in this form, so that we are accommodating a large
class of tax and expenditure processes.  Then notice that
$$ \EQNalign{ \left({A_1 \over 1 -\lambda_1^{-1} L^{-1}}\right) z_t
  & = A_1 G_z (I - \lambda_1^{-1}A_x)^{-1}  \bar x_t \EQN newsys2;a \cr
\left({A_2 \over 1 -\lambda_1^{-1} L^{-1}}\right) z_{t+1}
  & = A_2 G_z (I - \lambda_1^{-1}A_x)^{-1}A_x  \bar x_t \EQN newsys2;b \cr}$$
Substituting these expressions into
\Ep{ric14app} gives
\offparens
$$ \EQNalign{k_{t+1} = &  \lambda_2 k_t  -\lambda_2 \phi_2^{-1}
       \bigl[ (1-\lambda_1^{-1})^{-1} A_0+
       A_1 G_z (I - \lambda_1^{-1}A_x)^{-1}  \bar x_t  \cr
       &+  A_2 G_z (I - \lambda_1^{-1}A_x)^{-1}A_x  \bar x_t
         \bigr]. \EQN newsys1;c \cr}  $$
Taken together, system \Ep{newsys1} gives a complete description
of the joint evolution of the exogenous state variables $\bar x_t$
driving
 $z_t$ (our government
policy variables) and the capital stock.   System \Ep{newsys1} concisely
displays the \idx{cross-equation restrictions} that are the hallmark of
rational expectations models: nonlinear functions of the  parameter occurring in  $G_z, A_x$
in the law of motion for the exogenous processes appear in the equilibrium
representation \Ep{newsys1;c} for the endogenous state variables.



We can easily use the state space system \Ep{newsys1} to capture
the special case \Ep{path1}.  In particular, to portray $\bar
x_{j,t+1} = \bar x_{j+1,t}$, set the $T \times T$ matrix $A$ to be
$$ A = \left[\matrix{0_{T-1 \times 1} & I_{T-1 \times T-1} \cr
                     0_{1 \times T-1} & 1  \cr} \right]\EQN exmpl1  $$
and take the initial condition $\bar x_0 = \left[\matrix{0 & 0 &
\cdots & 0 & 1\cr}\right]' $.  To represent an element of $z_t$
that jumps once and for all from $0$ to $\overline z_j$  at $T=0$,
set the $j$th component of $G_z$ equal to $G_{zj} =
\left[\matrix{\overline z_j & 0 \cdots & 0\cr}\right] $.


\subsection{A one-time pulse}

We can modify the transition matrix \Ep{exmpl1} to model a one-time
pulse in a component of $z_t$ that occurs at and only at $t=T$.
To do this, we simply set
$$ A = \left[\matrix{0_{T-1 \times 1} & I_{T-1 \times T-1} \cr
                     0_{1 \times T-1} & 0  \cr} \right].\EQN exmpl1a  $$
%which alters the $(T, T)$  element of  $A$ from $1$ to $0$.

\subsection{Convergence rates and anticipation rates}\label{sec:lambda_2}%
Equation \Ep{ric14app} shows that up to a linear approximation, the  feedback coefficient $\lambda_2$ equals
 the geometric rate at which the model returns to
a steady state after a transient displacement away from a steady state.  For our benchmark values of our other parameters
$\delta=.2, \beta=.95,\alpha=.33$ and all distorting taxes set to zero,  we can compute
that $\lambda_2$ is the following function of the utility curvature parameter $\gamma$ that appears in
$u(c) = (1-\gamma)^{-1}c^{1-\gamma}$:\NFootnote{We used the Matlab
symbolic toolkit to compute this expression.}
$$ \lambda_2 = {\gamma \over a_1\gamma^{-1} + a_2 +a_3(\gamma^{-1} + a_4 \gamma^{-2}+a_5)^{1\over 2}} \EQN lambda_gamma$$
where
$a_1=.975, a_2=.0329, a_3=.0642, a_4=.00063, a_5=.0011$.  Figure \Fg{lambdaogamma} plots this function. When
$\gamma=0$,  the period utility function is linear and the household's willingness to substitute consumption
over time is unlimited.  In this case, $\lambda_2=0$, which means that in response to a perturbation of the
capital stock away from a steady state, the return to a steady state
is immediate.  Furthermore, as mentioned above, because there are no distorting taxes in the initial steady state,
we know that  $\lambda_1 = {1 \over \beta \lambda_2}$, so that according to \Ep{ric14app}, the feedforward response
to future $z$'s is a discounted sum that decays at rate $\beta \lambda_2$.  Thus, when $\gamma=0$, anticipations
of {\it future\/} $z$'s have no effect on current $k$. This is the other side of the coin of the immediate adjustment
associated with the feedback part.

As the curvature parameter $\gamma$ increases, $\lambda_2$
increases, more  rapidly at first, more slowly later. As $\gamma$
increases, the household values a smooth consumption path more and
more highly. Higher values of $\gamma$ impart to  the equilibrium
capital sequence  both a more sluggish feedback response and a
feedforward response that puts relatively more weight on
prospective  values of the $z$'s in the more distant future.


\midfigure{lambdaogamma}
\centerline{\epsfxsize=3truein\epsffile{lambdaogamma.eps}}
\caption{Feedback coefficient $\lambda_2$ as a function
$\gamma$, evaluated at $\alpha=.33, \beta=.95, \delta=.2, g=.2$.}
\infiglist{lambdaogamma}
\endfigure

\subsection{A remark about accuracy: Euler equation errors}\label{sec:accuracy}%
It is important to estimate the accuracy of approximations.  One
simple diagnostic tool is to  take a candidate solution for a
sequence $c_t, k_{t+1}$, substitute them into the two Euler
equations \Ep{ric20} and \Ep{ric21}, and call the deviations
between the left sides and the right sides the Euler equation
errors.\NFootnote{For more about this method, see Den Haan and
Marcet (1994) and Judd (1998).}  An accurate method makes these
errors small.\NFootnote{Calculating Euler equation errors, but for
a different purpose, goes back a long time. In chapter 2 of {\it
The General Theory of Interest, Prices, and Money\/}, John Maynard
Keynes noted that plugging in {\it data\/} (not a candidate {\it
simulation\/}) into \Ep{ric21}  gives big residuals. Keynes
therefore assumed that \Ep{ric21} does not hold (``workers are off their labor supply curve'').
}
%
%Figure \Fg{experiment09f} plots the consumption Euler equation
%errors that we obtained  when we used a linear approximation to
%study the consequences of a foreseen jump in $g$ (the experiment
%recorded in Figure \Fg{experiment01f}).  Although qualitatively
%the responses that the linear approximation recovers are
%indistinguishable from figure \Fg{experiment01f} (we don't display
%them), the Euler equation errors for the linear approximation are
%substantially larger than for the shooting method (we don't show
%the Euler equation errors for the shooting method because they are
%so minuscule that they couldn't be detected on the graph).
%
%%%%%%%%%%%%%
%%$$\grafone{experiment09_Euler.eps,height=2in}{{\bf Figure XXX.7.}
%%Error in consumption Euler equation for linear approximation
%%for response to foreseen increase in $g$ at
%%$t=10$.} $$
%%%%%%%%%%%%%%%%%%
%
%\midfigure{experiment09f}
%\centerline{\epsfxsize=3truein\epsffile{experiment09_Euler.eps}}
%\caption{Error in consumption Euler equation for linear approximation for response
%to foreseen increase in $g$ at $t=10$.}
%\infiglist{experiment09f}
%\endfigure



\section{Growth}\label{sec:growthKPR}%
   It is straightforward to alter the model to allow for exogenous growth.
We modify the production function to be
$$ Y_t = F(K_t, A_t n_t)  \EQN ric50 $$
where $Y_t$ is aggregate output, $N_t$ is total employment, $A_t$
is labor-augmenting technical change, and $F(K, AN)$ is the same
linearly homogeneous production function as before. We assume that
$A_t$ follows the process
$$ A_{t+1} = \mu_{t+1} A_t \EQN ric51  $$
and will usually but not always assume that $\mu_{t+1} = \overline
\mu >1$.  We exploit the linear homogeneity of \Ep{ric50}
to express the production function as
$$ y_t = f(k_t) \EQN ric52 $$
where $f(k) = F(k,1)$ and now $k_t = {K_t \over n_t A_t}, y_t =
{Y_t\over n_t A_t}$.  We say that $k_t$ and $y_t$ are measured per
unit of ``effective  labor'' $A_t n_t$. We also let $c_t = {C_t
\over A_t n_t} $ and $g_t = {G_t \over A_t n_t}$ where $C_t$ and
$G_t$ are total consumption and total government expenditures,
respectively. We consider the special case in which labor is
inelastically supplied.  Then feasibility can be summarized by the
following modified version of \Ep{ric7}:
$$ k_{t+1} = \mu_{t+1}^{-1}[f(k_t) +(1-\delta) k_t -g_t -c_t].
\EQN ric53 $$
Noting that per capita consumption is $c_t A_t$,
we obtain the following counterpart to
equation \Ep{tom100}:
$$\eqalign{ u'(c_t A_t)  = &
\beta u'(c_{t+1} A_{t+1}) {(1+\tau_{ct}) \over(1+\tau_{ct+1})}
 \cr &     \left[ (1-\tau_{kt+1}) (f'(k_{t+1})-\delta) + 1 \right] .\cr}\EQN ric54$$

We assume the power utility function $u'(c) = c^{-\gamma}$, which makes
the Euler equation become
$$ (c_t A_t)^{-\gamma} = \beta (c_{t+1} A_{t+1})^{-\gamma} \bar R_{t+1}, $$
where $\bar R_{t+1}$ continues to be defined by \Ep{formula1;e}, except
that now $k_t$ is capital per effective unit of labor.  The preceding
 equation can be represented as
$$ \left({c_{t+1} \over c_t}\right)^\gamma =
  \beta \mu_{t+1}^{-\gamma} \bar R_{t+1}. \EQN eulergrowth $$
In a steady state, $c_{t+1} = c_t$.
Then the steady-state version of the Euler equation \Ep{ric54} is
$$ 1 =  \mu^{-\gamma}
\beta[ (1-\tau_k) (f'(k)-\delta) +1 ],
\EQN ric55 $$
which can be solved for the steady-state capital stock.
It is easy to compute that
the steady-state   level of capital per unit of effective labor
satisfies
$$ f'(k) = \delta + \Biggl( {(1+\rho)\mu^\gamma -1 \over 1-\tau_k} \Biggr) % {1\over (1-\tau_k)} [(1+\rho) \mu^\gamma -(1-\delta)],
  \EQN steadstkgr $$
and that
$$ \bar R = (1+\rho)\mu^\gamma. \EQN steadstRgr  $$
%and that the steady-state value of capital $s/q$ is
%$$ s/q = (1-\tau_i)(1+\rho) \mu^\gamma + \tau_i (1-\delta) .
%  \EQN steadstsqgr $$
Equation \Ep{steadstRgr} immediately shows that {\it ceteris
paribus\/}, a jump in the rate of technical change raises $\bar R$.

Next we apply the shooting algorithm to compute equilibria. We augment
the vector of forcing variables $z_t$ by including $\mu_t$, so that
it becomes $z_t =\left[\matrix{g_t &   \tau_{kt} &
\tau_{ct} & \mu_t \cr} \right]'$, where $g_t$ is understood to be
measured in effective units of labor, then proceed as before.



\midfigure{fig_pr_1}
\centerline{\epsfxsize=3truein\epsffile{fig_pr_1.eps}}
\caption{Response to foreseen once-and-for-all increase in rate of growth of
productivity $\mu$ at $t=10$. From left to right, top to bottom:
$k, c, \bar R, \eta, \mu$, where now $k, c$ are measured in units of
effective unit of labor.}
\infiglist{fig_pr_1}
\endfigure


\midfigure{fig_pr_2}
\centerline{\epsfxsize=3truein\epsffile{fig_pr_2.eps}}
\caption{Response to increase in rate of growth of productivity $\mu$ at $t=0$.
From left to right, top to bottom: $k, c, \bar R, \eta, \mu$, where now $k, c$ are
measured in units of effective unit of labor.}
\infiglist{fig_pr_2}
\endfigure


\medskip
\noindent{\bf Foreseen jump in  productivity growth at $t=10$.}
Figure \Fg{fig_pr_1} shows effects of a permanent  increase
from $1.02$ to $1.025$ in the productivity gross growth rate  $\mu_t$ at
$t=10$. This figure and also Figure \Fg{fig_pr_2}
 now measure $c$ and $k$ in effective units of labor.  The
steady-state Euler equation \Ep{ric55} guides  main features of
the outcomes, and implies that a permanent  increase in $\mu$ will
lead to a decrease in the steady-state value of capital per unit
of effective labor.  Because capital is more efficient, even with
less of it, consumption per capita can be raised, and that is what
individuals care about. Consumption jumps immediately because
people are wealthier.  The increased productivity of capital
spurred by the increase in $\mu$ leads to an increase in the
 gross return  $\bar R$.  Perfect foresight makes
the  effects of the increase in the growth of capital precede
it.%{\bf check this}
%The value of capital $s/q$ rises.

\medskip\noindent{\bf Immediate (unforeseen) jump in productivity
growth at $t=1$.} Figure \Fg{fig_pr_2} shows effects of an
immediate jump in $\mu$ at $t=0$.  It is instructive to compare
these with the effects of the foreseen increase in Figure
\Fg{fig_pr_1}. In Figure \Fg{fig_pr_2}, the paths of all
variables are entirely dominated by the feedback part of the
solution, while before $t=10$ those in Figure \Fg{fig_pr_1}
have contributions from the feedforward part.  The absence of
feedforward effects makes  the paths of all variables in Figure
\Fg{fig_pr_2} smooth. Consumption per effective unit of labor
jumps immediately then declines smoothly toward its steady state
as the economy moves to a lower level of capital per unit of
effective labor. The after-tax gross return  $\bar R$  once
again comoves with the consumption growth rate to verify the Euler
equation \Ep{ric55}.

%%%%%%%%%%%%%%%%%%
%$$\grafone{experiment07.eps,height=3in}{{\bf Figure XXX.8.}
%Response to foreseen once-and-for-all  increase
% in rate of growth
%of productivity $\mu$ at $t=10$. From left to right, top to
%bottom: $k, c, R, w/q, s/q, r/q$, where now $k, c$ are measured in
%units of effective unit of labor.}
%$$
%%%%%%%%%%%%%%%%%%%%%






\section{Elastic labor supply}\label{sec:elasticlabor}

We now again shut down productivity  growth by setting the gross productivity growth rate  $\mu =1$, but we
     allow a possibly
nonzero labor supply elasticity by
specifying
$U(c,1-n)$ to include a preference for leisure. Again, we let $U_i$ be the partial derivative
of $U$ with respect to its $i$th argument. We assume an interior solution
for $n \in (0,1)$.
Now we have to carry along equilibrium conditions
 both for the intertemporal  evolution of capital and for the
labor-leisure choice. These are the two difference equations:
%{\ninepoint
 $$\eqalign{ &
 { 1
\over (1+\tau_{ct})} U_1\bigl(F(k_t, n_t) + (1-\delta) k_t - g_t - k_{t+1}, 1-n_t\bigr)
\cr
& = \beta (1+\tau_{ct+1})^{-1}
U_1\bigl(F(k_{t+1}, n_{t+1}) + (1-\delta) k_{t+1} - g_{t+1} - k_{t+2},
   1-n_{t+1}\bigr)
 \cr
   & \times \left[   (1-\tau_{kt+1}) (F_k(k_{t+1},n_{t+1}) - \delta) +1  \right] \cr}\EQN ric20 $$
%}%endninepoint
$$\eqalign{ & {U_2\bigl(F(k_t,n_t) + (1-\delta)k_t -g_t - k_{t+1},1- n_t\bigr)\over
U_1\bigl(F(k_t,n_t) + (1-\delta)k_t -g_t - k_{t+1}, 1-n_t\bigr) } \cr
  & = {(1-\tau_{nt})\over(1+\tau_{ct})} F_n(k_t,n_t). \cr} \EQN ric21 $$

  The  linear approximation method
applies equally well to this more general setting  with just one
additional step.
  We obtain a linear approximation to this dynamical system by
  proceeding as follows.
First, find steady-state values $(\overline k, \overline n)$ by solving
the two steady-state versions of equations \Ep{ric20}, \Ep{ric21}. (Now
$(\overline k, \overline n)$  are steady-state values of capital per person
and labor supplied per person, respectively.)
Then take the following
linear approximations to \Ep{ric20}, \Ep{ric21}, respectively, around
the steady state:
$$ \eqalign{ &H_{k_t}(k_t - \overline k) + H_{k_{t+1}} (k_{t+1} -\overline k)
    + H_{n_{t+1}} (n_{t+1} - \overline n)
+ H_{k_{t+2}}(k_{t+2} - \overline k) \cr &+ H_{n_t} (n_t - \overline n)
 +H_{z_t} (z_t - \overline z)
  + H_{z_{t+1}}(z_{t+1} - \overline z) = 0 \cr} \EQN ric22 $$

$$ G_k(k_t -\overline k) + G_{n_t}(n_t - \overline n)
+G_{k_{t+1}}(k_{t+1} -\overline k)
  + G_z(z_t - \overline z) = 0 \EQN ric23 $$
Solve \Ep{ric23} for $(n_t-\overline n)$ as functions
of the remaining terms, substitute into \Ep{ric22}
to get a version of equation \Ep{ric12}, and proceed
as before
with a difference equation of the form \Ep{ric10}.


\subsection{Steady-state calculations}

To compute a steady state for this version of the model, assume
that government expenditures and all  flat-rate taxes are
constant over time.
% and calculate steady state
%values of capital and labor.
Steady-state versions of \Ep{ric20}, \Ep{ric21} are
$$\EQNalign{ 1 & = \beta [(1 +(1-\tau_k)
  (F_k(\overline k,\overline n)-\delta) ] \EQN ric200 \cr
  {U_2 (\overline c, 1 - \overline n)\over U_1 (\overline c, 1 - \overline n)} & =  {(1-\tau_n)\over (1+\tau_c)}
   F_n(\overline k,\overline n)
   \EQN ric201 \cr} $$
and the steady state version of the feasibility condition
\Ep{ric02} is
$$ \overline c + \overline g + \delta \overline k = F(\overline k, \overline n). \EQN tom_e_2001_old $$
The linear homogeneity of $F(k,n)$ means that
equation \Ep{ric200} by itself determines the steady-state
capital-labor ratio ${\overline k \over \overline n}$.  In particular,
where $\tilde k = {k \over n}$,  notice that
$F(k,n) = n f(\tilde k) $
and $ F_k(k,n) = f'(\tilde k)$.
It is helpful to use these facts to write \Ep{tom_e_2001_old}
as
$$ {\frac{\overline c + \overline g}{\overline n}} = f(\tilde k) - \delta \tilde k . \EQN tom_e_2001$$
  Next,  letting
$\beta = {1 \over 1+\rho}$, \Ep{ric200}
can be expressed as
$$ \delta +{ \rho  \over (1-\tau_k)} =
  f'(\tilde k), \EQN ric202 $$
an equation that determines a steady-state capital-labor ratio
$\tilde k$. An increase in ${1\over(1-\tau_k)}$ decreases
the capital-labor ratio, but  the steady-state
capital-labor ratio is independent of the steady state values of  $\tau_c, \tau_n$.  However,
given the steady state value of the capital-labor ratio $\tilde k$,  flat rate taxes on consumption and labor income  influence
the steady-state levels of consumption and labor via  the steady state equations \Ep{ric201} and \Ep{tom_e_2001_old}.
Formula \Ep{ric201} reveals how both $\tau_c$ and $\tau_n$  distort the same labor-leisure margin.



If we define $\check \tau_c = {\tau_n + \tau_c \over 1 +
\tau_c}$ and $\check \tau_k = {\tau_k  \over
1-\tau_k}$, then it follows that ${(1-\tau_n) \over (1+\tau_c)} =
1- \check \tau_c$ and ${1 \over(1-\tau_k)} = 1
+\check \tau_k$. The wedge $1 -\check \tau_c$ distorts the
steady-state labor-leisure decision via \Ep{ric201} and the wedge
$1 +\check \tau_k$ distorts the steady-state capital-labor
ratio via \Ep{ric202}. \auth{Den Haan, Wouter} \auth{Marcet,
Albert} \auth{Judd, Kenneth L.} \auth{Keynes, John Maynard}
\auth{Keynes, John Maynard}

\subsection{Some experiments}

To make things concrete,  we use  the following preference specification
popularized by Hansen (1985) and Rogerson (1988):
$$ U(c, 1-n) = \ln c + B (1 - n)  \EQN hansen_rogerson_1 $$
where we set $B $ substantially greater than $1$ to assure an interior solution
$n \in (0,1)$ for labor supply.  In particular, we set $B =3$ in the experiments below.
In terms of steady states,
equation  \Ep{ric201}
 becomes
$$ B \overline c = {(1-\tau_n)\over (1+\tau_c)} \bigl[
   f(\tilde k) - \tilde k f'(\tilde k) \bigr] . \EQN eqlbrm_new_a $$
It is useful to collect equations \Ep{ric202},  \Ep{eqlbrm_new_a}, and \Ep{tom_e_2001},
 into the following system that recursively determines steady-state outcomes for $\tilde k, \overline c,$ and $\overline n$ in the
experiments to follow:
$$
\EQNalign{ \delta + {\rho  \over (1-\tau_k)} &=
  f'(\tilde k) \EQN ric2020 \cr
  B \overline c & = {(1-\tau_n)\over (1+\tau_c)} \bigl[
   f(\tilde k) - \tilde k f'(\tilde k) \bigr]  \EQN eqlbrmnewa0 \cr
%{\frac{\overline c + \overline g}{\overline n}} & = f(\tilde k) - \delta \tilde k  .
\overline c & = \overline n \left( f(\tilde k) - \delta \tilde k \right) - \overline g .
\EQN tom_e_20010
 }$$
%In the experiments below, we assume that the capital tax $\tau_{kt}$  always satisfies
%$$ {\frac{\rho}{\rho+\delta}} > \tau_{kt}. \EQN taxk_condn $$
%This condition assures that the steady state $\tilde k$ implied by \Ep{ric2020} is feasible.\NFootnote{Notice
%that the maximum sustainable feasible steady-state capital labor ratio $\check k$ solves
%$ f'(\check k) = \delta.$}


%\vfil\eject
\noindent{\bf Unforeseen jump in $g$.}   Figure \Fg{fig_gnu} displays the consequences
of an unforeseen and permanent  jump in $g$ at $t=0$, financed entirely by adjustments in lump sum taxes.
Equation \Ep{ric2020} determines $\tilde k$, which is unaltered.  Equation \Ep{eqlbrmnewa0} then implies that $\overline c$ is unaltered.
Equation \Ep{tom_e_20010} determines $\overline k$ and $\overline n$, their ratio having been determined by \Ep{ric2020}.
  The consequences of an unforeseen  increase in
$g$ differ markedly from those analyzed above for the case in which  the labor elasticity is zero. Then, the consequence was immediately and permanently
to {\it lower\/}
consumption per capita by the amount of the {\it increase\/} in government purchases per capita.  Now the effect
is to leave unaltered  both  steady state  consumption per capita and the steady state capital/labor ratio.  This is accomplished
 by raising the steady state levels of both capital and the labor supply. Thus, now the consequence of the increase in $g$ is to `grow the economy' enough eventually to leave  consumption unaffected despite the increase in $g$.

 These asymptotic outcomes immediately  drop out of our steady state equations.  The increase in
$g$ is accompanied by  increases in $k$ and $n$ that leave the steady state capital/labor ratio unaltered, as required by equation \Ep{ric202}.  Equation
\Ep{eqlbrm_new_a} then dictates that  steady-state consumption per capita also remain unaltered.







\midfigure{fig_gnu}
\centerline{\epsfxsize=3truein\epsffile{fig_gnu.eps}}
\caption{Elastic labor supply: response to unforeseen increase  in $g$ at $t=0$.
From left to right, top to bottom: $k, c, n, \bar R, w, g$. The dashed line is the original
steady state.}
\infiglist{fig_gnu}
\endfigure



%
%\midfigure{fig_gnu}
%\centerline{\epsfxsize=3truein\epsffile{fig_gnu.eps}}
%\caption{Elastic labor supply: response to unforeseen increase  in $g$ at $t=0$.
%From left to right, top to bottom: $k, c, n, w, \bar R, g$. The dashed line is the original
%steady state.}
%\infiglist{fig_gnu}
%\endfigure



%
%$$ {(\rho + \delta)  \over (1-\tau_k)} =
%  f'(\tilde k), \EQN ric202 $$
%$$ \overline c + \overline g + \delta \overline k = F(\overline k, \overline n). \EQN tom_e_2001 $$
%$$ B \overline c = {(1-\tau_n)\over (1+\tau_c)} \bigl[
%   f(\tilde k) - \tilde k f'(\tilde k) \bigr] . \EQN eqlbrm_new_a $$

\medskip
\noindent{\bf Unforeseen jump in $\tau_n$.}
Figure \Fg{fig_tnu} shows outcomes from an unforeseen increase in the marginal tax rate on labor $\tau_n$, once again
accompanied by an adjustment in the present value of lump sum taxes required to balance the government's budget.
Here the effect is to shrink the economy.  As required by equation \Ep{ric202}, the steady state
capital labor ratio is unaltered.  But equation \Ep{eqlbrm_new_a} then requires that steady state consumption
per capita must fall in response to the increase in $\tau_n$.  Both labor supplied $n$ and capital fall in the new steady state.
\medskip

\noindent{\bf Countervailing forces contributing to Prescott (2002)}
The preceding two experiments isolate forces that Prescott (2002) combines to reach his conclusion that Europe's
economic activity has been depressed relative to the U.S. because its tax rates have been higher.  Prescott's numerical
calculations activate the forces that shrink the economy in our second experiment that increases $\tau_n$  while shutting  down the force to grow
the economy implied by a larger $g$.  In particular, Prescott assumes that  cross-country outcomes are generated by  second experiment, with  lump sum transfers being used to rebate the revenues raised from the larger labor tax rate $\tau_n$ that  he estimates to prevail in  Europe.
If instead one assumes that higher taxes in Europe are used to pay for larger per capita government purchases, then  forces to grow the economy identified
in our first experiment are unleashed, making  the adverse consequences for the level of economic activity of larger $g, \tau_n$ pairs in Europe  become much smaller than Prescott calculated.
\auth{Prescott, Edward C.}%

\midfigure{fig_tnu}
\centerline{\epsfxsize=3truein\epsffile{fig_tnu.eps}}
\caption{Elastic labor supply: response to unforeseen increase  in $\tau_n$ at $t=0$.
From left to right, top to bottom: $k, c, n, \bar R, w, \tau_n$. The dashed line is the original
steady state.}
\infiglist{fig_tnu}
\endfigure


\vfil\eject
\medskip
\noindent{\bf Foreseen jump in $\tau_n$.}
Figure \Fg{fig_tn} describes consequences of a foreseen increase in $\tau_n$ that occurs at time $t=10$.   While the ultimate
effects are identical with those described in the preceding experiment,  transient outcomes differ.
The immediate effect of the foreseen increase in $\tau_n$ is to spark a boom in employment and capital accumulation, while leaving
consumption unaltered before time $t=10$.  People work more in response to the anticipation that rewards to working will decrease permanently
at $t=10$.  Thus, the foreseen increase in $\tau_n$ sparks a temporary employment and  investment boom.

\midfigure{fig_tn}
\centerline{\epsfxsize=3truein\epsffile{fig_tn.eps}}
\caption{Elastic labor supply: response to foreseen increase  in $\tau_n$ at $t=10$.
From left to right, top to bottom: $k, c, n, \bar R, w, \tau_n$. The dashed line is the original
steady state.}
\infiglist{fig_tn}
\endfigure


To interpret what is going on here, we begin by noting that with  preference specification \Ep{hansen_rogerson_1},
the following system of difference equations determines the dynamics of equilibrium allocations:
$$\EQNalign{ c_{t+1} & = \beta \bar R_{t+1} c_t \EQN bigsys1;a \cr
          \bar R_{t+1}& = {\frac{1 +\tau_{ct}}{1+\tau_{ct+1}}} \bigl[ 1+ (1-\tau_{kt+1}) ( f' \bigl( k_{t+1}/n_{t+1}\bigr) -\delta) \bigr] \EQN bigsys1;b \cr
          B c_t & = {\frac{(1-\tau_{nt})}{(1+ \tau_{ct})}} F_n (k_t, n_t) \EQN bigsys1;c \cr
          k_{t+1} & = F(k_t, n_t) + (1-\delta) k_t - g_t - c_t \EQN bigsys1;d } $$
These equations teach us that the foreseen increase in $\tau_n$ sparks a substantial rearrangement in how the household distributes its work over time.
The effect of the permanent increase in $\tau_n$ at $t=10$ is to reduce the after-tax wage from $t=10$ onward,  though initially
the real wage falls by less than the decrease in $(1-\tau_n)$ because of the increase in the capital labor ratio induced by the drastic fall in
$n$ at $t=10$.  Eventually,  as the pre-tax real wage $w$ returns
to its initial value, the real wage falls by the entire amount of the decrease in $(1-\tau_n)$.  The decrease in the after-tax wage after $t=10$ makes it relatively more attractive to work before $t=10$.  As a consequence,
$n_t$ rises above its initial steady state value before $t=10$. The household uses the extra income to purchase enough capital to keep the
capital-labor ratio and consumption equal to their respective initial steady state values for the first nine periods.  This force  increases $n_t$ in the
periods before $t=10$.  The effect of the build up of capital in the periods before $t=0$ is to attenuate the decrease in the after tax wage
that occurs at $t=10$ because the equilibrium marginal product of labor has been raised higher than it would have been if capital had remained at
its initial steady state value.  From $t=10$ onward, the capital stock is drawn down and the marginal product of labor falls, making the pre-tax
real wage eventually return to its value in the initial steady state.



Mertens and Ravn (2011) use these to  offer an interpretation of contractionary contributions that the Reagan tax cuts made
to the U.S. recession of the early 1980s.
\auth{Ravn, Morten 0.}%
\auth{Mertens, Karel}%




\section{A two-country model}

This section describes a two country version of the basic model of this chapter. The model has a structure similar to ones used in the international real business cycle literature
(e.g., Backus, Kehoe, and Kydland (1992))  and is in the spirit of an analysis of distorting taxes by Mendoza and Tesar (1998), though our  presentation differs from theirs.
We paste two countries together and allow them freely to trade goods, claims on future goods, but not labor.  %This means that there is international trade
%in goods, capital, and financial claims to future goods, but not in labor.
 We shall have to be careful in how we specify taxation of earnings by non residents. %In particular,
%we assume that a domestic resident  pays taxes on rentals from holdings of  foreign capital at the tax rate set by the foreign country.
\auth{Mendoza, Enrique G.}%
\auth{Tesar, Linda L.}%
\auth{Backus, David K.}
\auth{Kehoe, Patrick J.}
\auth{Kydland, Finn E.}

There are now two countries like the one in previous sections. Objects for  the first country are denoted without asterisks, while those for the second country
bear asterisks.  There is international trade in goods, capital, and debt, but not in labor.  We assume that leisure generates utility in neither
country. Preferences over consumption streams in the two countries are ordered by $\sum_{t=0}^\infty \beta^t u(c_t)$ and $\sum_{t=0}^\infty \beta^t u(c_t^*)$, respectively,
where $u(c) = {\frac{c^{1-\gamma}}{1-\gamma}}$ with $\gamma > 0$.  Feasibility
for the world economy is
$$ (c_t+c_t^*) + (g_t + g_t^*) + (k_{t+1} - (1-\delta ) k_t ) + (k_{t+1}^* - (1-\delta ) k_t^* ) = f(k_t) + f(k_t^*) \EQN twocountry20 $$
where $f(k) = A k^\alpha$ with $\alpha \in (0,1)$.

A consumer in country one can hold capital in either country, but pays taxes on rentals from  foreign holdings of capital at the rate set  by the foreign country.
At time $0$, residents in both countries can purchase consumption at date $t$ at a common Arrow-Debreu price $q_t$. Let $\tilde k_t$ be capital in country 2 held by a representative   consumer of country 1. Temporarily, in this paragraph only, let $k_t$  denote the amount of domestic capital owned by the domestic consumer. (In all other paragraphs of our exposition of the two-country model,
 $k_t$ denotes the amount
of capital in country 1.) Let $B_t^f$ be the amount of time $t$   goods  that  the representative  domestic consumer raises by issuing a one-period IOU to the  representative foreign consumer;
    so $B_t^f >0 $ indicates that the domestic consumer is borrowing from abroad at $t$ and $B_t^f < 0$ indicates that the domestic consumer is lending abroad at $t$.  For $t \geq 1$ let $R_{t-1,t}$ be the gross return on a one-period loan from period $t-1$ to period $t$. Define $R_{-1,0} \equiv 1$ and let the
    domestic consumer's initial debt to the foreign consumer be $R_{-1,0} B_{-1}^f$. We assume that returns on loans are not taxed by either
     country.  The budget constraint of a country 1 consumer
is
$$\EQNalign {  & \sum_{t=0}^\infty q_t \left(c_t + \left(k_{t+1} - (1-\delta )k_t\right) + (\tilde k_{t+1} - (1-\delta) \tilde k_t) + R_{t-1,t} B_{t-1}^f \right)  \cr
    &\leq \sum_{t=0}^\infty q_t \left( (\eta_t - \tau_{kt} (\eta_t - \delta))k_t +  (\eta_t^* - \tau_{kt}^* (\eta_t^* - \delta))\tilde k_t
  + (1-\tau_{nt}) w_t n_t - \tau_{ht} + B_t^f \right). \cr
   &  \EQN twocountry21 }  $$
A no-arbitrage condition for $k_0$ and $\tilde k_0$ is
$$ (1-\tau_{k0}) \eta_0 + \delta \tau_{k0}  =  (1-\tau_{k0}^*) \eta_0^* + \delta \tau_{k0}^*.$$
No-arbitrage conditions for $k_t$ and $\tilde k_t$ for  $t \geq 1$ imply
$$\EQNalign{ q_{t-1} & = \left[ (1-\tau_{kt})(\eta_t - \delta) + 1 \right] q_t \cr
             q_{t-1} & = \left[ (1-\tau_{kt}^*)(\eta_t^* - \delta) + 1 \right] q_t , \EQN twocountry22 }  $$
which together imply that after-tax rental rates on capital are equalized across the two countries:
$$ (1- \tau_{kt}^*)( \eta_t^* - \delta)  = (1-\tau_{kt})(  \eta_t - \delta). \EQN twocountry23 $$
No arbitrage conditions for $B_t^f$ for $t \geq 0$ are
$q_t = q_{t+1} R_{t,t+1}$,  which implies that
$$ q_{t-1} = q_t R_{t-1,t} \EQN twocountry23a $$
for $t \geq 1$.

Since domestic capital, foreign capital, and consumption loans bear the same rates of return  by virtue of \Ep{twocountry23} and \Ep{twocountry23a},
portfolios are indeterminate. We are free to set holdings of foreign capital equal to zero in each country if we allow $B_t^f$ to be nonzero.
Adopting this way of resolving portfolio indeterminacy is convenient because it  economizes on the number of initial conditions we have to specify.
Therefore, we set holdings of foreign capital equal to zero in both countries but allow international lending.
Then given an initial level $B_{-1}^f$ of debt from the domestic country to the foreign country $^*$, and
where $R_{t-1,t} = {\frac{q_{t-1}}{q_t}}$, international debt dynamics satisfy
$$ B_t^f = R_{t-1,t} B_{t-1}^f + c_t +(k_{t+1}- (1-\delta) k_t) + g_t - f(k_t) \EQN twocountry3 $$
and
$$ c_t^* + (k_{t+1}^* - (1-\delta) k_t^* ) + g_t^* - R_{t-1,t} B_{t-1}^f = f(k_t^*) - B_t^f . \EQN twocountry4 $$

Firms' first-order conditions in the two countries  are:
$$ \EQNalign{ \eta_t = f'(k_t), & \quad w_t = f(k_t) - k_t f'(k_t) \cr
           \eta_t^* = f'(k_t^*), & \quad w_t^* = f(k_t^*) - k_t^* f'(k_t^*). \EQN twocountry23 }  $$





International trade in goods establishes
$$ {\frac{q_t}{\beta^t}} = {\frac{u'(c_t)}{1+\tau_{ct}}} = \mu^*{\frac{u'(c_t^*)}{1+ \tau_{ct}^*}} , \EQN twocountry5 $$
where $\mu^*$ is a nonnegative number that is a function of the  Lagrange multiplier on the budget constraint for a consumer in country $^*$ and where we  have
normalized the Lagrange multiplier on the budget constraint of the domestic country to set the corresponding $\mu$ for the domestic country to unity.
Equilibrium requires that the following two national Euler equations be satisfied for $t \geq 0$:
$$\EQNalign{ u'(c_t) & = \beta u'(c_{t+1}) \left[ (1-\tau_{kt+1}) (f'(k_{t+1})- \delta ) + 1\right]\left[ {\frac{1+\tau_{ct+1}}{1+\tau_{ct}}} \right]
           \EQN twocountry6 \cr
            u'(c_t^*) & = \beta u'(c_{t+1}^*) \left[ (1-\tau_{kt+1}^*) (f'(k_{t+1}^*)- \delta ) + 1\right]\left[ {\frac{1+\tau_{ct+1}^*}{1+\tau_{ct}^*}} \right]
             \EQN twocountry7 } $$
Given that equation \Ep{twocountry5} holds for all $t\geq 0$, either equation \Ep{twocountry6} or equation \Ep{twocountry7} is redundant.


\subsection{Initial conditions}

As initial conditions, we take the pre-international-trade allocation of capital across countries $(\check k_0, \check k_0^*)$ and
an initial level $B_{-1}^f = 0 $ of international debt owed by the unstarred (domestic)  country to the starred (foreign) country.
\subsection{Equilibrium steady state values}

The following two equations determine steady values for $k$ and $k^*$.
$$\EQNalign{ f'(\overline k) &= \delta + {\frac{\rho}{1-\tau_k}} \EQN steadst_2100\cr
             f'(\overline k^*) & = \delta + {\frac{\rho}{1-\tau_k^*}} \EQN steadst_2101 }
             $$
Given the steady state capital-labor ratios $\overline k$ and $\overline k^*$, the following two equations
determine steady state values of domestic and foreign consumption $\overline c$ and $\overline c^*$ as functions
of a steady state value $\overline B^f$ of debt from the domestic country to country $^*$:
$$\EQNalign{ (\overline c + \overline c^*) & = f(\overline k) + f(\overline k^*) - \delta (\overline k + \overline k^*) -
             (\overline g + \overline g^*) \EQN steadst_2102 \cr
             \overline c & = f(\overline k) - \delta \overline k - \overline g - \rho \overline B^f \EQN steadst_2103
              }$$
Equation \Ep{steadst_2102} expresses  feasibility at a steady state while \Ep{steadst_2103} expresses trade balance, including interest payments,  at a steady state.

\subsection{Initial equilibrium values}
Trade in physical capital and time $0$ debt takes place before production and trade in other goods occurs at
time $0$. We shall always initialize international debt at zero:
$ B_{-1}^f = 0, $
a condition that we use to express that international  trade in capital begins at time $0$.
Given an initial total world-wide capital stock $\check k_0 + \check k_0^*$, initial values of $k_0$ and $k_0^*$
satisfy
$$ \EQNalign{ k_0 + k_0^* & = \check k_0 + \check k_0^* \EQN twocountry1 \cr
              (1-\tau_{k0}) f'(k_0) + \delta \tau_{k0} & =  (1-\tau_{k0}^*) f'(k_0^*) + \delta \tau_{k0}^*. \EQN twocountry2 } $$
              The price  of a unit of capital in either country at time $0$ is
$$ p_{k0} = \left[(1-\tau_{k0} )f'(k_0) + (1-\delta) + \delta \tau_{k0} \right] . \EQN twocountry6 $$
It follows that
$$ B_{k0} = p_{k0}[ k_0 - \check k_0 ] , \EQN twocountry7 $$
which says that the domestic country finances imports of physical capital from abroad by borrowing from the foreign country $^*$.


\subsection{Shooting algorithm}

To apply a shooting algorithm, we would search for pairs $c_0, \mu^*$ that yield a pair $(k_0, k_0^*)$ and paths  $\{c_t, c_t^*, k_t, k_t^*, B_t^f\}_{t=0}^T$ that solve equations
\Ep{twocountry1}, \Ep{twocountry2}, \Ep{twocountry6}, \Ep{twocountry7}, \Ep{twocountry3},
\Ep{twocountry5}, and \Ep{twocountry6}.  The shooting algorithm  `aims' for   $(\overline k, \overline k^*)$ that satisfy the steady-state
equations \Ep{steadst_2100}, \Ep{steadst_2101}.




\midfigure{fig1_kinit}
\centerline{\epsfxsize=3truein\epsffile{fig1_kinit.eps}}
\caption{Response to unforeseen opening of trade at time $1$.
From left to right, top to bottom: $k, c, \bar R, \eta, x,$ and $B^f$. The solid line is the domestic country, the dashed line is the foreign
country and the dashed dotted line is the original
steady state.}
\infiglist{fig1_kinit}
\endfigure
\subsection{Transition exercises}

 In the one-country exercises  earlier in this chapter, announcements of new policies always occurred at time $0$.
In the two-country exercises to follow, we assume that announcements of new paths of tax rates and/or expenditures or trade
regimes all occur at time $1$.  We do this to show some dramatic jumps in  particular variables that occur at
time $1$ in response to announcements about changes that will occur at time 10 and later.  Showing variables at times 0 and
1 helps display some of the outcomes on which we shall  focus here.
The production function is $f(k) = A k^\alpha$. Parameter values are $\beta=.95, \gamma =2, \delta = .2, \alpha = .33, A=1$;
 $g$ is initially
$.2$  in both countries and all distorting taxes are initially $0$ in both countries.

 We describe outcomes from three exercises that illustrate  two
economic forces. The first force  is consumers' desire to smooth
consumption over time, expressed through  households' consumption Euler equations.
The second  force is that equilibrium outcomes must offer no opportunities for arbitrage, expressed through  equations that
equate rates of returns on bonds and capital in both countries.

In the first two experiments, all taxes are lump sum in both countries.  In the third experiment we activate
a tax on capital in the domestic but not the foreign country.
In all experiments, we allow lump sum taxes in both countries  to adjust to satisfy government budget constraints in both countries.

\subsubsection{Opening International Flows}
In our
first example, we study the transition dynamics for two countries when  in period one
newly produced output and stocks of capital, but not labor,  suddenly  become internationally mobile. The two economies are initially
identical
in all  aspects except for one: we start the domestic
economy at its  autarkic steady state, while we start the  foreign economy
at an initial capital stock  below its autarkic steady state. Because  there are no
distorting  taxes on returns to physical capital, capital stocks in both economies   converge to the same level.

In this experiment the domestic country is at its steady state capital stock while the poorer foreign country
has a capital stock that is $.5$ less.  This means that initially, before trade is opened at $t=1$, the
marginal product of capital in the foreign country exceeded the marginal product capital in the domestic country, that the
foreign interest rate $R_{0,1}^*$ exceeded the domestic rate $R_{0,1}$, and that consequently the foreign consumption growth rate exceeded the
domestic consumption growth rate. The disparity of interest rates before trade is opened is a force for physical capital to flow from the domestic
country to the foreign country once when trade is opened at $t=1$.
Figure \Fg{fig1_g-2} presents the transitional dynamics. When countries
become open to trade in goods and capital in period one, there occurs an immediate reallocation of capital from the
capital-rich domestic country to the capital-poor foreign country. This transfer of capital has to take place because if it didn't,
  capital
in different countries  would yield different returns, providing consumers in both countries with arbitrage opportunities.
Those cannot occur in equilibrium. %Hence,
%in equilibrium, capital in the foreign economy will increase up to the point
%where the return to capital in both countries and the return on bonds are
%equated.

Before international  trade had opened, rental rates on capital and interest rates differed across country because
marginal products of capital differed and consumption growth rates differed.  When trade opens at
time $1$ and  capital is reallocated across countries to equalize returns,  the interest rate in the domestic country
jumps  at time one. Because $\gamma =2$, this means that consumption $c$ in the domestic country  must fall. The opposite is
true for the foreign economy. Notice also that figure \Fg{fig1_g-2} shows an
investment spike abroad while  there is a large decline in investment in
the domestic economy. This occurs because capital is reallocated from the domestic country to the foreign one.
This transfer is feasible because investment in capital is
reversible. The foreign country finances this import of physical capital by borrowing from the domestic
country, so $- B^f$ increases. % Budget constraints show that the increase in foreign
%investment is financed via a large increase in foreign private debt ($-B_{t}^f$%
%), so foreign assets $B_t^f$ increase in the domestic country.
 Foreign debt $- B^f$  continues to increase as both economies converge smoothly towards a
steady-state with a positive level of $- \overline{B}^f$. Ultimately, these
differences account for differences in steady-state consumption by $2 \rho
\overline{B^f}$.

%Standard arguments can be used to show that opening financial markets will
%benefit both economies.\footnote{%
%Setup a planner problem. $B_{t}=0$ would imply equal Pareto weights.}
Opening trade in goods and capital at time $1$ benefits consumers in both economies.
 By
 opening up  to capital flows, the foreign country achieves convergence to
a steady-state consumption level at an accelerated rate. This steady-state
consumption rate is lower than what it would be had the economy remained
closed, but this reduction in long-run consumption is more than compensated
by the rapid increase in consumption and output in  the short-run. In contrast, domestic consumption
falls in the short-run as trade allows domestic consumers to accumulate
foreign assets that  eventually support  greater steady-state consumption.

This experiment shows the importance of studying transitional dynamics for
welfare analysis. In this example, focusing only on steady-state consumption
would lead to the false conclusion that opening markets are detrimental for
poorer economies.


\midfigure{fig1_g-2}
\centerline{\epsfxsize=3truein\epsffile{fig1_g.eps}}
\caption{Response to  increase in $g$ at time $10$ foreseen at time $1$.
From left to right, top to bottom: $k, c, \bar R, x, g, B^f$. The dashed-dotted  line is the original
steady state in the domestic country.  The dashed line denotes the foreign country.}
\infiglist{fig1_g}
\endfigure





\subsubsection{Foreseen Increase in $g$}

Figure \Fg{fig1_g-2} presents  transition dynamics after an increase in $g$   in the
domestic economy from $.2$ to $.4$ that is announced ten periods in advance. We start both economies
 from a steady-state with $B_{0}^f = 0$. When the new $g$ path  is
announced at time $1$, consumption smoothing motives induce domestic households to increase
their savings in response to the  adverse shock to domestic private
wealth that is caused at time $1$ by the foreseen increase
in domestic government purchases $g$.  Domestic households plan to  use those savings to  dampen the impact on
consumption in periods after $g$ will have increased ten periods ahead. Households  save partly by accumulating more
domestic capital in the short-run, their only source of assets in the closed economy version of this experiment.
 In an open economy, they have other ways to save, namely,  by lending abroad.
 The no-arbitrage conditions connect adjustments of both types of saving: the increase in
savings  by domestic households will reduce the equilibrium return
on bonds and capital in the foreign economy to prevent arbitrage
opportunities. Confronting  the revised  interest rate path that now begins with lower
interest rates, foreign households increase their rates of consumption and
investment in physical capital. These increases in foreign absorbtion are funded
by  increases in foreign consumers' external debt. After the announcement of
the increase in $g$, the paths for consumption (and capital) in both countries follow
the same patterns because no-arbitrage conditions equate the ratios of their marginal utilities of consumption.
 Both countries continue to accumulate capital until the
increase in $g$ occurs. After that,  domestic households begin consuming some of their
 capital. Again by no-arbitrage conditions,  when  $g$  actually  increases both countries reduce their
investment rates. The domestic economy, in turn, starts
running current-account deficits partially to fund the increase in $g$. This
means that foreign households begin repaying part of their external debt by
reducing their  capital stock. Although not plotted in  figure \Fg{fig1_g-2}, there is
a sharp reduction in gross investment $x$ in both countries when the increase in $g$ occurs. After
$t=10$, all variable converge smoothly towards a new steady state where the
domestic economy persists with positive asset holdings $- B^f$. Ultimately, this
explains why the foreign country ends with lower steady state consumption
than in  the initial steady state.  In the new steady state, minus the sum of the {\it decreases\/} of
consumption rates across the two countries equals the {\it increase} in steady state government expenditures
in the domestic country.\NFootnote{Despite the decrease in its steady state consumption, we have calculated that
$\sum_{t=0}^\infty \beta^t u(c_t^*)$ is higher in the new equilibrium than in the old.}

The experiment teaches  valuable economic lessons. First, it shows how the
consequences of the foreseen increase in $g$ will be distributed across time and households. Second,
it tells how this distribution takes place: through time by accumulating or reducing the
capital stock, and across  households in different countries by running current-account deficits and
surpluses.
\midfigure{fig1_tauk}
\centerline{\epsfxsize=3truein\epsffile{fig1_tauk.eps}}
\caption{Response to  once-and-for-all increase in $\tau_k$ at $t=10$ foreseen at time $t=1$.
From left to right, top to bottom: $k, c, \bar R, \eta, \tau_k$ and $\tau_k^*$,
and $B^f$.  Domestic country (solid line), foreign country (dotted line) and
steady-state values (dot-line).}
\infiglist{fig1_tauk}
\endfigure


\subsubsection{Foreseen increase in $\tau_k$}
We now explore the impact of an increase in capital taxation in the domestic
economy 10 periods after its announcement at $t=1$. Figure \Fg{fig1_tauk} shows equilibrium outcomes. When the increase
in $\tau_k$ is announced,
domestic households become aware that the domestic capital stock will eventually
decline to increase gross returns to equalize  after-tax
returns across countries despite a higher domestic tax rate on returns from capital.
Domestic households will reduce their
capital stock by increasing their rate of consumption. The consequent higher equilibrium
world interest rates then also induces
foreign households to increase consumption. Prior to the  increase in $\tau_k$, the domestic country runs
a current account deficit. When $\tau_k$ is eventually increased,
 capital is rapidly reallocated across borders to
preclude arbitrage opportunities, leading to a lower interest rate on bonds. The fall in
the return on bonds occurs because the capital returns tax $\tau_k$ in the domestic country will
reduce the after-tax return on capital, and because the foreign economy has
a higher capital stock. Foreign households fund this large purchase of
capital with a sharp increase in external debt, to be interpreted as a current account deficit.
After $\tau_k$ has increased, the economies smoothly converge to a
new steady state that features lower consumption rates in both countries and where
the differences in the capital stock equate after-tax returns. It is useful to
note that steady-state consumption in the foreign economy is higher than in the domestic country despite
its perpetually  having  positive liabilities. This occurs because foreign output is larger because
the capital stock held abroad is also larger.

This example shows how, via the no-arbitrage conditions,  both countries share the impact of the shock
 and how  fluctuations in capital stocks  smooth  over
time the adjustments in consumption in both countries.


\section{Concluding remarks}

   In chapter \use{growth1} we shall describe a stochastic version of the basic growth
model and alternative ways of representing its competitive equilibrium.\NFootnote{It will be of particular
interest to learn how to achieve a recursive representation of an equilibrium by finding an appropriate
formulation of a state vector in terms of which to cast an equilibrium.  Because there are endogenous state
variables in the growth model, we shall have to extend the method used in chapter \use{recurge}.}
 Stochastic and nonstochastic versions
of the growth model are widely used throughout aggregative
economics to study a range of policy questions. Brock and Mirman
(1972), Kydland and Prescott (1982), and many others have used  a
stochastic version of the model to approximate features of the
business cycle. In much of the earlier literature on real
business cycle models, the phrase ``features of the business
cycle'' has meant ``particular moments of some aggregate time series
that have been filtered in a particular way to remove trends.''
Lucas (1990) uses a nonstochastic model like the one in this chapter
to prepare rough quantitative estimates of the eventual
consequences of lowering taxes on capital and raising those on
consumption or labor. Prescott (2002) uses a version of the model
in this chapter with leisure in the utility function together with
some illustrative
 (high) labor supply elasticities to construct the argument that in the last two
decades, Europe's economic activity has been depressed relative to
that of the United States because Europe has taxed labor more highly that the
United States.  Ingram, Kocherlakota, and Savin (1994) and  Hall (1997)
use actual data to construct the errors in the Euler equations
associated with stochastic versions of the basic growth model and
interpret them not as computational errors, as in the procedure
recommended in section
\use{sec:accuracy}, but as measures of additional shocks that have to be added to the basic model to make it fit the data.
In the basic stochastic growth model described  in chapter
\use{growth1}, the technology shock  is the only shock, but it
cannot by itself account for the discrepancies that emerge in
fitting all of the model's Euler equations to the data.   A
message of  Ingram, Kocherlakota, and Savin (1994) and  Hall
(1997) is
 that more shocks are required to account for the data.
Wen (1998) and Otrok (2001) build growth models with more shocks
and additional sources of dynamics, fit them to U.S. time series
using likelihood function-based methods, and discuss the
additional shocks and sources of data that are required to match the
data.  See Christiano, Eichenbaum, and Evans (2003) and
Christiano, Motto, and Rostagno (2003) for papers that add a
number of additional shocks and  measure their importance.
Greenwood, Hercowitz, and Krusell (1997) introduced what seems to
be an important additional shock in the form of a technology shock
that impinges directly on the relative price of investment goods.
Jonas Fisher (2006) develops econometric evidence attesting to the
importance of this shock in accounting for aggregate fluctuations.
Davig, Leeper, and Walker (2012) use stochastic versions of the types of models discussed in this chapter to
study issues of intertemporal fiscal balance.
\auth{Davig, Troy}
\auth{Leeper, Eric M.}
\auth{Walker, Todd B}
\auth{Fisher, Jonas} \auth{Christiano, Lawrence J.}
\auth{Eichenbaum, Martin} \auth{Evans, Charles} \auth{Ingram,
Beth Fisher} \auth{Greenwood, Jeremy}\auth{Hercowitz, Zvi}
\auth{Kocherlakota, Narayana R.} \auth{Savin, N.E.} \auth{Hall,
Robert E.} \auth{Prescott, Edward C.}\auth{Lucas, Robert E., Jr.}
\auth{Brock, William A.} \auth{Mirman, Leonard J.}

Schmitt-Grohe and Uribe (2004b) and Kim and Kim (2003) warn that
the linear and log linear approximations described in this chapter
can be treacherous when they are used to compare the welfare under alternative policies
of economies, like the ones described in this
chapter, in which distortions prevent equilibrium allocations from
being optimal ones.  They describe ways of attaining locally more
accurate welfare comparisons by constructing higher order
approximations to decision rules and welfare functions.
\index{approximation!higher order}
\auth{Kim, S.} \auth{Kim, J.} \auth{Schmitt-Grohe, Stephanie}
\auth{Uribe, Martin}


\appendix{A}{Log linear approximations}

Following Christiano (1990), a widespread practice is to obtain
log linear rather than linear approximations.  Here is how this
would be done for the model of this chapter.

  Let $\log k_t = \tilde k_t$ so that $k_t = \exp \tilde k_t$; similarly,
let $\log g_t = \tilde g_t$.   Represent $z_t$ as
$z_t =\left[\matrix{\exp(\tilde g_t)  & \tau_{kt} & \tau_{ct} \cr}
\right]'$ (note that only $g_t$ has been replaced by it's log here).
Then proceed as follows to get a log linear approximation.

\medskip
\noindent{\bf 1.}  Compute the steady state as before.
Set the government policy $z_t = \overline z$, a constant level.
Solve $H(\exp(\tilde k_\infty), \exp(\tilde k_\infty), \exp(\tilde
k_\infty), \overline z, \overline z) =0$ for a steady state
$\tilde k_\infty$. (Of course, this will give the same steady
state for the original unlogged variables  as we got earlier.)

\medskip
\noindent{\bf 2.}  Take first-order Taylor series approximation around $(\tilde
k_\infty,
\overline z)$:
$$ \eqalign{ &H_{\tilde k_t}(\tilde k_t - \tilde k_\infty) +
H_{\tilde k_{t+1}} (\tilde k_{t+1} -\tilde k_\infty)
   + H_{\tilde k_{t+2}}(\tilde k_{t+2} - \tilde k_\infty) \cr & +H_{z_t} (z_t - \overline z)
  + H_{z_{t+1}}(z_{t+1} - \overline z) = 0. \cr} \EQN ric12 $$
(But please remember here  that the first component of $z_t$ is now
$\tilde g_t$.)\medskip
\noindent{\bf 3.}  Write the resulting  system as
$$ \phi_0 \tilde k_{t+2} + \phi_1 \tilde k_{t+1} + \phi_2 \tilde
 k_t = A_0+ A_1 z_t + A_2 z_{t+1}
\EQN ric120 $$
or
$$ \phi(L) \tilde k_{t+2} = A_0 + A_1 z_t + A_2 z_{t+1} \EQN ric120a $$
where $L$ is the lag operator (also called the backward shift
operator). Solve the linear difference equation \Ep{ric120a}
exactly as before, but for the sequence  $\{\tilde k_{t+1}\}$.
\medskip
\noindent{\bf 4.}  Compute $k_t = \exp(\tilde k_t)$, and also remember to
exponentiate $\tilde g_t$, then use equations \Ep{formula1} to
compute the associated  prices and quantities. Compute the Euler
equation errors as before.





%\section{Exercises}
\showchaptIDfalse
\showsectIDfalse
\section{Exercises}
\showchaptIDtrue
\showsectIDtrue
\medskip
\noindent{\it Exercise \the\chapternum.1} \quad {\bf Tax reform: I} \medskip
\noindent Consider the following economy populated by a government
and a representative household. There is no uncertainty,  and the
economy and the representative household and government within it
live forever. The government consumes a constant amount $g_t = g
>0, t \geq 0$. The government also sets sequences for two
types of taxes, $\{\tau_{ct},\tau_{ht}\}_{t=0}^\infty$. Here
$\tau_{ct}, \tau_{ht}$ are, respectively, a  possibly time-varying
flat-rate tax  on consumption and a time-varying  lump-sum or
``head'' tax. The preferences of the household are ordered by
$$ \sum_{t=0}^\infty \beta^t u(c_t) , $$
where $\beta \in (0,1)$ and $u(\cdot)$ is strictly concave,
increasing, and twice continuously differentiable. The representative household is
endowed wtih one unit of labor each period and does not value leisure.  The feasibility
condition in the economy is
$$ g_t + c_t + k_{t+1} \leq f(k_t) + (1-\delta) k_t $$
where $k_t$ is the stock of capital owned by the household at the
beginning of time $t$ and $\delta \in (0,1)$ is a depreciation
rate. At time $0$, there are complete markets for dated
commodities. The household faces the budget constraint:
$$ \sum_{t=0}^\infty \left\{q_t[(1+\tau_{ct}) c_t +
k_{t+1} - (1-\delta) k_t] \right\}   \leq \sum_{t=0}^\infty q_t \left\{ \eta_t  k_t + w_t
      -  \tau_{ht} \right\}   $$
where we assume that the household inelastically supplies one unit
of labor, and $q_t$ is the price of date $t$ consumption goods measured in the numeraire at time $0$,
$\eta_t$ is the rental rate of date $t$ capital measured in consumption goods at time $t$, and $w_t$ is the
wage rate of date $t$ labor  measured in consumption goods at time $t$. Capital is neither taxed nor
subsidized.

A production firm rents labor and capital. The production function
is $f(k)n$, where $f'>0, f''<0$.  The value of the firm is
$$ \sum_{t=0}^\infty q_t \bigl[ f(k_t)n_t - w_t n_t  - \eta_t k_t\bigr], $$
where  $k_t $ is the firm's capital-labor ratio and $n_t $ is
the amount of labor it hires.

The government sets $g_t$ exogenously and must set $\tau_{ct},
\tau_{ht}$ to satisfy the budget constraint:

$$ \sum_{t=0}^\infty q_t (\tau_{ct} c_t  + \tau_{ht})
  = \sum_{t=0}^\infty q_t g_t . \leqno(1) $$



\medskip


\noindent{\bf a.}  Define a competitive equilibrium.

\medskip
\noindent{\bf b.}  Suppose that historically the government had
unlimited access to lump-sum taxes and availed itself of them.
Thus, for a long time the economy had  $g_t = \overline g >0,
\tau_{ct} =0$. Suppose that this situation had been expected to go
on forever. Tell how to find the steady-state capital-labor ratio
for this economy.
\medskip
\noindent{\bf c.}  In the economy depicted in b, prove that the
timing of lump-sum taxes is irrelevant.
\medskip
\noindent{\bf d.}  Let $\bar k_0$ be the steady value of $k_t$
that you found in part {\bf b}.   Let this be the initial value of
capital at time $t=0$  and consider the following  experiment.
Suddenly and unexpectedly, a court decision rules that lump-sum
taxes are illegal and that starting at time $t=0$, the government
must finance expenditures using the consumption tax $\tau_{ct}$.
The value of $g_t$ remains constant at $\overline g$. Policy
advisor number 1 proposes the following tax policy: find a {\it
constant\/} consumption tax  that satisfies the budget constraint
(1), and impose it from time $0$ onward. Please compute the new
steady-state  value of $k_t$ under this policy. Also, get as far
as you can in analyzing the transition path from the old steady
state to the new one.

\medskip
\noindent{\bf e.}  Policy advisor number 2 proposes the following
alternative policy.  Instead of imposing the increase in
$\tau_{ct}$ suddenly, he proposes to ease the pain by postponing
the increase for $10$ years. Thus, he/she proposes to set
$\tau_{ct} = 0$ for $t=0, \ldots, 9$, then to set $\tau_{ct}=
\overline \tau_{c}$ for $t\geq 10$. Please compute the steady-state
 level of capital associated with this policy. Can you say
anything about the transition path to the new steady-state $k_t$
under this policy?

\medskip
\noindent{\bf f.}  Which policy is better, the one  recommended
 in {\bf d} or the one in {\bf e}?


\medskip\noindent
{\it Exercise \the\chapternum.2}\quad {\bf Tax reform: II}
\medskip \noindent Consider the following economy populated by a
government and a representative household. There is no uncertainty,
and the economy and the representative household and government
within it last forever. The government consumes a constant amount
$g_t = g
>0, t \geq 0$. The government also sets sequences of   two
types of taxes, $\{\tau_{ct},\tau_{kt}\}_{t=0}^\infty$. Here
$\tau_{ct}, \tau_{kt}$ are, respectively, a  possibly time-varying
flat-rate tax on consumption and a time-varying flat-rate tax on
earnings from capital. The preferences of the household are
ordered by
$$ \sum_{t=0}^\infty \beta^t u(c_t) , $$
where $\beta \in (0,1)$ and $u(\cdot)$ is strictly concave,
increasing, and twice continuously differentiable. The household is endowed with one unit of labor
each period and does not value leisure.  The feasibility
condition in the economy is
$$ g_t + c_t + k_{t+1} \leq f(k_t) + (1-\delta) k_t $$
where $k_t$ is the stock of capital owned by the household at the
beginning of time $t$ and $\delta \in (0,1)$ is a depreciation
rate. At time $0$, there are complete markets for commodities at all dates. The household faces the budget constraint:
$$\eqalign{ & \sum_{t=0}^\infty \left\{ q_t[(1+\tau_{ct}) c_t +
k_{t+1} - (1-\delta) k_t] \right\}  \cr
 &      \leq \sum_{t=0}^\infty q_t \left\{ \eta_t k_t - \tau_{kt} (\eta_t - \delta)  k_t + w_t
       \right\} \cr }  $$
where we assume that the household inelastically supplies one unit
of labor, and $q_t$ is the price of date $t$ consumption goods in units of the numeraire at time $0$,
$\eta_t$ is the rental rate of date $t$ capital in units of time $t$ goods, and $w_t$ is the
wage rate of date $t$ labor in units of time $t$ goods.

A production firm  rents labor and capital. The value of the firm
is
$$ \sum_{t=0}^\infty q_t \bigl[ f(k_t)n_t - w_t n_t  - \eta_t k_t n_t\bigr], $$
where here $k_t $ is the firm's capital-labor ratio and $n_t $ is
the amount of labor it hires.

The government sets $\{g_t\}$ exogenously and must set the
sequences $\{\tau_{ct}, \tau_{kt}\}$ to satisfy the budget
constraint:

$$ \sum_{t=0}^\infty q_t \bigl(\tau_{ct} c_t  + \tau_{kt} (\eta_t  - \delta) k_t \bigr)
  = \sum_{t=0}^\infty q_t g_t . \leqno(1) $$



\medskip


\noindent{\bf a.}  Define a competitive equilibrium.

\medskip
\noindent{\bf b.}  Assume an initial situation in which from time
$t\geq 0$ onward, the  government finances a constant stream of
expenditures $g_t=\overline g$  entirely by levying a constant tax
rate $\tau_k$ on capital and a zero consumption tax. Tell how to
find steady-state levels of capital, consumption, and the rate of
return on capital.


\medskip
\noindent{\bf c.}  Let $\bar k_0$ be the steady value of $k_t$
that you found in part {\bf b}.   Let this be the initial value of
capital at time $t=0$  and consider the following  experiment.
Suddenly and unexpectedly, a new party comes into power that
repeals the tax on capital, sets $\tau_k =0$ forever, and finances
the same constant level of $\overline g$ with a flat-rate tax on
consumption.
  Tell what happens to the new steady-state values of capital,
  consumption, and the return on capital.

\medskip
\noindent{\bf d.}  Someone recommends comparing the two
alternative policies of (1) relying completely on the taxation of
capital as in the initial equilibrium and (2) relying completely
on the consumption tax, as in our second equilibrium, by comparing
the discounted utilities of consumption in steady state, i.e., by
comparing ${1\over 1-\beta} u(\overline c)$ in the two equilibria,
where $\overline c$ is the steady-state value of consumption.  Is
this a good way to measure the costs or gains of one policy {\it vis-a-vis\/}
 the other?


\medskip
\noindent{\it Exercise \the\chapternum.3} \quad {\bf Anticipated productivity shift} \medskip
\medskip
\noindent  An infinitely lived  representative household
has preferences over a  stream of consumption of a single good that are
ordered by
$$ \sum_{t=0}^\infty \beta^t u(c_t) , \quad \beta \in (0,1) $$
where $u$ is a strictly concave, twice continuously differentiable,
one-period utility function, $\beta $ is a discount factor, and
$c_t$ is time $t$ consumption.  The technology is:
$$\eqalign{ c_t + x_t & \leq f(k_t)n_t \cr
      k_{t+1} & = (1-\delta) k_t + \psi_t x_t \cr} $$
where for $t\geq 1$
$$ \psi_t = \cases{1 & for $t < 4$ \cr
                   2 & for $ t\geq 4$. \cr} $$
Here $f(k_t)n_t$ is output, where $f>0, f' >0, f''<0$, $k_t$ is capital
per unit of labor input,  and $n_t$ is labor input.
 The household supplies one unit of labor inelastically.  The initial capital
stock $k_0$ is given and is owned by the representative household.
In particular, assume that $k_0$ is at the optimal steady value for $k$ presuming that $\psi_t$ had been
equal to $1$ forever. There is no uncertainty. There is no government.

\medskip
\noindent{\bf a.}  Formulate the planning problem for this economy  in the space of
sequences and form the pertinent Lagrangian.  Find a formula for the optimal steady-state level of capital.
How does a permanent  increase in  $\psi$ affect the steady values of
$k, c$, and $x$?
\medskip
\noindent{\bf b.} Formulate the planning problem for this economy recursively
(i.e., compose a Bellman equation for the planner).  Be careful to give a complete
description of the state vector and its law  of motion.  (``Finding the state
is an art.'')
\medskip
\noindent{\bf c.}  Formulate an (Arrow-Debreu) competitive equilibrium
with time $0$ trades, assuming the following decentralization.
Let the household  own the stocks of capital and labor and in each period
 let the household rent them to the firm.  Let the household choose
the investment rate each period.  Define an appropriate price system and compute
the first-order necessary conditions for the household and for the firm.

\medskip
\noindent{\bf d.} What is the connection between a solution of the planning
problem and the competitive equilibrium in part c?
Please link the prices in part c to corresponding objects
in the planning problem.
\medskip
\noindent{\bf e.}  Assume
that $k_0$ is given by the steady-state value that corresponds
to  the assumption that $\psi_t$ had been
equal to $1$ forever, and had been expected to remain equal to $1$ forever.
Qualitatively
describe the evolution of the economy from time $0$ on.   Does the jump in $\psi$ at
$t=4$ have any effects that precede it?


\medskip
\noindent{\it Exercise \the\chapternum.4} \quad {\bf A capital levy} \medskip
\medskip
\noindent
A  nonstochastic economy produces one good that can be allocated among consumption, $c_t$, government purchases, $g_t$,  and gross investment, $x_t$.
The economy-wide resource constraints are
$$\eqalign{ c_t + g_t + x_t & \leq f(k_t) \cr
    k_{t+1} & = (1-\delta) k_t + x_t , \cr }$$
where $\delta \in (0,1)$ is a depreciation rate,  $k_t$ is the capital stock, and $f(k_t)$ gives production as a function of capital, where
 $f(k) = A k^\alpha$ with $\alpha  \in (0,1)$.  A single representative consumer owns the capital stock and one unit of labor.
 The consumer rents capital and labor to a competitive firm each period.  The consumer ranks consumption plans according to
 $$ \sum_{t=0}^\infty \beta^t u(c_t) $$
 where $u(c) = {c^{1-\gamma}\over 1-\gamma}$, with $\gamma \geq 1$.  The household supplies one unit of labor inelastically each period.

 The government has only one tax at its disposal, a one-time capital levy through which it confiscates part of the capital stock from the private sector. When the government imposes a capital levy, we
 assume that it sends the consumer a tax bill for a fraction $\phi$ of the beginning of period capital stock. Below, you will be asked to compare consequences
 of levying this tax either at the beginning of time $T=0$ or at the beginning of time $T=10$.  The fraction can  exceed
 $1$ if that is necessary to finance the government budget. The government is allowed to impose no other taxes. Among other things,
 this means that it cannot impose a direct lump sum or `head' tax .

 \medskip
 \noindent {\bf a.} Define a competitive equilibrium with time $0$ trading.

 \medskip
 \noindent{\bf b.}  Suppose that  before time $0$ the economy had been in a steady state in which $g$ had always been zero and had been
 expected always to equal zero.  Find a formula for the initial steady state capital stock in a competitive equilibrium with time
 zero trading. Let this value be $\overline k_0$.

 \medskip
 \noindent {\bf c.}  At time $0$, everyone suddenly wakes up to discover that from time $0$ on, government expenditures will
 be $\overline g> 0$, where $ \overline g + \delta \overline k_0 < f(\overline k_0)$, which implies that the new level of government expenditures
 would be feasible in the old steady state.  Suppose that the government finances the new path of expenditures by a capital levy at time
 $T=0$.  The government imposes a capital levy by  sending  the household  a bill for a fraction of the value of its capital at the time indicated.
  Find  the new steady state value of the capital stock in a competitive equilibrium. Describe an algorithm to compute the fraction of the capital stock that the government must tax away at time $0$ to finance
 its budget.  Find  the new steady state value of the capital stock in a competitive equilibrium. Describe the time paths of
 capital, consumption, and the interest rate from $t=0$ to $t= +\infty$ in the new equilibrium and compare them with their counterparts in the initial $g_t \equiv
 0$ equilibrium.
 \medskip

 \noindent {\bf d.}  Assume the same new path of government expenditures indicated in part {\bf c}, but now assume
 that the government imposes the one-time capital levy at time $T=10$, and that this is foreseen at time $0$.
 Find  the new steady state value of the capital stock in a competitive equilibrium that is associated with this tax policy.
 Describe an algorithm to compute the fraction of the capital stock that the government must tax away at time $T=10$ to finance
 its budget. Describe the time paths of
 capital, consumption, and the interest rate in this new equilibrium and compare them with their counterparts in part {\bf b} and
 in the initial $g_t \equiv
 0$ equilibrium.

 \medskip
 \noindent {\bf e.}  Define a competitive equilibrium with sequential trading of one-period Arrow securities.  Describe how to compute such an equilibrium. Describe the time path of the
 consumer's holdings of one-period securities  in a competitive equilibrium with one period Arrow securities
 under the government tax policy assumed in part {\bf d}. Describe the time path of government debt.






 %%%%% New exercises %%%%%%
%
%
%\medskip
%\noindent{\it Exercise \the\chapternum.5} \quad {\bf Investment tax credit} \medskip
%\medskip
%
%
%\noindent
%Consider a non-stochastic economy with
%an infinitely-lived  representative household. The household
%orders consumption, leisure streams $\{c_t, \ell_t\}_{t=0}^\infty$ according
%to
%$$
% \sum_{t=0}^\infty \beta^t u(c_t), \qquad \beta \in (0,1)
%$$
%where $u(\cdot)$ is  increasing, strictly concave, and
%twice continuously differentiable.
%The household is endowed with one unit of time that can be used for leisure
%$\ell_t$ and labor $n_t$; $\ell_t+n_t=1$.
%
%A single good is produced with labor $n_t$ and capital $k_t$ as inputs.
%The output can be consumed by households, consumed by the government, or used
%to augment the capital stock. The technology is described by
%$$c_t + g + k_{t+1} = F(k_t,n_t) + (1-\delta) k_t,                                                                          $$
%where $g>0$ is an exogenous constant amount of government purchases,
%$F(k,n)$ is a constant-returns-to-scale production function, and
%$\delta \in (0,1)$ is the rate at which capital depreciates.
%
%The government raises funds by levying lump-sum taxes (``head taxes'') and
%taxes on capital income, and by issuing one-period bonds.
%Specifically, in period $t$,
%the government levies a lump-sum tax $\tau^h_t$ on the household
%and a tax rate $\tau^k_{t}$ on capital income. Moreover, the
%government pays the household an invesment tax credit at the
%rate $\tau^i_t$ on new capital investments. Hence,
%the household's budget constraint in period $t$ becomes
%%\begin{eqnarray*}
%%c_t + k_{t+1} + \frac{b_{t+1}}{R_t} &\leq& w_t n_t
%%+ (1-\tau^k_t) r_t k_t - \tau^h_t\\
%%&& + \tau^i_t\left[k_{t+1} - (1-\delta)k_t\right]
%%+ (1-\delta) k_t + b_t,
%%\end{eqnarray*}%
%$$\eqalign{
% c_t + k_{t+1} + \frac{b_{t+1}}{R_t} &\leq w_t n_t
%+ (1-\tau^k_t) \eta_t k_t - \tau^h_t\cr
%& + \tau^i_t\left[k_{t+1} - (1-\delta)k_t\right]
%+ (1-\delta) k_t + b_t,
%}$$
%where $w_t$ and $\eta_t$ are the wage rate and the rental rate on capital,
%respectively. The term $\tau^i_t\left[k_{t+1} - (1-\delta)k_t\right]$
%is the household's investment tax credit when investing an amount
%$\left[k_{t+1} - (1-\delta)k_t\right]$ in new capital. The household's
%holdings of one-period bonds between periods $t$ and $t+1$ are denoted
%$b_{t+1}$, at the gross interest rate $R_t$.
%
%
%
%\medskip
%
%\noindent {\bf a.} Define a competitive equilibrium.\medskip
%
%\noindent {\bf b.} Use a no-arbitrage argument to derive a
%relationship between prices $R_t$ and $\eta_{t+1}$ that must hold
%in an equilibrium.\medskip
%
%\noindent {\bf c.} Suppose that historically the government had
%unlimited access to lump-sum taxes and had only levied lump-sum taxes
%and had neither taxed capital income nor subsidized new investments.
%Thus, it had been true for a long time that $\tau^k_t = \tau^i_t=0$. Suppose that this
%situation had been expected to go on forever. Derive the
%steady-state capital stock for this economy.\medskip
%
%\vskip.25cm
%\noindent
%Let $\bar{k}_{0}$ be the steady value of $k_{t}$ that
%you found in part c. Let this be the initial value of capital at time $t=0$
%and consider the following situation. Suddenly and unexpectedly, a court
%decision rules that lump-sum taxes are illegal and that starting at time $%
%t=0 $, the government must finance expenditures using the capital income
%tax $\tau^k_t$.
%Policy advisor number 1 proposes the following tax policy:
%set the capital tax rate and the investment tax credit equal to the same
%{\it constant\/} rate $\bar \tau$ from time $0$ onward, i.e.,
%$\tau^k_t=\tau^i_t=\bar\tau$ for $t\geq0$. The policy advisor
%says that she has found a rate $\bar\tau\in(0,1)$ for which her policy is government
%budget feasible.\medskip
%
%\vskip.25cm
%\noindent {\bf d.} Assume that policy maker 1's tax policy is
%indeed government budget feasible. How does the policy affect the
%household's incentives to accumulate capital? Compute the
%new steady-state value of $k_{t}$ under the policy. Also, please analyze
%the transition path from the initial steady state specified above to the
%new one.\medskip
%
%\noindent{\bf e.} In the new steady state, find an expression for
%tax revenues that are raised from capital taxation net of investment tax
%credits, i.e., find an expression for
%$\tau^k_t \eta_t k_t - \tau^i_t\left[k_{t+1} - (1-\delta)k_t\right]$
%when evaluated at the new steady-state capital stock.
%\medskip
%
%\noindent {\bf f.} Policy maker 2 argues that leaving out the investment
%tax credit would improve welfare, because the distortionary
%capital income tax can then be lowered. Hence, she proposes the following
%alternative tax policy: $\tau^i_t=0$ and $\tau^k_t=\hat\tau^k$
%for $t\geq0$. Assume that there exists such a policy $\hat\tau^k\in(0,1)$
%that is government budget feasible. Compute the steady-state value of
%$k_{t}$ under this policy.\medskip
%
%\noindent  {\bf g.} Which policy yields higher welfare, the one
%recommended by policy maker 1 or policy maker 2?
%
%
%
%
%
%%%%   HHHHH first Anmol insert starts
%
%
%


\medskip
\noindent{\it Exercise \the\chapternum.5}
\medskip
\noindent
 A representative consumer has preferences ordered by
 $$ \sum_{t=0}^\infty \beta^t \log(c_t), \;\;  0< \beta<1$$
 where $\beta = {\frac{1}{1+\rho}}, \rho>0$, and $c_t$ is the consumption per worker.
The technology is $$ y_t=f(k_t)=zk_t^\alpha, \;\; 0<\alpha<1, \;\; z>0 $$
where $y_t$ is the output per unit labor and $k_t$ is capital per unit labor
$$\eqalign{
y_t & =c_t+x_t+g_t \cr
  k_{t+1} & =(1-\delta)k_t+x_t, \;\;  0 < \delta<1 \cr
}$$
and $x_t$ is  gross investment per unit of labor and $g_t$ is  government expenditures per unit of labor.
Assume a competitive equilibrium with a price system $\{q_t,\eta_t,w_t\}_{t=0}^\infty$ and a government policy $\{g_t,\tau_{ht}\}_{t=0}^\infty$

 Assume that the government finances its expenditures by levying lump sum taxes. There are no distorting taxes. Assume that at time 0, the economy begins
 with a capital per unit of labor $k_0$ that equals the steady state value appropriate for an economy in which $g_t$ had been zero forever.
\medskip
\noindent {\bf a.} Find a formula for the steady state capital stock when $g_t=0 \; \forall t$.
\medskip

\noindent {\bf b.} Compare the steady state capital labor ratio $\overline{k}$ in the competitive equilibrium with the capital labor ratio $\tilde{k}$ that maximizes steady state consumption per capita, i.e , $\tilde{k}$ solves $$\tilde{c}=\max_k \bigl( f(k)-\delta k \bigr)$$
Is $\tilde{k}$ greater than or less that $\overline{k}$? If they differ, please tell why. Is $\tilde{c}$ greater or less than $\overline{c} = f(\overline{k})-\delta  \overline{k}$? Explain why.
\medskip

\noindent {\bf c.} Now assume that at time 0, $g_t$ suddenly jumps to the value $g={\frac{1}{2}}\overline{c}$ where
$\overline{c}$ is the value of consumption per capita in the initial steady state in which $g$ was zero forever. Starting from $k_0 = \overline{k}$ for the old $g=0$ steady state, find the time paths of $\{c_t,k_{t+1}\}_{t=0}^\infty$ associated with the new path $g_t=g>0 \; \forall t$ for government expenditures per capita.
Also show the time path for $\bar R_{t+1} \equiv (1-\delta) + f'(k_{t+1})$
Explain why the new paths are as they are.


\medskip
\noindent{\it Exercise \the\chapternum.6} \quad {\bf Trade and growth} \medskip
\medskip
\noindent
{\bf Part  I}
\medskip \noindent
Consider the problem of a planner in a small economy. When the economy is closed to international trade, the planner chooses $\{c_t,k_{t+1}\}_{t=0}^\infty$ to maximize
$$ \sum_{t=0}^\infty \beta^t u(c_t), $$
where $ 0< \beta<1, \beta = {\frac{1}{1+\rho}}, \rho>0$
subject to
$$c_t+k_{t+1}=f(k_t)+(1-\delta)k_t,  \quad  \delta \in (0,1)$$
\noindent where $u(c_t)={\frac{c_t^{1-\gamma}}{1-\gamma}}$, $\gamma>0$, $f(k_t)=zk_t^\alpha$, and $0<\alpha<1$.


\medskip
\noindent
Let $\overline{k}$ be the steady state value of $k_t$ under the optimal plan.
\medskip
\noindent{\bf a.} Find a formula for $\overline{k}$.
\medskip
\noindent{\bf b.} Assume that $k_0<\overline{k}$. Describe time paths for $\{c_t,k_{t+1}\}_{t=0}^\infty$ and $\bar R_{t+1}=(1-\delta) + f'(k_{t+1})$.
\medskip
\noindent{\bf c.} What is the steady state value of $\bar R_{t+1}$?\medskip

\noindent{\bf d.} Is $\bar R_{t+1}$ less or greater than its steady state value when $k_{t+1} < \overline{k}$?

\medskip\noindent
{\bf Part  II}
\medskip \noindent
Now assume that the economy is open to international trade in capital and financial assets. Assume that there is a fixed world gross rate of return $R=\beta^{-1}$ at which the planner can borrow or lend, what is often called a `small open economy' assumption.
The planner can use the proceeds of borrowing to purchase goods on the international market. These goods can be used to augment capital or to consume.\medskip \noindent
Let $\overline{k}_0$ be the level of initial capital $(\overline{k}_0<\overline{k})$ before the country opens up to trade just before time 0.  Let $\overline{k}_0$ be the same initial capital per capita $\overline{k}_0 < \overline{k}$ studied in parts {\bf a}-{\bf d}.\medskip \noindent
At time $t=-1$, after $\overline{k}_0$ was set, trade opens up. At time $t=-1$, the planner can issue IOU's or bonds in amount $B_{-1}$ and use the proceeds to purchase capital, thereby setting
$$k_0=\overline{k}_0+B_{-1} $$
where $B_{-1}$ is denominated in time -1 consumption goods. The bonds are one period in duration and bear the constant world gross interest rate $R=\beta^{-1}$.
For $t\geq 0$, the planner faces the constraints
$$ c_t+k_{t+1}+RB_{t-1}=f(k_t)+(1-\delta)k_t+B_t$$
Here $RB_{t-1}$ is the sum of interest and  principal on the bonds issued at $t-1$ and $B_t$ is the amount of one-period bonds issued at $t$. \medskip \noindent
The planning problem in the small open economy is now to choose $\{c_t,k_{t},B_t\}_{t=0}^\infty$ and $B_{-1}$, subject to $\overline{k}_0$ given. (Notice that $k_0$ is a choice variable and that $\overline{k}_0$ is an initial condition.)
Please solve the  planning problem in the small open economy and compare outcomes to those in the closed economy.
 Is  welfare $\sum_{t=0}^\infty \beta^t u(c_t)$ higher in the ``open'' or ``closed'' economy?



\medskip

\bigskip


The next several problems assume the following environment. A representative consumer has preferences ordered by
$$ \sum_{t=0}^\infty \beta^t u(c_t),   \;\;  0< \beta<1,\quad   \;\;  \beta = {\frac{1}{1+\rho}}, \quad   \;\; \rho>0$$
where $c_t$ is the consumption per worker and where
$$ u(c) = \cases{{\frac{c^{1-\gamma}}{1-\gamma}}  & for $\gamma>0$ and $\gamma \ne 1$ \cr
                   \log(c) & if $ \gamma=1.$ \cr
}$$
%$$ u(c) = \cases{{\frac{c^{1-\gamma}}{1-\gamma}}  &\gamma>0,\;\; \gamma \ne 1\cr
%                   \log(c) & if  \gamma=1.
%}$$
%$$ \psi_t = \cases{1 & for $t < 4$ \cr
%                   2 & for $ t\geq 4$. \cr} $$

\noindent The technology is
$y_t=f(k_t)=zk_t^\alpha, 0<\alpha<1,  z>0$,
where $y_t$ is the output per unit labor and $k_t$ is capital per unit labor and
$$\eqalign{
y_t & =c_t+x_t+g_t \cr
  k_{t+1} & =(1-\delta)k_t+x_t \quad  , \;\;  0 < \delta<1 \cr
}$$
where $ x_t$ is  gross investment per unit of labor and $g_t$ is  government expenditures per unit of labor.
The government finances its expenditures by levying some combination of a flat rate tax $\tau_{ct}$ on the value of consumption goods purchased at $t$,
a flat rate tax $\tau_{nt}$ on the value of labor earnings at $t$, a flat rate tax $\tau_{kt}$ on earnings from capital at $t$ and a lump sum tax of $\tau_{ht}$ in time $t$ consumption goods per worker at time $t$.\medskip
\noindent Let $\{q_t,\eta_t,w_t\}_{t=0}^\infty$ be a price system.


\medskip
\noindent{\it Exercise \the\chapternum.7}
\medskip
\noindent
 Consider an economy  in which $g_t=\bar g > 0 \;\; \forall\; t\geq0$
and in which initially the government finances all expenditures by lump sum taxes.
\medskip
\noindent {\bf a.} Find a formula for the steady state capital labor ratio $k_t$ for this economy.
Find  formulas for the steady state level of $c_t$ and $\bar R_t = [(1-\delta + f'(k_{t+1})]$
\medskip
\noindent {\bf b.} Now suppose that starting from $k_0=\bar k$, i.e., the steady state that you computed in part {\bf a}, the government suddenly increases the tax on earnings from capital to a constant level $\tau_{k} > 0 $. The government adjusts lump sum taxes to keep the government budget balances. Describe competitive equilibrium time paths for $c_t, k_{t+1},\bar R_t$ and their relationship to corresponding values in the old steady state that you described in part {\bf a}.
\medskip
\noindent {\bf c.} Describe how the \underbar{shapes} of the paths that you found in part {\bf b} depend on the curvature parameter $\gamma$ in the utility function  $u(c) = {\frac{c^{1-\gamma}}{1-\gamma}}$. Higher values of $\gamma$ imply higher curvature and more aversion to consumption path that fluctuate.
Higher values of $\gamma$ imply that the consumer values smooth consumption paths even more. \medskip\noindent
%Thus, here is what we expect:
%\medskip
%\midfigure{fig6}
%\centerline{\epsfxsize=3truein\epsffile{fig6.eps}}
%\caption{The dashed lines are the steady states.The solid line is for high $\gamma$ while the dashed-dotted line are for low $\gamma$}
%\infiglist{fig6}
%\endfigure
%
%\noindent Things look this way because of the consumer's attitude about consumption stability. The interest rate and $k_t$ adjust to make this happen.
%\medskip \noindent
%\medskip
%\noindent{\bf Hint:} Recall the forces behind formula  \Ep{lambda_gamma} in section \use{sec:lambda_2}.% verifies this by giving a formula for $\lambda$ as a function of $\gamma$ where $\lambda$.
\medskip
\noindent {\bf d.} Starting from the steady state $\bar k$ that you computed in part a, now consider a situation in which the government announces at time 0 that starting in period 10 the tax on earnings from capital $\tau_k$ will rise permanently to $\overline {\tau_k}>0$. The government adjusts its lump sum taxes to balance it's budget.
\medskip
\indent  {\bf i)} Find the new steady state values for $k_t,c_t,\bar R_t$.
\medskip
\indent  {\bf ii)} Describe the shapes of the transition paths from the initial steady states to the new one for $k_t,c_t,\bar R_t$.
\medskip
\indent  {\bf iii)} Describe how the shapes of the transition paths depend on the curvature parameter $\gamma$ in the utility function u(c).
\medskip
\underbar{Hint} : When $\gamma$ is bigger, consumers more strongly prefer smoother consumption paths.  Recall the forces behind formula  \Ep{lambda_gamma} in section \use{sec:lambda_2}.


\medskip
\noindent{\it Exercise \the\chapternum.8}
\quad {\bf Trade and growth, version II} \medskip\noindent
{\bf Part  I}
\medskip \noindent
Consider the problem of the planner in a small economy. When the economy is \underbar{closed to trade}, the planner chooses $\{c_t,k_{t+1}\}_{t=0}^\infty$ to maximize
$$ \sum_{t=0}^\infty \beta^t u(c_t), \quad  0< \beta<1, \quad  \beta = {{\frac{1}{1+\rho}}}, \rho>0$$
subject to
$$c_t+k_t+1=f(k_t)+(1-\delta)k_t,  \delta \in (0,1)$$
\noindent where% $$\eqalign{u(c_t)={\frac{c_t^{1-\gamma}}{1-\gamma}} \quad  &, \;\; \gamma>0\cr
%f(k_t)=zk_t^\alpha \quad &,\;\;0<\alpha<1
%}$$
\medskip
\noindent
Let $\overline{k}$ be the steady state value of $k_t$ under the optimal plan.
\medskip
\noindent{\bf a.} Find a formula for $\overline{k}$.
\medskip
\noindent{\bf b.} Assume that $k_0 > \overline{k}$. Describe time paths for $\{c_t,k_{t+1}\}_{t=0}^\infty$ and $\bar R_{t+1}=(1-\delta) + f'(k_{t+1})$.
\medskip
\noindent{\bf c.} What is the steady state value of $\bar R_{t+1}$?\medskip

\noindent{\bf d.} Is $\bar R_{t+1}$ less or greater than its steady state value when $k_{t+1} > \overline{k}$?

\medskip\noindent
{\bf Part  II}
\medskip
\noindent{\bf e.} Now assume that the economy is open to international trade in capital and financial assets. Assume that there is a fixed world gross rate of return $R=\beta^{-1}$ at which the planner can borrow or \underbar{lend}.
In particular, the planner is free to use the following plan. The planner can sell all of its capital $\overline{k}_0$ and simply consume the interest payments.
\noindent
Let $\overline{k}_0$ be the level of initial capital $(\overline{k}_0>\overline{k})$ just before the country opens up to trade just before time 0.  Let $\overline{k}_0$ be the initial capital per capita $\overline{k}_0 >\overline{k}$ studied in parts {\bf a}- {\bf d}.
At time $t=-1$, after $\overline{k}_0$ was set, trade opens up. At time $t=-1$, the planner \underbar{sells} $\overline{k}_0$ in exchange for IOU's or bonds from the rest of the world in the amount $A_{-1}=\overline{k}_0$.
\medskip \noindent
The bonds $A_{-1}$ are one-period in duration and bear the constant world gross interest rate $R=\beta^{-1}$.
After the sale of $\overline{k}_0=A_{-1}$, the planner has \underbar{zero} capital and so shuts down the technology. Instead, the planner uses the asset market to smooth consumption. The planner chooses $\{A_{t+1},c_t\}$ to maximize
$$ \sum_{t=0}^\infty \beta^t u(c_t),  \quad   0< \beta<1,\quad   \beta = {\frac{1}{1+\rho}},  \quad \rho>0$$
subject to
$c_t+\beta A_{t+1}=A_t,  A_0 = \overline{k}_0 .$
Please find the optimal path for consumption $\{c_t\}$.
%                   Form the Lagrangian
%$$L = \sum_{t=0}^\infty \beta^t \{u(c_t) + \lambda_t[\beta^{-1}(A_t-c_t)-A_{t+1}]\}$$
%
% $$\eqalign{ FONC &\colon \cr
%c_t &\colon\; u'(c_t)-\lambda_t\beta^{-1}=0 \cr
%A_t &\colon\; -\beta^-1\lambda_{t-1} + \beta^{-1}\lambda_t=0 \cr
%&\lim_{T\to\infty}\beta^{T}\lambda_T A_{T+1}=0 \cr
%\Rightarrow\;\; &\lambda_t  =\lambda_0 \;\; \forall t \cr
%&u'(c_t) =u'(c_0)\;\; \forall t \cr
%&c_t =c_0\;\; \forall t \cr
%$The$&$n rollover assets and just consume interest:$ \cr
%&A_{t+1} =\beta^{-1}[A_t -c_t] \Rightarrow \cr
%&\bar A =\beta^{-1}[\bar A -c_0] \cr
%&(1-\beta^{-1})\bar A = -c_0 \cr
%& \quad $or$ \cr
%& c_0  =(\beta^{-1}-1)\bar A \cr
%& c_0  =(1+\rho-1)\bar A \cr
%& c_0 =\rho \bar A \quad, \;\; \bar A= A_0=\overline{k}_0 \cr\
%}$$
%\medskip
%\noindent
%Thus, the optimal path for consumption is c_t = \rho \bar A=\rho \overline{k}_0
\medskip
\noindent{\bf f.} Compare the path of $(c_t,k_{t+1},\bar R_t)$ that you computed in parts {\bf a}-{\bf d} with ``no trade'' with the part {\bf e} path ``with trade'' in which the government ``shuts down the home technology'' and lives entirely from the returns on foreign assets. Can you say which path the representative consumer will prefer?
%\medskip \underbar{Hint:} I think it could go either way for reasons that might be easier to appreciate after you look at part g.
\medskip
\noindent{\bf g.} Now return to the economy in part {\bf e} with $\overline{k}_0>\bar k$ from part {\bf d}.  Assume that the planner is free to borrow or lend capital at the fixed gross interest of $\beta^{-1}=1+\rho$, as before. But now assume the planner chooses the \underbar{optimal} amount of $\overline{k}_0$ to sell off and so does not necessarily sell off the entire $\overline{k}_0$ and possibly continues to operate the technology.

\medskip
\indent {\bf i)} Find the solution of the planning problem.\medskip
%\underbar{Hint:} You might want to review your solution to problem 2 of the practice problem 4.
%\medskip
\indent {\bf ii)} Explain why it is optimal not to shut down the technology. {\bf Hint}: Starting from having shut the technology down, think of putting a small amount $\epsilon$ of capital into the technology-this earns $z\epsilon^\alpha$ and costs $\rho \epsilon$ in terms of foregone interest. Because $0<\alpha<1$, $\rho\epsilon<z\epsilon^\alpha$ for small $\epsilon$. Thus, the technology is \underbar{very productive} for small $\epsilon$, so it is efficient to use it.
\vfil\eject

\medskip
\noindent{\it Exercise \the\chapternum.9}
\medskip \noindent  Consider a consumer who wants to choose $\{c_t\}_{t=0}^T$ to maximize
$$\sum_{t=0}^T \beta^tc_t  \quad  , \;\;  0< \beta<1$$ subject to the intertemporal budget constraint
$$\sum_{t=0}^T R^{-t}[c_t-y_t]=0  \quad  , \;\;  R>1$$ where $c_t\geq0$ and $y_t>0$ for $t=0,\cdots,T$. Here $R>1$ is the gross interest rate $(1+r),\;\; r>0$. Assume $R$ is constant. Here $\{y_t\}_{t=0}^T $ is an exogenous sequence of ``labor income''.
\medskip
\noindent {\bf a.} Assume that $\beta R < 1$. Find the optimal path $\{c_t\}_{t=0}^T$.
\medskip
\noindent {\bf b.} Assume that $\beta R > 1$. Find the optimal path $\{c_t\}_{t=0}^T$.
\medskip
\noindent {\bf c.} Assume that $\beta R = 1$. Find the optimal path $\{c_t\}_{t=0}^T$.





\medskip
\noindent{\it Exercise \the\chapternum.10} \quad {\bf Term Structure of Interest Rates}\medskip
\medskip
\noindent
This problem assumes the following environment. A representative consumer has preferences ordered by
$$ \sum_{t=0}^\infty \beta^t u(c_t), \;\;  0< \beta<1, \;\;  \beta = {\frac{1}{1+\rho}}  , \;\; \rho>0$$
where $c_t$ is the consumption per worker and
 where
$$ u(c) = \cases{\frac{c^{1-\gamma}}{1-\gamma}  & if $\gamma>0$ and $\gamma \ne 1$\cr
                   \log(c) & if  $\gamma=1$\cr
}$$
%
%$$ \psi_t = \cases{1 & for $t < 4$ \cr
%                   2 & for $ t\geq 4$. \cr} $$
The technology is
$$y_t=f(k_t)=zk_t^\alpha,\;\; 0<\alpha<1, \;\; z>0$$
where $y_t$ is the output per unit labor and $k_t$ is capital per unit labor
$$\eqalign{
            y_t & =c_t+x_t+g_t \cr
                        k_{t+1} & =(1-\delta)k_t+x_t \quad  , \;\;  0 < \delta<1 \cr
        } $$
where $x_t$ is  gross investment per unit of labor and $g_t$ is government expenditures per unit of labor.
The government finances its expenditures by levying some combination of a flat rate tax $\tau_{ct}$ on the value of consumption goods purchased at $t$, a flat rate tax $\tau_{nt}$ on the value of labor earnings at $t$, a flat rate tax $\tau_{kt}$ on earnings from capital at $t$ and a lump sum tax of $\tau_{ht}$ in time $t$ consumption goods per worker at time $t$.
%\noindent Let $\{q_t^0,r_t^0,w_t^0\}_{t=0}^\infty$ be a price system. Here the superscript 0 means the time 0 price system.
 Define
 $  \bar R_{t+1}  = {(1+\tau_{ct}) \over(1+\tau_{ct+1})}
     \left[  1   + (1-\tau_{kt+1}) (    f'(k_{t+1} - \delta)\right] . $

\medskip

 \noindent
{\bf a.} Recall that we can represent
$$ q_t^0  = q_0^0m_{0,1}m_{1,2}\cdots m_{t-1,t} $$
where $m_{t-1,t}  = {\frac{q_t^0}{q_{t-1}^0}} $ and $ m_{t-1,t} \equiv \exp(-r_{t-1,t})\approx {\frac{1}{1+r_{t-1,t}}}$.
Further, recall that the $t$ period long yield  satisfies
$q_t^0  =\exp(-tr_{0,t}) $
and $
r_{0,t}  = {\frac{1}{t}}[r_{0,1}+r_{1,2}+\cdots+r_{t-1,t}] $.
Now suppose that at $t=0$, $k_0=\bar k$, where $\bar k$ is the steady state appropriate for an economy with constant $g_t=\bar g >0$ and all expenditures are financed by lump sum taxes. Find $q_t^0$ for this economy.

\medskip \noindent
{\bf b.} Plot $r_{t-1,t}$ for this economy for $t=1,2,\cdots,10$.
\medskip \noindent
{\bf c.} Plot $r_{0,t}$ for this economy for $t=1,2,\cdots,10$ (this is what Bloomberg plots).
\medskip
\noindent {\bf d.} Now assume that at time 0, starting from $k_0=\bar k$ for the steady state you computed in part a, the government unexpectedly and permanently raises the tax rate on income from capital $\tau_{kt}=\tau_k >0$ to a positive rate.
\medskip \indent
{\bf i)} Plot $r_{t-1t}$ for this economy for $t=1,2,\cdots,10$. Explain how you got this outcome.

\medskip \indent
{\bf ii)} Plot $r_{0,t}$ for this economy for $t=1,2,\cdots,10$. Explain how you got this outcome.



\medskip
\noindent{\it Exercise \the\chapternum.11}
\medskip
\noindent
 This problem assumes the same economic environment as the previous exercise i.e, the ``growth model'' with fiscal policy.
Suppose that you observe the path for consumption per capita in figure \Fg{fig_7_1}. Say what you can about the likely behavior over time of $k_t$, $\bar R_t =[1 + (1-\tau_{kt})(f'(k_t)-\delta)]$, $g_t$ and $\tau_{kt}$.(You are free to make up any story that is consistent with the model.)

\medskip


\midfigure{fig_7_1}
\centerline{\epsfxsize=2.5truein\epsffile{fig_7_1.eps}}
\caption{Consumption per capita.}
\infiglist{fig_7_1}
\endfigure




\medskip
\noindent{\it Exercise \the\chapternum.12}
\medskip
\noindent Assume the same economic environment as in the previous two problems. Assume that someone has observed the  time path for $c_t$ in figure \Fg{fig_7_2}:
\medskip
\midfigure{fig_7_2}
\centerline{\epsfxsize=2.5truein\epsffile{fig_7_2.eps}}
\caption{Consumption per capita.}
\infiglist{fig_7_2}
\endfigure
\medskip
\noindent {\bf a.} Describe a consistent set of assumptions about the fiscal policy that explains this time path for
$c_t$. In doing so, please distinguish carefully between changes in taxes and expenditures that are \underbar{foreseen} versus \underbar{unforeseen}.

\medskip
\noindent {\bf b.} Describe what is happening to $k_t,\bar R_t$ and $g_t$ over time. Make whatever assumptions you must to get a complete but consistent story - here ``consistent'' means ``consistent with the economic environment we have assumed''.

\medskip
\noindent{\it Exercise \the\chapternum.13}
\medskip \noindent
Consider the optimal growth model with a representative consumer with preferences
$$\sum_{t=0}^\infty \beta^tc_t , \;\;  0< \beta<1$$  with technology
$$c_t+k_{t+1}=f(k_t)+(1-\delta)k_t,\;\; \delta \in (0,1)$$
$$c_t\geq 0, \quad k_0>0 \;\hbox{given} $$
$$f'>0,\;\;f''<0,\;\;\lim_{k \to 0}f'(k)=+ \infty,\;\;\lim_{k \to +\infty}f'(k)= 0$$
 Let $\bar k$ be
the steady state capital stock for the optimal planning problem.
\medskip
\noindent {\bf a.} For $k_0$ given, formulate and solve the optimal planning problem.
\medskip
\noindent {\bf b.}  For $k_0>\bar k$,  describe the optimal time path of $\{c_t,k_{t+1}\}_{t=0}^\infty$.
\medskip
\noindent {\bf c.} For $k_0<\bar k$, describe the optimal path of $\{c_t,k_{t+1}\}_{t=0}^\infty$.
\medskip
\noindent {\bf d.} Let the \underbar{saving rate} $s_t$ be defined as the $s_t$ that satisfies $$k_{t+1}=s_tf(k_t)+(1-\delta)k_t.$$
%(In the Solow model, we assumed that $s_t=s$, as exogenous constant).
 Here $s_t$ in general varies along a $\{c_t,k_{t+1}\}$ sequence. Say what you can about how $s_t$ varies as a function of $k_t$.

%$\underline{Hint}$: Form the Lagrangian
%$$L=\sum_{t=0}^\infty\beta^t\bigg \{c_t+\lambda_t[f'(k_t)+(1-\delta)k_t-c_t-k_{t+1}]\bigg \}$$
%
%$$\leqalignno{ FONC &\colon \cr
%c_t &\colon\; 1-\lambda_t\leq 0\;\;,\;=0 \;$if$ \;c_t>0 \quad ,\;t\geq0 &(1)\cr
%k_t &\colon\; \lambda_t[f'(k_t)+(1-\delta)]-\beta^{-1}\lambda_{t-1} =0 \quad ,\;t\geq1 &(2)\cr
%}$$
%
%$$\eqalign{(1) \Rightarrow \lambda_t \geq 1 \;\;,\;\;
%\lambda_t=1 &\Rightarrow c_t>0 \cr
%\lambda_t>1 &\Rightarrow c_t=0 \cr
%}$$
%
%$$\eqalign{(2) & \Rightarrow \quad \lambda_t = (\frac{\beta^{-1}}{f'(k_t)+(1-\delta)}) \lambda_{t-1} \;\;,\beta=\frac{1}{1+\rho \cr
%& \Rightarrow \quad \lambda_t = (\frac{1+\rho}{f'(k_t)+(1-\delta)}) \lambda_{t-1}
%}
%}$$
%
%Thus,
%$$\frac{\lambda_t}{\lambda_{t-1}}  \cases{<1,\; \hbox{when} \; & $f'(k_t) >\rho+\delta$ \cr
%  =1,\; \hbox{when} \; & $f'(k_t) =\rho+\delta$ \cr
%  >1,\; \hbox{when} \; & $f'(k_t) <\rho+\delta$ \cr}$$
%
%The outcomes indicate the following optimal path
%
%\medskip
%
%\midfigure{fig_8}
%\centerline{\epsfxsize=3truein\epsffile{fig_8.eps}}
%\infiglist{fig_8}
%\endfigure
%\medskip
%Optimal policy is : \medskip
%\indent if $k_0>\bar k$,\medskip
%Set $c_0$ to solve $\bar k +c_0=f(k_0)+(1-\delta)k_0$. So that you jump to $\bar k$ in one step.
%\medskip
%\indent if $k_0<\bar k$,\medskip Save \underbar{everything} until you get to $\bar k \Rightarrow$ get to $\bar k$ as  \underbar{fast} as you can.
%\medskip
%So, the consumption path is :
%
%\medskip
%
%\midfigure{fig_82}
%\centerline{\epsfxsize=3truein\epsffile{fig_82.eps}}
%\infiglist{fig_82}
%\endfigure\medskip \noindent
%Note: with preferences $\sum_{t=0}^\infty\beta^tc_t$- consumer is willing to substitute consumption across time equally - leads to a very \underbar{unsmooth} consumption path under the optimal plan. \medskip
%\noindent Compare the optimal growth model above with the Solow model. The Solow model has the same technology. $$c_t+k_{t+1}=(1-\delta)k_t+f'(k_t)$$ or $$k_{t+1}=(1-\delta)k_t+(f'(k_t)-c_t)$$
%The solow model assumes $$\hbox{savings}\equiv f(k_t)-c_t=sf(k_t)$$ where s is a \underbar{constant} savings rate.
%\medskip
%
%\midfigure{fig_8solow}
%\centerline{\epsfxsize=3truein\epsffile{fig_8solow.eps}}
%\infiglist{fig_8solow}
%\endfigure\medskip \noindent
%
%Steady State :
%$$(*)\quad sf(\bar k)=\delta \bar k$$
%(*) is the equation that determines steady state $\bar k$ of Solow model as a function of assumed constant savings rate s.
%\medskip \noindent
%In the above optimal growth model with preferences $$\sum_{t=0}^\infty\beta^tc_t \qquad(\hbox{i.e.} u(c_t)=c_t)$$
%the saving rate is \underbar{not} constant. Instead of (*) the steady state is determined by the usual equation $$f'(\bar k)=\rho+\delta$$ the saving rate (s) adjusts to make this happen. Ofcourse, at the steady state of the optimal growth model, $$k_{t+1}=(1-\delta)k_t+(f(k_t)-c_0) \Rightarrow $$
%$$\delta \bar k= f(\bar k)-\bar c \equiv \bar sf(\bar k)$$ \centerline{[where $\bar s$ is the savings rate at the steady state]}
%or $$\bar s=\frac{\delta\bar k}{f(\bar k)}$$
%but here $\bar s$ is an outcome - to be solved for as a function of $\bar k$ and not a determinant of $\bar k$ as it is in the Solow model.
%\medskip
%What about the savings rate $\frac{f(k)-c}{f(k)}$ away from the steady state ? We can address the answer from our earlier picture :
%\medskip
%\midfigure{fig_84}
%\centerline{\epsfxsize=3truein\epsffile{fig_84.eps}}
%\infiglist{fig_84}
%\endfigure\medskip \noindent
%
%\medskip
%Note how the rate of return on capital covaries with the savings rate here. Compare this with the constant savings rate assumed in the Solow model.
%






%%%   HHHH first Anmol insert ends

%%%%  HHHHH second Anmol insert begins

\medskip
\noindent{\it Exercise \the\chapternum.14}
\medskip
\noindent
 A representative consumer has preferences over consumption streams ordered by
$\sum_{t=0}^\infty \beta^t  u(c),  \quad 0 < \beta < 1,$
where $\beta \equiv {\frac{1}{1+\rho}}$, $c_t$ is consumption per worker, $\gamma > 0$, and
\noindent
$$ u(c) = \cases{{\frac{c^{1-\gamma}}{1-\gamma}} & if  $\gamma > 0$ and $\gamma \ne 1$ \cr
                   \log(c) & if  $\gamma=1$. \cr
}$$
%
%$$ \psi_t = \cases{1 & for $t < 4$ \cr
%                   2 & for $ t\geq 4$. \cr} $$
 The consumer supplies one unit of labor inelastically.  The technology is
$$
y_t = f(k_t) = z k_t^\alpha, \quad  0 < \alpha < 1$$
where $y_t$ is output per worker, $k_t$ is capital per worker, $x_t$ is gross investment per worker,
$g_t$ = government expenditures per worker, and
$$\eqalign{
y_t & = c_t + x_t + g_t \cr
k_{t+1} & = (1-\delta ) k_t + x_t , \quad 0 < \delta < 1 .\cr
}$$
The government finances its expenditure stream  $\{g_t\}$ by levying a stream of flat rate taxes $\{\tau_{ct}\}$ on the value of the consumption good purchased at
$t$, a stream of flat rate taxes $\{\tau_{kt}\}$ on earnings from capital at $t$, and a stream of lump sum taxes $\{\tau_{ht}\}$.
Let $\{q_t, q_t \eta_t, q_t w_t\}_{t=0}^\infty$ be a {\it price system}, where $q_t $ is the price of time $t$ consumption and investment goods, $q_t \eta_t$ is the price of renting capital at time $t$,
and $q_t w_t$ is the price of renting labor at time $t$.  All trades occur at time $0$ and all prices are measured in units of the time $0$ consumption good.
The initial capital stock $k_0$ is given.\medskip

\noindent {\bf a.} Define a competitive equilibrium with taxes and government purchases. \medskip
\noindent{\bf b.}  Assume that $g_t =0 $ for all $t \geq 0$ and that all taxes are also zero.  Find the value $\bar k$ of the steady state capital per worker. Find a formula
for the saving rate $\frac{x_t}{f(k_t)}$ at the steady state value of the capital stock.\medskip

\noindent{\bf c.} Suppose that the initial capital stock $k_0 = .5 \bar k$, so that the economy starts below its steady state level.
Describe (i.e., draw graphs showing) the time paths of $\{c_t, k_{t+1}, \bar R_{t+1}\}_{t=0}^\infty $ where $\bar R_{t+1} \equiv [(1-\delta) + f'(k_{t+1})]$.\medskip
\noindent{\bf d.} Starting from the same initial $k_0$ as in part {\bf c}, assume now that $g_t = \bar g = \phi f(k_0) >0 $ for all $t \geq 0$ where $\phi \in (0, 1-\delta)$. Assume that the
government finances its purchases by imposing lump sum taxes.
Describe (i.e., draw graphs) showing the time paths of $\{c_t, k_{t+1}, \bar R_{t+1}\}_{t=0}^\infty $ % where $\bar R_{t+1} \equiv [(1-\delta) + f'(k_{t+1})]$
and compare them to the outcomes
that you obtained in part {\bf c}.  What outcomes differ? What outcomes, if any, are identical across the two economies? Please explain. \medskip

\noindent{\bf e.}  Starting from the same initial $k_0$ assumed in part {\bf c}, assume now that $g_t = \bar g = \phi f(k_0) >0 $ for all $t \geq 0$ where $\phi \in (0, 1-\delta)$. Assume that the
government must now finance these  purchases by imposing a time-invariant tax rate $\bar \tau_k$ on capital each period. The government {\it cannot} impose lump sum taxes or any other kind of taxes
to balance its budget.   Please describe how to find a competitive equilibrium. %({\em Hint:} Here you will have to go beyond the shooting method that we used in class when we assumed that the present value
%of lump sum taxes or subsidies could be adjusted to finance the amount not raised by the exogenous levels of the other tax rates that we had set. Now those lump sum taxes or subsidies are not available.)\medskip

%
%\medskip
%\midfigure{FN_1}
%\centerline{\epsfxsize=2truein\epsffile{FN_1.eps}}
%\caption{Yield to maturity $r_{0,t}$ at time $0$ as a function of term to maturity $t$.}
%\infiglist{FN_1}
%\endfigure
%\medskip

\medskip
\figure{FN_1}
\centerline{\epsfxsize=2truein\epsffile{FN_1_tom_2.eps}}
\caption{Yield to maturity $r_{0,t}$ at time $0$ as a function of term to maturity $t$.}
\infiglist{FN_1}
\endfigure
\medskip
%
%
\midfigure{FN_2}
%\centerline{\epsfxsize=2.25truein\epsffile{FN_2_tom.eps}}
\centerline{\epsfxsize=2.25truein\epsffile{fig_11.4.eps}}
\caption{Capital stock as function of time in two economies with different values of $\gamma$.}
\infiglist{FN_2}
\endfigure
\medskip
\noindent{\it Exercise \the\chapternum.15}
\medskip
\noindent
The structure of the  economy is identical to that described in the previous exercise.
 %In particular, a representative consumer has preferences over consumption streams ordered by
%\[ \sum_{t=0}^\infty \beta^t \frac{c_t^{1-\gamma}}{1-\gamma}, \quad, 0 < \beta < 1, \gamma > 0 \]
%where $\beta \equiv \frac{1}{1+\rho}$ and $c_t$ is consumption per worker. The consumer supplies one unit of labor inelastically.  The technology is
%\[
%y_t = f(k_t) = z k_t^\alpha, \quad  0 < \alpha < 1 \]
%where $y_t$ is output per worker, $k_t$ is capital per worker, $x_t$ is gross investment per unit of labor,
%$g_t$ = government expenditures per unit of labor, and
%\begin{eqnarray*}
%y_t & = & c_t + x_t + g_t \\
%k_{t+1} & = & (1-\delta ) k_t + x_t , \quad 0 < \delta < 1
%\end{eqnarray*}
%The government finances its expenditure stream  $\{g_t\}$ by levying a stream of flat rate taxes $\{\tau_{ct}\}$ on the value of the consumption good purchased at
%$t$, a stream of flat rate taxes $\{\tau_{kt}\}$ on earnings from capital at $t$, and a stream of lump sum taxes $\{\tau_{ht}\}$. There is a competitive equilibrium with the price system being
%$\{q_t, r_t, w_t\}_{t=0}^\infty$, where $q_t $ is the price of time $t$ consumption and investment goods, $r_t$ is the price of renting capital at time $t$,
%and $w_t$ is the price of renting labor at time $t$.  All trades occur at time $0$ and all prices are measured in units of the time $0$ consumption good.
%The initial capital stock $k_0$ is given.\\
Let $r_{0,t}$ be the yield to maturity on a $t$ period bond at time $0$, $t=1, 2, \ldots, $.  At time $0$, Bloomberg reports the term structure of interest rates
in figure \Fg{FN_1}. Please say what you can about the evolution of $\{c_t, k_{t+1}\}$ in this economy.  Feel free to make any assumptions you need  about fiscal policy
$\{g_t, \tau_{kt}, \tau_{ct}, \tau_{ht}\}_{t=0}^\infty$ to make your answer coherent.

%




\medskip
\noindent{\it Exercise \the\chapternum.16}
\medskip
\noindent
 The structure of the  economy is identical to that described in exercise \the\chapternum.14.
Assume that $\{g_t, \tau_{ct}, \tau_{kt}\}_{t=0}^\infty $ are all constant sequences (their values don't change over time).
In this problem, we ask you to infer  differences
across two economies in which all aspects of the economy are identical {\it except} the parameter $\gamma$ in the utility function.\NFootnote{It is possible that lump sum taxes differ across the two economies. Assume that lump sum taxes are adjusted to balance the government budget.} In both economies, $\gamma > 0$.  In one economy, $\gamma >0$ is high and in the other it is low.
Among other identical features, the two economies have identical government policies and identical initial capital stocks.\medskip
\medskip

\noindent{\bf a.}   Please look at figure \Fg{FN_2}. Please tell which outcome for $\{k_{t+1}\}_{t=0}^\infty$ describes the low $\gamma$ economy, and which describes the high $\gamma$ economy.
Please explain your reasoning.\medskip

\noindent{\bf b.}   Please look at figure \Fg{FN_3}. Please tell which outcome for $\{f'(k_t)\}_{t=0}^\infty$ describes the low $\gamma$ economy, and which describes the high $\gamma$ economy.
Please explain your reasoning.
\medskip

\noindent{\bf c.} Please plot time paths of consumption for the low $\gamma$ and the high $\gamma$ economies.
\medskip
%
%\midfigure{FN_2}
%\centerline{\epsfxsize=2.25truein\epsffile{FN_2_tom.eps}}
%\caption{Capital stock as function of time in two economies with different values of $\gamma$.}
%\infiglist{FN_2}
%\endfigure
%\medskip
%\midfigure{FN_3}
%\centerline{\epsfxsize=2.25truein\epsffile{FN_3_tom.eps}}
%\caption{Marginal product of capital as function of time in two economies with different values of $\gamma$.}
%\infiglist{FN_3}
%\endfigure
%\medskip
%





\medskip



\midfigure{FN_3}
%\centerline{\epsfxsize=2.25truein\epsffile{FN_3_tom.eps}}
\centerline{\epsfxsize=2.25truein\epsffile{fig_11_5.eps}}
\caption{Marginal product of capital as function of time in two economies with different values of $\gamma$.}
\infiglist{FN_3}
\endfigure
\medskip

\noindent{\it Exercise \the\chapternum.17}
\medskip
\noindent
 The structure of the  economy is identical to that described in exercise \the\chapternum.16.
 %In particular, a representative consumer has preferences over consumption streams ordered by
%\[ \sum_{t=0}^\infty \beta^t \frac{c_t^{1-\gamma}}{1-\gamma}, \quad, 0 < \beta < 1, \gamma > 0 \]
%where $\beta \equiv \frac{1}{1+\rho}$ and $c_t$ is consumption per worker. The consumer supplies one unit of labor inelastically.  The technology is
%\[
%y_t = f(k_t) = z k_t^\alpha, \quad  0 < \alpha < 1 \]
%where $y_t$ is output per worker, $k_t$ is capital per worker, $x_t$ is gross investment per unit of labor,
%$g_t$ = government expenditures per unit of labor, and
%\begin{eqnarray*}
%y_t & = & c_t + x_t + g_t \\
%k_{t+1} & = & (1-\delta ) k_t + x_t , \quad 0 < \delta < 1
%\end{eqnarray*}
%The government finances its expenditures stream  $\{g_t\}$ by levying a stream of flat rate taxes $\{\tau_{ct}\}$ on the value of the consumption good purchased at
%$t$, a stream of flat rate taxes $\{\tau_{kt}\}$ on earnings from capital at $t$, and a stream of lump sum taxes $\{\tau_{ht}\}$. There is a competitive equilibrium with the price system being
%$\{q_t, r_t, w_t\}_{t=0}^\infty$, where $q_t $ is the price of time $t$ consumption and investment goods, $r_t$ is the price of renting capital at time $t$,
%and $w_t$ is the price of renting labor at time $t$.  All trades occur at time $0$ and all prices are measured in units of the time $0$ consumption good.
%The initial capital stock $k_0$ is given.
Assume that $\{\tau_{ct}, \tau_{kt}\}_{t=0}^\infty $ are  constant sequences (their values don't change over time) but that $\{g_t\}_{t=0}^\infty$ follows the path described in panel {\bf c}
of figure \Fg{FN_4}  -- it takes a once and for all jump at time $t=10$. Lump sum taxes adjust to balance the government budget.   Panels {\bf a} and {\bf b} give consumption paths for two economies that are identical except in one respect.
In one of the economies, the time $10$ jump in $g$ had been anticipated since time $0$, while in the other, the jump in $g$ that occurs at time $10$ is completely unanticipated at time $10$.
Please tell which panel corresponds to which view of the arrival of news about the path of $g_t$. Please say as much as you can about how $\{k_{t+1}\}_{t=0}^\infty$ and the interest rate behave in these two economies.
%
%




\medskip
\noindent{\it Exercise \the\chapternum.18}
\medskip
\noindent
A planner chooses sequences $\{c_t, k_{t+1}\}_{t=0}^\infty$
to maximize
 $$\sum_{t=0}^\infty \beta^t u\bigl(c_t - \alpha c_{t-1}\bigr), \quad \alpha \in (-1,1), \quad \beta \in (0,1)$$
subject to
$$k_{t+1} + c_t = f(k_t) + (1-\delta) k_t, \quad  \delta \in (0,1) , \quad c_t \geq 0$$
where
\noindent

$$ u(x) = \cases{{\frac{x^{1-\gamma}}{1-\gamma}}  &if $\gamma >0$ and $\gamma \ne 1$\cr
                   \log(x) & if  $\gamma=1$\cr
}$$
%
%$$ \psi_t = \cases{1 & for $t < 4$ \cr
%                   2 & for $ t\geq 4$. \cr} $$
 and
$(k_0, c_{-1})$ are given initial conditions.  Here $u'>0, u''< 0$, and $\lim_{c\downarrow 0} u'(c) =0$. If $\alpha >0$, it indicates that the consumer has a `habit'; if
$\alpha < 0$, it indicates that the consumption good is somewhat durable.
\medskip

\noindent{\bf a.}  Find a complete set of first-order necessary conditions for the planner's problem.
\medskip

\noindent{\bf b.}  Define an optimal {\it steady state}.\medskip

\noindent{\bf c.}  Find optimal steady state values for $(k, f'(k), c)$.\medskip

\noindent{\bf d.}  For given initial conditions $(k_0, c_{-1})$, describe as completely as you can an algorithm for
computing a path $\{c_t, k_{t+1}\}_{t=0}^\infty$ that solves the planning problem.


\medskip
\figure{FN_4}
\centerline{\epsfxsize=2truein\epsffile{figv3_116_b.eps}}
%\centerline{\epsfxsize=2truein\epsffile{FN_4.eps}}
\caption{Panels {\bf a} and {\bf b}: consumption $c_t$ as function of time in two economies.  Panel {\bf c}: government expenditures $g_t$ as a function of time.}
\infiglist{FN_4}
\endfigure
\medskip
%%%%% HHHHH second Anmol insert ends


\medskip
\noindent{\it Exercise \the\chapternum.19} \quad {\bf The Invisible Hand, I} \medskip
\medskip
\noindent
Please consider once again the  ``Foreseen jump in $g$'' experiment studied in section %11.9
\use{sec:transinelastic}.  This exercise asks you to put yourself in the shoes of the representative household
and to think through the optimum problem that it faces within a competitive equilibrium  at time
$0$.


\medskip

\noindent{\bf a.}  Please describe the signals that the market sends to the household  {\it after\/}
the new $\{g_t\}_{t=0}^\infty$ policy materializes at time $0$. Please list  all of the  objects that ``the market'' (also known as
``the invisible hand'') presents as {\it exogenous\/} to the representative household.

\medskip
\noindent{\bf b.}  Please describe how the ``Big $K$'' part of a ``Big $K$, little $k$'' argument is used to determine  {\it all\/}
of those objects exogenous to the household.  {\bf Hint:} This is accomplished by applying the shooting algorithm in section \use{sec:transinelastic}.
\index{Big $K$, little $k$}%

\medskip
\noindent {\bf c.} Please explain thoroughly how the representative household chooses to respond to the signals presented to it
 by the market at time $0$. % I.e., explain how it responds to the exogenous objects presented to it at time $0$.

\medskip
\noindent{\bf d.} Given the objects that the market presents to the representative household, please tell how you would could a shooting algorithm
to compute the path of $\{c_t, k_{t+1}\}_{t=0}^\infty$ chosen by the household.

\medskip
\noindent{\bf e.}  Please describe how to complete a  ``Big $K$, little $k$'' argument using your answers to parts {\bf c} and {\bf d.}



\medskip
\noindent{\it Exercise \the\chapternum.20} \quad {\bf The Invisible Hand, II} \medskip
\medskip
\noindent Please consider again the  ``Foreseen jump in $\tau_n$''  experiment in section \use{sec:elasticlabor}.  % 11.12
Like the previous problem, this one puts you  into the shoes of the representative household
and asks you to think through the optimum problem that it faces within a competitive equilibrium with distorting taxes at time
$0$.

\medskip

\noindent{\bf a.}  Please describe the signals that the market sends to the household  {\it after\/}
the new $\{\tau_{nt}\}_{t=0}^\infty$ policy  materializes at time $0$. Please list  all  objects that ``the market''
presents as {\it exogenous\/} to the representative household.

\medskip
\noindent{\bf b.}  Keeping in mind that there is a ``Big $K$, little $k$'' argument in the background,
please provide a complete explanation for why the household chooses the paths of $\{c_t, n_t, k_t\}_{t=0}^\infty$ displayed in
figure \Fg{fig_tn}. %11.12.3

