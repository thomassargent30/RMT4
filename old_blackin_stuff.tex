
{\bf Move this material to the chapter-by-chapter discussions.}


Chapter \use{assetpricing2} extends and applies the asset pricing theory developed in chapter \use{recurge} on complete markets  and  chapter \use{incomplete} on incomplete markets
  to devise  empirical strategies for specifying and testing
asset pricing models.   Building on work by Darrell Duffie, Lars Peter Hansen,  and their co-authors,  chapter  \use{assetpricing2} discusses  ways of characterizing asset pricing puzzles
associated with the preference specifications and market structures commonly used in other parts of macroeconomics.
It then describes alterations of those structures that hold promise for  resolving some of those puzzles.
\auth{Hansen, Lars P.}%
\auth{Duffie, Darrell}%


Chapter \use{chang}  extends the chapter \use{credible} analysis of  credible government policies to a
more sophisticated  class of dynamic economies.  While the economic environments of  chapter \use{credible} are purposefully
cast in  ways to make {\it all\/} interesting dynamics be associated with reputation, chapter \use{chang} incorporates other sources of dynamics
 associated, for example, with  interactions between the demand and supply of money.


Chapter \use{macrolaborII} describes  new  macroeconomic models of aggregate labor supply.
Until recently, there was a sharp division between macroeconomists working in the real business cycle tradition and microeconomists
studying data on individual workers'  labor supply decisions and outcomes.  On the one hand, real business cycle researchers
had appealed to the Hansen (1985)-Rogerson (1988) employment lotteries model to justify a high aggregate labor supply elasticity.
On the other hand, labor  microeconomists had expressed skepticism about the empirical plausibility of those lotteries and the supplementary complete consumption insurance markets that
the Hansen-Rogerson theory uses to obtain that high labor supply elasticity.  The microeconomists typically based their much smaller labor supply elasticities
by working within some version of an incomplete markets life cycle model.  Chapter \use{macrolaborII} describes how prominent
real business cycle theorists have recently abandoned the employment lotteries model in favor of a life cycle model.
The adoption of a common theoretical framework means that the quest for the
aggregate labor supply elasticity is now in terms of the forces and institutions
that shape individuals' career length choices.
%The question of whether or not their doing that
%causes them to dispose of that high labor supply elasticity is the subject of this
%chapter. The chapter tells how the answer hinges on details of publicly funded
%retirement schemes.
\auth{Hansen, Gary D.}%
\auth{Rogerson, Richard}%


=======================================

We have added significant amounts of material to a number of chapters, including chapters
\use{timeseries}, \use{search1}, \use{recurpe}, \use{recurge}, \use{linappro}, \use{assetpricing1}, \use{optax}, \use{selfinsure}, \use{incomplete}, \use{stackel}, \use{credible},
\use{fiscalmonetary}, and \use{search2}. Chapter \use{timeseries} has a better treatment
of laws of large numbers, a prettier derivation of the Kalman filter,  and an extended economic example (a
permanent income model of consumption) that    illustrates some of  the time series
techniques introduced in the chapter. It also contains a new appendix describing the powerful  Markov chain Monte Carlo
method that has opened wide new avenues in empirical implementation of dynamic economic models.
Chapter \use{search1} on search models  has been improved throughout.  New material on a model of Burdett and Judd (1983) deepens the treatment of the basic McCall search model.  Chapter \use{recurpe} reaps more returns from the recursive competitive equilibrium concept in the context of partial equilibrium
  models in the tradition of Sherwin Rosen and his co-authors. The chapter shows how Rosen schooling models
  fit within the recursive competitive equilibrium framework. Chapter \use{recurge} has been revised and improved to say
 more about how to find a recursive structure within an
Arrow-Debreu pure exchange economy. Chapter \use{linappro} has been extensively revised. It now  extracts many more implications from
the basic model, including  links to the term structure of interest
rate, explanations of how sequence and recursive formulations imply things about movements in government debt.  It also includes
a section on a two country model that illustrates the transmission abroad of effects of changes in tax policies in one country.
A companion to this chapter entitled `Practicing Dynare' contains dynare code for computing all of the quantitative examples in the chapter.
 Chapter \use{assetpricing1} on asset pricing theory has been revised in a way designed to prepare the way for our new chapter \use{assetpricing2} on asset pricing empirics.
 Chapter \use{optax} on optimal taxation has been revised in several ways. For example, it  includes additional exercises that bring out aspects of the theory
 of optimal taxation. We also use a general equilibrium model with money to attempt to shed light on  contentious aspects of the fiscal theory of the price level.
 %Chapter \use{optax} also studies an
%optimal taxation problem in an economy in which there are
%incomplete markets. The supermartingale convergence theorem plays
%an important role in the analysis of this model.
 Chapter
\use{selfinsure} extends our discussion of the  supermartingale
convergence theorem and how it implies precautionary saving
 to settings with endogenous labor supply.
%like those of Zhu (2009) and  Marcet, Obiols-Homs, and Weil (2007)
%where there is a labor supply decision.
Chapter \use{incomplete} on incomplete markets has been revised in parts.  Chapter  \use{stackel} on Stackelberg problems in linear quadratic models has been extended and improved.
Chapter  \use{credible} on credible public policy has been set up in  ways that lead naturally to our new followup  chapter \use{chang} on credible economic policy
in models with other sources of dynamics.
Chapter \use{fiscalmonetary} has more to say about the fiscal theory of the price level
%building on insights provided by Marco Bassetto.
including game theoretic arguments by Bassetto (2002).
To elaborate on how congestion externalities are such a driving force
in matching models, chapter \use{search2}
includes an analysis of age dynamics in an overlapping generations model with
a single matching function.


\auth{Judd, Kenneth L.}%
\auth{Burdett, Kenneth}%
\auth{Rosen, Sherwin}%
%%\auth{Marcet, Albert}%
%%\auth{Obiols-Homs, Francesc}%
%%\auth{Weil,  Philippe}%
\auth{Bassetto, Marco}%
%%\auth{Zhu, Shenghao}%