
\input grafinp3
%\input grafinput8
\input psfig
%\eqnotracetrue

%\showchaptIDtrue
%\def\@chaptID{9.}

%\eqnotracetrue

\hbox{}

\def\toone{{t+1}}
\def\ttwo{{t+2}}
\def\tthree{{t+3}}
\def\Tone{{T+1}}
\def\TTT{{T-1}}
\def\rtr{{\rm tr}}

\chapter{Ricardian Equivalence\label{ricardian}}
\footnum=0

\section{Borrowing limits and Ricardian equivalence}

  This chapter studies whether the timing of taxes
matters.  Under some assumptions it does and under others it does
not.  The Ricardian doctrine describes assumptions under
which the timing of lump taxes does not matter.
In this chapter, we will study how the timing of taxes interacts
with restrictions on the ability of households to borrow.  We study
the issue in two equivalent settings:
(1) an infinite horizon economy with an infinitely lived
representative agent; and (2) an infinite horizon economy
with a sequence of one-period-lived agents, each of whom cares
about its immediate descendant. We assume that
the interest rate is exogenously given. For example,
the economy might be a small open economy that faces a given
interest rate determined in the
international capital market. Chapters \use{linappro} amd  \use{assetpricing1} will describe
 general equilibrium
analyses of the Ricardian doctrine where the interest
rate is determined within the model.

  The key findings of the chapter are that in the infinite horizon
model, Ricardian equivalence holds under what  we earlier
called the natural borrowing limit, \index{borrowing constraint!natural}%
but not under more stringent ones.  The natural borrowing limit
 lets households borrow up to the capitalized
value of their endowment sequences.  These results have limited counterparts
in the overlapping generations model, since that model is equivalent
to an infinite horizon model with a  no-borrowing constraint.\NFootnote{This is one of the insights in the
influential paper of Barro (1974) that reignited modern interest in Ricardian equivalence.}
  In the overlapping generations model,
a no-borrowing constraint translates into a requirement
that bequests be nonnegative.
Thus, in the overlapping generations  model, the domain
of the Ricardian proposition is restricted, at least relative
to the infinite horizon model under the natural borrowing limit.


\section{Infinitely lived agent economy}

Each of $N$ identical
households orders a consumption stream by
%of a single good by %with preferences
$$ \sum_{t=0}^\infty \beta^t u(c_t), \EQN ricardo1 $$
where $\beta \in (0,1)$ and $u(\cdot)$ is a strictly  increasing,
strictly concave, twice-differentiable
utility function.
We impose the Inada condition
$ \lim_{c \downarrow 0} u'(c) = +\infty.$
This Inada condition is important  because we will be stressing
the feature that $c \geq 0$.  There is no uncertainty.
The household can invest in a single risk-free asset bearing a fixed gross
one-period rate of return $R>1$, a loan either to foreigners or to the
government.  At time $t$, the household faces the budget
constraint
$$ c_t + R^{-1} b_{t+1} \leq y_t + b_t,  \EQN ricardo2$$
where $b_0$ is given.   Throughout this chapter,
we assume that $R \beta =1$.  Here
$\{y_t\}_{t=0}^\infty$ is a given nonstochastic
nonnegative endowment sequence and
$ \sum_{t=0}^\infty \beta^t y_t$
$< \infty$.

We   investigate two alternative restrictions on asset holdings
$\{b_t\}_{t=0}^\infty$.  One is that
$b_t \geq 0$ for all $t \geq 0$, which allows
the household to lend but not borrow.  The alternative
is to  permit the household to borrow, but only an amount
that it is feasible to repay.     To discover this amount,
set $c_t=0$ for all $t$ in formula \Ep{ricardo2} and solve forward for
$b_t$ to get
$$ \tilde b_t = - \sum_{j=0}^\infty R^{-j} y_{t+j},   \EQN ricardo3 $$
where we have ruled out Ponzi schemes by imposing the transversality condition
$$
\lim_{T\to\infty} R^{-T} b_{t+T} \,=\, 0.            \EQN ricardo_TVC
$$
Following Aiyagari (1994),
 we call $\tilde b_t$ the
{\it \idx{natural debt limit}}.\NFootnote{Even with $c_t=0$, the consumer
cannot repay more than $\tilde b_t$.} Thus, our alternative restriction
on assets is
$$ b_t \geq \tilde b_t , \EQN ricardo4 $$
which, because $\tilde b_t$ is typically negative,  is evidently weaker than $b_t \geq 0$.\NFootnote{We encountered
a more general version of equation \Ep{ricardo4} in chapter \use{recurge} when
we discussed Arrow securities.}\auth{Aiyagari, Rao}

\subsection{Optimal  consumption/savings decision when $b_{t+1} \geq 0$}
Consider the  household's problem of choosing $\{c_t, b_{t+1}\}_{t=0}^\infty$
to maximize expression \Ep{ricardo1}
subject to a given initial condition for initial assets $b_0$, the budget constraints \Ep{ricardo2},  and $b_{t+1} \geq 0$ for all $t$.  The first-order
 conditions for this problem are
$$ \EQNalign{ u'(c_t) & \geq \beta R u'(c_{t+1}),\quad \forall t \geq 0;
                                                                 \EQN ricardo5;a \cr
\noalign{\hbox{\rm and}}
              u'(c_t) & > \beta R u'(c_{t+1}) \quad {\rm implies} \quad
       b_{t+1} = 0 . \EQN ricardo5;b \cr} $$
Because $\beta R =1$, these conditions  and the constraint
\Ep{ricardo2} imply that $c_{t+1} = c_t$
when $b_{t+1} > 0$,  but when the consumer is
borrowing constrained, $b_{t+1}=0$    and $y_t + b_t = c_t < c_{t+1}$.
The optimal consumption plan  evidently
depends on the $\{y_t\}$ path, as the following examples illustrate.

\medskip
\noindent{\it Example 1:}
\quad   Assume $b_0 = 0$ and the endowment path
$\{y_t\}_{t=0}^\infty = \{y_h, y_l, y_h, y_l, \ldots\}$,
where $y_h > y_l > 0$.
The present value of the household's endowment is
$$ \sum_{t=0}^\infty \beta^t y_t =
   \sum_{t=0}^\infty \beta^{2t} (y_h + \beta y_l)
   = {y_h + \beta y_l \over 1-\beta^2}.
$$
The annuity value $\bar c$ that has the same present value as the
endowment stream satisfies
$$
{\bar c \over 1-\beta}= {y_h + \beta y_l \over 1-\beta^2},
\qquad \hbox{\rm or} \quad
 \bar c = {y_h + \beta y_l \over 1+\beta}.
$$
The solution to the household's optimization problem is the
constant consumption stream
$c_t = \bar c$ for all $t \geq 0$. Using the budget constraint
\Ep{ricardo2}, we can back out the associated savings plan:
$b_{t+1} =(y_h-y_l)/(1+\beta)$ for even $t$,
and $b_{t+1} =0$ for odd $t$.
The consumer is never borrowing constrained.\NFootnote{Note $b_t=0$
does not imply that the consumer is borrowing constrained. We say that he is borrowing constrained if the Lagrange multiplier
on the constraint $b_t \geq 0$ is not zero.}
\medskip

\noindent{\it Example 2:} \quad  Assume $b_0 =0$ and the endowment path
$\{y_t\}_{t=0}^\infty = \{y_l, y_h, y_l, y_h, \ldots\}$,
where $y_h > y_l > 0$.
The optimal plan  is $c_0=y_l$ and $b_1=0$, and from period 1 onward,
the solution is the same as in example 1. Hence, the consumer is borrowing
constrained the first period.\NFootnote{Examples 1 and 2 illustrate a
general result stated in chapter \use{selfinsure}. Given a borrowing constraint
and a nonstochastic endowment stream, the impact of the
borrowing constraint will not vanish until the household reaches
the period with the highest annuity value for the remainder
of the endowment stream.}



\medskip
\noindent{\it Example 3:} \quad  Assume $b_0=0$ and
$ y_t = \lambda^t$ where $1 < \lambda < R$.  Notice that
$\lambda \beta <1$.  The solution with the borrowing constraint
$b_t \geq 0$ is $c_t = \lambda^t , b_t=0$ for all $t \geq 0$.
The consumer is always borrowing constrained.
\medskip

\subsection{Optimal  consumption/savings decision when $b_{t+1} \geq \tilde b_{t+1}$}
\noindent{\it Example 4:} \quad  Assume the same  $b_0$ and same endowment
sequence $ y_t = \lambda^t$ as in example 3, but now impose only
the natural borrowing constraint \Ep{ricardo4}.   The present
value of the household's endowment is
$$ \sum_{t=0}^\infty \beta^t \lambda^t = {1 \over 1 -\lambda \beta}.$$
The household's budget constraint for each $t$ is
satisfied at a constant consumption level $\hat c$ satisfying
$$
{\hat c \over 1 -\beta} = {1 \over 1-\lambda \beta}, \qquad \hbox{\rm or}
\quad  \hat c = {1-\beta \over 1 - \lambda \beta}.
$$   Substituting
this consumption rate into formula \Ep{ricardo2} and solving forward gives
$$ b_t =
   {1  - \lambda^t \over 1 - \beta \lambda }. \EQN ricardo6$$
   The consumer issues more and more debt as time passes
and uses his rising endowment to service it.   The
consumer's debt always satisfies the
natural debt limit at $t$; in particular, $ b_t > \tilde b_t = -\lambda^t /(1 - \beta \lambda)$.
\medskip
\noindent{\it Example 5:} \quad   Take the specification of example 3,
but now impose  $\lambda <1$.    Note that the solution
\Ep{ricardo6} implies $b_t  \geq 0$, so that the constant consumption
path $c_t = \hat c$ in example 4
 is now the solution even if the borrowing constraint $b_t \geq 0$
is imposed.

\section{Government}
Add a government  to the model in a way that leaves the consumer's preferences over consumption plans continue to be ordered by
\Ep{ricardo1}.   The government
purchases a stream $\{g_t\}_{t=0}^\infty$  per household.  This stream does not appear in the consumer's utility functional.
The government levies
 a stream of lump-sum taxes $\{\tau_t\}_{t=0}^\infty$
on the   household, subject to the sequence of budget constraints
$$ B_t + g_t = \tau_t + R^{-1} B_{t+1}, \EQN ricardo7 $$
where $B_t$ is one-period  debt  due at $t$, denominated
in the time $t$ consumption good, that the government owes the households
or foreign investors.  Notice that we allow the government
to borrow, even though in  examples 1, 2,  3, and 5
 we did not permit the household to borrow. (If $B_t <0$, the
government lends to households or to foreign investors.)  Solving the
government's budget constraint forward gives the intertemporal constraint
$$ B_t = \sum_{j=0}^\infty R^{-j}(\tau_{t+j} - g_{t+j}) \EQN ricardo8$$
for $t \geq 0$, where we have ruled out Ponzi schemes by imposing the
transversality condition
$$
\lim_{T\to\infty} R^{-T} B_{t+T} \,=\, 0.
$$

\subsection{Effect on household}
We must now deduct $\tau_t$ from the household's endowment in equation
\Ep{ricardo2},
$$ c_t + R^{-1} b_{t+1} \leq y_t - \tau_t + b_t.  \EQN ricardo2tax$$
 Solving this tax-adjusted budget constraint forward
and invoking transversality condition \Ep{ricardo_TVC} yield
$$ b_t = \sum_{j=0}^\infty R^{-j} (c_{t+j} + \tau_{t+j} - y_{t+j}).
 \EQN ricardo9$$
The natural debt limit is obtained by setting $c_t=0$ for all $t$
in \Ep{ricardo9},
$$b_t \geq \tilde b_t = \sum_{j=0}^\infty R^{-j} (\tau_{t+j} - y_{t+j}).
    \EQN ricardo3a$$
A comparison of  equations \Ep{ricardo3} and \Ep{ricardo3a} indicates how taxes affect
$\tilde b_t$



We use the following definition:
\medskip\noindent{\sc Definition:} Given initial conditions $(b_0, B_0)$,
an {\it equilibrium} is
a household plan $\{c_t,   b_{t+1}\}_{t=0}^\infty$ and a government
policy $\{g_t, \tau_t, B_{t+1}\}_{t=0}^\infty$ such that (a) the government
policy satisfies the government budget    constraint \Ep{ricardo7}, and
(b) given $\{\tau_t\}_{t=0}^\infty$, the household's  plan is optimal.
\medskip
  We can now state a Ricardian proposition
 under the natural debt limit.

\medskip\noindent{\sc Proposition 1:} Suppose that  the natural
debt limit prevails. Given initial conditions $(b_0, B_0)$,
let $\{\bar c_t, \bar b_{t+1}\}_{t=0}^\infty$ and
$\{\bar g_t, \bar \tau_t , \bar B_{t+1}\}_{t=0}^\infty$ be an equilibrium.
Consider any other tax policy $\{\hat \tau_t\}_{t=0}^\infty$ satisfying
$$ \sum_{t=0}^\infty R^{-t} \hat \tau_t =
\sum_{t=0}^\infty R^{-t} \bar \tau_t . \EQN ricardo10$$
Then  $\{\bar c_t, \hat b_{t+1}\}_{t=0}^\infty$ and $\{\bar g_t, \hat \tau_t,
\hat B_{t+1}\}_{t=0}^\infty$ is also an equilibrium where
$$ \hat b_t = \sum_{j=0}^\infty R^{-j} (\bar c_{t+j} +
\hat \tau_{t+j} - y_{t+j})                         \EQN ricardobb$$
and
$$ \hat B_t = \sum_{j=0}^\infty R^{-j}(\hat \tau_{t+j} -\bar g_{t+j}).
                                                    \EQN ricardobbg$$
\medskip

\noindent{\sc Proof:} The first point of the proposition
is that the same consumption plan $\{\bar c_t\}_{t=0}^\infty$, but adjusted
borrowing plan $\{\hat b_{t+1}\}_{t=0}^\infty$, solve the household's optimum
problem under the altered government tax scheme.
    Under the natural debt limit,
the household in effect faces  a single intertemporal
budget constraint \Ep{ricardo9}. At time $0$, the household can be
thought of as choosing an optimal consumption plan subject to
the single constraint,
$$ b_0 = \sum_{t=0}^\infty R^{-t} (c_{t} - y_{t})
        +  \sum_{t=0}^\infty R^{-t} \tau_{t}.$$
Thus, the household's budget set, and therefore its optimal  plan,
does not depend
on the timing of taxes, only their present value.   The altered
tax plan leaves the household's   intertemporal   budget set unaltered
and therefore doesn't affect its optimal consumption plan. Next,
we construct the adjusted borrowing plan
$\{\hat b_{t+1}\}_{t=0}^\infty$ by solving the budget constraint \Ep{ricardo2tax}
forward to obtain \Ep{ricardobb}.\NFootnote{It is straightforward to verify that
the adjusted borrowing plan $\{\hat b_{t+1}\}_{t=0}^\infty$
must satisfy the transversality condition
\Ep{ricardo_TVC}. In any period $(k-1)\geq 0$, solving the budget constraint
\Ep{ricardo2tax} backward yields
$$
b_k = \sum_{j=1}^k R^j \left[ y_{k-j} - \tau_{k-j}
           - c_{k-j} \right] + R^k b_0.
$$
Evidently, the difference between $\bar b_k$ of the initial
equilibrium and $\hat b_k$ is equal to
$$
\bar b_k - \hat b_k = \sum_{j=1}^k R^j
          \left[ \hat \tau_{k-j} - \bar \tau_{k-j} \right],
$$
and after multiplying both sides by $R^{1-k}$,
$$
R^{1-k} \left(\bar b_k - \hat b_k \right) = R \sum_{t=0}^{k-1} R^{-t}
          \left[ \hat \tau_{t} - \bar \tau_{t} \right].
$$
The limit of the right side is zero when $k$ goes to infinity due to
condition \Ep{ricardo10}, and hence, the fact that the
equilibrium borrowing plan $\{\bar b_{t+1}\}_{t=0}^\infty$ satisfies transversality
condition \Ep{ricardo_TVC} implies that so must
$\{\hat b_{t+1}\}_{t=0}^\infty$.}
The adjusted borrowing plan trivially
satisfies  the (adjusted) natural debt limit in every period,
since the consumption
plan $\{\bar c_t\}_{t=0}^\infty$ is a nonnegative sequence.

The second point of the proposition is that the altered government tax
and borrowing plans continue to satisfy the government's budget constraint.
In particular, we see that the government's budget set at time $0$ does
not depend on the timing of taxes, only their present value,
$$
B_0  = \sum_{t=0}^\infty R^{-t} \tau_t - \sum_{t=0}^\infty R^{-t} g_{t}.
$$
Thus, under the altered tax plan with an unchanged present value of taxes,
the government can finance the same expenditure plan $\{\bar g_t\}_{t=0}^\infty$.
The adjusted borrowing plan $\{\hat B_{t+1}\}_{t=0}^\infty$ is computed
in a similar way as above to arrive at \Ep{ricardobbg}. \qed

\medskip

This proposition depends on imposing  on the household  the natural debt limit,
which is weaker than the no-borrowing constraint.
Under the no-borrowing constraint, we require that the asset choice
$b_{t+1}$ at time $t$ satisfy budget constraint \Ep{ricardo2tax}
and  not fall below zero. That is, under the no-borrowing constraint,
we have to check more than just a single  intertemporal budget
constraint for the household at time $0$.
Changes in the timing of taxes that
obey equation \Ep{ricardo10} evidently alter the right side of
equation \Ep{ricardo2tax}
and can, for example, cause a previously binding
 borrowing constraint no longer
to  be binding, and  vice versa.
Binding borrowing constraints in either the initial
$\{\bar \tau_t\}_{t=0}^\infty$ equilibrium or the new $\{\hat \tau_t\}_{t=0}^\infty$ equilibria eliminates
a Ricardian proposition as general as Proposition 1.
More restricted versions of the proposition evidently hold across restricted
equivalence classes of taxes that do not alter when the borrowing
constraints are binding
across the two equilibria being compared.


\medskip
\specsec{Proposition 2:}  Consider an initial equilibrium
with consumption path $\{\bar c_t\}_{t=0}^\infty$
in which $b_{t+1} >0$ for all $t \geq 0$.  Let $\{\bar \tau_t \}_{t=0}^\infty$
be the tax  rate in the initial equilibrium, and let
$\{\hat \tau_t\}_{t=0}^\infty$ be any other tax-rate sequence for which
$$ \hat b_t = \sum_{j=0}^\infty R^{-j} (\bar c_{t+j} +
\hat \tau_{t+j} - y_{t+j}) \geq 0 $$
for all $t \geq 0$.  Then
$\{\bar  c_t\}_{t=0}^\infty$ is also an equilibrium allocation
for the $\{\hat \tau_t\}_{t=0}^\infty$ tax sequence.
\medskip
We leave the proof of this proposition to the reader.
\section{Linked generations interpretation}

  Much of the
preceding analysis with borrowing constraints applies to a setting
with overlapping generations linked by a bequest motive.
Assume that there is a sequence of  one-period-lived agents.
For each $t\geq 0$ there is a one-period-lived agent who
values consumption and the utility of his direct descendant, a young
person at time $t+1$.  Preferences of a young person at $t$ are
ordered by
$$  u(c_t)  + \beta V(b_{t+1}), $$
where $u(c)$ is the same utility function
as in the previous section,
$b_{t+1}\geq 0$  are bequests from the  time $t$   person to
the time $t+1$ person, and $V(b_{t+1})$ is the maximized utility
function of a time $t+1$ agent.   The maximized utility function
is defined recursively by
$$ V(b_t) =  \max_{c_t, b_{t+1}} \{u(c_t) + \beta V(b_{t+1}) \} \EQN ricardo11$$
where the maximization is subject to
$$ c_t + R^{-1} b_{t+1} \leq y_t -\tau_t + b_t\EQN ricardo12$$
and $b_{t+1} \geq 0$.   The constraint $b_{t+1} \geq 0$ requires that
bequests cannot be negative.   Notice that a person cares about
his direct descendant, but not vice versa.
We continue to assume that there is an infinitely lived government
whose taxes and purchasing and borrowing strategies are as described in
the previous section.

In consumption outcomes, this model is equivalent to the previous model
with a no-borrowing constraint.  Bequests here play the
role of savings $b_{t+1}$ in the previous model.  A positive
savings condition $b_{t+1} >0$  in the previous version of the
model becomes an ``operational bequest motive'' in the overlapping
generations model.

  It follows that we can obtain a restricted Ricardian
equivalence proposition, qualified as in Proposition 2.
The qualification is that the initial equilibrium must have an operational
bequest motive for all $t \geq 0$, and that the new
tax policy must not be so different from  the initial one
that it renders the bequest motive inoperative.

\section{Concluding remarks}%
The arguments in this chapter were cast in a setting with
an exogenous interest rate $R$ and a capital market that is outside of the
model. When we discussed potential failures of Ricardian equivalence due to
households facing no-borrowing constraints, we were also implicitly
contemplating changes in the government's outside asset position. For example,
consider an altered tax plan $\{\hat \tau_t\}_{t=0}^\infty$ that satisfies \Ep{ricardo10}
and shifts taxes away from the future toward the present. A
large enough change will definitely ensure that the government is a lender
in early periods. But since the households are not allowed to become indebted,
the government must lend abroad and we can show that Ricardian equivalence
breaks down.

The readers might be able to  anticipate
 the nature of the general equilibrium proof of Ricardian
equivalence in chapter \use{assetpricing1}.
 First, private consumption and government
expenditures must then be consistent with the aggregate endowment in each
period, $c_t+g_t=y_t$, which implies that an altered tax plan cannot
affect the consumption allocation as long as government expenditures
are kept the same. Second, interest rates are determined by intertemporal
marginal rates
of substitution evaluated at the equilibrium consumption allocation, as
studied in chapter \use{recurge}.
 Hence, an unchanged consumption allocation implies
that interest rates are also unchanged.
Third, at those very interest rates, it can be shown that households
would like to choose asset positions that exactly offset any changes in
the government's asset holdings implied by an altered tax plan.
For example, in the case of the tax change contemplated in the preceding
paragraph, the households would demand loans exactly equal to the
rise in government lending generated by budget surpluses in early periods.
The households would use those loans to meet the higher taxes and thereby
 finance an unchanged consumption plan.

\auth{Blanchard, Olivier J.}   \auth{Diamond, Peter A.}
\auth{Barro, Robert J.}\auth{Bernheim, B. Douglas} \auth{Bagwell, Kyle}
The finding of Ricardian equivalence in the infinitely lived agent model
is a useful starting point for identifying alternative assumptions under
which the irrelevance result might fail to hold,\NFootnote{Seater (1993) reviews
the theory and empirical evidence on Ricardian equivalence.%
\auth{Seater, John, J}}%
such as our imposition of borrowing constraints that are tighter
than the ``natural debt limit.''
Another deviation from the benchmark model is finitely lived agents,
as analyzed by Diamond (1965) and Blanchard (1985).
But as suggested by Barro (1974)
and shown in this chapter, Ricardian equivalence will
continue to hold if agents are altruistic toward their descendants
and there is an operational bequest motive.
 Bernheim and Bagwell (1988)
take this argument to its extreme and formulate a model where all
agents are interconnected because of linkages across dynastic
families.  They show how those linkages can become extensive enough  to render neutral all redistributive
policies, including ones attained via distortionary taxes. But, in general, replacing
lump-sum taxes by distortionary taxes is a sure-fire way to undo
Ricardian equivalence (see, e.g.,\ Barsky, Mankiw, and Zeldes, 1986).
We will return to the question of the timing of distortionary
taxes in chapter \use{optax}. Kimball and Mankiw (1989) describe
how incomplete markets can make the timing of taxes interact with
a precautionary savings motive in a way that disarms
Ricardian equivalence.    We take up precautionary savings and
incomplete markets in chapters \use{selfinsure} and
\use{incomplete}.  Finally, by
allowing distorting taxes to be history dependent, Bassetto and
Kocherlakota (2004) attain a Ricardian equivalence result for a
variety of taxes. \auth{Bassetto, Marco} \auth{Kocherlakota,
Narayana R.}\auth{Kimball, Miles S.} \auth{Barsky, Robert}
\auth{Mankiw, Gregory} \auth{Zeldes, Stephen P.}
