
\input grafinp3
%\input grafinput8
\input psfig

%\showchaptIDtrue
%\def\@chaptID{19.}
%\input gayejnl.txt
%\input gayedef.txt
\def\lege{\raise.3ex\hbox{$>$\kern-.75em\lower1ex\hbox{$<$}}}

%\hbox{}
\footnum=0
\chapter{Equilibrium Search, Matching, and Lotteries\label{search2}}

\section{Introduction}

   This chapter presents various equilibrium models of search and matching. We
describe (1) Lucas and Prescott's version of an island model; (2) some matching
models in the style of Mortensen, Pissarides, and Diamond; and (3) a
search model of money along the lines of Kiyotaki and Wright.

Chapter \use{search1} studied the optimization problem of a single
unemployed agent who searched for a job by drawing from an
exogenous wage offer distribution.  We now turn to a model with a
continuum of agents who interact across a large number of
spatially separated labor markets. Phelps (1970, introductory
chapter)
\auth{Phelps, Edmund S.}%
describes such an ``island economy,''
\index{island model}%
and a formal framework is analyzed by Lucas and Prescott (1974).
\auth{Lucas, Robert E., Jr.}\auth{Prescott, Edward C.}%
The agents on an island can choose to work at
the market-clearing wage in their own labor market or seek their
fortune by moving to another island and its labor market. In an equilibrium,
agents tend to move to islands that experience good productivity shocks,
while an island with bad productivity may see some of its labor force depart.
Frictional unemployment arises because moves between labor markets take
time.

Another approach to model unemployment is the matching framework
\index{matching model}%
described by Diamond (1982), Mortensen (1982), and Pissarides (1990).
\auth{Diamond, Peter A.}\auth{Mortensen, Dale T.}\auth{Pissarides, Christopher}%
These models postulate the existence of a matching function that
maps measures of unemployment and vacancies into a measure of matches. A match
pairs a worker and a firm, who then have to bargain about how to share the
``match surplus,''
\index{matching model!match surplus}%
that is, the value that will be lost if the two parties cannot agree and
break the match. In contrast to the island model with price-taking behavior
and no externalities, the decentralized outcome in the matching framework is
in general not efficient. Unless parameter values satisfy a knife-edge
restriction, there will be either  too many or too few vacancies posted
in an equilibrium.  The efficiency problem is further exacerbated if it is
assumed that heterogeneous jobs must be created via a single matching function.
This assumption creates a tension between getting an efficient mix of jobs and
an efficient total supply of jobs.

As a reference point to models with search and matching frictions, we also study a
frictionless aggregate labor market but assume that labor is indivisible.  For
example, agents are constrained to work either full time or not at all.  This
kind of assumption has been used in the real business cycle literature to generate
unemployment. If markets for contingent claims exist,
Hansen (1985) and Rogerson (1988)
\auth{Hansen, Gary D.} \auth{Rogerson, Richard}%
show that employment lotteries
\index{employment lottery}%
can be welfare enhancing and that they imply that only a fraction  of agents will
be employed in an equilibrium. Using this model and the other two frameworks
that we have mentioned, we analyze how layoff taxes affect an economy's employment
level. The different models yield very different conclusions, shedding further
light on the economic forces at work in the various frameworks.

To illustrate another application of search and matching, we study Kiyotaki and
Wright's (1993) search model of money.
Agents who differ with respect to their taste for different goods meet pairwise
and at random. In this model, fiat money can potentially ameliorate the problem
of ``double coincidence of wants.''
\auth{Kiyotaki, Nobuhiro} \auth{Wright, Randall}%

\index{island model}\auth{Lucas, Robert E., Jr.} \auth{Prescott, Edward C.}%
\section{An island model}\label{sec:LucasPrescott}%
The model here is a simplified version of Lucas and
Prescott's (1974) ``island economy.''
There is a continuum of agents populating a large number of spatially separated
labor markets.  Each island is endowed with an aggregate production function
$\theta f(n)$, where $n$ is the island's employment level and $\theta>0$ is an
idiosyncratic productivity shock. The production function
satisfies
$$f'>0, \qquad f''<0, \qquad {\rm and} \qquad
               \lim_{n\to0} f'(n) \,=\, \infty\,.       \EQN prodfn
$$
The productivity shock takes on $m$ possible values,
$\theta_1 < \theta_2 < \cdots < \theta_m$, and the
shock is governed by strictly positive transition probabilities,
$\pi(\theta, \theta') > 0$. That is, an island with a current
productivity shock of $\theta$ faces a probability
$\pi(\theta, \theta')$ that its next period's shock is $\theta'$.
The productivity shock is persistent in the sense that the
cumulative distribution function,
${\rm Prob}\left(\theta' \leq \theta_k \vert \theta \right) =
\sum_{i=1}^k \pi(\theta, \theta_i)$, is a decreasing
function of $\theta$.


At the beginning of a period, agents are distributed in some
way over the islands. After observing the productivity shock, the
agents decide whether or not to move to another island. A mover
forgoes his labor earnings in the period of the move, whereas he
can choose the destination with complete information about
current conditions on all islands. An agent's decision to work or to move
is taken so as to maximize the expected
present value of his earnings stream. Wages are determined
competitively, so that each island's labor market clears with
a wage rate equal to the marginal product of labor.
We will study stationary equilibria.


\subsection{A single market (island)}

The state of a single market is given by its productivity level
$\theta$ and its beginning-of-period labor force $x$. In an
equilibrium, there will be functions mapping this state into
an employment level, $n(\theta, x)$, and a wage rate,
$w(\theta, x)$. These functions must satisfy the market-clearing
condition
$$w(\theta, x) \;=\; \theta f'\bigl[ n(\theta, x) \bigr]   $$
and the labor supply constraint
$$n(\theta, x) \;\leq\; x \,.
$$

Let $v(\theta, x)$ be the value of the optimization problem
for an agent finding himself in market $(\theta, x)$ at the
beginning of a period. Let $v_u$ be the expected value obtained
next period by an agent leaving the market, a value to be
determined by conditions in the aggregate economy.
The value now associated with leaving the
market is then $\beta v_u$.
 The Bellman equation can then be
written as
$$v(\theta, x) \;=\; \max \Bigl\{\beta v_u \,,\; w(\theta, x) \,+\,
   \beta E\left[v(\theta',x') \vert \theta, x \right]
                                                     \Bigr\}\,, \EQN value1
$$
where the conditional expectation refers to the evolution of
$\theta'$ and $x'$ if the agent remains in the same market.


The value function $v(\theta, x)$ is equal to $\beta v_u$ whenever
there are any agents leaving the market. It is instructive to
examine the opposite situation when no one leaves the market.
This means that the current employment level is
$n(\theta, x) = x$ and the wage rate becomes
$w(\theta, x) = \theta f'(x)$. Concerning the continuation
value for next period,
$\beta E\left[v(\theta',x') \vert \theta, x \right]$, there
are two possibilities:

%\vskip.2cm
\medskip
\noindent {\it Case i:} All agents remain, and some additional agents arrive
next period. The arrival of new agents corresponds to a continuation
value of $\beta v_u$ in the market. Any value less than $\beta v_u$ would not
attract any new agents, and a value higher than $\beta v_u$ would be driven
down by a larger inflow of new agents. It follows that the current
value function in equation \Ep{value1} can under these circumstances
be written as
$$v(\theta, x) \;=\; \theta \, f'(x) \,+\, \beta v_u \,.
$$
\medskip
\noindent {\it Case ii:} All agents remain, and no additional agents arrive
next period. In this case $x' = x$, and the lack of new arrivals
implies that the market's continuation value is less than or equal to
$\beta v_u$. The current value function becomes
$$v(\theta, x) \;=\; \theta \, f'(x) \,+\,
   \beta E\left[v(\theta',x) \vert \theta \right] \;\leq\;
                      \theta \, f'(x) \,+\, \beta v_u \,.
$$
\medskip
%\vskip.3cm
After putting both of these cases together, we can rewrite the value
function in equation \Ep{value1} as follows:
$$v(\theta, x) \;=\; \max \Bigl\{\beta v_u\,,\; \theta \, f'(x) \,+\,
  \min \bigl\{ \beta v_u\,,\; \beta E\left[v(\theta',x) \vert \theta \right]
      \bigr\} \Bigr\}\,. \EQN value2
$$
Given a value for $v_u$, this is a well-behaved functional equation
with a unique solution $v(\theta, x)$. The value function is
nondecreasing in $\theta$ and nonincreasing in $x$.

On the basis of agents' optimization behavior, we can study the
evolution of the island's
labor force. There are three possible cases:

%\vskip.2cm
\medskip
\noindent {\it Case 1:} Some agents leave the market. An implication is
that no
additional workers will arrive next period, when the beginning-of-period
labor force will be equal to the current employment level,
$x'=n$. The current employment level, equal to  $x'$, can then be computed from
the condition that agents remaining in the market receive the same
utility as the movers, given by $\beta v_u$,
$$\theta \, f'(x') \,+\, \beta E\left[v(\theta',x') \vert \theta \right]
      \;=\; \beta v_u \,.                                 \EQN value2aa
$$
This equation implicitly defines $x^+(\theta)$ such that
$x'=x^+(\theta)$ if $x\geq x^+(\theta)$.

%\vskip.2cm
\medskip
\noindent {\it Case 2:} All agents remain in the market, and some additional
workers arrive next period. The arriving workers must expect to attain the
value $v_u$, as discussed in case {\it i}. That is, next period's
labor force $x'$ must be such that
$$ E\left[v(\theta',x') \vert \theta \right]
      \;=\; v_u \,.                                  \EQN value2bb
$$
This equation implicitly defines $x^-(\theta)$ such that
$x'=x^-(\theta)$ if $x\leq x^-(\theta)$. It can be seen that
$x^-(\theta) < x^+(\theta)$.

%\vskip.2cm
\medskip
\noindent {\it Case 3:} All agents remain in the market, and no additional
workers arrive next period. This situation was discussed in case {\it ii}.
It follows here that $x'=x$ if $x^-(\theta) <x< x^+(\theta)$.


%\vskip.5cm

\subsection{The aggregate economy}

The previous section assumed an exogenous value to search,
$v_u$. This assumption will be maintained in the first part
of this section on the aggregate economy. The approach
amounts to assuming a perfectly elastic outside labor supply
with reservation utility $v_u$. We end the section by
showing how to endogenize the value to search in the face
of a given inelastic aggregate labor supply.

Define a set $X$ of possible labor forces in a market as
follows:
$$
X \equiv \cases{
 \Bigl\{x \in \bigl\{x^-(\theta_i)\,,\; x^+(\theta_i)\bigr\}_{i=1}^m \,:\;
   x^+(\theta_1) \leq x \leq x^-(\theta_m) \Bigr\}, & \cr
\noalign{\vskip.2cm}
\hskip5.5cm {\rm if} \;x^+(\theta_1) \leq x^-(\theta_m); & \cr
\noalign{\vskip.3cm}
 \Bigl\{x \in \left[x^-(\theta_m)\,,\; x^+(\theta_1) \right] \Bigr\}\,,
                     \hskip1.5cm {\rm otherwise.} &                \cr}
$$
The set $X$ is the ergodic set of labor forces in a stationary
equilibrium. This can be seen by considering a single market
with an initial labor force $x$. Suppose that $x > x^+(\theta_1)$;
the market will then
eventually experience the least advantageous productivity shock
with a next period's labor force of $x^+(\theta_1)$. Thereafter,
the island can at most attract a labor force $x^-(\theta_m)$
associated with the most advantageous productivity
shock.
Analogously, if the market's initial labor force is
$x < x^-(\theta_m)$, it will eventually have a labor force of
$x^-(\theta_m)$ after experiencing the most advantageous productivity
shock. Its labor force will thereafter never fall below
$x^+(\theta_1)$, which is the next period's labor force of a market
experiencing the least advantageous shock (given a current
labor force greater than or equal to $x^+(\theta_1)$). Finally,
in the case that $x^+(\theta_1) > x^-(\theta_m)$, any initial
distribution of workers such that each island's labor force
belongs to the closed interval
$\left[x^-(\theta_m)\,,\; x^+(\theta_1) \right]$
can constitute a stationary equilibrium. This would be a parameterization
of the model where agents do not find it worthwhile to relocate in
response to productivity shocks.


In a stationary equilibrium, a market's transition probabilities
among states $(\theta, x)$ are given by
$$\eqalign{
\Gamma(\theta', x' \vert \theta, x) \;=\; \pi(\theta, \theta')\;\cdot\;
I\Bigl(\,
&\left[x' = x^+(\theta) \; \hbox{and} \; x \geq x^+(\theta) \right]
\;\hbox{or}\;                                                     \cr
&\left[x' = x^-(\theta) \; \hbox{and} \; x \leq x^-(\theta) \right]
\;\hbox{or}\;                                                     \cr
&\left[x' = x \; \hbox{and} \; x^-(\theta) < x < x^+(\theta) \right]
                                                           \,  \Bigr)\,, \cr
\noalign{\vskip.2cm}
&\hskip1cm \hbox{for} \ \ x,x' \in X \;\; \hbox{and all}\ \
                          \theta, \, \theta'\,;                     \cr}
$$
where $I(\cdot)$ is the indicator function that
 takes on the value
1 if any of its arguments are true and 0 otherwise.
These transition probabilities define an operator $P$ on distribution
functions $\Psi_t(\theta,x;v_u)$ as follows: Suppose that at a point
in time, the distribution of productivity shocks and labor forces
across markets is given by $\Psi_t(\theta,x;v_u)$, then the next
period's distribution is
$$\eqalign{
\Psi_{t+1}(\theta',x';v_u) \;&=\;
P \Psi_{t}(\theta',x';v_u) \cr
\;&=\; \sum_{x \in X}\ \sum_{\theta}
            \Gamma(\theta', x' \vert \theta, x)\, \Psi_t(\theta,x;v_u)\,.\cr}
$$
Except for the case when the stationary equilibrium involves no
reallocation of labor, the described process has a
unique stationary distribution, $\Psi(\theta,x;v_u)$.

Using the stationary distribution $\Psi(\theta,x;v_u)$, we can
compute the economy's average labor force per market,
$$
\bar x(v_u) \;=\; \sum_{x \in X}\ \sum_{\theta} x \, \Psi(\theta,x;v_u)\,,
$$
where the argument $v_u$ makes explicit that the construction
of a stationary equilibrium rests on the maintained assumption
that the value to search is exogenously given by $v_u$. The
economy's equilibrium labor force $\bar x$ varies negatively
with $v_u$. In a stationary equilibrium with labor
movements, a higher value to search is only consistent with
higher wage rates, which in turn require higher marginal
products of labor, that is, a smaller labor force on the islands.

From an economy-wide viewpoint, it is the size of the labor
force that is fixed, let's say $\hat x$, and the value to
search that adjusts to clear the markets. To find a
stationary equilibrium for a particular $\hat x$, we trace
out the schedule $\bar x(v_u)$ for different values of $v_u$.
The equilibrium pair $(\hat x, v_u)$ can then be read off at the
intersection $\bar x(v_u) = \hat x$, as illustrated in
Figure \Fg{fig191f}. %19.1.

%%%%%%%%%%%
%\topinsert
%$$\grafone{fig191.eps,height=2.5in}{{\bf Figure 19.1}  The
%curve maps an economy's average labor force per market, $\bar x$,
%into the stationary-equilibrium value to search, $v_u$.}
%$$\endinsert
%%%%%%%%%%%%

\midfigure{fig191f}
\centerline{\epsfxsize=3truein\epsffile{fig191.eps}}
\caption{The curve maps an economy's average labor force per market, $\bar x$,
into the stationary equilibrium value to search, $v_u$.}
\infiglist{fig191f}
\endfigure

%\vfil\eject
 \index{matching model}
\section{A matching model}\label{sec:matchingmodel}%
Another model of unemployment is the matching framework, as described
by Diamond (1982), Mortensen (1982), and Pissarides (1990).
\auth{Diamond, Peter A.}\auth{Mortensen, Dale T.}\auth{Pissarides, Christopher}%
The basic
model is as follows: Let there be a continuum of identical workers
with measure normalized to 1. The workers are infinitely lived and
risk neutral. The objective of each worker is to maximize the expected
discounted value of leisure and labor income. The leisure enjoyed by
an unemployed worker is denoted $z$, while the current utility of an
employed worker is given by the wage rate $w$. The workers'
discount factor is $\beta = (1+r)^{-1}$.

The production technology is constant returns to scale, with labor
as the only input. Each employed worker produces $y$ units of output.
Without loss of generality, suppose each firm employs at most one
worker. A firm entering the economy incurs a vacancy cost $c$ in
\index{vacancy}%
each period when looking for a worker, and in a subsequent match
the firm's per-period earnings are $y-w$. All matches are exogenously
destroyed with per-period probability $s$.
 Free entry implies that the expected
discounted stream of a new firm's vacancy costs and earnings is
equal to zero. The firms have the same discount factor as the
workers (who would be the owners in a closed economy).

The measure of successful matches in a period is given by a matching
function  \index{matching model!matching function}%
 $M(u,v)$, where $u$ and $v$ are the aggregate measures of
unemployed workers and vacancies. The matching function is increasing
in both its arguments, concave, and homogeneous of degree 1. By the
homogeneity assumption, we can write the probability of filling a
vacancy as $q(v/u) \equiv M(u,v)/v$. The ratio between vacancies and
unemployed workers, $\theta \equiv v/u$, is commonly labelled the
{\it tightness\/} of the labor market.  \index{tightness of labor market}%
The probability that an unemployed
worker will be matched in a period is $\theta q(\theta)$. We
will assume that the matching function has the Cobb-Douglas form,
which implies constant elasticities,
$$\EQNalign{
M(u,v) & \;=\; A u^{\alpha} v^{1-\alpha} \,, \cr { \partial M(u,v) \over \partial u} {u \over M(u,v) }  & \;=\;
- q'(\theta) {\theta \over q(\theta)}  \;=\; \alpha \,, \EQN matchingfunction \cr}
$$
where $A>0$, $\alpha\in(0,1)$, and the last equality will be used
repeatedly in our derivations that follow.

Finally, the wage rate is assumed to be determined in a Nash bargain between
a matched firm and worker.  Let $\phi\in[0,1)$ denote the worker's bargaining
strength, or his weight in the Nash product, as described in the next subsection.


%\vfill\eject
\subsection{A steady state}

In a steady state, the measure of laid-off workers in a period,
$s(1-u)$, must be equal to the measure of unemployed workers
gaining employment, $\theta q(\theta) u$. The steady-state
unemployment rate can therefore be written as
$$
u \;=\; {s \over s \,+\, \theta q(\theta) } \,. \EQN unemp
$$
To determine the equilibrium value of $\theta$, we now turn to
the situations faced by firms and workers, and we impose the
no-profit condition for vacancies and the Nash bargaining
outcome on firms' and workers' payoffs.

A firm's value of a filled job $J$ and a vacancy $V$ are given by
$$\EQNalign{
J\;&=\; y\,-\, w\,+\,\beta \left[ sV \,+\, (1-s)J \right] \,,  \EQN J_eq \cr
V\;&=\; -c \,+\, \beta \bigl\{ q(\theta) J \,+\,
[1 - q(\theta)] V \bigr\}                     \,.  \EQN V_eq \cr}
$$
That is, a filled job turns into a vacancy with probability
$s$, and a vacancy turns into a filled job with probability
$q(\theta)$. After invoking the condition that vacancies earn
zero profits, $V=0$, equation \Ep{V_eq} becomes
$$
J \;=\; {c \over \beta q(\theta)} \,,                          \EQN V2_eq
$$
which we substitute into equation \Ep{J_eq} to arrive at
$$
w\;=\; y \,-\, {r+s \over q(\theta)} c \,.                     \EQN wage1
$$
The wage rate in equation \Ep{wage1} ensures that firms with vacancies
break even in an expected present-value sense. In other words,
a firm's match surplus must be equal to $J$ in equation \Ep{V2_eq}
in order for the firm to recoup its average discounted costs
of filling a vacancy.

\index{matching model!match surplus}
\index{Nash bargaining}%
The worker's share of the match surplus
is the difference between the value of an employed
worker $E$ and the value of an unemployed worker $U$,
$$\EQNalign{
E\;&=\; w\,+\, \beta \bigl[s U \,+\, (1-s) E \bigr] \,,     \EQN E_eq \cr
U\;&=\; z\,+\, \beta \bigl\{\theta q(\theta) E \,+\,
[1 - \theta q(\theta)] U \bigr\} \,,           \EQN U_eq \cr}
$$
where an employed worker becomes unemployed with probability
$s$ and an unemployed worker finds a job with probability
$\theta q(\theta)$.  The worker's share of the match surplus, $E-U$, has to
be related to the firm's share of the match surplus, $J$, in a particular
way to be consistent with Nash bargaining.
Let the total match
surplus be denoted $S=(E-U)+J$, which is shared according to
the Nash product
$$\EQNalign{
&\max_{(E-U), J} \; \; (E-U)^\phi J^{1-\phi} \EQN Nash_prod   \cr
\noalign{\vskip.2cm}
&\hbox{subject to} \quad S \,=\, E-U \,+\, J\,,  \cr}
$$
with solution
$$
E-U \;=\; \phi S \  \hbox{and} \ J \;=\; (1-\phi) S\,. \EQN split
$$

After solving equations \Ep{J_eq} and \Ep{E_eq} for $J$ and $E$,
respectively, and substituting them into equations \Ep{split}, we get
$$
w\;=\; {r \over 1+r} U \,+\, \phi\left(y\,-\, {r \over 1+r} U\right)\,.
                                                     \EQN wage2
$$
The expression is quite intuitive when seeing $r(1+r)^{-1}U$
as the annuity value of being unemployed. The wage rate is
just equal to this outside option plus the worker's share $\phi$ of
the one-period match surplus. The annuity value of being unemployed
can be obtained by solving equation \Ep{U_eq} for $E-U$ and
substituting this expression and equation \Ep{V2_eq} into equations
\Ep{split},
$$
{r \over 1+r} U \;=\; z \,+\, {\phi\,\theta\,c \over 1-\phi}\,.  \EQN U_value
$$
Substituting equation \Ep{U_value} into equation \Ep{wage2},
we obtain still another expression for the wage rate,
$$
w\;=\; z \,+\, \phi (y-z\,+\,\theta c)\,.              \EQN wage3
$$
That is, the Nash bargaining results in the worker receiving
compensation for lost leisure $z$ and a fraction $\phi$ of both
the firm's output in excess of $z$ and the economy's average
vacancy cost per unemployed worker.

The two expressions for the wage rate in equations \Ep{wage1} and
\Ep{wage3} determine jointly the equilibrium value for $\theta$,
$$
y-z \;=\; {r\,+\,s\,+\,\phi\,\theta\,q(\theta) \over
                    (1-\phi) q(\theta) } \; c \,.            \EQN equili
$$
This implicit function for $\theta$ ensures that
vacancies are associated with zero profits, and that firms' and
workers' shares of the match surplus are the outcome of Nash
bargaining.



\subsection{Welfare analysis}
A  planner would choose an allocation that maximizes the discounted
value of output and leisure net of vacancy costs. The
social optimization problem does not involve any uncertainty because
the aggregate fractions of successful matches and destroyed matches are
just equal to the probabilities of these events. The social
planner's problem of choosing the measure of vacancies,
$v_t$, and next period's employment level, $n_{t+1}$, can then be
written as
$$\EQNalign{
&\max_{\{v_t,n_{t+1}\}_{t}} \;\; \sum_{t=0}^\infty \beta^t
     \left[y n_t \,+\, z(1-n_t) \,-\, c v_t \right] \,,   \EQN soc_obj  \cr
&\hbox{subject to} \quad n_{t+1} \;=\; (1-s)n_t \,+\,
                                q\left({v_t \over 1-n_t}\right) v_t \,,
                                                           \EQN law_mot \cr
&\hbox{given \ \ } n_0\,.                                               \cr}
$$
The first-order conditions with respect to $v_t$ and $n_{t+1}$,
respectively, are
%$$\EQNalign{
%-\beta^t c \,&+\, \lambda_t \left[q'\left(\theta_t\right) \theta_t
%              \,+\, q\left(\theta_t\right) \right] \;=\; 0 \,, \EQN foc_v \cr
%\noalign{\vskip.2cm}
%-\lambda_t \,&+\, \beta^{t+1} (y-z) \cr
%\noalign{\vskip.1cm}
%           \,&+\, \lambda_{t+1}
%  \left[ (1-s) \,+\, q'\left(\theta_{t+1}\right) \theta_{t+1}^2 \right]
%                                               \;=\; 0\,,     \EQN foc_n  \cr}
%$$
$$\EQNalign{
-\beta^t c \, +\, \lambda_t \left[q'\left(\theta_t\right) \theta_t
              \,+\, q\left(\theta_t\right) \right] & \;=\; 0 \,, \EQN foc_v \cr
\noalign{\vskip.2cm}
-\lambda_t \, +\, \beta^{t+1} (y-z)
           \,+\, \lambda_{t+1}
  \left[ (1-s) \,+\, q'\left(\theta_{t+1}\right) \theta_{t+1}^2 \right]
                                               & \;=\; 0\,,   \hskip1cm  \EQN foc_n  \cr}
$$
where $\lambda_t$ is the Lagrangian multiplier on equation \Ep{law_mot}.
Let us solve for $\lambda_t$ from equation \Ep{foc_v}, and substitute
into equation \Ep{foc_n} evaluated at a stationary solution
$$
y-z \;=\; {r\,+\,s\,+\,\alpha\,\theta\,q(\theta) \over
                    (1-\alpha) q(\theta) } \; c \,.            \EQN soc_opt
$$
A comparison of this social optimum to the private outcome in equation
\Ep{equili} shows that the decentralized equilibrium is only efficient
if $\phi = \alpha$. If the workers' bargaining strength $\phi$ exceeds
(falls below) $\alpha$, the equilibrium job supply is too low (high).
Recall that $\alpha$ is both the elasticity of the matching function with
respect to the measure of unemployment, and the negative of the
elasticity of the probability of filling a vacancy with respect to
$\theta_t$. In its latter meaning, a high $\alpha$ means that
an additional vacancy has a large negative impact on all firms'
probability of filling a vacancy; the social planner would therefore
like to curtail the number of vacancies by granting workers a relatively
high bargaining power. Hosios (1990) shows how the efficiency condition
$\phi=\alpha$ is a general one for the matching framework.

\auth{Hosios, Arthur J.}
It is instructive to note that the social optimum is equivalent
to choosing the worker's bargaining power $\phi$ such that the value of
being unemployed is maximized in a decentralized equilibrium. To see
this point, differentiate the value of being unemployed \Ep{U_value}
to find the slope of the indifference in the space of $\phi$ and
$\theta$,
$$
{ \partial \theta \over \partial \phi} \;=\; - {\theta \over \phi (1-\phi)}\,,
$$
and use the implicit function rule to find the corresponding slope of
the equilibrium relationship \Ep{equili},
$$
{ \partial \theta \over \partial \phi} \;=\; - {y\,-\,z\,+\,\theta c \over
\left[\phi \,-\, (r+s)\, q'(\theta)\, q(\theta)^{-2} \right] \, c }\;.
$$
We set the two slopes equal to each other because a maximum would be
attained at a tangency point between the highest attainable
indifference curve and equation \Ep{equili} (both curves are negatively
sloped and convex to the origin):
$$
y-z \;=\; {(r+s) {\displaystyle \alpha \over \displaystyle  \phi}
           \,+\,\phi\,\theta\,q(\theta) \over
                    (1-\phi) q(\theta) } \; c \,.            \EQN U_opt
$$
When we also require that the point of tangency satisfy the
equilibrium condition \Ep{equili}, it can be seen that
$\phi = \alpha$ maximizes the value of being unemployed in a
decentralized equilibrium. The solution is the same as the
social optimum because the social planner and an unemployed
worker both prefer an optimal rate of investment
in vacancies, one that takes matching externalities into account.
\index{matching model!match surplus}

\subsection{Size of the match surplus}
The size of the match surplus depends naturally on the output $y$
produced by the worker, which is lost if the match breaks up and
the firm is left to look for another worker. In principle,
this loss includes any returns to production factors used by the
worker that cannot be adjusted immediately. It might then
seem puzzling that a common assumption in the matching literature
is to exclude payments to physical capital when determining the
size of the match surplus (see, e.g., Pissarides, 1990).
\auth{Pissarides, Christopher}%
Unless capital can be moved without friction in the economy,
this exclusion of payments to physical capital must rest on some
implicit assumption of outside financing from a third party that
is removed from the wage bargain between the firm and the worker.
For example, suppose the firm's capital is financed by a financial
intermediary that demands specific rental payments in order
not to ask for the firm's bankruptcy.
As long as the financial intermediary can credibly distance itself
from the firm's and worker's bargaining, it would be rational for
the two latter parties to subtract the rental payments from the
firm's gross earnings and bargain over the remainder.

In our basic matching model, there is no physical capital, but
there is investment in vacancies. Let us consider the possibility
that a financial intermediary provides a single firm funding for this
investment. The simplest contract would
be that the intermediary hand over funds $c$ to a firm with a
vacancy in exchange for a promise that the firm pay
$\epsilon$ in every future period of operation. If the firm cannot
find a worker in the next period, it fails and the intermediary
writes off the loan, and otherwise the intermediary receives the
stipulated interest payment $\epsilon$ so long as a successful
match stays in business. This agreement with a single firm will
have a negligible effect on the economy-wide values of market
tightness $\theta$ and
the value of being unemployed $U$. Let us examine the consequences
for the particular firm involved and the worker it meets.

Under the conjecture that a match will be acceptable to both the
firm and the worker, we can compute the interest payment $\epsilon$
needed for the financial intermediary to break even in an expected
present-value sense,
$$
c\;=\; q(\theta) \, \beta  \sum_{t=0}^\infty
                          \beta^t (1-s)^t \epsilon
\quad \Longrightarrow \quad \epsilon \;=\; {r+s \over q(\theta)} \, c\,.
                                                            \EQN interest
$$
A successful match will then generate earnings net of the interest
payment equal to $\tilde y = y - \epsilon$. To determine how the
match surplus is split between the firm and the worker, we replace
$y$, $w$, $J$, and $E$ in equations \Ep{J_eq} and \Ep{E_eq}, and
\Ep{Nash_prod} by $\tilde y$, $\tilde w$, $\tilde J$, and $\tilde E$.
That is, $\tilde J$ and $\tilde E$ are the values to the firm and the worker,
respectively, for this particular filled job. We treat $\theta$, $V$, and
$U$ as constants, since they are determined in the rest of the
economy. The Nash bargaining can then be seen to yield
$$
\tilde w \;=\; {r \over 1+r} U \,+\,
                  \phi\left(\tilde y\,-\, {r \over 1+r} U\right)
         \;=\; {r \over 1+r} U \,+\, \phi {\phi \, (r+s) \over
                  (1-\phi)\,q(\theta) } \,c \,,
$$
where the first equality corresponds to the previous equation \Ep{wage2}.
The second equality is obtained after invoking
$\tilde y = y - \epsilon$ and equations \Ep{U_value}, \Ep{equili},
and \Ep{interest}, and the resulting
expression confirms the conjecture that the match is
acceptable to the worker who receives a wage in excess of the
annuity value of being unemployed. The firm will, of course, be
satisfied with any positive $\tilde y - \tilde w$ because it has
not incurred any costs whatsoever in order to form the match,
$$
\tilde y \,-\, \tilde w \;=\; {\phi \, (r+s) \over q(\theta)} \, c\;>\;0\,,
$$
where we once again have used $\tilde y = y - \epsilon$, equations
\Ep{U_value}, \Ep{equili}, and \Ep{interest}, and the
preceding expression for $\tilde w$. Note that $\tilde y - \tilde w =
\phi \epsilon$ with the following interpretation:
If the interest payment on the firm's investment, $\epsilon$, was
not subtracted
from the firm's earnings prior to the Nash bargain, the worker
would receive an increase in the wage equal to his share $\phi$ of
the additional ``match surplus.'' The present financial arrangement
saves the firm this extra wage payment, and the saving becomes the
firm's profit.  Thus, a single firm with the proposed contract
would have a strictly positive present value when entering the
economy of the previous subsection. If there were unlimited  entry of new firms having
access to intermediaries offering such a contract, those profits would be
competed away.  We ask the reader to characterize equilibrium outcomes
under free entry.
\index{matching model!heterogeneous jobs}

\auth{Caballero, Ricardo J.}
 \auth{Acemoglu, Daron} \auth{Bertola, Giuseppe} \auth{Davis, Steven J.}%
\section{Matching model with heterogeneous jobs}\label{Sec_matching_heterojobs}%
Acemoglu (1997), Bertola and Caballero (1994), and Davis (1995)
explore matching models
where heterogeneity on the job supply side must be negotiated through a
single matching function, which gives rise to additional externalities.
Here, we will study an infinite horizon version of Davis's model,
which assumes that heterogeneous jobs are
created in the same labor market with only one matching
function. We extend our basic matching framework as follows:
Let there be $I$ types of jobs. A filled job of type $i$
produces $y^i$. The cost in each period of creating a measure $v^i$
of vacancies of type $i$ is given by a strictly convex upward-sloping cost
schedule, $C^i(v^i)$.
In a decentralized equilibrium, we will assume that vacancies are
competitively supplied at a price equal to the marginal cost of
creating an additional vacancy, ${C^i}'(v^i)$, and we
retain the assumption that firms employ at most one worker.
Another implicit assumption is that $\{y^i, C^i(\cdot)\}$ are such
that all types of jobs are created in both the decentralized steady
state and the socially optimal steady state.


\subsection{A steady state}
In a steady state, there will be a time-invariant distribution of
employment and vacancies across types of jobs. Let $\eta^i$
be the fraction of type $i$ jobs
among all vacancies. With respect to a job of type $i$,
the value of an employed worker, $E^i$, and a firm's values
of a filled job, $J^i$, and a vacancy, $V^i$, are given by
$$\EQNalign{
J^i\;&=\; y^i\,-\, w^i\,+\,\beta \bigl[ sV^i \,+\, (1-s)J^i \bigr] \,,
                                                               \EQN Ji_eq \cr
V^i\;&=\; -{C^i}'(v^i)\,+\, \beta \bigl\{ q(\theta) J^i \,+\,
\bigl[1 - q(\theta)\bigr] V^i     \bigr\}        \,,  \EQN Vi_eq \cr
E^i\;&=\; w^i\,+\, \beta \left[s U \,+\, (1-s) E^i \right] \,, \EQN Ei_eq \cr
U\;&=\; z\,+\, \beta \Bigl\{\theta q(\theta) \sum_j \eta^j E^j \,+\,
\bigl[1 - \theta q(\theta) \bigr] U \Bigr\} \,,             \EQN Ui_eq \cr}
$$
where the value of being unemployed, $U$, reflects that the
probabilities of being matched with different types of jobs are
equal to the fractions of these jobs among all vacancies.

After imposing a zero-profit condition on all types of vacancies,
we arrive at the analogue to equation \Ep{wage1},
$$
w^i\;=\; y^i \,-\, {r+s \over q(\theta)} {C^i}'(v^i) \,.         \EQN wage1i
$$
As before, Nash bargaining can be shown to give rise to still another
characterization of the wage,
$$
w^i\;=\; z \,+\, \phi\Bigl[y^i-z\,+\,\theta
                \sum_j \eta^j {C^j}'(v^j)  \Bigr]\,,    \EQN wage3i
$$
which should be compared to equation \Ep{wage3}. After
setting the two wage expressions \Ep{wage1i} and \Ep{wage3i} equal to
each other, we arrive at a set of equilibrium conditions for
the steady-state distribution of vacancies and
the labor market tightness,
$$
y^i-z \;=\; {\displaystyle r\,+\,s\,+\,\phi\,\theta\,q(\theta)
              {\sum\nolimits_j \eta^j {C^j}'(v^j)
              \over  {C^i}'(v^i) }
             \over \displaystyle  (1-\phi) q(\theta) } \; {C^i}'(v^i) \,.
                                                           \EQN equilii
$$


When we next turn to the efficient allocation in the current setting,
it will be useful to manipulate equation \Ep{equilii}
in two ways. First, subtract from this equilibrium expression for job $i$
the corresponding expression for job $j$,
$$
y^i-y^j \;=\; {r\,+\,s\, \over  (1-\phi) q(\theta) } \,
            \left[ {C^i}'(v^i) \,-\,  {C^j}'(v^j) \right] \,.   \EQN equilii2
$$
Second, multiply equation \Ep{equilii} by $v^i$ and sum over all types of
jobs,
$$
\sum_i v^i (y^i-z) \;=\; {r\,+\,s\,+\,\phi\,\theta\,q(\theta)  \over
      (1-\phi) q(\theta) } \; \sum_i v^i {C^i}'(v^i) \,. \EQN equilii3
$$ (This expression is reached after invoking $\eta^j \equiv v^j / \sum_h v^h$, and an interchange of summation signs.)


\subsection{Welfare analysis}

The social planner's optimization problem becomes
$$\EQNalign{
& \max_{\{v_t^i,n_{t+1}^i\}{}_{t,i}} \;\; \sum_{t=0}^\infty \beta^t
     \Bigl[\sum_j y^j n_t^j \,+\, z\Bigl(1-\sum_j n_t^j\Bigr)
          \,-\, \sum_j C^j(v_t^j) \Bigr],      \EQN soc_obji;a  \cr
\noalign{\vskip.2cm}
%%%\noalign{\hbox{subject to }}
&\hbox{subject to }\ \
 n_{t+1}^i \;=\; (1-s)n_t^i \,+\,
      q\left({\sum_j v_t^j \over 1-\sum_j n_t^j}\right) v_t^i \,,
\hskip.3cm          \forall i, \; t\geq 0, \hskip1.5cm  \EQN soc_obji;b  \cr
&\hbox{given \ \ } \left\{n_0^i\right\}_i\,.       \EQN soc_obji;c  \cr}
$$
The first-order conditions with respect to $v_t^i$ and $n_{t+1}^i$,
respectively, are
$$\EQNalign{
&-\beta^t {C^i}'(v_t^i) \,+\, \lambda_t^i q(\theta_t) \,+\,
              { q'\left(\theta_t\right)
               \over 1 - \sum\nolimits_j n_t^j } \,
             \sum_j \lambda_{t}^j \, v_t^j \;=\; 0 \,,     \EQN foc_vi \cr
\noalign{\vskip.2cm}
&-\lambda_t^i \,+\, \beta^{t+1} (y^i-z) \,+\, \lambda_{t+1}^i (1-s) \cr
\noalign{\vskip.2cm}
&\hskip 2.5cm   \,+\, {q'\left(\theta_{t+1}\right)\theta_{t+1}
            \over 1 - \sum\nolimits_j n_{t+1}^j }\,
            \sum_j \lambda_{t+1}^j \, v_{t+1}^j \;=\; 0\,. \hskip1cm \EQN foc_ni  \cr}
$$
To explore the efficient relative allocation of different types of
jobs, we subtract from equation \Ep{foc_vi} the corresponding expression
for job $j$,
$$
\lambda_t^i \,-\, \lambda_t^j \;=\; {\beta^t \left[{C^i}'(v_t^i)\,-\,
                   {C^j}'(v_t^j) \right] \over q(\theta_t) }\,.  \EQN lamb_i
$$
Next, we do the same computation for equation \Ep{foc_ni} and substitute
equation \Ep{lamb_i} into the resulting expression evaluated at a
stationary solution,
$$
y^i-y^j \;=\; {r\,+\,s\, \over  q(\theta) } \,
            \left[ {C^i}'(v^i) \,-\,  {C^j}'(v^j) \right] \,.   \EQN soc_opti
$$
A comparison of equation \Ep{soc_opti} to equation \Ep{equilii2}
suggests that there will be an efficient {\it relative\/} supply
of different types of jobs in a
decentralized equilibrium only if $\phi=0$. For any strictly positive
$\phi$, the difference in marginal costs of creating vacancies for
two different jobs is smaller in the decentralized equilibrium as
compared to the social optimum; that is, the decentralized equilibrium
displays smaller differences in the distribution of vacancies across
types of jobs. In other words, the decentralized equilibrium creates
relatively too many ``bad jobs'' with low $y$'s or, equivalently,
relatively too few ``good jobs'' with high $y$'s.
The inefficiency in the mix of jobs disappears if the workers have no
bargaining power so that the firms reap all the benefits of upgrading
jobs.\NFootnote{The interpretation that $\phi=0$, which is needed to attain an
efficient relative supply of different types of jobs in a decentralized
equilibrium, can be made precise in the following way:  Let $v$ and $n$
denote any sustainable stationary values of the economy's measure of
total vacancies and employment rate, that is,
$s n = q\left({v \over1-n}\right) v$.  Solve the social planner's
optimization problem in equation \Ep{soc_obji} subject to the
additional constraints
%$$
%\sum_i v_t^i = v\,, \hskip.25cm  \sum_i n_{t+1}^i = n\,,\hskip.25cm
%\forall t\geq 0,
%$$
$
\sum_i v_t^i = v\,,   \sum_i n_{t+1}^i = n
\forall t\geq 0,
$
given $\{n_0^i: \sum_i n_{0}^i = n \}$. After applying the
steps in the main text to the first-order conditions of this problem,
we arrive at the very same expression \Ep{soc_opti}.
Thus, if $\{v, n\}$
is taken to be the steady-state outcome of the decentralized economy,
it follows that equilibrium condition \Ep{equilii2} satisfies
efficiency condition \Ep{soc_opti} when $\phi = 0$.}
But from before we know that workers' bargaining power is
essential to correct an excess supply of the {\it total\/} number of
vacancies.

To investigate the efficiency with respect to the total number of
vacancies, multiply equation \Ep{foc_vi} by $v^i$ and sum over all types of
jobs,
$$
\sum_i \lambda_t^i \, v_t^i \;=\; {\beta^t \sum\nolimits_i v_t^i
 {C^i}'(v_t^i) \over q(\theta_t) \,+\, q'(\theta_t) \theta_t} \,. \EQN lamb_i2
$$
Next, we do the same computation for equation \Ep{foc_ni} and substitute
equation \Ep{lamb_i2} into the resulting expression evaluated at a stationary
solution,
$$
\sum_i v^i (y^i-z) \;=\; {r\,+\,s\,+\,\alpha\,\theta\,q(\theta)
    \over  (1-\alpha) q(\theta) } \; \sum_i v^i {C^i}'(v^i) \,. \EQN soc_opti2
$$
A comparison of equations \Ep{soc_opti2} and \Ep{equilii3} suggests
the earlier result from the basic matching model; that is, an
efficient {\it total\/}
supply of jobs in a decentralized equilibrium calls for
$\phi = \alpha$.\NFootnote{The suggestion that $\phi = \alpha$, which is
needed to attain an efficient total supply of jobs in a decentralized
equilibrium, can be made precise in the following way. Suppose that the
social planner is forever constrained to some arbitrary relative
 distribution, $\{\gamma^i\}$, of types of jobs and vacancies,
where $\gamma^i\geq 0$ and
$\sum_i \gamma^i = 1$. The constrained social planner's problem is then
given by equations \Ep{soc_obji} subject to the additional restrictions
%$
%v_t^i = \gamma^i v_t \,, \hskip.5cm  n_t^i = \gamma^i n_t \,,\hskip.5cm
%                                                            \forall t\geq 0.
%$
$
v_t^i = \gamma^i v_t \,, n_t^i = \gamma^i n_t
                                                            \forall t\geq 0.
$
That is, the only choice variables are now total vacancies
and employment, $\{v_t,n_{t+1}\}$.  After consolidating the
two first-order conditions with respect to $v_t$ and $n_{t+1}$, and evaluating
at a stationary solution, we obtain
$$
\sum_j y^j \gamma^j - z \;=\; {r\,+\,s\,+\,\alpha\,\theta\,q(\theta)
    \over  (1-\alpha) q(\theta) } \;
           \sum_j \gamma^j {C^j}'(\gamma^j v) \,.
$$
By multiplying both sides by $v$, we arrive at the very same expression
\Ep{soc_opti2}. Thus, if the arbitrary distribution $\{\gamma^i\}$
is taken to be the steady-state outcome of the decentralized economy,
it follows that equilibrium condition \Ep{equilii3} satisfies
efficiency condition \Ep{soc_opti2} when $\phi = \alpha$.}
Hence, Davis (1995) concludes that there is a
fundamental tension between the condition for an efficient mix
of jobs ($\phi=0$) and the standard condition for an efficient
total supply of jobs ($\phi=\alpha$).
\index{matching model!separate markets}
\subsection{The allocating role of wages  I: separate markets}

The last section clearly demonstrates Hosios's (1990) characterization
\auth{Hosios, Arthur J.}%
of the matching framework: ``Though wages in matching-bargaining
models are completely flexible, these wages have nonetheless been
denuded of any allocating or signaling function: this is because
matching takes place before bargaining and so search effectively
precedes wage-setting.'' In Davis's matching model, the problem of
wages having no allocating role is compounded through the existence
of heterogeneous jobs. But as discussed by Davis, this latter
complication would be overcome if different types of jobs were
{\it ex ante\/} sorted into separate markets. Equilibrium movements
of workers across markets would then remove the tension between
the optimal mix and the total supply of jobs. Different wages in
different markets would serve an allocating role for the labor
supply across markets, even though the equilibrium wage in each market
would still be determined through bargaining after matching.

Let us study the outcome when there are such separate markets for
different types of jobs and each worker can  participate in only one
market at a time. The modified model is described by equations
\Ep{Ji_eq}, \Ep{Vi_eq}, and \Ep{Ei_eq} where
the market tightness variable is now also indexed by $i$ and
$\theta^i$, and the new expression for the value of being
unemployed is
$$
U\;=\; z\,+\, \beta \bigl\{\theta^i q(\theta^i) E^i \,+\,
\bigl[1 - \theta^i q(\theta^i) \bigr] U \bigr\} \,.          \EQN Uis_eq
$$
In an equilibrium, an unemployed worker attains the value
$U$ regardless of which labor market he participates in.
The characterization of a steady state proceeds along the
same lines as before.  Let us here reproduce only three
equations that will be helpful in our reasoning. The wage
in market $i$ and the annuity value of an unemployed worker can
be written as
$$\EQNalign{
w^i \;&=\; \phi y^i \,+\, (1-\phi) {r \over 1+r} U\,,  \EQN wageis     \cr
\noalign{\vskip.2cm}
{r \over 1+r} U \;&=\; z\,+\,{\phi \theta^i {C^i}'(v^i) \over 1-\phi} \,,
                                                     \EQN U_valueis  \cr}
$$
and the equilibrium condition for market $i$ becomes
$$
y^i-z \;=\; {r\,+\,s\,+\,\phi\,\theta^i\,q(\theta^i)
    \over  (1-\phi) q(\theta^i) } \, {C^i}'(v^i) \,.          \EQN equiliis
$$

The social planner's objective function is the same as expression
\Ep{soc_obji;a}, but the earlier constraint \Ep{soc_obji;b} is now
replaced by
$$\EQNalign{
n_{t+1}^i \;&=\; (1-s)n_t^i \,+\,
      q\left({v_t^i \over u_t^i}\right) v_t^i \,,
                                                   \cr
1\;&=\;\sum_j \left(u_t^j \,+\, n_t^j \right) \,,  \cr}
$$
where $u_t^i$ is the measure of unemployed workers in
market $i$. At a stationary solution, the first-order
conditions with respect to
$v_t^i$, $u_t^i$, and $n_{t+1}^i$ can be combined to
read
$$
y^i-z \;=\; {r\,+\,s\,+\,\alpha\,\theta^i\,q(\theta^i) \over
                 (1-\alpha) q(\theta^i) } \; {C^i}'(v^i) \,.  \EQN soc_optis
$$
Equations \Ep{equiliis} and \Ep{soc_optis} confirm Davis's
finding that the social optimum can be attained with
$\phi = \alpha$ as long as different
types of jobs are sorted into separate markets.

It is interesting to note that the socially optimal wages, that is,
equation \Ep{wageis} with $\phi=\alpha$, imply wage differences for
{\it ex ante\/} identical workers. Wage differences here are not a sign of any
inefficiency but rather necessary to ensure an optimal
supply and composition of jobs. Workers with higher pay are
compensated for an unemployment spell in their job market, which is
on average longer.

\index{matching model!wage announcements}%
\subsection{The allocating role of wages  II: wage announcements}
According to Moen (1997), we can reinterpret the socially optimal
\auth{Moen, Espen R.}%
steady state in the last section as an economy with competitive wage
announcements instead of wage bargaining with $\phi=\alpha$.
Firms are assumed to
freely choose a wage to announce, and then they join the market
offering this wage without any bargaining. The socially optimal equilibrium
is attained when workers as wage takers choose between labor markets,
so that the value of an unemployed worker is equalized in the economy.

To demonstrate that wage announcements are consistent with the
socially optimal steady state, consider a firm with a vacancy
of type $i$ which is free to choose any wage $\tilde w$ and
then join a market with this wage. A labor market with wage
$\tilde w$ has a market tightness $\tilde \theta$ such that
the value of unemployment is equal to the economy-wide value
$U$. After replacing $w$, $E$, and $\theta$ in equations \Ep{Ei_eq}
and \Ep{Uis_eq} by $\tilde w$, $\tilde E$, and $\tilde \theta$,
we can combine these two expressions to arrive at a
relationship between $\tilde w$ and $\tilde \theta$,
$$
\tilde w \;=\; {r \over 1+r} U \,+\, {r+s \over
\tilde \theta q(\tilde \theta)} \Bigl({r \over 1+r} U \,-\, z\Bigr) \,.
                                                        \EQN tildew
$$
The expected present value of posting a vacancy of type $i$
for one period in market $(\tilde w, \tilde \theta)$ is
$$
-{C^i}'(v^i) \,+\, q(\tilde \theta) \beta \sum_{t=0}^\infty
\beta^t (1-s)^t (y^i-\tilde w) \;=\;
-{C^i}'(v^i) \,+\, q(\tilde \theta) {y^i-\tilde w \over r+s} \,.
$$
After substituting equation \Ep{tildew} into this expression, we can
compute the first-order condition with respect to $\tilde \theta$ as
$$
q'(\tilde \theta) {y^i \over r+s} \,-\, {z \over \tilde \theta^{2}}
\,+\, \left[{1\over\tilde \theta^{2}} \,-\,
 {q'(\tilde \theta) \over r+s} \right]
\,{r \over 1+r}\, U \;=\; 0\,.
$$
Since the socially optimal steady state is our conjectured equilibrium,
we get the economy-wide value $U$ from equation \Ep{U_valueis} with $\phi$
replaced by $\alpha$. The substitution of this value for $U$ into the
first-order condition yields
$$
y^i-z \;=\; {r\,+\,s\,+\,\alpha\,\tilde \theta\,q(\tilde \theta) \over
         (1-\alpha) q(\tilde \theta)  } \;\;
{\theta^i \over \tilde \theta} \; {C^i}'(v^i) \,.  \EQN equiliwa
$$
The right side is strictly decreasing in $\tilde \theta$, so by
equation \Ep{soc_optis} the equality can only hold with
$\tilde \theta=\theta^i$.
We have therefore confirmed that the wages in an optimal steady state
are such that firms would like freely to  announce them and to participate
in the corresponding markets without any wage bargaining. The equal
value of an unemployed worker across markets  ensures  the
participation of workers, who now also act as wage takers.

%%
%%
\def\lege{\raise.3ex\hbox{$>$\kern-.75em\lower1ex\hbox{$<$}}}
%%
%%
\index{lotteries!households}
\index{employment lottery}
\section{Model of employment lotteries}\label{Empl_lottery}%
Consider a labor market without search and matching frictions but
where labor is indivisible. An individual can supply either
one unit of labor or no labor at all, as assumed by Hansen (1985)
and Rogerson (1988).
\auth{Hansen, Gary D.}\auth{Rogerson, Richard}%
In such a setting, employment lotteries can
be welfare enhancing. The argument is best understood in
Rogerson's static model, but with physical capital (and its
implication of diminishing marginal product of labor)
removed from the analysis.  We assume that a single good
can be produced with labor, $n$, as the sole input in a
constant returns to scale technology,
$$
f(n) \,=\, \gamma n\,,  \qquad \hbox{\rm where} \quad \gamma >0\,.  \EQN Elott_tech
$$
In a competitive equilibrium, the equilibrium wage is then equal to
$\gamma$.
Following Hansen and Rogerson, the preferences of an individual
are assumed to be additively separable in consumption, $c$, and labor,
$$
u(c) \,-\, v(n)\,.
$$
The standard assumptions are that both $u$ and $v$ are twice continuously
differentiable and increasing, but while $u$ is strictly concave,
$v$ is convex. However, as pointed out by Rogerson, the precise
properties of the function $v$ are not essential because of the
indivisibility of labor. The only values of $v(n)$ that matter
are $v(0)$ and $v(1)$. Let $v(0)=0$
and $v(1)=A>0$. An individual who can supply one unit of labor
in exchange for $\gamma$ units of goods would then choose to do
so if
$$
u(\gamma) \,-\, A \;\geq\; u(0)\,,
$$
and otherwise the individual would choose not to work.

The proposed allocation might be improved upon by introducing
employment lotteries. That is, each individual chooses
a probability of working, $\psi \in [0, 1]$, and he
trades his stochastic labor earnings in contingency markets.
We assume a continuum of agents so that the
idiosyncratic risks associated with employment lotteries
do not pose any aggregate risk, and the
contingency prices are then determined by the probabilities
of events occurring. (See chapters \use{recurge}, \use{assetpricing1}, and \use{assetpricing2}.)
Let $c_1$ and $c_2$ be the individual's choice of consumption
when working and not working, respectively. The optimization
problem becomes
$$\eqalign{
\max_{c_1, c_2, \psi} \;\;&\psi \left[u(c_1) \,-\, A\right] \;+\;
(1-\psi) \, u(c_2) \,,                                           \cr
\hbox{\rm subject to} \quad &\psi c_1 \,+\, (1-\psi) c_2 \;\leq\;
\psi \gamma \,,                                                  \cr
 \quad &c_1, c_2 \,\geq\,0\,, \quad \psi \in [0, 1]\,.           \cr}
$$
At an interior solution for $\psi$, the first-order
conditions for consumption imply that $c_1=c_2$,
$$\eqalign{
 \psi \, u'(c_1) \;&=\; \psi \, \lambda \,,                      \cr
 (1-\psi) \, u'(c_2) \;&=\; (1-\psi) \, \lambda \,,              \cr}
$$
where $\lambda$ is the multiplier on the budget constraint.
Since there is no harm in also setting $c_1=c_2$ when
$\psi=0$ or $\psi=1$,
the individual's maximization problem can be simplified to
read
$$\eqalign{
\max_{c, \psi} \;\; &u(c) \,-\, \psi \, A  \,,               \cr
 \hbox{\rm subject to} \quad & c \;\leq\;
\psi \gamma \,,
\quad c \,\geq\,0\,, \quad \psi \in [0, 1]\,.           \cr}
\EQN lott_max
$$
The welfare-enhancing potential of employment lotteries is
implicit in the relaxation of the earlier constraint that
$\psi$ could only take on two values, 0 or 1. With
employment lotteries, the marginal rate of transformation
between leisure and consumption is equal to $\gamma$.

The solution to expression \Ep{lott_max} can be characterized
by considering three possible cases:
\medskip
\noindent {\it Case 1.} $A/u'(0) \geq \gamma$.

\noindent {\it Case 2.} $A/u'(0) < \gamma < A/u'(\gamma)$.

\noindent {\it Case 3.} $A/u'(\gamma) \leq \gamma$.
\medskip
\noindent
The introduction of employment lotteries will only affect
individuals' behavior in the second case. In the first case,
if $A/u'(0) \geq \gamma$, it will under all circumstances be
optimal not to work ($\psi=0$), since the marginal value of
leisure in terms of consumption exceeds the marginal rate of
transformation even
at a zero consumption level. In the third case, if
$A/u'(\gamma) \leq \gamma$, it will always be optimal to
work ($\psi=1$), since the marginal value of
leisure falls short of the marginal rate of transformation
when evaluated at the highest feasible consumption per worker.
The second case implies that expression \Ep{lott_max} has an
interior solution with respect to $\psi$ and that employment
lotteries are welfare enhancing.
The optimal value, $\psi^*$, is then given by the first-order
condition
$$
{A  \over u'(\gamma \psi^*)}  \;=\; \gamma \,.
$$
An example of the second case is shown in Figure \Fg{fig192f}. %19.2.
The situation here is such that the individual would choose
to work in the absence of employment lotteries, because the
curve $u(\gamma n) - u(0)$ is above the curve $v(n)$ when
evaluated at $n = 1$. After the introduction of employment
lotteries, the individual chooses the probability $\psi^*$
of working, and his welfare increases by $\triangle_{\psi} -
\triangle$.

%%%%%%%%%%%
% \topinsert
%$$\grafone{fig192.eps,height=2.5in}{{\bf Figure 19.2}
%The optimal employment lottery is given by probability $\psi^*$ of working,
%which increases expected welfare by  $\triangle_{\psi} -
%\triangle$ as compared to working full-time $n=1$.} $$\endinsert
%%%%%%%%%%%%%

\midfigure{fig192f}
\centerline{\epsfxsize=3truein\epsffile{fig192.eps}}
\caption{The optimal employment lottery is given by probability $\psi^*$ of working,
which increases expected welfare by  $\triangle_{\psi} - \triangle$ as compared
to working full-time, $n=1$.}
\infiglist{fig192f}
\endfigure
\index{layoff taxes}


%\section{Lotteries for households versus lotteries for firms}
\section{Lotteries for households versus lotteries for firms}\label{sec:lotteries_firms_households}%
\index{lotteries!firms}%
Prescott (2005b) focuses on the role of nonconvexities at the level of
individual households and production units in the study of business
cycles. On the household side, he envisions indivisibilities in labor
supply like those in the previous section, while on the firm side, he
uses capacity constraints as an example. In spite of these
nonconvexities at the micro level, where all units are assumed to be
infinitesimal, Prescott points out that the
aggregate economy is convex when there are lotteries for
households and lotteries for firms that serve to smooth
the nonconvexities and that thereby deliver both a stand-in
household and a stand-in firm. \auth{Prescott, Edward C.}%

\index{family!one big happy}
Prescott thus recommends an  aggregation theory to rationalize a
 stand-in household that is analogous to  better-known
  aggregation results that underlie the stand-in firm and
the aggregate production
function. He emphasizes the formal similarities
associated with smoothing out nonconvexities
by aggregating over firms, on the one hand,  and aggregating over households,
on the other.
Here we shall argue that
the  economic interpretations that attach
to these two types of aggregation
make the two aggregation theories
very different.\NFootnote{Our argument is based on
Ljungqvist and Sargent's (2005) comment on Prescott (2005b).
\auth{Ljungqvist, Lars}
\auth{Sargent, Thomas J.}}
 Perhaps this
explains why this aggregation method has been applied more to firms
than to households.\NFootnote{Sherwin Rosen often used a lottery model
for the household. Instead of analyzing why a particular individual
chose higher education, Rosen modeled
a family with a continuum of members
that allocates fractions of its members to distinct educational  choices
that involve different numbers of years of schooling. See Ryoo and Rosen
 (2003).\auth{Rosen, Sherwin}\auth{Ryoo, Jaewoo}}



Before turning to a critical comparison of the two aggregation theories,
we first describe a simple technology that will capture the essence
of Prescott's example of nonconvexities on the firm side, while
leaving intact most of our analysis in section \use{Empl_lottery}.

\subsection{An aggregate production function}
We replace our earlier linear technology \Ep{Elott_tech} with
a production technology based on a strictly concave function
$g(\cdot)$,
$$
g(0)=0,\hskip.5cm g(1)=\gamma,\hskip.5cm g'>0,\hskip.5cm g''<0
\hskip.25cm \Longrightarrow \hskip.25cm g'(1)<\gamma.
$$
The point of normalization, $g(1)=\gamma$, will be a focal point in
our analysis, and the strict concavity of $g$ ensures that
$g'(1)<\gamma$.

We assume that there is a continuum of firms  and each firm
has a production technology given by
 $$
f(n) = \cases{ g(n)   & if $n\geq1$, \cr
%%\noalign{\vskip.1cm}
  0      & otherwise, \cr} \EQN Elott_tech2
$$
where $n$ is the amount of labor employed in the firm.
Note that production can  take place only if the firm employs
at least $n\geq1$.\NFootnote{As a clarification,
note that we do not impose an integer constraint on employment in a
firm. It is true that each household in section
\use{Empl_lottery} faces the integer
constraint of  supplying either one unit of labor or no labor at all.
However, a household that chooses to work can very well divide
its one unit of labor across several firms.}
We normalize the measure of households in section \use{Empl_lottery}
to unity and assume that there is a measure of firms
equal to $Z<1$, i.e., there are more households than firms.
Each household owns an equal share of all firms.

In a competitive equilibrium where $N$ households are working, there
will be $\min\{N,Z\}$ firms in operation. Because of the
technological constraint in \Ep{Elott_tech2}, firms
will  choose to operate only if they can employ at least one worker.
Moreover, competitive forces will guarantee that the maximum number
of firms is operating subject to the constraint
that each firm employs at least one worker. This is an implication
of our assumption of decreasing returns to scale in each
firm. The assumption guarantees that it is profitable to operate
many small firms rather than one large firm and that all operating
firms will employ the same amount of labor. Thus, in a competitive
equilibrium, aggregate output as a function of aggregate employment
$N$ is given by
$$
F(N) = \cases{ N \,\gamma   & if $N<Z$, \cr
%%\noalign{\vskip.1cm}
  Z\, g\!\left({N\over Z}\right)& if $N\geq Z$. \cr} \EQN Elott_tech2agg
$$

The aggregate production function in \Ep{Elott_tech2agg} can be
understood as follows. In the first case, only $N$ firms
are active and each  employs one worker. Hence, aggregate
output is equal to $N\,g(1)=N\,\gamma$, and the economy is operating
at less than full capacity because there are idle
firms ($N<Z$). In the second case, all $Z$ firms are active and
each one employs the same amount of labor, $n=N/Z$. Hence,
aggregate output is equal to $Z\,g(N/Z)$, and the economy
is operating at full capacity in the sense that there are no idle
firms. Note that the aggregate production function
in \Ep{Elott_tech2agg} is convex even though individual
firms are subject to a nonconvexity in \Ep{Elott_tech2}.

We now turn to an example of time-varying capacity utilization
to compare and criticize the aggregation theory underlying
the stand-in household in section \use{Empl_lottery} and
the aggregation theory underlying the aggregate production
function in \Ep{Elott_tech2agg}.\NFootnote{There are three
differences between Prescott's (2004) example and ours, but none
 materially effects  our illustration of time-varying
capacity utilization. First, Prescott postulates an additional
production factor, capital, that can also be freely allocated
across firms. Second, Prescott assumes a technology for
creating new firms. Third, Prescott studies
technology shocks while we explore preference shocks.}

\subsection{Time-varying capacity utilization}
\index{capacity utilization}%
We assume that the stand-in household in section \use{Empl_lottery}
is subject to an aggregate preference shock where the disutility of
working can take on two different values, $A\in\{0,\, \bar A\}$.
The parameters satisfy the following restrictions:
$$
{\bar A \over u'(0)} < \gamma, \hskip1cm
g'\!\left({ 1 \over Z}\right) <
 {\bar A \over u'\!\Bigl[Z\,g\!\left({1 \over Z}\right)\Bigr] }. \EQN Elott_para
$$
These parameter restrictions are the analogue to
the parameter restriction in case 2 of section \use{Empl_lottery}.
In particular, restrictions \Ep{Elott_para}
guarantee an interior solution with
respect to the employment lottery when $A=\bar A$, i.e., employment,
will then satisfy $N\in(0,1)$ where $N$ is both the measure of
households working and the probability of an individual household working
since the population of all households is normalized to one.

To see that parameter restrictions \Ep{Elott_para} guarantee an
interior solution with respect to $N$ when $A=\bar A$, we will examine
why neither $N=0$ nor $N=1$ can constitute an equilibrium. First,
we can reject $N=0$ with the following argument. Whenever $N<Z$,
competition among firms drives up the equilibrium
wage to $w=g(1)=\gamma$. That is, firms are then not
a scarce input in production and, therefore, earn no rents. Given
the equilibrium wage $w=\gamma$, the first inequality in
\Ep{Elott_para} states that the stand-in household's
first-order conditions  would be
violated if $N=0$. Second, we can reject an equilibrium
outcome with $N=1$ as follows. At full employment, all
firms are operating and aggregate output is given by
$Z\,g(1/Z)$, which is also equal to per capita output since the
measure of households is normalized to one. Moreover,
according to section \use{Empl_lottery}, households will
trade in contingent claims prior to the outcome of the employment
lottery so that each household's consumption is also given
by $c=Z\,g(1/Z)$. The equilibrium wage at full employment is
given by $w=g'(1/Z)$, i.e, the marginal product of labor in an
individual firm that employs the same amount of labor
as all other firms. Given the  consumption outcome and
wage rate when $N=1$, we can ask if the stand-in household
would indeed choose the probability of working equal to one
that would be required in order for this allocation to constitute
an equilibrium. According to the second inequality in
\Ep{Elott_para}, the answer is no because the stand-in household
would then value a marginal increase in leisure more than the
loss of wage income. Thus, we can conclude that parameter
restrictions \Ep{Elott_para} guarantee an interior solution
with respect to the probability of working when $A=\bar A$.
%Let $\psi^*=N\in(0,1)$ denote the associated equilibrium value
%of the probability of working.

In contrast, when the preference shock is $A=0$, the stand-in
household will inelastically supply one unit of labor since there
is no disutility of working. The economy will then be operating
at full employment with no idle firms. Hence, different realizations
of the preference
shock $A\in\{0,\bar A\}$ will trigger changes in unemployment
and potentially changes in capacity utilization, where the
latter depends on the size of the given measure of firms.
Everything else being equal, a higher $Z$  makes it  more likely that
the preference shock $\bar A$ entails idle firms in an equilibrium.
The households and firms
that are designated to be unemployed and idle, respectively,
are determined by the outcome of lotteries among households and
lotteries among firms. Prescott's assertion that the
aggregation theory for households is the analogue of the
aggregation theory for firms seems to be accurate. So what
is the difference between these two aggregation theories?

An important distinction between firms and households is that
firms have no independent
preferences.  They  serve only as vehicles for
generating rental payments for employed factors and profits for their owners.
When a firm becomes  inactive, the ``firm'' itself does not
care whether it continues or ceases to exist. Our
example of a nonconvex production technology that generates
time-varying capacity utilization
illustrates this point very well. The firms that do not find any workers
stay idle;  that is just as well for those idle firms because
the firms in operation earn zero rents. In short, whether individual
firms operate or remain idle is the end of the
story in the aggregation theory behind the aggregate production
function in \Ep{Elott_tech2agg}. But in the aggregation theory behind the stand-in
household's utility function in \Ep{lott_max}, it is really just the beginning.
Individual households do have preferences and
care about alternative states of the world. So the aggregation theory behind
the stand-in household has an additional aspect that is not present
in the theory that aggregates over firms, namely,
it says how consumption
and leisure are smoothed  across households with the help of an
extensive set of contingent claim markets.
This market arrangement and randomization device  stands at the
center of the employment lottery model.  To us, it seems that
they make the aggregation
theory behind the stand-in household fundamentally different than
the well-known aggregation theory for the firm side.

Next, we explore how models with employment lotteries that are
used to generate unemployed individuals in a frictionless framework
can have very different implications than models embodying
frictional unemployment. In particular,
%we study the employment effects of layoff taxes.
models with employment  lotteries predict effects from layoff
taxes that are
opposite to those in search models.




\section{Employment effects of layoff taxes}

The models of employment determination in this chapter can be
used to address the question, how do layoff taxes affect an
economy's employment? Hopenhayn and Rogerson (1993)
\auth{Hopenhayn, Hugo A.}\auth{Rogerson, Richard}%
apply the model of employment lotteries to this very question
and conclude that a layoff tax would reduce the level
of employment. Mortensen and Pissarides (1999b) reach the opposite
conclusion in a matching model.
We will here examine these results by
scrutinizing the economic forces at work in different
frameworks. The purpose is both to
gain further insights into the workings of our theoretical
models and to learn about possible effects of layoff
taxes.\NFootnote{The analysis is based on Ljungqvist's (2002)
\auth{Ljungqvist, Lars}%
study of layoff taxes in different models
of employment determination.}

Common features of many analyses of layoff taxes
are as follows: The productivity of a job evolves
according to a Markov process, and a sufficiently poor
realization triggers a layoff. The government imposes
a layoff tax $\tau$ on each layoff. The tax revenues
are handed back as equal lump-sum transfers to all
agents, denoted by $T$ per capita.

Here, we assume the simplest possible Markov process
for productivities. A new job has productivity $p_0$.
In all future periods, with
probability $\xi$, the worker keeps the productivity
from last period, and with probability $1-\xi$,
the worker draws a new
productivity from a distribution $G(p)$.

In our numerical examples, the model period is 2 weeks,
and the assumption that $\beta=0.9985$ then implies an
annual real interest rate of 4 percent. The initial productivity
of a new job is $p_0=0.5$, and $G(p)$ is taken to be
a uniform distribution on the unit interval. An
employed worker draws a new productivity on average
once every two years when we set $\xi =0.98$.
\index{employment lottery!layoff taxes}
\index{layoff taxes!employment lottery}
\subsection{A model of employment lotteries with layoff taxes}

In a model of employment lotteries,  a market-clearing wage $w$
  equates the demand and supply of labor. The constant-returns-to-scale
technology implies that this wage is determined from the supply side, as follows.
At the beginning of a period, let $V(p)$ be the firm's value of an
employee with productivity $p$,
$$\eqalign{
V(p) &\;=\; \max \, \Bigl\{p \,-\, w \,+\,
\beta \Bigl[ \xi V(p) \,+\, (1-\xi) \int V(p') \,\;
dG(p')\Bigr]\,, \;\, \cr
&\hskip7.5cm
-\tau \, \Bigr\} \,. \cr}  \EQN lott_V
$$
Given a value of $w$, the solution to this Bellman
equation is a reservation productivity $\bar p$.
If there exists an equilibrium with strictly positive
employment, the equilibrium wage must be such that new
hires exactly break even, so that
$$
V(p_0) \;=\; p_0 \,-\, w \,+\,
\beta \Bigl[ \xi V(p_0) \,+\, (1-\xi) \int V(p') \,
dG(p')\Bigr] \;=\; 0
$$
$$
\Rightarrow  w \;=\; p_0 \,+\, \beta (1-\xi) \tilde V\,, \EQN lott_0
$$
where
$$
\tilde V \;\equiv\; \int V(p') \, dG(p') \,.
$$
To compute $\tilde V$, we first look at the value of
$V(p)$ when $p \geq \bar p$,
$$\EQNalign{
V(p)\Big\vert_{p \geq \bar p} \;&=\; p \,-\, w \,+\,
\beta \left[ \xi V(p) \,+\, (1-\xi) \tilde V \right]        \cr
&=\; p \,-\, w \,+\,
\beta \xi \Bigl\{ p \,-\, w \,+\,
\beta \bigl[ \xi V(p) \,+\, (1-\xi) \tilde V \bigr] \Bigr\}
\,+\, \beta (1-\xi) \tilde V                            \hskip.5cm  \cr
&=\; (1\,+\,\beta \xi)\left[p \,-\, w \,+\, \beta (1-\xi)
\tilde V \right] \,+\, \beta^2 \xi^2 V(p)                   \cr
&=\; {p \,-\, w \,+\, \beta (1-\xi) \tilde V   \over
1\,-\,\beta \xi}  \;=\;
{p \,-\, p_0   \over   1\,-\,\beta \xi}  \,,  \EQN lott_V+ \cr}
$$
where the first equalities are obtained through successive
substitutions of $V(p)$, and the last equality incorporates equation
\Ep{lott_0}. We can then use equation \Ep{lott_V+} to find an expression
for $\tilde V$,
$$\EQNalign{
\tilde V \;&=\;  \int_{-\infty}^{\bar p} -\tau\, dG(p)
\,+\, \int_{\bar p}^{\infty} V(p) \, dG(p)            \cr
&=\; -\tau\, G(\bar p)
\,+\, \int_{\bar p}^\infty
{p \,-\, p_0   \over   1\,-\,\beta \xi}\, dG(p)
                                                \,.   \EQN lott_tildeV \cr}
$$
From Bellman equation \Ep{lott_V}, the reservation productivity satisfies
$$
\bar p \,-\, w \,+\,
\beta \left[ \xi V(\bar p) \,+\, (1-\xi) \tilde V \right]\;=\;\, -\tau\,.
$$
After imposing equation \Ep{lott_0} and $V(\bar p) = -\tau$, we find
$$
\bar p \;=\; p_0 \,-\, (1-\beta \xi) \tau \;\equiv\; \bar p(\tau) \,.    \EQN lott_barp
$$
Equations \Ep{lott_barp}, \Ep{lott_tildeV}, and \Ep{lott_0} can
be used to solve for the equilibrium wage $w=w(\tau)$.


In a stationary equilibrium, let $\mu $ be the mass of new jobs
created in every period. The mass of jobs with productivity $p_0$
that have not yet experienced a new productivity draw can then be
expressed as
$$
\mu  \sum_{i=0}^{\infty} \xi^i = {\mu  \over 1-\xi}\,,\EQN lott_firm1
$$
and the mass of jobs that have experienced a new productivity draw and
are still operating is given by
$$\EQNalign{
\sum_{i=0}^{\infty} \xi^i \mu  (1-\xi) \left[1 - G(\bar p)\right]
&\sum_{j=0}^{\infty} \bigl\{\xi + (1-\xi)\left[1-G(\bar p)\right]\bigr\}^j \cr
&\hskip.5cm
\;=\; {\mu  \over 1- \xi} {1-G(\bar p) \over G(\bar p)} \,.  \EQN lott_firm2 \cr}
$$
After
equating the sum
of these two kinds of jobs
to $N$ (which
we use to denote the total mass of all jobs), we get the
following steady-state relationship:
$$
\mu  = N  G(\bar p) (1-\xi).  \EQN lott_muPsi
$$
The firms generate aggregate profits $\Pi$. These profits are here computed
 gross of aggregate layoff taxes, i.e., $\Pi+T$. (Recall that the
government hands back layoff tax revenues to the representative
agent as a lump-sum transfer $T$.) Using the masses of jobs in expressions
\Ep{lott_firm1} and \Ep{lott_firm2}, we have
$$\EQNalign{
\Pi \,+\, T \;&=\; {\mu  \over 1-\xi} \left(p_0 - w \right)  +
{\mu  \over 1- \xi} {1-G(\bar p) \over G(\bar p)}
\int_{\bar p}^{\infty}{p-w  \over 1-G(\bar p)} \,dG(p) \cr
&=\; N  \left[ G(\bar p)  \left(p_0 - w \right)   +
\int_{\bar p}^{\infty}\left( p-w  \right) \,dG(p) \right]\,,
                           \EQN lott_profit  \cr}
$$
where the last inequality invokes relationship \Ep{lott_muPsi}.


In a stationary equilibrium with wage $w$ and a gross interest rate $1/\beta$, the
representative agent's optimization problem reduces to the static problem:
$$\eqalign{
\max_{c, \psi} \;\; &u(c) \,-\, \psi \, A  \,,               \cr
 \hbox{\rm subject to} \quad & c \;\leq\;
\psi w \,+\, \Pi \,+\,T\,,
\quad c \,\geq\,0\,, \quad \psi \in [0, 1]\,,           \cr}
\EQN lott_max2
$$
where the profits $\Pi$ and the lump-sum transfer $T$ are taken
as given by the agents.\NFootnote{As above, we are normalizing
the measure of agents to unity so that aggregate variables also
represent per capita outcomes.}
We let
$\psi^*$ denote the optimally chosen probability of working.  It is equal to $N$ in an equilibrium
and the corresponding
optimal consumption level is
$$ c^* \;=\; N w \,+\, \Pi \,+\, T \;=\;
 N  \left[ G(\bar p) p_0    +
\int_{\bar p}^{\infty} p \,dG(p) \right]\,,
                           \EQN lott_consumption
$$
where we have invoked expression \Ep{lott_profit}. Hence,
the steady-state expected
lifetime utility of an agent before seeing the outcome of any
employment lottery is equal to
$$
\sum_{t=0}^\infty \beta^t \Bigl[ u\left( c^* \right)\,-\, \psi^* A\Bigr]\,.
$$

Following Hopenhayn and Rogerson (1993), the preference specification
is $u(c) = \hbox{\rm log}(c)$ and the disutility of work is calibrated
to match an employment to population ratio equal to $0.6$, which leads
us to choose $A=1.6$. Figures \Fg{hsm_pic1f}--\Fg{hsm_pic5f} %19.3--19.7
show how equilibrium
outcomes vary with the layoff tax. The curves labelled $L$ pertain to
the model of employment lotteries. As derived in equation
\Ep{lott_barp}, the reservation
productivity in Figure \Fg{hsm_pic1f} %19.3
  falls when it becomes more costly to lay
off workers. Figure \Fg{hsm_pic3f} %19.4
 shows how the decrease in number of layoffs
is outweighed by the higher tax per layoff, so total layoff taxes
as a fraction of GNP increase over almost the whole range.
Figure \Fg{hsm_pic4f} %19.5
 reveals changing job prospects, where the probability
of working falls with a higher layoff tax (which is  equivalent
to falling employment in a model of employment lotteries).
The welfare loss associated with a layoff tax is depicted in
Figure \Fg{lay_p5f} %19.6
as the amount of consumption that an agent would be
willing to give up in exchange for a steady state
with no layoff tax,            %rid the economy of the layoff tax,
and the ``willingness to pay'' is expressed as a fraction of per
capita consumption at a zero layoff tax.



Figure \Fg{hsm_pic5f} %19.7
reproduces Hopenhayn and Rogerson's (1993) result
that employment falls with a higher layoff tax (except at the
highest layoff taxes). Intuitively, from a private perspective,
a higher layoff tax
is like a deterioration in
the production technology; the optimal change in the agents'
employment lotteries will therefore depend on the strength of
the substitution effect versus the income effect. The income
effect is largely mitigated by the government's lump-sum transfer
of the tax revenues back  to the private economy. Thus,
layoff taxes in models of employment lotteries have strong
negative employment implications that are caused by substituting
leisure for work. Formally, the
logarithmic preference specification gives rise to an optimal
choice of the probability of working, which is equal to
the employment outcome, as given by
$$
\psi^* \;=\; {1 \over A} \,-\, {T + \Pi \over w} \,. \EQN lott_psiopt
$$
The precise employment effect here is driven by profit flows
from firms gross of layoff taxes expressed in terms of
the wage rate. Since these profits are to a large extent
generated in order to pay for firms' future layoff taxes, a
higher layoff tax tends to increase the accumulation of such
funds with a corresponding negative effect on the optimal
choice of employment.


\vskip.5cm
\medskip
\noindent{\bf Negative employment effect of layoff taxes, when evaluated at $\tau=0$}
\smallskip
\noindent
Under the assumption that $p_0=1$, i.e., the initial productivity of a new job is
equal to the upper support of the uniform distribution $G(p)$ on
the unit interval $[0,1]$, we will show that the derivative of equilibrium
employment is strictly negative with
respect to the layoff tax when evaluated at $\tau=0$.

Expressions \Ep{lott_tildeV} and \Ep{lott_profit} can then
be evaluated as follows:
$$\EQNalign{
\tilde V \;&=\;  -\tau\, \bar p + \left[ {1+\bar p \over 2} -1 \right]
{1-\bar p \over 1 - \beta \xi}\,,  \EQN lott_tildeV_uni \cr
\noalign{\hbox{\rm and}}
\Pi \,+\, T \;&=\;
N \left[ \bar p + (1-\bar p) {1+\bar p \over 2} - w \right]\,.
                           \EQN lott_profit_uni  \cr}
$$
From equations \Ep{lott_0} and \Ep{lott_tildeV_uni},
$$
w \;=\; 1 \,+\, \beta (1-\xi) \left[ - \tau \bar p -
{(1-\bar p)^2 \over 2 (1 - \beta \xi)   } \right]\,,
$$
and after substituting for $\bar p$ from \Ep{lott_barp}
$$
w \;=\; 1 \,-\, \beta (1-\xi) \tau \left[1 -
                {(1 - \beta \xi) \tau \over 2} \right] \equiv w(\tau) \,.
\EQN lott_0_uni
$$

By substituting \Ep{lott_profit_uni} into \Ep{lott_psiopt} and using
expressions \Ep{lott_barp} and \Ep{lott_0_uni}, we
arrive at an equilibrium expression for $N$,
$$
N(\tau) = { 2 w(\tau) \over \gamma \left[ 2 \bar p(\tau) + 1
- \bar p(\tau)^2 \right] }
$$
with its derivative
$$
{d\, N(\tau)  \over d\, \tau \hfill}=
{-2 \beta (1-\xi) \bar p(\tau)  \left[ 2 \bar p(\tau) + 1 - \bar p(\tau)^2 \right]
+ 4 (1-\beta \xi) \left[1-\bar p(\tau)\right] w(\tau) \over
\gamma \left[ 2 \bar p(\tau) + 1 -  \bar p(\tau)^2 \right]^2 }\,.
$$
Evaluating the derivative at $\tau=0$, where $\bar p(0) = p_0 = 1$, we have
$$
{d\, N(\tau) \over d \,\tau \hfill}\Bigg|_{\tau=0} =
{-\beta (1-\xi) \over \gamma } < 0.
$$
This states that in general equilibrium, employment falls
in response to the introduction
of a layoff tax in our employment lottery model.



%%%%%%%%%%%%%
%$$\grafone{hsm_pic1.ps,height=2.5in}{{\bf Figure 19.3} Reservation
%productivity for different values of the layoff tax.}$$
%%%%%%%%%%%%%%%%%%

\midfigure{hsm_pic1f}
\centerline{\epsfxsize=3truein\epsffile{hsm_pic1.ps}}
\caption{Reservation productivity for different values of the layoff tax.}
\infiglist{hsm_pic1f}
\endfigure

\vfill\eject

%%%%%%%%%%%%
%$$\grafone{hsm_pic3.ps,height=2.25in}{{\bf Figure 19.4} Total layoff taxes
%as a fraction of GNP for different values of the layoff tax.}$$
%%%%%%%%%%%%%%%%

\midfigure{hsm_pic3f}
\centerline{\epsfxsize=3truein\epsffile{hsm_pic3.ps}}
\caption{Total layoff taxes as a fraction of GNP for different values of the
layoff tax.}
\infiglist{hsm_pic3f}
\endfigure

%%%%%%%%%%%%
%$$\grafone{hsm_pic4.ps,height=2.25in}{{\bf Figure 19.5}  Probability
%of working in the model with employment lotteries and probability
%of finding a job within 10 weeks in the other models, for different
%values of the layoff tax.} $$
%%%%%%%%%%%%%%%%%

\midfigure{hsm_pic4f}
\centerline{\epsfxsize=3truein\epsffile{hsm_pic4.ps}}
\caption{Probability of working in the model with employment lotteries
and probability of finding a job within 10 weeks in the other models, for
different values of the layoff tax.}
\infiglist{hsm_pic4f}
\endfigure

%%%%%%%%%%%%%%%%%
%$$\grafone{lay_p5.ps,height=2.25in}{{\bf Figure 19.6} A job finder's
%welfare loss due to the presence of a layoff tax, computed
%as a fraction of per capita consumption at a zero layoff tax.} $$
%%%%%%%%%%%

\midfigure{lay_p5f}
\centerline{\epsfxsize=3truein\epsffile{lay_p5.ps}}
\caption{A job finder's welfare loss due to the presence of a layoff tax,
computed as a fraction of per capita consumption at a zero layoff tax.}
\infiglist{lay_p5f}
\endfigure

%%%%%%%%%%%%%%%%%
%$$\grafone{hsm_pic5.ps,height=2.25in}{{\bf Figure 19.7}  Employment
%index for different values of the layoff tax. The index is equal
%to one at a zero layoff tax.} $$
%%%%%%%%%%%%%%%

\midfigure{hsm_pic5f}
\centerline{\epsfxsize=3truein\epsffile{hsm_pic5.ps}}
\caption{Employment index for different values of the layoff tax.  The index
is equal to $1$ at a zero layoff tax.}
\infiglist{hsm_pic5f}
\endfigure

\ \vfil \eject
\ \vfil \eject


\index{island model!layoff taxes} \index{layoff taxes!island model}
\subsection{An island model with layoff taxes}
To stay with the described technology in an island framework, let each
job represent a separate island, and an agent moving to a new island
experiences productivity $p_0$.  We retain the feature that every agent
bears the direct consequences of his decisions.  He receives his marginal
product $p$ when working and incurs the layoff tax $\tau$ if leaving his
island. The Bellman equation can then be
written as
$$\EQNalign{
V(p) \;=\; \max \, \Biggl\{p \,-\, z \,+\,
\beta &\left[ \xi V(p) \,+\, (1-\xi) \int V(p') \, dG(p')\right]\,,   \cr
&\qquad \qquad \;\, -\tau \,+\,\beta^T\,V(p_0) \, \Biggr\} \,,
                                                       \EQN island_V \cr}
$$
where $z$ is the forgone utility of leisure when working and $T$ is
the number of periods it takes to move to another island.\NFootnote{Note
that we have left out the lump-sum transfer from the government because
it does not affect the optimization problem.}
 The solution
to this equation is a reservation productivity $\bar p$.

If there
exists an equilibrium with agents working, we must have
$$\EQNalign{
&V(p_0) \;=\; p_0 \,-\, z \,+\,
\beta \left[ \xi V(p_0) \,+\, (1-\xi) \int V(p') \, dG(p')\right]\,   \cr
&\Longrightarrow \quad
\beta (1-\xi) \tilde V \;=\; (1 - \beta \xi) V(p_0) \,+\, z \,-\, p_0\,,
                                                         \EQN island_V0   \cr}
$$
where
$$
\tilde V \;\equiv\; \int V(p') \, dG(p') \,.
$$
If the equilibrium entails agents moving between islands, the
reservation productivity, by equation \Ep{island_V}, satisfies
$$
 \bar p \,-\, z \,+\,
\beta \left[ \xi V(\bar p) \,+\, (1-\xi) \tilde V \right]  \;=\;
 -\tau \,+\,\beta^T\,V(p_0)  \,,
$$
and, after imposing equation \Ep{island_V0} and
$V(\bar p)= -\tau \,+\,\beta^T\,V(p_0)$,
$$
\bar p \;=\; p_0 \,-\, (1-\beta \xi )\left[ \tau \,+\,
(1- \beta^T) \, V(p_0) \right]\,.                             \EQN island_barp
$$
Note that if agents could move instantaneously between islands,
$T=0$, the reservation productivity would be the same as in the
model of employment lotteries, given by equation \Ep{lott_barp}.

A higher layoff tax  also reduces the reservation productivity
in the island model; that is, an increase in $\tau$ outweighs the
drop in the second term in square brackets in equation \Ep{island_barp}.
For a formal proof, let us make explicit that
the value function and the reservation
productivity are functions of the layoff tax, $V(p; \tau)$ and
$\bar p(\tau)$. Consider two layoff taxes, $\tau$ and $\tau'$, such
that $\tau'>\tau\geq0$, and denote the difference $\triangle \tau =
\tau'-\tau$. We can then construct a lower bound for $V(p; \tau')$ in
terms of $V(p; \tau)$. In response to the higher layoff tax
$\tau'$, the agent can always keep his decision rule associated
with $V(p; \tau)$ and an upper bound for his extra layoff tax
payments would be that he paid $\triangle \tau$ in the current
period and every $T$th period from there on,
$$
V(p;\tau') \;>\; V(p,\tau) \,-\,
\sum_{i=0}^\infty \beta^{iT} \triangle \tau \,,            \EQN island_bound
$$
where the strict inequality follows from the fact that it cannot be
optimal to constantly move. In addition, the agent might be able to select
a better decision rule than the one associated with $\tau$. In fact,
the reservation productivity must fall in response to a higher layoff
tax whenever there is an interior solution with respect to $\bar p$,
as given by equation \Ep{island_barp}. By using equations \Ep{island_barp}
and \Ep{island_bound}, we have
$$\EQNalign{
\bar p(\tau') \,-\, \bar p(\tau) \;&=\; - (1-\beta \xi )
\bigl\{ \triangle \tau  \,+\,
(1- \beta^T) \bigl[V(p_0; \tau')-V(p_0; \tau) \bigl] \bigr\}   \cr
&<\;- (1-\beta \xi )
\Bigl[ \triangle \tau  \,-\,
(1- \beta^T) \sum_{i=0}^\infty \beta^{iT} \triangle \tau \Bigr] \;=\;0\,. \cr}
$$

The numerical illustration in Figures \Fg{hsm_pic1f} through \Fg{hsm_pic5f}  %%19.3--19.7
is based
on a value of leisure $z=0.25$ and a length of transition between
jobs $T=7$; that is, unemployment spells last 14 weeks. The curves
that pertain to the island model are labeled $S$. The effects
of layoff taxes on the reservation productivity, the economy's
total layoff taxes, and the welfare of a recent job finder are
all similar to the outcomes in the model of employment lotteries.
The sharp difference appears in Figure \Fg{hsm_pic5f} %%19.7
depicting the effect
on the economy's employment. In the island model where agents are
left to fend for themselves, a lower
reservation productivity is synonymous with both less labor
reallocation and lower unemployment. Lower
unemployment is thus attained at the cost of a less efficient
labor allocation.

Mobility costs also cause employment  to rise in the general
version of the island model, as mentioned by Lucas and Prescott
(1974, p. 205).
\auth{Lucas, Robert E., Jr.}\auth{Prescott, Edward C.}%
 For a given expected value of arriving on a new
island $v_u$, the value function in equation \Ep{value2} is
replaced by
$$\EQNalign{v(\theta, x) \;=\; \max \Bigl\{&\beta v_u\,-\,\tau\,,\;\;
 \cr
\,&\theta \, f'(x) \,+\,
  \min \bigl\{ \beta v_u\,,\; \beta E [v(\theta',x) \vert \theta ]
      \bigr\} \Bigr\}\,, \hskip.5cm \EQN value2_lay \cr}
$$
which lies below equation \Ep{value2}, but with a drop of at
most $\tau$. Similarly, equation \Ep{value2aa} changes to
$$\theta \, f'(n) \,+\, \beta E\left[v(\theta',n) \vert \theta \right]
      \;=\; \beta v_u \,-\, \tau \,.             \EQN Case1_lay
$$
An implication here is that $x^+(\theta)$ rises in response to a higher
layoff tax. The unchanged expression \Ep{value2bb} means that $x^-(\theta)$
falls as a result of the preceding drop in the value function. In
other words, the range of an island's employment levels characterized by no
labor movements is enlarged. This effect will shift the curve in
Figure \Fg{fig191f}  %%19.1
downward and decrease the equilibrium value of $v_u$. Less
labor reallocation maps directly into a lower unemployment
rate.

\index{matching model!layoff taxes}
\index{layoff taxes!matching model}
\subsection{A matching model with layoff taxes}\label{sec:layofftaxmatch}%
We now modify the matching model to incorporate
a layoff tax, and the exogenous destruction of jobs is replaced
by the described Markov process for a job's productivity.  A job
is now endogenously destroyed when the outside option, taking the
layoff tax into account, is higher than the value of maintaining
the match. The match surplus, $S_i(p)$, is a function of the
job's current productivity $p$ and can be expressed as
$$\EQNalign{
  S_i(p)+U_i \;=\;\max \Biggl\{p
    +\beta & \left[ \xi S_i(p) \,+\, (1-\xi) \int S_i(p') \,
          dG(p') \,+\, U_i \right]  , \cr
              & \hskip3.5cm U_i \,-\, \tau \Biggr\}\,,  \EQN lay_S \cr}
$$
where $U_i$ is once again the agent's outside option, that is, the
value of being unemployed. Both $S_i(p)$ and $U_i$ are
indexed by $i$, since we will explore the implications of two
alternative specifications of the Nash product, $i\in\{a,b\}$,
$$\EQNalign{
& \bigl[E_a(p) - U_a \bigr]^\phi J_a(p)^{1-\phi}\,,
                                                             \EQN Nash_a \cr
& \bigl[E_b(p) - U_b \bigr]^\phi
       \bigl[J_b(p) + \tau \bigr]^{1-\phi} \,.        \EQN Nash_b \cr  }
$$
Specification \Ep{Nash_a} leads to the usual result that the worker
receives a fraction $\phi$ of the match surplus, while
the firm gets the remaining fraction $(1-\phi)$,
$$  E_a(p) - U_a \,=\, \phi S_a(p)  \quad {\rm and}
     \quad J_a(p) \,=\, (1-\phi) S_a(p) \,.  \EQN split_a
$$
The alternative specification \Ep{Nash_b} adopts the assumption of
Saint-Paul (1995)
\auth{Saint-Paul, Gilles}%
that the layoff cost changes the firm's threat point from 0 to
$-\tau$, and thereby increases the worker's relative share of
the match surplus.
Solving for the corresponding surplus-sharing rules, we get
$$\eqalign{  E_b(p) - U_b \,&=\, \phi \bigl(S_b(p) + \tau \bigr)\,, \cr
 J_b(p) \,&=\, (1-\phi) S_b(p) - \phi \tau  \,.
 \cr} \EQN split_b
$$

The worker's continuation value outside of the match associated with
Nash product \Ep{Nash_a} or \Ep{Nash_b}, respectively, is
$$\EQNalign{
U_a\;&=\; z\,+\, \beta \bigl[\theta q(\theta) \phi S_a(p_0) \,+\,
 U_a \bigr] \,,           \EQN U_a                                   \cr
\noalign{\vskip.2cm}
U_b\;&=\; z\,+\, \beta \bigl\{ \theta q(\theta) \phi \bigl[ S_b(p_0)
\,+\, \tau\bigr] \,+\,  U_b \bigr\} \,.           \EQN U_b \cr}
$$
The equilibrium conditions that firms post vacancies until the expected
profits are driven down to zero become
$$\EQNalign{
(1-\phi) S_a(p_0) \;&=\; {c \over \beta q(\theta)} \,,
                                                         \EQN noprofit_a \cr
(1-\phi) S_b(p_0)\,-\, \phi \tau \;&=\; {c \over \beta q(\theta)} \,,
                                                         \EQN noprofit_b \cr}
$$
for Nash product \Ep{Nash_a} or \Ep{Nash_b}, respectively.

In the calibration, we choose a matching function $M(u,v)=0.01 u^{0.5}
v^{0.5}$, a worker's bargaining strength $\phi=0.5$, and the same value
of leisure as in the island model, $z=0.25$. Qualitatively, the results
in Figures \Fg{hsm_pic1f} through \Fg{lay_p5f}  %%19.3--19.6   XXXXXX
are the same across all the models
considered here. The curve labeled $Ma$ pertains to the matching model
in which the workers' relative share of the match surplus is constant,
while the curve $Mb$ refers to the model in which the share is positively
related to the layoff tax. However, matching model $Mb$ does stand
out. Its reservation productivity plummets in
response to the layoff tax in Figure \Fg{hsm_pic1f},  %%19.3
and is close to zero
at $\tau=11$. A zero reservation productivity means that labor
reallocation comes to a halt, and the economy's tax revenues
fall to zero in Figure \Fg{hsm_pic3f}.   %%19.4
The more dramatic outcomes under $Mb$
have to do with layoff taxes increasing workers' relative
share of the match surplus.
The equilibrium condition
\Ep{noprofit_b} requiring that firms finance incurred vacancy
costs with retained earnings from the matches becomes exceedingly
difficult to satisfy when a higher layoff tax erodes the fraction
of match surpluses going to firms.
Firms can  break even only if the expected time to fill a vacancy
is cut dramatically; that is, there has to be a large number of
unemployed workers for each posted vacancy. This equilibrium outcome
is reflected in the sharply falling probability of a worker
finding a job within 10 weeks in Figure \Fg{hsm_pic4f}.     %%19.5
As a result, there
are larger welfare costs in model $Mb$, as shown by the welfare
loss of a job finder in Figure \Fg{lay_p5f}.   %%19.6
The welfare loss of
an unemployed agent is even larger in model $Mb$, whereas the
differences between employed and unemployed agents in the three other
model specifications are negligible (not shown in any figure).

In Figure \Fg{hsm_pic5f},   %%19.7
matching model $Ma$ looks very much like
the island model with increasing employment, and matching model
$Mb$ displays initially falling employment, similar to the model of
employment lotteries. The later sharp reversal of the employment
effect in the $Mb$ model is driven by our choice of a Markov
process with rather little persistence. (For a comparison, see
Ljungqvist, 2002, who explores Markov formulations with more
persistence.)

Mortensen and Pissarides (1999a)
\auth{Mortensen, Dale T.} \auth{Pissarides, Christopher}%
 propose still another bargaining specification where
expression \Ep{Nash_a} is the Nash product when a worker and a firm
meet for the first time, while the Nash product in expression \Ep{Nash_b}
characterizes all their consecutive negotiations. The idea is
that the firm will not incur any layoff tax if the firm and worker
do not agree on a wage in the first encounter; that is, there is
never an employment relationship. In contrast, the firm's threat point
is weakened in future negotiations with an already employed worker
because the firm would then have to pay a layoff tax if the match
were broken up. We will here show that, except for the wage profile,
this alternative specification is equivalent to just assuming
Nash product \Ep{Nash_a} for all periods. The intuition is
that the modified wage profile under the Mortensen and Pissarides
assumption is equivalent to a new hire posting a bond equal to his
share of the future layoff tax.

First, we compute the wage associated with expression \Ep{Nash_a},
$w_a(p)$, from the expression for a firm's match surplus,
$$
J_a(p) \;=\; p \,-\,w_a(p) \,+\, \beta
\Bigl[ \xi J_a(p) \,+\, (1-\xi) \int J_a(p') \, dG(p') \Bigr]\,,  \EQN J_a
$$
which together with equation \Ep{split_a} implies
$$\EQNalign{
w_a(p) \;=\; p \,-\, (1&-\phi)\,S_a(p) \,+\, \beta
\Bigl[ \xi (1-\phi)\,S_a(p)                                           \cr
&\,+\,(1-\xi) \int (1-\phi)\,S_a(p') \, dG(p') \Bigr]\,.  \EQN wage_a \cr}
$$
Second, we verify that the present value of these wages
is exactly equal to that of Mortensen and Pissarides' bargaining scheme
for any completed job, under the maintained hypothesis that the two
formulations have the same match surplus $S_a(p)$. Let $J_1(p)$ and
$J_+(p)$ denote the firm's match surplus with Mortensen and Pissarides'
specification in the first period and all future periods, respectively.
The solutions to the maximization of their Nash products are
$$\eqalign{
J_1(p) \;&=\; (1-\phi) S_a(p) \,,\cr
J_+(p) \;&=\; (1-\phi) S_a(p) - \phi \tau   \,. \cr} \EQN split_c
$$
The associated wage functions can be written as
$$\EQNalign{
w_1(p) \;&=\; p \,-\, J_1(p) \,+\, \beta
\Bigl[ \xi J_+(p) \,+\, (1-\xi) \int J_+(p') \, dG(p') \Bigr]  \cr
\;&=\; w_a(p) \,-\, \beta \,\phi \tau \,,\cr
w_+(p) \;&=\; p \,-\, J_+(p) \,+\, \beta
\Bigl[ \xi J_+(p) \,+\, (1-\xi) \int J_+(p') \, dG(p') \Bigr]  \cr
\;&=\; w_a(p) \,+\, r\, \beta \,\phi \tau \,,\cr}
$$
where the second equalities follow from equations \Ep{wage_a} and
\Ep{split_c}, and $r\equiv \beta^{-1}-1$. It can be seen that the
wage under the Mortensen and Pissarides' specification is reduced
in the first period by the worker's share of any future layoff tax,
and future wages are increased by an amount equal to
the net interest on this posted ``bond.'' In other words,
the present value of a worker's total compensation for any
completed job is identical for the two specifications.
It follows that the present value of a firm's match surplus
is also identical across specifications. We have thereby
confirmed that the same equilibrium allocation is supported
by Nash product \Ep{Nash_a} and Mortensen and Pissarides'
alternative bargaining formulation.


\index{search model!money}
\index{money!search model}
\section{Kiyotaki-Wright search model of money}

We now explore a discrete-time version of Kiyotaki and Wright's (1993)
\auth{Kiyotaki, Nobuhiro} \auth{Wright, Randall}%
search model of money.\NFootnote{Our main simplification is that
the time to produce is deterministic rather than stochastic.  We
also
alter the way money is introduced into the model.}
Let us first study their environment without money.
The economy is populated by a continuum of infinitely lived agents,
with total population normalized to unity. There is also a
number of differentiated commodities, which are indivisible and
come in units of size one.  Agents have idiosyncratic
tastes over these consumption goods as captured by a parameter
$x\in(0,1)$. In particular, $x$ equals the proportion of commodities
that can be consumed by any given agent, and $x$ also equals
the proportion of agents that can consume any given commodity. If a
commodity can be consumed by an agent, then we say that it is one of
his consumption goods. An
agent derives utility $U>0$ from consuming one of his consumption
goods, while the goods that he cannot consume yield zero utility.

Initially, let each agent be endowed with one good, and let these
goods be randomly drawn from the set of all commodities. Goods are
costlessly storable, but each agent can store at most
one good at a time. The only input in the production of goods
is the agents' own prior consumption. After consuming one of his
consumption goods, an agent produces next period a new good drawn
randomly from the set of all commodities. We assume that agents can
consume neither their own output nor their initial endowment, so
for consumption and production to take place there must be exchange
among agents.

Agents meet pairwise and at random. In each period, an agent meets
another agent with probability $\theta \in (0,1]$ and he has no encounter
with probability $1-\theta$. Two agents who meet will trade if there is
a mutually agreeable transaction. Any transaction must be quid pro quo
because private credit arrangements are ruled out by the assumptions of a
random matching technology and a continuum of agents.
We also assume that there is a transaction cost
$\epsilon \in (0, U)$ in terms of disutility, which is incurred
whenever accepting a commodity in trade. Thus, a trader who is indifferent
between holding two goods will never trade one for the other.

Agents choose trading strategies in order to maximize their expected
discounted utility from consumption net of transaction costs, taking as
given the strategies of other traders. Following Kiyotaki and Wright
(1993), we restrict our attention to symmetric Nash equilibria, \index{Nash equilibrium}%
where all agents follow the same strategies and all goods are treated
the same, and to steady states, where strategies and aggregate variables
are constant over time.

In a symmetric equilibrium, an agent will  trade only if he is offered
a commodity that belongs to his set of consumption goods, and then
consumes it immediately. Accepting a commodity that is not one's consumption
good would only give rise to a transaction cost $\epsilon$ without
affecting expected future trading opportunities. This statement
is true because no commodities are treated as special in a
symmetric equilibrium, and therefore the probability of a commodity
being accepted by the next agent one meets is independent of the
type of commodity
one has.\NFootnote{Kiyotaki and Wright (1989) analyze commodity money in
a related model with nonsymmetric equilibria, where some goods become
media of exchange.}
It follows that $x$ is the probability that a trader located
at random is willing to accept any given commodity, and $x^2$ becomes
the probability that two traders consummate a barter in a situation
of ``double coincidence of wants.''
\index{double coincidence of wants}

At the beginning of a period before the realization of the matching
process, the value of an agent's optimization problem becomes
$$
V^n_c \;=\; \theta\,x^2\,(U-\epsilon) \,+\, \beta V^n_c \,,
$$
where $\beta \in (0,1)$ is the discount factor. The superscript and
subscript of $V^n_c$ denote a nonmonetary equilibrium and a commodity
trader, respectively, to set the stage for our next exploration of the
role for money in this economy. How will fiat money affect welfare?
Keep the benchmark of a barter economy in mind,
$$
V^n_c \;=\; { \theta\,x^2\,(U-\epsilon) \over 1 - \beta }\,.   \EQN VNc
$$

\subsection{Monetary equilibria}

At the beginning of time, suppose a fraction $\bar M \in [0,1)$ of all
agents are each offered one unit of fiat money. The money is
indivisible, and an agent can store at most one unit of money or one
commodity at a time. That is, fiat money will enter into circulation only
if some agents accept money and discard their endowment of goods. These
decisions must be based solely on agents' beliefs about other traders'
willingness to accept money in future transactions, because fiat money
is by definition unbacked and intrinsically worthless. To determine
whether or not fiat money will initially be accepted, we will therefore
first have to characterize monetary equilibria.\NFootnote{If money is valued
in an equilibrium, the relative price of goods and
money is trivially equal to $1$, since both objects are indivisible
and each agent can carry at most one unit of the objects.
Shi (1995) and Trejos and Wright (1995)
\auth{Shi, Shouyong}\auth{Trejos, Alberto}\auth{Wright, Randall}%
endogenize the price level
by relaxing the assumption that goods are indivisible.}

Fiat money adds two state variables in a symmetric steady state:
the probability that a commodity trader accepts money, $\Pi\in[0,1]$,
and the amount of money circulating, $M\in[0, \bar M]$, which is
also the fraction of all agents carrying money. An equilibrium pair
$(\Pi, M)$ must be such that an individual's choice of probability of
accepting money when being a commodity trader, $\pi$, coincides
with the economy-wide $\Pi$, and the amount of money $M$ is consistent
with the decisions of those agents who are initially free to replace
their commodity endowment with fiat money.

In a monetary equilibrium, agents can be divided into two types of
traders. An agent brings either a commodity
or a unit of fiat money to the trading process; that is, he is either
a commodity trader or a money trader. At the beginning of a period,
the values associated with being
a commodity trader and a money trader are denoted $V_c$ and $V_m$,
respectively. The Bellman equations can be written
$$\EQNalign{
V_c \;&=\; \theta (1-M) x^2 \bigl(U-\epsilon+\beta V_c\bigr)
\,+\, \theta M x \max_{\pi} \bigl[\pi \beta V_m + (1-\pi) \beta V_c\bigr] \cr
\noalign{\vskip.15cm}
      &\hskip.4cm
    +\, \bigl[1 \,-\, \theta (1-M) x^2 \,-\, \theta M x\bigr] \beta V_c \,,
                                                               \EQN Vc \cr
\noalign{\vskip.3cm}
V_m \;&=\; \theta (1-M) x  \Pi \bigl(U-\epsilon+\beta V_c\bigr)
    +  \bigl[1 - \theta (1-M) x \Pi \bigr] \beta V_m .
                                                               \EQN Vm \cr}
$$
The value of being a commodity trader in equation \Ep{Vc} equals the sum of
three terms. The first term is the probability of the agent meeting other
commodity traders, $\theta (1-M)$, times the probability that both want
to trade, $x^2$, times the value of trading, consuming, and returning
as a commodity trader next period, $U-\epsilon+\beta V_c$. The second
term is the probability of the agent meeting money traders, $\theta M$,
times the probability that a money trader wants to trade, $x$, times
the value of accepting money with probability $\pi$,
$\pi \beta V_m + (1-\pi) \beta V_c$, where $\pi$ is
chosen optimally. The third term captures the complement to the two
previous events when the agent stores his commodity to the next
period with a continuation value of $\beta V_c$.
According to equation \Ep{Vm}, the value of being a money trader
equals the sum of two terms. The first term is the probability of
the agent meeting a commodity trader, $\theta (1-M)$, times the
probability of both wanting to trade, $x \Pi$, times the value of
trading, consuming, and becoming a commodity trader next period,
$U-\epsilon+\beta V_c$. The second term is the probability of
the described event not occurring times the value of keeping
the unit of fiat money to the next period, $\beta V_m$.

The optimal choice of $\pi$ depends solely on $\Pi$. First, note
that if $\Pi<x$ then equations \Ep{Vc} and \Ep{Vm} imply that $V_m<V_c$,
so the individual's best response is $\pi=0$. That is, if money
is being accepted with a lower probability than a barter offer,
then it is harder to trade using money than barter, so agents
would never like to exchange a commodity for money. Second, if
$\Pi>x$, then equations \Ep{Vc} and \Ep{Vm} imply that $V_m>V_c$, so the
individual's best response is $\pi=1$. If money is being accepted
with a greater probability than a barter offer, then it is easier
to trade using money than barter, and agents would always like to
exchange a commodity for money whenever possible. Finally, if
$\Pi=x$, then equations \Ep{Vc} and \Ep{Vm} imply that $V_m=V_c$,
so $\pi$ can be anything in $[0, 1]$. If monetary exchange and
barter are equally easy, then traders are indifferent between carrying
commodities and fiat money, and they could accept money with
any probability. Based on these results, the individual's best-response
\index{best response}%
correspondence is as shown in Figure \Fg{fig194f}, %19.8,
and there are exactly
three values consistent with $\Pi=\pi$: $\Pi=0$, $\Pi=1$, and $\Pi=x$.

%%%%%%%%%%%%
%\topinsert
%$$\grafone{fig194.eps,height=2.5in}{{\bf Figure 19.8.}  The
%best-response correspondence.}
%$$\endinsert

\midfigure{fig194f}
\centerline{\epsfxsize=3truein\epsffile{fig194.eps}}
\caption{The best-response correspondence.}
\infiglist{fig194f}
\endfigure

We can now answer our first question, namely, how many of the agents
who are initially free to exchange their commodity endowment for fiat
money will choose to do so? The answer is implicit in our
discussion of the best-response correspondence. Thus, we have the
following three types of symmetric equilibria:

\noindent{1. } A nonmonetary equilibrium with $\Pi=0$ and $M=0$, which
is identical to the barter outcome in the previous section: Agents
expect that money
will be valueless, so they never accept it, and this expectation
is self-fulfilling. All agents become commodity traders associated
with a value of $V^n_c$, as given by equation \Ep{VNc}.
\smallskip
\noindent{2.} A pure monetary equilibrium with $\Pi=1$ and $M=\bar M$:
Agents expect that money will be universally acceptable. From
our previous discussion
we know that agents will then prefer to bring money rather than
commodities to the trading process. It is therefore a dominant strategy
to accept money whenever possible; that is, expectation is self-fulfilling.
Another implication is that the fraction
$\bar M$ of agents who are initially free to exchange their commodity
endowment for fiat money will also do so. Let $V^p_c$ and $V^p_m$
denote the values associated with being a commodity trader and a
money trader, respectively, in a pure monetary equilibrium.
\smallskip
\noindent{3. } A mixed monetary equilibrium with $\Pi=x$ and $M\in[0, \bar M]$:
Traders are indifferent between accepting and rejecting money as long
as future trading partners take it with probability $\Pi=x$, so partial
acceptability with agents setting $\pi=x$ can also be self-fulfilling.
However, a mixed monetary equilibrium has no longer a unique mapping to
the amount of circulating money $M$. Suppose the initial choices
between commodity endowment and fiat money are separate
from agents' decisions on trading strategies. It
follows that any amount of money between $[0, \bar M]$
can constitute a mixed monetary equilibrium because of the indifference
between a commodity endowment and a unit of fiat money.
Of course, the allocation in a
mixed monetary equilibrium with $M=0$ is identical to the one in a
nonmonetary equilibrium. Let $V^i_c(M)$ and $V^i_m(M)$
denote the values associated with being a commodity trader and a
money trader, respectively, in a mixed monetary equilibrium with an
amount of money equal to $M\in[0, \bar M]$.


\subsection{Welfare}

To compare welfare across different equilibria, we set $\pi=\Pi$ in
equations \Ep{Vc} and \Ep{Vm} and solve for the reduced-form expressions
$$\EQNalign{
V_c \;&=\; { \psi \over 1-\beta } \bigl\{ (1-\beta) x +
           \beta \theta x \Pi \bigl[M \Pi + (1-M) x \bigr] \bigr\} ,
                                                               \EQN Vc_2 \cr
\noalign{\vskip.3cm}
V_m \;&=\; { \psi \over 1-\beta } \bigl\{ (1-\beta) \Pi +
           \beta \theta x \Pi \bigl[M \Pi + (1-M) x \bigr] \bigr\} ,
                                                               \EQN Vm_2 \cr}
$$
where $\psi=[\theta (1-M) x (U-\epsilon)]/[1- \beta(1- \theta x \Pi)]>0$.
The value
$V_m$ is greater than or equal to $V_c$ in a monetary equilibrium, since
a necessary condition is that monetary exchange is at least as easy as
barter ($\Pi \geq x$),
$$
V_m \;=\;  V_c \,+\, \psi \, (\Pi - x) \,.
$$

After setting $\Pi=x$ in equations \Ep{Vc_2} and \Ep{Vm_2}, we
see that a mixed monetary equilibrium with $M>0$ gives rise to
a strictly lower welfare as compared to the barter outcome in
equation \Ep{VNc},
$$
V^i_c(M) \;=\; V^i_m(M) \;=\; (1-M) V^n_c \,.
$$
Even though some agents are initially willing to switch their commodity
endowment for fiat money, it is detrimental for the economy as a
whole. Since money is accepted with the same probability as commodities,
money does not ameliorate the problem of ``double coincidence of wants''
but only diverts real resources from the economy.\NFootnote{This welfare
result differs from that of Kiyotaki and Wright (1993), who assume that
a fraction $\bar M$ of all agents are initially endowed with fiat money
without any choice. It follows that those agents endowed with money are
certainly better off in a mixed monetary equilibrium as compared to
the barter outcome, while the other agents are indifferent. The latter
agents are indifferent because the existence of the former agents
has the same crowding-out effect on their consumption arrival rate
in both types of equilibria. Our welfare results reported here are
instead in line with Kiyotaki and Wright's (1990) original working
paper based on a slightly different environment where agents can at
any time dispose of their fiat money and engage in production.}
In fact, as noted by Kiyotaki and Wright (1990), the mixed monetary
equilibrium is isomorphic to the nonmonetary equilibrium of another
economy where the probability of meeting an agent is reduced
from $\theta$ to $\theta (1-M)$.

In a pure monetary equilibrium ($\Pi=1$), the value of being a money
trader is strictly greater than the value of being a commodity trader.
A natural welfare criterion is the {\it ex ante\/} expected utility
before the quantity $\bar M$ of fiat money is randomly distributed,
$$\EQNalign{
W \;&=\;  \bar M V^p_m  \,+\, (1-\bar M) V^p_c                        \cr
\noalign{\vskip.3cm}
    &=\; {\theta (1-\bar M) x (U - \epsilon) \over 1-\beta }\,
         \bigl[\bar M \,+\, (1-\bar M)x \bigr] \,.    \EQN W_pure      \cr}
$$
The first and second derivatives of equation \Ep{W_pure} are
$$\EQNalign{
{ \partial W \over \partial \bar M } \;&=\;
  { \theta x (U - \epsilon) \over 1 - \beta }
  \Bigl\{1 \,-\,2\bigl[\bar M + (1-\bar M) x \bigr] \Bigr\}  \,,\EQN Wder1 \cr
\noalign{\vskip.3cm}
{ \partial^2 W \over \partial \bar M^2 } \;&=\;
  -\, 2 \,{ \theta x (U - \epsilon) \over 1 - \beta }\,
  (1 - x) \;<\; 0  \,.                                        \EQN Wder2 \cr}
$$
Since the second derivative is negative, fiat money can
only have a welfare-enhancing role if the first derivative is positive
when evaluated at $\bar M=0$. Thus, according to equation \Ep{Wder1},
money can (cannot) increase welfare if $x<.5$ ($x\geq .5$). Intuitively
speaking, when $x\geq .5$, each agent is willing to consume
(and therefore accept) at least half of all commodities, so barter
is not very difficult. The introduction of money would here only reduce
welfare by diverting real resources from the economy. When $x< .5$,
barter is sufficiently difficult so that the introduction of some
fiat money improves welfare. The optimum quantity of money is then
found by setting equation \Ep{Wder1} equal to zero,
$\bar M^\star = (1-2x)/(2-2x)$. That is, $\bar M^\star$ varies
negatively with $x$, and the optimum quantity of money increases
when $x$ shrinks and the problem of ``double coincidence of wants''
becomes more difficult. In particular, $\bar M^\star$ converges to
$.5$ when $x$ goes to zero.


\section{Concluding remarks}

The frameworks of search and matching present various ways of departing
from the frictionless Arrow-Debreu economy where all agents meet in
a complete set of markets. This chapter has mainly focused on labor markets
as a central application of these theories. The presented models have
the concept of frictions in common, but there are also differences.
The island economy has frictional unemployment without any externalities.
An unemployed worker does not inflict any injury
on other job seekers other than what a seller of a good imposes on his
competitors. The equilibrium value to search, $v_u$, serves the function
of any other equilibrium price of signaling to suppliers the correct
social return from an additional unit supplied. In contrast, the matching
model with its matching function is
associated with externalities. Workers and firms impose
congestion effects when they enter as unemployed
in the matching function or add another vacancy in the matching
function. To arrive at an efficient allocation in the economy, it is
necessary that the bilaterally bargained wage be exactly right.
In a labor market with homogeneous firms and workers,
efficiency prevails only if the workers' bargaining strength,
$\phi$, is exactly equal to the elasticity of the matching function
with respect to the measure of unemployment, $\alpha$. In the case of
heterogeneous jobs in the same labor market with a single matching
function, we established the impossibility of efficiency without
government intervention.

The matching model unarguably offers  a richer analysis through its
extra interaction effects, but it comes at the cost of the model's
microeconomic structure. In an explicit economic environment,
feasible actions can be clearly envisioned
for any population size, even if there is only one Robinson Crusoe.
The island economy is an example of such a model with its
microeconomic assumptions,
such as the time it takes to move from one island to another.
In contrast, the matching model with its matching function imposes
relationships between aggregate outcomes. It is therefore not
obvious how the matching function arises when gradually increasing
the population from one Robinson Crusoe to an economy with more
agents. Similarly, it is an open question what
determines when heterogeneous firms and labor have to be matched through
a common matching function and when they have access to
separate matching functions.

Peters (1991) and Montgomery (1991)
\auth{Peters, Michael} \auth{Montgomery, James D.}%
suggest some microeconomic underpinnings to
labor market frictions, which are further pursued by Burdett, Shi, and Wright (2001).
\auth{Burdett, Kenneth}\auth{Shi, Shouyong}\auth{Wright, Randall}%
Firms post vacancies with announced wages, and unemployed workers can  apply
to only one firm at a time. If the values of filled jobs differ across firms, firms
with more valued jobs will have an incentive to post higher wages to attract job
applicants. In an equilibrium, workers will be indifferent between applying to
different jobs, and they are assumed to use identical mixed strategies in
making their applications. In this way, vacancies may remain
unfilled because some firms do not receive any
applicants, and some workers may find themselves ``second in line''
for a job and therefore remain unemployed.
When assuming a large number of firms that take market tightness as
given for each posted wage, Montgomery finds that the decentralized
equilibrium does maximize welfare for reasons similar to Moen's (1997)
identical finding that was discussed earlier in this chapter.

Lagos (2000)
\auth{Lagos, Ricardo}%
 derives a matching function from a model without
any exogenous frictions at all. He studies a dynamic market for taxicab
rides in which taxicabs seek potential passengers on a spatial grid and
the fares are regulated exogenously. In each location, the shorter side
determines the number of matches. It is shown that a matching function
exists for this model, but this matching function is an equilibrium
object that changes with policy experiments. Lagos sounds a warning
that assuming an exogenous matching function when doing policy analysis
might be misleading.

Throughout our discussion of search and matching models, we have
assumed risk-neutral agents. Acemoglu and Shimer (1999), and
Gomes, Greenwood, and Rebelo (2001)
 analyze a matching model and a search model,
respectively, where agents are risk averse and hold precautionary
savings because of imperfect insurance against unemployment.
\auth{Gomes, Joao} \auth{Greenwood, Jeremy} \auth{Rebelo, Sergio}
\auth{Acemoglu, Daron} \auth{Shimer, Robert}

%\note{Compare to chapter XXX where we resort %to a ``black-box'' transaction cost
%function to make fiat money %valuable. See especially subsection XXX
%that discusses a government's %use of legal restrictions to boost demand for its currency.}


%\section{Exercises}
\showchaptIDfalse
\showsectIDfalse
\section{Exercises}
\showchaptIDtrue
\showsectIDtrue
\medskip
\medskip\noindent
{\it Exercise \the\chapternum.1} \quad {\bf An island economy}\  (Lucas and Prescott, 1974)
\auth{Lucas, Robert E., Jr.} \auth{Prescott, Edward C.}%
\index{island model}%
\medskip\noindent
Let the island economy in this chapter have
a productivity shock that takes on two possible values, $\{\theta_L,
\theta_H\}$ with $0 < \theta_L < \theta_H$. An island's
productivity remains constant from one period to another
with probability $\pi \in (.5, 1)$, and its productivity changes
to the other possible value with probability $1-\pi$. These
symmetric transition probabilities imply a stationary distribution
where half of the islands experience a given $\theta$ at any point
in time. Let $\hat x$ be the economy's labor supply (as an
average per market).

\medskip
\noindent{\bf a.}  If there exists a stationary equilibrium with
labor movements, argue that an island's labor force has two
possible values, $\{x_1, x_2\}$ with $0 < x_1 < x_2$.


\medskip
\noindent{\bf b.} In a stationary equilibrium with labor
movements, construct a matrix $\Gamma$ with the transition
probabilities between states $(\theta, x)$, and explain what
the employment level is in different states.

\medskip
\noindent{\bf c.} In a stationary equilibrium with labor
movements, we observe only four values of the value function
$v(\theta,x)$ where $\theta\in\{\theta_L, \theta_H\}$ and
$x\in\{x_1, x_2\}$. Argue that the value function takes on
the same value for two of these four states.

\medskip
\noindent{\bf d.} Show that the condition for the existence
of a stationary equilibrium with labor movements is
$$
\beta (2 \pi - 1) \theta_H \;>\; \theta_L \,,    \eqno(1)   %%EQN cond_move
$$
and, if this condition is satisfied, an implicit
expression for the equilibrium value of $x_2$ is
$$
\left[ \theta_L \,+\, \beta (1-\pi) \theta_H \right] f'(2 \hat x - x_2)
\;= \; \beta \pi \theta_H f'(x_2)\,.             \eqno(2)   %%\EQN x2_opt
$$
\noindent


\medskip
\noindent{\bf e.}
Verify that the allocation of labor in part d coincides with a social
planner's solution when maximizing the present value of the
economy's aggregate production. Starting from an initial equal
distribution of workers across islands, condition (1)   %%\Ep{cond_move}
indicates when it is optimal for the social planner to increase
the number of workers on high-productivity islands.
The first-order condition for the social planner's choice of
$x_2$ is then given by equation (2).           %%\Ep{x2_opt}.

%\noindent
%{\it Hint}:
%Consider an employment plan $(x_1, x_2)$ such that the next
%period's labor force is $x_1$ ($x_2$) for an island currently experiencing
%productivity shock $\theta_L$ ($\theta_H$). If $x_1 \leq x_2$, the present
%value of the economy's production (as an island average) becomes
%$$
%.5 \sum_{t=0}^\infty \beta^t \left[ \theta_L f(2 \hat x - x_2)
%\,+\, (1-\pi) \theta_H f(2 \hat x - x_2) \,+\, \pi \theta_H f(x_2)
%                                                    \right] \,.
%$$
%Examine the effect of a once-and-for-all increase in the number
%of workers allocated to high-productivity islands.
\vfil\eject
\medskip\noindent
{\it Exercise \the\chapternum.2}\quad  {\bf Business cycles and search} \ \ (Gomes, Greenwood,
and Rebelo, 2001)
\auth{Gomes, Joao}\auth{Greenwood, Jeremy}\auth{Rebelo, Sergio}%
\index{search model!business cycles}

\medskip\noindent
{\bf Part I} \ \ {\it The worker's problem}

\medskip\noindent
Think about an economy in which workers  all confront the following common
environment: Time is discrete.  Let $t=0, 1, 2, \ldots$ index time.
At the beginning of each period, a  previously employed
worker can choose to work at her last
period's wage or  draw a new wage.
 If she draws a new wage, the old wage is
lost and she will be unemployed in the current period.  She can start work
at the new wage in the next period.
New wages are independent and identically
distributed from the cumulative
distribution function $F$, where $F(0)=0$, and $F(M)=1$ for $M<\infty$. Unemployed workers face a similar problem. At the beginning of each period, a previously unemployed worker can choose to work at last period's wage offer or to draw a new wage from $F$. If she draws a new wage, the old wage offer is lost and she can start working at the new wage in the following period. Someone offered a wage is free to work at that wage for as long as she
chooses (she cannot be fired).  The income of an unemployed
worker is $b$, which includes unemployment insurance and the
value of home production.  Each worker seeks to maximize
$E_0 \sum_{t=0}^\infty (1-\mu)^t \beta^t I_t, $
where $\mu$ is the probability that a worker dies at the end of a period,
$\beta$ is the subjective discount factor, and $I_t$ is the worker's income
in period $t$; that is, $I_t$ is equal to the wage $w_t$ when employed
and the income $b$ when unemployed.  Here, $E_0$ is the mathematical
expectation operator, conditioned on information known at time $0$.
Assume that $\beta \in (0,1)$ and $\mu \in (0,1)$.

\medskip
\noindent{\bf a.} Describe the  worker's optimal decision rule.  In particular,
what should an employed worker do?  What should an unemployed worker do?
\vskip.1cm

\noindent{\bf b.} How would an unemployed  worker's
behavior be affected by an increase in $\mu$?
\medskip\noindent
{\bf Part II}\ \ {\it  Equilibrium unemployment rate}

\medskip\noindent
The economy is populated with a continuum of the workers just
described. There is an exogenous rate of new workers entering
the labor market  equal to $\mu$, which equals
the death rate. New entrants are unemployed and must draw a new wage.

\vskip.2cm

\noindent{\bf c.} Find an expression for the economy's unemployment
rate in terms of exogenous parameters and the endogenous reservation
wage.  Discuss the determinants of the unemployment rate.

\medskip
\noindent
We now change the technology so that
the economy fluctuates between booms ($B$) and recessions ($R$).
In a boom, all employed workers are paid an extra $z>0$.
That is, the income of a worker with wage $w$ is $I_t=w+z$ in a boom and
$I_t=w$ in a recession.  Let whether the economy is
in a boom or a recession define the {\it state\/} of the economy.
Assume that
the state of the economy is i.i.d.\ and that booms and
recessions have the same probabilities of
$.5$. The state of the economy is publicly known at the beginning
of a period before any decisions are made.

\vskip.2cm
\noindent{\bf d. }  Describe the optimal behavior of employed and unemployed
 workers.  When, if ever, might workers choose to quit?

\vskip.2cm
\noindent{\bf e. } Let $w_{\scriptscriptstyle B}$ and $w_{\scriptscriptstyle R}$
be the reservation wages in booms and recessions, respectively. Assume
that $w_{\scriptscriptstyle B} < w_{\scriptscriptstyle R}$.
Let $G_t$ be the fraction of workers employed at wages
$w \in [w_{\scriptscriptstyle B}, w_{\scriptscriptstyle R}]$ in period $t$.
Let $U_t$ be the fraction
of workers unemployed in period $t$.
Derive difference equations for $G_t$ and $U_t$ in terms
of the parameters of the model and the reservation wages,
$\{F, \mu, w_{\scriptscriptstyle B}, w_{\scriptscriptstyle R}\}$.

\vskip.1cm
\noindent{\bf f. } Figure \Fg{pic11f} contains a simulated time series from the solution
of the model with booms and recessions.  Interpret the time series in terms
of the model.


%%%%%%%%%%%%%
%$$
%\grafone{pic1.ps,height=2.1in}{\bf } % replaced from Lars, Aug 10, 2001
%\grafone{pic11.ps,height=2.1in}{{\bf Figure 19.9.}  Unemployment during
%business cycles.}  $$
%%%%%%%%

\midfigure{pic11f}
\centerline{\epsfxsize=3truein\epsffile{pic11.ps}}
\caption{Unemployment during business cycles.}
\infiglist{pic11f}
\endfigure

\medskip
\noindent {\it Exercise \the\chapternum.3} \quad {\bf Business cycles and search again}
\medskip
\noindent
The economy is either in a boom ($B$) or recession ($R$) with
probability $.5$.  The state of the economy ($R$ or $B$) is
i.i.d. through time.  At the beginning of each period, workers
know the state of the economy for that period.
At the beginning of each period, a previously
employed worker can choose to work at her last
period's wage or draw a new wage. If she draws a new wage,
the old wage is lost, $b$ is received this period,
and she can start working at the new wage in the following period.
During recessions, new wages (for jobs to start next period)
are i.i.d. draws from the c.d.f. $F$, where $F(0)=0$ and $F(M)=1$
for $M<\infty$. During booms, the worker can choose to quit and
take {\it two\/} i.i.d. draws of a possible new wage (with the
option of working at the higher wage, again for a job to start
the next period) from the {\it same\/} c.d.f. $F$ that prevails
during recessions.  (This ability to choose
is what ``jobs are more plentiful during booms'' means
to workers.)  Workers who are unemployed at the beginning of
a period receive $b$ this period and draw either one (in recessions)
or two (in booms) wage offers from the c.d.f. $F$ to start
work next period.
\medskip
\noindent
A worker seeks to maximize
$E_0 \sum_{t=0}^\infty (1-\mu)^t \beta^t I_t$,
where $\mu$ is the probability that a worker dies at the end of
a period, $\beta$ is the subjective discount factor, and $I_t$
is the worker's income in period $t$; that is, $I_t$ is equal
to the wage $w_t$ when employed and the income $b$ when unemployed.

\vskip.2cm
\noindent{\bf a.} Write the Bellman equation(s) for a previously
employed worker.

\vskip.1cm
\noindent{\bf b.} Characterize the worker's quitting policy.
If possible, compare reservation wages in booms and recessions.
Will employed workers ever quit?  If so, who will quit and when?

\medskip
\noindent
{\it Exercises \the\chapternum.4--\the\chapternum.6} \quad {\bf European unemployment}
\index{unemployment!European}
\medskip\noindent
The following three exercises are based on work by Ljungqvist
and Sargent (1998), Marimon and Zilibotti (1999), and Mortensen
and Pissarides (1999b),
who calibrate versions of search and
matching models to explain high European unemployment. Even
though the specific mechanisms differ, they all
attribute the rise in unemployment to generous benefits in
times of more dispersed labor market outcomes for job seekers.
\auth{Ljungqvist, Lars} \auth{Sargent, Thomas J.} \auth{Marimon, Ramon}
\auth{Zilibotti, Fabrizio} \auth{Mortensen, Dale T.} \auth{Pissarides, Christopher}
\medskip\noindent
{\it Exercise \the\chapternum.4} \quad {\bf Skill-biased technological change} \
              (Mortensen and Pissarides, 1999b)
\index{matching model!skill-biased technological change}

\medskip\noindent
Consider a matching model in discrete time with infinitely lived
and risk-neutral
workers who are endowed with different skill levels. A worker
of skill type $i$ produces $h_i$ goods in each period that
she is matched to a firm, where $i\in\{1,2,\ldots,N\}$
and $h_{i+1}>h_i$. Each skill type has its own but identical
matching function $M(u_i,v_i)=A u_i^\alpha v_i^{1-\alpha}$,
where $u_i$ and $v_i$ are the measures of unemployed workers
and vacancies in skill market $i$. Firms incur a vacancy
cost $c \, h_i$ in every period that a vacancy is posted in
skill market $i$; that is, the vacancy cost is proportional to
the worker's productivity. All matches are exogenously
destroyed with probability $s\in(0,1)$ at the beginning of a
period. An unemployed worker receives unemployment compensation
$b$. Wages are determined in Nash bargaining between matched
firms and workers. Let $\phi \in [0,1)$ denote the worker's
bargaining weight in the Nash product, and we adopt the
standard assumption that $\phi=\alpha$.

\vskip.2cm
\noindent{\bf a.} Show analytically how the unemployment rate in
a skill market varies with the skill level $h_i$.

\vskip.1cm
\noindent{\bf b.} Assume an even distribution of workers across skill
levels. For different benefit levels $b$, study numerically
how the aggregate steady-state unemployment rate is affected by
mean-preserving spreads in the distribution of skill levels.

\vskip.1cm \noindent{\bf c.} Explain how the results would change if
unemployment benefits are proportional to a worker's productivity.


\medskip\noindent
{\it Exercise \the\chapternum.5} \quad {\bf Dispersion of match values} \
                                    (Marimon and Zilibotti, 1999)

\index{matching model!dispersion of match values}
\medskip\noindent
We retain the matching framework of exercise {\it \the\chapternum.4\/}  %%19.4
but assume that all
workers have the same innate ability $h=\bar h$ and any earnings
differentials are purely match specific. In particular, we assume that
the meeting of a firm and a worker is associated with a random draw of a
match-specific productivity $p$ from an exogenous distribution $G(p)$.
If the worker and firm agree to stay together, the output of the
match is then $p \cdot h$ in every period as long as the match
is not exogenously destroyed as in exercise {\it \the\chapternum.4\/}.   %%19.4.
We also keep the
assumptions of a constant unemployment compensation $b$ and
Nash bargaining over wages.

\vskip.1cm
\noindent{\bf a.} Characterize the equilibrium of the model.

\vskip.1cm
\noindent{\bf b.} For different benefit levels $b$, study numerically
how the steady-state unemployment rate is affected by
mean-preserving spreads in the exogenous distribution $G(p)$.

\medskip
\noindent {\it Exercise \the\chapternum.6} \quad {\bf Idiosyncratic shocks to human capital}
 \ \ (Ljung\-qvist and Sargent, 1998)
\index{search model!shocks to human capital}

\medskip\noindent
We retain the assumption of exercise {\it \the\chapternum.5\/}    %%19.5
that a worker's output is the product of
his human capital $h$ and a job-specific component which we now
denote $w$, but we replace the matching framework with a search
model. In each period of unemployment, a worker draws a value
$w$ from an exogenous wage offer distribution $G(w)$ and, if the worker
accepts the wage $w$, he starts working in the following period.
The wage $w$ remains constant throughout the employment spell that
ends either because the worker quits or the job is exogenously
destroyed with probability $s$ at the beginning of each period.
Thus, in a given job with wage $w$, a worker's earnings $w h$
can only vary over time because of changes in human capital $h$.
For simplicity,
we assume that there are only two levels of human capital,
$h_1$ and $h_2$ where $0<h_1<h_2<\infty$. At the beginning of
each period of employment, a worker's human capital is unchanged
from last period with probability $\pi_e$ and is equal to
$h_2$ with probability $1-\pi_e$. Losses of human capital
are only triggered by exogenous job destruction. In the
period of an exogenous job loss, the laid off worker's human
capital is unchanged from last period with probability $\pi_u$
and is equal to $h_1$ with probability $1-\pi_u$. All unemployed
workers receive unemployment compensation, and
the benefits are equal to a replacement ratio $\gamma \in [0,1)$
times a worker's last job earnings.

\vskip.1cm
\noindent{\bf a.} Characterize the equilibrium of the model.

\vskip.1cm
\noindent{\bf b.} For different replacement ratios $\gamma$, study numerically
how the steady-state unemployment rate is affected by
changes in $h_1$.

\vskip.3cm
\noindent{\it  Comparison of models}

\vskip.3cm
\noindent{\bf c.} Explain how the different models in exercises
{\it \the\chapternum.4\/} through {\it \the\chapternum.6\/}    %%19.4--19.6
address the observations
that European welfare states have experienced less of an increase
in earnings differentials as compared to the United States, but
suffer more from long-term unemployment where the probability
of gaining employment drops off sharply with the length of the
unemployment spell.

\vskip.1cm
\noindent{\bf d.} Explain why
the assumption of infinitely lived agents is innocuous for the
models in exercises {\it \the\chapternum.4\/} and {\it \the\chapternum.5\/},  %%19.4 and 19.5,
but the alternative assumption
of finitely lived agents can make a large difference for
the model in exercise {\it \the\chapternum.6\/}.                     %%19.6
\medskip
  \noindent {\it Exercise \the\chapternum.7} \quad  {\bf Temporary jobs and layoff costs}
  \medskip \noindent Consider a search model with temporary jobs. At the beginning of
each period, a
previously employed worker loses her job with probability $\mu$, and she can keep
her job and wage rate from last period with  probability $1-\mu$. If she loses
 her job (or chooses to quit),  she draws a new wage and can start working at the
new wage in  the following period with probability $1$. After a first period on
the new job, she will again in each period face  probability $\mu$ of losing her
job. New wages are independent and identically distributed from the cumulative
distribution function $F$, where $F(0)=0$, and $F(M)=1$ for $M<\infty$. The
situation during unemployment is as follows. At the beginning of each period, a
previously unemployed worker can choose to start working at last period's wage
offer or to draw a new wage from $F$. If she draws a new wage, the old wage offer
is lost and she can start working at the new wage in the following period. The
income of an unemployed  worker is $b$, which includes unemployment insurance and
the  value of home production.  Each worker seeks to maximize
$E_0 \sum_{t=0}^\infty \beta^t I_t$, where  $\beta$ is the subjective discount
factor, and $I_t$ is the worker's income in period $t$; that is, $I_t$ is equal
to the wage $w_t$ when employed  and the income $b$ when unemployed.
Here $E_0$ is the mathematical expectation operator, conditioned on information
known at time $0$. Assume that $\beta \in (0,1)$ and $\mu \in (0,1]$.
\medskip
\noindent{\bf a.} Describe the worker's optimal decision rule.
\medskip
\noindent Suppose that there are two types of temporary jobs: short-lasting
jobs with $\mu_s$ and long-lasting jobs with $\mu_l$, where $\mu_s>\mu_l$.
When the worker draws a new wage from the distribution $F$, the job is now
randomly designated as  either short-lasting with probability $\pi_s$ or
long-lasting with probability $\pi_l$, where $\pi_s+\pi_l=1$. The worker
observes the characteristics of a job offer, $(w, \mu)$.
\medskip
\noindent{\bf b.} Does the worker's reservation wage depend on whether a job
is short-lasting or long-lasting? Provide intuition for your answer.
\medskip
 We now consider the effects of layoff costs. It is
assumed that the government imposes a cost $\tau>0$ on each worker that
loses a job (or quits).

\medskip
\noindent{\bf c.} Conceptually, consider the following two reservation wages,
for a given value of $\mu$: (i) a previously unemployed worker sets a
reservation wage for accepting last period's wage offer;
(ii) a previously employed worker sets a reservation wage for continuing
working at last period's wage. For a given value of $\mu$, compare these
two reservation wages.

\medskip
\noindent{\bf d.} Show that an unemployed worker's reservation wage for
a short-lasting job exceeds her reservation wage for a long-lasting job.

\medskip
\noindent{\bf e.} Let $\bar w_s$ and $\bar w_l$ be
an unemployed worker's reservation wages for
short-lasting jobs and long-lasting jobs, respectively. In period $t$,
let $N_{st}$ and $N_{lt}$ be the fractions of workers employed
in short-lasting jobs and long-lasting jobs, respectively. Let $U_t$
 be the fraction of workers unemployed in period $t$. Derive difference
equations for $N_{st}$, $N_{lt}$ and $U_t$ in terms of the parameters
of the model and the reservation wages,
$\{F,\mu_s,\mu_l,\pi_s,\pi_l,\bar w_s,\bar w_l\}$.

\medskip

\noindent
{\it Exercise \the\chapternum.8} \quad  {\bf Productivity shocks,
 job creation, and job destruction}, donated by Rodolfo Manuelli
\medskip \noindent
Consider an economy populated by a large number of identical individuals.
  The utility function of each individual is
$$ \sum_{t=0}^\infty \beta ^t  x_t, $$
where $0 < \beta < 1$, $\beta=1/(1+r)$, and $x_t$ is income
at time $t$.  All individuals are endowed with one unit of labor
that is supplied inelastically: If the individual is working in the market,
its productivity is $y_t$, while if he or she works at home,
productivity is $z$.  Assume that $z < y_t$.  Individuals who
are producing at home can also, at no cost, search for a
market job.  Individuals who are searching and jobs that are vacant get
randomly matched.
    Assume that the number of matches per period is given by
$$ M(u_t, x_t), $$
where $M$ is concave, increasing in each
argument, and homogeneous of degree $1$. In this
setting, $u_t$ is interpreted as the total number of unemployed
workers, and $v_t$ is the total number of vacancies.
Let $\theta \equiv v/u$, and let $q(\theta) = M(u,v)/v$ be
 the probability that a vacant job (or firm) will meet a worker.
  Similarly, let $\theta q(\theta) = M(u,v)/u$ be the probability that
an unemployed worker is matched with a vacant job.  Jobs are exogenously
destroyed with probability $s$.
    In order to create a vacancy, a firm must pay a cost $c > 0$ per
period in which the vacancy is ``posted'' (i.e., unfilled).  There is a large
number of potential firms (or jobs), and this guarantees that the expected
value of a vacant job, $V$, is zero.
    Finally, assume that when a worker and a vacant job meet, they
bargain according to the Nash bargaining solution, with the worker's share
equal to $\varphi$.  Assume that $y_t = y$ for all $t$.
\medskip
 \noindent{\bf a.} Show that the zero-profit condition implies that
$$ w = y - (r+s)c/q(\theta). $$

\noindent{\bf b.}  Show that if workers and firms negotiate wages according to
the Nash bargaining solution (with worker's share equal to $\varphi$),
wages must also satisfy
$$ w = z + \varphi (y - z + \theta c). $$

\noindent{\bf c.}  Describe the determination of the equilibrium level
 of market tightness, $\theta$.
\medskip
\noindent{\bf d.}  Suppose that at $t = 0$, the economy is at its steady
state.  At this point, there is a once-and-for-all
increase in productivity.  The new value of $y$ is $y' > y$.
Show how the new steady-state value of $\theta$, $\theta'$, compares
with the previous value.  Argue that the economy ``jumps'' to the new
value right away.  Explain why there are no ``transitional dynamics'' for
the level of market tightness, $\theta$.
\medskip

\noindent{\bf e.}  Let $u_t$ be the unemployment rate at time $t$.
  Assume that at time $0$ the economy is at the steady-state
 unemployment rate corresponding to $\theta$, the ``old''
market tightness, and display this rate.  Denote this rate as $u_0$.
  Let $\theta_0 = \theta'$.  Note that change in unemployment rate is
equal to the difference between job destruction at $t, JD_t$ and
job creation at $t, JC_t$.  It follows that
$$ \eqalign{JD_t &= (1-u_t) s,  \cr
    JC_t &= \theta_t q (\theta_t) u_t,  \cr
    u_{t+1} - u_t &= JD_t - JC_t.  \cr }  $$
    Go as far as you can characterizing job creation and job
destruction at $t = 0$ (after the shock).  In addition, go as far
as you can describing the behavior of both $JC_t$ and $JD_t$ during
the transition to the new steady state (the one corresponding to $\theta'$).
\vfil\eject
\medskip
\noindent{\it Exercise \the\chapternum.9} \quad  {\bf Workweek restrictions,
unemployment, and welfare}, donated by
Rodolfo Manuelli
\medskip \noindent
Recently, France has moved to a shorter workweek of about 35 hours per week.   In this
exercise you are asked to evaluate the consequences of such a move.
To this end, consider an economy populated by risk-neutral, income-maximizing  workers with preferences given by

$$ U = E_t \sum_{j=0}^\infty \beta^j y_{t+j}, \quad
0 < \beta < 1, \quad
1 + r = \beta^{-1}.  $$
 \noindent
Assume that workers produce $z$ at home if they are unemployed, and that they are
endowed with one unit of labor.  If a worker is employed, he or she can spend $x$ units
of time at the job, and $(1-x)$ at home, with $0\leq x \leq 1$.  Productivity on the
job is $yx$, and $x$ is perfectly observed by both workers and firms.
\medskip
\noindent
Assume that if a worker works $x$ hours, his or her wage is $wx$.
\medskip
\noindent
Assume that all jobs have productivity $y>z$, and that to create a vacancy
firms have to pay a cost of $c>0$ units of output per period.  Jobs are destroyed
with probability $s$.  Let the number of matches per period be given by
$$ M(u,v),  $$
where $M$ is concave, increasing in each argument, and homogeneous of degree one.
In this setting, $u$ is interpreted as the total number of unemployed workers,
and $v$ is the total number of vacancies.  Let $\theta \equiv v/u$, and let
$q(\theta) = M(u,v)/v$.
\medskip
\noindent
Assume that workers and firms bargain over wages, and that the outcome is
described by a Nash bargaining outcome with the workers' bargaining power
equal to $\varphi$.
\medskip

\noindent{\bf a.} Go as far as you can describing the unconstrained (no restrictions on
$x$ other than it be a number between $0$ and $1$) market equilibrium.
\medskip

\noindent{\bf b.} Assume that $q(\theta) = A\theta^{-\alpha}$, for some $0 < \alpha < 1$.
Does the solution of the planner's problem coincide with the market equilibrium?
\medskip

\noindent{\bf c.} Assume now that the workweek is restricted to be less than or
equal to $x^* < 1$.  Describe the equilibrium.
\medskip

\noindent{\bf d.} For the economy in part c, go as far as you can (if necessary, make
additional assumptions) describing the impact of this workweek restriction on wages,
unemployment rates, and the total number of jobs.  Is the equilibrium optimal?
\medskip
\noindent{\it Exercise \the\chapternum.10} \quad {\bf Costs of creating a vacancy and
 optimality}, donated by Rodolfo Manuelli
\medskip \noindent
Consider an economy populated by risk-neutral, income-maximizing workers with
preferences given by
$$ U = E_t \sum_{j=0}^\infty \beta^j y_{t+j}, \quad
0 < \beta < 1, \quad
1 + r = \beta^{-1}.  $$
Assume that workers produce $z$ at home if they are unemployed.  Assume that
 all jobs have productivity $y>z$, and that to create a vacancy firms
have to pay $p_A$, with $p_A = C'(v)$, per period when they have an open
 vacancy, with $v$ being the total number of vacancies.  Assume that the
function $C(v)$ is strictly convex, twice differentiable and increasing.
  Jobs are destroyed with probability $s$.
\medskip
\noindent
Let the number of matches per period be given by
$$ M(u,v),  $$
where $M$ is concave, increasing in each argument, and homogeneous
 of degree $1$.  In this setting, $u$ is interpreted as the total
number of unemployed workers, and $v$ is the total number of vacancies.
Let $\theta \equiv v/u$, and let $q(\theta) = M(u,v)/v$.
\medskip
\noindent
Assume that workers and firms bargain over wages and that the outcome is
described by a Nash bargaining outcome with the worker's bargaining power
equal to $\varphi$.
\medskip

\noindent{\bf a.} Go as far as you can describing the market equilibrium.
In particular, discuss how changes in the exogenous variables, $z$, $y$, and
the function $C(v)$, affect the equilibrium outcomes.
\medskip

\noindent{\bf b.} Assume that $q(\theta) = A \theta^{-\alpha}$ for some  $0 < \alpha < 1$.
Does the solution of the planner's problem coincide with  the market equilibrium?
Describe instances, if any, in which this is the case.
\vfil\eject
\medskip
\noindent{\it Exercise \the\chapternum.11} \quad  {\bf Financial wealth, heterogeneity,
 and unemployment}, donated by Rodolfo Manuelli
\medskip \noindent
Consider the behavior of a risk-neutral worker who seeks to maximize the expected
present discounted value of wage income.  Assume that the discount factor is fixed
and equal to $\beta$, with $0<\beta<1$.  The interest rate is also constant and
satisfies $1+r=\beta^{-1}$.  In this economy, jobs last  forever.  Once the worker
has accepted a job, he or she never quits and the job is never destroyed.  Even though
preferences are linear, a worker needs to consume a minimum of $a$ units of
consumption per period.  Wages are drawn from a distribution with support
on $[a,b]$.  Thus, any employed individual can have a feasible consumption
level.  There is no unemployment compensation.

\medskip
\noindent
Individuals of type $i$ are born with wealth $a^i$, $i=0,1,2$, where $a^0=0$,
$a^1=a$, $a^2=a(1+\beta)$.  Moreover, in the period that they are born, all
individuals are unemployed.  Population, $N_t$, grows at the constant rate $1+n$.
Thus, $N_{t+1} = (1+n) N_t$.  It follows that, at the beginning of period $t$,
at least $nN_{t-1}$ individuals --- those born in that period --- will be unemployed.
Of the $nN_{t-1}$ individuals born at time $t$, $\varphi^0$ are of type $0$,
$\varphi^1$ of type $1$, and the rest, $1-\varphi^0 - \varphi^1$, are of type $2$.
Assume that the mean of the offer distribution (the mean offered, not necessarily
accepted, wage) is greater than $a/\beta$.
\medskip

\noindent{\bf a.} Consider the situation of an unemployed worker who has $a^0=0$.
Argue that this worker will have a reservation wage $w^*(0)=a$.  Explain.
\medskip

\noindent{\bf b.} Let $w^*(i)$ be the reservation wage of an individual with wealth
$i$.  Argue that $w^*(2) > w^*(1) > w^*(0)$.  What does this say about the cross-sectional
relationship between financial wealth and employment probability?
Discuss the economic reasons underlying this result.
\medskip

\noindent{\bf c.} Let the unemployment rate be the number of unemployed
individuals at $t, U_t$, relative to the population at
$t, N_t$.  Thus, $u_t = U_t/N_t$.  Argue that in this economy,
the unemployment rate is constant.
\medskip

\noindent{\bf d.} Consider a policy that redistributes wealth in the form
of changes in the fraction of the population that is born with
wealth $a^i$.  Describe as completely as you can the effect upon
the unemployment rate of changes in $\varphi^i$.  Explain your results.
\medskip

\noindent{\it Extra credit:} Go as far as you can describing the distribution
of the random variable ``number of periods unemployed'' for an individual
of type 2.
