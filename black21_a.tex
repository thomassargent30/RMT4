
\input grafinp3
%\input grafinput8
\input psfig
%
%\showchaptIDtrue
%\def\@chaptID{21.}
%\input form1
%\chapternum=28
\def\cov{\mathop{\rm cov}\nolimits} \overfullrule=0pt
%\hbox{}
\footnum=0
\Appendix{B}{Linear Projections and Hidden Markov Models\label{lincontrol}}
%
%
\section{Linear projections} %to Chapter 2}

For reference we state the following theorems about linear least-squares
projections.  We let $Y$ be an $(n \times 1)$ vector of random variables and
$X$ be an $(h \times 1)$ vector of random variables.  We assume that the
following first and second moments exist:
$$\eqalign{EY &= \mu_Y,\ EX = \mu_X , \cr
EXX^\prime &= S_{XX},\ EYY^\prime = S_{YY},\ EYX^\prime = S_{YX} .\cr}$$
Letting $x=X - EX$ and $y = Y - EY$, we define the following covariance matrices
$$Exx^\prime = \Sigma_{xx},\ E_{yy}^\prime = \Sigma_{yy},\ Eyx^\prime =
\Sigma_{yx}.$$

We are concerned with estimating $Y$ as a linear function of $X$.  The
estimator of $Y$ that is a linear function of $X$ and that minimizes the
mean squared error between each component $Y$ and its estimate is called the
{\it linear projection of $Y$ on $X$.}
\medskip\noindent
%\specsec{Definition 21.2:}
\definition{def21.2} The {\it linear projection\/} of $Y$ on $X$ is the
affine
function $\hat Y = AX + a_0$ that minimizes $E\hbox{ trace } \{(Y-\hat Y)\,
(Y-\hat Y)^\prime\}$ over all affine functions $a_0+AX$ of $X$.  We denote
this linear
projection as $\widehat E [Y \mid X]$, or sometimes as $\widehat E\, [Y\mid x,
\, 1]$ to emphasize that
a constant is included in the ``information set.''
\enddefinition
\par
The linear projection of $Y$ on $X$, $\widehat E \, [Y \mid X]$ is also
sometimes called the {\it wide sense expectation of $Y$ conditional on $X$.}
We have the following theorems:
\medskip \noindent
%\specsec{Theorem 21.1:}
\theorem{th21.1}
$$\widehat E\,[Y \mid X] = \mu_y + \Sigma_{yx} \Sigma^{-1}_{xx} (X-\mu_x) .
\EQN A1$$
\endtheorem
\medskip \noindent
%\specsec{Proof:}
\proof
The theorem follows immediately by writing out $E\, {\rm trace}
\ (Y-\hat Y) (Y - \hat
Y)^\prime$ and completing the square, or else by writing out $E \, {\rm trace}
(Y-\hat Y) (Y - \hat Y)^\prime$ and obtaining first-order necessary conditions
(``normal equations'') and solving them. \endproof %\qed
\medskip \noindent
%\specsec{Theorem 21.2:}
\theorem{th21.2}
$$\widehat E\,\biggl[\bigl(Y - \widehat E [Y \mid x]\bigr) \mid X^\prime
\biggr] = 0.$$
\endtheorem
\noindent This equation states that the errors from the projection are orthogonal to each
variable included in $X$.
\medskip\noindent
%\specsec{Proof:}
\proof Immediate from the normal equations.\endproof % \qed
\medskip\noindent
%\specsec{Theorem 21.3:}
\theorem{th21.3} \quad (Orthogonality principle)
$$E\Bigl[ [Y-\widehat E\,(Y\mid x)]\,x^\prime\Bigr]=0 .$$
\endtheorem
\medskip\noindent
%\specsec{Proof:}
\proof Follows from Theorem 21.3. \endproof %\qed

%\specsec{Theorem 21.4:}
\theorem{th21.4} \quad (Orthogonal regressors)
\medskip \noindent
 Suppose that\hfil\break
$X^\prime = (X_1, X_2, \ldots, X_h)^\prime, EX^\prime= \mu^\prime = (\mu_{x1},
\ldots, \mu_{xh})^\prime$, and $E (X_i - \mu_{xi})\, (X_j-\mu_{xj}) = 0$
for $i \not= j$.  Then
$$\widehat E \, [Y \mid x_1,\ldots, x_n, 1] = \widehat E\,[Y \mid x_1]+\widehat
E\,[Y\mid x_2] + \ldots + \widehat E\, [Y\mid x_n]-(n-1) \mu_y . \EQN A2$$
\endtheorem
\medskip\noindent
%\specsec{Proof:}
\proof Note that from the hypothesis of orthogonal regressors, the
matrix $\Sigma_{xx}$ is diagonal.  Applying equation
\Ep{A1} then gives equation \Ep{A2}. \endproof %\qed
\index{Markov chain!hidden}
\index{filter!nonlinear}
\section{Hidden Markov models}\label{sec:HMM}%
This section gives a brief introduction to hidden Markov models,
a tool that is useful to study a variety of nonlinear filtering
problems in finance and economics.  We display a solution to
a nonlinear filtering problem that a reader might want
to compare to the linear filtering problem described earlier.

    Consider an $N$-state Markov chain.   We can represent the
state space in terms of the unit vectors
$S_x = \{e_1,\ldots, e_N\}$, where $e_i$  is the $i$th
 $N$-dimensional unit vector.  Let the $N \times N$ transition
matrix be $P$, with $(i,j)$ element
$$P_{ij} = {\rm Prob}(x_{t+1} = e_j\mid x_t = e_i).$$
With these definitions, we have
$$E x_{t+1} \mid x_t = P^\prime x_t.$$
%
Define the ``residual''
$$v_{t+1} = x_{t+1} - P^\prime x_t,$$
which implies the linear ``state-space'' representation
$$x_{t+1} = P^\prime x_t + v_{t+1}.$$
Notice how it follows that
$E \ v_{t+1} \mid x_t = 0 ,$ which qualifies $v_{t+1}$ as a ``martingale
difference process adapted to $x_t$.''\index{martingale difference}%

  We want to append a ``measurement equation.''   \index{measurement equation}%
Suppose that $x_t$ is not observed, but that $y_t$, a   noisy function
of $x_t$, is observed.  Assume that $y_t$ lives in the $M$-dimensional
space $S_y$, which we represent in terms of
$M$ unit  vectors:
$S_y = \{f_1,\ldots, f_M\}$, where $f_i$ is the $i$th $M$-dimensional
unit vector.
To specify a  linear measurement equation $y_t = C(x_t, u_t)$, where
$u_t$ is a measurement noise, we begin by defining the $N \times M$
matrix $Q$ with
$$\hbox{Prob } (y_t = f_j \mid x_t = e_i) = Q_{ij}.$$
It follows that
 $$E\ (y_t\mid x_t) = Q^\prime x_t.$$
Define the residual
$$u_t \equiv y_t - E\ y_t\mid x_t,$$
which suggests the ``observer equation''  \index{observer equation}
$$y_t = Q^\prime x_t + u_t.$$
It follows from the definition of $u_t$ that $E \ u_t \mid x_t = 0$.
Thus, we have the linear state-space system
$$\eqalign{ x_{t+1} &= P' x_t + v_{t+1} \cr
             y_t &= Q' x_t + u_t. \cr}  $$
Using the definitions,
it is straightforward to calculate the conditional second moments
of the error processes
 $v_{t+1},u_t$.\NFootnote{Notice that
 $$\eqalign{x_{t+1} x_{t+1}^\prime &= P' x_t (P' x_t)' + P' x_t v'_{t+1}\cr
&+ v_{t+1} (P' x_t)' + v_{t+1} v'_{t+1}\cr}$$
Substituting into this equation the facts that
$x_{t+1} x^\prime_{t+1} = \hbox{diag } x_{t+1} = \hbox{diag } (P^\prime
x_t) + \hbox{diag } v_{t+1}$ gives
$$\eqalign{v_{t+1} v^\prime_{t+1}&= \hbox{diag } (P^\prime x_t) + \hbox{diag }
(v_{t+1}) - P^\prime \hbox{diag } x_t P \cr
&- P^\prime x_t v^\prime_{t+1} (P^\prime x_t)^\prime.\cr}$$
It follows that
$$E\  [v_{t+1} v^\prime_{t+1} \mid x_t] = \hbox{diag } (P^\prime x_t) -
P^\prime \hbox{diag } x_t P.$$
Similarly,
$$E\ [u_t\ u^\prime_t \mid x_t] = \hbox{diag } (Q^\prime x_t) - Q^\prime
\hbox{diag } x_t Q.$$}



\section{Nonlinear filtering}
We seek a recursive formula for  computing the conditional
distribution of the hidden state:
%
%
$$\rho_i (t) = {\rm Prob}\{x_t = i\mid y_1=\eta_1, \ldots, y_t = \eta_t\}.$$
%$$P_{ij} (t) = {\rm Prob}\{x_t = j\mid x_{t-1}=i\}$$
%$$\sum_j P_{ij}(t) = 1\ ,\quad P_{ij} (t) \geq 0$$
%$$q_{ij} = {\rm Prob} \{y_t =
%    j\mid x_t=i\}\ ,\quad \sum_j q_{ij}=1,\quad q_{ij}\geq
%0$$
   Denote the history of observed $y_t$'s up to $t$ as
$\eta^t = \hbox{ col } (\eta_1,\ldots,\eta_t).$
Define the conditional probabilities
$$p(\xi_t,\eta_1,\ldots,\eta_t) = {\rm Prob} \ (x_t=\xi_t, y_1= \eta_1,\ldots,
y_t=\eta_t),$$
and assume $p(\eta_1,\ldots, \eta_t)\not= 0$.
Then apply the calculus of conditional expectations to compute\NFootnote{Notice
 that
$$\eqalign{p(\xi_t,\eta_t \mid \eta^{t-1}) &= \sum_{\xi_{t-1}}
 p(\xi_t, \eta_t,
\xi_{t-1}\mid \eta^{t-1}) \cr\noalign{\medskip}
&= \sum_{\xi_{t-1}} p (\xi_t, \eta_t\mid \xi_{t-1}, \eta^{t-1})
p (\xi_{t-1}\mid \eta^{t-1})\cr}$$

$$\eqalign{p(\xi_t,\eta_t \mid \xi_{t-1}, \eta^{t-1}) &=
p(\xi_t \mid \xi_{t-1}, \eta^{t-1}) p(\eta_t \mid \xi_t,\ \xi_{t-1},
\eta^{t-1})\cr \noalign{\smallskip}
&= p(\xi_t\mid \xi_{t-1}) p(\eta_t\mid \xi_t) . \cr}$$
Combining these results gives the formula in the text.}
$$\eqalign{p(\xi_t\mid \eta^t) &= {p(\xi_t, \eta_t\mid \eta^{t-1})\over
p (\eta_t\mid \eta^{t-1})}\cr
        & = { \sum_{\xi_{t-1}} p(\eta_t \mid \xi_t)\ p(\xi_t
\mid \xi_{t-1}) p(\xi_{t-1}\mid \eta^{t-1})\over
\sum_{\xi_t} \sum_{\xi_{t-1}} p(\eta_t\mid \xi_t) p(\xi_t \mid \xi_{t-1})
p(\xi_{t-1}\mid \eta^{t-1})} .\cr}$$
This result can be written
$$\rho_i (t+1) = {\sum_s Q_{ij} P_{si} \rho_s(t)\over \sum_s \sum_i Q_{ij}
P_{si} \rho_s(t)},  $$
where $\eta_{t+1} = j$ is the value of $y$ at $t+1$
%$$\rho_i(t+1) = \hbox{Prob } \{x_{t+1} = i\mid y_{t+1} = j, \eta^{t-1}\}$$
We can represent  this recursively as
$$\eqalign{\tilde \rho(t+1) &= \hbox{ diag } (Q_j) P^\prime
 \rho(t)\cr \rho(t+1) &=
{\tilde \rho(t+1)\over < \tilde \rho(t+1), \underline{1}>} . \cr}$$
where $Q_j$ is the $j$th column of $Q$, and diag$(Q_j)$ is a diagonal matrix
with $Q_{ij}$ as the $i$th diagonal element; here $< \cdot, \cdot >$ denotes
the inner product of two vectors, and $\underline 1$ is the
unit vector.

\chapternum=0
