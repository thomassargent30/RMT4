\input grafinp3
\input psfig
%\eqnotracetrue
\offparens

\footnum=0
\overfullrule=0pt

\def\bull{\vrule height .9ex width .8ex depth -.1ex}
\def\bh{\penalty-100}
\footnum=0
\chapter{Equilibrium without Commitment \label{socialinsurance2}}
%%%%%%%%%%%\chapter{Enforcement and Equilibrium \label{socialinsurance2}}

\index{commitment!two-sided lack of}
\section{Two-sided lack of commitment}
In section \use{sec:moneylender1} of the previous chapter, we studied
insurance without commitment. That was a ``small open economy''  analysis
since the moneylender could borrow or lend resources outside
of the village at a given interest rate. Recall also the asymmetry
in the environment where villagers could not make any commitments while
the moneylender was assumed to be able to commit.
We will now study a closed system without access to an outside credit
market. Any household's consumption in excess of its own endowment
must then come from the endowments of  other households in the
economy. We will also adopt the symmetric assumption that no one
is able to make
commitments. That is, any contract prescribing an exchange of goods
today in anticipation of future exchanges of goods represents a
sustainable allocation only if current and future exchanges satisfy participation constraints
for all households involved in the contractual
arrangement. Households are free to walk away from the arrangement
at any point in time and thereafter to live in autarky. Such a contract design
problem with participation constraints on both sides of an exchange
represents a problem with two-sided lack of commitment, as compared to
the problem with one-sided lack of commitment in
section \use{sec:moneylender1}.
%% MOVED to black15.tex \NFootnote{For an earlier two-period
%%                      model of a one-sided
%%commitment problem, see Holmstr\"om (1983).} \auth{Holmstr\"om,
%%Bengt}

This chapter draws on the work of Thomas and Worrall (1988,
1994) and Kocherlakota (1996b). At the end of the chapter, we also
discuss market arrangements for decentralizing the constrained
Pareto optimal allocation, as studied by Kehoe and Levine (1993)
and Alvarez and Jermann (2000).



\section{A closed system}\label{sec:closed}%
Thomas and Worrall's (1988) model
\auth{Thomas, Jonathan}\auth{Worrall, Tim}%
of self-enforcing wage contracts is an
antecedent to our villager-moneylender environment.
  The counterpart to our moneylender  in their model is  a risk-neutral firm that forms a long-term relationship
with a risk-averse worker. In their model, there is also a competitive spot
market for labor where a worker is paid $y_t$ at time $t$. The
worker is always free to walk away from the firm and work in that
spot market.  But if he does, he can never again enter into a long-term
relationship with another firm.  The firm seeks to maximize the
discounted stream of expected future profits by designing a
long-term wage contract that is self-enforcing in the sense that it never gives
the worker
an incentive to quit.  In a contract that
stipulates a wage $c_t$ at time $t$, the firm earns time $t$
profits of $y_t - c_t$
(as compared to hiring a worker in the spot market for labor).
If Thomas and Worrall had assumed a commitment problem only on the part of
the worker, their model  would be formally identical to our
villager-moneylender environment.
  However, Thomas and Worrall also assume that
the firm itself can renege on a wage contract and buy labor at the
random spot market wage. Hence, they require that a self-enforcing wage contract
 be one in which neither party ever has an incentive to
renege.

Kocherlakota
\auth{Kocherlakota, Narayana R.}%
(1996b) studies a model that has important features in common with
Thomas and Worrall's.\NFootnote{The working paper of Thomas and Worrall
(1994) also analyzed a multiple agent closed model like Kocherlakota's.
Thomas and Worrall's (1994) analysis evolved into an article by
Ligon, Thomas, and Worrall (2002) that we discuss in section \use{LTW2002}.} %
 Kocherlakota's  counterpart to Thomas and Worrall's   firm is  a  risk-averse second household.
  In Kocherlakota's  model, two
 households
receive stochastic endowments. The \idx{contract design} problem
is  to find an insurance/transfer arrangement that reduces
consumption risk while respecting participation constraints: both households must be induced each period not
to walk away from the arrangement. Kocherlakota
uses his model in an interesting way to help interpret empirically
estimated conditional consumption-income covariances that seem to
violate the hypothesis of complete risk sharing. Kocherlakota
investigates the extent to which those failures reflect
impediments to enforcement that are captured by his participation
constraints. \auth{Thomas, Jonathan}
\auth{Worrall, Tim}


For the purpose of studying those conditional covariances in a stationary stochastic environment,
Kocherlakota's use of an environment with two-sided lack of commitment is important.
In our model of villagers facing a moneylender in section \use{sec:moneylender1},
imperfect risk
sharing is temporary and so would not prevail in a stochastic steady state.
In Kocherlakota's model, imperfect risk sharing can be perpetual.
There are equal numbers of two types of households in
the village.  Each of the households has the preferences,
endowments, and autarkic utility possibilities described
in chapter \use{socialinsurance}. %%earlier.
%%In addition,  we now impose an Inada condition $$
%%\lim_{c \searrow 0} u'(c) = +\infty$$ that is designed to keep
%%consumers off corners.
Here we assume that the endowments of the two types of households
are perfectly negatively correlated. Whenever a household of type
1 receives $\overline y_s$, a household of type 2 receives $1 -
\overline y_s$. We assume that  $y_t$ is independently and
identically distributed according to the discrete probability
distribution ${\rm Prob} (y_t = \overline y_s) = \Pi_s$, where we
assume that $\overline y_s \in [0,1]$. We also assume  that the
$\Pi_s$'s are such that the distribution of $y_t$ is identical to
that of $1-y_t$.
%%\NFootnote{This last assumption guarantees an
%%identical reservation value $v_{\rm aut}$ for both types of
%%consumer; it is easy to  relax this assumption.}
  Also, now the planner has
access to neither borrowing nor lending opportunities,
and is confined to reallocating consumption goods
between the two types of households. This limitation leads to two
participation constraints.  At time $t$, the type 1 household
receives endowment $y_t$ and consumption $c_t$,
while the type 2 household receives $1-y_t$ and
$1-c_t$.

  In this setting, an allocation is said to be
{\it sustainable\/}\NFootnote{Kocherlakota says
{\it subgame perfect\/}   rather than
{\it sustainable}.}
if for all $t\geq 0$ and for all histories $h_t$
$$ \EQNalign{
 u(c_t) - u(y_t) + &\beta E_t \sum_{j=1}^\infty \beta^{j-1}
\left[ u(c_{t+j}) - u(y_{t+j}) \right] \geq 0,       \EQN sustain1;a \cr
    u(1-c_t) -  u(1-y_t) + &\beta E_t \sum_{j=1}^\infty \beta^{j-1}
\left[ u(1-c_{t+j}) - u(1-y_{t+j}) \right] \geq 0. \hskip1.3cm \EQN sustain1;b \cr}$$

Let $\Gamma$ denote the set of sustainable allocations.
%Denote by $v_{\rm max}$ the maximal expected utility to agent 1 among
%all sustainable allocations.
We seek the following function:
$$ \EQNalign { Q(\triangle)  = &\max_{\{c_t\}} E_{-1}  \sum_{t=0}^\infty
  \beta^t \left[ u(1-c_t) - u(1-y_t) \right] \EQN Pdef;a   \cr
\noalign{\hbox{\rm subject to }}
 \ &  \{c_t\}  \in \Gamma, \EQN Pdef;b \cr
       \ & E_{-1} \sum_{t=0}^\infty \beta^t \left[ u(c_t) - u(y_t) \right]
                                          \geq \triangle.
  \EQN Pdef;c  \cr} $$
The function $Q(\triangle)$  depicts a (constrained) Pareto
frontier by  portraying the maximized value of the expected
lifetime utility of the type 2 household subject to  the type 1 household receiving an
expected lifetime utility that exceeds its autarkic welfare level
by at least $\triangle$ utils. To find this Pareto frontier, we
first solve for the consumption dynamics that characterize all
efficient contracts. From these optimal consumption dynamics, it
will be straightforward to compute the {\it ex ante} division of
 gains from an efficient contract.


 \index{contract
design!dynamic program}
\section{Recursive formulation}
We choose to study Kocherlakota's model using the approach %a method like that
proposed by Thomas and Worrall.\NFootnote{Kocherlakota instead extended the approach that we used in the
villager-moneylender model of section
\use{sec:moneylender1} to an environment with two-sided lack of commitment. We followed Kocherlakota
in chapter 15 of the first edition of this book.
%For some special examples with two-sided lack of commitment, this approach
%encounters a technical difficulty associated with possible kinks in the Pareto frontier.
However, when applied to problems with two-sided lack of commitment,
this approach encounters a technical difficulty associated with possible
kinks in the Pareto frontier.
(We first encountered this difficulty when we assigned a version of exercise
{\it \the\chapternum.3\/} to our students.)
%We seem to side-step this non-differentiability
%problem  by using Thomas and Worrall's approach.
Thomas and Worrall's approach avoids this nondifferentiability
problem by using conditional Pareto frontiers, one for each
realization of the endowment.} %
Thomas and Worrall (1988) formulate
the contract design problem as a dynamic program, where the state
of the system {\it prior} to the current period's endowment
realization is given by a vector $[x_1 \; x_2 \; \ldots \;x_s\;
\ldots \; x_S]$.
% $\{x_s\}_{s=1}^S$.
Here $x_s$ is the value of the expression on the left side of
\Ep{sustain1;a} that is promised to a type 1 agent conditional on
the current period's endowment realization being $\overline y_s$.
Let $Q_s(x_s)$ then denote the corresponding value of expression
\Ep{sustain1;b} that is promised to a type 2
agent.\NFootnote{$Q_s(\cdot)$ is a Pareto frontier conditional on
the endowment realization $\overline y_s$, while $Q(\cdot)$ in
\Ep{Pdef;a} is an {\it ex ante} Pareto frontier before observing
any endowment realization.} When the endowment realization
$\overline y_s$ is associated with a promise to a type 1 agent
equal to $x_s=x$, we can write the Bellman equation as
$$ Q_s(x) = \max_{c,\,\{\chi_j\}_{j=1}^S}
\bigl\{ u(1 - c) - u(1 - \overline y_s)
                  + \beta   \sum_{j=1}^S \Pi_j Q_j(\chi_j) \bigr\} \EQN new1;a  $$
subject to
$$\EQNalign{  & u(c) - u(\overline y_s) + \beta \sum_{j=1}^S \Pi_j  \chi_j  \geq x,
                                                                         \EQN new1;b \cr
             & \chi_j \geq 0,       \hskip1.5cm    j=1, \ldots, S;          \EQN new1;c \cr
          &    Q_j(\chi_j) \geq 0,  \hskip.8cm j=1, \ldots, S;  \EQN new1;d   \cr
             &  c \in [0, 1],  \EQN new1;e \cr}
$$
where expression \Ep{new1;b} is the promise-keeping constraint,
expression \Ep{new1;c} is the participation constraint for the
type 1 agent, and expression \Ep{new1;d} is the participation constraint
for the type 2 agent. The set of feasible $c$ is given by
expression \Ep{new1;e}.

\auth{Thomas, Jonathan} \auth{Worrall, Tim}
 Thomas and Worrall
prove the existence of a compact interval that contains all
permissible continuation values $\chi_j$:
$$
\chi_j  \in [0, \overline x_j] \ \hbox{\rm for }\;j = 1, 2, \ldots, S.
                                                                \EQN new1;f
$$
Thomas and Worrall also show that the Pareto-frontier $Q_j(\cdot)$
is decreasing, strictly concave, and continuously differentiable
on $[0, \overline x_j]$. The bounds on $\chi_j$ are motivated as
follows. The contract cannot award the type 1 agent a value of
$\chi_j$ less than zero because that would correspond to an
expected future lifetime utility below the agent's autarky level.
There exists an upper bound $\overline x_j$ above which the
planner would never find it optimal to award the type 1 agent a
continuation value conditional on next period's endowment
realization being $\overline y_j$. It would simply be impossible
to deliver  a higher continuation value because of the
participation constraints. In particular, the upper bound
$\overline x_j$ is such that
$$ Q_j(\overline x_j) = 0.                                  \EQN neww1
$$
Here a type 2 agent  receives an expected lifetime utility
equal to his autarky level if the next period's endowment
realization is $\overline y_j$ and a type 1 agent is
promised the upper bound $\overline x_j$. Our two- and three-state examples
in sections \use{sec:thekink} and \use{sec:3state} illustrate what
determines $\overline x_j$.


   Attach Lagrange multipliers $\mu$, $\beta \Pi_j \lambda_j$,
and $\beta \Pi_j \theta_j$ to expressions \Ep{new1;b},
\Ep{new1;c}, and \Ep{new1;d}, then get the following first-order conditions for $c$
and $\chi_j$:\NFootnote{Here we are proceeding under the
conjecture that the nonnegativity constraints on consumption in
\Ep{new1;e}, $c\geq 0$ and $1-c\geq 0$, are not binding. This
conjecture is confirmed below when it is shown that optimal
consumption levels satisfy $c\in[\overline y_1,\, \overline y_S]$.\label{nonneg}}
$$\EQNalign{ c\,\rm{:}& \hskip.25cm
     -u'(1-c) + \mu u'(c) =0,
                \EQN ffonc1;a \cr
             \chi_j\,\rm{:}& \hskip.25cm
              \beta \Pi_j Q'_j(\chi_j) + \mu \beta \Pi_j + \beta \Pi_j \lambda_j
              +\beta \Pi_j \theta_j Q'_j(\chi_j) = 0 .
                  \EQN ffonc1;b \cr }$$
By the envelope theorem,
$$ Q'_s(x)=-\mu.       \EQN envel
$$
After substituting \Ep{envel} into \Ep{ffonc1;a} and
\Ep{ffonc1;b}, respectively, the optimal choices of $c$ and
$\chi_j$ satisfy
$$\EQNalign{
 Q'_s(x) & =  -{u'(1 - c) \over u'(c)},\EQN marg1;a \cr
 Q'_s(x) & =  (1+\theta_j) Q'_j(\chi_j) + \lambda_j . \EQN marg1;b \cr}
$$

\section{Equilibrium consumption}\label{sec:equil_consumption}%
\vskip-.5cm
\subsection{Consumption dynamics}\label{sec:consdynamic}%
From equation \Ep{marg1;a}, the consumption $c$ of a type 1 agent
is an increasing function of the promised value $x$. The
properties of the Pareto frontier $Q_s(x)$ imply that $c$ is a
differentiable function of $x$ on $[0, \overline x_s]$. Since
$x\in[0,\overline x_s]$, $c$ is contained in
the nonempty compact interval $[\underline c_s, \overline c_s]$,
where
$$
Q'_s(0) = -{u'(1 - \underline c_s) \over u'(\underline c_s)}
\hskip1cm \hbox{\rm and } \hskip1cm
Q'_s(\overline x_s) = -{u'(1 - \overline c_s) \over u'(\overline c_s)} .
$$
Thus, if $c=\underline c_s$, $x=0$, so that a type 1 agent gets no
gain from the contract from then on. If
$c=\overline c_s$, $Q_s(x) = Q_s(\bar x_s)=0$, so that a type 2
agent gets no gain.


Equation \Ep{marg1;a} can be expressed as
$$
c = g\!\left(Q'_s(x)\right),                      \EQN g_func
$$
where $g$ is a continuously and strictly decreasing function.
By substituting the inverse of that function into equation \Ep{marg1;b},
we obtain the expression
$$
g^{-1}(c) = (1+\theta_j) \,g^{-1}(c_j) + \lambda_j,  \EQN c_func
$$
where $c$ is  again the current consumption of a type 1 agent
and $c_j$ is his next period's consumption when next
period's endowment realization is $\overline y_j$. The optimal
consumption dynamics implied by an efficient contract are
evidently governed by whether or not agents' participation
constraints are binding. For any given endowment realization
$\overline y_j$  next period, only one of the participation
constraints in \Ep{new1;c} and \Ep{new1;d} can bind. Hence, there
are three regions of interest for any given realization $\overline
y_j$:

\medskip
\noindent{1.} Neither participation constraint binds.  When
$\lambda_j = \theta_j =0$, the consumption dynamics in \Ep{c_func}
satisfy
$$
g^{-1}(c) = g^{-1}(c_j) \hskip1cm \Longrightarrow \hskip1cm c=c_j ,
$$
where $c=c_j$ follows from the fact that $g^{-1}(\cdot)$ is a
strictly decreasing function. Hence, consumption is independent of
the endowment and the agents are offered full insurance against
endowment realizations so long as there are no binding
participation constraints. The constant consumption allocation is
determined by the ``temporary relative Pareto weight'' $\mu$
in equation \Ep{ffonc1;a}.

\medskip
\noindent{2.}   The participation constraint of a type 1 person
binds ($\lambda_j > 0$), but $\theta_j =0$. Thus,
condition \Ep{c_func} becomes
$$
g^{-1}(c) = g^{-1}(c_j) + \lambda_j  \hskip.5cm \Longrightarrow
\hskip.5cm g^{-1}(c) > g^{-1}(c_j)  \hskip.5cm \Longrightarrow
\hskip.5cm c<c_j.
$$
The planner raises the consumption of the type 1 agent in order
to satisfy his participation constraint. The strictly positive Lagrange
multiplier, $\lambda_j>0$, implies that \Ep{new1;c} holds with
equality, $\chi_j=0$.  That is, the planner raises the welfare of
a type 1 agent just enough to make her indifferent between
choosing autarky and staying with the optimal insurance contract.
In effect, the planner minimizes the change in last period's
relative welfare distribution that is needed to induce the type 1
agent not to abandon the  contract. The welfare of the type 1
agent is raised both through the mentioned higher consumption
$c_j>c$ and through the expected higher future consumption. Recall
our earlier finding that implies that the new higher consumption
level will remain unchanged so long as there are no binding
participation constraints. It follows that the contract for agent
1 displays {\it amnesia\/}  when agent 1's participation
constraint is binding, because the previously promised value $x$
becomes irrelevant for the consumption allocated to agent 1 from
now on.
\index{amnesia!of risk-sharing contract}%




\medskip
\noindent{3.}   The participation constraint of a type 2 person
binds ($\theta_j > 0$), but $\lambda_j =0$. Thus,
condition \Ep{c_func} becomes
$$
g^{-1}(c) = (1+\theta_j) \,g^{-1}(c_j)  \hskip.5cm \Longrightarrow
\hskip.5cm g^{-1}(c) < g^{-1}(c_j) \hskip.5cm \Longrightarrow
\hskip.5cm c>c_j,
$$
where we have used the fact that $g^{-1}(\cdot)$ is a negative
number. This situation is the mirror image of the previous case.
When the participation constraint of the type 2 agent binds, the
planner induces the agent to remain with the optimal contract by
increasing her consumption $(1-c_j)>(1-c)$ but only by enough that
she remains indifferent to the alternative of choosing autarky,
$Q_j(\chi_j)=0$. And once again, the change in the welfare
distribution  persists in the sense that the new consumption level
will remain unchanged so long as there are no binding
participation constraints.  The amnesia property prevails again.

\vskip.3cm

We can assemble  these results to characterize the contract by
arguing that when $c < \underline c_j$, the participation
constraint of the type 1 agent binds, while if $c > \overline
c_j$, the participation constraint of a type 2 agent binds. Thus,
assume that $c < \underline c_j$.  Then since it must be that $c_j
\geq \underline c_j$, it follows that $c \leq \underline c_j$, so
we must be in the region where the participation constraint of the
type 1 agent binds, which in turn implies that $c_j= \underline
c_j$.  A symmetric argument that applies when the participation
constraint of a type 2 argument applies, allowing us to summarize
the consumption dynamics of an efficient contract as follows.
Given the current consumption $c$ of the type 1 agent, next
period's consumption conditional on next period's  endowment realization
$\overline y_j$ satisfies
$$
c_j = \cases{ \underline c_j &if $c<\underline c_j$ \hskip1cm (p.c. of type 1 binds), \cr
              c              &if $c\in[\underline c_j, \overline c_j]$
                                        \hskip.35cm (p.c. of neither type binds), \cr
              \overline c_j &if $c>\overline c_j$  \hskip1cm (p.c. of type 2 binds). \cr}
 \EQN c_update
$$


\subsection{Consumption intervals cannot contain
  each other}\label{sec:intervals_contain}%
  We will show that
$$
\overline y_k > \overline y_q \hskip.5cm \Longrightarrow \hskip.5cm
\overline c_k > \overline c_q \hskip.25cm \hbox{\rm and} \hskip.25cm
\underline c_k > \underline c_q.                                 \EQN c_intervals
$$
Hence, no consumption interval can contain another. Depending on
parameter values, the consumption intervals can  be
either overlapping or disjoint.

As an intermediate step, it is useful to first verify that the
following assertion is correct for any $k, q = 1, 2, \ldots, S$,
and for any $x\in[0, \overline x_q]$:
$$ Q_k\!\left(x+u(\overline y_q) - u(\overline y_k) \right)
= Q_q(x) + u(1-\overline y_q) - u(1-\overline y_k).    \EQN intermediate_step
$$
After invoking  functional equation \Ep{new1}, the left side of
\Ep{intermediate_step} is equal to
$$ Q_k\!\left(x+u(\overline y_q) - u(\overline y_k) \right) = \max_{c,\,\{\chi_j\}_{j=1}^S}
\bigl\{ u(1 - c) - u(1 - \overline y_k)
                  + \beta   \sum_{j=1}^S \Pi_j Q_j(\chi_j) \bigr\}   $$
%\vskip-.3cm
subject to        %$$\EQNalign{  \hbox{subject to} \hskip1cm
$$
 u(c) - u(\overline y_k) + \beta \sum_{j=1}^S \Pi_j  \chi_j \geq
x  +u(\overline y_q) - u(\overline y_k) $$
and \Ep{new1;c} -- \Ep{new1;e};
and the right side of \Ep{intermediate_step} is equal to
$$\EQNalign{
&Q_q(x) + u(1-\overline y_q) - u(1-\overline y_k)  \cr
         &= \max_{c,\,\{\chi_j\}_{j=1}^S}
\bigl\{ u(1 - c) - u(1 - \overline y_q)
                  + \beta   \sum_{j=1}^S \Pi_j Q_j(\chi_j) \bigr\}
                           + u(1-\overline y_q) - u(1-\overline y_k) \cr}
$$
%\vskip-.3cm
subject to     %$$\EQNalign{  \hbox{subject to} \hskip1cm
$$
 u(c) - u(\overline y_q) + \beta \sum_{j=1}^S \Pi_j  \chi_j \geq
x   $$
and \Ep{new1;c} -- \Ep{new1;e}.
We can then verify  \Ep{intermediate_step}.\NFootnote{The
two optimization problems on the left  and
the right sides of expression \Ep{intermediate_step}
share the common objective of maximizing the expected utility of
the type 2 agent, minus an identical constant. The optimization is
subject to the same constraints, $u(c) - u(\overline y_q) + \beta
\sum_{j=1}^S \Pi_j \chi_j \geq x$ and \Ep{new1;c} -- \Ep{new1;e}.
Hence, they are identical  well-defined optimization problems. The
observant reader should not be concerned with the fact that
$Q_k(\cdot)$ on the left side of \Ep{intermediate_step} might be
evaluated at a promised value outside of the range $[0, \bar
x_k]$. This causes no problem because  the
optimization problem  imposes no  participation constraint
in the current period, in contrast to the restrictions on future
continuation values in \Ep{new1;c} and \Ep{new1;d}.} And after
differentiating that expression with respect to $x$,
$$ Q'_k\!\left(x+u(\overline y_q) - u(\overline y_k) \right)
= Q'_q(x)  .                                              \EQN intermediate_step2
$$

To show that $\overline y_k > \overline y_q$ implies
$\overline c_k > \overline c_q$,\ set $x=\overline x_q$ in expression
\Ep{intermediate_step},
$$ Q_k\!\left(\overline x_q +u(\overline y_q) - u(\overline y_k) \right)
= u(1-\overline y_q) - u(1-\overline y_k) > 0,           \EQN intermediate_step3
$$
where we have used $Q_q(\overline x_q) = 0$. After also invoking
$Q_k(\overline x_k) = 0$ and the fact that $Q_k(\cdot)$ is
decreasing, it follows from
$ Q_k\!\left(\overline x_q +u(\overline y_q) - u(\overline y_k) \right) > 0$
that
$$
\overline x_k > \overline x_q +u(\overline y_q) - u(\overline y_k).
$$
So by the strict concavity of $Q_k(\cdot)$, we have
$$
Q'_k(\overline x_k) < Q'_k\!\left(\overline x_q +u(\overline y_q) - u(\overline y_k)\right)
= Q'_q(\overline x_q),                         \EQN intermediate_extra
$$
where the equality is given by \Ep{intermediate_step2}.
Finally, by using function \Ep{g_func} and the present
finding that $Q'_k(\overline x_k) < Q'_q(\overline x_q)$, we can verify our
assertion that
$$
\overline c_k = g\!\left(Q'_k(\overline x_k)\right) \; > \;
g\!\left(Q'_q(\overline x_q)\right) = \overline c_q.
$$


 We leave it to the reader as an
exercise to construct a symmetric argument to show that $\overline
y_k > \overline y_q$ implies $\underline c_k > \underline c_q$.



\subsection{Endowments are contained in the consumption intervals}\label{sec:y_in_c}%
We will show that
$$
\overline y_s \in [\underline c_s, \overline c_s], \hskip.25cm
\forall s; \hskip.5cm \hbox{\rm and} \hskip.5cm
\overline y_1 = \underline c_1 \hskip.25cm \hbox{\rm and} \hskip.25cm
\overline y_S = \overline c_S.                          \EQN y_c_intervals
$$
First, we show that $\overline y_s \leq \overline c_s$ for all $s$; and
$\overline y_S = \overline c_S$. Let $x=\overline x_s$ in the functional
equation \Ep{new1}, then $c=\overline c_s$ and
$$ u(1 - \overline c_s) - u(1 - \overline y_s)
                  + \beta   \sum_{j=1}^S \Pi_j Q_j(\chi_j) = 0  \EQN optimum
$$
with $\{\chi_j\}_{j=1}^S$ being optimally chosen. Since $Q_j(\chi_j)\geq 0$, it
follows immediately that
$$ u(1 - \overline c_s) - u(1 - \overline y_s) \leq 0
\hskip.5cm \Longrightarrow \hskip.5cm \overline y_s \leq \overline c_s.
$$
%{\bf XXX: Lars. A student spotted that the inequalities below should be reversed (as I have done)
%from what is in RMT2. Agree?}
To establish strict equality for $s=S$, we note that
$$
Q'_j(\chi_j) \geq Q'_j(\overline x_j) \geq  Q'_S(\overline x_S) ,
$$
where the first weak inequality follows from the fact that all
permissible $\chi_j \leq \overline x_j$ and $Q_j(\cdot)$ is
strictly concave, and the second weak inequality is given by
\Ep{intermediate_extra}. In fact, we showed above that the second
inequality holds strictly for $j<S$ and therefore, by the
condition for optimality in \Ep{marg1;b},
$$
 Q'_S(\overline x_S) =  (1+\theta_j) Q'_j(\chi_j)
\hskip.5cm \hbox{\rm with} \hskip.5cm \theta_j>0, \hskip.15cm\hbox{\rm for } j<S;
\hskip.5cm \hbox{\rm and} \hskip.5cm \theta_S=0,
$$
which imply $\chi_j=\overline x_j$ for all $j$. After also invoking the corresponding
expression \Ep{optimum} for $s=S$, we can complete the argument:
$$
\beta \sum_{j=1}^S \Pi_j Q_j(\overline x_j) = 0
\hskip.5cm \Longrightarrow \hskip.5cm
 u(1 - \overline c_S) - u(1 - \overline y_S) = 0
\hskip.5cm \Longrightarrow \hskip.5cm \overline y_S = \overline c_S.
$$

We leave it as an exercise for the reader to construct a symmetric
argument showing that $\overline y_s \geq \underline c_s$ for all
$s$; and $\overline y_1 = \underline c_1$.



\subsection{All consumption intervals are nondegenerate
 (unless autarky is the only sustainable allocation)}\label{sec:c_nondeg}%
Suppose that the consumption interval associated with endowment
realization $\overline y_k$ is degenerate, i.e., $\underline c_k =
\overline c_k = \overline y_k$. (The last inequality follows from
section \use{sec:y_in_c}, where we established that the endowment
is contained in the consumption interval.) Since the consumption
interval is degenerate, it follows that the range of permissible
continuation values associated with endowment realization
$\overline y_k$ is also degenerate, i.e., $\chi_k  \in [0,
\overline x_k]= \{0\}$. Recall that $\chi_k$ is the number of
utils awarded to the type 1 household over and above its autarkic
welfare level, given endowment realization $\overline y_k$:
$$
0=\chi_k = u(\overline y_k) - u(\overline y_k)
           + \beta \sum_{j=1}^S \Pi_j \,\chi_j ,
$$
where we have invoked the degenerate consumption interval,
$c=\overline y_k$,
and where $\chi_j$ are optimally chosen subject to the constraints
$\chi_j\geq 0$ for all $j\in {\bf S}$. It follows immediately that
$\chi_j=0$ for all $j\in {\bf S}$, given the current endowment
realization $\overline y_k$.

Due to the degenerate range of continuation values associated
with endowment realization $\overline y_k$, i.e.,
$\chi_k  \in \{0\}$, it must be the case that the type 2
household also receives its autarkic welfare level, given
endowment realization $\overline y_k$:\NFootnote{Obviously, there
would be a contradiction if the type 2 household were to receive
$Q_k(\chi_k)>0$. The reason is that then it would be possible to transfer current
consumption from the type 2 household to the type 1 household without
violating the type 2 household's participation constraint,
and hence the consumption interval associated with
endowment realization $\overline y_k$ could not be degenerate.}
$$
0=Q_k(\chi_k) = u(1-\overline y_k) - u(1-\overline y_k)
           + \beta \sum_{j=1}^S \Pi_j \,Q_j(\chi_j)
$$
where we have invoked the degenerate consumption interval,
and where continuation values $Q_j(\chi_j)$ are subject to the constraints
$Q_j(\chi_j)\geq 0$ for all $j\in {\bf S}$. It follows immediately that
$Q_j(\chi_j)=0$ for all $j\in {\bf S}$, given the current endowment
realization $\overline y_k$.
Hence, given endowment realization $\overline y_k$, we have
continuation values of the type 2 household
satisfying $Q_j(\chi_j)=0$, so it must be the case that
the optimally chosen $\chi_j$ are set at their maximum
permissible values, $\chi_j = \overline x_j$. Moreover,
we know from above that the optimal values also satisfy $\chi_j=0$, and
therefore we can conclude that $\overline x_j=0$ for all $j\in {\bf S}$.

We have shown that if one consumption interval is degenerate, then
all consumption intervals must be degenerate, i.e., $\underline
c_s = \overline c_s = \overline y_s$ for all $s\in {\bf S}$. This
finding seems rather intuitive. A degenerate consumption interval
associated with any endowment realization $\overline y_k$ implies
that, given the realization of $\overline y_k$, none of the
households has anything to gain from the optimal contract,
neither from current transfers nor from future risk sharing. That
can only happen if autarky is the only sustainable allocation.



\section{Pareto frontier and ex ante division of the gains}\label{sec:Koch_Pareto1}%
We have characterized the optimal consumption dynamics of any
efficient contract. The consumption intervals $\{[\underline c_j,
\overline c_j]\}_{j=1}^S$ and the updating rules in \Ep{c_update}
are identical for all efficient contracts. The {\it ex ante}
division of  gains from an efficient contract can be viewed as
being determined by an implicit past consumption level,
$c_{\triangle}\in[\underline c_1, \overline c_S]$: by
\Ep{y_c_intervals}, this can also be written as
$c_{\triangle}\in[\overline y_1, \overline y_S]$). A contract with
an implicit past consumption level $c_{\triangle}=\underline c_1$ gives
all of the surplus to the type 2 agent and none to the type 1
agent. This follows immediately from the updating rules in
\Ep{c_update} that prescribe a first-period consumption level
equal to $\underline c_j$ if the endowment realization is
$\overline y_j$. The corresponding promised value to the type 1
agent, conditional on endowment realization $\overline y_j$, is
$\chi_j = 0$. Thus, the {\it ex ante} gain to the type 1 agent in
expression \Ep{Pdef;c} becomes
$$
\triangle \Bigl|_{c_{\triangle}=\underline c_1}
   = \sum_{j=1}^S \Pi_j \chi_j \Bigl|_{c_{\triangle}=\underline c_1}= 0.
$$
Similarly, we can  show that a contract with an implicit consumption
level $c_{\triangle}=\overline c_S$ gives all of the surplus to
the type 1 agent and none to the type 2 agent. The updating rules
in \Ep{c_update} will then prescribe a first-period consumption
level equal to $\overline c_j$ if the endowment realization is
$\overline y_j$ with a corresponding promised value of $\chi_j =
\overline x_j$. We can compute the {\it ex ante} gain to the type
1 agent as
$$
\triangle \Bigl|_{c_{\triangle}=\overline c_S}
          = \sum_{j=1}^S \Pi_j \overline x_j \equiv \triangle_{\rm max}.
$$

For these two endpoints of the interval
$c_{\triangle}\in[\underline c_1, \overline c_S]$, the {\it ex ante}
gains attained by the type 2 agent in expression \Ep{Pdef;a} become
$$\EQNalign{
Q(\triangle) \Bigl|_{c_{\triangle}=\underline c_1}
          &= Q(0) = \sum_{j=1}^S \Pi_j Q_j(0) = \triangle_{\rm max}, \cr
Q(\triangle) \Bigl|_{c_{\triangle}=\overline c_S}
          &= Q(\triangle_{\rm max}) = \sum_{j=1}^S \Pi_j Q_j(\overline x_j) = 0,  \cr}
$$
where the equality $Q(0) = \triangle_{\rm max}$ follows from the
symmetry of the environment with respect to the type 1 and type 2 agents'
preferences and endowment processes.


\section{Consumption distribution}
\vskip-.5cm
\subsection{Asymptotic distribution}\label{sec:Koch_asympdist}%
The asymptotic consumption distribution depends sensitively on
whether  there exists a first-best sustainable allocation.
  We say that a sustainable allocation
is {\it first best\/} if the participation constraint of neither
agent ever binds.  As we have seen, nonbinding participation
constraints imply that consumption remains constant over time.
Thus, a first-best sustainable allocation can  exist only if the
intersection of all the consumption intervals $\{[\underline c_j,
\overline c_j]\}_{j=1}^S$ is nonempty. Define the following two
critical numbers
$$\EQNalign{
\overline c_{\rm min} &\equiv \min \{\overline c_j\}_{j=1}^S
                                       = \overline c_1, \EQN c_int_bound;a \cr
\underline c_{\rm max}&\equiv \max \{\underline c_j\}_{j=1}^S
                                       = \underline c_S, \EQN c_int_bound;b \cr}
$$
where the two equalities are implied by \Ep{c_intervals}. A
necessary and sufficient condition for the existence of a
first-best sustainable allocation is that $\overline c_{\rm min}
\geq \underline c_{\rm max}$. Within a first-best sustainable
allocation, there is complete risk sharing.

 For high enough values of $\beta$, sufficient endowment
risk, and  enough curvature of $u(\cdot)$, there will exist a set
of first-best sustainable allocations, i.e., $\overline c_{\rm
min} \geq \underline c_{\rm max}$. If the {\it ex ante} division
of the gains is then given by an implicit initial consumption
level $c_{\triangle}\in[\underline c_{\rm max},\,\overline c_{\rm
min}]$, it follows by the updating rules in \Ep{c_update} that
consumption remains unchanged forever, and therefore the
asymptotic consumption distribution is  degenerate.

But what happens if the {\it ex ante} division of gains is
associated with an implicit initial consumption level outside of
this range, or if there does not exist any first-best sustainable
allocation ($\overline c_{\rm min} < \underline c_{\rm max}$)? To
understand the convergence of consumption to an asymptotic
distribution in general, we make the following observations.
According to the updating rules in \Ep{c_update}, any increase in
the consumption of a type 1 person
 between two consecutive periods has
consumption attaining the lower bound of some consumption
interval. It follows that in periods of increasing consumption,
the consumption level is bounded above by $\underline c_{\rm max}$
($=\underline c_S$) and hence increases can occur only  if the
initial consumption level is less than $\underline c_{\rm max}$.
Similarly,  any decrease in consumption  between two consecutive
periods has consumption attain the upper bound of some
consumption interval. It follows that in periods of decreasing
consumption,  consumption  is bounded below by $\overline
c_{\rm min}$ ($=\overline c_1$) and hence decreases can only occur
if  initial consumption is higher than $\overline c_{\rm
min}$. Given a current consumption level $c$, we can then
summarize the permissible range for next-period consumption $c'$
as follows:
$$\EQNalign{
\hbox{\rm if} \hskip.25cm c \leq \underline c_{\rm max}&
\hskip.25cm \hbox{\rm then } \hskip.25cm
c'\in \left[\min \{c, \overline c_{\rm min}\},\; \underline c_{\rm max}\right],
                                                              \EQN c_range;a \cr
\hbox{\rm if} \hskip.25cm c \geq \overline c_{\rm min}&
\hskip.25cm \hbox{\rm then } \hskip.25cm
c'\in \left[\overline c_{\rm min}, \; \max \{c, \underline c_{\rm max}\}\right].
                                                              \EQN c_range;b \cr}
$$
\subsection{Temporary imperfect risk sharing}
We now return to the case that there exist first-best sustainable
allocations, $\overline c_{\rm min} \geq \underline c_{\rm max}$,
but we let the {\it ex ante} division of gains be given by an
implicit initial consumption level
$c_{\triangle}\not\in[\underline c_{\rm max},\,\overline c_{\rm
min}]$. The permissible range for next-period consumption, as
given in \Ep{c_range}, and the support of the asymptotic
consumption becomes
$$\EQNalign{
\hbox{\rm if} \hskip.25cm c \leq \underline c_{\rm max}&
\hskip.5cm \hbox{\rm then } \hskip.5cm
c'\in \left[c,\, \underline c_{\rm max}\right] \hskip.5cm \hbox{\rm and}
\hskip.5cm \lim_{t \rightarrow \infty} c_t = \underline c_{\rm max}
             =\underline c_S,  \hskip1.5cm                \EQN c_rangeFB;a  \cr
\hbox{\rm if} \hskip.25cm c \geq \overline c_{\rm min}&
\hskip.5cm \hbox{\rm then } \hskip.5cm
c'\in \left[\overline c_{\rm min}, \, c\right]  \hskip.5cm \hbox{\rm and}
\hskip.5cm \lim_{t \rightarrow \infty} c_t = \overline c_{\rm min}
             =\overline c_1.  \hskip1.5cm                  \EQN c_rangeFB;b  \cr}
$$
We have monotone convergence in \Ep{c_rangeFB;a} for two
reasons. First, consumption is bounded from above by $\underline
c_{\rm max}$. Second, consumption cannot decrease when $c \leq
\overline c_{\rm min}$ and by assumption $\overline c_{\rm min}
\geq \underline c_{\rm max}$, so consumption cannot decrease when
$c \leq \underline c_{\rm max}$. It follows immediately that
$\underline c_{\rm max}$ is an absorbing point that is attained as
soon as the endowment $\overline y_S$ is realized with its
consumption level $\underline c_S = \underline c_{\rm max}$.
Similarly, the explanation for monotone convergence in
\Ep{c_rangeFB;b} goes as follows. First, consumption is bounded
from below by $\overline c_{\rm min}$. Second, consumption cannot
increase when $c \geq \underline c_{\rm max}$ and by assumption
$\overline c_{\rm min} \geq \underline c_{\rm max}$, so
consumption cannot increase when $c \geq \overline c_{\rm min}$.
It follows immediately that $\overline c_{\rm min}$ is an
absorbing point that is attained as soon as the endowment
$\overline y_1$ is realized with its consumption level $\overline
c_1 = \overline c_{\rm min}$.

These convergence results assert that imperfect risk sharing is at
most temporary if the set of first-best sustainable allocations is
nonempty. Notice that when  an economy begins with an implicit initial
consumption outside of the interval of sustainable constant
consumption levels, the subsequent monotone convergence to the
closest endpoint of that interval is reminiscent  of our earlier
analysis  of the moneylender and the villagers with one-sided lack
of commitment in section \use{sec:moneylender1}. In the current setting, the agent who is relatively
disadvantaged under the initial welfare assignment will see her
consumption weakly increase over time until she has experienced
the endowment realization that is most favorable to her. From
there on, the consumption level remains constant forever, and the
participation constraints will never bind again.

\subsection{Permanent imperfect risk sharing}
If the set of first-best sustainable allocations is empty
($\overline c_{\rm min} < \underline c_{\rm max}$), it breaks the
 monotone convergence to a constant consumption level.
The updating rules in \Ep{c_update} imply that the permissible
range for next-period consumption in \Ep{c_range} will ultimately
shrink to $\left[\overline c_{\rm min},\, \underline c_{\rm
max}\right]$, regardless of the initial welfare assignment. If the
 implicit initial consumption lies outside of that set, consumption
is bound to converge to it, again because of the monotonicity of
consumption when $c \leq \overline c_{\rm min}$ or $c \geq
\underline c_{\rm max}$. And as soon as there is a binding
participation constraint with an associated consumption level that
falls inside of the interval $\left[\overline c_{\rm min},\,
\underline c_{\rm max}\right]$, the updating rules in
\Ep{c_update} will never take us outside of this interval again.
Thereafter, the only observed consumption levels belong to the
ergodic set
$$
%\hbox{\rm the ergodic consumption set,} \hskip.4cm
\left\{\left[\overline c_{\rm min},\, \underline c_{\rm max}\right]
\bigcap \, \{\underline c_j,\, \overline c_j\}_{j=1}^S\right\}, \EQN c_ergodic
$$
with a unique asymptotic distribution. Within this invariant set,
the participation constraints of both agents occasionally bind,
reflecting imperfections in risk sharing.

If autarky is the only sustainable allocation, then each consumption
interval is degenerate with
$\underline c_j =\overline c_j = \overline y_j$ for all $j\in {\bf S}$,
as discussed in section \use{sec:c_nondeg}. Hence, the ergodic
consumption set in \Ep{c_ergodic} is then trivially equal to the set
of endowment levels, $\{\overline y_j\}_{j=1}^{S}$.



\section{Alternative recursive formulation}
Kocherlakota (1996b)
\auth{Kocherlakota, Narayana R.}%
used an alternative recursive formulation of the
contract design problem, one that more closely resembles our treatment of the moneylender
villager economy of section \use{sec:moneylender1}.   After replacing the argument in the function of
\Ep{Pdef;a} by the expected utility of the type 1 agent, Kocherlakota
writes the Bellman equation as
$$ P(v) = \max_{\{c_s,w_s\}_{s=1}^S}
  \sum_{s=1}^S \Pi_s \bigl\{ u(1 - c_s) + \beta P(w_s) \bigr\} \EQN newKoch0;a  $$
subject to
$$\EQNalign{  & \sum_{s=1}^S \Pi_s [u(c_s) + \beta w_s]  \geq v,   \EQN newKoch0;b \cr
             & u(c_s) + \beta w_s  \geq u(\overline y_s) + \beta v_{\rm aut},
       \hskip2cm   s=1, \ldots, S;  \hskip1cm
                  \EQN newKoch0;c \cr
          &    u(1 - c_s) + \beta P(w_s)  \geq u(1-\overline y_s) +
                  \beta v_{\rm aut},
             \ s=1, \ldots, S;  \EQN newKoch0;d \cr
             &  c_s \in [0, 1],  \EQN newKoch0;e \cr
              & w_s  \in [v_{\rm aut}, v_{\rm max}] .\EQN newKoch0;f \cr}$$
Here the planner comes into a period with a state variable $v$
that is a promised expected utility to the type 1 agent. Before
observing the current endowment realization, the planner chooses a
consumption level $c_s$ and a continuation value $w_s$ for each
possible realization of the current endowment. This
state-contingent portfolio $\{c_s, w_s\}_{s=1}^S$ must deliver
at least the promised value $v$ to the type 1 agent, as stated in
\Ep{newKoch0;b}, and must also be consistent with the agents' participation
constraints in \Ep{newKoch0;c} and \Ep{newKoch0;d}.

Notice the difference in timing with our presentation, which we
have based on  Thomas and Worrall's (1988) analysis.
Kocherlakota's planner leaves the current period with only one
continuation value $w_s$ and postpones the question of how to
deliver that promised value across future states until the
beginning of next period  but {\it  before\/} observing next
period's endowment. In contrast, in our setting, in the current
period the planner chooses a state-contingent set of continuation
values for the next period, $\{\chi_j\}_{j=1}^S$, where $\chi_j$
is the number of utils that the type 1 agent's expected utility
should exceed her autarky level in the next period if that
period's endowment is $\overline y_j$. We can evidently express
Kocherlakota's one state variable in terms of our state vector,
$$
w_s = \sum_{j=1}^S \Pi_j \left[\chi_j + u(\overline y_j) + \beta v_{\rm aut} \right]
    = v_{\rm aut} + \sum_{j=1}^S \Pi_j \chi_j ,
$$
where $v_{\rm aut}$ is the {\it ex ante} welfare level in autarky as given by
\Ep{autarky_value}. Similarly, Kocherlakota's
upper bound on permissible  values of next period's continuation value in \Ep{newKoch0;f}
is related to our upper bounds $\{\overline x_j\}_{j=1}^S$,
$$
v_{\rm max} =  v_{\rm aut} +  \sum_{j=1}^S \Pi_j \overline x_j
            = v_{\rm aut} + \triangle_{\rm max}.
$$



\section{Pareto frontier revisited}\label{sec:Koch_Pareto2}%
Given our earlier characterization of the optimal solution, we can
map Kocherlakota's promised value $v$ into an implicit promised
consumption level $c_{\triangle}\in[\underline c_1, \overline c_S]
= [\overline y_1, \overline y_S]$. Let that mapping be encoded in
the function $v(c_{\triangle})$. Hence, given a promised a value
$v(c_{\triangle})$, the optimal consumption dynamics in section
\use{sec:consdynamic} instruct us to set Kocherlakota's choice
variables, $\{c_s,w_s\}_{s=1}^S$, as follows:
$$\EQNalign{
c_s &= c_{\triangle} + \max\{0,\, \underline c_s - c_{\triangle} \}
           - \max\{0,\, c_{\triangle}  - \overline c_s \},   \EQN optstar;a \cr
w_s &= v(c_s).                                   \EQN optstar;b \cr}
$$


For a given value of $c_{\triangle}$, we define the following
three sets that partition the set ${\bf S}$ of endowment
realizations:
$$\EQNalign{
{\bf S}^o(c_{\triangle}) &\equiv \Bigl\{j\in {\bf S}: \, c_{\triangle}\in
                    (\underline c_j, \overline c_j) \Bigr\}, \EQN Sdef;a \cr
{\bf S}^-(c_{\triangle}) &\equiv \Bigl\{j\in {\bf S}: c_{\triangle}\geq
                    \overline c_j \Bigr\}, \EQN Sdef;b \cr
{\bf S}^+(c_{\triangle}) &\equiv \Bigl\{j\in {\bf S}: c_{\triangle}\leq
                    \underline c_j \Bigr\}.                  \EQN Sdef;c \cr}
$$
According to our characterization of consumption intervals in
\Ep{c_intervals}, these three sets are mutually exclusive and their
union is equal to ${\bf S}$.
${\bf S}^o(c_{\triangle})$ is the set of states, i.e.,
endowment realizations, for which
the optimal consumption level is $c_s = c_{\triangle}$.
But if the endowment realization falls outside of ${\bf S}^o(c_{\triangle})$,
the optimal consumption $c_s$ is
determined by either the upper or lower bound of the consumption interval
associated with that endowment realization.
In particular, for $s\in {\bf S}^-(c_{\triangle})$, consumption should drop
to the upper bound of the consumption interval, $c_s = \overline c_s$;
and for
$s\in {\bf S}^+(c_{\triangle})$, consumption should increase
to the lower bound of the consumption interval, $c_s = \underline c_s$.

The continuation value $v(c_{\triangle})$ can then be expressed as
$$\EQNalign{
v(c_{\triangle}) = & \sum_{s\in {\bf S}^o(c_{\triangle})}
\Pi_s \Bigl[u(c_{\triangle}) + \beta v(c_{\triangle}) \Bigr]
                                + \sum_{s\in {\bf S}^-(c_{\triangle})}
\Pi_s \Bigl[u(\overline c_s) + \beta v(\overline c_s) \Bigr]      \hskip1cm  \cr
                              &+    \sum_{s\in {\bf S}^+(c_{\triangle})}
\Pi_s \Bigl[u(\underline c_s) + \beta v(\underline c_s) \Bigr] \EQN Koch_value1 \cr}
$$
which can be rewritten as
$$\EQNalign{
v(c_{\triangle}) &=
\left( 1- \beta \sum_{s\in {\bf S}^o(c_{\triangle})} \Pi_s \right)^{-1}
\left\{  u(c_{\triangle}) \sum_{s\in {\bf S}^o(c_{\triangle})} \Pi_s  \right.\cr
%\noalign{\vskip.2cm}
                      &\left. + \sum_{s\in {\bf S}^-(c_{\triangle})}
\Pi_s \Bigl[u(\overline c_s) + \beta v(\overline c_s) \Bigr]
                       +    \sum_{s\in {\bf S}^+(c_{\triangle})}
\Pi_s \Bigl[u(\underline c_s) + \beta v(\underline c_s)
                                   \Bigr]\right\}. \hskip1.3cm \EQN Koch_value2 \cr}
$$
Similarly, the promised value to the type 2 household can be expressed
as
\offparens
$$\EQNalign{
P\!\left(v(c_{\triangle})\right) &=
\left( 1- \beta \sum_{s\in {\bf S}^o(c_{\triangle})} \Pi_s \right)^{-1}
\left\{  u(1-c_{\triangle}) \sum_{s\in {\bf S}^o(c_{\triangle})} \Pi_s  \right.\cr
\noalign{\vskip.2cm}
                      &\hskip3cm + \sum_{s\in {\bf S}^-(c_{\triangle})}
\Pi_s \Bigl[u(1-\overline c_s) + \beta P\!\left(v(\overline c_s)\right) \Bigr] \cr
                      &\left. \hskip3cm +    \sum_{s\in {\bf S}^+(c_{\triangle})}
\Pi_s \Bigl[u(1-\underline c_s) + \beta P\!\left(v(\underline c_s)\right)
                                   \Bigr]\right\}. \hskip1.3cm \EQN Koch_pareto\cr}
$$

\subsection{Values are continuous in implicit consumption}\label{sec:Koch_continuous_c}%
Both $v(c_{\triangle})$ and
$P\!\left(v(c_{\triangle})\right)$ are continuous in the implicit
consumption level $c_{\triangle}$. From \Ep{Koch_value2} and \Ep{Koch_pareto}
this is trivially true when variations
in $c_{\triangle}$ do not change the partition of states given by
the sets ${\bf S}^o(\cdot)$, ${\bf S}^-(\cdot)$ and
${\bf S}^+(\cdot)$. It can also be shown to be true when variations in
$c_{\triangle}$ do involve changes in the partition of states.
As an illustration, let us compute the limiting values of
$v(c_{\triangle})$ when $c_{\triangle}$ approaches $\overline c_k$
from below and from above, respectively, where we recall that
$\overline c_k$ is the upper bound of the consumption interval
associated with endowment $\overline y_k$.

We can choose a sufficiently small $\epsilon>0$ such that
$$
\{\underline c_s,\, \overline c_s\}_{s=1}^S \bigcap \,
[\overline c_k - \epsilon, \, \overline c_k + \epsilon] = \overline c_k.
$$
In particular,
the findings in \Ep{c_intervals} ensure that we can choose
a sufficiently small $\epsilon$ so that this intersection contains
no upper bounds on consumption intervals other than $\overline c_k$.
Similarly, $\epsilon$ can be chosen sufficiently small  that
the intersection does not contain any lower bound on consumption
intervals {\it unless} there exists a consumption interval with
a lower bound that is exactly equal to $\overline c_k$, i.e., if for
some $j\geq 1$, $\underline c_{k+j} = \overline c_k$.
We will have to keep this possibility in mind as we proceed in our
characterization of the sets ${\bf S}^o(\cdot)$, ${\bf S}^-(\cdot)$
and ${\bf S}^+(\cdot)$.

All three sets are constant for an implicit consumption
$c_{\triangle}\in[\overline c_k-\epsilon,\, \overline c_k)$ with
$\max\{{\bf S}^-(c_{\triangle} )\}=k-1$. For an implicit consumption
$c_{\triangle} \in[\overline c_k,\, \overline c_k+\epsilon]$, the set
${\bf S}^-(c_{\triangle})$ is constant with
$\max\{{\bf S}^-(c_{\triangle})\}=k$, while
the configuration of the other two sets depends on which one
of the following two possible cases applies.



\vskip.5cm
\noindent
{\it Case a:} $\overline c_k \not= \underline c_s$ for all $s\in {\bf S}$.
Here it follows that the set ${\bf S}^+(c_{\triangle})$ is constant for any
implicit consumption
$c_{\triangle}\in[\overline c_k-\epsilon,\, \overline c_k+\epsilon]$. Using
\Ep{Koch_value1}, the
limiting values of
$v(c_{\triangle})$ when $c_{\triangle}$ approaches $\overline c_k$
from below and from above, respectively, are then equal to
$$\EQNalign{
\lim_{ c_{\triangle} \uparrow \overline c_k}
v(c_{\triangle}) = & \sum_{s\in {\bf S}^o(\overline c_k - \epsilon)}
\Pi_s \Bigl[u(\overline c_k) + \beta v(\overline c_k) \Bigr]
                                + \sum_{s=1}^{k-1}
\Pi_s \Bigl[u(\overline c_s) + \beta v(\overline c_s) \Bigr]            \cr
                   &+   \sum_{s\in {\bf S}^+(\overline c_k - \epsilon)}
\Pi_s \Bigl[u(\underline c_s) + \beta v(\underline c_s) \Bigr]          \cr
                   = & \sum_{s\in {\bf S}^o(\overline c_k + \epsilon)}
\Pi_s \Bigl[u(\overline c_k) + \beta v(\overline c_k) \Bigr]
                                + \sum_{s=1}^{k}
\Pi_s \Bigl[u(\overline c_s) + \beta v(\overline c_s) \Bigr]            \cr
                   &+   \sum_{s\in {\bf S}^+(\overline c_k + \epsilon)}
\Pi_s \Bigl[u(\underline c_s) + \beta v(\underline c_s) \Bigr]
%\hskip1cm
=   \lim_{ c_{\triangle} \downarrow \overline c_k}
v(c_{\triangle}) .                                   \EQN Koch_continuous \cr}
$$




\vskip.5cm
\noindent
{\it Case b:} $\overline c_k = \underline c_{k+j}$ for some $j\geq 1$.
Here it follows that the set ${\bf S}^+(c_{\triangle})$ is constant
with $\min\{{\bf S}^+(c_{\triangle} )\}=k+j$ for any
implicit consumption
$c_{\triangle}\in[\overline c_k-\epsilon,\, \overline c_k]$;
and ${\bf S}^+(c_{\triangle})$ is constant
with $\min\{{\bf S}^+(c_{\triangle} )\}=k+j+1$ for any
implicit consumption
$c_{\triangle}\in(\overline c_k,\, \overline c_k+\epsilon]$. Using
\Ep{Koch_value1} and invoking  the fact that $\underline c_{k+j}=\overline c_k$, the limiting values of
$v(c_{\triangle})$ when $c_{\triangle}$ approaches $\overline c_k$
from below and from above, respectively, are then equal to
$$\EQNalign{
\lim_{ c_{\triangle} \uparrow \overline c_k}
v(c_{\triangle}) = & \sum_{s=k}^{k+j-1}
\Pi_s \Bigl[u(\overline c_k) + \beta v(\overline c_k) \Bigr]
                                + \sum_{s=1}^{k-1}
\Pi_s \Bigl[u(\overline c_s) + \beta v(\overline c_s) \Bigr]            \cr
                   &+   \sum_{s=k+j}^S
\Pi_s \Bigl[u(\underline c_s) + \beta v(\underline c_s) \Bigr]          \cr
                 = & \sum_{s=k+1}^{k+j}
\Pi_s \Bigl[u(\overline c_k) + \beta v(\overline c_k) \Bigr]
                                + \sum_{s=1}^{k}
\Pi_s \Bigl[u(\overline c_s) + \beta v(\overline c_s) \Bigr]            \cr
                   &+   \sum_{s=k+j+1}^S
\Pi_s \Bigl[u(\underline c_s) + \beta v(\underline c_s) \Bigr]
%\hskip1cm
=   \lim_{ c_{\triangle} \downarrow \overline c_k}
v(c_{\triangle}) .                                   \EQN Koch_continuous \cr}
$$



We have shown that $v(c_{\triangle})$ is continuous at the upper
bound of any consumption interval even though the partition of states
changes at such a point. Similarly, we can show that $v(c_{\triangle})$ is
continuous at the lower bound of any consumption interval. And in
the same manner, we can also establish that
$P\!\left(v(c_{\triangle})\right)$ is continuous in the implicit
consumption $c_{\triangle}$.




\subsection{Differentiability of the Pareto frontier}\label{sec:Koch_nondiff}%
Consider an implicit consumption level $c_{\triangle} \in
[\overline y_1, \overline y_S]$ that falls strictly inside at
least one consumption interval. We can then use expressions
\Ep{Koch_value2} and \Ep{Koch_pareto} to compute the derivative of
the Pareto frontier at $v(c_{\triangle})$ by differentiating with
respect to $c_{\triangle}$:
$$ P'\!\left(v(c_{\triangle})\right) =
{ \displaystyle {\displaystyle d P\!\left(v(c_{\triangle})\right)
\over
   \displaystyle d c_{\triangle}  }           \over
  \displaystyle {\displaystyle d v(c_{\triangle}) \over
   \displaystyle d c_{\triangle} }   }
=  -{u'(1 -c_{\triangle})
   \over u'(c_{\triangle}) }.                      \EQN Koch_diff
$$
It can be verified that \Ep{Koch_diff} is the derivative of the
Pareto frontier so long as the set ${\bf S}^o(c_{\triangle})$
remains nonempty. That is, changes in the set
${\bf S}^o(c_{\triangle})$ induced by varying $c_{\triangle}$
do not affect the expression for the
derivative in \Ep{Koch_diff}. This follows from the fact that
the derivatives are the same to the left and to the right of an
implicit consumption level where the set ${\bf S}^o(c_{\triangle})$
changes, and the fact that $v(c_{\triangle})$ and
$P\!\left(v(c_{\triangle})\right)$ are continuous in
the implicit consumption level, as shown in section
\use{sec:Koch_continuous_c}.
%%{\bf Tom and Lars: please rewrite the following
%%sentence.}  Changes in the set ${\bf S}^o(c_{\triangle})$ induced
%%by varying $c_{\triangle}$ do not affect the expression for the
%%derivative in \Ep{Koch_diff} because the consumption intervals
%%make transitions smoothly  in and out of the sets ${\bf
%%S}^-(c_{\triangle})$ and ${\bf S}^+(c_{\triangle})$, i.e., without
%%causing any jumps in $P\!\left(v(c_{\triangle})\right)$ and
%%$v(c_{\triangle})$.
It can also be verified that the derivative in
\Ep{Koch_diff} exists in the knife-edged case that occurs when
${\bf S}^o(c_{\triangle})$ becomes empty at a single point because
two adjacent consumption intervals share only one point, i.e.,
when $\overline c_k = \underline c_{k+1}$, which implies that ${\bf
S}^o(\overline c_k) = {\bf S}^o(\underline c_{k+1}) =\emptyset$.

The Pareto frontier becomes nondifferentiable when two adjacent
consumption intervals are disjoint. Consider such a situation
where an implicit consumption level $c_{\triangle} \in [\overline
y_1, \overline y_S]$ does not fall inside any consumption
interval, which implies that the set ${\bf S}^o(c_{\triangle})$ is
empty. Let $\overline y_k$ and $\overline y_{k+1}$ be the
endowment realizations associated with the consumption interval to
the left and to the right of $c_{\triangle}$, respectively. That
is,
$$
\overline c_k < c_{\triangle} < \underline c_{k+1}.
$$
 According to
\Ep{Koch_value1}, the continuation value for any implicit
consumption level $c\in[\overline c_k,\, \underline c_{k+1}]$ is
then constant and equal to
$$
\hat v = \sum_{s\in {\bf S}^-(c_{\triangle})}
\Pi_s \Bigl[u(\overline c_s) + \beta v(\overline c_s) \Bigr]
                             +    \sum_{s\in {\bf S}^+(c_{\triangle})}
\Pi_s \Bigl[u(\underline c_s) + \beta v(\underline c_s) \Bigr]. \EQN constant_v
$$
By using expression \Ep{Koch_diff}, we can compute the derivative
of the Pareto frontier on the left side and the right side of
$\hat v$,
$$\EQNalign{
\lim_{v \uparrow \hat v}
P'\!\left(v\right) &=
\lim_{c \uparrow \overline c_k}
{ \displaystyle {\displaystyle d P\!\left(v(c)\right) \over
   \displaystyle d c  }           \over
  \displaystyle {\displaystyle d v(c) \over
   \displaystyle d c  }   }
=  -{u'(1 -\overline c_k)
   \over u'(\overline c_k) },                      \cr
\noalign{\vskip.2cm}
\lim_{v \downarrow \hat v}
P'\!\left(v\right) &=
\lim_{c \downarrow \underline c_{k+1}}
{ \displaystyle {\displaystyle d P\!\left(v(c)\right) \over
   \displaystyle d c  }           \over
  \displaystyle {\displaystyle d v(c) \over
   \displaystyle d c  }   }
=  -{u'(1 -\underline c_{k+1})
   \over u'(\underline c_{k+1}) }.                      \cr}
$$
Since $\overline c_k < \underline c_{k+1}$, it follows that
$$
\lim_{v \uparrow \hat v} P'\!\left(v\right) >
\lim_{v \downarrow \hat v} P'\!\left(v\right)
$$
and hence, the Pareto frontier is not differentiable at $\hat
v$.\NFootnote{Kocherlakota (1996b) prematurely assumed that Thomas
and Worrall's (1988) demonstration of the differentiability of the
Pareto frontier $Q_s(\cdot)$ would imply that his conceptually
different frontier $P(\cdot)$ would be differentiable. Koeppl
(2003) uses the approach of Benveniste and Scheinkman (1979) to
establish a sufficient condition for differentiability of the
Pareto frontier $P(v)$. For a given value of $v$, the sufficient
condition is that there exists at least one realization of the
endowment such that the participation constraints are not binding
for any household in that state, i.e., our set ${\bf
S}^o(c_{\triangle})$ should be nonempty for the implicit
consumption level $c_{\triangle}$ associated with that particular
value of $v$. That condition is  sufficient but not  necessary,
since we have seen above that $P(v)$ is also differentiable at a
knife-edged case with $c_{\triangle} = \overline c_k = \underline
c_{k+1}$, even though the set ${\bf S}^o(c_{\triangle})$ would
then be empty.
%In terms of our consumption intervals,
%that sufficient condition translates into at least one endowment
%realization with an optimal consumption level that lies strictly
%inside the consumption interval associated with that endowment.
}
\auth{Koeppl, Thorsten V.}
\auth{Benveniste, Lawrence}  \auth{Scheinkman, Jose}%


\section{Continuation values {\`a la} Kocherlakota}
\vskip-.5cm
\subsection{Asymptotic distribution is nondegenerate for imperfect
risk sharing (except when $S=2$)}\label{sec:Koch_distnondeg}%
Here we assume that there exist sustainable allocations other than
autarky but that first-best outcomes are not attainable, i.e.,
there exist sustainable allocations with imperfect risk sharing.
Kocherlakota (1996b, Proposition 4.2) states that the continuation
values will then converge to a unique nondegenerate distribution.
Here we will verify that the claim of a nondegenerate asymptotic
distribution is correct except for when there are only two states
($S=2$).

The assumption that the distribution of $y_t$ is identical to that
of $1-y_t$ means that
$$\EQNalign{
\Pi_j &= \Pi_{S+1-j},                    \EQN symmetry;a \cr
\overline y_j &= 1- \overline y_{S+1-j}, \EQN symmetry;b \cr}
$$
for all $j\in {\bf S}$. The symmetric environment bestows symmetry
on the consumption intervals of section
\use{sec:equil_consumption}
$$
\overline c_j = 1 - \underline c_{S+1-j}, \EQN symmetry;c
$$
for all $j\in {\bf S}$, and symmetry on the continuation values of
the type 1 and type 2 household
$$
v(c_{\triangle}) =  P\!\left(v(1-c_{\triangle})\right). \EQN symmetry;d
$$


As discussed in section \use{sec:Koch_asympdist}, the condition
for the nonexistence of first-best sustainable allocations is that
$\overline c_{\rm min} < \underline c_{\rm max}$, which by
\Ep{c_int_bound} is the same as
$$
\overline c_1 < \underline c_S \hskip1cm \Longrightarrow
\hskip1cm       \underline c_S > 0.5               \EQN symmetry2
$$
where the implication follows from using $\overline c_1 = 1
-\underline c_S$ as given by \Ep{symmetry;c}. It is quite
intuitive that the consumption interval $[\underline c_S,\,
\overline c_S]$ associated with the highest endowment realization
$\overline y_S$ cannot contain the average value of the stochastic
endowment, $\sum_{i=1}^{S} \Pi_i \overline y_i = 0.5$. Otherwise,
there would certainly exist first-best sustainable allocations,
 a contradiction.

To prove the existence of a nondegenerate asymptotic distribution
of continuation values, it is sufficient to show that the
continuation value of an agent experiencing the highest endowment,
say, the type 1 household, exceeds the continuation value of the
other agent who is then experiencing the lowest endowment, say,
the type 2 household. Given her current realization of the highest
endowment $\overline y_S$, the type 1 household is awarded the
highest consumption level $\underline c_S$ ($= \underline c_{\rm
max}$) in the ergodic consumption set of \Ep{c_ergodic}.
Conditional on next period's endowment realization $\overline
y_i$, the type 1 household's consumption $\hat c_i$ in the next
period is determined by \Ep{optstar;a}, where $c_{\triangle} =
\underline c_S$. From \Ep{c_intervals} we know that $\underline
c_S \geq \underline c_i$ for all $i\in {\bf S}$, so next period's
consumption of the type 1 household as determined by
\Ep{optstar;a} can be written as
$$
\hat c_i = \min \{ \overline c_i, \, \underline c_S \}.   \EQN symmetry3
$$
Given the vector $\{\hat c_i\}_{i=1}^{S}$ for next period's consumption,
we can use \Ep{Koch_value1} to compute the type 1
household's outgoing continuation value in the current period,
$$
v(\underline c_S) = \sum_{i=1}^{S} \Pi_i \Bigl[u(\hat c_i) + \beta v(\hat c_i)
                                                                         \Bigr].
$$
Next, we are interested in computing the difference between the
continuation values of the type 1 and the type 2 households,
$$\EQNalign{
v(\underline c_S) - &P\!\left(v(\underline c_S))\right)
= \sum_{i=1}^{S} \Pi_i \Bigl[u(\hat c_i) + \beta v(\hat c_i)
                 -u(1-\hat c_i) - \beta P\!\left(v(\hat c_i)\right) \Bigr] \cr
&= \sum_{i=1}^{S} \Pi_i \Bigl[u(\hat c_i) + \beta v(\hat c_i)
       -u(1-\hat c_{S+1-i}) - \beta P\!\left(v(\hat c_{S+1-i})\right) \Bigr]\cr
&= \sum_{i=1}^{S} \Pi_i \Bigl[u(\hat c_i) + \beta v(\hat c_i)
       -u(1-\hat c_{S+1-i}) - \beta v(1-\hat c_{S+1-i}) \Bigr],\hskip1cm
                                                            \EQN symmetry4 \cr}
$$
where the second equality emerges from the symmetric probabilities
in \Ep{symmetry;a}, and the third equality follows from \Ep{symmetry;d}.
To establish that the difference in \Ep{symmetry4} is strictly positive,
it is sufficient to show that
$$
\hat c_i \geq 1- \hat c_{S+1-i},
$$
or, by using \Ep{symmetry3},
$$
\min \{ \overline c_i, \, \underline c_S \} \geq
           1 - \min \{ \overline c_{S+1-i}, \, \underline c_S \} \EQN symmetry5
$$
for all $i\in {\bf S}$, and at least one of them holds with strict
inequality. The proof proceeds by considering four possible cases
for each $i\in {\bf S}$.

\vskip.5cm
\noindent
{\it Case a:} $\overline c_i \leq \underline c_S$ and
$\overline c_{S+1-i} \leq \underline c_S$. According to
\Ep{symmetry;c} $\overline c_{S+1-i} = 1-\underline c_i$,
so inequality \Ep{symmetry5} can then be written as
$$
\overline c_i \geq 1 - \overline c_{S+1-i} = \underline c_i
\ \hbox{\rm which is true since} \
\overline c_i > \underline c_i \hskip.3cm \hbox{\rm for all}
\hskip.3cm  i\in {\bf S},
$$
as established in section \use{sec:c_nondeg}.


\vskip.5cm
\noindent
{\it Case b:} $\overline c_i \leq \underline c_S$ and
$\overline c_{S+1-i} > \underline c_S$. According to
\Ep{symmetry;c} $\overline c_i = 1-\underline c_{S+1-i}$,
so inequality \Ep{symmetry5} can then be written as
$$
\overline c_i = 1 - \underline c_{S+1-i}
\geq 1 - \underline c_S
\ \hbox{\rm which is true since} \
\underline c_{S+1-i} < \underline c_S \hskip.3cm \hbox{\rm for all}
\hskip.3cm  i\not=1,
$$
as established in section \use{sec:intervals_contain}.


\vskip.5cm
\noindent
{\it Case c:} $\overline c_i > \underline c_S$ and
$\overline c_{S+1-i} \leq \underline c_S$. According to
\Ep{symmetry;c} $\overline c_{S+1-i} = 1-\underline c_i$,
so inequality \Ep{symmetry5} can then be written as
$$
\underline c_S \geq 1 - \overline c_{S+1-i} = \underline c_i
\ \hbox{\rm which is true since} \
\underline c_S > \underline c_i \hskip.3cm \hbox{\rm for all}
\hskip.3cm  i\not=S,
$$
as established in section \use{sec:intervals_contain}.


\vskip.5cm
\noindent
{\it Case d:} $\overline c_i > \underline c_S$ and
$\overline c_{S+1-i} > \underline c_S$.
The inequality \Ep{symmetry5} can then be written as
$$
\underline c_S \geq 1 - \underline c_S
\ \hbox{\rm which is true since} \
\underline c_S > 0.5
$$
as established in \Ep{symmetry2}.

\vskip.5cm

We can conclude that the inequality \Ep{symmetry5} holds with {\it
strict} inequality with only two exceptions: (1) when $i=1$
and case b applies; and (2) when $i=S$ and case c applies.
It follows that the difference in \Ep{symmetry4} is definitely
strictly positive if there are more than two states, and hence the
asymptotic distribution of continuation values is nondegenerate.
But what about when there are only two states ($S=2$)? Since
$\overline c_1 < \underline c_S$ by \Ep{symmetry2} and $\overline
c_S > \underline c_S$, it follows that case b applies when $i=1$
and case c applies when $i=2=S$. Therefore, the difference in
\Ep{symmetry4} is zero, and thus the continuation value of an
agent experiencing the highest endowment is equal to that  of the
other agent who is then experiencing the lowest endowment. Since
there are no other continuation values in an economy with only two
possible endowment realizations, it follows that the asymptotic
distribution of continuation values is degenerate when there are
only two states ($S=2$).

A two-state example in section \use{sec:thekink} illustrates our
findings. The intuition for the degenerate asymptotic distribution
of continuation values is straightforward. On the one hand,
the planner would
like to vary continuation values and thereby avoid large changes
in current consumption that would otherwise be needed to satisfy
binding participation constraints. But, on the other hand,
different continuation
values presuppose that there exist ``intermediate'' states in
which a higher continuation value can be  awarded. In our
two-state example, the participation constraint  of either one or
the other type of agent always  binds, and the asymptotic
distribution is degenerate with only one continuation value.



\subsection{Continuation values do not always respond to binding
participation constraints}
Evidently, continuation values will eventually not respond to
binding participation constraints in a two-state economy, since we
have just shown that the asymptotic distribution is degenerate
with only one continuation value. But the outcome that
continuation values might not respond to binding participation
constraints occurs even with more states when endowments are
i.i.d. In fact, it is present whenever the consumption intervals
of two adjacent endowment realizations, $\overline y_k$ and
$\overline y_{k+1}$, do not overlap, i.e., when $\overline c_k <
\underline c_{k+1}$. Here is how the argument goes.

Since $\overline c_k < \underline c_{k+1}$ it follows from
\Ep{c_ergodic} that both $\overline c_k$ and $\underline c_{k+1}$
belong to the ergodic set of consumption. Moreover,
\Ep{c_intervals} implies that ${\bf S}^o(\overline c_k) = {\bf
S}^o(\underline c_{k+1}) = \emptyset$, where ${\bf S}^o(\cdot)$ is
defined in \Ep{Sdef;a}. Using expression \Ep{Koch_value1}, we can
compute a common continuation value $v(\overline c_k)=v(\underline
c_{k+1}) = \hat v$, where $\hat v$ is given by \Ep{constant_v}
when that expression is evaluated for any $c_{\triangle} \in
[\overline c_k,\, \underline c_{k+1}]$. Given this identical
continuation value, it follows that there are situations where
households' continuation values will not  respond to binding
participation constraints.

As an example, let the current consumption and continuation value
of the type 1 household be $\overline c_k$ and $v(\overline c_k) =
\hat v$, and suppose that the household next period realizes the
endowment $\overline y_{k+1}$. It follows that the participation
constraint of the type 1 household is binding and that the optimal
solution in \Ep{optstar} is to award the household a consumption
level $\underline c_{k+1}$ and continuation value $v(\underline
c_{k+1})$. That is, the household is induced not to defect into
autarky by increasing its consumption, $\underline c_{k+1} >
\overline c_k$, but its continuation value is kept unchanged,
$v(\underline c_{k+1}) = \hat v$. Suppose next that the type 1
household experiences $\overline y_k$ in the following period.
This time it will be the participation constraint of the type 2
households that binds and the optimal solution in \Ep{optstar}
prescribes that the type 1 household is awarded consumption
$\overline c_k$ and continuation value $v(\overline c_k)=\hat v$.
Hence, only consumption levels but not continuation values are
adjusted in these two realizations with alternating binding
participation constraints.

We use a three-state example in section \use{sec:3state} to
elaborate on the point that even though an incoming continuation
value lies in the interior of the range of permissible
continuation values in \Ep{newKoch0;f}, a binding participation
constraint still might  not trigger a change in the outgoing
continuation value because there may not exist any efficient way
to deliver a changed continuation value. Continuation values that
do not respond to binding participation constraints are a
manifestation of the possibility that the Pareto frontier
$P(\cdot)$ need not be differentiable everywhere on the interval
$[v_{\rm aut}, v_{\rm max}]$, as shown in section
\use{sec:Koch_nondiff}.





%When computing efficient allocations in our two- and three-state examples below,
%it will be useful to refer to Kocherlakota's continuation value but rely on
%Thomas and Worrall's characterization of the optimal consumption dynamics.
%Our examples will also serve as counterexamples to two assertions of
%Kocherlakota.\NFootnote{The small technical glitches that prevent Kocherlakota's analysis
%from applying to our two-- and three--state examples  do nothing to diminish
%the value of Kocherlakota's main contribution in analyzing the conditions
%under which occasionally binding enforcement constraints can lead to empirically
%interesting conditional covariances of consumption and income.}
%We like these counterexamples because they help illuminate the workings of
%Kocherlakota's model both under the situations that prevail in the counterexamples and also
%in the less pathological cases that  he had in mind.  Thus, it is instructive
%to understand why the following two claims must be qualified:
%\vskip.2cm
%\item{1. } If there exist no first-best sustainable allocation, then the continuation
%values will converge to a unique non-degenerate distribution. [Kocherlakota (1996b),
%Proposition 4.2 on page 603]

%\item{2. } If the participation constraint of the type 1 agent
%binds and $v<v_{\rm max}$, then $w_s > v$. On the other hand, if
%the participation constraint of the type 2 agent binds and
%$v>v_{\rm aut}$, then $w_s<v$. [Kocherlakota (1996b), page 600]
%\vskip.2cm \noindent The first claim that there is a {\it non-degenerate}
%asymptotic distribution does not hold in our two-state example.
%The intuition for this is straightforward. Yes, the planner would
%like to vary continuation values and thereby avoid large changes
%in current consumption that would otherwise be needed to satisfy
%binding participation constraints. However, different continuation
%values presuppose that there exist ``intermediate'' states in
%which a higher continuation value can be  awarded. In our
%two-state example, the participation constraint  of either one or
%the other type of agent always  binds and the asymptotic
%distribution is degenerate with only one continuation value. Thus,
%the two-state example serves as a counterexample not  only to the first
%claim but also to the second claim. We use a three-state example
%to  elaborate on the point that even though an incoming
%continuation value lies in the interior of the range of
%permissible continuation values, a binding participation
%constraint still might  not trigger a change in the outgoing
%continuation value because there may not exist any efficient way
%to deliver a changed continuation value. A manifestation of the
%failure of the second claim is that the Pareto frontier $P(\cdot)$
%is not differentiable everywhere on the interval $[v_{\rm aut},
%v_{\rm max}]$.\NFootnote{Kocherlakota prematurely assumed that Thomas and
%Worrall's (1988) demonstration of the differentiability of the Pareto frontier
%$Q_s(\cdot)$ %that is conditional on $y_s$
%would imply that his conceptually
%different frontier $P(\cdot)$ would be differentiable.  The counterexamples seem to say
%that unfortunately this
%is not true in general.}




\section{A two-state example: amnesia overwhelms
memory}\label{sec:thekink}%
In this example and the three-state example of the following
section, we use the term ``continuation value'' to denote the
state variable of Kocherlakota (1996b) as described in the
preceding section.\NFootnote{See Krueger and Perri (2003b) for
another analysis of a two-state example.} That is, at the end of a
period, the continuation value $v$ is the promised expected
utility to the type 1 agent that will be delivered at the start of the next
period.
\auth{Krueger, Dirk}\auth{Perri, Fabrizio}%

Assume that there are only two possible endowment realizations,
$S=2$, with $\{\overline y_1, \overline y_2\}=\{1-\overline y,
\overline y\}$, where $\overline y \in (.5,1)$. Each endowment
realization is equally likely to occur, $\{\Pi_1, \Pi_2\}=\{0.5,
0.5\}$. Hence, the two types of agents face the same {\it ex ante}
welfare level in autarky,
$$v_{\rm aut} = {.5 \over 1-\beta}\left[ u(\overline y) +
  u(1-\overline y)\right].
$$
We will focus on parameterizations for which there exist no
first-best sustainable allocations (i.e., $\overline c_{\rm min} <
\underline c_{\rm max}$, which here amounts to $\overline c_1 <
\underline c_2$). An efficient allocation will then asymptotically
enter the ergodic consumption set in \Ep{c_ergodic} that here is
given by two points, $\{\overline c_1, \underline c_2\}$. Because
of the symmetry in preferences and endowments, it must be true
that $\underline c_2 = 1 - \overline c_1 \equiv \overline c$,
%% (and for that matter, $\overline c_2 = 1- \underline c_1$).
where we let $\overline c$ denote the consumption allocated to an
agent whose participation constraint is binding and $1 -\overline
c$ be the consumption allocated to the other agent.

Before determining the optimal values $\{1-\overline c, \overline
c\}$, we will first verify that any such stationary allocation
delivers the same continuation value to both types of agent. Let
$v^+$ be the continuation value for the consumer who last received
a high endowment and let $v^-$ be the continuation value for the
consumer who last received a low endowment. The promise-keeping
constraint for $v^+$ is
$$v^+ = .5[u(\overline c) + \beta v^+] +
  .5[u(1-\overline c) + \beta v^- ]  $$
and  the promise-keeping constraint for
$v^-$ is
$$v^- = .5[u(\overline c) + \beta v^+] +
  .5[u(1-\overline c) + \beta v^- ].
$$
Notice that the promise-keeping constraints make  $v^+$ and
$v^-$  identical.
Therefore, there is a unique
stationary continuation value $\overline v \equiv v^+ =v^-$ that is independent
of the current period endowment, as established in section
\use{sec:Koch_distnondeg} for $S=2$.
  Setting $v^+ = v^- = \overline v$
 in one of the two equations above
and solving gives the stationary continuation value:
$$ \overline v = {.5\over 1-\beta}
\left[u(\overline c) + u(1-\overline c)\right].  \EQN kochex1  $$


To determine the optimal $\overline c$ in this two-state example,
we use the following two facts. First, $\overline c$ is the lower
bound of the consumption interval $[\underline c_2, \overline
c_2]$; $\overline c$ is the consumption level that should be
awarded to the type 1 agent when she experiences the highest
endowment $\overline y_2 = \overline y$ and we want to maximize
the welfare of the type 2 agent subject to the type 1 agent's
participation constraint. Second, $\overline c$ belongs also to
the ergodic set $\{\overline c_1, \underline c_2\}$ that
characterizes the stationary efficient allocation, and we know
that the associated efficient continuation values are then the
same for all agents and given by $\overline v$ in \Ep{kochex1}.
The maximization problem above can therefore be written as
$$\EQNalign{
\max_{\overline c}\hskip.2cm &u(1-\overline c) + \beta \overline v  \EQN kochex2;a \cr
\hbox{\rm subject to }
\hskip.2cm  &u(\overline c) + \beta \overline v  - \left[ u(\overline y)
  + \beta v_{\rm aut} \right] \geq 0,                        \EQN kochex2;b \cr
%%%%           &\overline c\in[0.5, 1], \cr
}$$ where $\overline v$ is given by \Ep{kochex1}. We graphically
illustrate how $\overline c$ is chosen in order to maximize
\Ep{kochex2;a} subject to \Ep{kochex2;b} in Figures \Fg{koch2s1bf}
and  \Fg{koch2s1af} for utility function $(1-\gamma)^{-1}
c^{1-\gamma}$ and
 parameter values $(\beta, \gamma, \overline y)
= (.85, 1.1, .6)$. It can be verified numerically
that $\overline c  = .536$.
Figure  \Fg{koch2s1bf}
     shows \Ep{kochex2;a} as a decreasing
function of $\overline c$ in the interval $[.5,.6]$.
Figure
\Fg{koch2s1af} plots the left side of \Ep{kochex2;b} as a function of
$\overline c$.  Values
of $\overline c$ for which the expression is negative are not sustainable
(i.e., values less than .536).  Values of $\overline c$ for
which the expression is nonnegative are sustainable.  Since the welfare
of the agent with a low endowment realization in \Ep{kochex2;a} is
decreasing as a function of $\overline c$ in the interval
$[.5,.6]$, the best sustainable value of $\overline c$ is the
lowest value for which the expression in \Ep{kochex2;b} is nonnegative.
This value for $\overline c$ gives the  most
risk sharing that is compatible with the participation constraints.

%%%%%%%%%%%
%\midinsert
%$$\grafone{koch2s1b.eps,height=2.25in}{{\bf Figure XXX.} Continuation value
%as function of $\overline c$.}
%$$ \endinsert
%%%%%%%%%%%%%%%%%%%%%%%
%%%%%%%%%%%%%
%\midinsert $$
%\grafone{koch2s1a.eps,height=2in}{{\bf Figure XXX.} The participation
%constraint is satisfied for values of  $\overline c$ for which
%the difference $u(\overline c) + \beta v -\left(u(\overline y)
%+ \beta v_{\rm aut}\right) $ plotted here is positive.}
%$$
%\endinsert
%%%%%%%%%%%%


\midfigure{koch2s1bf}
\centerline{\epsfxsize=3.1truein\epsffile{koch2s1b.ps}}
\caption{Welfare of the agent with low endowment as a function of $\overline c$.}
\infiglist{koch2s1bf}
\endfigure

\midfigure{koch2s1af}
\centerline{\epsfxsize=3.1truein\epsffile{koch2s1a.ps}}
\caption{The participation constraint is satisfied for values of $\overline c$
for which the difference $u(\overline c) + \beta \overline v -\left[u(\overline y)
+ \beta v_{\rm aut}\right]$ plotted here is positive.}
\infiglist{koch2s1af}
\endfigure



\subsection{Pareto frontier}
It is instructive to find the entire set of sustainable values
$V$. In addition to the value $\overline v$ above associated with
a stationary sustainable allocation, other values can be sustained, for
example,  by
promising a value $\hat v >  \overline v$ to a type 1 agent who has
yet to receive a low endowment realization.  Thus, let $\hat v$ be
a promised value to such a consumer and let $c^+$ be the
consumption assigned to that consumer in the event that his endowment is
high.  Then promise keeping for the two types of agents requires
$$\EQNalign{\hat v & = .5[ u(c^+) + \beta \hat v ]
    + .5[ u(1-\overline c) + \beta \overline v], \EQN pk1;a \cr
   P(\hat v) & = .5[u(1-c^+) + \beta P(\hat v) ]
    + .5[ u(\overline c)  + \beta \overline v] . \EQN pk1;b  \cr} $$
If the type 1 consumer receives the high endowment,
sustainability of the allocation requires
$$\EQNalign{ u(c^+) + \beta \hat v    &\geq u(\overline y) + \beta
       v_{\rm aut}, \EQN sust1;a \cr
     u(1-c^+) + \beta P(\hat v) & \geq u(1-\overline y) + \beta v_{\rm aut}.
  \EQN sust1;b \cr}  $$
If the type 2 consumer receives the high endowment, awarding him
$\overline c,\overline v$    automatically satisfies the sustainability
requirements because these are already built into the construction
of the stationary sustainable value $\overline v$.

Let's solve for the {\it highest\/} sustainable initial value  of
$\hat v$, namely, $v_{\rm max}$.
To do so, we must solve the three equations formed
by the  promise-keeping constraints
\Ep{pk1;a} and  \Ep{pk1;b}    and the participation
constraint \Ep{sust1;b}
of a type 2 agent when it receives $1-\overline y$ at equality:
$$ u(1-c^+) + \beta P(\hat v) = u(1-\overline y) + \beta v_{\rm aut} .
\EQN number3
$$
Equation \Ep{pk1;b} and \Ep{number3} are two equations
in $(c^+, P(v_{\rm max}))$. After solving them, we can solve
\Ep{pk1;a} for $v_{\rm max}$.
Substituting \Ep{number3} into \Ep{pk1;b} gives
$$P(v_{\rm max}) = .5[u(1-\overline y) +\beta v_{\rm aut}]
   + .5 [u(\overline c) + \beta \overline v].  \EQN number4 $$
But from the participation constraint of a high endowment
household in a stationary allocation,  recall that
$u(\overline c) + \beta \overline v = u(\overline y) + \beta v_{\rm aut}$.
Substituting this into \Ep{number4} and rearranging gives
$$ P(v_{\rm max}) = v_{\rm aut}  $$
and therefore by \Ep{number3}, $c^+ = \overline
y$.\NFootnote{According to our general characterization of
the {\it ex ante} division of the gains of an efficient contract
in section \use{sec:Koch_Pareto1},
it can be viewed as determined by an implicit initial consumption
level $c_{\triangle}\in [\overline y_1, \overline y_S]$. Notice
that the present calculations have correctly computed the upper
bound of that interval for our two-state example, $\overline y_S =
\overline y_2 \equiv \overline y$.} Solving \Ep{pk1;a} for $v_{\rm
max}$ we find
$$v_{\rm max} = {1 \over 2-\beta}[u(\overline y)  +u(1-\overline c)
    + \beta \overline v] .  \EQN maxhatv $$

%%\subsubsection{A kink}

  Now let us study what happens when we set $v \in
  (\overline v, v_{\rm max})$ and drive
$v $ toward $\overline v$ from above.  Totally differentiating \Ep{pk1;a}
and \Ep{pk1;b}, we find
$$ {d P(\hat v) \over d \hat v} = -{u'(1-c^+) \over u'(c^+) } . $$
Evidently
$$\lim_{v \downarrow \overline v}
{d P(v) \over d v}
= -{u'(1 -\overline c)
   \over u'(\overline c)} < -1.  $$
By symmetry,
$$\lim_{v \uparrow \overline v}
{d P(v) \over d v}
  = -{ u'(\overline c) \over u'(1-\overline c)
  } > -1 .$$
Thus, there is  a kink in the value function $P(v)$ at $v=\overline v$.
At $\overline v$, the value function is not differentiable as
established in section \use{sec:Koch_nondiff} when two adjacent consumption
intervals are disjoint.
At $\overline v$, $P'(v)$ exists only in the sense of
a subgradient in the interval $\left[
 -u'(1 -\overline c) / u'(\overline c),\;
  - u'(\overline c) / u'(1-\overline c)\right]$.
Figure \Fg{rm1f} %15.rm1XXX
depicts the kink in $P(v)$.


%%Figures 15.rm1XXX and 15.rm2XXX depict
%%the kink in $P(v)$  and
%%the consequent flat spot in the function that depicts the tradeoff
%%between $w_s$ and $c_s$.  The flat spot reflects the fact that
%%$P'(v)$ is defined as a subgradient at the stationary value
%%of $v$.

%%%%%%%%%
%\midinsert
%$$\grafone{rm1.ps,height=3.25in,angle=90}{{\bf Figure 15.rm1XXX.}
%The kink in $P(v)$ at the stationary value of $v$ for the two-state
%symmetric example.}
%$$ \endinsert
%%%%%%%%%%%%%

%%%%%%%%%%%
%\topinsert$$
%\grafone{rm2.ps,height=3.25in,angle=90}{{\bf Figure 15.rm2XXX.}
%Tradeoff between continuation value and consumption in the two-state
%symmetric example.  The kink in $P(v)$ at $v$ leads to a flat spot in
%the function expressing $w_s$ as a function of $y_s$.}
%$$ \endinsert
%%%%%%%%%%%%


\midfigure{rm1f}
\centerline{\epsfxsize=3truein\epsffile{rm1.eps}}
\caption{The kink in $P(v)$ at the stationary value of $v$ for the two-state symmetric
example.}
\infiglist{rm1f}
\endfigure

%\midfigure{rm2f}
%\centerline{\epsfxsize=3.5truein\epsffile{rm2.eps}}
%\caption{Tradeoff between continuation value and consumption in the two-state symmetric
%example.  The kink in $P(v)$ at $v$ leads to a flat spot in the function expressing
%$w_s$ as a function of $y_s$.}
%\infiglist{rm2f}
%\endfigure

\subsection{Interpretation}
Recall our characterization of the optimal consumption dynamics in
\Ep{c_update}. Consumption remains unchanged between periods when
neither participation constraint binds, and hence the efficient
contract displays memory or history dependence. When either of the
participation constraints binds, history dependence is limited to
 selecting either the lower or the upper bound of a
consumption range $[\underline c_j, \overline c_j]$, where the
range and its bounds are functions of the current endowment
realization $\overline y_j$. After someone's participation
constraint has once been binding, history becomes irrelevant,
because past consumption has no additional impact on the level of
current consumption. %Kocherlakota calls this feature
\index{amnesia}%

Now, in the case of our two-state example, there are only two
consumption ranges, $[\underline c_1, \overline c_1]$ and
$[\underline c_2, \overline c_2]$. And as a consequence, the
asymptotic consumption distribution has only two points,
$\overline c_1$ and $\underline c_2$ (or in our notation,
$1-\overline c$ and $\overline c$). It follows that history
becomes  irrelevant because consumption is then determined by the
endowment realization. Thus, it can be said that ``amnesia
overwhelms memory'' in this example, and the asymptotic
distribution of continuation values becomes degenerate with a
single point $\overline v$.\NFootnote{If we adopt the recursive
formulation of Thomas and Worrall in \Ep{new1},  amnesia manifests
itself as a time-invariant state vector $[x_1, x_2]$ where
$x_1 = u(1-\overline c) - u(1-\overline y) + \beta [\overline v - v_{\rm aut} ]
$ and $
x_2 = u(\overline c) - u(\overline y) + \beta [\overline v - v_{\rm aut} ]$.}

We further explore the variation or the lack of variation
in continuation values in the three-state example of the following
section.




\section{A three-state example}\label{sec:3state}%
As the two-state example stresses, any variation of continuation values
in an efficient allocation requires that
%%the graphical presentation
%%in Figure 15.XXXX (from Kocherlakota (1996b) implicitly assumes that
 the environment be such that when a household's participation constraint
is binding, the planner has room to increase both the current
consumption {\it and\/} the continuation value of that household.
In the stationary allocation in the two-state example, there is no
room to adjust the continuation value because of the restrictions
that promise keeping imposes. We now analyze the stationary
allocation of a three-state $(S=3)$ example in which
%%the small number of states
the environment still limits the planner's ability to manipulate
continuation values, but nevertheless sometimes allows adjustments
in the continuation value.

Thus, consider an environment in which $S=3$. We assume
that the distributions of $y_t$ and $1-y_t$ are identical.  In particular,
we let $\{\overline y_1,\overline y_2,\overline y_3\}=
\{1-\overline y, 0.5, \overline y\}$ and
$\{\Pi_1, \Pi_2, \Pi_3\}=\{\Pi/2, 1-\Pi, \Pi/2\}$ where
$\overline y\in(.5, 1]$ and $\Pi\in[0,1]$.
Given parameter values such that there is no first-best
sustainable allocation
(i.e., $\overline c_1 < \underline c_3$),
we will study the efficient allocation that is attained
asymptotically. According to \Ep{c_ergodic}, this ergodic
consumption set is given by
$$
\left\{\left[\overline c_1,\, \underline c_3\right]
\bigcap \, \{\overline c_1,\, \underline c_2,\, \overline c_2,
           \, \underline c_3\} \right\}, \EQN c_ergodic3
$$
which contains at least two points ($\overline c_1$, $\underline c_3$)
and maybe two additional points ($\underline c_2$, $\overline c_2$).

When there are no first-best sustainable allocations, the
efficient stationary allocation must be such that the
participation constraints of a type 1 person and a type 2 person
bind in state 3 and state 1, respectively. Let $\overline
c\in[0.5, 1]$ and $\bar w^+$ be the consumption and continuation
value allocated to the agent whose participation constraint is
binding because his endowment is equal to $\overline y$:
$$ u(\overline c) + \beta \bar w^+ = u(\overline y)
  + \beta v_{\rm aut} . \EQN koch3ex1 $$
In such a state, the agent whose participation constraint
is {\it not\/} binding consumes $1-\overline c$ and is assigned
continuation value
$\bar w^-$. Because of the assumed symmetries with respect to
preferences and endowments, we have
$\overline c = \underline c_3 = 1-\overline c_1$.

The consumption allocation in state 2 depends on
the different promised continuation values with which
agents enter a period.
The symmetry in our
environment and the existence of only three states imply that
there is a single consumption level $\hat c$ that is granted to
the type of person that last realized the highest endowment
$\overline y$.
Let  $\hat w^+$ be the continuation
value that in state 2 is
allocated to the type of person that last received endowment
$\overline y$.
 According to our earlier characterization of an
efficient allocation, the agents who realize the highest endowment
$\overline y$ are induced not to defect into autarky by granting them
both higher current consumption and a higher continuation value.
Hence, state 2 is ``payback time'' for the agents who were promised
a higher continuation value and it must be true that
$\hat c \in [0.5, 1]$. In state 2, the type of person that
did not last receive $\overline y$ is
allocated consumption $1-\hat c$ and continuation value
$\hat w^-$. The participation constraint of this type of person
might conceivably be binding in state 2,
$$u(1-\hat c) + \beta \hat w^- \geq u(0.5)
  + \beta v_{\rm aut} . \EQN koch3ex2 $$
According to the optimal consumption dynamics in \Ep{c_update},
we know that $\hat c = \min \{\overline c, \overline c_2\}$. That is,
a person who had the highest endowment realization $\overline y$
with associated consumption level $\overline c$ will retain
that consumption level when moving into state 2
($\hat c = \overline c$) unless
the participation constraint of the other agent becomes binding
in state 2. In the latter case, the person who had the highest
endowment realization is awarded consumption
$\hat c = \overline c_2$ in state 2
and the participation constraint for the other person in \Ep{koch3ex2}
will hold with strict equality.


While there can exist four different consumption levels in the
efficient stationary allocation, $\{1-\overline c, 1-\hat c, \hat
c, \overline c\}$, it is possible to have at most two distinct
continuation values:
$$\EQNalign{
\bar w^+ = \hat w^+  = &
(\Pi/2) \left[u(\overline c)+\beta \bar w^+ \right]
+(1-\Pi) \left[u(\hat c)+\beta \hat w^+\right] \cr
& + (\Pi/ 2) \left[u(1-\overline c)+\beta \bar w^-\right],
                                                   \EQN koch3ex3;a\cr
\noalign{\vskip.3cm}
\bar w^- = \hat w^-  = &
(\Pi/ 2) \left[u(\overline c)+\beta \bar w^+ \right]
+(1-\Pi) \left[u(1-\hat c)+\beta \hat w^-\right]            \cr
& + (\Pi/ 2) \left[u(1-\overline c)+\beta \bar w^-\right].
                                                  \EQN koch3ex3;b\cr}
$$
As can be seen on the right side of \Ep{koch3ex3;a}, the
expressions for $\bar w^+$ and $\hat w^+$ are the same, and so
$\bar w^+ = \hat w^+\equiv w^+$. The same holds true for $\bar
w^-$ and $\hat w^-$ in \Ep{koch3ex3;b}, and hence $\bar w^- = \hat
w^-\equiv w^-$. By manipulating equations \Ep{koch3ex3}, we can
express the two continuation values in terms of $(\overline c,
\hat c)$:
$$\EQNalign{
w^+ &= \Bigl\{ (\Pi/2) \left[1+\beta \kappa \Pi/2 \right]
          \left[u(\overline c) + u(1-\overline c)\right]             \cr
     &\hskip1cm + (1-\Pi) \left[u(\hat c)
                    +\beta \kappa \Pi u(1-\hat c)/2\right] \Bigr\}   \cr
     &\hskip.5cm \cdot \Bigl\{ \left[1-\beta(1-\Pi)\right]
              (1-\beta)\kappa \Bigr\}^{-1}         \EQN koch3ex4;a \cr
w^- &= w^+ \,-\, {1-\Pi \over 1-\beta(1-\Pi) }
       \left[u(\hat c) - u(1-\hat c)\right] ,  \EQN koch3ex4;b \cr}
$$
where $\kappa = [1-(1-\Pi/2)\beta]^{-1}$.

To determine the optimal $\{\overline c, \hat c\}$ in this
three-state example, it is helpful to focus on a state in which
the agents realize different endowments,  say, state 3 in which
the type 1 agent realizes the highest endowment $\overline y$ and
is awarded consumption level $\overline c$. We can then exploit
the following two facts. First, $\overline c$ is the lower bound
of the consumption interval $[\underline c_3, \overline c_3]$, so
$\overline c$ is the consumption level that should be awarded to
the type 1 agent when she experiences the highest endowment
$\overline y_3 = \overline y$ and we want to maximize the welfare
of the type 2 agent subject to the type 1 agent's participation
constraint. Second, $\overline c$ belongs also to the ergodic set
in \Ep{c_ergodic3} that characterizes the stationary efficient
allocation, and we know that the associated efficient continuation
values are $w^+$ for the agents with high endowment and $w^-$ for
the other agents. By invoking functions \Ep{koch3ex4} that express
these continuation values in terms of $\{\overline c, \hat c\}$
and by using participation constraint \Ep{koch3ex1} that
determines permissible values of $\hat c$, the optimization
problem above becomes:
$$\EQNalign{
\max_{\overline c,\,\hat c}\hskip.2cm &u(1-\overline c) + \beta w^-  \EQN koch3ex5;a \cr
\hbox{\rm subject to }
\hskip.2cm  &u(\overline c) + \beta w^+  - \left[ u(\overline y)
  + \beta v_{\rm aut} \right] \geq 0                        \EQN koch3ex5;b \cr
&u(1-\hat c) + \beta w^- - \left[ u(0.5)
  + \beta v_{\rm aut} \right] \geq 0,                        \EQN koch3ex5;c \cr
%%%%            &\overline c, \hat c \in[0.5, 1], \cr
}$$
where $w^-$ and $w^+$ are given by \Ep{koch3ex4}.

To  illustrate graphically how an efficient stationary allocation
$\{\overline c, \hat c\}$ can be computed from optimization problem
\Ep{koch3ex5}, we assume a utility function $c^{1-\gamma}/(1-\gamma)$ and
parameter values $(\beta, \gamma, \Pi, \overline y)
= (0.7, 1.1, 0.6, 0.7)$. It should now be evident that
we can restrict attention
to consumption levels $\overline c\in[0.5, \overline y]$
and $\hat c\in[0.5, \overline c]$. Figure
%15.XXX.a and 15.XXX.b
%\Fg{icpairs }a and \Fg{icpairs}b
shows the sets
$(\overline c, \hat c)\in [0.5, \overline y]
\times[0.5, \overline c]$ that
satisfy participation constraint \Ep{koch3ex5;b}
and \Ep{koch3ex5;c}, respectively. The intersection of these
sets is depicted in Figure %Figure 15.XXX.c
\Fg{ic_bothf} where the
circle indicates the efficient stationary allocation
that maximizes \Ep{koch3ex5;a}.


%%%%%%% GGGGG Here is one of the April 2017 problems

%%%% Here is the attempted fix



\midfigure{icpairs}
\centerline{\epsfxsize=2.0truein\epsffile{ic_high.ps}, \epsfxsize=2.0truein\epsffile{ic_low.ps}}
\caption{Left panel: Pairs
of $(\overline c, \hat c)$ that satisfy
$u(\overline c) + \beta w^+ \geq u(\overline y)
  + \beta v_{\rm aut}$.  Right panel: Pairs of
$(\overline c, \hat c)$ that satisfy $u(1-\hat c) + \beta w^- \geq
u(0.5)
  + \beta v_{\rm aut}$. }
\infiglist{icpairs}
\endfigure
\medskip
%
%%\midinsert
%\midfigure{icpairs}
%$$\graftwo{ic_high.ps,height=1.8in}{{\bf Figure \Fg{icpairs}a:} Pairs
%of $(\overline c, \hat c)$ that satisfy
%$u(\overline c) + \beta w^+ \geq u(\overline y)
%  + \beta v_{\rm aut}$.}
%{ic_low.ps,height=1.8in}{{\bf Figure \Fg{icpairs}b:} Pairs of
%$(\overline c, \hat c)$ that satisfy $u(1-\hat c) + \beta w^- \geq
%u(0.5)
%  + \beta v_{\rm aut}$.} $$
%\endfigure
%%\endinsert




%%%%%%%%%%%%%
%$$\grafone{ic_both.ps,height=2in}{{\bf Figure 15.XXX.c} Pairs
%of $(\overline c, \hat c)$ that satisfy
%$u(\overline c) + \beta w^+ \geq u(\overline y)
%  + \beta v_{\rm aut}$ and $u(1-\hat c) + \beta w^- \geq u(0.5)
%  + \beta v_{\rm aut}$. The social maximun within this set is
%marked with a circle.}$$
%%%%%%%%%%%%%%%%%

\midfigure{ic_bothf}
\centerline{\epsfxsize=3truein\epsffile{ic_both.ps}}
\caption{Pairs of $(\overline c, \hat c)$ that satisfy
$u(\overline c) + \beta w^+ \geq u(\overline y)
  + \beta v_{\rm aut}$ and $u(1-\hat c) + \beta w^- \geq u(0.5)
+ \beta v_{\rm aut}$.  The efficient stationary allocation
within this set is marked with a circle.}
\infiglist{ic_bothf}
\endfigure

\subsection{Perturbation of parameter values}

We also compute efficient stationary allocations for different
values of $\Pi\in[0,1]$ while retaining all other parameter
values. As a function of $\Pi$, the two panels of Figure \Fg{contvpi}
  depict consumption levels and continuation values,
respectively. For low values of $\Pi$, we see that there cannot be
any risk sharing among the agents, so that autarky is the only
sustainable allocation. The explanation for this is as follows.
Given a low value of $\Pi$, an agent who has realized the high
endowment $\overline y$ is heavily discounting the insurance value
of any transfer in a future state when her endowment might drop to
$1-\overline y$ because such a state occurs only with a small
probability equal to $\Pi/2$. Hence, in order for that agent to
surrender some of her endowment in the current period, she must be
promised a significant combined payoff in that unlikely event of a
low endowment in the future and a positive transfer in the most
common state 2. But such promises are difficult to make compatible
with participation constraints, because all agents will be
discounting the value of any insurance arrangement as soon as the
common state 2 is realized since then there is once again only a
small probability of experiencing anything else.

When the probability of experiencing extreme values of the
endowment realization is set sufficiently high, there exist
efficient allocations that deliver risk sharing. When $\Pi$
exceeds $0.4$ in
the left panel of Figure %15.XXXX.a
\Fg{contvpi}, the lucky agent is persuaded to surrender some
of her endowment, and her consumption becomes $\overline c < \overline y$.
The lucky agent is compensated for her sacrifice not only
through the insurance value of being entitled to an equivalent transfer
in the future when she herself might realize the low endowment
$1-\overline y$ but also through a higher consumption level in
state 2, $\hat c > 0.5$.\NFootnote{Recall that we established in
section \use{sec:c_nondeg} that {\it all} consumption intervals are
nondegenerate if there is risk sharing. We can use this fact to
prove that as soon as the parameter value for $\Pi$ exceeds the
critical value where risk sharing becomes viable, it follows that
$\hat c = \overline c_2 > \overline y_2=0.5$.}
 In fact, if the consumption smoothing motive
could operate unhindered in this situation, the lucky agent's
consumption would indeed by equalized across states. But what hinders
such an outcome is the participation constraint of the unlucky agent
when entering state 2. It must be incentive compatible for that
earlier unlucky agent to give up parts of her endowment in state 2
when both agents now have the same endowment and the value of the insurance
arrangement lies in the future. Notice that this participation constraint of
the earlier unlucky agent is no longer binding in our example when
$\Pi$ is greater than $0.94$, because the efficient allocation prescribes
$\hat c = \overline c$.
In terms of Thomas and Worrall's characterization of the optimal
consumption dynamics, the parameterization is then such that
$\overline c_2 > \underline c_3$ and the ergodic set in
\Ep{c_ergodic3} is given by $\{\overline c_1, \underline c_3\}$
or, in our notation, by $\{1-\overline c, \overline c\}$.



%%%%%%% GGGGG Here another one of the April 2017 problems

%% here is the fixed version

\midfigure{contvpi}
\centerline{\epsfxsize=2.0truein\epsffile{c07.ps}, \epsfxsize=2.0truein\epsffile{v07.ps}}
\caption{Left panel:  Consumption
levels as a function of $\Pi$.
The solid line depicts $\overline c$, i.e., consumption in
states 1 and 3 of a person who realizes the highest
endowment $\overline y$. The dashed line depicts
$\hat c$, i.e., consumption in state 2 of the type of person
that was the last one to have received $\overline y$. Right panel: Continuation values
as a function of $\Pi$.
The solid line depicts $w^+$, i.e., continuation value of the
type of person that was the last one to have received
$\overline y$. The dashed line is the continuation value of
the other type of person, i.e., $w^-$.}
\infiglist{contvpi}
\endfigure
\medskip
%
%
%%\vfil\eject
%%\vskip.2cm%
%%\midinsert
%\midfigure{contvpi}
%$$\graftwo{c07.ps,height=1.7in}{{\bf Figure \Fg{contvpi}a:} Consumption
%levels as a function of $\Pi$.
%The solid line depicts $\overline c$, i.e., consumption in
%states 1 and 3 of a person who realizes the highest
%endowment $\overline y$. The dashed line depicts
%$\hat c$, i.e., consumption in state 2 of the type of person
%that was the last one to have received $\overline y$.}
%{v07.ps,height=1.7in}{{\bf Figure \Fg{contvpi}b:} Continuation values
%as a function of $\Pi$.
%The solid line depicts $w^+$, i.e., continuation value of the
%type of person that was the last one to have received
%$\overline y$. The dashed line is the continuation value of
%the other type of person, i.e., $w^-$.}  $$
%\endfigure
%\endinsert



The fact that the efficient allocation raises the consumption of
the lucky agent in future realizations of state 2 is reflected in
the
spread of continuation values in the right panel of Figure %15.XXX.b
\Fg{contvpi}. The spread vanishes only in the limit when $\Pi=1$
because then the three-state example turns into our two-state
example of the preceding section where there is only a single
continuation value. But while the planner is able to vary
continuation values in the three-state example, there remains an
important limitation to when those continuation values can be
varied. Consider a parameterization with $\Pi\in(0.4,\, 0.94)$ for
which we know that $\hat c < \overline c$ in  Figure
\Fg{contvpi}. The agent who last experienced the highest
endowment $\overline y$ is consuming $\hat c$ in state 2 in the
efficient stationary allocation, and is awarded continuation value
$w^+$. Suppose now that agent  once again realizes the highest
endowment $\overline y$ and his participation constraint becomes
binding. To prevent him from defecting to autarky, the planner
responds by raising his consumption to $\overline c$ ($> \hat c$)
{\it but\/} keeps his continuation value unchanged at $w^+$. In
other words, the optimal consumption dynamics in the efficient
stationary allocation leaves no room for increasing the
continuation value  further.
%%%Kocherlakota's erroneous claim above
%%%that $w^+$ should be raised in response to a binding participation
%%%constraint so long as $w^+<v_{\rm max}$, is associated with his
%%%mistaken assertion that the Pareto frontier $P(v)$ is
%%%differentiable.
The unchanging continuation value is a reflection of the
nondifferentiability of the Pareto frontier at $v=w^+$.
%%%Next we shall  compute the Pareto frontier to show
%%%that it is not differentiable at $v=w^+$.


\subsection{Pareto frontier}
As described in section \use{sec:Koch_Pareto2},
the {\it ex ante} division of the gains from
an efficient contract can be viewed as determined by an implicit
initial  consumption level, $c_{\triangle}\in[\overline y_1,
\overline y_S]$. In our symmetric environment, it is
sufficient to focus on half of this range because the other half
will just be the mirror image of those computations. Let us
therefore compute the Pareto frontier for $c_{\triangle}\in[0.5,
\overline y_3]\equiv[0.5, \overline y]$. We assume a
parameterization such that the consumption intervals,
$\{[\underline c_j, \overline c_j]\}_{j=1}^3$, are disjoint, i.e.,
the parameterization is such that $\hat c < \overline c$, which
corresponds to a parameterization with $\Pi\in(0.4,\, 0.94)$
in the left panel of Figure %15.XXX.
\Fg{contvpi}.

First, we study the formulas for computing $v$ and $P(v)$ in the
range $c_{\triangle} \in [0.5, \hat c]$:
$$\EQNalign{
v & = {\Pi\over2} \Bigl\{u(1-\overline c) + \beta w^-\Bigr\}
    + (1-\Pi) \Bigl\{u(c_{\triangle}) + \beta v \Bigr\}
    +  {\Pi\over2} \Bigl\{u(\overline c) + \beta w^+\Bigr\},    \cr
P(v) & = {\Pi\over2} \Bigl\{u(\overline c) + \beta w^+\Bigr\}
    + (1-\Pi) \Bigl\{u(1-c_{\triangle}) + \beta P(v) \Bigr\}   \cr
&    +  {\Pi\over2} \Bigl\{u(1-\overline c) + \beta w^-\Bigr\}. \cr}
$$
When the type 1 agent is assigned an implicit initial  consumption
level $c_{\triangle} \in [0.5, \hat c]$, her consumption is indeed
equal to $c_{\triangle}$ for any initial uninterrupted string of
realizations of state 2 with some continuation value $v$, and the
corresponding consumption of the type 2 agent is $1-c_{\triangle}$
with an associated continuation value $P(v)$. But as soon as
either state 1 or state 3 is realized for the first time, the
updating rules in \Ep{c_update} imply that the economy enters the
ergodic set of the efficient stationary allocation. In particular,
if state 1 is realized and the participation constraint of the
type 2 agent becomes binding, the type 1 agent is awarded
consumption $\overline c_1 \equiv 1-\overline c$ and continuation
value $w^-$ while the type 2 agent consumes $1-\overline c_1\equiv
\overline c$ with continuation value $P(w^-)=w^+$. But if state 3
is realized and the participation constraint of the type 1 agent
becomes binding, the type 1 agent is awarded consumption
$\underline c_3 \equiv \overline c$ and continuation value $w^+$
while the type 2 agent consumes $1-\underline c_3\equiv
1-\overline c$ with continuation value $P(w^+)=w^-$. Given the
implicit initial  consumption level $c_{\triangle}$, it is
straightforward to solve for the initial welfare assignment $\{v,
P(v)\}$ from the equations above.

Similarly, we can use the updating rules \Ep{c_update} to get
formulas for computing $v$ and $P(v)$ in the range $c_{\triangle}
\in [\overline c, \overline y]$:
$$\EQNalign{
v = & {\Pi\over2} \Bigl\{u(1-\overline c) + \beta w^-\Bigr\}
    + (1-\Pi) \Bigl\{u(\hat c) + \beta w^+ \Bigr\}
    +  {\Pi\over2} \Bigl\{u(c_{\triangle}) + \beta v\Bigr\},    \cr
P(v) = & {\Pi\over2} \Bigl\{u(\overline c) + \beta w^+\Bigr\}
    + (1-\Pi) \Bigl\{u(1-\bar c) + \beta w^- \Bigr\}              \cr
&    +  {\Pi\over2} \Bigl\{u(1-c_{\triangle}) + \beta P(v)\Bigr\}. \cr}
$$
Concerning the remaining range of implicit initial consumption
levels $c_{\triangle} \in (\hat c, \overline c)$, we can
immediately verify that either pair of equations above can be used
 when setting $c_{\triangle}=\hat c$ in the first pair of
equations or $c_{\triangle}=\overline c$ in the second pair of
equations. Hence, the initial welfare assignment is the same for
implicit initial consumption $c_{\triangle} \in [\hat c, \overline
c]$, and it is
given by $\{w^+, P(w^+)\}$. At this point in Figure \Fg{pareto3f} %16.XXXX,
the Pareto frontier becomes nondifferentiable.


\topfigure{pareto3f}
\centerline{\epsfxsize=3.5truein\epsffile{pareto3.ps}}
\caption{Pareto frontier $P(v)$ for the three-state symmetric example.
Kinks occur at coordinates $(v, P(v)) = (w^-, w^+)$ and
$(v, P(v)) = (w^+, w^-)$.}
\infiglist{pareto3f}
\endfigure




\section{Empirical motivation}
Kocherlakota was interested in the case of perpetual imperfect
risk sharing because he wanted to use his model to think about the
empirical findings from panel studies by Mace (1991), Cochrane
(1991), and Townsend (1994). Those studies found that,
 after conditioning
on aggregate income, individual consumption  and earnings are
positively correlated, belying the risk-sharing implications of
the complete markets models with recursive utility of the type we
studied in chapter \use{recurge}.   %When continuation utilities
%converge to a unique nontrivial invariant distribution,
So long as no first-best allocation is sustainable, the action of
the occasionally binding participation constraints lets the model
with two-sided lack of commitment reproduce that positive
conditional covariation. In recent work, Albarran and Attanasio
(2003) and Kehoe and Perri (2003a, 2003b) pursue more implications
of models like Kocherlakota's. \auth{Cochrane, John H.}
\auth{Mace, Barbara}\auth{Townsend, Robert M.} \auth{Attanasio,
Orazio} \auth{Albarran, Pedro} \auth{Krueger, Dirk} \auth{Perri,
Fabrizio}



%%By the way, you write on page 708 that if promised utility converges to a nontrivial invariant distribution, then consumption and income have a positive conditional correlation.  This is true - but perhaps a little imprecise.  Actually, as long as no first-best allocation is sustabinable, then consumption and income have a positive conditional correlation.  The nondenegeracy of the invariant distribution is necessary to guarantee that consumption is positively conditionally correlated with PAST income.


\section{Generalization}\label{LTW2002}%
Our formal analysis has followed the approach taken by Thomas and
Worrall (1988). We have converted the risk-neutral firm
into a risk-averse household, as suggested by Kocherlakota (1996b).
Another difference is that our analysis is cast in a general
equilibrium setting while Thomas and Worrall formulate a partial
equilibrium model where the firm  implicitly has access
to an outside credit market with a given gross interest rate
of $\beta^{-1}$ when maximizing the expected present value of
profits. However, this difference is not material, since
an efficient contract is such that wages never exceed
output.\NFootnote{The outcome that efficient wages do not
exceed output in Thomas and Worrall's (1988) analysis is
related to our ability to solve optimization problem
\Ep{new1} without imposing nonnegativity constraints on
consumption. See footnote \use{nonneg}.}
Hence, Thomas and Worrall's (1988)
analysis can equally well be thought of as a general equilibrium
analysis.

Ligon, Thomas, and Worrall (2002) further generalize the environment
by assuming that the endowment follows a Markov process.
This allows for the possibility of both aggregate and idiosyncratic
risk and serial correlation. The efficient contract is characterized
by an updating rule for the ratio of the marginal utilities of the
two households that resembles our updating rule for consumption
in \Ep{c_update}. Each state of nature is associated with a particular
interval of permissible ratios of marginal utilities. Given the current
state and the previous period's ratio of marginal utilities, the new
ratio lies within the interval associated with the current state, such
that the change is minimized. That is, if last period's ratio falls
outside of the current interval, then the ratio must
change to an endpoint of the current interval, and one of the
households will be constrained. But whenever possible, the ratio is
kept constant over time. This is consistent with
our analysis in chapter \use{recurge} of competitive equilibria with complete
markets (and full commitment). Expression \Ep{eq5}
states that these unconstrained first-best allocations are such that
ratios of marginal utilities between pairs of agents are constant
across all histories and dates.



\section{Decentralization}
By imposing constraints on each household's budget sets
above and beyond those imposed
by the standard household's budget constraint,
  Kehoe and Levine (1993)
\auth{Kehoe, Timothy}\auth{Levine, David K.}%
describe how to decentralize the optimal allocation
in an economy like   Kocherlakota's
with complete competitive markets at time $0$.
 Thus,
let $q^0_t(h_t)$ be the Arrow-Debreu time $0$ price of a unit of
time $t$ consumption after history $h_t$.
The two households' budget constraints
are
$$\EQNalign{
 \sum_{t=0}^\infty \sum_{h_t} q_t^0(h_t) c_t(h_t) &  \leq
 \sum_{t=0}^\infty \sum_{h_t} q_t^0(h_t) y_t \EQN kl1;a \cr
 \sum_{t=0}^\infty \sum_{h_t} q_t^0(h_t) (1- c_t(h_t)) & \leq
 \sum_{t=0}^\infty \sum_{h_t} q_t^0(h_t) (1-y_t) .\EQN kl1;b \cr}$$
Kehoe and Levine augment these standard budget constraints with
what were the planner's  ``participation constraints''
\Ep{sustain1;a}, \Ep{sustain1;b}, but which now have to be
interpreted as exogenous restrictions on the households' budget
sets, one restriction for each consumer for each $t\geq 0$
for each history $h_t$.

 Adding those restrictions leaves the household's   budget
sets convex.   That allows all of the assumptions of
the second welfare theorem to be fulfilled. That then
 implies that a competitive equilibrium (defined
in the standard way to include optimization and
market clearing, but with household budget sets being
further restricted by \Ep{sustain1}) will implement the
planner's optimal allocation.

%  Although mechanically this decentralization  works like a charm, it
%can nevertheless
%be argued that it conflicts with the
%spirit that at every time and contingency, households should be free
%to walk away from the contract.  In Kehoe and Levine's
%decentralization, all decisions and trades are
%made at time $0$:  households cannot
%renege on those time $0$ contracts because
%they confront no choices after date $0$.
%Partly because of this controversial feature of the
%Kehoe-Levine decentralization,
%Alvarez and Jermann use another  decentralization, one that is
%cast in terms of sequential trading of Arrow securities.
%\index{Arrow securities}%
%We turn to their work in the next section.


  Although mechanically this decentralization  works like a charm, it
can nevertheless
be argued that it conflicts with the
spirit of a competitive equilibrium in which agents take prices
as given and budget constraints are the only restrictions on agents'
consumption sets. In contrast,
participation constraints \Ep{sustain1;a} and \Ep{sustain1;b}
are now modelled as direct restrictions on agents' consumption possibility sets.
Partly because of this controversial feature of the
Kehoe-Levine decentralization,
Alvarez and Jermann use another  decentralization, one that
imposes portfolio/solvency constraints and is cast in terms of sequential
trading of Arrow securities.
\index{Arrow securities}%
The endogenously determined solvency constraints are agent and
state specific and ensure that the participation constraints are
satisfied. We turn to the Alvarez-Jermann decentralization in the next section.


In all fairness, one could argue that the alternative decentralization solely
converts one set of participation constraints into another one.
For both specifications we have a substantial departure from a decentralized
equilibrium under full commitment. When
removing the assumption of commitment, we are assigning a very demanding
task to the ``invisible hand'' who must not only look for market-clearing
prices but also check participation/solvency
constraints for all agents and all states of the world.


\section{Endogenous borrowing constraints}
Alvarez and Jermann (2000) alter Kehoe and Levine's
decentralization to attain a model with sequentially complete
markets in which households face what can be interpreted as
endogenous borrowing constraints.
\index{borrowing constraint!endogenous}%
Essentially, they accomplish this by showing how the standard
quantity constraints on Arrow securities (see chapter
\use{recurge}) can be appropriately tightened  to implement the
optimal allocation as constrained by the participation
constraints.  Their idea is to find borrowing constraints tight
enough to make the highest endowment agents adhere to the
allocation, while letting prices alone prompt lower endowment
agents to go along with it.\auth{Kehoe, Timothy} \auth{Levine,
David K.} \auth{Alvarez, Fernando} \auth{Jermann, Urban}

 For expositional simplicity,
we let $y^i(y)$ denote the endowment of a household of type $i$
when a representative household of type 1 receives
$y$. Recall the earlier assumption
that $[y^1(y), y^2(y)]=(y,1-y)$.
The state of the economy
is the current endowment realization $y$ and the
beginning-of-period asset holdings $A=(A_1, A_2)$, where
$A_i$ is the asset holding of a household of type $i$ and $A_1+A_2=0$.
Because asset holdings add to zero, it is sufficient to use $A_1$ to
characterize the wealth distribution.
Define the {\it state\/} of the economy as $X = \left[\matrix{y & A_1\cr}
\right]'$.
There is a
complete set of  markets in
one-period Arrow securities.
  In particular, let   $Q(X'|X)$
be the price of one unit of consumption in state $X'$ tomorrow
given state $X$ today. A household of type $i$ with
beginning-of-period assets $a$ can purchase and sell these securities
subject to the budget constraint
$$ c + \sum_{X'} Q(X' | X) a(X') \leq y^i(y) + a , \EQN budconstr$$
where $a(X')$ is the quantity purchased (if positive) or sold (if negative)
of Arrow securities that pay one unit
of consumption tomorrow if $X'$ is realized, and also
subject to the borrowing constraints
$$ a(X') \geq B^i(X') .  \EQN bconstr $$
Notice that
there is one constraint for each next period
state $X'$ and that
 the borrowing constraints reflect
history dependence through the presence of
$A'$.

 The Bellman equation for the household in the decentralized
economy is
$$ V^i(a,X) = \max_{c,\{a(X')\}_{X' \in {\bf X}}} \Bigl\{
               u(c) + \beta \sum_{X'} V^i[a(X'),X'] \Pi(X'|X) \Bigr\}
$$
subject to the budget constraint \Ep{budconstr} and   borrowing
constraints \Ep{bconstr}.
The equilibrium law of motion for the
asset distribution, $A_1$ is embedded in the conditional
distribution $\Pi(X'|X)$.



Alvarez and Jermann define a competitive equilibrium
with borrowing constraints in a standard way,
with the qualification that among the equilibrium
objects are the borrowing constraints $B^i(X')$,
functions that the households take
as given.  Alvarez and Jermann show how to choose the borrowing constraints
to make the allocation that solves the planning
problem be an equilibrium allocation.   They do so by construction,
identifying the elements of the borrowing constraints that are binding
from having identified the states in the planning problem where
one or another agent's participation constraint is binding.

  It is easy for Alvarez and Jermann to compute the equilibrium
pricing kernel from the allocation that solves the planning
problem.  The pricing kernel satisfies
$$ q(X' | X) = \max_{i=1,2} \beta {u'[c^i(a',X')] \over
                             u'[c^i(a,X)] } \Pi(X'|X) , \EQN AJ10 $$
where $c^i(a,X)$ is the consumption decision rule of a household
of type $i$ with beginning-of-period assets $a$.\NFootnote{For the
two-state example with $\beta=.85, \gamma =1.1, \overline
y=.6$, described in Figure \Fg{koch2s1bf}, we computed that
$\overline c=.536$, which implies that the risk-free
interest rate is
%claim to consumption next period is
 $1.0146$.  Note that with complete markets the risk-free
claim would be  $\beta^{-1} = 1.1765$.}
  People with the highest valuation of an asset {\it buy\/} it.  Buyers
of state-contingent securities are unconstrained, so they equate
their marginal rate of substitution to the price of the asset. At
equilibrium prices, sellers of state-contingent securities will occasionally like
to issue more, but are constrained from doing so by state-by-state
restrictions on the amounts that they can sell.  Thus, the intertemporal
marginal rate of substitution of an agent whose participation constraint
(or borrowing constraint) is {\it not\/} binding determines the
pricing kernel. % {\bf XXXX: Lars: I have added the following sentences in response to a student's
% suggestion that they help clarify matters.  }
A binding {\it participation\/} constraint translates
  into a binding {\it borrowing\/} constraint in the previous period.  A participation constraint for some state
  at $t$ restricts the amount of state-contingent debts that can be issued for that state at $t-1$.
   In effect, constrained and unconstrained agents have
their own ``personal interest rates'' at which they are just indifferent
between borrowing or lending a infinitesimally more.  A constrained
agent wants to consume {\it more\/} today at equilibrium prices
(i.e., at the shadow prices \Ep{AJ10} evaluated at the solution of the
planning problem), and thus has a high personal interest rate.  He
would like to sell more of the state-contingent security than he is
allowed to at the equilibrium state-date prices.  An agent would like
to {\it sell\/} state-contingent claims on consumption tomorrow in those
states in which he will be well endowed tomorrow.  But those high
endowment states are also the ones in which he will have an incentive
to default.  He must be restrained from doing so by limiting the volume
of debt that he is able to carry into those high endowment states.
This limits his ability to smooth consumption across high and low endowment
states.
%  \NFootnote{However, this statement has
%to be modified in the steady state allocation in the two-state
%example described above, because there is not enough room to move the
%continuation value.}
Thus, his consumption and continuation
value increases when  he enters one of those high endowment states
precisely because he has been prevented from selling enough
claims to smooth his consumption over time and across states.
%But an increased
%continuation value means that his consumption can be
%anticipated to increase over time, which the constrained agent
%would like to smooth out by borrowing.  He is deterred from doing so
%by the state-by-state quantity constraints on the amount that he can  credibly
%promise to repay next period.


From a general equilibrium perspective, when sellers of a state-contingent security are constrained with respect to the quantities that they can issue, it follows that the price is bid up when unconstrained buyers are competing for a smaller volume of that security. This tendency of lowering the yield on individual Arrow securities explains Alvarez and Jermann's result that interest rates are lower, when compared to a corresponding complete markets economy, a property shared with the Bewley economies studied in chapter 18.\NFootnote{In exercise {\it \the\chapternum.4,\/} we ask the reader
to compute the allocation and interest rate in such an economy.}


%XXXXX deleted in response to Lars's recommendation, May 2, 2012
% The quantity constraints do not bind for the agents who receive
%a low endowment and a declining continuation value.  Price signals
%in the form of {\it low\/} interest rates can reconcile them to
%accept the declining consumption allocations that they face.  Quantity
%constraints for the unconstrained agents  don't bind, but the
%unconstrained agents must be induced to accept a consumption trajectory
%that they expect to be {\it decreasing\/},  i.e., their
%continuation values must decrease as the planner moves continuation
%values along the Pareto frontier $P(v)$ in order to increase
%the continuation values of the constrained agents.  In the decentralization,
%higher one-period state-contingent prices, or what are the same
%things, lower state-contingent interest rates (see \Ep{AJ10}),
%induce the unconstrained agents to accept a decreasing consumption
%trajectory.  Thus, when compared to a corresponding complete markets
%economy without enforcement problems, this is a {\it low-interest-rate
%economy\/}, a property it shares with the Bewley economies studied in
%chapter \use{incomplete}.\NFootnote{In exercise {\it \the\chapternum.4\/}, we ask the reader
%to compute the allocation and interest rate in such an economy.}

Alvarez and Jermann study how the state-contingent prices
\Ep{AJ10} behave as they vary the discount factor and the
stochastic process for $y$.  They use the additional fluctuation
in the stochastic discount factor injected by the participation
constraints to explain some asset pricing puzzles.  See Zhang
(1997) and Lustig (2000, 2003) for further work along these lines.
\auth{Lustig, Hanno} \auth{Zhang, Harold}

\section{Concluding remarks}

   The model in this chapter assumes that the economy
   reverts to an autarkic allocation in the event that a household
chooses to deviate from the allocation assigned in the contract.
Of course, assigning  autarky continuation values to {\it
everyone\/} puts us inside the Pareto frontier and so is
inefficient. In terms of sustaining an allocation, the important
feature of  the autarky allocation  is just the continuation value
that it assigns to an agent who is tempted to default, i.e., an
agent whose participation constraint binds. Kletzer and Wright
(2000) recognize that it can be possible to promise an agent who
is tempted to default an autarky continuation value while giving
those agents whose participation constraints aren't binding enough
to stay on the Pareto frontier. Continuation values that lie on
the Pareto frontier are said to be ``\idx{renegotiation proof}''.

Further thought about how to model the consequences of default in
these settings is likely to be fruitful. By  permitting coalitions
of consumers to break away and thereafter share risks among
themselves, Genicot and Ray (2003)  refine a notion of
sustainability in
 a  multi-consumer economy.

\auth{Kletzer, Kenneth M.} \auth{Wright, Brian D.}
 \auth{Genicot, Garance}\auth{Ray, Debraj}
 \showchaptIDfalse \showsectIDfalse
\section{Exercises}
\showchaptIDtrue
\showsectIDtrue
\medskip
\noindent{\it Exercise \the\chapternum.1}\quad {\bf Lagrangian method with
two-sided no commitment}
\smallskip
\noindent
   Consider the model of Kocherlakota with two-sided
lack of commitment. Two consumers each
have preferences $E_0 \sum_{t=0}^\infty \beta^t u[c_i(t)]$,
where $u$ is increasing, twice differentiable, and strictly
concave, and
where $c_i(t)$ is the consumption of consumer $i$.
The good is not storable, and the consumption allocation must satisfy
$c_1(t) + c_2(t) \leq 1$.  In period $t$, consumer $1$ receives
an endowment of
$y_t \in [0, 1]$, and consumer   2 receives an endowment
of $1 - y_t$.  Assume that $y_t$ is i.i.d.\ over time and is
distributed according to the discrete distribution
${\rm Prob} (y_t = y_s) = \Pi_s$.  At the start of  each period,
after the realization of $y_s$ but before consumption has occurred,
each consumer is free to walk away from the loan contract.
\medskip
\noindent {\bf a.}    Find expressions for
the expected value of autarky, before the state $y_s$ is revealed, for
consumers of each type.  ({\it Note:} These need not be equal.)

\medskip
\noindent{\bf b.}  Using the Lagrangian method, formulate
the contract design problem of finding an optimal allocation
that for each history respects feasibility and the participation constraints
of the two types of consumers.

\medskip
\noindent{\bf c.}  Use the Lagrangian method to characterize the
optimal contract as completely as you can.

\medskip

\noindent{\it Exercise  \the\chapternum.2}\quad {\bf A model of Dixit, Grossman, and Gul (2000)}
\medskip
\noindent For each date $t \geq 0$,
 two political parties divide a ``pie'' of fixed size $1$.  Party
$1$ receives  a sequence of shares $ y =\{y_t\}_{t \geq 0}$
and has   utility function
$E\sum_{t=0}^\infty \beta^t U(y_t)$, where $\beta \in (0,1)$,
$E$ is  the  mathematical  expectation operator,
and $U(\cdot)$ is an increasing, strictly concave, twice
differentiable period utility function. Party $2$ receives
share $1-y_t$ and has utility function
$E \sum_{t=0}^\infty \beta^t U(1-y_t)$.   A state variable
 $X_t$ is governed by  a
 Markov   process; $X$ resides in one of $K$ states.   There
is a partition $S_1, S_2$ of the state space.  If     $X_t \in S_1$,
party $1$ chooses the division $y_t, 1-y_t$, where
$y_t$ is the share of party $1$.  If $X_t \in S_2$, party
$2$ chooses the division.
At each point in  time, each  party has  the option
of choosing ``autarky,'' in which  case its share is $1$ when
it is in power and zero when it is not in power.
\auth{Gul, Faruk}\auth{Grossman, Gene M.}\auth{Dixit, Avinash K.}

  Formulate the optimal history-dependent sharing rule as a
recursive contract.  Formulate the Bellman equation.
({\it Hint:} Let $V[u_0(x),x]$ be the optimal value for party $1$ in
state $x$ when party $2$ is promised value $u_0(x)$.)

\medskip
\noindent{\it Exercise \the\chapternum.3}\quad {\bf Two-state numerical example of
social insurance}
\medskip
\noindent
Consider an endowment economy populated by a large number of individuals
with identical preferences,
$$
E \sum_{t=0}^\infty \beta^t u(c_t) =
E \sum_{t=0}^\infty \beta^t \left(4 c_t - {c_t^2 \over 2}\right),
\qquad \hbox{with} \ \ \beta=0.8.
$$
With respect to endowments, the individuals are divided into two types of
equal size.  All individuals of a particular type receive zero goods with
probability 0.5 and two goods with probability 0.5 in any given period.
The endowments of the two types of individuals are perfectly
negatively correlated so that the per capita endowment is always one good in
every period.

The planner attaches the same welfare weight to all individuals.
Without access to outside funds or borrowing and lending
opportunities, the planner seeks to provide insurance by simply
reallocating goods between the two types of individuals. The
design of the social insurance contract is constrained by a lack
of commitment on behalf of the individuals. The individuals are
free to walk away from any social arrangement, but they must then
live in autarky evermore.

\medskip
\noindent {\bf a.}   Compute the optimal insurance contract when
the planner lacks memory; that is, transfers in any given period
can be a function only of the current endowment realization.

\medskip
\noindent {\bf b.} Can the insurance contract in part a be
improved if we allow for history-dependent transfers?
%(Let the optimal recursive contract be designed prior to seeing
%the endowment realization.)

\medskip
\noindent {\bf c.}  Explain how the optimal contract changes when the
parameter $\beta$ goes to $1$. Explain how the optimal contract
changes when the parameter $\beta$ goes to zero.

\medskip
\noindent{\it Exercise \the\chapternum.4} \quad {\bf Kehoe-Levine without risk}

\medskip
\noindent Consider an economy in which each of two types
of households has preferences over streams of
a single good that are ordered by
$ v= \sum_{t=0}^\infty \beta^t u(c_t) $, where
$u(c) = (1-\gamma)^{-1} (c+b)^{1-\gamma}$ for $\gamma \geq 1$ and
$\beta \in (0,1)$, and $b>0$.
For $\epsilon >0$ and $t\geq 0$,
households of type 1 are endowed with an endowment stream
$y_{1,t} = 1+\epsilon$ in even-numbered periods and
$y_{1,t}=1-\epsilon$ in odd-numbered periods.  Households
of type 2 own an endowment stream of  $y_{2,t}$ that
equals $1-\epsilon$ in even periods and $1+\epsilon$ in
odd periods.  There are equal numbers of the two types of household.
For convenience, you can assume that there is one of each type of
household.

\smallskip
\noindent Assume that $\beta=.8$, $b=5$, $\gamma=2$,  and $\epsilon=.5$.

\medskip
\noindent{\bf a.}  Compute autarky levels of  discounted
utility  $v$ for the two types of
households.  Call them $v_{{\rm aut},h}$ and $v_{{\rm aut}, \ell}$.

\medskip
\noindent{\bf b.}  Compute the competitive equilibrium allocation
and prices.  Here assume that there are no enforcement
problems.


\medskip
\noindent{\bf c.} Compute the  discounted utility to each household for the
competitive equilibrium allocation.   Denote them
$v_i^{CE}$ for $i=1,2$.

\medskip
\noindent{\bf d.}  Verify that the competitive equilibrium
allocation is not self-enforcing in the sense that at each
$t >0$, some households  would prefer autarky to the competitive
equilibrium allocation.

\medskip
\noindent{\bf e.}   Now assume that there are enforcement problems
because at the beginning of each period, each  household can
renege on contracts and other social arrangements  with the consequence
that it receives the autarkic allocation from that period on.
 Let $v_i$  be the discounted utility at time $0$ of consumer $i$.
Formulate the consumption smoothing problem of a planner
who wants to maximize $v_1$ subject to $v_2 \geq \tilde v_2$,
and constraints that   make the allocation
 self-enforcing.

\medskip
\noindent{\bf f.}   Find an efficient
 self-enforcing allocation of the periodic form
$c_{1,t} = \check c, 2- \check c, \check c , \ldots$
and
$c_{2,t} = 2- \check c, \check c, 2-  \check c , \ldots$,
where continuation utilities of the two agents
oscillate between two values  $v_h$ and $v_\ell$.
Compute $\check c$.
Compute discounted utilities $v_h$ for the agent who receives
$1+\epsilon$ in the period and $v_\ell$ for the agent
who receives $1-\epsilon$ in the period.

\medskip
\noindent Plot consumption paths for the two agents for (i) autarky,
(ii) complete markets without enforcement problems, and
(iii) complete markets with the enforcement constraint.
Plot continuation utilities for the two agents for  the
same three allocations.  Comment on them.

\medskip
\noindent{\bf  g.}   Compute one-period gross interest rates
in the complete markets economies with and without enforcement
constraints.  Plot them over time. In which economy
is the interest rate higher?  Explain.

\medskip
\noindent{\bf h.}   Keep all parameters the same, but gradually
increase the discount factor.   As you raise $\beta$ toward
$1$, compute interest rates
 as in part g.  At  what value of $\beta$ do interest rates
in the two economies become equal? At that value of $\beta$
is either participation constraint ever binding?


\bigskip
\noindent{\it Exercise \the\chapternum.5} \quad {\bf The kink}
\medskip
\noindent A pure endowment  economy consists of two {\it
ex ante\/} identical consumers each of whom values streams of a
single nondurable consumption good according to the utility
functional
$$ v= E \sum_{t=0}^\infty \beta^t u(c_t), \quad \beta
\in (0,1)   $$ where $E$  is the mathematical expectation
operator and $u(\cdot)$ is a strictly concave, increasing, and
twice continuously differentiable function. The endowment sequence
 of consumer $1$ is an i.i.d.\ process with ${\rm Prob} (y_t =
 \overline y )=.5$ and ${\rm Prob} (y_t =
 1-\overline y )=.5$ where $\overline y \in [.5,1)$.  The
 endowment sequence of consumer $2$ is identically distributed
 with that of consumer $1$, but perfectly negatively correlated
 with it: whenever consumer $1$ receives $\overline y$, consumer
 $2$ receives $1 -\overline y$.\vfil\eject
 \medskip
 \noindent{\bf Part I. (Complete markets)}
 \medskip
 \noindent In this part, please assume that
there are no enforcement (or commitment) problems.
 \medskip
 \noindent{\bf a.} Solve the Pareto problem for this economy,
 attaching equal weights to the two types of consumer.
 \medskip
 \noindent{\bf b.}  Show how to decentralize the allocation that
 solves the Pareto problem with a competitive equilibrium with
 {\it ex ante\/} (i.e., before time $0$) trading of a complete set
 of
 history-contingent commodities. Please calculate the
 price of a one-period risk-free security.

 \medskip
 \noindent {\bf Part II. (Enforcement problems)}
 \medskip
 \noindent
In this part, assume that there are enforcement problems. In
particular, assume that there is two-sided lack of
 commitment.
 \medskip
 \noindent{\bf c.}
  Pose an {\it ex ante\/} Pareto problem in which,
 after having observed its current
 endowment but before receiving his allocation from the Pareto
 planner, each
 consumer is free at any time to defect from the social contract and live thereafter in
 autarky.  Show how to compute the value of autarky for each type of consumer.

 \medskip\noindent{\bf d.}
 Call an allocation {\it sustainable\/} if neither
 household would ever choose to defect to autarky.
Formulate the enforcement-constrained Pareto problem
 recursively.  That is, please write a programming problem  that can be used to
 compute an optimal sustainable allocation.
 \medskip
 \noindent{\bf e.}  Under what circumstance will the allocation
 that you found in part I solve the enforcement-constrained Pareto
 problem in part d?  I.e., state conditions on $u, \beta,
 \overline y$ that are sufficient to make the enforcement
 constraints never bind.

 \medskip
\noindent{\bf Some useful background:} For the remainder of this
problem, please assume that $u,\beta, \overline y, $ are such that
the allocation
 computed in part I is not sustainable.     Recall that
the amnesia property  implies that the
 consumption allocated to an agent whose participation constraint
 is binding is independent of the  {\it ex ante\/} promised value with which he enters the
 period.  With the
 present i.i.d., two-state, symmetric endowment pattern, {\it ex ante\/},
 each period each of
 our two agents has an equal chance that it is his participation constraint
 that is binding.  In a symmetric sustainable allocation, let
each agent {\it enter\/} the period with the {\it same\/} {\it ex
ante\/} promised value $v$, and
   let $\overline c$ be the consumption allocated to
 the high endowment agent whose participation constraint is binding and let
 $1-\overline c$ be the consumption allocated to the low endowment agent whose
 participation constraint is not binding. By the above argument,
  $\overline c$ is independent of the promised value
 $v$ that an agent enters the period with, which means that the current allocation
 to {\it both\/} types of agent does not depend on the promised value with which they entered the period.
   And in a symmetric stationary sustainable allocation,
 both consumers enter each period with the same promised value $v$.

 \medskip
 \noindent{\bf f.}
 Please give a formula for the promised value $v$ within
 a symmetric stationary sustainable allocation.


\medskip
\noindent{\bf g.} Use a graphical argument to show how to
determine the $v, \overline c$ that are associated with an optimal
stationary symmetric allocation.

\medskip
\noindent{\bf h.}  In the optimal stationary sustainable
allocation that you computed in part g, why doesn't the planner
adjust the continuation value of the consumer whose participation
constraint is binding?

\medskip
\noindent{\bf i.} Alvarez and Jermann showed that, provided that
the usual constraints on issuing Arrow securities are tightened
enough,  the optimal sustainable allocation can be decentralized
by trading in a complete set of Arrow securities with price
$$ q(y' | y) = \max_{i=1,2} \beta {u'(c^i_{t+1}(y')) \over
                             u'(c^i_t(y)) }.5, $$
where $q(y'|y)$ is the price of one unit of consumption tomorrow,
contingent on tomorrow's endowment of the type $1$ person being
$y'$ when it is $y$ today. This formula has each Arrow security
being priced by the agent whose participation constraint is {\it
not\/} binding.  Heuristically, the agent who wants to {\it buy\/}
the state-contingent security determines its price because the agent who
wants to sell it is constrained from selling more by a limitation
on the quantity of Arrow securities that he can promise to deliver
in that future state. Evidently the gross rate of interest on a
one-period risk-free security is
$$ R(y)= {1 \over \sum_{y'} q(y'|y) } ,$$
for $y= \overline y$ and $y=1-\overline y$.
\medskip
 For the case in which the parameters are such that the
allocation computed in part I is {\it not\/} sustainable (so that
the participation constraints bind), please compute the risk-free
rate of interest.  Is it higher or lower than that for the
complete markets economy without enforcement problems that you
analyzed in part I?
