%% file: index.tex - creating an index - TeXsis 2.18
% @(#) $Id: index.tex,v 18.0 2000/05/16 22:37:11 myers Exp $
%======================================================================*
%  index.tex  --  Macros for creating an index using MakeIndex
%
%  Use \index{key} or \idx{word} to make an entry in the index file
%  [\idx{word} is more or less the same as ``word\index{word}''].
%  At the end of the job the file \jobname.idx is the raw index file, 
%  ready to be fed into the MakeIndex program to create the index.  
%  The MakeIndex output file \jobname.ind is then fed back through TeX with
%  the output formatting macros below to print the final formatted index.
%
%  To process the index file with MakeIndex the command is simply
%
%       makeindex  jobname.idx
%
%  You can leave off the .idx extension for the raw file and makeindex
%  will assume it.  This will produce the processed index file, called
%  \jobname.ind.  To print the index you can either include it in 
%  another pass of the full document with ``\input \jobname.ind'' or
%  print it separately by adding ``\input index'' to the begining of the 
%  \jobname.ind file and running it through TeX.
%
%  If you set \markindextrue then indexed words selected with \idx
%  will appear in the printed output inside a ruled box.  This is
%  useful for draft copies when you are putting together the index,
%  but obviously you do not want to use it in a final copy.
%  The box does not interfere with or change the line spacing.
%
%  These macros are completely compatible with Plain TeX, although
%  they can also be used with TeXsis (some of the macros used below
%  are a part of TeXsis). Note however that these macros support output 
%  from MakeIndex in it's default LaTeX mode.  You don't need a separate
%  style file for MakeIndex!
%
%  For more information about how to make full use of MakeIndex read
%  the man page for makeindex(1) and/or the paper ``Index Preparation 
%  and Processing'' by Pehong Chen and Michale A. Harrison, 
%  Software: Practice and Experience, Vol 19, No. 9, Sep. 1988, pp. 897-915  
%  (a copy is available at ftp://ftp.texsis.org/texsis/index/index.tex )
%
%  It is possible to make more than one index by creating a few extra
%  macros based on the ones below.  See the documentation at the end
%  of this file for details.
%
%======================================================================*
% (C) Copyright 1992, 1999, 2000 by Eric Myers and Frank Paige.
% This file is a part of TeXsis.  Distribution and/or modifications
% are allowed under the terms of the LaTeX Project Public License (LPPL).
% See the file COPYING or ftp://ftp.texsis.org/texsis/LPPL
%======================================================================*
\message{- indexing macros}
\catcode`@=11                           % @ is a letter here to hide workings

\newif\ifmarkindex \markindexfalse      % put proof marks in text?
\newif\ifmakeindex \makeindextrue       % set false to suppress .idx output

%---------------------------*
% Index Entries:
%
% \idx{word} is the quickest way to make an index entry.
% The word is put in the text and an index entry is made at the
% same time.  It is the same as typing ``word\index{word}''.
% If \markintdextrue is selected the word appears in the text with a 
% box around it so you know it has been included in the index.


\def\idx#1{% include a word in the index output file
   #1\index{#1}%                        % put word in text and \index it
   \ifmarkindex                         % proof marks?
     \llap{\lower\jot\vbox to 0pt{\vss  % box it
        \tightboxit{\loosebox{\phantom{#1}}}}\hskip-\jot}%
   \fi}                                  % put a ruled box around word


% \index{key} makes an index entry for the given key.
% Since MakeIndex can use special characters like @ | ! and " in the key
% we must make sure that they are printed correctly to the raw index file.

\def\index{%  make an index entry for a given key
   \begingroup                  % turn MakeIndex special characters to letters
     \catcode`@=11 \catcode`"=11 \catcode`!=11 \catcode`|=11 \catcode`-=11 
    \@index}

\def\@index#1{%  get the index entry and write it to the file
   \endgroup                    % special characters back to before
     \ifmakeindex               % .idx file output enabled?
       \indexwrite\indexout{\indexentry{#1}}% write entry to file
     \fi}


% \loosebox put a loose  box around a box.
% \tightboxit draws a rule box around the box #1 without any added space

\def\loosebox#1{%
    \vbox{\vskip\jot
        \hbox{\hskip\jot #1\hskip\jot}%
        \vskip\jot}}

\def\tightboxit#1{\vbox{\hrule\hbox{\vrule\vbox{#1}\vrule}\hrule}}

%---------------------------*
% Index Formatting (printing the processed index file) 


% The index is begun by \theindex or \begin{theindex} and ended
% with \endtheindex or \end{theindex}


\def\theindex{%  begin a formatted index
   \begingroup
     \parskip = 0pt                     % no extra space between items
     \def\item{\indexitem{0pt}}%        % level 0 item separator
     \def\subitem{\indexitem{20pt}}%    % level 1 item separator
     \def\subsubitem{\indexitem{30pt}}% % level 2 item separator
     \def\indexspace{\medskip}%         % intergroup vertical space
     \def\setcounter##1##2{\global\csname ##1no\endcsname=##2}% to set page counter
     \def\bold##1{{\bf ##1}}%           % boldface for |bold
     \def\italic##1{{\it ##1}}%         % italics for |italic
     \def\slanted##1{{\sl ##1}}%        % slanted for |slanted
     \def\see##1##2{{\it see} ##1}%     % italics for |see
     \def\seealso##1{{\it see also}##1}%% italics for |seealso
     \def\also##1{{\it see also}##1}%   % italics for |also
   }

\def\indexitem#1{%  how to make an entry for an item in the index
     \vskip 0pt                         % force vertical mode
     \hangindent=30pt\hangafter=1       % hanging indent for extra lines
     \noindent\hskip #1}                % level N item separator

\def\endtheindex{%  end a formatted index
   \endgroup}


% If you don't have MakeIndex you can still make a sorted index.
% (Even though it may not look as good.  If you are really going to
% do it right, do get MakeIndex).  The Unix sort(1) command or a similar
% program can be used to sort the \jobname.idx file.  You can then
% edit the file to collect all the page numbers for a key into 
% one \indexentry, like ``\indexentry{foo}{29,47-48}'' , and \indexentry
% will format the items as best it can. 


\def\indexentry#1#2{%  prints an index entry (from the raw file)
     \indexitem{0pt}#1, #2}             % keyword, page number


%-------------------------*
% Simple \begin .. \end support (like in LaTeX):  There is a conspiracy
% to preserve job security for typists by making sure one has to 
% type \begin{thing} and \end{thing} everywhere.  In TeXsis it's just
% \thing and \endthing, so here we just make \begin{thing} become \thing
% and \end{thing} become \endthing.  Then we can use MakeIndex in it's
% default mode, even though that is for LaTeX.
% Note: we don't check that you begin and end the same thing.  After all,
% you might want to say \begin{midfigure} followed by \end{figure}.


\def\begin#1{%  \begin{foo} just becomes \foo
   \begingroup                  % changes to \end are temporary
     \let\end=\endbegin         % now \end will end the \begin
     \expandafter\ifx\csname #1\endcsname\relax\relax % is \thing defined?
        \def\next{\beginerror{#1}}% NO: flag an error
     \else                      %
        \def\next{\csname #1\endcsname}% YES: just invoke it
     \fi\next}

\def\endbegin#1{%  \end{foo} just becomes \endfoo
   \endgroup                    % now \end goes back to normal
   \expandafter\ifx\csname end#1\endcsname\relax\relax % is \endthing defined?
      \def\next{\begingroup\beginerror{end#1}}% % NO: flag an error
   \else                        %
      \def\next{\csname end#1\endcsname}% YES: just invoke it
   \fi\next}


\newhelp\beginhelp{begin: 
    The \string\begin\space or \string\end\space marked above is for a^^M% 
    non-existant environment.  Check for spelling errors and such.}


\def\beginerror#1{% \begin{foo} or \end{foo} fail if no \foo or \endfoo
   \endgroup                    % make sure \end is back to normal
   \errhelp=\beginhelp          % longer help message
   \newlinechar=`\^^M           % ^^M is line break
   \errmessage{Undefined environment for \string\begin\space or \string\end}}


%---------------------------*
% Index file output:  
%

\newwrite\indexout              % raw index output file


% \indexinit will only get called once.  It opens the output file,
% write an identifying comment to it, and then undefines itself.

\def\indexinit{%                     open and initialize raw index file
  \immediate\openout\indexout=\jobname.idx % open file as \jobname.idx
  \immediate\write\indexout{\@comment Raw index file ``\jobname.idx''}%
  \immediate\write\indexout{\@comment ====================================}%
  \gdef\indexinit{\relax}}      % so we initialize only once

%  \@comment is for writing % to a file.  It expands to '% '.
     
{\catcode`\%=11 \gdef\@comment{% }}


% \indexwrite  writes an entry to the raw index file named by #1. Argument
% #2 is the \indexentry{foo}, so it is not expanded (which is why this is so 
% complicated), while \@indexpage is the page number, which *is* expanded.

\def\indexwrite#1#2{%           % write entry to raw index file
   \indexinit                   % open output file if not already
   \def\@finwrite{\write#1}%    % to write to index output file (#1)
   \begingroup                  % begin collecting tokens on stack
    \aftergroup\@finwrite       % put \write\indexout on stack
    \aftergroup{\relax          % put { on stack
    \@stackNX#2\@endstack       % puts \noexpand\token... on stack 
    \aftergroup\@indexpage      % expanded {enclosed} page number
    \aftergroup}\relax          % put } on stack
   \endgroup}                   % now dump from stack to \write

\def\@indexpage{{\folio}}       % to get the {enclosed} page number 

% Saying \@stackNX <stuff>\@endstack will take all the tokens
% in <stuff> and put them on the stack, with every control sequence
% prefaced by \noexpand.  \indexwrite uses this for unexpanded write.

\def\@stackNX{\futurelet\next\@stackswitch} 

\def\@stackswitch{%        decides what to do with various types of tokens
    \ifx\next\@endstack\relax           % if \@endstack then we'll quit
    \else\ifcat\noexpand\next\@stoken   % if token is a space,
        \aftergroup\space\let\next=\@eatnext%   write \space and eat token
    \else\ifcat\noexpand\next\bgroup    % if token is \bgroup
        \aftergroup{\let\next=\@eatnext %   write { and eat it
    \else\ifcat\noexpand\next\egroup    % if token is \egroup
        \aftergroup}\let\next=\@eatnext %   write a } and eat it
    \else                               % otherwize
        \let\next=\@stacktoken          %   just copy to the stack with \NX
     \fi\fi\fi\fi                       % end all \if's
     \next}                             % do it to the token

\def\\{\global\let\@stoken= }\\ 

% \@eatnext ignores the \next token and goes on to the 
%  one following in the argument list

\def\@eatnext{\afterassignment\@stackNX\let\next= } 

% \@stacktoken copies one token to the stack with \aftergroup,
%  then calls \@stackNX to get the one following.

\long\def\@stacktoken#1{% write token to stack, unexpanded    
    \ifcat\noexpand#1\relax             % if token is a control sequence
        \aftergroup\noexpand            %   preface with \noexpand
    \else\ifcat\noexpand#1\noexpand~\relax % if token is an active character
        \aftergroup\noexpand            %   preface with \noexpand
    \fi\fi                              %
    \aftergroup#1\relax                 % put the next token on the list
    \@stackNX}                          % do next token from argument...

\def\@endstack\@endstack{}              % just marks the end of token list

\catcode`@=12                           % @ goes back to \other, not a letter

%---------------------------*
% MULTIPLE INDICES:  
%  
% If you want more than one index I can't guess how you will want to
% use them, so all I can do is tell you how to use the macros above to
% create another (or several other) additional indices:
%
% 1) Create another output file and open it, with something like:
%
%       \newwrite\otherindexout   
%       \immediate\openout\otherindexout=other.idx
%
% 2) Define a macro to make the entry to the second index, following 
%    the pattern of \index and \@index above, but using the other index
%    file (but don't use @ in a macro name).  For example:
%
%       \def\OtherIndex{\begingroup     % special characters letters
%           \catcode`@=11 \catcode`"=11 \catcode`!=11 \catcode`|=11
%               \catcode`-=11  
%           \Oindex}
%
%       \def\Oindex#1{\endgroup    % special characters return
%              \indexwrite\otherindexout{\indexentry{#1}}}
%
%    Then saying \OtherIndex{word} will make an entry into the alternate
%    index for ``word''.
%
%>>> EOF index.tex <<<
