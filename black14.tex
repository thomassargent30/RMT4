%%% XXXXX Tom: check whether \bbR and {\bf R} are consistent throughout book
\input grafinp3
%\input grafinput8
\input psfig
%\eqnotracetrue
%\ReadAUX
%\input form1

%\input psfig
%\input grafinput8

%\showchaptIDtrue
%\def\@chaptID{14.}

%\hbox{}
\footnum=0
\chapter{Incomplete Markets Models\label{incomplete}}
\offparens
\section{Introduction}

 In the complete markets model of chapter \use{recurge},
the optimal consumption allocation is not history dependent; the allocation
depends on the current value of the  Markov state variable only.  This
outcome reflects the comprehensive opportunities to insure risks that
markets provide.   This chapter and chapter \use{socialinsurance}
describe settings with more impediments to exchanging risks.  These
reduced opportunities make allocations history dependent.  In this
chapter, the history dependence is encoded in the dependence
of a household's  consumption on the household's current asset
holdings.  In chapter \use{socialinsurance}, history dependence
is encoded in the dependence of the consumption allocation on a
continuation value promised by a planner or principal.

\auth{Bewley, Truman}%
\index{Bewley models}%
\index{equilibrium!incomplete markets}
  The present chapter describes a particular type of incomplete
markets model.
 The models have a large number of {\it ex ante\/} identical but
{\it ex post\/} heterogeneous agents who trade a single security.
For most of this chapter, we study models with no
aggregate uncertainty and no variation of an aggregate
state variable over time (so macroeconomic time series
variation is absent).  But there is much uncertainty at
the individual level.  Households' only option
is to ``self-insure'' \index{self-insurance} by managing a
stock of a single  asset to buffer their consumption against
adverse shocks.  We study several models that differ mainly
with respect to the particular asset that is the vehicle
for \idx{self-insurance},
\index{risk-sharing mechanisms}
for example, fiat currency or  capital.
\auth{Bewley, Truman}%
\auth{Aiyagari, Rao} \auth{Huggett, Mark}%

The tools for constructing these models are discrete-state
discounted dynamic programming, used to formulate and solve
problems of the individuals, and Markov chains, used to
compute a stationary wealth distribution. The models produce
a stationary wealth distribution that is determined
simultaneously with various aggregates that are defined
as means across corresponding individual-level variables.

We begin by recalling our discrete-state formulation of a
single-agent infinite horizon savings problem. We then describe
several economies in which  households face some version of
this infinite horizon savings problem, and where some of the
prices taken parametrically in each household's problem
are determined by the {\it average\/} behavior of
all households.%\NFootnote{Most of the
%heterogeneous-agent models in this chapter have been
%arranged to shut down aggregate variations over time,
%to avoid the ``curse of dimensionality'' that comes into play
%in formulating the household's dynamic programming problem when
%there is an aggregate state variable. But we also describe a  model of
%Krusell and Smith (1998) that has an aggregate state variable.}

   This class of models was invented by Bewley (1977, 1980, 1983, 1986),
partly to study a set of classic issues in monetary theory.
The second half of this chapter joins that enterprise by using
the model to represent inside and outside money, a free banking
regime, a subtle limit to the scope of Friedman's optimal quantity
of money, a model of international exchange rate indeterminacy,
and some related issues.  The chapter closes by describing work of Krusell and Smith (1998) that extended the domain
of such models to include a time-varying stochastic aggregate state
variable.  As we shall see, this innovation makes the state
of the household's problem include the time $t$ cross-section
distribution of wealth, an immense object.

  Researchers have used calibrated versions of Bewley models to give
quantitative answers to questions including the welfare costs of inflation
(\.Imrohoro\u glu, 1992),
the risk-sharing benefits of unfunded social security systems
(\.Imrohoro\u glu, \.Imrohoro\u glu, and Joines, 1995),
the benefits of insuring unemployed people
(Hansen and \.Imrohoro\u glu, 1992),
and the welfare costs of taxing capital (Aiyagari, 1995).  Also see
Heathcote,  Storesletten, and    Violante (2008), and
Krueger, Perri,  Pistaferri, and   Violante (2010).
See Kaplan and  Violante (2010) for a quantitative study of how much insurance consumers seem to attain
beyond the self-insurance allowed in Bewley models.
Heathcote, Storesletten, and  Violante (2012) combine ideas of Bewley with those of  Constantinides and Duffie (1996) to build
a model of partial insurance.  Heathcote,  Perri, and Violante (2010) present an enlightening account  of recent  movements
in the distributions of   wages, earnings, and consumption across people and across time in the U.S.  \auth{Hansen, Gary D.}
 % \auth{Imrohoroglu, Selahattin} %\auth{Imrohorouglu, Ayse}
 \auth{Aiyagari, Rao}
\auth{Joines, Douglas} \auth{Imrohoro\u glu, Selahattin}
\auth{Imrohoro\u glu, Ay\c se}
\auth{Heathcote, Jonathan}
\auth{Storesletten, Kjetil}
\auth{Violante, Giovanni L.}
\auth{Krueger, Dirk}
\auth{Perri, Fabrizio}
\auth{Pistaferri,  Luigi}
\auth{Kaplan, Greg}
\auth{Constantinides, George M.}%
\auth{Duffie, Darrell}%

\section{A savings problem}

Recall the discrete-state savings problem
described in chapters \use{practical}.  % and \use{selfinsure}.
The household's labor income at time $t$, $s_t$,
evolves according to an $m$-state Markov chain with transition
matrix $\cal P$. Think of initiating the process from the invariant distribution
of $\cal P$ over $\bar s_i$'s.   If the realization of the process at $t$ is
$\bar s_i$, then at time $t$ the household receives labor income
$w \bar s_i$.   Thus, employment opportunities determine the
labor income process.
We shall sometimes assume that $m$ is 2, and that $s_t$
takes the value 0 in an unemployed state and 1 in an
employed state.

We constrain holdings of a single asset to a grid
${\cal A} = [ 0 <\bar a_1 < \bar a_2 < \cdots < \bar a_n]$.
For given values
of ($w,r$) and given initial values ($a_0, s_0$), the household
chooses a policy for
 $\{c_t, a_{t+1}\}_{t=0}^\infty$ to maximize
$$ E_0 \sum_{t=0}^\infty \beta^t u(c_t) , \EQN ob1 $$
subject to
$$\eqalign{ c_t + a_{t+1} &= (1+r) a_t + w s_t \cr
       a_{t+1} &\in {\cal A}\cr }\EQN ob2
$$
where
$\beta \in (0,1)$ is a discount factor; $u(c)$ is a strictly
increasing, strictly concave, twice continuously
differentiable one-period utility function satisfying
the Inada condition $\lim_{c \downarrow 0} u'(c) = +\infty$;
 and $\beta(1+r) < 1$.\NFootnote{The Inada condition makes consumption
nonnegative, and this fact plays a role in justifying the
natural debt limit below.}

%$$ v(a,s) = \max_{a' \in {\cal A}} \{u((1+r)a +ws - a') + \beta
%    E v(k',s') \vert s \} ,$$
%or
The Bellman equation, for each
$i \in [1, \ldots, m]$ and each $h \in [1, \ldots, n]$,
is
$$ v(\bar a_h, \bar s_i) = \max_{a' \in {\cal A}} \{u[(1+r)\bar a_h+w \bar s_i - a']
 + \beta \sum_{j=1}^m {\cal P}(i,j) v(a',\bar s_j) \} ,\EQN ob3$$
where $a'$ is next period's value of asset holdings. Here $v(a,s)$
is the optimal value of the objective function, starting from
asset-employment state ($a,s$).   Note  that the grid ${\cal A}$
incorporates upper and lower limits on the quantity that can be
borrowed (i.e., the amount of the asset that can be issued). The
upper bound on ${\cal A}$ is restrictive.  In some of our prior
theoretical discussions, especially in chapter \use{selfinsure},  it  was important to
dispense with that upper bound.

In chapter \use{practical}, we described how to solve equation \Ep{ob3}
for a value function $v(a,s)$
and an associated policy function
$a' = g (a,s)$ mapping this period's ($a,s$) pair into an optimal
choice of assets to carry into next period.
%\vfil\eject
\index{wealth-employment distributions}
\subsection{Wealth-employment distributions}
Define the unconditional distribution of $(a_t,s_t)$ pairs,
$\lambda_t (a,s) =
{\rm Prob}(a_t = a,s_t=s)$.
The exogenous Markov transition matrix ${\cal P}$ on $s$ and the optimal policy function
$a'=g(a,s)$ induce a law of motion for the distribution $\lambda_t$, namely,
$$\eqalign{ {\rm Prob}(a_{t+1} &= a', s_{t+1} = s') =
  \sum_{a_t} \sum_{s_t} {\rm Prob}(a_{t+1} = a'\vert a_t=a, s_{t} = s)\cr
  &\cdot {\rm Prob}(s_{t+1} = s' \vert s_t = s)
  \cdot {\rm Prob}(a_t =a, s_t = s),\cr}$$
or
$$ \lambda_{t+1} (a',s') = \sum_a \sum_s \lambda_t(a,s) {\rm Prob}
  (s_{t+1} =s' \vert s_t = s) \cdot {\cal I}(a',s,a),$$
where we define the
indicator function ${\cal I} (a',a,s) = 1$ if $a'=g(a,s)$, and $0$
otherwise.\NFootnote{This construction exploits the fact that
 the optimal policy is a deterministic function of
the state, which comes from the concavity of the objective
function and the convexity of the constraint set.}
The indicator function  ${\cal I} (a',a,s) = 1$
identifies the time $t$ states $a,s$ that are sent into
$a'$ at time $t+1$.
The preceding equation can be expressed
as
$$ \lambda_{t+1} (a',s') = \sum_s \sum_{\{a: a' = g(a,s)\}}
  \lambda_t(a,s) {\cal P}(s,s'). \EQN steady $$
A time-invariant probability distribution $\lambda$ that solves equation \Ep{steady}
(i.e., one for which $\lambda_{t+1} = \lambda_t$) is
called a {\it stationary distribution}. \index{distribution!stationary}%
 In chapter \use{timeseries}, we described two ways to compute a stationary distribution for a Markov chain.
  One way is in effect to iterate to
convergence on equation \Ep{steady}.  An alternative is
to create a Markov chain that describes the solution of the
optimum problem, then to compute an invariant distribution from
a left eigenvector associated with a unit eigenvalue
 of the stochastic
matrix (see chapter \use{timeseries}).

To deduce this \idx{Markov chain}, we map the
pair ($a,s$) of vectors into a single state vector $x$
as follows. For $i=1, \ldots ,n$, $h=1, \ldots, m$,
let the $j$th element of $x$ be the {\it pair\/}
($a_i, s_h$), where $j= (i-1) m +h$. Denote $x' = [(\bar a_1, \bar s_1), (\bar a_1, \bar s_2), \ldots, (\bar a_1, \bar s_m),
(\bar a_2, \bar s_1),  \ldots,   (\bar a_2, \bar s_m), \ldots ,(\bar a_n, \bar s_1),
 \ldots, (\bar a_n,\bar s_m)]$.  \hfil\break
 The optimal policy function $a' = g(a,s)$
and the Markov chain $\cal P$ on $s$ induce a Markov
chain for $x$ via the formula
%\ninepoint{
$$\eqalign{ &  {\rm Prob} [(a_{t+1} = a', s_{t+1} = s') \vert (a_t
= a, s_t = s)] \cr & ={\rm Prob}(a_{t+1} =a' \vert a_t=a, s_t =s )
\cdot {\rm Prob}( s_{t+1}=s' \vert s_t = s)  \cr
& = {\cal I} (a',a,s) {\cal P}(s,s'), \cr} $$
%}%endninepoint
where ${\cal I} (a',a,s) = 1$ is defined as above.
%if $k'=g(k,s)$, and $0$ otherwise.
This formula defines an $N \times N$ matrix $P$, where $N = n \cdot m$.
This is the Markov chain on the household's state vector
$x$.\NFootnote{Matlab
programs to be described later in this chapter create the
Markov chain for the joint $(a,s)$ state.}
\mtlb{markov.m}

Suppose that the Markov chain associated with $P$ is asymptotically
stationary and has a unique
invariant distribution $\pi_\infty$. Typically,
all states in the Markov chain will be recurrent, and
the individual  will occasionally revisit each state.  For
long samples, the distribution $\pi_\infty$ tells  the fraction
of time that the household spends in each state.
We can ``unstack'' the state vector $x$ and use
$\pi_\infty$ to deduce the stationary probability measure
$\lambda (\bar a_i, \bar s_h)$ over ($\bar a_i,\bar s_h$) pairs, where
$$ \lambda(\bar a_i,\bar s_h) = {\rm Prob}(a_t = \bar a_i, s_t = \bar s_h)=\pi_\infty(j),$$
and where $\pi_\infty(j)$ is the $j$th component of the vector
$\pi_\infty$, and $j=(i-1)m+h$.

\subsection{Reinterpretation of the distribution $\lambda$}
The solution of the household's optimum savings
problem induces a stationary distribution $\lambda(a,s)$ that tells
the fraction of time that an infinitely lived agent spends
in state $(a,s)$. We want to reinterpret $\lambda(a,s)$.
Thus, let $(a,s)$
index the state of a particular household at a particular time period
$t$, and assume that there is a cross-section of households distributed
over states $(a,s)$. We
start the economy at time $t=0$ with a cross-section
$\lambda(a,s)$ of households that we want  to repeat
over time. The models in this chapter
arrange the initial distribution and other things so that the
cross-section  distribution of agents over individual state variables
$(a,s)$ remains constant over time even though the state of the
individual household is a stochastic process.%
\index{Bewley models}%
\auth{Huggett, Mark}%

In a model of this type, for a given interest rate $r$,  the population mean
$$ E(a)(r) =  \sum_{a,s} \lambda(a,s) g(a,s) $$
has two interpretations. First, it is the average asset level experienced by a single
household, where here the average is {\it across time}.  Second, it is the
average asset level held by the economy as a whole, where here the average is
across households indexed by $(a,s)$ pairs.  The spirit of what we shall call `Bewley models'
is to make  $r$ an equilibrium object that adjusts to set  $E (a)(r)$  equal to a particular value.
 We shall study
several models of this type, where the models differ in how we formulate the value to which $E(a)(r)$ must be equated through
an appropriate adjustment of $r$.%


\subsection{Example 1: a pure credit model}
 Mark Huggett (1993) studied a \idx{pure consumption loans economy}.
\index{pure credit model} Each of a continuum of  households has
access to a centralized loan market in which it can borrow or lend
at a constant net risk-free interest rate of $r$.  Each
household's endowment is governed by the Markov chain $({\cal P},
\bar s)$. The household can either borrow or lend at a constant
risk-free rate. However, total borrowing cannot exceed $\phi >0$,
where $\phi$ is a parameter set by Huggett.
 A household's
setting of next period's level of assets is restricted
to the discrete set ${\cal A} = [\bar a_1, \ldots, \bar a_m]$, where
the lower bound on assets is $\bar a_1 = -\phi$.  Later we'll discuss
alternative ways to set $\phi$, and how it relates to
a natural borrowing limit.\NFootnote{For a related  discussion of
borrowing limits in economies with sequential trading of IOU's, see chapter \use{recurge}.}
For now, we simply  note  the fact that  $\phi$ must be set so that it is feasible for the consumer
to  honor his loans with probability $1$.  Otherwise, it is not coherent to posit that loans are risk-free.


The solution of a typical household's problem is a policy
function $a' = g(a,s)$ that induces a stationary
distribution $\lambda(a,s)$ over states.
Huggett uses the following definition:

\index{equilibrium!stationary}
\medskip\noindent{\sc Definition:} Given a borrowing limit $\phi$, a {\it stationary equilibrium\/}
is an interest rate $r$, a policy function $g(a,s)$, and a
 stationary distribution $\lambda(a,s)$ for which
\medskip
\noindent(a) Given $r$, the policy function $g(a,s)$ solves the household's
optimum problem;
\medskip
\noindent(b) The probability distribution  $\lambda(a,s)$ is the invariant distribution of the Markov chain on $(a,s)$
induced by the Markov chain $({\cal P}, \bar s)$ and the optimal policy $g(a,s)$;
\medskip
\noindent(c) When $\lambda(a,s)$ describes the cross-section of households at each date, the loan market clears
$$ \sum_{a,s} \lambda(a,s) g(a,s) = 0 .$$



\subsection{Equilibrium computation}
 Huggett computed  equilibria by using an iterative  algorithm that adjusted
 $r$ to make $\sum_{a,s} \lambda(a,s) g(a,s) = 0$.
 He fixed an
$r=r_j$ for $j=0$, and for that $r$ solved the household's problem
for a policy function $g_j(a,s)$ and an associated stationary
distribution $\lambda_j(a,s)$. Then he checked to see
whether the loan market clears at $r_j$ by computing
$$ \sum_{a,s} \lambda_j(a,s) g(a,s) = e_j^*.$$
If $e_j^* >0$, Huggett lowered $r_{j+1}$ below $r_j$ and
recomputed excess demand, continuing these iterations until he
found an $r$ at which excess demand for loans is zero.



\subsection{Example 2: a model with capital}
 The next model
was created by Rao Aiyagari (1994).   He used
a version of the savings problem in an economy with
many agents  and interpreted the single asset
as homogeneous physical capital, denoted $k$.
The capital holdings of a household evolve
according to
$$ k_{t+1} = (1-\delta) k_t + x_t $$
where $\delta \in (0,1)$ is a depreciation rate  and
$x_t$ is gross investment.  The household's consumption
is constrained by
$$ c_t + x_t = \tilde r k_t + w s_t,$$
where $\tilde r$ is the rental rate on capital and $w$ is
a competitive wage, to be determined later.  The
preceding two equations can be combined to become
$$ c_t + k_{t+1} = (1+\tilde r - \delta ) k_t + w s_t, $$
which agrees with equation \Ep{ob2} if we take $a_t \equiv k_t$
and $r \equiv \tilde r - \delta $.

There is a large number of households with identical preferences
\Ep{ob1} whose
distribution across $(k,s)$ pairs is given by $\lambda(k,s)$,
and whose average behavior determines $(w,r)$ as follows:
\auth{Aiyagari, Rao}%
Households are identical in their preferences,
the Markov processes governing their employment opportunities,
and the prices that they face. However, they differ in their
histories $s^t_0 = \{s_h\}_{h=0}^t$ of employment opportunities,
 and therefore in the capital that they have
accumulated. Each household has its own history $s^t_0$ as well
as its own  initial capital $k_0$. The
productivity processes are assumed to be independent across
households. The behavior of the collection of these households
determines the wage and interest rate ($w,r$).

  Assume an initial distribution {\it across\/} households
of $\lambda(k,s)$.
 The average level of capital per
household $K$ satisfies
$$ K = \sum_{k,s} \lambda(k,s) g(k,s),$$
where $k' = g(k,s)$.
Assuming that we start from the invariant distribution,
the
average level of employment is
$$ N = \xi_\infty' \bar s,$$
where $\xi_\infty$ is the invariant distribution associated
with $\cal P$ and $\bar s$ is the exogenously
specified vector of individual employment rates. The
average employment rate is exogenous, but
the average level of capital is endogenous.

There is an aggregate production function whose
arguments are the average levels of capital and employment.
The production function determines the rental
rates on capital and labor from the first-order conditions
$$\eqalign{ w  & = \partial F(K,N) / \partial N \cr
      \tilde r   &= \partial F(K,N) / \partial K , \cr} $$
where $F(K,N)= A K^\alpha N^{1-\alpha}$ and
$\alpha \in (0,1)$.
%\NFootnote{We omit $N$ as an explicit argument of $w(\cdot)$ and $r(\cdot)$
%because it is exogenous. Its presence should be understood.}

We now have identified
all of the objects in terms of which a stationary equilibrium
is defined.

\index{equilibrium!stationary}
\medskip\noindent{\sc Definition of Equilibrium:} A {\it stationary equilibrium\/}
is a policy function $g(k,s)$, a probability distribution
$\lambda (k,s)$, and positive real numbers ($K, r,w$) such that
\medskip

% \beginleftbox Note: no original eq. nos. 4, 5, 6, 7 or 8 \endleftbox

\noindent(a) The prices ($w,r$) satisfy
$$\eqalign{ w& = \partial F(K,N) / \partial N \cr
       r &= \partial F(K,N) / \partial K - \delta ; \cr} \EQN ob9 $$
\medskip
\noindent(b) The policy function $g(k,s)$ solves the household's
optimum problem;
\medskip
\noindent(c) The probability distribution $\lambda(k,s)$ is a
stationary distribution associated with $[g(k,s), {\cal P}]$;
that is, it satisfies
$$ \lambda(k',s') = \sum_s \sum_{\{ k: k'=g(k,s) \}} \lambda(k,s)
 {\cal P}(s,s');$$
\medskip
\noindent(d) The cross-section average value of $K$ is implied by the average
of the  households'
decisions $$ K = \sum_{k,s} \lambda(k,s) g(k,s)  .$$



\subsection{Computation of equilibrium}
 Aiyagari computed an equilibrium of the model
by defining a mapping from $K \in {\bbR}$
into $\bbR$, with the property that a fixed point of the mapping is
an equilibrium $K$. Here is
an algorithm for finding a fixed point:
\medskip
\noindent{\bf 1.} For fixed value of $K=K_j$ with $j=0$, compute ($w,r$) from
equation
\Ep{ob9}, then solve the household's optimum problem. Use the optimal
policy $g_j(k,s)$ to deduce an associated stationary distribution
$\lambda_j(k,s)$.
\medskip
\noindent{\bf 2.} Compute the average value of capital associated
with $\lambda_j(k,s)$, namely,
$$ K_j^* = \sum_{k,s} \lambda_j(k,s) g_j(k,s)  .$$

\medskip
\noindent{\bf 3.} For a fixed ``relaxation parameter'' $\xi \in (0,1)$,
compute a new estimate of $K$ from method\NFootnote{By setting $\xi <1$,
the  relaxation method often converges to a fixed point
in cases in which direct iteration (i.e., setting $\xi=0 $)
fails to converge.}
\index{relaxation!parameter}
$$ K_{j+1} = \xi K_j + (1-\xi) K_j^* .$$
\medskip
\noindent{\bf 4.} Iterate on this scheme to convergence.
\index{relaxation!method}

\medskip
Later, we shall display some computed examples of
equilibria of both Huggett's model and Aiyagari's model.
But first we shall analyze some features of both models
more formally.

\section{Unification and further analysis}\label{sec:unif}%
     We can display salient features of several models by using
a graphical apparatus of Aiyagari (1994).  We shall
show relationships  among several
models that have identical household sectors but
make different assumptions about the single asset being traded.

  For convenience, recall the basic savings problem.
 The household's objective is to maximize
$$ E_0 \sum_{t=0}^\infty \beta^t u(c_t)  \EQN rao1;a $$
$$ c_t + a_{t+1} = w s_t + (1+r) a_t  \EQN rao1;b $$
subject to the borrowing constraint
$$ a_{t+1} \geq -  \phi . \EQN rao1;c $$
We now temporarily suppose that $a_{t+1}$ can take any
real value   exceeding $-\phi$.    Thus, we now suppose
that $a_t \in [-\phi, + \infty)$.
We occasionally  find it useful to express the discount factor
$\beta \in (0,1)$ in terms of a discount {\it rate\/}
$\rho$ as $\beta = {1 \over 1+ \rho}$.
In equation \Ep{rao1;b}, we sometimes express $w$ as a given function
$\psi(r)$ of the net interest rate $r$.

\section{The nonstochastic savings problem when $\beta(1+r) < 1$}
It is useful briefly to study the nonstochastic version of the
savings problem when $\beta (1+r) < 1$.   For $\beta(1+r)=1$, we studied
this problem in chapter \use{selfinsure}.
To get the nonstochastic
savings problem, assume that
$s_t$ is permanently fixed at some positive level $s$.     Associated with
the household's maximum  problem is the Lagrangian
%
%$$ \sum_{t=0}^\infty \beta^t u(c_t)  \EQN save1 $$
%subject to
%$$ c_t + a_{t+1} \leq (1+r) a_t + w s  \EQN save2 $$
%and
%$$ a_{t+1} \geq - \phi $$
$$ L =
 \sum_{t=0}^\infty \beta^t \left\{ u(c_t)
 + \theta_t \left[ (1+r) a_t + w s  - c_t -a_{t+1}  \right] \right\} ,
 \EQN save3 $$
where $\{\theta_t \}_{t=0}^\infty$ is a  sequence of
 nonnegative Lagrange multipliers on the budget constraint.
The first-order conditions for this problem are
%$$ \EQNalign{ u'(c_t) &= \lambda_t , \quad  \EQN foc1 \cr
%             \lambda_t & \geq \beta(1+r) \lambda_{t+1}, \ \  \quad
%    = \ {\rm if} \ a_{t+1} > - \phi    \EQN foc2 \cr }   $$
%$$ \bar a = \cases{ - \phi, & if $r < \rho$; \cr
%                    a_0 , & if $r + \rho$ , \cr} $$
$$ u'(c_t)   \geq \beta (1+r) u'(c_{t+1} ),
            \quad =   {\rm if}  \ a_{t+1} > - \phi.    \EQN foc3   $$
When $a_{t+1} > - \phi$, the first-order condition implies
$$u'(c_{t+1}) = {1 \over \beta (1+r) } u'(c_t), \EQN driftc $$
which because  $\beta (1+r) < 1 $ in turn implies that
$u'(c_{t+1}) > u'(c_t)$ and
$c_{t+1} < c_t$.
Consumption is declining during periods when the household
is not borrowing constrained, so  $\{c_t\}_{t=0}^\infty$ is a monotone decreasing
sequence. If  $\{c_t\}_{t=0}^\infty$  is bounded below, either because of an Inada condition
 $\lim_{c \downarrow 0} u'(c) = +\infty$ % $u(\cdot)$ at $0$
or a nonnegativity constraint on $c_t$,
then
$c_t$ will converge as $t \rightarrow +\infty$.   When it converges,
the household
will be ``borrowing constrained''.

 We can compute the steady level of consumption when
the household eventually becomes permanently stuck at the
borrowing constraint.
Set $a_{t+1} = a_t = - \phi$.  This and \Ep{rao1;b} gives
$$ c_t =  \bar c = w s - r \phi.  \EQN climit $$
This is the amount of labor income remaining after paying the
net interest on the debt at the borrowing limit.  The household would like
to shift consumption from tomorrow to today but can't.

   If we solve the budget constraint \Ep{rao1;b} forward, we obtain
the present-value budget constraint
$$  a_0 = (1+r)^{-1} \sum_{t=0}^\infty (1+r)^{-t}(c_t - ws). \EQN budc $$
 Thus, when $\beta (1+r) < 1$, the household's   consumption plan
can be found by solving equations \Ep{budc}, \Ep{climit}, and
\Ep{driftc} for an initial $c_0$ and a date $T$ after which the
debt limit is binding and $c_t$ is constant.

If consumption is required
to be nonnegative,\NFootnote{Consumption must be nonnegative,
for example, if we impose the Inada condition
discussed earlier.}
equation \Ep{climit} implies that
 the  debt limit must satisfy
$$ \phi \leq {w s   \over r}. \EQN debtlim1 $$
We call the right side  the {\it natural debt limit}.  If $\phi
< {w s \over r}$, we say that we have imposed an {\it ad hoc\/} debt limit.

  We have deduced that when $\beta (1+r) < 1$, if a steady-state
level exists, consumption is given by equation \Ep{climit} and
assets by $a_t = - \phi$.

  Now turn  to the case that $\beta (1+r) =1$.  Here equation
\Ep{driftc} implies that $c_{t+1} = c_t$ and the budget constraint
implies $c_t = w s + r a$ and $a_{t+1} = a_t = a_0$.  So when
$\beta (1+r) =1$, {\it any} $a_0$ is a stationary value of $a$.
It is optimal forever to roll over  initial assets.

   In summary, in the deterministic case, the steady-state demand
for assets is $-\phi$ when $(1+r)< \beta^{-1}$ (i.e., when $r < \rho$);
and it equals  $a_0$  when $r = \rho$.
Letting the steady-state level be $\bar a$, we have
$$ \bar a = \cases{ - \phi, & if $r < \rho$; \cr
                    a_0 , & if $r = \rho,$ \cr} $$
where $\beta = (1+\rho)^{-1}$.   When $r=\rho$, we say
that the steady-state asset level $\bar a$ is indeterminate.

\section{Borrowing limits: natural and ad hoc}
  We return to the stochastic case and take up the issue
of debt limits.
  Imposing $c_t \geq 0$ implies the emergence of
what Aiyagari calls a natural debt  limit.  Thus,
imposing $c_t \geq 0$  and solving equation \Ep{rao1;b} forward gives
$$ a_t \geq -{1 \over 1+r} \sum_{j=0}^\infty
 w s_{t+j} (1+r)^{-j}.\EQN constr1$$
Since the right side is a random variable, not known
at $t$, we have to supplement equation \Ep{constr1} to obtain
the borrowing constraint.  One possible approach is to
replace the right side of equation \Ep{constr1} with its conditional
expectation, and to require equation \Ep{constr1}
to hold in expected value. But this expected value formulation
is incompatible with the notion that the
loan is risk free, and
that the household can repay it for sure. We want to impose a restriction
that will guarantee that it is feasible for the household to repay its debt for all
possible sequences of income realizations.
      If we insist that equation \Ep{constr1} hold almost surely
for all $t\geq 0$,
then  we obtain the constraint that emerges by replacing  $s_t$
with ${\min s} \equiv \bar s_1$, which yields
$$ a _t \geq - {\bar s_1 w \over r}.  \EQN constr2 $$
Aiyagari (1994) calls this the natural debt limit.  To accommodate
possibly more stringent debt limits, beyond those dictated by
the notion that it is feasible to repay the debt for sure,
Aiyagari specifies the debt limit as
$$ a_t \geq -\phi, \EQN constr3$$
where
 $$ \phi =  \min \Bigl[b, {\bar s_1 w\over r}\Bigr] , \EQN constr4 $$
and $b>0$ is an arbitrary parameter defining an
``ad hoc'' debt limit.

\subsection{A candidate for a single state variable}
For the special case in which $s$ is i.i.d., Aiyagari
showed how to cast the model in terms of a single state
variable to appear in the household's value function.
To synthesize a single state  variable, note that
the ``disposable resources'' available to be allocated at  $t$
are
$ z_t = w s_t + (1+r) a_t + \phi$.
 Thus, $z_t$ is the sum of
the current endowment, current savings at the beginning of the
period, and the maximal borrowing capacity
$\phi$.  This can be rewritten as
$$z_t = w s_t + (1+r) \hat a_t - r \phi$$
where  $\hat a_t \equiv a_t +\phi$.
In terms of the single state variable $z_t$,
the household's budget set can   be represented recursively as
$$\EQNalign{ c_t + \hat a_{t+1} &  \leq z_t   \EQN rao3;a \cr
            z_{t+1} & = w s_{t+1} + (1+r) \hat a_{t+1}  - r \phi
         \EQN rao3;b \cr}$$
where we must have $\hat a_{t+1} \geq 0$.
The Bellman equation is
$$ v(z_t,s_t) = \max_{\hat a_{t+1} \geq 0} \left\{u(z_t - \hat a_{t+1})
    + \beta E v(z_{t+1}, s_{t+1})  \right\}. \EQN rao4 $$
Here $s_t$ appears in the state vector purely as an
information variable for predicting the
employment component
$s_{t+1}$ of next period's disposable resources $z_{t+1}$,
conditional on the choice of $\hat a_{t+1}$ made this period.
Therefore, it disappears from both the value function and the decision
rule in the i.i.d.\ case.

More generally, with a serially correlated state,
associated with the solution of the Bellman
equation is a policy function
$$ \hat a_{t+1} = A(z_t, s_t). \EQN rao5$$

\subsection{Supermartingale convergence again}
Let's revisit a main issue from chapter \use{selfinsure}, but now
consider the possible case $\beta (1+r) < 1$.
From equation \Ep{rao3;a}, optimal consumption satisfies
$c_t = z_t - A(z_t,s_t)$.
The optimal policy obeys the Euler inequality:
$$ u'(c_t) \geq \beta (1+r) E_t u'(c_{t+1}), \quad = {\rm \ if \ }
     \hat a_{t+1} > 0.  \EQN rao6  $$
We can use equation \Ep{rao6} to deduce significant aspects of the
limiting behavior of mean assets as a function of $r$.
 Following Chamberlain and Wilson (2000) and others,
  to deduce the effect of $r$ on the mean of assets, we analyze
the limiting behavior of consumption implied by the Euler inequality
\Ep{rao6}.
Define
 $$M_t = \beta^t (1+r)^t u'(c_t) \geq 0. $$
Then
$ M_{t+1} - M_t= \beta^t(1+r)^t[\beta (1+r) u'(c_{t+1}) - u'(c_t)]$.
Equation \Ep{rao6}  can be written
$$ E_t (M_{t+1} - M_t) \leq 0, \EQN rao7 $$
which asserts that $M_t$ is a supermartingale.  Because
$M_t$ is  nonnegative, the \idx{supermartingale convergence theorem}
applies.  It
asserts that $M_t$ converges almost surely to a nonnegative
random variable $\bar M$: $M_t \rightarrow_{\rm a.s.} \bar M$.

   It is interesting to consider three cases: (1) $\beta(1+r) > 1$;
(2) $\beta (1+r) < 1$, and (3) $\beta (1+r) =1$.  In case 1,
the fact
that $M_t$ converges implies that $u'(c_t)$ converges to
zero almost surely. % If $u(\cdot)$ is unbounded (has no satiation
%point),
Because $u'(c_t)>0$ and $u''(c_t)<0$,
this fact then implies that $c_t \rightarrow +\infty$ and
that the consumer's asset holdings diverge
to $+\infty$.   Chamberlain and Wilson (2000) show
that such results also characterize the borderline case
(3) (see chapter \use{selfinsure}).
  In case 2, convergence of $M_t$ leaves
open the possibility that $u'(c)$ does not converge almost surely.
To take a simple example of nonconvergence in case 2, consider the case of a nonstochastic endowment.
Under the natural borrowing constraint, the consumer chooses to drive $u'(c) \rightarrow +\infty$ as time passes  and so asymptotically  chooses
to impoverish himself.  The marginal utility $u'(c)$ diverges. %Indeed, when $\beta (1+r) < 1$,
%the average level of assets remains finite, and so does the level
%of consumption.

It is easier to analyze the borderline case
$\beta (1+r)=1$ in the special case that the employment
process is independently and identically distributed, meaning
that the stochastic matrix ${\cal P}$ has identical rows.\NFootnote{See chapter
\use{selfinsure} for a closely related proof.}
  In this case, $s_t$ provides no information
about $z_{t+1}$, and so $s_t$ can be dropped as an argument
of both  $v(\cdot)$ and  $A(\cdot)$.
For the case in which $s_t$ is i.~i.~d.,
Aiyagari (1994) uses the following
argument by contradiction to show that  if $\beta (1+r) =1$, then
$z_t$ diverges to $+\infty$.  Assume that there is some upper
limit $z_{\rm max}$ such that $z_{t+1} \leq z_{\rm max} = w s_{\rm max}
+ (1+r) A(z_{\rm max}) - r \phi$.  Then when $\beta (1+r)=1$,
the strict concavity of the value function, the
\idx{Benveniste-Scheinkman formula}, and equation \Ep{rao6} imply
$$\EQNalign{
v'(z_{\rm max}) &\geq E_t v'\bigl[w s_{t+1} + (1+r)A(z_{\rm max}) - r \phi
\bigr] \cr
&> v'\left[w s_{\rm max} + (1+r) A(z_{\rm max}) - r \phi\right] =
v'(z_{\rm max}), \cr}
$$
which is a contradiction.


% Assume
%that there is some $z_{\rm max}$ such that
%$z_{t+1} \leq A(z_{\rm max}) + (1+r) w s_{\rm max}$   Then
%when $\beta (1+r)=1$,
%the strict concavity of the value function, the
%\idx{Benveniste-Scheinkman formula}, and \Ep{rao6}
%imply
%$v_z(z_{\rm max}) \geq E_t v_z(w s_{t+1} + (1+r)A(z_{\rm max})
%> v_z(w s_{\rm max}) + (1+r) A(z_{\rm max})) = v_z(z_{\rm max})$,
%which is a contradiction.

\section{Average assets as a function of $r$}

In the next several sections, we use versions of a graph
of Aiyagari (1994) to analyze several models.  The graph
plots the average level of assets as a function of $r$.
In the model with capital, the graph is constructed to
incorporate the equilibrium dependence of the wage
$w$ on $r$.  In models without capital, like
Huggett's, the wage is fixed.  We shall focus on situations
where $\beta (1+r) < 1$.  We consider cases where
the optimal decision rule $A(z_t, s_t)$ and the Markov chain
for $s$ induce a Markov chain jointly for assets and $s$ that
has a unique invariant distribution.  For fixed $r$, let $E a(r)$
denote the mean level of assets $a$ and let $ E \hat a(r) = E a(r)+\phi$
be the mean level of assets plus borrowing capacity $\hat a = a+\phi$, where the mean is taken
with respect to the invariant distribution.  Here it
is understood that $E a(r)$ is a function of $\phi$; when
we want to make the dependence explicit we write
$Ea(r;\phi)$.  Also, as we have said, where the single
asset is capital, it is appropriate to
make the wage $w$ a function of $r$. This approach incorporates
the way different values of $r$ affect average capital, the
marginal product of labor, and therefore the wage.

The preceding analysis applying the  supermartingale convergence theorem implies
that as $\beta (1+r)$ goes to 1 from below (i.e., $r$ goes
to $\rho$ from below), $Ea (r)$ diverges to $+ \infty$.  This
feature is reflected in the shape of the $E a(r)$ curve
in Figure \Fg{rao1f}.\NFootnote{As discussed
in Aiyagari (1994), $E a(r)$ need not be a monotonically
increasing function of $r,$ especially because $w$ can be a function of $r$.}

%%%%%%%%
%%$$
%\grafone{rao1.eps,height=2.5in}{{\bf Figure 1.}
%Demand for capital and determination of interest rate.}
%$$
%%%%%%%%%%%%%%

\midfigure{rao1f}
\centerline{\epsfxsize=3truein\epsffile{rao1.eps}}
\caption{Demand for capital and determination of interest rate. The $E a(r)$
curve is constructed for a fixed wage that equals the marginal product of labor
at level of capital $K_1$.  In the nonstochastic version of the model with capital,
the equilibrium interest rate and capital stock are $(\rho, K_0)$, while in the
stochastic version they are $(r, K_1)$.  For a version of the model without
capital in which $w$ is fixed at this same fixed wage, the equilibrium interest
rate in Huggett's pure credit economy occurs at the intersection of the $E a(r)$
curve with the $r$-axis.}
\infiglist{rao1f}
\endfigure

%%%%%%%%%%%%%%%%%
%\topinsert{
%$$\grafone{rao2.eps,height=2.5in}{{\bf Figure 14.1}
%Demand for capital and determination
%of interest rate. The $E a(r)$ curve is
%constructed for a fixed wage that equals
%the marginal product of labor at level of capital
%$K_1$.  In the nonstochastic version of the
%model with capital, the equilibrium interest
%rate and capital stock are $(\rho, K_0)$,
%while in the stochastic version they are
%$(r, K_1)$.  For a version of the model without
%capital in which $w$ is fixed at this same fixed wage,
%the equilibrium interest rate in Huggett's pure credit
%economy occurs at the intersection of the $E a(r)$ curve
%with the $r$ axis.}
%$$
%}\endinsert
%%%%%%%%%%%%%%%%%%
%
%\midfigure{rao2f}
%\centerline{\epsfxsize=3truein\epsffile{rao2.eps}}
%\caption{Demand for capital and determination of interest rate. The $E a(r)$
%curve is constructed for a fixed wage that equals the marginal product of labor
%at level of capital $K_1$.  In the nonstochastic version of the model with capital,
%the equilibrium interest rate and capital stock are $(\rho, K_0)$, while in the
%stochastic version they are $(r, K_1)$.  For a version of the model without
%capital in which $w$ is fixed at this same fixed wage, the equilibrium interest
%rate in Huggett's pure credit economy occurs at the intersection of the $E a(r)$
%curve with the $r$-axis.}
%\infiglist{rao2f}
%\endfigure


\smallskip

  Figure \Fg{rao1f} assumes that the wage $w$ is fixed in drawing the $E a(r)$
curve.  Later, we will discuss how to draw a similar curve, making $w$
adjust as the function of $r$ that is induced by the marginal
productivity conditions for positive values of $K$. For now, we just assume
that $w$ is fixed at the value equal to the marginal product of labor
when $K = K_1$, the equilibrium level of capital in the model.  The
equilibrium interest rate is determined at the intersection of the
$E a(r)$ curve with the marginal productivity of capital curve.
Notice that the equilibrium interest rate $r$ is lower than $\rho$, its
value in the nonstochastic version of the model, and that the equilibrium
value of capital $K_1$ exceeds the equilibrium value $K_0$ (determined by
the marginal productivity of capital at $r=\rho$ in the nonstochastic version
of the model.)

  For a pure credit version of the model like Huggett's, but the same $E a(r)$
curve, the  equilibrium interest rate is determined by the intersection of
the $E a(r)$ curve with the $r$-axis.

\medskip

%%%%%%%%%%%
%\topinsert{
%$$\grafone{rao3.eps,height=2.7in}{{\bf Figure 14.2}
%The effect of a shift in $\phi$ on the $E a(r)$ curve.
%Both $E a(r)$ curves are drawn assuming that the wage
%is fixed.}
%$$
%}\endinsert
%%%%%%%%%%%%%%

\midfigure{rao3f}
\centerline{\epsfxsize=3truein\epsffile{rao3.eps}}
\caption{The effect of a shift in $\phi$ on the $E a(r)$ curve.  Both $E a(r)$
curves are drawn assuming that the wage is fixed.}
\infiglist{rao3f}
\endfigure


For the purpose of comparing some of the models that follow, it is useful to note
the following aspect of the dependence of
$E a(0)$ on $\phi$:

\medskip
\medskip\noindent{\sc Proposition 1:} When $r=0$, the optimal rule
$ \hat a_{t+1} = A(z_t,s_t)$
is independent of $\phi$.  This implies that for $\phi >0$, $E a(0;\phi)
   = Ea(0; 0) - \phi$.

\medskip
\noindent{\sc Proof:}  It is sufficient to note that
when $r=0$, $\phi$ disappears  from the right side of equation
\Ep{rao3;b} (the consumer's budget constraint).
  Therefore, the optimal rule
$\hat a_{t+1} = A(z_t,s_t)$ does not depend on $\phi$ when
$r=0$.  More explicitly,  when $r=0$, add $\phi$ to both sides
of the household's budget constraint to get
$$(a_{t+1} + \phi)  + c_t \leq (a_t+ \phi) + w s_t .$$
If the household's problem with $\phi=0$ is solved by the decision rule
$a_{t+1} = g(a_t, z_t)$, then the household's problem
with $\phi >0$ is solved with the same decision rule
evaluated at $a_{t+1} +\phi = g(a_t +\phi, z_t)$. \qed

\smallskip

Thus, it follows that at $r=0$, an increase in $\phi$ displaces the
$Ea(r)$ curve to the left by the same amount. See Figure \Fg{rao3f}.
We shall use this
result to analyze several models.

In the following sections, we use a version of Figure \Fg{rao1f} to compute equilibria
of various models.  For models without capital, the figure is drawn
assuming that the wage is fixed.  Typically, the $E a(r)$ curve will have the
same shape as Figure \Fg{rao1f}.  In Huggett's model, the equilibrium interest rate
is determined by the intersection of the $Ea(r)$ curve with the $r$-axis,
reflecting that the asset (pure consumption loans) is available in zero
net supply.  In some models with money, the availability of fiat currency as  a perfect substitute
for consumption loans  creates a positive net supply.

%%%%%%%%%%%%
%\topinsert{
%$$\grafone{bewley00w1.eps,height=2.5in}{{\bf Figure 14.3}
%Two $Ea(r)$ curves, one with $b=6$, the other with $b=3$,
%with $w$ fixed at $w=1$. Notice that at $r-0$, the difference between
%the two curves is $3$, the difference in the $b$'s.} $$
%}\endinsert
%%%%%%%%%%%%%

\midfigure{bewley00w1f}
\centerline{\epsfxsize=3truein\epsffile{bewley00w1.eps}}
\caption{Two $Ea(r)$ curves, one with $b=6$, the other with $b=3$, with $w$ fixed
at $w=1$. Notice that at $r=0$, the difference between the two curves is $3$, the
difference in the $b$'s.}
\infiglist{bewley00w1f}
\endfigure

%\bye  OK to here
\section{Computed examples}

  We used some Matlab programs that solve
 discrete-state dynamic programming problems to compute some
examples.\NFootnote{The Matlab programs
used to compute the  $Ea(r)$ functions
are {\tt bewley99.m}, {\tt bewley99v2.m}, {\tt aiyagari2.m},
{\tt bewleyplot.m}, and {\tt bewleyplot2.m}.  The program
{\tt markovapprox.m} implements Tauchen's method for
approximating a continuous autoregressive process with a Markov
chain. A program {\tt markov.m} simulates
a Markov chain.  The programs can be downloaded from
$<$www.tomsargent.com/source\_code/mitbook.zip$>$.}
%$<$https://files.nyu.edu/ts43/public/books.html$>$.}
We discretized the space  of assets from $-\phi$ to
a parameter $a_{\max} =16$ with step size $.2$.
\mtlb{bewley99v2.m}
\mtlb{bewley99.m}
\mtlb{aiyagari2.m}
\mtlb{bewleyplot.m}
\mtlb{bewleyplot2.m}
\mtlb{markovapprox.m}

 The utility function is $u(c) =(1- \mu)^{-1} c^{1-\mu}$, with $\mu =3$.
We set $\beta=.96$. We used two specifications of the Markov
process for $s$.  First, we used Tauchen's (1986) method to
get a discrete-state Markov chain to approximate a first-order
autoregressive process
$$ \log s_t =  \rho \log s_{t-1} + u_t, $$
where $u_t$ is a sequence of  i.i.d.\ Gaussian  random variables.
We set $\rho =.2$ and the standard deviation of $u_t$ equal
to $.4 \sqrt{1-\rho^2}$.  We used Tauchen's method with
$N=7$ being the number of points in the grid for $s$.

 For the second specification, we assumed that $s$ is i.i.d.\ with
mean $1.0903$. For this case, we compared two settings for
the variance: .22 and .68.
Figures \Fg{bewley00w1f} %14.3
 and
 \Fg{newfig5f} % 14.5
 plot the $Ea(r)$ curves for these various
specifications.
%For future reference, they also plot a
%downward-sloping marginal-productivity-of-capital curve,
%from a Cobb-Douglas specification of a production function.
  Figure \Fg{newfig5f} %14.3
  plots $E a(r)$ for the first case of serially
correlated $s$.  The two $E [a(r)]$ curves correspond to
two distinct settings of the ad hoc debt constraint.
One is for $b=3$, the other for $b=6$.
Figure \Fg{newfig4f} %14.4
plots the invariant distribution of asset holdings
for the case in which $b=3$ and  the interest rate is determined at the
intersection of the $Ea(r)$ curve and the $r$-axis.
%curve $F_k - \delta$.    This curve is drawn assuming
%a production function in Aiyagari's model   $K^\alpha N^{1-\alpha}$
%with $\alpha =.36$ and $\delta = .08$.

  Figure \Fg{newfig5f} % 14.5
   summarizes a precautionary savings experiment for
the i.i.d.\ specification of $s$.  Two $E a(r)$ curves are
plotted. For each, we set the ad hoc debt limit $b=0$.  The
$E a(r)$ curve further to the right is the one for the higher
variance of the endowment shock $s$.  Thus, a larger variance in
the random shock causes increased savings.

%%%%%%%%%%%%%%%
%$$\grafone{newfig4.eps,height=2.25in}{{\bf Figure 14.4}
%The invariant distribution of capital when $b=3$.}$$
%%%%%%%%%%%



%\vfill\eject


%%%%%%%%%%%%
%$$\grafone{newfig5.eps,height=2.25in}{{\bf Figure 14.5}
%Two $E a(r)$ curves
%when $b=0$ and the endowment shock
%$s$ is i.i.d.\ but with different variances;
%the curve with circles
%belongs to the economy with the higher variance.}$$
%%%%%%%%%%%%%

\midfigure{newfig5f}
\centerline{\epsfxsize=3truein\epsffile{newfig5.eps}}
\caption{Two $E a(r)$ curves when $b=0$ and the endowment shock $s$ is i.i.d.\
but with different variances; the curve with circles belongs to the economy with
the higher variance.}
\infiglist{newfig5f}
\endfigure

\midfigure{newfig4f}
\centerline{\epsfxsize=3truein\epsffile{newfig4.eps}}
\caption{The invariant distribution of capital when $b=3$.}
\infiglist{newfig4f}
\endfigure

 Keep these graphs in mind as we turn to analyze
some particular models in more detail.


\section{Several Bewley models}

   We consider several models in which  a continuum of households faces the
same savings problem.  Their behavior generates the asset demand function $E a(r;\phi)$.
The models share the same family  of $E a(r;\phi)$ curves as functions of $\phi$, but differ in their
settings of $\phi$ and in their interpretations of the supply of the asset.
The models are (1) Aiyagari's (1994, 1995) model in which the risk-free asset
is either physical capital or private IOUs, with physical capital being the
net supply of the asset; (2) Huggett's model (1993), where the asset is private
IOUs, available in zero net supply; (3) Bewley's model of fiat currency;
(4) modifications of Bewley's model to permit an inflation tax;
and (5) modifications of Bewley's model to pay interest on currency,
either explicitly or implicitly through deflation.



\subsection{Optimal stationary allocation}

Because there is no aggregate
risk and the aggregate endowment is constant,
 a stationary optimal  allocation
would have consumption constant over time for each
household. Each household's  consumption plan would have constant
consumption over time.
The implicit risk-free interest rate associated with such an
allocation would be  $r = \rho$, where recall that $\beta = (1+\rho)^{-1}$.  In the version
of the model with capital, the stationary aggregate capital stock
solves
$$ F_K(K,N) - \delta =   \rho. \EQN casskoop  $$
Equation \Ep{casskoop} restricts the stationary
optimal capital stock in the nonstochastic optimal
growth model of Cass (1965) and Koopmans (1965).
%(See equation \Ep{growth0} in the chapter on economic growth.)
  The stationary
level of capital is  $K_0$ in Figure \Fg{rao1f}, depicted as the ordinate
of the intersection
of the marginal productivity net of depreciation  curve with
a horizontal line  $r = \rho$.   As we saw before,
the horizontal line at $r=\rho$ acts as a ``long-run'' demand
curve for savings for a nonstochastic version of the savings
problem.
   The stationary optimal allocation matches the one produced
by a nonstochastic growth model.
We shall use the risk-free interest rate $r=\rho$ as a benchmark against
which to compare some alternative incomplete market allocations.
    Aiyagari's (1994) model  replaces the horizontal
line  $r=\rho$ with an upward-sloping curve $E a(r)$, causing the stationary
equilibrium interest rate to fall  and the  capital stock to rise
relative to the  savings model with a risk-free endowment sequence.


\section{A model with capital and private IOUs}

  Figure \Fg{rao1f} can be used to depict the equilibrium of Aiyagari's model
described above.  The single asset is capital.
    There is an aggregate production function
$Y = F(K,N)$, and $w = F_N(K,N)$, $r + \delta = F_K(K,N)$.
  We can  invert
the marginal  condition for capital to deduce
a downward-sloping curve
$K=K(r)$. This is drawn as the curve labeled
$F_K - \delta$ in Figure \Fg{rao1f}.  We can  use  the marginal
productivity conditions  to deduce a
factor price frontier $w=\psi(r)$.  For fixed  $r$,
we use $w=\psi(r)$  as the wage in the
savings problem and then deduce $E a(r)$. We want
the equilibrium $r$ to satisfy
$$ E a(r) = K(r). \EQN capital $$
The equilibrium interest rate occurs at the intersection
of $E a(r)$ with the $F_K - \delta$ curve.  See Figure \Fg{rao1f}.\NFootnote{Recall
that Figure \Fg{rao1f} was drawn for a fixed wage $w$, fixed at the value
equal to the marginal product of labor when $K=K_1$. Thus, the
new version of Figure \Fg{rao1f} that incorporates $w=\psi(r)$ has a new
curve $E a(r)$ that intersects the $F_K - \delta$ curve at the same
point $(r_1,K_1)$ as the old curve $E a(r)$  with the fixed wage. Further, the
new $E a(r)$ curve would not be defined for negative values of
$K$.}

It follows from the shape of the curves that the equilibrium
capital stock $K_1$ exceeds $K_0$, the capital stock required
at the given level of total labor to make the interest rate equal
$\rho$.  There is capital overaccumulation in
the stochastic version of the  model.


\section{Private IOUs only}
It is easy to compute the equilibrium of
   Mark Huggett's (1993) model with Figure \Fg{rao1f}.
Recall that in Huggett's model the one asset   consists of risk-free
loans issued by
other  households.  There are no ``outside'' assets.  This fits
the basic model, with $a_t$ being the quantity of loans owed to
the individual at the beginning of $t$.  The equilibrium
condition is
$$   E a(r,\phi) = 0,  \EQN clearing $$
which is depicted as the intersection of the $E a(r)$ curve in
Figure \Fg{rao1f} with  the $r$-axis.  There is a family of such
curves, one for each value of the ``ad hoc'' debt limit.  Relaxing
the ad hoc debt limit (by driving $b \rightarrow + \infty$) sends
the equilibrium   interest rate upward toward the intersection
of the furthest to the left  $E a(r)$ curve, the one that is
associated with the
natural debt limit,  with the $r$-axis.


\subsection{Limitation of what credit can achieve}

  The equilibrium condition \Ep{clearing} and
$\lim_{r \nearrow \rho} E a(r) = + \infty$
imply that the equilibrium value of $r$ is less than $\rho$, for
all values of the debt limit respecting the natural debt limit.
This outcome supports the following conclusion:

\medskip\noindent{\sc Proposition 2:} (Suboptimality of equilibrium with credit)
   The equilibrium interest rate associated with the natural debt limit
is the highest one that Huggett's model can support.  This
interest rate falls short of $\rho$, the interest rate that
would prevail in a complete markets world.\NFootnote{Huggett used
the model to study how tightening the ad hoc debt limit
parameter $b$ would reduce the risk-free rate
far enough  below $\rho$ to explain the ``risk-free rate'' puzzle.}

\subsection{Proximity of $r$ to  $\rho$}

  Notice how in Figure \Fg{bewley00w1f} the equilibrium interest rate $r$ gets closer
to $\rho$ as the borrowing constraint is relaxed.  How close it can get
under the natural borrowing limit depends on several key parameters
of the model: (1) the discount factor $\beta$,
(2) the curvature of $u(\cdot)$, (3) the persistence of the endowment
process, and (4) the volatility of the innovations to the endowment
process.  When he selected a plausible $\beta$ and $u(\cdot)$ and then
calibrated the persistence and volatility of the
endowment process to U.S. panel data on workers' earnings,
  Huggett (1993) found that under the natural borrowing limit,
$r$  is quite close to $\rho$ and that the household can achieve
substantial self-insurance.\NFootnote{This result depends sensitively
on how one specifies the left tail of the endowment distribution.  Notice
that if the minimum endowment $\bar s_1$ is set to zero, then the
natural borrowing limit is zero. However, Huggett's calibration permits
positive borrowing under the natural borrowing limit.}
We shall encounter an echo of this finding when we review
Krusell and Smith's (1998) finding that under their calibration
of idiosyncratic risk, a real business cycle model with complete markets does
a good job of approximating the prices and the aggregate allocation
of a  model with identical preferences and technology but  in which only a single asset, physical capital, can be traded.

\subsection{Inside money or free banking interpretation}

  Huggett's can be viewed as  a model of  pure ``inside money,'' or
of circulating private IOUs.  Every person is a ``banker'' in
this setting, being entitled to issue ``notes'' or evidences of indebtedness, subject
to the debt limit \Ep{constr3}.    A household has
issued  more IOU notes of its own  than it holds of those issued by others     whenever $a_{t+1} < 0$.



  There are several ways to think about
the ``clearing'' of notes imposed by equation \Ep{clearing}.
Here is one:  In period $t$, trading occurs in subperiods as
follows. First, households realize their $s_t$.  Second, some
households  choose to set $a_{t+1} < a_t \leq 0$ by  issuing new
IOUs in the amount $-a_{t+1} +a_t$.
    Other households with $a_t < 0$ may decide
to set $a_{t+1} \geq  0$, meaning that they want to ``redeem''  their
outstanding notes and possibly acquire notes issued by others.
Third, households go to the market and
exchange goods  for notes.  Fourth, notes are ``cleared''
or ``netted out'' in a centralized clearinghouse:
  positive holdings of  notes
issued by  others are used to retire possibly negative initial holdings of
one's own notes.   If a person holds positive
amounts of notes issued by others, some of these are used to retire
any of his own notes outstanding. This clearing operation leaves each person
with a particular
 $a_{t+1}$ to carry into the next period, with no owner of IOUs
also
being in the position of having some notes outstanding.

  There are other ways to interpret the trading arrangement in terms
of circulating notes that implement multilateral long-term lending
among corresponding ``banks'': notes issued by individual A and
owned by B are ``honored'' or redeemed  by individual C by being
exchanged for goods.\NFootnote{It is possible to tell versions of
this story in which notes issued by one individual or group of
individuals  are ``extinguished'' by another.}  In a different
setting, Kocherlakota (1996b) and Kocherlakota and Wallace (1998)
describe such trading mechanisms.


Under the natural borrowing limit, we might think of this pure consumption
loans or inside money model as a model of free banking.  In
the model,  households' ability to issue IOUs is restrained only by the
requirement that all loans be  risk-free and of one period in duration.
Later, we'll use the equilibrium allocation of this free banking
model as a benchmark against which to judge the celebrated
Friedman rule in a model with outside money and a severe borrowing
limit.

We now tighten the borrowing limit enough to make
room for  some ``outside money.''

\subsection{Bewley's basic model of fiat money}
   This version of the  model is set up to generate a demand for fiat money,
an inconvertible currency  supplied in a fixed nominal
amount by  an
entity outside the model called the government.  Individuals can hold currency,
but not issue it.  To map the individual's problem into
problem \Ep{rao1},
we let $m_{t+1} / p = a_{t+1} , b= \phi =0$, where $m_{t+1}$
is the individual's holding of currency from $t$ to $t+1$,
and $p$ is a constant price level. With a constant price
level, $r=0$.   With $b=\phi=0$,
$\hat a_t = a_t$. Currency is the only
asset that can   be held.   The fixed supply of currency is $M$.
The condition for a stationary  equilibrium
is
$$ E a(0) = { M   \over p}. \EQN qtheory  $$
This equation is to be solved for $p$.  The equation states
a version of the  quantity theory of money.

  Since $r=0$, we need {\it some\/} ad hoc borrowing constraint
(i.e., $b < \infty$)
to make this model have a stationary equilibrium.
    If we relax the borrowing constraint from $b=0$ to permit
some borrowing (letting $b > 0$), the $E a(r) $ curve
shifts to the left, causing  $E a(0)$ to fall
and  the stationary price level to rise.

Let $ \bar m = E a(0, \phi=0)$ be the solution of equation
\Ep{qtheory} when $\phi =0$.    Proposition
1 tells how to construct a set of stationary equilibria,
indexed  by $\phi \in (0, \bar m)$, which
have identical allocations but different price levels.
Given an initial stationary equilibrium with $\phi =0$ and a price
level satisfying equation \Ep{qtheory}, we construct the  equilibrium
for $\phi \in (0, \bar m)$ by
setting  $\hat a_t $ for the new equilibrium equal to $\hat a_t$ for
the old equilibrium for each person for each period.

\index{money!outside}
\index{money!inside}
       This set of equilibria
highlights how  expanding the amount
of ``inside money,'' by substituting for ``outside'' money,
causes the value of outside money (currency) to fall.
The construction also indicates that if we set $\phi > \bar m$,
then there exists no stationary monetary equilibrium with
a finite positive price level.
For $\phi > \bar m$,  $E a(0) < 0$, indicating
a force for the interest rate to rise and for
private IOUs to dominate currency in rate of return and to
drive it out of the model.
 This outcome leads us to consider proposals to get currency
back into the model by paying
interest on it.   Before we do, let's consider some situations
more often observed, where a government raises revenues
by an inflation tax.




\section{A model of seigniorage}

The household side of the model is described in the previous section;
we continue to summarize this in a stationary demand function $Ea(r)$.
We suppose that $\phi = 0$, so  individuals cannot borrow.
But now the government augments the nominal
supply of currency over time to finance a fixed aggregate
flow of real purchases  $G$.
The government budget constraint at $t\geq 0$ is
$$ M_{t+1} = M_t + p_t G, \EQN govbud0 $$
which for $t \geq 1$ can be expressed
$$ {M_{t+1} \over p_t} = {M_t \over p_{t-1}} \left({p_{t-1}
     \over p_t} \right) + G. $$
We shall seek a stationary  equilibrium with ${p_{t-1}\over p_t} = (1+r)$
for $t \geq 1$ and ${M_{t+1} \over p_t } = \bar a $ for
$t\geq 0$.  These guesses make the previous equation
become
$$ \bar a = {G \over -r}. \EQN govbud1 $$
For $G > 0$, this is a rectangular hyperbola in the southeast
quadrant.  A stationary equilibrium value of  $r$ is determined
at an intersection of this curve with $E a(r)$ (see Figure \Fg{rao4f}).
 Evidently,
when $G > 0$, an equilibrium net interest rate
 $r <0$; $-r$ can be regarded as an inflation tax.
Notice that if there is one equilibrium net interest rate, there is typically
more than one.  This is a consequence of the Laffer curve present
in this model.\NFootnote{A Laffer curve exists when government revenues from a tax  are not a monotonic function
of a tax rate.}
\index{Laffer curve}%
 Typically if a
stationary equilibrium exists,
there are at least two stationary inflation rates that finance
the government budget. This conclusion follows from the fact that
both curves in Figure \Fg{rao4f} have positive slopes.
%\vfil\eject

%%%%%%%%%%%%%
%$$\grafone{rao4.eps,height=2.7in}{{\bf Figure 14.6}
%Two stationary equilibrium  rates of return on currency
%that finance the constant government deficit $G$.}
%$$
%%%%%%%%%%%%

\midfigure{rao4f}
\centerline{\epsfxsize=3truein\epsffile{rao4.eps}}
\caption{Two stationary equilibrium rates of return on currency that finance the
constant government deficit $G$.}
\infiglist{rao4f}
\endfigure

After $r$ is determined, the initial price level can be determined
by the time $0$ version of the government budget constraint \Ep{govbud0},
namely,
$$ \bar a = M_0/p_0 + G.   $$
This is the version of the quantity theory of money that prevails in this model.
An increase in $M_0$ increases $p_0$ and all subsequent prices proportionately.

  Since there are generally multiple stationary equilibrium inflation
rates, which one should we select?  We recommend choosing
the one with the highest rate of return to currency, that is, the
lowest inflation tax.  This selection gives
``classical'' comparative statics: increasing $G$ causes $r$ to fall.
In  distinct but related settings, Marcet and Sargent (1989) and
Bruno and Fischer (1990) give learning procedures that select
the same equilibrium we have recommended.  Marimon and
Sunder (1993) describe experiments with human subjects
that they interpret as supporting this selection.

  Note the effects of alterations in the debt limit $\phi$ on the
inflation rate.  Raising $\phi$ causes the $E a(r)$ curve to
shift to the left, and {\it  lowers\/} $r$.  It is even possible
for such an increase in $\phi$ to cause all stationary equilibria
to vanish.  This experiment indicates why
governments intent on raising seigniorage might want to restrict
private borrowing.  See Bryant and Wallace (1984) for an
extensive theoretical elaboration of this and related points.
See Sargent and Velde (1995) for a practical example from the
French Revolution.
\auth{Kareken, John}
\auth{Wallace, Neil}
\section{Exchange rate indeterminacy}

We can modify the preceding model to display a version of Kareken
and Wallace's (1980) %and Helpman's (1981)
%\auth{Helpman, Elhanan}
 theory of exchange rate indeterminacy.
\index{exchange rate!indeterminacy}
Consider a model consisting of two countries, each of which  is
a Bewley economy with stationary money demand function
$E a_i(r)$ in country $i$. The same single  consumption
good is available in each country.  Residents of both countries
are free to hold the currency of either country.  Households
of either country are indifferent between the two currencies as long
as their rates of return are equal.  Let $p_{i,t}$ be the price
level in country $i$, and let $p_{1,t} = e_t p_{2,t}$ define
the time $t$ exchange rate $e_t$.  The gross return on
currency $i$ between $t-1$ and $t$ is $(1+r) = \left({p_{i,t-1}
\over p_{i,t}}\right)$ for $i=1,2$.
Equality of rates of return implies $e_t = e_{t-1}$ for all
$t$ and therefore
$ p_{1,t} = e p_{2,t}$ for all $t$, where $e$ is a {\it constant}
exchange rate to be determined.

Each of the two countries finances a fixed
expenditure level $G_i$ by printing its own currency.
Let $\bar a_i$ be the stationary level of real balances in
country $i$'s currency.
Stationary versions  of the two  countries' budget constraints are
$$\EQNalign{ \bar a_1 & = \bar a_1 (1+r) + G_1 \EQN govbud1 \cr
             \bar a_2 & = \bar a_2 (1+r) + G_2 \EQN govbud2 \cr} $$
Sum these to get
 $$  \bar a_1 + \bar a_2 ={ \left( G_1 + G_2  \right) \over - r} .$$
Setting this curve against $E a_1(r) + E a_2(r)$ determines
a stationary equilibrium rate of return $r$.
To determine the initial price level  and exchange rate, we use the
time $0$ budget constraints of the two governments.  The time $0$
budget constraint for country $i$ is
$$ {M_{i,1} \over p_{i,0}}  = {M_{i,0} \over p_{i,0} } + G_i $$
or
$$ \bar a_i = {M_{i,0} \over p_{i,0}} + G_i.  \EQN wqtheory $$
Add these
and use $p_{1,0} = e p_{2,0}$ to get
$$ (\bar a_1 + \bar a_2) - (G_1 + G_2)  =
 {M_{1,0} + e M_{2,0} \over p_{1,0}}. $$
This is one equation in two variables $(e, p_{1,0})$.
If there is a solution for some $e \in (0, +\infty)$, then
there is a solution for any other $e \in (0, + \infty)$.
In this sense, the equilibrium exchange rate is indeterminate.

Equation \Ep{wqtheory} is a quantity theory of money stated in
terms of the initial ``world  money supply'' $M_{1,0} + e M_{2,0}$.




\section{Interest on currency}
    Bewley (1980, 1983)  studied whether
Friedman's recommendation to pay interest on currency could
improve outcomes in a stationary equilibrium, and possibly even
support an optimal allocation.  He found  that when
$\beta < 1$, Friedman's
rule could improve things but could
 not implement an optimal allocation, for reasons
we now describe.


As in the earlier fiat money model,
there is one asset,   fiat currency, issued by a government.
Households cannot borrow ($b=0$).
The consumer's budget constraint is
$$ m_{t+1} + p_t c_t \leq (1+\tilde r) m_t + p_t w s_t - \tau p_t $$
where $m_{t+1} \geq 0$ is currency carried over from
$t$ to $t+1$,  $p_t$ is the price level at $t$, $\tilde r$ is
nominal interest  on currency paid by the government, and
$\tau $ is a real lump-sum tax.   This tax is used
to finance the interest payments on currency.
The government's budget constraint   at $t$ is
$$ M_{t+1} = M_t +  \tilde r M_t  - \tau p_t , $$
where $M_t$ is the nominal stock of currency per person
at the beginning of $t$.

There are two versions of  this model: one
where the government pays explicit interest while keeping
the nominal stock of currency fixed, another where the
government pays no explicit interest but varies the stock of
currency to pay interest through deflation.

For each setting, we can  show that
paying interest on currency, where currency holdings
continue to obey $m_t \geq 0$, can be viewed
as a device for weakening the impact of this
nonnegativity constraint.  We establish this point for each
setting
by showing that the household's problem is isomorphic
with Aiyagari's problem as expressed in  \Ep{rao1}, \Ep{constr3},
and \Ep{constr4}.

\subsection{Explicit interest}
In the first setting,
the government leaves the money supply fixed, setting $M_{t+1} = M_t
\ \forall t$,
 and undertakes to support a constant price
level.
These settings make the government budget constraint imply
$$ \tau = \tilde r M / p.$$
Substituting this  into the household's budget constraint and
rearranging
gives
$$ {m_{t+1} \over p} + c_t  \leq {m_t  \over p} (1+ \tilde r)
  + w s_t  -\tilde r {M \over p}$$
% $$ \left( {m_{t+1} \over p} - {M \over p} \right) + c_t
%   \leq  \left({m_t\over p} - {M \over p} \right) (1+\tilde r) + w s_t, $$
where the choice of currency is subject to $m_{t+1} \geq 0$.
With appropriate transformations of variables,
this matches Aiyagari's setup of expressions \Ep{rao1}, \Ep{constr3},
and \Ep{constr4}.  In particular,  take
 $r = \tilde r$, $\phi ={M \over p}, {m_{t+1} \over p} =
 \hat a_{t+1}  \geq 0$.
With these choices, the solution of the savings problem  of a household
living in an economy with aggregate real balances of ${M \over p}$ and
with nominal interest $\tilde r$ on  currency can be read from the
solution  of the savings problem with the real interest rate $\tilde r$
and a borrowing constraint parameter  $\phi \equiv {M  \over p}$.
Let the solution of this problem be given by the
policy function $a_{t+1} = g(a,s;r, \phi)$.  Because we have set
${m_{t+1}\over p} = \hat a_{t+1} \equiv a_{t+1} + {M \over p}$,
the condition that the supply of real balances equals the demand
    $ E {m_{t+1}\over p}  = {M \over p}$ is equivalent with
$E \hat a(r) = \phi$.  Note that because $a_t = \hat a_t - \phi$, the
equilibrium can also be expressed as $E a(r) = 0$, where as usual
$E a(r)$ is the  average of $a$ computed with respect to the
invariant distribution $\lambda(a,s)$.


   The preceding argument shows that
an equilibrium  of the money economy with $m_{t+1} \geq 0$,
equilibrium real balances ${M \over p}$, and explicit
interest on currency $r$ therefore
is isomorphic to a pure credit economy
with borrowing constraint $\phi = {M \over p}$.
We formalize this conclusion in the following proposition:

\medskip
\medskip\noindent{\sc Proposition 3:}  A stationary equilibrium with  interest
on currency financed by lump-sum taxation
 has the same allocation and interest rate
as an equilibrium of Huggett's free banking model
for debt limit $\phi$ equaling the equilibrium real balances
from the monetary economy.
\medskip
\auth{Prescott, Edward C.}
\auth{Sims, Christopher A.}
\auth{Alvarez, Fernando}
\auth{Fitzgerald, T.}
\auth{Diaz-Gim\' enez, J.}
   To compute an equilibrium with interest on currency, we use
a ``back-solving'' method.\NFootnote{See Sims (1989) and Diaz-Gim\' enez,
Prescott, Fitgerald, and Alvarez (1992) for an explanation and
application of back-solving.}
  Thus, even though the spirit of the model is that the government
names $\tilde r =r$ and commits itself to set the lump-sum tax
needed to finance interest payments on whatever ${M \over p}$
emerges, we can compute the equilibrium by naming ${M \over p}$
{\it first\/}, then finding an $r$ that makes things work.  In
particular, we use the following steps:\index{back-solving}%
\medskip
\noindent{\bf 1.} Set $\phi$ to satisfy $0 \leq \phi \leq {w s_1 \over r}$.
(We will elaborate on the upper bound in the next section.)
  Compute real balances and therefore $p$ by
solving  ${M \over p} = \phi$.
\medskip
\noindent{\bf 2.} Find $r$ from  $ E \hat a(r) = {M \over p} $ or $E a(r) = 0$.
\medskip
\noindent{\bf 3.} Compute the equilibrium tax rate from
the government budget constraint $\tau = r  {M\over p}  $.
\medskip
\noindent
This construction finds a constant tax
that satisfies  the government budget
constraint and that supports  a level  of
real balances in the interval $ 0 \leq    {M \over p} \leq {w s_1
\over r}$.  Evidently,  the largest level of real balances that
can be supported in equilibrium
is the one associated with the natural debt limit.    The levels
of interest rates that are associated with monetary equilibria
are in the range $0 \leq r \leq r_{FB}$,   where
$E a(r_{FB}) =0$ and $r_{FB}$ is the equilibrium  interest
rate in the pure credit economy (i.e., Huggett's model)
under the natural debt limit.

\subsection{The upper bound on ${M \over p}$}
  To interpret the upper  bound on attainable ${M \over p}$, note that
the government's budget constraint and the budget constraint
of a  household with zero real balances  imply
that $ \tau  = r  {M \over p} \leq  w s $ for all realizations
of $s$.    Assume that the stationary distribution
of real balances has a positive fraction
of agents with real balances arbitrarily close to zero.   Let the
distribution of employment shocks $s$  be such that a positive
fraction of these low-wealth consumers receive income $w s_1$ at any
time.   Then, for it to be feasible for the lowest wealth
consumers to pay their lump-sum taxes, we must
have $\tau \equiv {r M \over p} \leq w s_1$ or
 $ {M \over p} \leq { w s_1 \over r} $.
\medskip


In Figure \Fg{rao1f}, the equilibrium real interest
rate $r$ can be read from the intersection of the $E  a(r)$ curve and
the $r$-axis.  Think of a graph with two $E a(r)$ curves, one with the
natural debt limit $\phi = {s_1 w \over r}$, the other one with an
ad hoc debt limit $\phi = \min[b, {s_1 w \over r}]$ shifted to the
right.  The highest interest rate that can be supported by an
interest on currency policy is evidently determined by the point where
the $E a(r)$ curve for the natural debt limit passes through the $r$-axis.
This is higher than the equilibrium interest rate associated with any of
the ad hoc debt limits, but must be below $\rho$.  Note that $\rho$ is
the interest rate associated with the  optimal quantity of money.
Thus, we have Aiyagari's (1994) graphical version of Bewley's (1983) result
that the optimal quantity of money (Friedman's rule) cannot be implemented
in this setting.
\index{Friedman rule!Bewley's model}
\medskip
We summarize this discussion with a proposition about free banking and
Friedman's rule:\index{Friedman rule!and free banking}

\medskip
\medskip\noindent{\sc Proposition 4:}
\ \   The highest interest rate that can be supported by paying interest
on currency equals that associated with the pure credit (i.e., the pure inside money)
model with the natural debt limit.

\medskip

 If $\rho >0$, Friedman's rule---to pay real interest on
currency at the rate $\rho$---cannot be implemented in this
model.   The most that can be achieved by paying interest
on currency   is to
eradicate the restriction  that prevents households from
issuing currency in competition with the government and
to implement the free banking outcome.

\auth{Levine, David K.}    \auth{Zame, William}
\subsection{A very special case}
  Levine and Zame (2002)
 have studied a special limiting case of the preceding model
in which  the free banking equilibrium, which we have seen is
equivalent to  the best stationary equilibrium with
interest on currency, is optimal.  They attain this
special case as the limit of a sequence of economies
with $\rho \downarrow 0$.  Heuristically, under the natural debt limits,
the  $Ea(r)$ curves converge to a horizontal line at $r =0$.
At the limit $\rho=0$, the argument leading to Proposition 4 allows
for the optimal $r=\rho$ equilibrium.

\subsection{Implicit interest through deflation}
There is another arrangement equivalent to paying explicit interest on
currency.  Here the government aspires to pay interest through deflation,
but abstains from paying explicit interest.  This purpose is accomplished
by setting $\tilde r=0$ and
 $ \tau p_t = - g M_t$, where it is intended that the outcome will be
$ (1+r)^{-1} = (1+g)$, with $g < 0$.  The government budget constraint
becomes
$ M_{t+1} = M_t(1+g)$.  This can be written
$$ {M_{t+1} \over p_t} = {M_t \over p_{t-1}} {p_{t-1} \over p_t} (1+g). $$
We seek a steady state with constant real balances and inverse of the
gross inflation rate ${p_{t-1} \over p_t} = (1+r)$.  Such a steady state
implies that the preceding equation gives
$ (1+r) = (1+g)^{-1},$  as desired.  The implied lump-sum tax rate is
$ \tau = - {M_t \over p_{t-1}} (1+r) g. $  Using
$(1+r)=(1+g)^{-1}$, this can be expressed
$$ \tau =  {M_t \over p_{t-1}} r. $$

  The household's budget constraint with taxes set in this way
becomes
$$ c_t + {m_{t+1} \over p_t} \leq {m_{t} \over p_{t-1}} (1+r)
      + w s_t - {M_t \over p_{t-1}} r  \EQN rao10$$
This matches Aiyagari's setup with
$ {M_t \over p_{t-1}} = \phi.$

   With these matches the steady-state equilibrium is determined
just as though explicit interest were paid on currency.   The intersection
of the $E a(r)$ curve with the $r$-axis determines the real interest
rate.  Given the parameter $b$ setting the debt limit, the interest
rate equals that for the economy with explicit interest on currency.


\section{Precautionary savings}
As we have seen in the production economy with idiosyncratic labor
income shocks, the steady-state capital stock is larger when agents
have no access to insurance markets as compared to
the capital stock in a complete markets economy. The
``excessive'' accumulation of capital can be thought of as the
economy's aggregate amount of {\it \idx{precautionary savings}}---a point
emphasized by Huggett and Ospina (2000).
The precautionary
demand for savings is usually described as the extra savings
caused by future income being random rather than determinate.\NFootnote{Neng Wang (2003)
describes an analytically tractable Bewley model with exponential utility. He is
able to decompose the   savings  of an infinitely lived agent into three pieces:
(1) a part reflecting a ``rainy day'' motive that would also be present with quadratic preferences;       (2)       a part coming from a precautionary motive; and (3) a   dissaving component due to impatience that reflects the relative sizes of the interest rate and the consumer's discount rate.  Wang  computes the equilibrium of a Bewley  model by hand and  shows that, at the equilibrium
interest rate, the second and third components cancel, effectively leaving the consumer to behave  as a permanent-income consumer having a martingale consumption policy.}
\auth{Wang, Neng}
\auth{Huggett, Mark}
\auth{Ospina, Sandra}\auth{Leland, Hayne E.}
\auth{Sandmo, Agnar}

In a partial equilibrium savings problem, it has been known since
Leland (1968) and Sandmo (1970)
that precautionary savings in response
to risk are associated with convexity of the marginal utility
function, or a positive third derivative of the utility function.
In a two-period model, the intuition can be obtained
from the Euler equation, assuming an interior solution with respect
to consumption:
$$
u'[(1+r)a_0+w_0-a_1] = \beta (1+r) E_0 u'[(1+r)a_1+w_1],
$$
where $1+r$ is the gross interest rate, $w_t$ is labor
income (endowment) in period $t=0,1$,   $a_0$ is an initial
asset level, and $a_1$ is the optimal amount of savings between periods $0$ and
$1$. Now compare the optimal choice of $a_1$ in two economies
where next period's labor income $w_1$ is either determinate
and equal to $\bar w_1$, or random with a mean value of $\bar w_1$.
Let $a^n_1$ and $a^s_1$ denote the optimal choice of savings in
the nonstochastic and stochastic economy, respectively,
that satisfy the Euler equations:
$$\EQNalign{
u'[(1+r)a_0+w_0- a^n_1]
   &= \beta (1+r) u'[(1+r) a^n_1+ \bar w_1]                    \cr
\noalign{\vskip.2cm}
u'[(1+r)a_0+w_0- a^s_1]
   &= \beta (1+r) E_0 u'[(1+r) a^s_1+ w_1]                 \cr
\noalign{\vskip.1cm}
   &> \beta (1+r) u'[(1+r) a^s_1+ \bar w_1],   \cr}
$$
where the strict inequality is implied by Jensen's inequality
under the assumption that $u'''>0$. It follows immediately from
these expressions that
the optimal asset level is strictly greater in the stochastic
economy as compared to the nonstochastic economy, $a^s_1>a^n_1$.

\auth{Caballero, Ricardo J.}
\auth{Kimball, Miles S.}
\auth{Carroll, Christopher D.}
\auth{Miller,  Bruce L.}
\auth{Zeldes, Stephen P.}
\auth{Sibley, David S.}
Versions of precautionary savings have been analyzed by
Miller (1974), Sibley (1975), Zeldes (1989), Caballero (1990),
Kimball (1990, 1993), and Carroll and Kimball (1996),
 to mention just a few other studies in a vast literature.
Using numerical methods for a finite horizon savings problem
and assuming a constant relative risk aversion utility function,
Zeldes (1989) found that introducing
labor income uncertainty made the optimal consumption function
concave in assets. That is, the marginal propensity to
consume out of assets or transitory income declines with
the level of assets. In contrast, without uncertainty and when
$\beta (1+r)=1$ (as assumed by Zeldes),
the marginal propensity to consume depends only on the number of
periods left to live, and is neither a function
of the agent's asset level nor the present value of lifetime
wealth.\NFootnote{When
$\beta (1+r)=1$ and there are $T$ periods left to live in
a nonstochastic economy, consumption smoothing prescribes
a constant consumption level $c$ given by
$\sum_{t=0}^{T-1} {c \over (1+r)^t} = \Omega$, which
implies % \qquad \Longrightarrow
 $c = {r \over 1+r} \left[1 - {1 \over (1+r)^T} \right]^{-1} \Omega
\,\equiv \, {\rm MPC}_T \;\Omega,$
where $\Omega$ is the agent's current assets plus the present
value of her future labor income. Hence, the marginal propensity
to consume out of an additional unit of assets or transitory income,
${\rm MPC}_T$, is only a function of the time horizon $T$.}
Here we briefly summarize Carroll and Kimball's (1996) analytical
explanation for the concavity of the consumption function that
income uncertainty seemed to induce.

\auth{Kimball, Miles S.}
\auth{Carroll, Christopher D.}
In a finite horizon model where
both the interest rate and endowment are stochastic
processes, Carroll and Kimball
cast their argument in terms
of the class of hyperbolic absolute risk aversion (HARA)
\index{hyperbolic absolute risk aversion}%
one-period utility functions.  These are defined by ${u''' u'
\over u^{\prime \prime 2}} = k$ for some number $k$. To induce
precautionary savings, it must be true that $k >0$. Most commonly
used utility functions are of the HARA class: quadratic utility
has $k=0$, constant absolute risk aversion (CARA) corresponds to
$k=1$, and constant relative risk aversion (CRRA) utility
functions satisfy $k>1$. \index{constant relative risk aversion}
\index{constant absolute risk aversion}

 Carroll and Kimball show that
if $k >0$, then consumption is a concave function of
wealth. Moreover, except for some special cases, they
show that the consumption function is {\it strictly\/} concave;
that is, the marginal propensity to consume out of wealth declines with
increases in wealth. The exceptions to strict concavity include
two well-known cases: CARA utility if all of the risk is to labor
income (no rate-of-return risk), and CRRA utility if all of
the risk is rate-of-return risk (no labor-income risk).

In the course of the proof, Carroll and Kimball generalize the
result of Sibley (1975) that a positive third derivative of the
utility function is inherited by the value function. For there to
be precautionary savings, the third derivative of the value
function with respect to assets must be {\it positive\/}; that is,
the marginal utility of assets must be a convex function of
assets.  The case of the quadratic one-period utility is an example
where  there is no precautionary saving.  Off corners, the value
function is quadratic, and the third derivative of the value
function is zero.\NFootnote{In linear-quadratic models, decision
rules for
 consumption and asset accumulation are independent of
the variances of innovations to exogenous income processes.}
%%\auth{Weil, Phillipe}

Where precautionary saving occurs, and where the marginal utility of
consumption is  always positive, the consumption function becomes
approximately linear for large asset levels.\NFootnote{Roughly
speaking, this follows from applying the Benveniste-Scheinkman
formula and noting that, where $v$ is the value function, $v''$ is
increasing in savings and $v''$ is bounded.}
  This feature of the consumption function plays a decisive role
in governing the behavior of a model of Krusell and Smith (1998),
to which we now turn.
\auth{Smith, Anthony}
\auth{Krusell, Per}



%  Within our model of the savings problem,
%particular conditions on the one-period utility function $u$
%make the household engage in {\it \idx{precautionary savings}}.
%Precautionary savings are said to occur if in response
%to a hypothetical increase in the variance of labor income, holding
%its mean constant,
% the household saves more at any given level of assets.
%  Leland (1987), %Sandmo (19??),
% Miller (1974),
% Zeldes (1989), Cabellero (1990), Kimball (1990, 1993),
%and Carroll and Kimball (1996)
% have analyzed
%versions of precautionary savings.
%In the context of the stochastic saving problem,
%we briefly summarize some results from the precautionary savings
%literature.
%\auth{Cabellero, Ricardo J.}
%\auth{Kimball, Miles S.}
%\auth{Carroll, Christopher D.}
%\auth{Miller,  Bruce L.}
%\auth{Leland, Hayne E.}
%\auth{Zeldes, Stephen P.}
%
%   We are interested in the situation that $\beta (1+r) < 1$.
%  Whether precautionary savings emerge depends on the third derivative
%of the value function, which in turn depends on the  third derivative
%of the one-period utility function.  For a model in
%which both the interest rate and endowment are stochastic
%processes, Carroll and
%Kimball (1996) give conditions under which the
%third-derivative properties of the utility function
%are transmitted to the value function.  For there to be precautionary
%savings, the third derivative of the value function with
%respect to assets must be {\it positive\/}; that is, the
%marginal utility of assets must be a convex function of
%assets.
%\index{constant relative risk aversion}
%\index{hyperbolic absolute risk aversion}

%  Carroll and Kimball cast their argument in terms
%of the class of hyperbolic absolute risk-aversion (HARA)
%one-period utility functions.  These are defined by
%${u''' u' \over u^{\prime \prime 2}} = k$ for some number $k$.
%To induce precautionary savings, it must be true that
%$k >0$.       Carroll and Kimball give the following
%examples of HARA utility functions: $u(c) = -(B-c)^\alpha$ for a constant
%$B$;
%for $\alpha = 1.5, 2, 2.5$, the associated values of $k$ are
%$k=-1, 0, .33$.
%Carroll and Kimball note that HARA utility functions with $k >1$ are
%the constant relative risk-aversion CRRA functions.  These display
%decreasing absolute risk aversion.

% Carroll and Kimball show   that
%if $k >0$, then consumption   is a concave function of
%wealth, except in some special cases.   Therefore
%the marginal propensity to consume out of wealth declines with
%increases in wealth.
%\medskip
%\item{1.}  Sketch argument   based on small noise expansion.
%\medskip


\section{Models with fluctuating aggregate variables}
  That the aggregate equilibrium state variables are constant helps
makes the preceding models tractable.  This section describes a way to
extend such models to situations with time-varying stochastic aggregate
state variables.\NFootnote{See Duffie, Geanakoplos, Mas-Colell, and
McLennan (1994) for a general formulation and equilibrium existence
theorem for such models.   These authors cast doubt on whether
in general the current distribution of wealth is enough to serve
as a complete description of the history of the aggregate state.
They show that in addition to the distribution of wealth, it can be necessary to add
a sunspot to the state.  See Miao (2003) for a later treatment and for an interpretation of
the additional state variable in terms of a distribution of continuation values.
  See Marcet and Singleton (1999) for a computational
strategy for incomplete markets models with a finite number of
heterogeneous agents.}
\auth{Mas-Colell, Andreu}      \auth{Singleton, Kenneth J.}
\auth{McLennan, Andrew}        \auth{Geanakoplos, John} \auth{Miao, Jianjun}
\auth{Marcet, Albert}

  Krusell and Smith (1998) modified Aiyagari's (1994) model
by adding an aggregate state variable $z$, a technology shock
that follows a Markov process.  Each household continues
to receive  an idiosyncratic labor-endowment shock $s$ that
averages to the same constant value for each value  of the aggregate shock
$z$.  The aggregate shock causes the size of the state of the
economy to expand dramatically, because every household's wealth
will depend on the history of the {\it aggregate\/} shock $z$, call it
$z^t$, as well as the history of the household-specific shock $s^t$.
That makes the joint histories of $z^t,s^t$ correlated across households,
which in turn makes the cross-section distribution of $(k,s)$ vary
randomly over time.  Therefore, the interest rate and wage will also vary
randomly over time.

  One way to specify the state is to include the cross-section distribution
$\lambda(k,s)$  {\it each period\/} among the state variables.  Thus, the
state includes a cross-section probability distribution of
(capital, employment) pairs.  In addition, a description of a recursive
competitive equilibrium must include a law of motion mapping today's
distribution $\lambda(k,s)$ into tomorrow's distribution.

\subsection{Aiyagari's model again}
To prepare the way for Krusell and Smith's way of handling
such a model, we recall the structure of
Aiyagari's model.  The household's Bellman equation in Aiyagari's model
is
$$ v(k,s) = \max_{c, k'}\left\{ u(c) + \beta  E [v(k', s') \vert s] \right\}
   \EQN bellmanrao $$
where the maximization is subject to
$$ c+ k'  = \tilde r k + w s + (1-\delta) k, \EQN constr0 $$
and the prices  $\tilde r$ and $w$ are   fixed numbers satisfying
$$ \EQNalign{ \tilde r & = \tilde r(K, N) = \alpha \left({K \over N }
           \right)^{\alpha -1} \EQN constr1;a \cr
                 w & = w(K, N) = (1- \alpha) \left({K \over N }\right)^\alpha
    .   \EQN constr1;b \cr }$$
Recall that  aggregate capital and labor $K,N$
 are the average values of $k,s$ computed from
$$\EQNalign{K & = \int k \lambda (k,s) d k d s  \EQN meank \cr
        N & = \int s \lambda (k,s) d k d s . \EQN meann \cr} $$
Here we are following Aiyagari by assuming a Cobb-Douglas
aggregate production function.
The definition of a stationary equilibrium
requires that $\lambda(k,s)$ be the stationary distribution of $(k,s)$ across
households induced by the decision rule that attains the
right side of equation \Ep{bellmanrao}.

\subsection{Krusell and Smith's extension}
  Krusell and Smith (1998) modify Aiyagari's model by adding an
aggregate productivity shock $z$ to the price equations, emanating
from the presence of $z$ in the production function. The shock $z$
is governed by an exogenous Markov process.
Now the state must include $\lambda $ and
$z$ too, so the household's Bellman equation
becomes
  $$ v(k,s;\lambda,z) = \max_{c, k'}\left\{
 u(c) + \beta  E [v(k', s';\lambda',z') \vert (s,z,\lambda)] \right\}
   \EQN bellmanrao2 $$
where the maximization is subject to
$$ \EQNalign{ c+ k' & = \tilde r(K,N,z)k + w(K,N,z) s +
 (1-\delta) k \EQN constr2;a \cr
                 \tilde r & = \tilde r(K, N, z) =z \alpha \left({K \over N }
           \right)^{\alpha -1} \EQN constr2;b \cr
                 w & = w(K, N, z) = z (1- \alpha) \left({K \over N }\right)^\alpha
    \EQN constr2;c \cr
         \lambda' & = H(\lambda,z) \EQN constr2;d \cr}$$
where $(K,N)$ is a  stochastic processes determined from\NFootnote{In our
simplified formulation, $N$ is actually
constant over time.  But in Krusell
and Smith's model, $N$ too can be a stochastic process, because
leisure is in the one-period utility function.}
$$\EQNalign{K_t  & = \int k \lambda_t (k,s) d k d s  \EQN bmeank \cr
        N_t & = \int s \lambda_t (k,s) d k d s. \EQN bmeann \cr}$$
Here $\lambda_t(k,s)$ is the distribution of $k,s$ across
households at time $t$.  The {\it distribution} is itself
a random function disturbed by the aggregate shock $z_t$.

Krusell and Smith make the plausible guess that $\lambda_t(k,s)$
is enough to complete the description of the state.\NFootnote{However,
in general settings, this guess remains to be verified. Duffie, Geanakoplos,
Mas-Colell, and McLennan (1994) give an example of an incomplete markets
economy in which it is necessary to keep track of a longer history of
the distribution of wealth.}$^,$\NFootnote{Loosely speaking,
that the individual moves through the distribution of wealth as time
passes indicates that his implicit Pareto weight is fluctuating.}
The Bellman equation and the pricing functions induce the
household to want to forecast the average capital stock $K$,
in order to forecast future prices.  That desire makes the household want
to forecast the cross-section distribution of holdings of
capital. To do so it consults the law of motion \Ep{constr2;d}.

\medskip
\medskip\noindent{\sc Definition:}  A recursive competitive equilibrium is
a pair of price functions $\tilde r, w$, a  value function,
a decision rule $k'=f(k,s;\lambda,z)$, and a law of motion $H$
for $\lambda(k,s)$ such that
\medskip
\noindent (a) given the price functions
and $H$, the value function solves the Bellman equation \Ep{bellmanrao2} and
the optimal  decision rule  is $f$;
\medskip
\noindent (b) the decision rule $f$ and the
Markov processes for $s$ and $z$ imply that today's distribution
$\lambda(k,s)$ is mapped into tomorrow's  $\lambda'(k,s)$ by $H$.

\medskip

\auth{Honkapohja, Seppo}%
\auth{Evans, George W.}%
\auth{Sargent, Thomas J.}%
\auth{Marcet, Albert}%
The curse of dimensionality makes an equilibrium difficult to compute.
Krusell and Smith propose a way to approximate an equilibrium using
simulations.\NFootnote{These simulations  can be justified formally using lessons learned from the literature
on convergence of least squares learning to rational expectations in self-referential environments. See footnote \use{FT:MSEH} of chapter \use{recurpe}, the paper by
   Marcet and Sargent (1989), and the book with extensions and many applications by
   Evans and Honkapohja (2001).} First, they characterize the distribution $\lambda(k,s)$
by a finite set of moments of capital $m = (m_1, \ldots, m_I)$.  They
assume a parametric functional form for $H$ mapping today's $m$ into
next period's value $m'$.  They assume a form that can be conveniently
estimated using least squares.  They assume initial values for the parameters
of $H$.  Given $H$, they use numerical dynamic programming to solve the
Bellman equation
  $$ v(k,s;m,z) = \max_{c, k'}\left\{
 u(c) + \beta  E [v(k', s';m',z') \vert (s,z,m)] \right\}$$
subject to the assumed  law of motion $H$ for $m$.   They take the solution
of this problem and draw a single long realization from the Markov process
for $\{z_t\}$, say, of length $T$.  For that particular realization of $z$, they
then simulate paths of $\{k_t,s_t\}$ of length $T$ for a large number $M$
of households.  They assemble these $M$ simulations into
a history  of $T$ empirical cross-section distributions $\lambda_t(k,s)$.
They use the cross section at $t$ to compute the cross-section
moments $m(t)$, thereby assembling a time series of length $T$ of the
cross-section moments $m(t)$.  They use this sample and nonlinear least squares
to estimate the transition function $H$ mapping $m(t)$ into $m(t+1)$.
 They return to the beginning of the procedure, use this new guess at
$H$, and continue, iterating to convergence of the function $H$.

   Krusell and Smith
compare the  aggregate time series $K_t, N_t,\tilde r_t, w_t$
from this model with a corresponding representative agent
(or complete markets) model.  They find that the statistics
for the  aggregate quantities and prices for the two types of models
are very close.   Krusell and Smith interpret this result in terms
of an ``approximate aggregation theorem'' that follows from two properties
of their parameterized model.  First, consumption as a function of wealth
 is concave but close to linear for moderate to high wealth levels.
Second, most of the saving is done by the high-wealth people.  These two
properties mean that fluctuations in the distribution of wealth have only
a small effect on the aggregate amount saved and invested.  Thus, distribution
effects are small.  Also, for these high-wealth people, self-insurance
works quite well, so aggregate consumption is not much lower than it would
be for the complete markets economy.

  Krusell and Smith  compare the distributions of wealth
from their model to the U.S. data.  Relative to the data, the model
with a constant discount factor generates too few very poor
people and too many rich people.   Krusell and Smith modify the
model by making the discount factor an exogenous stochastic process.
The discount factor switches occasionally between two values.  Krusell and
Smith find   that a modest difference between two discount factors can
bring the model's wealth distribution much closer to the data.
Patient people become wealthier; impatient people eventually become
poorer.


\section{Concluding remarks}
  The models in this chapter pursue some of the adjustments
that households  make when their preferences and
endowments give a motive to insure but markets offer
limited opportunities to do so.  We have studied
settings where households' saving  occurs through a single
risk-free asset.  Households use the asset to ``self-insure,''
by making intertemporal adjustments of the asset holdings to smooth
their consumption. Their consumption  rates at a given
date become a function of their asset holdings,
which in turn depend on the  histories of their
endowments.   In pure exchange versions of the model,
the equilibrium allocation becomes individual history specific,
in contrast to the history-independence of the corresponding
complete markets model.


  The models of this chapter arbitrarily shut down or allow markets
without explanation.   The market structure is imposed, its consequences
then analyzed.  In chapter \use{socialinsurance}, we study a class of  models
for similar environments that, like the models of this chapter,
make consumption allocations history dependent. But
the spirit of the models in chapter \use{socialinsurance} differs from
those in this chapter in requiring that the trading structure
be more firmly motivated by the environment.  In particular,
the models in chapter \use{socialinsurance}
posit a particular   reason that complete markets do not exist,
coming from enforcement or information problems, and then study how
risk sharing among people can best be arranged.


%\section{Exercises}
\showchaptIDfalse
\showsectIDfalse
\section{Exercises}
\showchaptIDtrue
\showsectIDtrue
\noindent{\it Exercise \the\chapternum.1}\quad {\bf Random discount factor}
\ \ (Bewley-Krusell-Smith)
\medskip
\noindent A household has preferences
over consumption of a single good ordered by a value function
defined recursively by
$v(\beta_t, a_t, s_t) = u(c_t) +  \beta_t  E_t v(\beta_{t+1}, a_{t+1},
 s_{t+1}),$
where $\beta_t \in (0,1) $ is the time $t$ value of a discount
factor, and $a_t$ is time $t$ holding of a single asset. Here
$v$ is the discounted utility for a consumer with
asset holding $a_t$, discount factor $\beta_t$, and
employment state $s_t$.
The discount factor evolves according to a three-state Markov
chain with transition probabilities $P_{i,j}=
{\rm Prob} (\beta_{t+1} = \bar \beta_j \vert \beta_t = \bar \beta_i)$.
The discount factor and employment state at $t$ are both known.
The household faces the sequence of budget constraints
$$ a_{t+1} + c_t \leq (1+r) a_t + w s_t $$
where $s_t$ evolves according to an $n$-state Markov
chain with  transition matrix ${\cal P}$.   The household
faces the borrowing constraint $a_{t+1} \geq - \phi$ for all
$t$.
\medskip
\noindent Formulate Bellman equations for the household's problem.
Describe an algorithm for solving the Bellman equations.
({\it Hint:} Form three coupled Bellman equations.)

\medskip

\noindent{\it Exercise \the\chapternum.2}\quad {\bf Mobility costs}\ \ (Bertola)
\medskip\noindent
A worker seeks to maximize
$E \sum_{t=0}^\infty \beta^t u(c_t)$, where
$\beta \in (0,1)$ and $u(c) = {c^{1-\sigma} \over (1-\sigma)}$, and
$E $ is the expectation operator.  Each period,
the worker supplies one unit of labor inelastically
(there is no unemployment) and
either $w^g$ or $w^b$, where $w^g > w^b$.  A  new ``job''
starts off paying $w^g$ the first period.  Thereafter,
a job earns a wage governed by the two-state Markov process governing
transition between good and bad wages on all jobs; the transition matrix
is $\left[\matrix{ p & (1-p) \cr
                    (1-p) & p \cr}\right]$.
A new (well-paying) job is always available, but the worker must pay mobility
cost $m>0$ to change jobs.  The mobility cost is paid at the beginning of the
period that a worker decides to move.  The worker's period $t$ budget constraint
is
$$ A_{t+1} + c_t + m I_t \leq R A_t + w_t, $$
where $R$ is a gross interest rate on assets, $c_t$ is consumption
at $t$, $m>0$ is moving costs, $I_t$ is an indicator equaling $1$
if the worker moves in period $t$, zero otherwise, and $w_t$ is the wage.
Assume that $A_0 >0$ is given and that the worker faces the no-borrowing
constraint, $A_t \geq 0$ for all $t$.

\medskip
\noindent{\bf a.} Formulate the Bellman equation for the worker.
\medskip
\noindent{\bf b.}  Write a Matlab program to  solve the worker's
Bellman equation.  Show the optimal decision rules computed for
the following parameter values: $m=.9, p=.8, R=1.02,  \beta = .95,
w^g= 1.4, w^b =1, \sigma=4$.  Use a range of assets levels of
$[0, 3]$.  Describe how the decision to move depends on wealth.

\medskip
\noindent{\bf c.}  Compute the Markov chain governing the transition
of the individual's state $(A,w)$.   If it exists, compute
the invariant distribution.

\medskip
\noindent{\bf d.}  In the fashion of Bewley,
 use the invariant distribution  computed in part  c
to describe the distribution of wealth across  a large number of workers
all facing this same optimum problem.

\vfil\eject
\medskip\noindent
{\it Exercise \the\chapternum.3} \quad {\bf Unemployment}
\medskip\noindent
 There is a continuum of workers with identical
probabilities $\lambda$ of being fired each period
when they are employed. With probability $\mu \in (0,1)$,
each unemployed worker receives one offer to work
at wage $w$ drawn from the cumulative distribution
function $F(w)$. If he accepts the offer,
the worker receives the offered wage each
period until he is fired. With probability $1-\mu$, an unemployed
worker receives no offer this period. The probability
$\mu$ is determined by the function
$ \mu= f(U)$, where $U$ is the unemployment rate, and
$f'(U) < 0, f(0) =1, f(1)=0$. A worker's
utility is given by $E \sum_{t=0}^\infty \beta^t y_t$, where
$\beta \in (0,1)$ and $y_t$ is income in period $t$,
which equals the wage if employed and zero otherwise.
There is no unemployment compensation. Each worker
regards $U$ as fixed and constant over time in making his
decisions.
\medskip
\noindent {\bf a.} For fixed $U$, write the Bellman equation
for the worker. Argue that his optimal policy
has the reservation wage property.
\medskip
\noindent {\bf b.} Given the typical worker's
policy (i.e., his reservation wage),
 display a difference equation for the unemployment
rate. Show that a stationary unemployment rate
must satisfy
$$ \lambda (1-U) = f(U) \bigl[ 1-F(\bar w)\bigr] U , $$
where $\bar w$ is the reservation wage.
\medskip
\noindent{\bf c.} Define a {\it stationary equilibrium}.
\medskip
\noindent{\bf d.} Describe how to compute a stationary
equilibrium. You don't actually have to compute it.

\medskip

\noindent{\it Exercise \the\chapternum.4} \quad  {\bf Asset insurance}
\index{deposit insurance}
\medskip
\noindent
Consider the following setup.
There is a continuum of households that maximize
$$ E \sum_{t=0}^\infty \beta^t u(c_t) ,$$
subject to
$$ c_t + k_{t+1} + \tau \leq y + \max(x_t,g) k_t^\alpha , \quad
           c_t \geq 0,\ k_{t+1}\geq 0,\ t \geq 0, $$
where $y> 0$ is a constant level of income not derived from capital,
$\alpha \in (0,1)$, $\tau$ is a fixed lump-sum tax, $k_t$ is the
capital held at the beginning of $t$, $g \leq 1$ is an ``investment
insurance'' parameter set by the government, and $x_t$ is a
stochastic household-specific gross rate of return on capital.
We assume that $x_t$ is governed by a two-state Markov process
with stochastic matrix $\cal P$, which takes on the two values
$\bar x_1 > 1$ and $\bar x_2 < 1$. When the bad investment return
occurs, ($x_t = \bar x_2$), the government supplements the
household's return by $\max(0, g - \bar x_2)$.

  The household-specific randomness is distributed identically and
independently across households. Except for paying taxes and possibly
receiving insurance payments from the government, households have
no interactions with one another; there are no markets.

  Given the government policy parameters $\tau, g$, the household's
Bellman equation is
$$ v(k,x) = \max_{k'} \{ u \bigl[ \max(x,g) k^\alpha - k' - \tau\bigr] +
\beta \sum_{x'} v(k',x') {\cal P} (x,x') \} .$$
The solution of this problem is attained by a decision rule
$$k' = G(k,x),$$
that induces a stationary distribution $\lambda(k,x)$ of agents across
states $(k,x)$.

  The average (or per capita) physical output of the economy is
$$ Y = \sum_k \sum_x (x \times k^\alpha) \lambda(k,x) .$$
The average return on capital to households, {\it including\/}
the investment insurance, is
$$ \nu = \sum_k \bar x_1 k^\alpha \lambda(k,x_1) +
\max(g,\bar x_2) \sum_k k^\alpha \lambda(k,x_2) ,$$
which states that the government pays out
insurance to all households for which $g > \bar x_2$.

Define a stationary equilibrium.

%%%AAAAAAA

\medskip
\noindent{\it Exercise \the\chapternum.5} \quad {\bf Matching and job quality}
\medskip
\noindent
Consider the following Bewley model, a version of which Daron Acemoglu
and Robert Shimer (2000) calibrate to deduce quantitative statements
about the effects of government-supplied unemployment insurance on the
equilibrium level of unemployment, output, and workers' welfare.  Time
is discrete.  Each of a continuum of {\it {\it ex ante\/}\/} identical workers
can accumulate nonnegative amounts of a single risk-free asset bearing
gross one-period rate of return $R$; $R$ is exogenous and satisfies
$\beta R < 1$.  There are good jobs with wage $w_g$ and bad jobs with
wage $w_b< w_g$.  Both wages are exogenous.  Unemployed workers must decide
whether to search for good jobs or bad jobs.  (They cannot search for both.)
If an unemployment worker devotes $h$ units of time to search for a good job,
a good job arrives with probability $m_g h$; $h$ units of time devoted to
searching for bad jobs makes a bad job arrive with probability $ m_b h$.
Assume that $m_g < m_b$.  Good jobs terminate exogenously each period
with probability $\delta_g$, bad jobs with probability $\delta_b$.
Exogenous terminations entitle  an unemployed worker to unemployment
compensation of $b$, which is independent of the worker's lagged earnings.
However, each period, an unemployed worker's entitlement to unemployment
insurance is exposed to an i.i.d.\ probability of $\phi$ of expiring.
Workers who quit are not entitled to unemployment insurance.

Workers choose $\{c_t, h_t\}_{t=0}^\infty$ to maximize
$$ E_0 \sum_{t=0}^\infty \beta^t
(1-\theta)^{-1}(c_t (\bar h - h_t)^\eta)^{1-\theta},$$
where $\beta \in (0,1)$, and $\theta$ is a  coefficient of relative
risk aversion, subject to the asset accumulation equation
$$ a_{t+1} = R (a_t + y_t -  c_t) $$
and the no-borrowing condition $a_{t+1} \geq 0$; $\eta$ governs the
substitutability between consumption and leisure.  Unemployed workers
eligible for unemployment insurance receive income $y_t = b$, while those not eligible
receive $0$.  Employed workers with good jobs receive after-tax income
of $y_t = w_g h (1-\tau)$, and those with bad jobs receive
$y_t = w_b h(1-\tau)$.  In equilibrium,  the flat-rate tax is set so that
the government budget for unemployment insurance balances.  Workers with bad jobs have
the option of quitting to search for good jobs.

Define a worker's composite {\it state\/} as his asset level, together with
one of four possible employment states: (1) employed in a good job, (2)
employed in a bad job,
 (3) unemployed and eligible for unemployment insurance;
 (4) unemployed and ineligible for unemployment insurance.
\medskip
\noindent{\bf a.}  Formulate value functions for the four types
of employment states, and
describe Bellman equations that link them.
\medskip
\noindent{\bf b.}  In the fashion of Bewley,
define a  stationary   stochastic equilibrium, being careful to define
all of the objects composing an equilibrium.
\medskip
\noindent{\bf c.}   Adjust the Bellman equations to accommodate
the following modification.  Assume that every period that a worker
finds himself in a bad job, there is a probability $\delta_{upgrade}$
that the following period, the bad job is upgraded to a good job,
conditional on not having been fired.

\medskip
\noindent{\bf d.}  Acemoglu and Shimer calibrate their model
to U.S. high school graduates, then perform a local analysis of
the consequences of increasing the unemployment compensation rate $b$.
For their calibration, they find that there are substantial benefits
to raising the unemployment compensation rate and that this conclusion
prevails despite the presence of a ``moral hazard problem'' associated
with providing unemployment insurance benefits in their model.  The reason is that too
many workers choose to search for bad rather than good jobs.  They calibrate
$\beta$ so that  workers are sufficiently impatient that most workers
with low assets search for bad jobs.  If workers were more fully insured,
more workers would search for better jobs. That would put a larger fraction
of workers in good jobs and  raise average productivity.  In equilibrium,
unemployed workers with high asset levels {\it do\/} search for good jobs,
because their assets provide them with the ``self-insurance'' needed to
support their investment in search for good jobs.  Do you think that
the modification suggested in part c would affect the outcomes
of increasing unemployment compensation $b$?


\medskip
\noindent{\it Exercise \the\chapternum.6} \quad {\bf Gluing stationary equilibria}
\medskip
\noindent At time $-1$, there is a continuum of {\it ex ante\/} identical consumer named $i \in (0,1)$.
Just before time $0$, net assets $\tilde a(i;0)$ drawn from a cumulative distribution function $F$  are distributed to agents. Net assets may be positive
or negative.  Agent
$i$'s net assets at the beginning of time $0$ are then $a_0 = \tilde a(i,0)$.  (To conserve notation, we'll usually supress the $i$.)
A typical consumer's labor income at time $t$ is $w s_t$ where where $w$ is a fixed positive number and $s_t$
evolves according to an $m$-state Markov chain with transition
matrix $\cal P$. Think of initiating the process from the invariant distribution
of $\cal P$ over $\bar s_i$'s.   If the realization of the process at $t$ is
$\bar s_i$, then at time $t$ the household receives labor income
$w \bar s_i$.  Let $a_t$ be the household's net assets at the beginning of period $t$.
For given initial values ($a_0, s_0$) and a given net risk-free interest rate $r$, a household
chooses a policy for
 $\{c_t, a_{t+1}\}_{t=0}^\infty$ to maximize
$$ E_0 \sum_{t=0}^\infty \beta^t u(c_t) , \leqno(1) $$
subject to
$$\eqalign{ c_t + a_{t+1} &= (1+r) a_t + w s_t \cr
       a_{t+1} &\in {\cal A}(r)\cr } \leqno(2)
$$
 where $\beta \in (0,1)$ is a discount factor; $u(c)$ is a strictly
increasing, strictly concave, twice continuously
differentiable one-period utility function satisfying
the Inada condition $\lim_{c \downarrow 0} u'(c) = +\infty$;
 and $\beta(1+r) < 1$; whether the set $ {\cal A}(r) $ might depend on $r$ depends on whether
or not we construct it from a natural borrowing limit or some tighter ad hoc limit.

\medskip

\noindent{\bf a.} Describe an equilibrium of an incomplete markets model in the style of Mark Huggett in which the only asset traded is a risk-free private IOU.  Please
define all objects of which an equilibrium consists and the restrictions that equilibrium imposes on those objects.

\medskip

\noindent{\bf b.}  Let  $\Lambda(a,s)$ be the joint probability distribution over $a, s$ in the equilibrium that you described in part {\bf a}.  How does
$\Lambda(a,s)$ relate to the cumulative distribution function $F$ in the set up of the problem?

\medskip
\noindent{\bf c.}  Suppose that parameters are such that the equilibrium net  interest rate  $r$ is negative in your Hugget model.  Call this interest rate
$r_H < 0$.  Please draw a graph with the net interest rate on the coordinate axis and average net assets across agents on the ordinate axis and show $r_H$ as well
as average savings across agents as a funtion of $r$.

\medskip
\noindent{\bf d.}  Now add a government to the  model, starting from initial conditions consistent with the equilibrium $\Lambda(a,s)$ distribution
associated with the Huggett equilibrium that you described in parts {\bf b} and {\bf c}.  The government acts only at time $0$, when it has  budget constraint
$$ T_{-1} = {\frac{M_0}{p}} ,  \quad T_{-1} = \int  \tilde T_{-1}(i) d i , $$
 where $p >0$ is the price level, $T_{-1}$ is the value of aggregate transfers of unbacked fiat currency from the government to consumers before  the beginning of time $0$,
and $\tilde T_{-1} (i)$ is  the real value of
transfers awarded to  consumer $i$.  %in the Hugget equilibrium with interest rate $r_H$,
A typical household's budget constraint  becomes
$$ c_0 + a_1 \leq w s_0 + (1 + r_H) a_0 +  \tilde T_{-1}(i)  \leqno (3) $$
at time $0$ and remains inequality   (2)  for $t \geq 1$.

\medskip
\noindent{\bf e.}  First, please define a stationary equilbrium of an incomplete markets model with valued unbacked fiat currency in the style of Bewley.  Let
$G$ be the cumulative distribution function of net assets $a$ in this equilbrium.

\medskip
\noindent{\bf f.}  {\it Extra credit:}  Recalling that $F$ is the cross section CDF of net assets in the equliibrium of the Huggett model of part {\bf c}  and that $G$ is
the cross section CDF of net assets in the equilibrium of the Bewley model of part {\bf e} please describe a scheme for awarding
the transfers of fiat money before the beginning of time $0$ that (i) preserves the ranks of all assets in the wealth distribution, and (ii) moves the initial asset distribution from $F$ to $G$,
where a consumer's initial net assets in the Bewley model are $a_B(i) = (1+ r_0) a_H(i) + \tilde T_{-1} (i)$, where $a_H(i)$ were his initial assets in
the Huggett model.
\medskip

\noindent{\it Hint:} Recall that individuals $i$ are distributed according to a uniform distribution on $[0,1]$. Measured in units of time $0$ consumption goods,
let $\tilde T_{-1} (i)$
be the transfer to agent $i$.  Guess that the transfer is $\tilde T_{-1} (i) = G^{-1}(i) - F^{-1}(i)$.  Let $a_G(i) = a_F(i)+ \tilde T_{-1} (i)$
be agent $i$'s initial assets after the transfer.
Verify that the CDF of $a_G(i)$  is $G$ as desired.\NFootnote{In effect, the guess  recommends applying what is known as an ``inverse probability integral transform'' or
``Smirnov transform''.}
\index{Smirnov transform}%
\index{inverse probability integral transform}%


\medskip
\noindent{\bf g.} {\it Extra credit:} Given the above transfer scheme for moving immediately from a stationary equilibrium of a Huggett model to a stationary
equilibrium of a Bewley model with valued fiat money, taking as given their rankings in the initial wealth distribution, do all agents prefer one   stationary equilibrium to the other? If so, which equilibrium
do they prefer? If not all agents prefer one to the other, please describe computations that would allow you to sort agents into those who  prefer to stay in the
Huggett equilibrium and those who  prefer to move to the Bewley equilibrium.



\medskip
\noindent{\it Exercise \the\chapternum.7} \quad {\bf Real bills}
\medskip

\noindent Consider the Bewley model with a constant stock $M_0$ of  valued fiat money described in exercise \the\chapternum.6.  Consider a monetary authority
that issues additional currency and that uses it to purchase one-period risk-free IOUs issued by consumers.  The monetary authority's
budget constraint is
$$ A_{t+1} = A_t + {\frac{M_{t+1} - M_t}{p}} , \quad t \geq 0 \leqno(1) $$
subject to $A_{t+1} \geq 0$ for $t \geq 0$  and $A_0 =0$, where $A_{t+1}$ is the stock of one-period IOU's that the monetary authority purchases at time $t$.    Here we guess that
$r=0$ and that s$p$ is the constant equilibrium price level in the Bewley economy with valued fiat currency
from exercise \the\chapternum.6. The equilibrium condition in the market for risk-free securities is now
$$ A_{t+1} + E a(0) = {\frac{M_{t+1}}{p}  } .  \leqno (2) $$

\medskip
\noindent{\bf a.} Solve the difference equation (1) backwards to verify that
$$ A_{t+1} = {\frac{M_{t+1} - M_0}{p} } . \leqno(3) $$

\medskip
\noindent{\bf b.}  Verify that if a constant price level $p$ satisfies $E a(0) = {\frac{M_0}{p}}$, as in the original Bewley economy,
then equilibrium condition (2) is satisfied at the same price level $p$ for any nonnegative sequence $\{A_{t+1}\}_{t=0}^\infty$ and associated
money supply sequence $\{M_{t+1}\}_{t=0}^\infty$ that satisfy constraint (1).

\medskip
\noindent{\bf c.}  Argue that an increase in $M_0$, engineered as described in exercise \the\chapternum.6, leads to an increase in the equilibrium
$p$; but that  increases in $M_{t}$ for $t \geq 1$ leave the price level unaffected.

\medskip
\noindent{\bf d.}  {\it Remark:} Sometimes $M_0$ is called outside money and $M_{t+1} - M_0$ is called inside money.  It is called
inside money because it is backed 100\% by safe private IOUs. The real bills doctrine asserts that increases in inside money are not inflationary.
\index{real bills doctrine}



