
\input grafinp3
%\input grafinput8
\input psfig

\showchaptIDtrue
\def\@chaptID{12.}

%\eqnotracetrue

%\hbox{}

\def\toone{{t+1}}
\def\ttwo{{t+2}}
\def\tthree{{t+3}}
\def\Tone{{T+1}}
\def\TTT{{T-1}}
\def\rtr{{\rm tr}}
\footnum=0
\chapter{Optimal Taxation with Commitment\label{optax}}
\index{optimal taxation!commitment}

\section{Introduction}

This chapter formulates a dynamic optimal taxation problem called a
\idx{Ramsey problem} whose solution is called a \idx{Ramsey plan}.
The government's goal is to maximize households' welfare subject to
raising prescribed revenues through distortionary taxation.  When designing
an optimal policy, the government takes into
account the competitive equilibrium reactions by consumers and firms to the
tax system. We first study a nonstochastic economy, then
a stochastic economy.

The model is a competitive equilibrium
 version of the basic neoclassical growth model
with a government  that finances an exogenous stream of government
purchases. In the simplest version, the production factors are
raw labor and physical capital on which the government levies
distorting flat-rate taxes.  The problem is
to determine optimal sequences for
the two  tax rates. In a nonstochastic economy,
Chamley (1986) and Judd (1985b)
show in related settings that if an equilibrium has an
asymptotic steady state, then the optimal policy is eventually
to set the tax rate on capital to zero.\NFootnote{Straub and
Werning (2015) offer  corrections  to Chamley's (1986)
and Judd's (1985) results about an asymptotically zero tax rate on
capital for specifications in which preferences are nonadditive intertemporally,
the government budget must be  balanced each period, and an
infinite sequence of restrictions is imposed on the sequence of tax rates on capital. Our treatment here
steers clear of these situations by assuming time-additively
separable utility, a government that can  freely access  debt markets
(subject to the usual no-Ponzi constraints),  and
a restriction on the capital tax rate only in the initial period.
\auth{Straub, Ludwig}\auth{Werning, Iv\'an}}
This remarkable result
asserts that capital income taxation serves neither efficiency
nor redistributive purposes in the long run.  The conclusion
follows immediately from time-additively separable utility,
a constant-returns-to-scale production technology,
competitive markets, and a complete set of flat-rate taxes.
However, if the tax system
is incomplete, the limiting value of the optimal capital tax
can  differ from zero.  To illustrate this possibility,
we follow Correia (1996)
and study a case with an additional fixed production factor that
cannot be taxed by the government.
\auth{Chamley, Christophe}\auth{Judd, Kenneth L.}%
\auth{Correia, Isabel H.}%

In a stochastic version of the model with complete markets, we find
indeterminacy of state-contingent debt and capital taxes.
Infinitely many plans implement the same
competitive equilibrium allocation. For example,
two such plans are  (1) that the government issues risk-free
bonds and lets the capital tax rate depend on the current state,
or (2) that the government  fixes the capital tax rate one period ahead and lets
debt be state contingent. While the state-by-state capital tax
rates cannot be pinned down, an optimal plan
does determine the current market value of next period's
tax payments across states of nature. Dividing
by the current market value of capital income gives
a measure that we call the {\it ex ante capital tax rate}.
If there exists a stationary Ramsey allocation, Zhu (1992)
shows that for some special
utility functions, the Ramsey plan prescribes a zero {\it ex ante\/} capital
tax rate that can be implemented by setting a zero tax on capital
income. But except for those   preferences, Zhu concludes
that the {\it ex ante\/} capital tax rate should vary around zero, in the
sense that there is a positive measure of states with positive tax
rates and a positive measure of states with negative tax rates.  Chari,
Christiano, and Kehoe (1994)
perform numerical simulations and conclude that
there is a quantitative presumption that the {\it ex ante\/} capital tax rate is
approximately zero.
\auth{Zhu, Xiaodong}%
\auth{Chari, V.V.}%
\auth{Christiano, Lawrence J.}%
\auth{Kehoe, Patrick J.}%
\auth{Lucas, Robert E., Jr.}%
\auth{Stokey, Nancy L.}%

To gain further insights into optimal taxation and debt policies,
we turn to Lucas and Stokey (1983)
who analyze a complete-markets model without physical capital.  Examples of deterministic
and stochastic government expenditure streams bring out the important
role of government debt in smoothing tax distortions over both time
and states. State-contingent government debt is used as an ``insurance policy'' that allows the government to smooth
taxes across  states.  In this complete markets model, the current value
of the government's debt reflects the current and likely future path
of government expenditures rather than anything about its past.  This
feature of an optimal debt policy is especially apparent when government
expenditures follow a Markov process because then the beginning-of-period
state-contingent government debt is a function of the current state only
and hence there are no lingering effects of past government expenditures.
 Aiyagari, Marcet, Sargent, and Sepp\"al\"a (2002)
alter that outcome % feature of  optimal policy in Lucas and Stokey's model
by assuming that the government can  issue only risk-free debt. Not
having access to state-contingent debt constrains the government's
ability to smooth taxes over states and allows  past values of
government expenditures to have persistent effects on both future
tax rates and debt levels.  Reasoning by analogy from the savings
problem of chapter \use{selfinsure} to an optimal taxation
problem, Barro (1979) asserted that tax revenues would be a
martingale that is  cointegrated with government debt.
%%, an outcome
%%possessing a dramatic version of such persistent effects
Barro thus predicted  persistent effects of
government expenditures that are absent from the Ramsey plan  in Lucas and Stokey's
model. Aiyagari et.\ al.'s suspension of complete markets in Lucas
and Stokey's environment goes a long way  toward rationalizing
outcomes Barro had predicted.\auth{Aiyagari, Rao} \auth{Marcet,
Albert} \auth{Sargent, Thomas J.} \auth{Sepp\" al\" a, Juha}

\auth{Jones, Larry E.} \auth{Manuelli, Rodolfo} \auth{Rossi, Peter E.}
Returning to a nonstochastic setup, Jones, Manuelli, and Rossi
(1997)
augment the model by allowing human capital accumulation. They
make the  particular assumption that the technology for human
capital accumulation is linearly homogeneous in a stock of human
capital and a flow of  inputs coming from current output. Under
this special constant returns assumption, they show that a zero
limiting tax applies also to labor income; that is, the return to
human capital should not be taxed in the limit. Instead, the
government should resort to a consumption tax. But even this
consumption tax, and therefore all taxes, should be zero in the
limit for a particular class of preferences where it is optimal
during a transition period
for the government to amass so many
claims on the private economy that the interest earnings suffice
to finance government expenditures.
%%%%While results that taxes rates
%%%%for non-capital taxes
While these successive results on optimal taxation
require ever more stringent assumptions, the
basic prescription for a zero {\it capital\/} tax in a
nonstochastic steady state is an immediate implication of
time-additively separable utility, a
constant-returns-to-scale production technology,
competitive markets, and a complete set of flat-rate taxes.

Throughout the chapter we maintain the assumption that the
government can commit to future tax rates.

\section{A nonstochastic economy}

An infinitely lived  representative household
 likes consumption, leisure streams $\{c_t, \ell_t\}_{t=0}^\infty$
that give higher values of
$$ \sum_{t=0}^\infty \beta^t u(c_t, \ell_t), \ \beta \in (0,1) \EQN cham1 $$
where $u$ is  increasing, strictly concave, and
three times  continuously differentiable in consumption  $c$ and leisure $\ell$.
The household is endowed with one unit of time that can be used for leisure
$\ell_t$ and labor $n_t$:
$$
\ell_t+n_t=1.                                                  \EQN cham0
$$

The single good is produced with labor $n_t$ and capital $k_t$.
Output can be consumed by the household, used by the government, or used
to augment the capital stock. The technology is
$$\EQNalign{ c_t + g_t + k_{t+1} &= F(k_t,n_t) + (1-\delta) k_t, \EQN cham2
                                                                        \cr}$$
where $\delta \in (0,1)$ is the rate at which capital depreciates and
$\{g_t\}_{t=0}^\infty$ is an exogenous sequence of government
purchases.
We assume a standard concave production function
$F(k,n)$ that exhibits constant returns to scale. By
Euler's theorem on homogeneous functions,  linear homogeneity of $F$   implies
$$ F(k,n) = F_k k + F_n n. \EQN cham3 $$

Let $u_c$   be the
derivative of $u(c_t, \ell_t)$ with respect to consumption;
$u_\ell$ is the derivative with
respect to $\ell$.
 We use $u_c(t)$ and $F_k(t)$ and so on
to denote the time $t$ values of the indicated objects,
evaluated at an allocation to be understood from the
context.


\subsection{Government}

The government finances its stream of purchases
$\{g_t\}_{t=0}^\infty$ by levying
flat-rate, time-varying taxes on earnings from  capital  at
rate $\tau^k_{t}$ and earnings from labor at rate $\tau^n_{t}$.
The government can also trade one-period bonds, sequential
trading of which
suffices to accomplish any intertemporal
trade in a world without uncertainty. Let $b_t$ be government
indebtedness to the private sector, denominated in time $t$-goods,
maturing at the beginning of period $t$. The government's
budget constraint is
$$ g_t =  \tau^k_t r_t k_t + \tau^n_t w_t n_t + {b_{t+1} \over R_t} - b_t ,
                                                              \EQN T_gov $$
where $r_t$ and $w_t$ are the market-determined rental rate of capital
and the wage rate for labor, respectively, denominated
in units of time $t$ goods,  and  $R_t$ is the gross rate
of return on one-period bonds held from $t$ to $t+1$. Interest earnings
on bonds are assumed to be tax exempt; this assumption is innocuous
for bond exchanges  between the government and the private sector.


\subsection{Household}\label{Household_PVbc}

A representative household chooses $\{c_t, n_t, k_{t+1}, b_{t+1}\}_{t=0}^\infty$ to  maximize expression
\Ep{cham1} subject to the following
sequence of budget constraints:
$$ c_t + k_{t+1} + {b_{t+1} \over R_t} = (1-\tau^n_t) w_t n_t
       + (1-\tau^k_t) r_t k_t + (1-\delta) k_t + b_t ,
                                                              \EQN T_bc $$
for $ t \geq 0$.
With $\beta^t \lambda_t$ as the Lagrange multiplier on the time $t$
budget constraint, the first-order conditions are
$$\EQNalign{
c_t\rm{:}&\ \ \ u_c(t) = \lambda_{t},                          \EQN T_foc1 \cr
n_t\rm{:}&\ \ \ u_{\ell}(t) = \lambda_{t} (1-\tau^n_t) w_t ,    \EQN T_foc2 \cr
k_{t+1}\rm{:}&\ \ \
\lambda_{t} = \beta \lambda_{t+1} \left[(1-\tau^k_{t+1}) r_{t+1}
   +1-\delta\right]  ,                                    \EQN T_foc3 \cr
b_{t+1}\rm{:}&\ \ \
\lambda_{t} {1 \over R_t} = \beta \lambda_{t+1}.          \EQN T_foc4 \cr}
$$
Substituting equation  \Ep{T_foc1} into equations
\Ep{T_foc2} and \Ep{T_foc3}, we obtain
$$\EQNalign{
u_\ell(t) &= u_c(t) (1-\tau^n_t) w_t  ,                      \EQN T_focX;a \cr
u_c(t) &= \beta u_c(t+1) \left[(1-\tau^k_{t+1}) r_{t+1}
   +1-\delta\right] .                                    \EQN T_focX;b \cr}
$$
Moreover, equations  \Ep{T_foc3} and \Ep{T_foc4} imply
$$
R_t = (1-\tau^k_{t+1}) r_{t+1} +1-\delta,                \EQN T_focR
$$
which is a condition not involving any quantities that the household
is free to adjust. Because only one financial asset is needed to
accomplish all intertemporal trades in a world without uncertainty,
condition \Ep{T_focR} constitutes a no-\idx{arbitrage} condition for
trades in capital and bonds that ensures that these two assets have
the same rate of return. This no-arbitrage condition can be obtained
by consolidating two consecutive budget constraints; constraint \Ep{T_bc}
and its counterpart for time $t+1$ can be merged by eliminating the
common quantity $b_{t+1}$ to get
$$\EQNalign{c_t &+ {c_{t+1} \over R_t} + {k_{t+2} \over R_t} + {b_{t+2} \over R_t R_{t+1}}
= (1-\tau^n_t) w_t n_t  \cr
& + {(1-\tau^n_{t+1}) w_{t+1} n_{t+1} \over R_t}
+\left[{(1-\tau^k_{t+1}) r_{t+1} + 1-\delta \over R_t} - 1\right] k_{t+1} \cr
&+ (1-\tau^k_t) r_t k_t + (1-\delta) k_t + b_t , \EQN T_bc2 \cr} $$
where the left side is the use of funds
and the right side measures the resources at the household's disposal.
If the term multiplying $k_{t+1}$ is not zero, the household can make
its budget set unbounded  either by buying an arbitrarily large $k_{t+1}$
when $(1-\tau^k_{t+1}) r_{t+1} + 1-\delta > R_t$, or, in the opposite
case, by selling capital short to achieve an arbitrarily large negative $k_{t+1}$.
In such arbitrage transactions, the household would finance purchases
of capital or invest the proceeds from short sales in the bond market
between periods $t$ and $t+1$.
Thus, to ensure the existence of a competitive equilibrium with bounded
budget sets, condition \Ep{T_focR} must hold.

If we continue the process of recursively using successive budget constraints
to eliminate successive $b_{t+j}$ terms, begun in
equation \Ep{T_bc2}, we arrive at the household's present-value budget
constraint,
$$\EQNalign{
\sum_{t=0}^\infty \left(\prod_{i=0}^{t-1} R_i^{-1} \right) c_t =
& \sum_{t=0}^\infty \left(\prod_{i=0}^{t-1} R_i^{-1} \right)
  (1-\tau^n_t) w_t n_t                                                    \cr
\noalign{\vskip.3cm}
&   + \left[(1-\tau^k_{0}) r_{0} + 1-\delta \right] k_0 + b_0,
                                                              \EQN T_bcPV \cr}
$$
where we have imposed the transversality conditions
$$\EQNalign{
&\lim_{T\to\infty} \left(\prod_{i=0}^{T-1} R_i^{-1} \right) k_{T+1} \,=\, 0,
                                                       \EQN T_TVC1 \cr
&\lim_{T\to\infty} \left(\prod_{i=0}^{T-1} R_i^{-1} \right)
                   {b_{T+1} \over R_{T}} \,=\, 0.    \EQN T_TVC2 \cr}$$
As discussed in chapter \use{assetpricing1},
 the household would not like to violate
these transversality conditions by choosing $k_{t+1}$ or
$b_{t+1}$ to be larger, because alternative
feasible allocations with higher consumption in finite time would
yield higher lifetime
utility. A consumption/savings plan that made either expression
negative would not be possible because the
household would not find anybody willing to be on the lending
side of the implied transactions.



\subsection{Firms}

In each period, the representative firm takes $(r_t, w_t)$ as given,
rents capital and labor from households, and  maximizes profits,
$$
\Pi = F(k_t, n_t) - r_t k_t - w_t n_t .                   \EQN T_profit
$$
The first-order conditions for this problem are
$$ \EQNalign{ r_t & = F_k(t), \EQN cham5;a \cr
              w_t & = F_n(t). \EQN cham5;b \cr }$$
In words, inputs should be employed until the marginal product of the
last unit is equal to its rental price. With constant returns to
scale, we get the standard result that pure profits are zero and
the size of an individual firm is indeterminate.

An alternative way of establishing the equilibrium conditions for the
rental price of capital and the wage rate for labor is to
substitute equation \Ep{cham3} into  equation \Ep{T_profit} to get
$$ \Pi = [F_k(t) - r_t]k_t + [F_n(t) - w_t] n_t .
$$
If the firm's profits are to be nonnegative and finite,
the terms multiplying $k_t$ and $n_t$ must be zero; that is, condition
\Ep{cham5} must hold. These conditions imply
that in any equilibrium, $\Pi =0$.

%\vfil\eject
\section{The Ramsey problem}
%%\subsection{Definitions}
 We shall use symbols without subscripts to denote
the one-sided infinite sequence for the corresponding variable,
e.g., $c \equiv \{c_t\}_{t=0}^\infty$.

\medskip\noindent{\sc Definition:} A {\it feasible allocation} is a sequence
$(k, c, \ell, g)$ that satisfies equation \Ep{cham2}.

\medskip\noindent{\sc Definition:}  A {\it price system}
is a 3-tuple of nonnegative bounded sequences $(w,r,R)$.

%\vfil\eject
\medskip\noindent{\sc Definition:}  A {\it government policy}
is a 4-tuple of sequences $(g, \tau^k, \tau^n, b)$.

\medskip\noindent{\sc Definition:} A {\it competitive equilibrium}
is a feasible allocation, a price system, and a government policy
such that  (a) given the price system and the government policy,
the allocation solves both the firm's  problem and the household's problem;
and  (b) given the allocation and the price system,
the government policy   satisfies the sequence of  government
budget constraints \Ep{T_gov}.

\medskip

There are many competitive equilibria, indexed by
different government policies.  This multiplicity  motivates the
\idx{Ramsey problem}.

\medskip\noindent{\sc Definition:}  Given $k_0$ and $b_0$, the {\it Ramsey problem\/} is
to choose a competitive equilibrium that maximizes expression
\Ep{cham1}.

\medskip

To make the Ramsey problem interesting, we always impose a restriction
on $\tau^k_{0}$, for example, by taking it as given at a small
number, say, $0$.  This approach rules out taxing the initial capital stock
via a so-called capital levy that would constitute a lump-sum tax, since
$k_0$ is in fixed supply.%
%SW%  Many papers impose  other restrictions on
%SW% $\tau^k_{t}, t \geq 1$, namely, that they be bounded above by some
%SW% arbitrarily given numbers. These bounds play an important role in
%SW% shaping  the near-term temporal properties of the optimal tax
%SW% plan, as discussed by Chamley (1986) and explored in computational
%SW% work by Jones, Manuelli, and Rossi (1993). In the analysis
%SW% that follows, we shall impose the bound on $\tau^k_t$ only for $t
%SW% =0$.
\NFootnote{According to our assumption on the technology in
equation \Ep{cham2}, capital is reversible and can be transformed
back into the consumption good.  Thus, the capital stock is a
fixed factor for only one period at a time, so $\tau^k_0$ is the
only tax that we need to restrict to ensure an interesting Ramsey
problem.}
%SW% \auth{Chamley, Christophe} \index{optimal taxation!zero
%SW% capital tax}\auth{Jones, Larry E.}
%SW% \auth{Manuelli, Rodolfo} \auth{Rossi, Peter E.}


\section{Zero capital tax}
\auth{Chamley, Christophe}
\index{optimal taxation!zero capital tax}
Following Chamley (1986),
we formulate the Ramsey problem as if the government chooses the after-tax
rental rate of capital $\tilde r_t$, and the after-tax wage rate $\tilde w_t$:
$$ \EQNalign{ \tilde r_t & \equiv (1-\tau^k_t) r_t,  \cr
              \tilde w_t & \equiv (1-\tau^n_t) w_t. \cr }$$
Using equations \Ep{cham5} and \Ep{cham3}, Chamley expresses government
tax revenues as
$$ \EQNalign{ \tau^k_t r_t k_t + \tau^n_t w_t n_t & =
(r_t - \tilde r_t) k_t + (w_t - \tilde w_t) n_t                 \cr
&= F_k(t) k_t + F_n(t) n_t - \tilde r_t k_t - \tilde w_t n_t \cr
&= F(k_t,n_t) - \tilde r_t k_t - \tilde w_t n_t.                 \cr}
$$
Substituting this expression into equation
 \Ep{T_gov} consolidates the
firm's first-order conditions with the government's budget constraint. The
government's policy choice is also constrained by the
aggregate resource constraint \Ep{cham2} and the
household's first-order conditions \Ep{T_focX}. To solve the Ramsey problem,
form a  Lagrangian  %HHHHH
$$ \EQNalign{ L = \sum_{t=0}^\infty \beta^t
  &\Bigl\{ u(c_t, 1-n_t)  \cr
& + \Psi_t \left[F(k_t,n_t) - \tilde r_t k_t - \tilde w_t n_t
                 + {b_{t+1} \over R_t} - b_t - g_t \right]       \cr
& + \theta_t \left[F(k_t,n_t) + (1-\delta) k_t -c_t -g_t -k_{t+1}\right]\cr
\noalign{\vskip.1cm}
& + \mu_{1t} \left[u_\ell(t) - u_c(t) \tilde w_t \right]            \cr
&
  + \mu_{2t} \left[u_c(t) - \beta u_c(t+1) \left(\tilde r_{t+1}
   +1-\delta\right)\right] \Bigr\},                       \EQN T_RP \cr}
$$
where $R_t= \tilde r_{t+1}+1-\delta$, as given by equation \Ep{T_focR}.
Note that the household's budget
constraint is not explicitly included because it is redundant when the
government satisfies its budget constraint and the resource constraint
holds.

The first-order condition for maximizing the Lagrangian \Ep{T_RP}  with respect to $k_{t+1}$ is
$$
\theta_t = \beta\left\{ \Psi_{t+1} \left[F_k(t+1) - \tilde r_{t+1}\right]
  + \theta_{t+1} \left[F_k(t+1) + 1-\delta \right] \right\}.     \EQN T_RPk
$$
The equation has a straightforward interpretation. A marginal increment of
capital investment in period $t$ increases the quantity of available goods
at time $t+1$ by the amount $[F_k(t+1) + 1-\delta]$, which has a social
marginal value $\theta_{t+1}$. In addition, there is an increase in tax
revenues equal to $[F_k(t+1) -\tilde r_{t+1}]$,
which enables the government to reduce its debt or other taxes by the
same amount. The reduction of the ``excess burden'' equals
$\Psi_{t+1} [F_k(t+1) -\tilde r_{t+1}]$. The sum of these two effects in
period $t+1$ is discounted by the discount factor $\beta$ and set
equal to the social marginal value of the initial investment good
in period $t$, which is  given by $\theta_t$.

Suppose that government expenditures stay constant after some period $T$,
and assume that the solution to the Ramsey problem converges to a steady
state; that is, all endogenous variables remain constant.
Using equation \Ep{cham5;a}, the steady-state version of equation \Ep{T_RPk}
is
$$
\theta = \beta\left[ \Psi \left(r - \tilde r\right)
  + \theta \left(r + 1-\delta \right)\right].         \EQN T_RPkSS
$$
Now with a constant consumption stream,
 the steady-state version of the household's
optimality condition for the choice of capital in equation \Ep{T_focX;b} is
$$
1 = \beta \left(\tilde r +1-\delta \right).   \EQN T_focKSS
$$
A substitution of equation  \Ep{T_focKSS} into equation \Ep{T_RPkSS} yields
$$
\left(\theta + \Psi \right) (r-\tilde r) = 0.     \EQN T_Kopt
$$
Since the marginal social value of goods $\theta$ is strictly positive
and the marginal social value of reducing government debt or taxes
$\Psi$ is nonnegative, it follows that $r$ must be equal to $\tilde r$,
so that $\tau^k=0$. This  analysis
establishes the following celebrated result,
versions of which were attained by Chamley (1986) and Judd (1985b).
\auth{Chamley, Christophe} \auth{Judd, Kenneth L.}

\medskip
\medskip\noindent{\sc Proposition 1:} If there exists a steady-state Ramsey allocation,
the associated limiting tax rate on capital is zero.

\medskip

It is important to keep in mind that the zero tax on capital result pertains only to
the limiting steady state. Our analysis is silent about how much
capital is taxed in the transition period.

%SW%  Its ability to borrow and {\it  lend\/} a risk-free one period asset
%SW% makes it feasible for the government to amass a stock of claims on
%SW% the private economy that is so large that eventually the interest earnings
%SW% suffice to finance the stream of government expenditures.\NFootnote{Below
%SW% we shall describe a stochastic economy in which the government cannot issue
%SW% state-contingent debt. For that economy, such a policy
%SW% would actually be the optimal one.}
%SW% Then it can set {\it all\/} tax rates to zero.  But  this is {\it not\/}
%SW%  the force that underlies the above result that $\tau_k$
%SW% should be zero asymptotically.  The zero-capital-tax outcome would prevail
%SW% even if we were to  prohibit the government from borrowing or lending by
%SW% requiring it to run a balanced budget in each period.  To see this, notice
%SW% that if we had set $b_t$ and $b_{t+1}$ equal to zero in equation \Ep{T_RP},
%SW% nothing would change in our derivation of the conclusion that $\tau^k=0$.
%SW% Thus, even when the government must perpetually raise positive revenues
%SW% from {\it some\/} source each period,  it is optimal eventually
%SW% to set $\tau_k$ to zero.
%SW% \index{redistribution}
%SW% \index{optimal taxation!redistribution}
%SW% \auth{Chamley, Christophe} \auth{Judd, Kenneth L.}
%SW% \section{Limits to redistribution}
%SW% The optimality of a limiting zero capital tax extends to
%SW% an economy with heterogeneous agents, as mentioned by Chamley (1986)
%SW% and explored in depth by Judd (1985b).
%SW% Assume a finite number of different classes of agents, $N$, and for
%SW% simplicity, let each class be the same size.  The consumption, labor
%SW% supply, and capital stock of the representative agent in class $i$
%SW% are denoted $c^i_t$, $n^i_t$, and $k^i_t$, respectively.  The utility
%SW% function might also depend on the class, $u^i(c^i_t,1-n^i_t)$, but
%SW% the discount factor is assumed to be identical across all agents.

%SW% The government can make positive class-specific lump-sum transfers
%SW% $S^i_t\geq 0$, but there are no lump-sum taxes. As before, the
%SW% government must rely on flat-rate taxes on earnings from capital
%SW% and labor. We assume that the government has a \idx{social welfare function}
%SW%  that is a positively weighted average of individual
%SW% utilities with weight $\alpha^i\geq 0$ on class $i$.  We assume that
%SW% the government runs a balanced budget, which does not affect the limiting value of zero for the tax rate
%SW% on income from capital.
%SW%  The Lagrangian associated with
%SW% the government's optimization problem becomes
%SW% $$ \EQNalign{ L = \sum_{t=0}^\infty &\beta^t
%SW% \left\{ \sum_{i=1}^N \alpha^i u^i(c^i_t, 1-n^i_t) \right. \cr
%SW% \noalign{\vskip.3cm}
%SW% & + \Psi_t \left[F(k_t,n_t) - \tilde r_t k_t - \tilde w_t n_t
%SW%                  - g_t -S_t \right]       \cr
%SW% \noalign{\vskip.3cm}
%SW% & + \theta_t \left[F(k_t,n_t) + (1-\delta) k_t -c_t -g_t -k_{t+1}\right]\cr
%SW% & + \sum_{i=1}^N \epsilon^i_t \left[\tilde w_t n^i_t
%SW%        + \tilde r_t k^i_t + (1-\delta) k^i_t + S^i_t - c^i_t - k^i_{t+1} \right]\cr
%SW% & + \sum_{i=1}^N \mu^i_{1t} \left[u^i_\ell(t) - u^i_c(t) \tilde w_t \right] \cr
%SW% & \left.
%SW%   + \sum_{i=1}^N \mu^i_{2t} \left[u^i_c(t) - \beta u^i_c(t+1) \left(\tilde r_{t+1}
%SW%    +1-\delta\right)\right] \right\},                              \EQN %SW% T_RP2 \cr}
%SW% $$
%SW% where $x_t\equiv \sum_{i=1}^N x^i_t$, for $x=c,n,k,S$.
%SW% Here we have to include the budget constraints and the first-order
%SW% conditions for each class of agents.

%SW% The social marginal value of an increment in the capital stock depends now
%SW% on whose capital stock is augmented. The Ramsey problem's first-order
%SW% condition with respect to $k^i_{t+1}$ is
%SW% $$\EQNalign{
%SW% \theta_t + \epsilon^i_t =
%SW%    \beta\Bigl\{ &\Psi_{t+1} \left[F_k(t+1) - \tilde r_{t+1}\right]
%SW%                 + \theta_{t+1} \left[F_k(t+1) + 1-\delta \right]   \cr
%SW% & + \epsilon^i_{t+1}\left(\tilde r_{t+1} +1-\delta\right) \Bigr\}. \EQN T_RPk2 \cr}
%SW% $$
%SW% If an asymptotic steady state exists in equilibrium, the time-invariant
%SW% version of this condition becomes
%SW% $$
%SW% \theta + \epsilon^i \left[1-\beta \left( \tilde r +1-\delta\right)\right]
%SW%  =     \beta\left[ \Psi \left(r - \tilde r\right)
%SW%   + \theta \left(r + 1-\delta \right)\right].           \EQN T_RPkSS2
%SW% $$
%SW% Since the steady-state condition \Ep{T_focKSS} holds for each individual
%SW% household, the term multiplying $\epsilon^i$ is zero, and we can once
%SW% again deduce condition \Ep{T_Kopt} asserting that
%SW% the limiting capital tax must
%SW% be zero in any convergent Pareto-efficient tax program.

%SW% Judd (1985b) discusses one extreme version of heterogeneity with two
%SW% classes of agents. Agents of class 1 are workers who do not save, so
%SW% their budget constraint is
%SW% $$
%SW% c^1_t = \tilde w_t n^1_t + S^1_t.
%SW% $$
%SW% Agents of class 2 are capitalists who do not work, so their budget
%SW% constraint is
%SW% $$
%SW% c^2_t + k^2_{t+1} = \tilde r_t k^2_t + (1-\delta) k^2_t + S^2_t.
%SW% $$
%SW% Since this setup is also covered by the preceding analysis,
%SW%  a limiting zero capital tax remains optimal if there is
%SW% a steady state. This fact implies, for example, that if the government only
%SW% values the welfare of workers ($\alpha^1>\alpha^2=0)$, there will not be
%SW% any recurring redistribution in
%SW% the limit. Government expenditures will be
%SW% financed solely by levying wage taxes on workers.

%SW% It is important to keep in mind that the zero tax on capital result pertains only to
%SW% the limiting steady state. Our analysis is silent about how much
%SW% redistribution is accomplished in the transition period.


\index{Ramsey problem!primal approach}
\index{primal approach}
\section{Primal approach to the Ramsey problem}
In the formulation of the Ramsey problem in expression \Ep{T_RP}, Chamley
reduced  a pair of taxes $(\tau^k_t, \tau^n_t)$ and a pair of
prices $(r_t, w_t)$ to just one pair of numbers
$(\tilde r_t, \tilde w_t)$ by utilizing the firm's first-order
conditions and equilibrium outcomes in factor markets. In a
similar spirit, we will now eliminate all prices and taxes so that
the government can be thought of as directly choosing a feasible
allocation, subject to constraints that ensure the existence
of prices and taxes such that the chosen allocation is
consistent with the optimization behavior of households and firms.
This primal approach to the Ramsey problem, as opposed to the
dual approach in which tax rates are viewed as governmental
decision variables, is used in Lucas and Stokey's (1983)
analysis of an economy without capital. Here we will follow the setup
of Jones, Manuelli, and Rossi (1997).
\auth{Jones, Larry E.}\auth{Manuelli, Rodolfo}\auth{Rossi, Peter E.}%
\auth{Lucas, Robert E., Jr.} \auth{Stokey, Nancy L.}%

%%To facilitate comparison to the formulation in equation
%%\Ep{T_RP}, we will now only
%%consider the case when the government is free to trade in the bond
%%market.
It is useful to compare our primal approach to the Ramsey problem
with the formulation in \Ep{T_RP}.
%SW% First, we will now  consider only
%SW% the case when the government is free to trade in the bond market.
Following the derivations in section \use{Household_PVbc},
the constraints associated with Lagrange multipliers $\Psi_t$ in \Ep{T_RP}
can be replaced with a
single present-value budget constraint
for either the government or the representative household. (One
of them is redundant, since we are also imposing the aggregate resource
constraint.) The problem simplifies nicely if
we choose the present-value budget constraint of the household
\Ep{T_bcPV}, in which future capital stocks have been eliminated with
the use of no-arbitrage conditions. For convenience, we repeat the
household's present-value budget constraint \Ep{T_bcPV} here in the form:
$$
\sum_{t=0}^\infty q^0_t c_t =
\sum_{t=0}^\infty q^0_t (1-\tau^n_t) w_t n_t
   + \left[(1-\tau^k_{0}) r_{0} + 1-\delta \right] k_0 + b_0\,.
                                                              \EQN T_bcPV2
$$
In  equation \Ep{T_bcPV2}, $q^0_t$ is the Arrow-Debreu price
$$
q^0_t = \prod_{i=0}^{t-1} R_i^{-1}, \hskip.5cm \forall t\geq1;    \EQN T_q
$$
with the numeraire $q^0_0=1$. Second, we use two
constraints in expression \Ep{T_RP} to replace prices
$q^0_t$ and $(1-\tau^n_t) w_t$ in equation \Ep{T_bcPV2} with the household's
marginal rates of substitution.

A stepwise summary of the primal approach is:
\medskip
\noindent{\bf 1.} Obtain the first-order conditions of the household's
and the firm's problems, as well as any arbitrage pricing conditions.
Solve these conditions for
$\{q^0_t, r_t, w_t, \tau^k_t$,
$\tau^n_t\}_{t=0}^\infty$
as functions
of the allocation $\{c_t, n_t, k_{t+1}\}_{t=0}^\infty$.
\medskip
\noindent{\bf 2.}  Substitute these expressions for taxes  and
prices in terms of the allocation into the household's
present-value budget constraint.   This is an intertemporal constraint
involving only the allocation.
\medskip
\noindent{\bf 3.}  Solve for the Ramsey allocation by maximizing
expression \Ep{cham1} subject to  equation \Ep{cham2} and the
``implementability condition'' derived in step 2.
\medskip
\noindent{\bf 4.}  After the Ramsey  allocation is solved,
use the formulas from step 1 to find taxes and
prices.

\subsection{Constructing the Ramsey plan}
We now carry out the steps outlined in the preceding list of
instructions.

\medskip

\noindent{\it Step 1.}  Let $ \lambda$ be a Lagrange
multiplier on the household's budget constraint \Ep{T_bcPV2}.
The first-order conditions for the household's problem are
$$\EQNalign{
c_t\rm{:}&\ \ \ \ \beta^t u_c(t)- \lambda q^0_t = 0,
                                                  \cr
n_t\rm{:}&\ \ \ -\beta^t u_{\ell}(t) + \lambda
                     q^0_t (1-\tau^n_t) w_t = 0.  \cr}
$$
With the numeraire $q^0_0=1$, these conditions imply
$$\EQNalign{ q^0_t  & = \beta^t {u_c(t) \over u_c(0)},    \EQN cham14;a \cr
   (1-\tau^n_t) w_t & = {u_\ell(t) \over u_c(t)}. \EQN cham14;b \cr}
$$
As before, we can derive the arbitrage condition \Ep{T_focR}, which now
reads
$$
{q^0_{t} \over q^0_{t+1}} = (1-\tau^k_{t+1}) r_{t+1} +1-\delta. \EQN T_focRq
$$
Profit maximization and factor market equilibrium imply
equations \Ep{cham5}.


\medskip
\noindent{\it Step 2.}   Substitute equations  \Ep{cham14} and $r_0=F_k(0)$
 into equation \Ep{T_bcPV2},
 so that we can write the household's budget constraint
as
$$ \sum_{t=0}^\infty \beta^t[u_c(t) c_t - u_\ell(t) n_t]
      - A = 0, \EQN cham15 $$
where $A$ is given by
$$ A = A(c_0, n_0,\tau^k_0,b_0)
               = u_c(0)
   \left\{[(1-\tau^k_0) F_k(0) + 1-\delta] k_0 + b_0 \right\}. \hskip.5cm\EQN cham16
$$


\medskip
\noindent{\it Step 3}.
The Ramsey problem is to choose an allocation to maximize expression \Ep{cham1}
subject to equation  \Ep{cham15} and the feasibility constraint
\Ep{cham2}. As before, we proceed by assuming that
government expenditures are small enough
that the problem has a convex constraint set
and that we can approach it using Lagrangian
methods.  In particular,
let $\Phi$ be a Lagrange multiplier on equation \Ep{cham15}
and define
$$ V(c_t, n_t, \Phi) = u(c_t,1-n_t)  + \Phi \left[ u_c(t) c_t
        -  u_\ell(t) n_t \right]. \EQN cham17 $$
Then form the Lagrangian
$$\EQNalign{
 J = &\sum_{t=0}^\infty \beta^t \left\{ V(c_t, n_t, \Phi)
     + \theta_t\left[ F(k_t,n_t) + (1-\delta) k_t     \right. \right.\cr
 & \left. \left. - c_t - g_t -
   k_{t+1} \right] \right\} \,-\, \Phi A , \EQN chamlag  \cr}
$$
where $\{\theta_t\}_{t=0}^\infty$ is a sequence of Lagrange
multipliers on the sequence of feasibility conditions \Ep{cham2}. For given $k_0$ and $b_0$,  we fix $\tau^k_0$ and maximize $J$
with respect to $\{c_t, n_t, k_{t+1} \}_{t=0}^\infty$.
First-order conditions for this problem are\NFootnote{Comparing the first-order
condition for $k_{t+1}$ to the earlier one in equation \Ep{T_RPk}, obtained
under Chamley's alternative formulation of the Ramsey problem, note that
the Lagrange multiplier $\theta_t$ is different across formulations.
Specifically, the
present specification of the objective function $V$ subsumes parts of the
household's present-value budget constraint. To bring out this difference,
a more informative notation would be to write $V_j(t,\Phi)$ for $j=c,n$ rather
than just $V_j(t)$.}
$$\eqalign{ c_t\rm{:}& \ \ V_c(t)  = \theta_t  , \ \ \ t \geq 1 \cr
            n_t\rm{:}& \ \ V_n(t)  = -\theta_t F_n(t),  \ \ \ t \geq 1 \cr
        k_{t+1}\rm{:}& \ \ \theta_{t} = \beta \theta_{t+1} [ F_k(t+1) + 1 -\delta],
                                                     \ \ \ t \geq 0 \cr
            c_0\rm{:}& \ \ V_c(0) = \theta_0 + \Phi A_c,  \cr
            n_0\rm{:}& \ \ V_n(0) = -\theta_0 F_n(0) + \Phi A_n.  \cr}$$
These conditions become
$$\EQNalign{ V_c(t) & = \beta V_c(t+1) [ F_k(t+1) + 1 -\delta ],  \ \ \ t \geq 1
      \EQN cham19;a \cr
         V_n(t) & = -V_c(t) F_n(t) , \ \ \ t \geq 1 \EQN cham19;b \cr
 V_c(0)-\Phi A_c & = \beta V_c(1) [ F_k(1) + 1 -\delta ],
      \EQN cham19;c \cr
         V_n(0) & =  \left[\Phi A_c - V_c(0)\right] F_n(0)  + \Phi A_n .
     \EQN cham19;d \cr}  $$
To these we add equations \Ep{cham2} and \Ep{cham15}, which we repeat
here for convenience:
\vskip-.8cm
$$ \EQNalign{ &c_t + g_t + k_{t+1} = F(k_t,n_t) + (1-\delta) k_t,
                                        \ \ \ t \geq 0 \EQN cham20;a  \cr
   & \sum_{t=0}^\infty \beta^t[u_c(t) c_t - u_\ell(t) n_t]
      - A  = 0. \EQN cham20;b \cr }  $$
We seek an allocation $\{c_t, n_t, k_{t+1}\}_{t=0}^\infty$, and a
multiplier $\Phi$ that satisfies the system of difference
equations formed by equations
\Ep{cham19}--\Ep{cham20}.\NFootnote{This system of nonlinear
equations can be solved iteratively. First, fix $\Phi$, and solve
equations \Ep{cham19} and \Ep{cham20;a} for an allocation.  Then
check the implementability condition \Ep{cham20;b}, and increase
or decrease $\Phi$ depending on whether the budget is in deficit
or surplus. Note that the multiplier $\Phi$ is nonnegative because
we are facing the constraint that the left-hand side of equation
\Ep{cham20;b} is {\it greater\/} than or equal to zero. That is,
we are constrained by the equilibrium outcome that households
fully exhaust their incomes and, hence, are not free to choose
households' expenditures strictly less than their incomes.}

\medskip
\noindent{\it Step 4:}
After an allocation has been found, obtain $q^0_t$
from equation  \Ep{cham14;a},  $r_t$ from equation \Ep{cham5;a},  $w_t$ from
equation \Ep{cham5;b},   $\tau^n_t$ from equation \Ep{cham14;b}, and finally
$\tau^k_t$ from equation \Ep{T_focRq}.
\index{optimal taxation!zero capital tax}
\subsection{Revisiting a zero capital tax}
  Consider the special case in which there
is a $T \geq 0$ for which $g_t = g$  for all $t \geq T$.
Assume that there exists a solution to the Ramsey problem and
that it converges to a  time-invariant allocation, so
that $c, n$, and $k$ are constant after some time.
Then because $V_c(t)$ converges to a constant, the
stationary version of equation \Ep{cham19;a}
implies
$$ 1 = \beta( F_k + 1 -\delta ).   \EQN cham21 $$
Now because $c_t$ is  constant in the limit, equation
  \Ep{cham14;a} implies that
$\left(q^0_{t} / q^0_{t+1}\right)$ $\rightarrow \beta^{-1}$ as $t \rightarrow
\infty$.  Then the  no-arbitrage condition for capital \Ep{T_focRq} becomes
$$ 1 = \beta[(1-\tau^k) F_k +1-\delta]. \EQN cham22 $$
Equalities \Ep{cham21} and \Ep{cham22} imply
that $\tau^k = 0$.
\index{optimal taxation!initial capital}
\section{Taxation of initial capital}
 Thus far, we have set $\tau^k_0$ at zero (or some other small
fixed number).  Now suppose that the government is free to choose
$\tau^k_0$.  The derivative of $J$ in equation \Ep{chamlag} with respect
to $\tau^k_0$ is
$$ {\partial J \over \partial \tau^k_0}
   =   \Phi u_c(0) F_k(0) k_0,  \EQN cham23 $$
which is strictly positive for all $\tau^k_0$ as long as $\Phi>0$.
The nonnegative Lagrange multiplier $\Phi$
measures the utility costs of raising government revenues
through distorting taxes.  Without distortionary taxation, a
competitive equilibrium would attain the first-best outcome
for the representative household, and $\Phi$ would be equal to
zero, so that  the household's (or equivalently, by Walras' Law,
 the government's) present-value budget constraint would not  constrain
 the Ramsey planner
beyond  the  technology constraints \Ep{cham2}.
In contrast, when the government has to use some of the tax rates
$\{\tau^n_t, \tau^k_{t+1}\}_{t=0}^\infty$, the multiplier $\Phi$ is
strictly positive and reflects the welfare cost of
the distorted margins, implicit in the present-value budget constraint
\Ep{cham20;b}, that govern the household's optimization behavior.

By raising $\tau^k_0$ and thereby increasing the revenues from
lump-sum taxation of $k_0$, the government reduces its need to
rely on future distortionary taxation, and hence the value of $\Phi$ falls.
In fact, the ultimate implication of condition \Ep{cham23} is that
the government should set $\tau^k_0$ high enough to drive $\Phi$
down to zero. In other words, the government should raise {\it all\/}
revenues through a time $0$ capital levy, then lend the proceeds
to the private sector and finance government expenditures
by using the interest from the loan; this would enable
the government to set $\tau^n_t = 0$ for all $t\geq0 $ and
$ \tau^k_t = 0 $ for all $t \geq 1$.

%SW% \NFootnote{The scheme may involve
%SW% $\tau^k_0>1$ for high values of $\{g_t\}_{t=0}^\infty$ and $b_0$.
%SW% However, such a scheme cannot be implemented if the household could
%SW%  avoid the tax
%SW% liability by not renting out its capital stock at time $0$. The
%SW% government would then be constrained to choose $\tau^k_0\leq 1$.
%SW%
%SW% In the rest of the chapter, we do not impose that $\tau^k_t\leq 1$.
%SW% If we were to do so, an extra constraint in the Ramsey problem
%SW% would be
%SW% $$
%SW% u_c(t) \geq \beta (1-\delta) u_c(t+1),
%SW% $$
%SW% which can be obtained by substituting equation  \Ep{cham14;a} into
%SW% equation \Ep{T_focRq}.}


\index{optimal taxation!incomplete taxation}%

\section{Nonzero capital tax due to incomplete taxation}
The result that the limiting capital tax should be zero
hinges on a complete set of flat-rate taxes. The consequences of incomplete
taxation are illustrated by Correia (1996),
 who introduces an additional production factor $z_t$ in fixed supply
$z_t=Z$ that cannot be taxed, $\tau^z_t=0$.\auth{Correia, Isabel H.}

The new production function $F(k_t,n_t,z_t)$ exhibits constant returns
to scale in all of its inputs. Profit maximization implies that the rental
price of the new factor equals its marginal product:
$$
p^z_t = F_z(t).
$$
The only change to the household's present-value budget constraint \Ep{T_bcPV2}
is that a stream of revenues is added to the right side:
$$
\sum_{t=0}^\infty q^0_t p^z_t Z.
$$

Following our scheme of constructing the Ramsey plan, step 2 yields the
following implementability condition:
$$
 \sum_{t=0}^\infty \beta^t \left\{u_c(t)[ c_t - F_z(t) Z]
           - u_\ell(t) n_t \right\}  - A = 0,      \EQN cham15z $$
where $A$ remains defined by equation \Ep{cham16}. In step 3 we formulate
$$\EQNalign{
V(c_t, n_t, k_t, \Phi) &= u(c_t,1-n_t)                           \cr
    &+ \Phi \left\{ u_c(t) [ c_t - F_z(t) Z] - u_\ell(t) n_t \right\}.
                                                   \EQN cham17z \cr}$$
In contrast to equation \Ep{cham17}, $k_t$ enters now as an argument
in $V$ because of the presence of the marginal product of the factor
$Z$ (but we have chosen to suppress the quantity $Z$ itself, since
it is in fixed supply).

Except for these changes of the functions $F$ and $V$, the Lagrangian
of the Ramsey problem is the same as equation \Ep{chamlag}. The first-order
condition with respect to $k_{t+1}$ is
$$
\theta_{t} = \beta V_k(t+1) + \beta \theta_{t+1} [ F_k(t+1) + 1 -\delta].
                                                       \EQN T_corrK
$$
Assuming the existence of a steady state, the stationary version of
equation \Ep{T_corrK} becomes
$$ 1 = \beta( F_k + 1 -\delta ) + \beta {V_k \over \theta}.
                                                         \EQN cham21z $$
Condition \Ep{cham21z} and the no-arbitrage condition for
capital \Ep{cham22} imply an optimal value for $\tau^k$:
$$
\tau^k = {-V_k \over \theta F_k } =
         {\Phi u_c Z \over \theta F_k } F_{zk}.
$$
As discussed earlier, in a second-best solution with distortionary
taxation, $\Phi>0$, so the limiting tax rate on capital is zero only if $F_{zk}=0$.
Moreover, the sign of $\tau^k$ depends on the
direction of the effect of capital on the marginal product of the untaxed
factor $Z$. If $k$ and $Z$ are complements, the limiting capital
tax is positive, and it is negative in the case where the two factors
are substitutes.

Other examples of a nonzero limiting capital tax are presented by
Stiglitz (1987) and Jones, Manuelli, and Rossi (1997),
who assume that two types of labor must be taxed at the same tax rate.  Once
again, the incompleteness of the tax system makes the optimal capital tax
depend on how capital affects the marginal products of the other factors.
\auth{Stiglitz, Joseph E.}  \auth{Jones, Larry E.}
\auth{Manuelli, Rodolfo}    \auth{Rossi, Peter E.}

\section{A stochastic economy}%
We now turn to optimal taxation in a stochastic version of our economy.
With the notation of chapter \use{recurge}, we follow the setups of
Zhu (1992) and Chari, Christiano, and Kehoe (1994).
The stochastic state $s_t$ at time $t$ determines
an exogenous shock both to the production function $F(\cdot,\cdot,s_t)$
and to government purchases $g_t(s_t)$. We use the history of events
$s^t$ to define history-contingent commodities:
 $c_t(s^t)$, $\ell_t(s^t)$, and $n_t(s^t)$ are the
household's consumption, leisure, and labor at time $t$ given history
$s^t$, and $k_{t+1}(s^t)$ denotes the capital stock carried over to
next period $t+1$. Following our earlier convention, $u_c(s^t)$
and $F_k(s^t)$ and so on denote the values
of the indicated objects at time $t$ for history $s^t$,
evaluated at an allocation to be understood from the
context.\auth{Zhu, Xiaodong}           \auth{Chari, V.V.}
\auth{Christiano, Lawrence J.} \auth{Kehoe, Patrick J.}

The household's preferences are ordered by
$$
\sum_{t=0}^\infty\ \ \sum_{s^t} \beta^t \pi_t(s^t) u[c_t(s^t), \ell_t(s^t)] .
                                                              \EQN TS_pref $$
The production function has constant returns to scale in labor and
capital. Feasibility requires that
$$\EQNalign{ c_t(s^t) &+ g_t(s_t) + k_{t+1}(s^t)   \cr
&= F[k_t(s^{t-1}),n_t(s^t),s_t] + (1-\delta) k_t(s^{t-1}).    \EQN TS_tech \cr}
 $$

\subsection{Government}
Given history $s^t$ at time $t$, the government finances its exogenous
purchase $g_t(s_t)$ and any debt obligation by levying flat-rate taxes on earnings
from capital at rate
$\tau^k_t(s^t)$ and from labor at rate $\tau^n_t(s^t)$, and by issuing
state-contingent debt. Let $b_{t+1}(s_{t+1}|s^t)$ be government indebtedness to
the private sector at the beginning of period $t+1$ if event $s_{t+1}$ is
realized. This state-contingent asset is traded in period $t$ at the price
$p_t(s_{t+1}|s^t)$, in terms of time $t$ goods. The government's budget
constraint becomes
$$\EQNalign{
 g_t(s_t) =  &\tau^k_t(s^t) r_t(s^t) k_t(s^{t-1})
               + \tau^n_t(s^t) w_t(s^t) n_t(s^t)                   \cr
             & + \sum_{s_{t+1}} p_t(s_{t+1} | s^t) b_{t+1}(s_{t+1} | s^t)
               - b_t(s_t | s^{t-1}),
                                                              \EQN TS_gov \cr}
$$
where $r_t(s^t)$ and $w_t(s^t)$ are the market-determined rental rate of capital
and the wage rate for labor, respectively.


\subsection{Households}
The representative household chooses $\{ c_t(s^t), n_t(s^t), k_{t+1}(s^t), b_{t+1}(s_{t+1}|s^t) \}_{t=0}^\infty$ to maximize
 expression
 \Ep{TS_pref} subject to the following
sequence of budget constraints:
$$\EQNalign{
&c_t(s^t) + k_{t+1}(s^t)
+ \sum_{s_{t+1}} p_t(s_{t+1} | s^t) b_{t+1}(s_{t+1} | s^t)     \cr
&= \left[1-\tau^k_t(s^t)\right] r_t(s^t) k_t(s^{t-1})
  +\left[1-\tau^n_t(s^t)\right] w_t(s^t) n_t(s^t)                   \cr
&\ + (1-\delta) k_t(s^{t-1}) + b_t(s_t | s^{t-1}) \quad \forall t.
         \EQN TS_bc \cr}
$$
The first-order conditions for this problem imply
$$\EQNalign{
{ u_\ell(s^t) \over u_c(s^t) } =& [1-\tau^n_t(s^t)] w_t(s^t),  \EQN TS_focX;a \cr
p_t(s_{t+1} | s^t) =& \beta { \pi_{t+1}(s^{t+1}) \over \pi_t(s^t) }
                        { u_c(s^{t+1}) \over u_c(s^{t}) }, \EQN TS_focX;b \cr
u_c(s^t) =& \beta E_{t} \Bigl\{ u_c(s^{t+1}) \cr
            & \cdot \left[(1-\tau^k_{t+1}(s^{t+1})) r_{t+1}(s^{t+1})
              +1-\delta\right] \Bigr\},    \hskip1cm         \EQN TS_focX;c \cr}
$$
where $E_{t}$ is the mathematical expectation conditional on
information available at time $t$, i.e., history $s^t$:
$$\EQNalign{
E_t x_{t+1}(s^{t+1}) &= \sum_{s^{t+1}\vert s^t} { \pi_{t+1}(s^{t+1}) \over \pi_t(s^t) }
                                             x_{t+1}(s^{t+1})  \cr
&=\sum_{s^{t+1}\vert s^t} \pi_{t+1}(s^{t+1} | s^t) x_{t+1}(s^{t+1}) \,,
\cr}
$$
where the summation over $s^{t+1} \vert s^t$ means that we sum
over all possible histories $\tilde s^{t+1}$ such that
$\tilde s^t=s^t$.

Corresponding to the no-arbitrage condition \Ep{T_focR} in the nonstochastic economy,
conditions \Ep{TS_focX;b} and \Ep{TS_focX;c} imply
$$
1 = \sum_{s_{t+1}} p_t(s_{t+1} | s^t)
    \left\{[1-\tau^k_{t+1}(s^{t+1})] r_{t+1}(s^{t+1})+1-\delta\right\}.  \EQN TS_arb
$$
And once again, this no-arbitrage condition can be obtained by
consolidating the budget constraints of two consecutive periods.
Multiply the time $t+1$ version of equation \Ep{TS_bc}
by $p_t(s_{t+1}|s^t)$ and
sum over all realizations $s_{t+1}$. The resulting expression can
be substituted into equation \Ep{TS_bc} by eliminating
$\sum_{s_{t+1}} p_t(s_{t+1} | s^t) b_{t+1}(s_{t+1} | s^t)$. To rule out
arbitrage transactions in capital and state-contingent assets,
the term multiplying $k_{t+1}(s^t)$ must be zero;  this approach amounts to
imposing condition \Ep{TS_arb}.
Similar no-arbitrage arguments were made in chapters \use{recurge} and
\use{assetpricing1}.

As before, by repeated substitution of one-period
budget constraints, we can obtain
the household's present-value budget constraint:
$$\EQNalign{
\sum_{t=0}^\infty\ \sum_{s^t} q^0_t(s^t) c_t(s^t) &=
\sum_{t=0}^\infty\ \sum_{s^t} q^0_t(s^t) [1-\tau^n_t(s^t)] w_t(s^t) n_t(s^t) \cr
   &+ \left[(1-\tau^k_0) r_0 + 1-\delta \right] k_0 + b_0,
                                                              \EQN TS_bcPV2  \cr}
$$
where we denote time $0$ variables by the time subscript $0$.
The price system $q^0_t(s^t)$ conforms to the following formula,
versions of which were displayed in chapter \use{recurge}:
$$
q^0_{t+1}(s^{t+1}) = p_t(s_{t+1}|s^t) q^0_t(s^t) = \beta^{t+1} \pi_{t+1}(s^{t+1})
                               {u_c(s^{t+1})  \over u_c(s^0)}.    \EQN TS_foci
$$
Alternatively, equilibrium price \Ep{TS_foci} can be computed from the
first-order conditions for maximizing expression \Ep{TS_pref} subject to
equation \Ep{TS_bcPV2} (and choosing the numeraire $q^0_0=1$). Furthermore, the no-arbitrage condition \Ep{TS_arb} can be expressed as
$$\EQNalign{
q^0_{t}(s^t) = &\sum_{s^{t+1}\vert s^t} q^0_{t+1}(s^{t+1})        \cr
  &\cdot \left\{[1-\tau^k_{t+1}(s^{t+1})] r_{t+1}(s^{t+1})+1-\delta\right\}.
                                                     \EQN TS_arbi \cr}
$$

In deriving the present-value budget constraint \Ep{TS_bcPV2},
we imposed two transversality conditions that specify that
for any infinite history $s^\infty$,
$$\EQNalign{
& \lim_{t \rightarrow +\infty} q^0_t(s^t) k_{t+1}(s^t) =0,                        \EQN TS_transv;a \cr
\noalign{\vskip.3cm}
&\lim_{t \rightarrow  +\infty} \sum_{s_{t+1}} q^0_{t+1}(\{s_{t+1},s^t\})
               b_{t+1}(s_{t+1} | s^t)=0,\EQN TS_transv;b \cr}
$$
where the limits are taken over sequences of histories $s^t$ contained in the
infinite history $s^\infty$.

\subsection{Firms}
The static maximization problem of the representative firm remains the
same.  Thus, in a competitive equilibrium, production factors
are paid their marginal products:
$$ \EQNalign{ r_t(s^t)  & = F_k(s^t)  , \EQN TS_firm;a \cr
              w_t(s^t)  & = F_n(s^t)  . \EQN TS_firm;b \cr }$$

\index{optimal taxation!indeterminacy of state-contingent debt and capital taxes}

\section{Indeterminacy of state-contingent debt and capital taxes}
Consider a feasible government policy
$\{g_t(s_t),\, \tau^k_t(s^t),\, \tau^n_t(s^t),\,
 b_{t+1}(s_{t+1}|s^t);\, \forall s^t$,
$s_{t+1}\}_{t\geq0}$ with  associated competitive allocation
$\{c_t(s^t), n_t(s^t), k_{t+1}(s^t); \forall s^t\}_{t\geq0}$.
Note that the labor tax is uniquely
determined by equations \Ep{TS_focX;a} and \Ep{TS_firm;b}.
However, there are infinitely many plans for state-contingent
debt and capital taxes that can implement a particular competitive
allocation.

Intuition for the indeterminacy of state-contingent debt and
capital taxes can be gleaned from the household's first-order condition
\Ep{TS_focX;c}, which states that capital tax rates affect the household's
intertemporal allocation by changing the current market value of after-tax
returns on capital. If a different set of capital taxes induces the same
current market value of after-tax returns on capital, then they will also
be consistent with the same competitive allocation. It remains
only to verify that the change of capital tax receipts in different states
can be offset by restructuring the government's
issue of state-contingent debt. Zhu (1992)
shows how such
feasible alternative policies can be constructed.\auth{Zhu, Xiaodong}

Let $\{\epsilon_t(s^t); \forall s^t\}_{t\geq0}$ be
a random process such that
$$
E_{t} u_c(s^{t+1}) \epsilon_{t+1}(s^{t+1}) r_{t+1}(s^{t+1})=0.      \EQN TS_epsilon
$$
We can then construct an alternative policy for capital taxes and
state-contingent debt,
$\{\hat \tau^k_t(s^t), \hat b_{t+1}(s_{t+1}|s^t);
\forall s^t, s_{t+1}\}_{t\geq0}$, as follows:
$$\EQNalign{ \hat \tau^k_0 &= \tau^k_0,\EQN TS_alt;a \cr
\hat \tau^k_{t+1}(s^{t+1}) &= \tau^k_{t+1}(s^{t+1}) + \epsilon_{t+1}(s^{t+1}),
 \EQN TS_alt;b \cr
\hat b_{t+1}(s_{t+1}|s^t) &= b_{t+1}(s_{t+1}|s^t) +
 \epsilon_{t+1}(s^{t+1}) r_{t+1}(s^{t+1}) k_{t+1}(s^{t}),\hskip1.25cm  \EQN TS_alt;c \cr}
$$
for $t\geq0$.
Compared to the original fiscal policy, we can verify that this alternative policy
does not change the following:
\smallskip
\noindent{\bf 1.} The household's intertemporal consumption choice, governed by
first-order condition \Ep{TS_focX;c}.
\smallskip
\noindent{\bf 2.} The current market value of all government debt issued at time $t$,
when discounted with the equilibrium expression for
$p_t(s_{t+1}|s^t)$ in equation \Ep{TS_focX;b}.
\smallskip
\noindent{\bf 3.} The government's revenue from capital taxation net of maturing
government debt in any state $s^{t+1}$.
\medskip

\noindent Thus, the alternative policy is feasible and leaves the
competitive allocation unchanged.

Since there are infinitely many ways of constructing sequences of
random variables $\{\epsilon_t(s^t)\}$ that satisfy equation \Ep{TS_epsilon},
it follows that the competitive allocation can be implemented by
many different plans for capital taxes and state-contingent debt.
It is instructive to consider two special cases where there is no
uncertainty one period ahead about one of the two policy instruments.
We first take the case of risk-free one-period bonds. In period
$t$, the government issues bonds that promise to pay
$\bar b_{t+1}(s^t)$ at time $t+1$ with certainty. Let the amount
of bonds be such that their present market value is the same as that
for the original fiscal plan,
$$
\sum_{s_{t+1}} p_t(s_{t+1} | s^t) \bar b_{t+1}(s^t)
= \sum_{s_{t+1}} p_t(s_{t+1} | s^t) b_{t+1}(s_{t+1} | s^t).
$$
After invoking the equilibrium expression for prices \Ep{TS_focX;b},
we can solve for the constant $\bar b_{t+1}(s^t)$
$$
\bar b_{t+1}(s^t) = { E_{t} u_c(s^{t+1})
b_{t+1}(s_{t+1}|s^{t})  \over
E_{t} u_c(s^{t+1})  }.                                        \EQN TS_rfbond1
$$
The change in capital taxes needed to offset this shift to risk-free
bonds is then implied by equation \Ep{TS_alt;c}:
$$
\epsilon_{t+1}(s^{t+1}) = { \bar b_{t+1}(s^t) - b_{t+1}(s_{t+1}|s^{t})
\over r_{t+1}(s^{t+1}) k_{t+1}(s^t) }.                               \EQN TS_rfbond2
$$
We can check that equations
 \Ep{TS_rfbond1} and \Ep{TS_rfbond2} describe a permissible policy by
substituting these expressions into equation \Ep{TS_epsilon} and verifying
that the restriction is indeed satisfied.

Next, we examine a policy where the capital tax is not contingent on the
realization of the current state but is already set in the previous
period. Let $\bar \tau_{t+1}(s^t)$ be the capital tax rate in period
$t+1$, conditional on information at time $t$. We choose $\bar \tau_{t+1}(s^t)$ so
that the household's first-order condition \Ep{TS_focX;c} is
un\-affected:
$$\EQNalign{
&E_{t} \left\{ u_c(s^{t+1})
            \left[(1-\bar \tau^k_{t+1}(s^{t})) r_{t+1}(s^{t+1})
              +1-\delta\right] \right\}                        \cr
&=E_{t} \left\{ u_c(s^{t+1})
            \left[(1-\tau^k_{t+1}(s^{t+1})) r_{t+1}(s^{t+1})
              +1-\delta\right] \right\} ,                       \cr}
$$
which gives
$$
\bar \tau^k_{t+1}(s^t) = { E_{t} u_c(s^{t+1})
\tau^k_{t+1}(s^{t+1}) r_{t+1}(s^{t+1}) \over
E_{t} u_c(s^{t+1}) r_{t+1}(s^{t+1}) }.                        \EQN TS_fixtau
$$
Thus, the alternative policy in equations \Ep{TS_alt} with capital taxes known
one period in advance is accomplished by setting
$$
\epsilon_{t+1}(s^{t+1}) = \bar \tau^k_{t+1}(s^t) - \tau^k_{t+1}(s^{t+1}).
$$
\index{Ramsey problem!uncertainty}
\section{The Ramsey plan under uncertainty}
We now ask what competitive allocation should be chosen by a benevolent
government; that is, we solve the Ramsey problem for the stochastic economy.
The computational strategy is in principle the same given in our recipe
for a nonstochastic economy.

Step 1, in which we use private first-order conditions to solve for prices and
taxes in terms of the allocation, has already been accomplished with
equations
 \Ep{TS_focX;a}, \Ep{TS_foci}, \Ep{TS_arbi}, and \Ep{TS_firm}. In step 2,
we use these expressions to eliminate prices and taxes from the
household's present-value budget constraint \Ep{TS_bcPV2}, which
leaves us with
$$ \sum_{t=0}^\infty \ \sum_{s^t} \beta^t \pi_t(s^t)
         [u_c(s^t) c_t(s^t) - u_\ell(s^t) n_t(s^t)]
      - A = 0,                                           \EQN TS_cham15 $$
where $A$ is still given by equation
 \Ep{cham16}. Proceeding to step 3, we define
$$\EQNalign{
 V\left[c_t(s^t), n_t(s^t), \Phi\right] = & u[c_t(s^t),1-n_t(s^t)]  \cr
+ & \Phi \left[ u_c(s^t) c_t(s^t)
        -  u_\ell(s^t) n_t(s^t) \right],  \hskip1cm          \EQN TS_cham17 \cr}$$
where $\Phi$ is a Lagrange multiplier on equation \Ep{TS_cham15}.
Then form the Lagrangian
\offparens
$$\EQNalign{
 J  = \sum_{t=0}^\infty\ \sum_{s^t} &\beta^t \pi_t(s^t) \Bigl\{
 V[c_t(s^t), n_t(s^t), \Phi]  \cr
 & + \theta_t(s^t)
\Bigl[ F\left(k_t(s^{t-1}),n_t(s^t),s_t\right)
  + (1-\delta) k_t(s^{t-1}) \cr
 & - c_t(s^t) - g_t(s_t) -
 k_{t+1}(s^t) \Bigr] \Bigr\} \,-\, \Phi A ,\EQN TS_chamlag  \cr}
$$
\autoparens
where $\{\theta_t(s^t); \forall s^t\}_{t\geq0}$ is a sequence of Lagrange
multipliers.  For given $k_0$ and $b_0$,  we fix $\tau^k_0$ and maximize $J$
with respect to $\{c_t(s^t), %%%%%\hfil \break
n_t(s^t),
 k_{t+1}(s^t); \forall s^t \}_{t\geq0}$.

The first-order conditions for the Ramsey problem are
$$\eqalign{ c_t(s^t)\rm{:}& \ \ V_c(s^t)  = \theta_t(s^t)  , \hskip3.75cm t \geq 1; \cr
            n_t(s^t)\rm{:}& \ \ V_n(s^t)  = -\theta_t(s^t) F_n(s^t),  \hskip2.5cm t \geq 1; \cr
            k_{t+1}(s^t)\rm{:}& \ \ \theta_t(s^t) = \beta \sum_{s^{t+1}\vert s^t}
                         {\pi_{t+1}(s^{t+1}) \over \pi_t(s^t) }
                         \theta_{t+1}(s^{t+1}) \cr
                          & \hskip3cm \cdot[ F_k(s^{t+1}) + 1 -\delta],
                                                     \ \ \  t \geq 0; \cr}$$
where we have left out the conditions for $c_0$ and $n_0$, which are
different because they include terms related to the initial stocks of
capital and bonds.
The first-order conditions for the problem imply, for $t\geq1$,
$$\EQNalign{
V_c(s^t) & = \beta E_t V_c(s^{t+1}) [ F_k(s^{t+1}) + 1 -\delta ],
                                                           \EQN TS_cham19;a \cr
         V_n(s^t) & = -V_c(s^t) F_n(s^t) .                 \EQN TS_cham19;b \cr}
$$
These expressions reveal an interesting property of the Ramsey allocation.
If the stochastic process $s$ is Markov, equations
 \Ep{TS_cham19} suggest that the allocations from period $1$ onward can
be described by time-invariant allocation rules $c(s,k)$, $n(s,k)$,
and $k'(s,k)$.\NFootnote{To emphasize
that the second-best allocation depends critically on
the extent to which the government has to resort to
distortionary taxation, we might want to include the
constant $\Phi$ as an explicit argument in $c(s,k)$,
$n(s,k)$, and $k'(s,k)$. }
\index{optimal taxation!{\it ex ante\/} capital tax varies around zero}
\index{capital tax}
\section{Ex ante capital tax varies around zero}
In a nonstochastic economy, we proved that if the equilibrium
converges to a steady state, then the optimal limiting capital
tax is zero.  The counterpart to a steady state in a stochastic
economy is a stationary equilibrium.  Therefore, we now assume
that the process on $s$ follows a Markov process with transition
probabilities $\pi(s'|s)\equiv{\rm Prob}(s_{t+1}=s'|s_t=s)$.
As noted in the previous section, this assumption implies that
the allocation rules are time-invariant functions of $(s,k)$.
If the economy converges to a stationary equilibrium, the stochastic
process $\{s_t,k_t\}$ is a stationary, ergodic Markov process on
the compact set ${\bf S} \times  [0, \bar k]$ where ${\bf S}$ is a
finite set of possible realizations for $s_t$ and
$\bar k$ is an upper bound on the capital stock.\NFootnote{An upper bound
on the capital stock can be constructed as follows:
$$
\bar k = \max\{\bar k(s): F[\bar k(s),1,s]=\delta \bar k(s); s\in {\bf S}\}.$$
\vskip-1cm}

Because of the indeterminacy of state-contingent government debt and
capital taxes, it is not possible uniquely to characterize
a stationary distribution of realized capital tax rates, but we can study
the {\it ex ante capital tax rate} defined as
$$
\bar \tau^k_{t+1}(s^t) = {\sum_{s_{t+1}} p_t(s_{t+1}|s^t)
\tau^k_{t+1}(s^{t+1}) r_{t+1}(s^{t+1}) \over
\sum_{s_{t+1}} p_t(s_{t+1}|s^t) r_{t+1}(s^{t+1}) }.            \EQN TS_etau
$$
That is, the {\it ex ante\/} capital tax rate is the ratio of current market
value of taxes on capital income to the present market value of
capital income. After invoking the equilibrium price of equation
 \Ep{TS_focX;b},
we see that this
expression is identical to equation \Ep{TS_fixtau}. Recall that equation \Ep{TS_fixtau}
resolved the indeterminacy of the Ramsey
plan by pinning down a unique fixed capital tax rate for period $t+1$
conditional on information at time $t$. It follows that the alternative
interpretation
of $\bar \tau^k_{t+1}(s^t)$ in equation \Ep{TS_etau} as the {\it ex ante\/} capital tax
rate offers a unique measure across the multiplicity of capital tax
schedules under the Ramsey plan. Moreover, it is quite intuitive that one
way for the government to tax away, in present value terms, a fraction
$\bar \tau^k_{t+1}(s^t)$ of next period's capital income is to set
a constant tax rate exactly equal to that number.

Let $P^\infty(\cdot)$ be the probability measure over the outcomes in
such a stationary equilibrium.  We now state the proposition of Zhu (1992)
that the {\it ex ante\/} capital tax rate in a stationary equilibrium either
equals zero or varies around zero.\auth{Zhu, Xiaodong}

\medskip
\medskip\noindent{\sc Proposition 2:}  If there exists a stationary Ramsey allocation,
the {\it ex ante\/} capital tax rate is such that
\smallskip
\noindent(a) either $P^\infty(\bar \tau^k_t=0)=1$, or
           $P^\infty(\bar \tau^k_t>0)>0$ and $P^\infty(\bar \tau^k_t<0)>0$;
\vskip.1cm
\noindent(b) $P^\infty(\bar \tau^k_t=0)=1$ if and only if
               $P^\infty[V_c(c_t,n_t,\Phi)/ u_c(c_t,\ell_t) =\Lambda]=1$
for some constant $\Lambda$.

\medskip\noindent
A sketch of the proof is provided in the next subsection.
 Let us just add here that the two
possibilities with respect to the {\it ex ante\/} capital tax rate are not vacuous.
One class of utilities that imply $P^\infty(\bar \tau^k_t=0)=1$ is
$$
u(c_t,\ell_t) = {c_t^{1-\sigma} \over 1-\sigma} + v(\ell_t),
$$
for which the ratio $V_c(c_t,n_t,\Phi)/u_c(c_t,\ell_t)$ is equal
to $[1+\Phi(1-\sigma)]$, which plays the role of the constant $\Lambda$
required by Proposition 2. Chari, Christiano, and Kehoe (1994)
solve numerically for Ramsey plans when the preferences do not satisfy
this condition. In their simulations, the {\it ex ante\/} tax on capital income
remains approximately equal to zero.\auth{Chari, V.V.} \auth{Christiano, Lawrence J.}
\auth{Kehoe, Patrick J.}

To revisit the result on the optimality of
a zero capital tax in a nonstochastic economy, it is trivially true that
the ratio $V_c(c_t,n_t,\Phi)/u_c(c_t,\ell_t)$ is constant in a nonstochastic
steady state.  In a stationary equilibrium of a stochastic economy,
Proposition 2 extends this result: for some utility functions, the
Ramsey plan prescribes a zero {\it ex ante\/} capital tax rate that can be
implemented by setting a zero tax on capital income. But except for such
special classes of preferences, Proposition 2 states that the {\it ex ante\/}
capital tax rate should fluctuate around zero, in the sense that
$P^\infty(\bar \tau^k_t>0)>0$ and $P^\infty(\bar \tau^k_t<0)>0$.

\subsection{Sketch of the proof of Proposition 2}
Note from equation \Ep{TS_etau} that $\bar \tau^k_{t+1}(s^t) \geq (\leq) \ 0$ if and
only if
$$
\sum_{s_{t+1}} p_t(s_{t+1}|s^t) \tau^k_{t+1}(s^{t+1}) r_{t+1}(s^{t+1}) \geq (\leq)  \ 0,
$$
which, together with equation \Ep{TS_arb},  implies
$$
1 \leq (\geq) \sum_{s_{t+1}} p_t(s_{t+1} | s^t)
    \left[r_{t+1}(s^{t+1})+1-\delta\right].
$$
Substituting equations
 \Ep{TS_focX;b} and \Ep{TS_firm;a} into this expression yields
$$
u_c(s^{t}) \leq (\geq) \beta E_t u_c(s^{t+1})
                           \left[F_k(s^{t+1})+1-\delta\right]   \EQN TS_p1
$$
if and only if $\bar \tau^k_{t+1}(s^t) \geq   (\leq) \ 0$.

Define
$$
H(s^t) \equiv { V_c(s^t) \over u_c(s^t) } .                      \EQN TS_p2
$$
Using equation \Ep{TS_cham19;a}, we have
$$
u_c(s^t) H(s^t) = \beta E_t u_c(s^{t+1}) H(s^{t+1}) [ F_k(s^{t+1}) + 1 -\delta ].
                                                                 \EQN TS_p3
$$
By formulas
 \Ep{TS_p1} and \Ep{TS_p3}, $\bar \tau^k_{t+1}(s^t) \geq (\leq) \ 0$ if and
only if
$$
H(s^t) \geq (\leq)  { E_t \omega(s^{t+1}) H(s^{t+1}) \over
                       E_t \omega(s^{t+1})                 },    \EQN TS_p4
$$
where $\omega(s^{t+1}) \equiv  u_c(s^{t+1}) [ F_k(s^{t+1}) + 1 -\delta ]$.

Since a stationary Ramsey equilibrium has time-invariant allocation rules
$c(s,k)$, $n(s,k)$, and $k'(s,k)$, it follows that $\bar \tau^k_{t+1}(s^t)$,
$H(s^t)$, and $\omega(s^t)$ can also be expressed as functions of $(s,k)$.
The stationary version of expression
 \Ep{TS_p4} with transition probabilities
$\pi(s'|s)$ becomes
$$\eqalign{
\bar \tau^k(s,k) &\geq (\leq) \ 0 \hskip.5cm {\hbox{\rm if and only if}} \cr
H(s,k) &\geq (\leq) \;\; {  \sum_{s'}  \pi(s'|s)
                        \omega[s',k'(s,k)] H[s',k'(s,k)]  \over
                        \sum_{s'} \pi(s'|s) \omega[s',k'(s,k)] }  \cr
                         & \hskip1.5cm    \equiv \Gamma H(s,k). \cr} \EQN TS_p5
$$
Note that the operator $\Gamma$ is a weighted average of $H[s',k'(s,k)]$
and that it has the property that $\Gamma H^*= H^*$
for any constant $H^*$.

Under some regularity conditions, $H(s,k)$ attains a minimum $H^-$ and a
maximum $H^+$ in the stationary equilibrium. That is, there exist
equilibrium states $(s^-,k^-)$ and  $(s^+,k^+)$
such that
$$\EQNalign{
& P^\infty\left[H(s,k)\geq H^-\right] = 1,       \EQN TS_p6;a \cr
& P^\infty\left[H(s,k)\leq H^+\right] = 1,       \EQN TS_p6;b \cr}
$$
where
$H^-=H(s^-,k^-)$ and
$H^+=H(s^+,k^+)$.
We will now show that if
$$P^\infty\left[H(s,k)\geq \Gamma H(s,k) \right] = 1,       \EQN TS_p7;a
$$
or
$$P^\infty\left[H(s,k)\leq \Gamma H(s,k) \right] = 1,       \EQN TS_p7;b
$$
then there must exist a constant $H^*$ such that
$$P^\infty\left[H(s,k) = H^* \right] = 1.                   \EQN TS_p7;c
$$
First, take equation \Ep{TS_p7;a} and consider the state
$(s,k)=(s^-,k^-)$ that is associated with a set of
possible states in the next period, $\{s',k'(s,k); \forall s'\in {\bf S}\}$.
By equation \Ep{TS_p6;a}, $H(s',k')\geq H^-$, and since $H(s,k)=H^-$,
condition \Ep{TS_p7;a} implies that $H(s',k')=H^-$.
We can repeat the same argument for each
$(s',k')$, and thereafter for the equilibrium states that they
map into, and so on.
Thus, using the ergodicity of $\{s_t, k_t\}$, we obtain equation \Ep{TS_p7;c}
with $H^*=H^-$.
A similar reasoning can be applied to equation \Ep{TS_p7;b}, but we now
use $(s,k)=(s^+,k^+)$ and equation \Ep{TS_p6;b} to show that equation \Ep{TS_p7;c}
is implied.

By the correspondence in expression \Ep{TS_p5} we have established part (a) of
Proposition 2. Part (b) follows after recalling
definition \Ep{TS_p2}; the constant $H^*$ in equation
 \Ep{TS_p7;c} is the
sought-after $\Lambda$.
\index{optimal taxation!labor tax smoothing}
\auth{Lucas, Robert E., Jr.} \auth{Stokey, Nancy L.}
\section{A stochastic economy without capital}\label{Lucas-Stokey}%
To gain some insight into optimal tax policies, we consider several
examples of stochastic processes of  government expenditures to be financed in a model without
physical capital. The technology is now described by
$$
c_t(s^t) + g_t(s_t) = n_t(s^t).                        \EQN TSs_tech
$$
Since one unit of labor yields one unit of output, the competitive equilibrium
wage is  $w_t(s^t)=1$. The model is otherwise identical to the
previous framework.  This very model is analyzed by
Lucas and Stokey (1983).\NFootnote{The optimal tax policy is in general time
inconsistent, as studied in chapter \use{fiscalmonetary} and as
indicated by the preceding discussion about taxation of initial capital.
 However, Lucas and Stokey (1983) show that the optimal tax policy
in the model without physical capital can be made time consistent
if the government can issue debt at all maturities (and so is
not restricted to issue only one-period debt as in our formulation).
There exists a period-by-period strategy for structuring a term
structure of history-contingent claims that preserves the initial
Ramsey allocation
$\{c_t(s^t), n_t(s^t); \forall s^t\}_{t\geq0}$
as the Ramsey allocation for the continuation economy.  By induction,
the argument extends to subsequent periods.
Alvarez, Kehoe and Neumeyer (2004)
apply the argument to the maturity structure of both real and
{\it nominal\/} bonds in a monetary economy. For a class of economies where the
Friedman rule of setting nominal interest rates to zero is optimal under
commitment, they show that optimal monetary and fiscal policies are time consistent.
When the Friedman rule is optimal, households are satiated with money balances
-- as if money has disappeared -- so that the economy is equivalent to a real
economy with one consumption good and labor.}
\auth{Alvarez, Fernando} \auth{Kehoe, Patrick J.} \auth{Neumeyer, Pablo Andr\'es}


%Persson, Persson, and Svensson (1988)
%apply the argument to the maturity structure of both real and
%{\it nominal\/} bonds in a monetary economy.}
%\auth{Persson, Mats} \auth{Persson, Torsten} \auth{Svensson, Lars E.O.}

 The household's present-value budget constraint is
given by equation \Ep{TS_bcPV2} after we delete
the part involving physical capital.
After using conditions \Ep{TS_foci} and \Ep{TS_focX;a} to express prices and taxes as functions of the
allocation,
the implementability condition, equation \Ep{TS_cham15},
becomes
$$ \sum_{t=0}^\infty\  \sum_{s^t} \beta^t \pi_t(s^t)
         [u_c(s^t) c_t(s^t) - u_\ell(s^t) n_t(s^t)]
      - u_c(s^0) b_0 = 0.                                    \EQN TSs_cham15 $$
We then form the Lagrangian in the same way as before.
 After writing out the derivatives
$V_c(s^t)$ and $V_n(s^t)$,
 the first-order
conditions of this Ramsey problem are
\vfil\eject$$\EQNalign{
c_t(s^t)\rm{:}& \ \ \ (1+\Phi) u_c(s^t) + \Phi \left[u_{cc}(s^t) c_t(s^t)
        -  u_{\ell c}(s^t) n_t(s^t) \right]    \hskip1.5cm                     \cr
&\hskip3cm - \theta_t(s^t) = 0, \hskip1cm t \geq 1;
                                                            \EQN TSs_foc;a \cr
n_t(s^t)\rm{:}& \ \ -(1+\Phi) u_{\ell}(s^t) - \Phi \left[u_{c\ell}(s^t) c_t(s^t)
        -  u_{\ell \ell}(s^t) n_t(s^t) \right]                        \cr
&\hskip3cm  + \theta_t(s^t) = 0, \hskip1cm t \geq 1;
                                                            \EQN TSs_foc;b \cr
c_0(s^0, b_0)\rm{:}& \ \ \ (1+\Phi) u_c(s^0) + \Phi \left[u_{cc}(s^0) c_0(s^0)
        -  u_{\ell c}(s^0) n_0(s^0) \right]                        \cr
&\hskip3cm  - \theta_0(s^0)
        - \Phi u_{cc}(s^0) b_0 = 0;
                                                            \EQN TSs_foc;c \cr
n_0(s^0, b_0)\rm{:}& \ \ -(1+\Phi) u_{\ell}(s^0) - \Phi \left[u_{c\ell}(s^0) c_0(s^0)
        -  u_{\ell \ell}(s^0) n_0(s^0) \right]                        \cr
&\hskip3cm  + \theta_0(s^0)
        + \Phi u_{c \ell}(s^0) b_0 = 0.
                                                            \EQN TSs_foc;d \cr}
$$
In equations \Ep{TSs_foc;c} and \Ep{TSs_foc;d} it is to be understood that $\ell(s^0)$ and
$c(s^0)$ are  functions of both $s^0$ and $b_0$.
We retain our assumption that the government  commits to a policy at time $0$.

To uncover an important property of the optimal allocation for $t\geq1$,
it is instructive to use first-order conditions
\Ep{TSs_foc;a} and \Ep{TSs_foc;b}
to elminate the multiplier $\theta_t(s^t)$:
\offparens
$$\EQNalign{
(1+\Phi) u_c(&c,1-c-g) + \Phi \bigl[c u_{cc}(c,1-c-g)\cr
     &   -  (c+g) u_{\ell c}(c,1-c-g) \bigr]                           \cr
=(1+\Phi) u_{\ell}(&c,1-c-g) + \Phi \bigl[c u_{c\ell}(c,1-c-g) \cr
     &   -  (c+g) u_{\ell \ell}(c,1-c-g)  \bigr],   \EQN TS_barg       \cr}
$$
\autoparens
where we have also invoked the resource constraints \Ep{TSs_tech} and
$\ell_t(s^t)+n_t(s^t)=1$. We have  suppressed the time
subscript and the index $s^t$ for the quantities of consumption,
leisure, and government purchases in order to highlight a key
property of the Ramsey allocation. In particular, if
the quantities of government purchases are equal after two
histories $s^t$ and $\tilde s^j$ for $t,j\geq0$, i.e., if $g_t(s_t) =
g_j(\tilde s_j)=g$, then it follows from equation \Ep{TS_barg}
that the optimal choices of consumption and leisure,
$(c_t(s^t),\ell_t(s^t))$ and $(c_j(\tilde s^j),\ell_j(\tilde
s^j))$, must be identical. Hence,
the optimal allocation is a function  of the current realized
quantity of government purchases $g$  only and does {\it not} depend
on the specific history leading up to that outcome. This outcome is connected to an analogous history
independence of the competitive equilibrium allocation with complete markets
in chapter \use{recurge}.


Still taking  $\Phi$ temporarily as given, it remains to compute $c_0(s^0, b_0)$
and $n_0(s^0, b_0)$ from  first-order conditions \Ep{TSs_foc;c} and \Ep{TSs_foc;d}, respectively.
While for $t \geq 1$, $c$ and $n$ depend on  the time $t$ realization of $g$ only,
for $t=0$, $c$ and $g$ also depends on the government's initial debt $b_0$.
Thus, while $b_0$ influences $c_0$ and $n_0$,  $b_t$ does not
appear as an independent variable influencing $c_t$ and $n_t$ at $t \geq 1$.  The absence of $b_t$ as a determinant of
the Ramsey allocation for $t \geq 1$ but not for $t=0$
is a tell-tale sign of  time inconsistency of a Ramsey plan, a point to which we shall
return below.


% The following preliminary calculations will be useful in shedding further
% light on
% optimal tax policies for some  examples of government expenditure streams.
Now what about $\Phi$? It has to take a value that assures
that  the consumer's and the government's  budget constraints are both satisfied
at a candidate Ramsey allocation and price system associated with that $\Phi$.
To see how this requirement restricts $\Phi$, we reason as follows.
First, substitute equations
 \Ep{TS_focX;a} and \Ep{TSs_tech} into equation \Ep{TSs_cham15} to
get
$$
\sum_{t=0}^\infty\  \sum_{s^t} \beta^t \pi_t(s^t)
         u_c(s^t) \left[ \tau^n_t(s^t) n_t(s^t) - g_t(s_t)\right]
    - u_c(s^0) b_0 = 0.                                  \EQN TSs_calc1 $$
Then multiplying equation
 \Ep{TSs_foc;a} by $c_t(s^t)$ and equation \Ep{TSs_foc;b} by $n_t(s^t)$
and summing, we find
$$\EQNalign{
&(1+\Phi) \left[ c_t(s^t) u_c(s^t) - n_t(s^t) u_{\ell}(s^t) \right]     \cr
&+ \Phi \left[ c_t(s^t)^2 u_{cc}(s^t) - 2 n_t(s^t) c_t(s^t) u_{\ell c}(s^t)
 + n_t(s^t)^2 u_{\ell \ell}(s^t) \right]                              \cr
&- \theta_t(s^t) \left[c_t(s^t) - n_t(s^t)\right] = 0, \hskip1cm t \geq 1.
                                        \hskip1cm              \EQN TSs_calc2;a \cr}
$$
Similarly, multiplying equation
 \Ep{TSs_foc;c} by $[c_0(s^0)-b_0]$ and equations \Ep{TSs_foc;d} by $n_0(s^0)$
and summing, we obtain
$$\EQNalign{
&(1+\Phi) \left\{ \left[ c_0(s^0)-b_0\right] u_c(s^0)
                                         - n_0(s^0) u_{\ell}(s^0) \right\}     \cr
&+ \Phi \Bigl\{ \left[c_0(s^0)-b_0\right]^2 u_{cc}(s^0) - 2 n_0(s^0)
                \left[c_0(s^0)-b_0\right] u_{\ell c}(s^0)
                              \cr
& + n_0(s^0)^2 u_{\ell \ell}(s^0) \Bigr\}- \theta_0(s^0)
\left[c_0(s^0) - b_0 - n_0(s^0)\right] = 0. \hskip1cm                \EQN TSs_calc2;b \cr}
$$
Note that since the utility function is strictly concave, the quadratic forms multiplying $\Phi$ appearing
in equations \Ep{TSs_calc2;a} and  \Ep{TSs_calc2;b} are both negative.\NFootnote{To see that the quadratic
term in equation
 \Ep{TSs_calc2;a} is negative, complete the square by adding and subtracting
the quantity $n^2 u_{\ell c}^2/u_{cc}$ (where we have suppressed the time
subscript and the argument $s^t$):
$$\EQNalign{
&c^2 u_{cc} - 2 n c u_{\ell c} + n^2 u_{\ell \ell}
+  n^2 {u_{\ell c}^2\over u_{cc}} - n^2 {u_{\ell c}^2\over u_{cc}}    \cr
&= u_{cc} \left( c^2 - 2 n c {u_{\ell c} \over u_{cc}}
   + n^2 {u_{\ell c}^2 \over u_{cc}^2}\right)
+  \left(u_{\ell \ell} - {u_{\ell c}^2\over u_{cc}}\right) n^2        \cr
&= u_{cc} \left( c - {u_{\ell c} \over u_{cc}} n\right)^2
+  {u_{cc} u_{\ell \ell} - u_{\ell c}^2\over u_{cc}} n^2.             \cr}
$$
Since the conditions for a strictly concave $u$ are
$u_{cc}<0$ and $u_{cc} u_{\ell \ell} - u_{\ell c}^2>0$, it follows
immediately that the quadratic term in equation \Ep{TSs_calc2;a} is negative.
The same argument applies to the quadratic term in equation \Ep{TSs_calc2;b}.}
Finally, multiplying equation \Ep{TSs_calc2;a} by $\beta^t \pi_t(s^t)$, summing
over $t$ and $s^t$, and adding equation \Ep{TSs_calc2;b}, we find that
$$\EQNalign{
&(1+\Phi)  \left( \sum_{t=0}^\infty\  \sum_{s^t} \beta^t \pi_t(s^t)
\left[ c_t(s^t) u_c(s^t) - n_t(s^t) u_{\ell}(s^t) \right] - u_c(s^0) b_0 \right) \cr
&+ \Phi Q - \sum_{t=0}^\infty\  \sum_{s^t} \beta^t \pi_t(s^t)
    \theta_t(s^t) \left[c_t(s^t) - n_t(s^t)\right] + \theta_0(s^0) b_0 = 0,
                                                                              \cr}
$$
where $Q$ is the sum of negative (quadratic) terms.
Using equations \Ep{TSs_cham15} and
 \Ep{TSs_tech}, we arrive at
$$\EQNalign{
& \Phi Q + \sum_{t=0}^\infty \ \sum_{s^t} \beta^t \pi_t(s^t)
    \theta_t(s^t) g_t(s_t) + \theta_0(s^0) b_0 = 0.
                                                            \EQN TSs_calc3 \cr}
$$


We can use
equation  \Ep{TSs_calc3} to increase  our understanding of the Lagrange
multiplier $\Phi$ on
the household's present value budget constraint and how it relates
to
shadow values associated with the economy's resource constraints
 $\{\theta_t(s^t); \forall s^t\}_{t\geq0}$.
We first examine circumstances in which
$\Phi$  equals zero. Setting $\Phi=0$  in equations \Ep{TSs_foc} and
\Ep{TSs_calc3} yields
$$
u_c(s^t) = u_{\ell}(s^t) = \theta_t(s^t), \hskip1.5cm t \geq 0;   \EQN TSs_Phi0
$$
and, thus,
$$
\sum_{t=0}^\infty \ \sum_{s^t} \beta^t \pi_t(s^t) u_c(s^t)
g_t(s_t) + u_c(s^0) b_0 = 0.
$$
Dividing this expression by $u_c(s^0)$ and using equation \Ep{TS_foci},
we find that
$$
\sum_{t=0}^\infty\ \sum_{s^t} q^0_t(s^t) g_t(s_t) = - b_0,
$$
which asserts that    the present value of
government expenditures equals the government's initial claims $-b_0$ against
the private sector. That means that  the Lagrange multiplier $\Phi$ is zero so that
 neither the household's nor the government's present-value budget constraint puts
additional constraints on welfare maximization beyond those inherent
in the physical  technology. The reason
is that the government does not have to resort to  distorting
taxes, as can be seen from conditions \Ep{TS_focX;a} and \Ep{TSs_Phi0},
which imply $\tau^n_t(s^t)=0$. If the government's initial claims
against the private sector were to exceed the present value of
future government expenditures, a trivial implication would be
that the government would like to return this excess financial
wealth as lump-sum transfers to the households, and our argument here with
$\Phi=0$ would remain applicable. But when
the present value of all government expenditures exceeds the
value of any initial claims against the private sector, the
Lagrange multiplier $\Phi>0$.
For example, suppose $b_0=0$ and that there is some $g_t(s_t)>0$. Since
$Q<0$ and $\theta_t(s^t)>0$, it follows from equation
 \Ep{TSs_calc3}
that $\Phi>0$.


\subsection{Computational  strategy}

To
compute a Ramsey allocation, we can proceed as follows:

\medskip

\item{1.} Start with a guess of the value for $\Phi$,  then use the first-order conditions   \Ep{TSs_foc} and the
feasibility conditions \Ep{TSs_tech}
as above to compute $c_t(s^t), n_t(s^t)$ for $t, s^t$ for all $t \geq 1$ and
$c_0(s^0,b_0)$ and $n_0(s^0, b_0)$.

\medskip
\item{2.} Use the first-order conditions \Ep{TSs_foc} to solve for $\theta_t(s^t)$
and $\theta_0(s^0)$ as functions of $\Phi$ and  the candidate Ramsey allocation at $\Phi$ computed in step 1.

\medskip

\item{3.}  Compute the negative term $Q$ as described above and check
whether equation \Ep{TSs_calc3} is satisfied.  If it is, we have constructed
a Ramsey allocation. If not, proceed to step 4 and try another $\Phi$.

\medskip

\item{4.}  Find a $\Phi$ that satisfies equation \Ep{TSs_calc3} by gradually raising $\Phi$ if the
left side of \Ep{TSs_calc3} is positive and lowering $\Phi$ if the left side of \Ep{TSs_calc3}
is negative.

\medskip

\noindent After computing a Ramsey allocation, we can recover the flat tax rate on labor from
$$ (1 - \tau_t^n(s^t)) = {\frac{u_l(s^t)}{u_c(s^t)} }  \EQN LSA_map;a $$
and  implied one-period Arrow securities prices from
$$ p_{t}(s_{t+1}| s^t) = \beta \pi_{t+1}(s_{t+1} | s^t) {\frac{u_c(s^{t+1})}{u_c({s^t})}} .\EQN LSA_map;b $$
% {\bf XXXXXX} We shall use these expressions for Arrow securities prices when we formulate a recursive version of the
% Ramsey problem.






\subsection{More specialized computations}
%{\bf Tom XXXXX: make sure you add that the formulation here is valid for $t \geq 1$ -- where we impose stationarity.}
 %  We now assume that $s$ is Markov with transition matrix $\Pi$.
%  For the remainder of this appendix,  we assume that $s_t$ is Markov with transition matrix $\Pi$ and that $g_t$ is a time invariant function $g(s_t)$ of
%  the Markov state $s_t$.
It is useful to specialize the model in the following way.
Assume that  $s$ is governed by a finite state Markov chain with
states $s\in [1, \ldots, S]$ and  transition matrix $\Pi$, where $\Pi(s'|s) = {\rm Prob}(s_{t+1} = s'| s_t =s)$. Also,
assume that government purchases $g$ are an exact time-invariant function  $g(s)$ of
$s$. We maintain these assumptions throughout this subsection.


Substituting from \Ep{LSA_map}, and the feasibility condition \Ep{TSs_tech}
into a  version of the household's
budget constraint \Ep{TS_bc} without the physical capital
terms and with an equilibrium wage rate of $w_t(s^t)=1$ gives
$$ \eqalign{ u_c(s^t) [ n_t(s^t) - g_t(s_t)] &+ \beta \sum_{s_{t+1}} \Pi (s_{t+1}| s_t) u_c(s^{t+1}) b_{t+1}(s_{t+1} | s^t)
 \cr & - u_l (s^t) n_t(s^t) = u_c(s^t) b_t(s_t | s^{t-1}) . } \EQN LSA_budget1 $$
Define the product $x_t(s^t) = u_c(s^t) b_t(s_t | s^{t-1})$.
% the level of government debt
%scaled by the marginal utility of consumption  today.
Notice that $x_t(s^t)$ appears on the right side of \Ep{LSA_budget1} while
 $\beta$ times the conditional expectation of $x_{t+1}(s^{t+1})$ appears on the left side, so that the equation shares much of the structure of
 a simple asset pricing equation with $x_t$ being analogous to the price of the asset at time $t$.
%  Equation \Ep{LSA_budget1}  will be an ingredient of a recursive formulation of a Ramsey problem for the % section \use{Lucas-Stokey}
%  Lucas-Stokey (1983) model.
%It is useful to define the marginal utility adjusted debt  $x_t(s^t) = u_c(s^t) b_t(s^t)$.


%%In section \use{Lucas-Stokey}, we learned that   for a Ramsey allocation,
%%$c_t(s^t), n_t(s^t)$ and  $b_t(s_t|s^{t-1})$, and therefore also $x_t(s^t)$, %%are each functions only of $s_t$, being independent of the history $s^{t-1}$  %%for $t \geq 1$.  That means that

Above we learned that   in a Ramsey allocation, for $t \geq 1$
 $c_t(s^t)$ and $n_t(s^t)$ are each functions only of the current realized quantity of
government purchases $g_t(s_t)$.
%, independent of the history
%$s^{t-1}$  leading up to that outcome.
Under our present specialization in which $g_t$ is a time invariant function of a Markov
state $s_t$, that means that we can express the allocation as
$c(s)$ and $n(s)$. Moreover, as we will learn in
section \use{LS_lessons_debt}, an additional implication of this
specialization is that government debt $b_t(s_t|s^{t-1})$, and
therefore  $x_t(s^t)$, are each functions of the realized state only.
Therefore,
we can express \Ep{LSA_budget1} as
% \NFootnote{Here we are using the section \use{Lucas-Stokey} outcome that
% along a Ramsey allocation $c$ and $n$ can each be expressed
% as exact functions of $s$ only.}
$$  u_c(s)\left(n(s) - g(s)\right) - u_l(s) n(s)  + \beta \sum_{s'} \Pi(s' | s) x'(s')  =  x(s), \EQN LSA_budget2 $$
where $s'$ denotes a next period value of $s$ and $x'(s')$ denotes a next period value of $x$.
%  Here $u_c(n-g)$  means $u_c n - u_c g$, i.e., the marginal utility of consumption times $n$ minus the marginal utility of consumption times $g$.
%In writing $x(s)$ in equation \Ep{LSA_budget2}, we are guessing that $x_t$ can be represented
%as a time-invariant function of the Markov state $s_t$.


Equation \Ep{LSA_budget2} is easy to solve for $x(s)$ for $s = 1, \ldots , S$.
If we let $\vec n, \vec g, \vec x$ denote $S \times 1$ vectors whose $i$th elements are the respective $n, g$, and $x$ values when $s=i$, and
let $\Pi$ be the transition matrix for the Markov state $s$, then
we can express \Ep{LSA_budget2} as the matrix equation\NFootnote{In these equations, by $\vec u_c \vec n$, for example,   we mean element-by-element multiplication of the two vectors.}
$$ \vec u_c(\vec n - \vec g) -\vec u_l \vec n  + \beta \Pi \vec x = \vec x . \EQN LSA_budget200 $$ %\EQN LSA_budget200 $$
This is a system of $S$ linear equations in the $S \times 1$ vector $x$, whose solution is\NFootnote{Here $(I - \beta \Pi )^{-1}$
is the resolvent operator of  section \use{sec:resolvent} of chapter \use{timeseries}.} \index{resolvent operator}%
$$ \vec x= (I - \beta \Pi )^{-1} [ \vec u_c (\vec n-\vec g) - \vec u_l \vec n] . \EQN LSA_xsoln $$
After solving \Ep{LSA_xsoln} for $\vec x$, we can find $b(s_t|s^{t-1})$ in Markov state $s_t=s$
from $ b(s) = {\frac{x(s)}{u_c(s)}} $ or
the matrix equation
$$ \vec b = {\frac{ \vec x }{\vec u_c}} ,  \EQN LSA_bsoln $$
where division here means element-by-element division of the respective components of the $S \times 1$ vectors $\vec x$ and $\vec u_c$.


Here is a computational algorithm:

\medskip

\item{1.} Start with a guess for the value for $\Phi$,  then use the first-order conditions   \Ep{TSs_foc} and the
feasibility conditions \Ep{TSs_tech}
 to compute $c(s), n(s)$ for  $s \in [1,\ldots , S] $  and
$c_0(s_0,b_0)$ and $n_0(s_0, b_0)$, given $\Phi$.  These are $2  (S+1)$ equations in $2  (S+1)$ unknowns.

\medskip

\item{2.} Solve the $S$ equations \Ep{LSA_budget200} for the $S$ elements of $\vec x$.  These depend on $\Phi$. % as does
%Then compute $\vec b$ from \Ep{LSA_bsoln}.

\medskip

\item{3.}  Find a $\Phi$ that satisfies
$$   u_{c,0} b_0 = u_{c,0} (n_0 - g_0) - u_{l,0} n_0  + \beta \sum_{s=1}^S \Pi(s | s_0) x(s) \EQN Bellman2consr $$
 by gradually raising $\Phi$ if the
left side of \Ep{TSs_calc3}  exceeds the right side  and lowering $\Phi$ if the left side is smaller.

\medskip

\noindent After computing a Ramsey allocation, we can recover the flat tax rate on labor and implied one-period Arrow securities
prices as before.
% $$ (1 - \tau_t^n(s^t)) = {\frac{u_l(s^t)}{u_c(s^t)} }  \EQN LSA_tax $$
% and  implied one-period Arrow securities prices from
% $$ p_{t+1}(s_{t+1}| s^t) = \beta \pi_{t+1}(s_{t+1} | s^t) {\frac{u_c(s^{t+1})}{u_c({s^t})}} .\EQN LSA_Arrow $$
% % {\bf XXXXXX} We shall use these expressions for Arrow securities prices when we formulate a recursive version of the
% % Ramsey problem.

%
% It remains to describe how to determine $\Phi= \Phi_0$.  Given
%  values of the $3 S + 2$  variables consisting of $n_0, c_0, \vec n, \vec c, \vec x$ computed thus far, all of which are to be
%  thought of  as  functions of $\Phi_0 = \Phi_1$,
% it is necessary that $\Phi_0= \Phi_0(b_0, s_0)$ be such that it verifies
% the time $0$ implementability constraint \Ep{Bellman2cons}.
% An  iterative algorithm like that described in section \use{Lucas-Stokey} applies.
%
% {\bf Tom XXXXX: update and revise the above paragraph -- refer to equations in this section.}



% {\bf Tom XXXXX: update and revise -- refer to equations in this section.}
In summary,  when $g_t$ is a time invariant function of a Markov state $s_t$, a Ramsey plan can be constructed
 by solving $3S +3$ equations in $S$ components each of  $\vec c$, $\vec n$, and $\vec x$
together with
$n_0, c_0$, and $\Phi$.











\subsection{Time consistency}

\specsec{Definition:}  Let  $\{\tau_t^n(s^t)\}_{t=0}^\infty, \{b_{t+1}(s_{t+1}| s^t)\}_{t=0}^\infty$ be a  time $0$, state $s_0$ Ramsey plan.
Then $\{\tau_j^n(s^j)\}_{j=t}^\infty, \{b_{j+1}(s_{j+1}| s^j)\}_{j=t}^\infty$ is a  time $t$, history $s^t$
  continuation of a time $0$, state $s_0$  Ramsey plan.


\medskip
\specsec{Definition:} A time $t$, history $s^t$  Ramsey plan is a Ramsey plan  that starts from initial conditions $s^t, b_t(s_t|s^{t-1})$.


\medskip

\specsec{Key outcome:}  For the Lucas-Stokey model, a time $t$, history $s^t$ continuation  of a time $0$, state $0$ Ramsey planner is
{\it not\/} a time $t$, history $s^t$ Ramsey plan.
A short hand way to say this is to say that
a Ramsey plan is not ``time consistent''.
\index{time inconsistency!of Ramsey plan}%

\medskip

\noindent We shall explore the time inconsistency  of a  time  $0$, state $0$ Ramsey plan in more depth  % \use{Ramsey_appa1}
 where we formulate
the Ramsey problem recursively in chapter \use{optaxrecur}.



\section{Examples of labor tax smoothing}\label{sec:labor_tax_smoothing}
Following Lucas and Stokey (1983),
we now exhibit examples of government expenditure streams and how
they affect optimal tax policies. % in the economy of the preceding section.
Throughout we assume that
$b_0=0$.


\subsection{Example 1: $g_t=g$ for all $t\geq0$}
Given  constant  government purchases $g_t=g$, the first-order condition
\Ep{TS_barg} is the same for all $t, s^t$, and we conclude that the
optimal allocation is constant over time: $(c_t,n_t)=(\hat c, \hat n)$
for $t\geq 0$. It then follows from condition \Ep{TS_focX;a} or
\Ep{LSA_map;a} that the tax rate
that  implements the optimal allocation is also constant over
time: $\tau^n_t=\hat \tau^n$, for $t\geq 0$.
%Consequently, equation \Ep{TSs_calc1}
%implies that
The government budget is balanced  each period.

Government debt issues in this economy  serve to smooth
distortions over time. Because government expenditures are already smooth
in this economy, they are optimally financed from contemporaneous
taxes. Nothing is gained by using debt to change the timing of tax
collections.


\subsection{Example 2: $g_t=0$ for $t\not= T$ and nonstochastic $g_T>0$}
Setting $g=0$ in expression \Ep{TS_barg}, the optimal
allocation  $(c_t,n_t)=(\hat c, \hat n)$ is the same  for all $t\not= T$,
and consequently, from condition \Ep{TS_focX;a}, the tax rate is also constant
over these periods, $\tau^n_t=\hat \tau^n$ for $t\not= T$. Using
equations \Ep{TSs_calc2}, we can deduce tax revenues.
Recall that $c_t-n_t=0$ for $t\neq T$  and that $b_0=0$. Thus, the
last terms in equations \Ep{TSs_calc2}
drop out. Since $\Phi>0$, the second (quadratic) term is negative, so
the first term must be positive. Since $(1+\Phi)>0$, this fact implies
$$
0<\hat c - {u_{\ell} \over u_c} \hat n = \hat c - (1-\hat \tau^n) \hat n
 = \hat \tau^n \hat n,
$$
where the first equality invokes condition \Ep{TS_focX;a}. We conclude that
tax revenue is positive for $t\not= T$. For period $T$, the last term
in equation \Ep{TSs_calc2}, $\theta_T g_T$, is positive. Therefore, the sign
of the first term is indeterminate: labor may be either taxed or
subsidized in period $T$.

This example is a stark illustration of tax smoothing in which the Ramsey planner uses government  debt
used to redistribute tax distortions over time. With the same tax
revenues in all periods before and after time $T$, the optimal debt
policy is as follows: in each period $t=0,1, \ldots ,T-1$, the government
runs a surplus, using it to accumulate  bonds issued by the representative household.
So $b_t < b_{t-1} < 0$ for $t = 1, \ldots , T$.
In period $T$, the expenditure $g_T$ is met by selling all of these
bonds, possibly levying a tax on current labor income, and issuing
new bonds that are thereafter rolled over forever. Interest payments
on that constant outstanding government debt are equal to the constant
tax revenue for $t\not= T$,
$\hat \tau^n \hat n$. Thus, the tax distortion is the same in all periods
surrounding period $T$, regardless of their proximity to  date $T$. %This
%symmetry was first noted by Barro (1979).
%\auth{Barro, Robert J.}

\subsection{Example 3: $g_t=0$ for $t\not= T$, and $g_T$ is stochastic}
We assume that $g_T=g>0$ with probability $\alpha \in (0,1)$ and $g_T=0$ with
probability $1-\alpha$.  As in the previous example, there is an
optimal constant allocation $(c_t,n_t)=(\hat c, \hat n)$ for all
periods $t\not= T$ (although the optimum values of $\hat c$ and
$\hat n$ will not, in general, be the same as in example 2). In addition,
equation \Ep{TS_barg} implies that $(c_T,n_T)=(\hat c, \hat n)$ if $g_T=0$.
The argument in example 2 shows that tax revenue is positive. Consequently, debt issues are as follows.

In each period $t=0,1, \ldots ,T-2$, the government
runs a surplus, using it to accumulate risk-free one-period bonds issued by the
private sector. A significant difference from example 2 occurs in
period $T-1$.  In the present case, the government now sells all of the bonds that it has accumulated and uses
the proceeds plus current labor tax revenue to buy one-period Arrow securities
that  pay off at $T$ only if $g_T=g$. In addition, the government {\it buys\/} more of these contingent
claims in period $T-1$. It finances these additional purchases of Arrow securities by simultaneously {\it issuing\/} one-period risk-free claims.
As in example 2,
at $t > T$, the government just rolls over its risk-free debt and pays out net
payments
equal to $\hat \tau^n \hat n$, only  here the risk-free debt is issued one period earlier.
At time $t=T$, there are two cases to consider, depending on the realization of  government expenditures at date $T$, a random variable.
If $g_T=0$, the government clearly satisfies
its intertemporal budget constraint. If $g_T=g$, the construction
of our Ramsey equilibrium ensures that the payoff on the government's
holdings of contingent claims against the private sector  equal
$g$ plus interest payments of $\hat \tau^n \hat n$ on government debt net
of any current labor tax/subsidy in period $T$. In periods $T+1,T+2,\ldots$,
the situation is as in example 2, regardless of whether $g_T=0$ or
$g_T=g$.

This is another example of tax smoothing over time in which the  tax distortion is the same in all periods around time $T$. It also
demonstrates the risk-sharing aspects of fiscal policy under uncertainty.
In effect, the government in period $T-1$ buys insurance from the private
sector against the event that $g_T=g$.
\auth{Lucas, Robert E., Jr.} \auth{Stokey, Nancy L.}
\section{Lessons for optimal debt policy}\label{LS_lessons_debt}
Lucas and Stokey (1983)
draw three lessons from their analysis of the model in our previous section. The
first is built into the model at the outset: budget balance in a present-value
 sense must be respected. In a stationary economy,  deficits in some states and dates must
 necessarily be offset by  surpluses
at other dates and states. Thus, in  examples with erratic government expenditures,
good times are associated with budget surpluses. Second, in the face of
erratic government spending, the role of government debt is to smooth
tax distortions over time, and the government should not seek
to balance its budget on a continual basis. Third, the
contingent-claim character of government debt is important for an
optimal policy.\NFootnote{Aiyagari, Marcet, Sargent, and Sepp\"al\"a (2002)
\auth{Aiyagari, Rao}\auth{Marcet, Albert}\auth{Sargent, Thomas J.}\auth{Sepp\" al\" a, Juha}%
 offer a qualification to the importance of state-contingent government
debt in the model of Lucas and Stokey (1983). In numerical simulations,
they explore Ramsey outcomes under the assumption that contingent
claims cannot be traded. (We present their  setup  in
 section \use{sec:AMSS}.) They find that the incomplete markets Ramsey
allocation is very close to the complete markets Ramsey allocation. This
proximity comes from the Ramsey policy's use of self-insurance through risk-free
borrowing and lending with households.  Compare this outcome to our chapter \use{incomplete}
 on heterogeneous agents and how self-insurance can soften the effects of
market incompleteness.}

To highlight the role of an optimal state-contingent government debt policy
further, we study the government's budget constraint at time $t$ after
history $s^t$:
$$\EQNalign{
& b_t(s_t | s^{t-1})  = \tau^n_t(s^t) n_t(s^t) - g_t(s_t)   \cr
&\hskip.5cm +  \sum_{j=1}^\infty \sum_{s^{t+j}\vert s^t} q^t_{t+j}(s^{t+j})
    \left[\tau^n_{t+j}(s^{t+j}) n_{t+j}(s^{t+j}) - g_{t+j}(s_{t+j})\right]  \cr
&\hskip.5cm= \sum_{j=0}^{\infty} \sum_{s^{t+j}\vert s^t} \beta^j \pi_{t+j}(s^{t+j}\vert s^t)
{u_c(s^{t+j})\over u_c(s^{t})}
\left\{\Biggl[1-{u_\ell(s^{t+j})\over u_c(s^{t+j})}\right]               \cr
& \hskip1cm \cdot
\left[c_{t+j}(s^{t+j})+g_{t+j}(s_{t+j})\right] - g_{t+j}(s_{t+j}) \Biggr\},      \EQN TS_bc2_LS \cr}
$$
where we have invoked the resource constraint \Ep{TSs_tech} and
conditions \Ep{TS_focX;a} and \Ep{TS_foci} that express taxes and prices
in terms of the allocation. Recall from our discussion of first-order
condition \Ep{TS_barg} that the optimal allocation
$\{c_{t+j}(s^{t+j}), \ell_{t+j}(s^{t+j})\}$
is history independent and depends only on the present realization of government
purchases in any given period. We now ask, on what aspects of history does the
optimal amount of state-contingent debt that matures in period $t$ after
history $s^t$ depend?  Investigating the right side of expression \Ep{TS_bc2_LS},
we see that history dependence would  arise only through history dependence in   the transition
probabilities
$\{\pi_{t+j}(s^{t+j}\vert s^t)\}$ that govern government purchases. Hence, if
government purchases are governed by a Markov process, we conclude that
there can be no history dependence: the beginning-of-period
state-contingent government debt is a function only of the current state $s_t$,
since everything on the right side of \Ep{TSs_tech} depends solely
on $s_t$. This is a remarkable feature of the optimal tax-debt policy.
By purposefully trading in state-contingent debt markets,
the government insulates its net indebtedness to the private sector from any lingering effects of past
shocks to government purchases. Its beginning-of-period indebtedness is
completely tailored to its present circumstances as captured by the realization
of the current state $s_t$. In contrast, our stochastic example 3
above is a nonstationary environment where the debt policy associated
with the optimal allocation depends on both calender time and past
events.\NFootnote{An alternative way to express this statement about example 3 is
that time $t$ must be added to the state $s_t$ to acquire an augmented state that is Markov.}

Finally, we take a look at the value of contingent government debt in
our earlier model with physical capital. Here we cannot expect any
sharp result concerning beginning-of-period debt because of our finding
about the indeterminacy of state-contingent debt and capital taxes.
However, the derivation of that specific finding suggests that instead we
should look at the value of outstanding debt at the end of a period.
By multiplying equation \Ep{TS_bc}
by $p_{t-1}(s_t|s^{t-1})$ and summing over $s_t$, we express the household's
budget constraint for period $t$ in terms of time $t-1$ values,
$$\EQNalign{
& k_t(s^{t-1}) + \sum_{s_{t}} p_{t-1}(s_{t} | s^{t-1}) b_t(s_t | s^{t-1})        \cr
& =  \sum_{s_{t}} p_{t-1}(s_{t} | s^{t-1})
     \Biggl\{ c_t(s^t) - \left[1-\tau^n_t(s^t)\right] w_t(s^t) n_t(s^t) \cr
& \hskip1.5cm +
     k_{t+1}(s^t) + \sum_{s_{t+1}} p_t(s_{t+1} | s^t) b_{t+1}(s_{t+1} | s^t)
    \Biggr\} ,                            \hskip1.5cm            \EQN TS_bc2  \cr}
$$
where the unit coefficient on $k_t(s^{t-1})$ is obtained by invoking conditions
\Ep{TS_focX;b} and \Ep{TS_focX;c}. Expression \Ep{TS_bc2} states that the
household's ownership of capital and contingent debt at the
end of period $t-1$ is equal to the present value of next period's
contingent purchases of goods and financial assets net of labor earnings.
We can eliminate next period's purchases of capital and state-contingent
bonds by using next period's version of
equation \Ep{TS_bc2}. After invoking transversality conditions \Ep{TS_transv},
continued substitutions yield
$$\EQNalign{
& \sum_{s_{t}} p_{t-1}(s_{t} | s^{t-1}) b_t(s_t | s^{t-1})                               \cr
& =  \sum_{j=t}^\infty \ \sum_{s^j\vert s^{t-1}} \beta^{j+1-t} \pi_j(s^j | s^{t-1})
     { u_c(s^j) c_j(s^j) - u_{\ell}(s^j) n_j(s^j) \over u_c(s^{t-1}) }
\cr &
     - k_t(s^{t-1}),                                                  \EQN TS_bc22 \cr}
$$
where we have invoked conditions
 \Ep{TS_focX;a} and \Ep{TS_focX;b}. Suppose now
 $s$ follows a Markov process. Then recall
from earlier that  the allocations
from period 1 onward can be described by time-invariant allocation rules
with the current state $s$ and beginning-of-period
capital stock $k$ as arguments.
Thus, equation \Ep{TS_bc22} implies that the end-of-period government debt is also
a function of the state vector $(s,k)$, since the current state fully determines the
end-of-period capital stock and is the only information needed to form
conditional expectations of future states. Putting together the lessons
of this section with earlier ones, reliance on state-contingent
debt and/or state-contingent capital taxes enables the government to
avoid any lingering effects on indebtedness from past shocks to government
expenditures and past productivity shocks that affected labor tax revenues.


  This striking lack of history dependence
contradicts the extensive history-dependence of the stock of
government debt that Robert Barro (1979) identified as one of the
salient characteristics of his model  of optimal fiscal policy.
According to Barro, government debt should be cointegrated with
tax revenues, which in turn should follow a random walk, with
innovations that are perfectly correlated with innovations in the
government expenditure process. Important aspects of
such behavior of government debt
seem to be observed. For example, Sargent and Velde
(1995) display long series of government debt for eighteenth century
Britain that more closely resembles the outcome from Barro's model
than from Lucas and Stokey's. Partly inspired by those observations,
Aiyagari, Marcet, Sargent, and Sepp\"al\"a (2002) returned to the environment of Lucas and Stokey's
model and altered  the market structure in a way that brought
outcomes closer to Barro's.
%Their model also sheds more light on the lack of history dependence
%in Lucas and Stokey's model.
We create their model by closing almost all of the markets that
Lucas and Stokey had allowed.\NFootnote{Werning (2007) extends the Lucas-Stokey model in another
interesting direction. He assumes that there are complete markets in consumption, that agents are heterogeneous
in the efficiencies of their labor supplies, and that taxes can be nonlinear functions of labor earnings.  For example,
with affine, rather than linear taxes, he explores  how distorting taxes on labor are imposed to redistribute income as well as
    to raise revenues for financing expenditures.  Without heterogeneity of labor efficiencies, no distorting taxes on labor are imposed.}\index{incomplete markets}
\auth{Aiyagari, Rao} \auth{Marcet, Albert} \auth{Sargent, Thomas
J.} \auth{Sepp\" al\" a, Juha}
\auth{Werning, Ivan}



\section{Taxation without state-contingent debt}\label{sec:AMSS}%
Returning to the model without physical capital, we
follow Aiyagari, Marcet, Sargent, and Sepp\"al\"a (2002)
and study optimal taxation without state-contingent debt. The government's
budget constraint in expression \Ep{TS_gov} has to be modified by replacing
state-contingent debt by risk-free government bonds. In period $t$
and history $s^t$, let $b_{t+1}(s^t)$ be the amount of government indebtedness
carried over to and maturing in period $t+1$, denominated in time $(t+1)$ goods. The
market value at time $t$ of that government indebtedness equals
$b_{t+1}(s^t)$ divided by the risk-free gross interest rate between periods $t$
and $t+1$, denoted  $R_t(s^t)$.  Thus,
the government's budget constraint in period $t$ at history $s^t$ becomes
$$\EQNalign{
 b_t(s^{t-1}) &=    \tau^n_t(s^t) n_t(s^t) - g_t(s_t) - T_t(s^t)
                    + {b_{t+1}(s^t) \over R_t(s^t )} \cr
            &\equiv z(s^t) + {b_{t+1}(s^t) \over R_t(s^t )},                 \EQN TS_gov_wo \cr}
$$
where $T_t(s^t)$ is a nonnegative lump-sum transfer to the
representative household and $z(s^t)$  the
net-of-interest government surplus. We include
the term $T_t(s^t) \geq 0 $ to allow for the possibility of  a nonnegative lump-sum transfer to the
private sector. In an  allocation that solves the Ramsey problem and  that
levies distorting  taxes on labor, why would the government ever want to
   hand revenues back to the private sector? Certainly that would never happen in an economy
with state-contingent debt, since any such allocation could be improved
by lowering distortionary taxes rather than handing out lump-sum
transfers. But  without state-contingent debt we shall see that  eventually there
can be circumstances when a government would like to make lump-sum
transfers to the private sector. %However, this usually won't happen.

To rule out Ponzi schemes, we assume that the government is
subject to versions of the natural debt limits defined in chapters
\use{recurge} and \use{incomplete}.
%no-arbitrage argument,
The consumption Euler equation for the representative household able
to trade risk-free debt with one-period gross interest  rate
$R_t(s^t)$ is
$$
{1 \over R_t(s^t)} = \sum_{s_{t+1}} p_t(s_{t+1} | s^t)
= \sum_{s^{t+1}\vert s^t} \beta  \pi_{t+1}(s^{t+1} | s^t)
                        { u_c(s^{t+1}) \over u_c(s^{t}) }.
$$
Substituting this expression into the government's budget constraint
\Ep{TS_gov_wo} yields:
$$
 b_t(s^{t-1}) =  z(s^t) + \sum_{s^{t+1}\vert s^t} \beta  \pi_{t+1}(s^{t+1} | s^t)
                        { u_c(s^{t+1}) \over u_c(s^{t}) } \; b_{t+1}(s^t).  \EQN TS_gov_wo2
$$
Note that  $b_{t+1}(s^t)$ is the same for all realizations of
$s_{t+1}$. We will now replace this constant $b_{t+1}(s^t)$ by another
expression of the same magnitude. In fact, we have as many candidate
expressions of that magnitude as there are possible states $s_{t+1}$,
i.e., for each state $s_{t+1}$ there is a government budget
constraint that is the analogue to expression \Ep{TS_gov_wo} but
where the time index is moved one period forward. And all those budget
constraints have a right side that  equals $b_{t+1}(s^t)$.
Instead of picking one of these candidate expressions to replace
all occurrences of $b_{t+1}(s^t)$ in equation \Ep{TS_gov_wo2}, we replace
$b_{t+1}(s^t)$ when the summation index in equation \Ep{TS_gov_wo2} is
$s_{t+1}$ by the right  side of next period's budget constraint
that is associated with that particular realization $s_{t+1}$.
These substitutions give rise to the following expression:
$$\EQNalign{
 b_t(s^{t-1}) =  z(s^t) + \sum_{s^{t+1}\vert s^t} \beta  \pi_{t+1}(s^{t+1} | s^t)
                        &{ u_c(s^{t+1}) \over u_c(s^{t}) } \cr
&\cdot \left[z(s^{t+1}) + {b_{t+2}(s^{t+1}) \over R_{t+1}(s^{t+1})}\right].  \cr}
$$
After similar repeated substitutions for all future occurrences
of government indebtedness, and by invoking the natural debt limit,
we arrive at a final expression:
$$\EQNalign{
 b_t(s^{t-1}) &=  \sum_{j=0}^\infty \sum_{s^{t+j} | s^t} \beta^j  \pi_{t+j}(s^{t+j} | s^t)
               { u_c(s^{t+j}) \over u_c(s^{t}) } \;z(s^{t+j})  \cr
            &= E_{t} \sum_{j=0}^\infty \beta^j
               { u_c(s^{t+j}) \over u_c(s^{t}) } \;z(s^{t+j}).     \EQN TS_gov_wo3 \cr}
$$
Expression \Ep{TS_gov_wo3} at time $t=0$ and initial state $s^0$,
constitutes an implementability condition derived from the
present-value budget constraint that the government
must satisfy when seeking a solution to the Ramsey taxation
problem:
$$
 b_0(s^{-1}) = E_{0} \sum_{j=0}^\infty \beta^j
               { u_c(s^{j}) \over u_c(s^{0}) } \;z(s^{j}).    \EQN TS_gov_wo4
$$

It is instructive to compare the present economy without
state-contingent debt to the earlier economy with state-contingent
debt. Suppose that the initial government debt in period 0 and
state $s^0$ is the same across the two economies, i.e.,
$b_0(s^{-1})=b_0(s_0|s^{-1})$. Implementability condition \Ep{TS_gov_wo4}
of the present economy is then exactly the same as the one for the
economy with state-contingent debt, as given by expression \Ep{TS_bc2_LS}
evaluated in period $t=0$. But while this is the only implementability
condition arising from budget constraints in the complete markets economy,
many more implementability conditions must be satisfied in the
economy without state-contingent debt. Specifically, because the
beginning-of-period indebtedness is the same across any two histories,
for any two realizations $s^t$ and $\tilde s^t$ that share the same
history  until the previous period, i.e., $s^{t-1}=\tilde s^{t-1}$,
we must impose equality across the right sides of their
respective budget constraints, as depicted in expression
\Ep{TS_gov_wo3}.\NFootnote{Aiyagari et al. (2002)
regard these conditions as imposing measurability of the right-hand
side of \Ep{TS_gov_wo3} with respect to $s^{t-1}$.}
Hence, the Ramsey taxation problem
without state-contingent debt becomes
$$\EQNalign{
\max_{\{c_t(s^t),b_{t+1}(s^t)\}} &E_{0} \sum_{t=0}^\infty \beta^t
                         u\left(c_t(s^t),1-c_t(s^t)-g_t(s_t)\right)  \cr
\noalign{\vskip.2cm}
\hbox{\rm s.t.    }E_{0} \sum_{j=0}^\infty &\beta^j
      { u_c(s^{j}) \over u_c(s^{0}) } \;z(s^{j}) \geq b_0(s^{-1}); \EQN AMSS_st;a\cr
\noalign{\vskip.2cm}
                         E_{t} \sum_{j=0}^\infty &\beta^j
      { u_c(s^{t+j}) \over u_c(s^{t}) } \;z(s^{t+j}) = b_t(s^{t-1}),
          \hskip.3cm \hbox{\rm for all }    s^t;\hskip1.5cm \EQN AMSS_st;b \cr
\noalign{\vskip.2cm}
\hbox{\rm given         }&b_0(s^{-1}),                                            \cr}
$$
where we have substituted the resource constraint \Ep{TSs_tech} into the
utility function. It should also be understood that we have substituted
the resource constraint into the net-of-interest government surplus and
used the household's first-order condition, $1-\tau^n_t(s^t)= u_{\ell}(s^t)
/u_c(s^t)$, to eliminate the labor tax rate. Hence, the net-of-interest
government surplus now reads as
$$
z(s^t) = \left[1 - {u_{\ell}(s^t) \over u_c(s^t)}\right] \left[c_t(s^t)+g_t(s_t)\right]
-g_t(s_t) - T_t(s^t)\,.                                                   \EQN AMSS_z
$$

Next, we compose a Lagrangian for the Ramsey problem.
Let $\gamma_0(s^0)$ be the nonnegative Lagrange multiplier on constraint
\Ep{AMSS_st;a}. As in the earlier economy with state-contingent debt,
this multiplier is strictly positive if the government must resort to
distortionary taxation, and otherwise equal to zero.
The force of the assumption that markets in state-contingent
securities have been shut down but that a market in a risk-free security
remains is that we have to attach stochastic
processes $\{\gamma_t(s^t)\}_{t=1}^\infty$ of Lagrange multipliers to
the new implementability constraints
\Ep{AMSS_st;b}. These multipliers might be positive or
negative, depending on direction in which the constraints are
binding:
$$\EQNalign{
\gamma_t(s^t) &\;\geq\; (\leq)\;\, 0 \hskip.4cm
                   \hbox{\rm if the constraint is binding in the direction} \cr
&E_{t} \sum_{j=0}^\infty \beta^j
 { u_c(s^{t+j}) \over u_c(s^{t}) } \;z(s^{t+j}) \;\geq \;(\leq)\;\, b_t(s^{t-1}).\cr}
$$
A negative multiplier $\gamma_t(s^t)<0$ means that if we could relax
constraint \Ep{AMSS_st;b}, we would like to {\it increase} the
beginning-of-period indebtedness for that particular
realization of history $s^t$, which would presumably enable us to
reduce the beginning-of-period indebtedness for some other history.
In particular, as we will soon see from the
first-order conditions of the Ramsey problem, there would then
exist another realization $\tilde s^t$ with the same
history up until the previous period, i.e., $\tilde s^{t-1}= s^{t-1}$,
but where the multiplier on constraint \Ep{AMSS_st;b} takes on a
positive value $\gamma_t(\tilde s^t)>0$. All this is indicative of
the fact that the government cannot use state-contingent debt
and therefore cannot allocate its indebtedness most efficiently
across future states.


%  \endchapter   OKOK

We apply two transformations to the
Lagrangian. We multiply constraint \Ep{AMSS_st;a} by $u_c(s^0)$
and the constraints \Ep{AMSS_st;b} by
$\beta^t u_c(s^{t})$. The Lagrangian for the Ramsey problem can then
be represented as follows, where the second equality invokes the law of
iterated expectations and uses Abel's summation
formula:\NFootnote{See Apostol (1974,
p.\ 194). For another application, see chapter
\use{socialinsurance}, page \use{abelsummation}.}
\offparens
$$\EQNalign{
 J &= E_{0} \sum_{t=0}^\infty \beta^t
                     \biggl\{ u\left(c_t(s^t), 1-c_t(s^t)-g_t(s_t)\right)\cr
&  \hskip1cm + \gamma_t(s^t) \Bigl[ E_{t} \sum_{j=0}^\infty \beta^j
      u_c(s^{t+j}) \,z(s^{t+j}) - u_c(s^{t}) \,b_t(s^{t-1}) \Bigr]  \biggr\}           \cr
&= E_{0} \sum_{t=0}^\infty \beta^t
                      \biggl\{ u\left(c_t(s^t), 1-c_t(s^t)-g_t(s_t)\right)   \cr
&  \hskip1cm + \Psi_t(s^t)\, u_c(s^{t}) \,z(s^{t})
                     - \gamma_t(s^t)\, u_c(s^{t}) \, b_t(s^{t-1}) \Bigr]  \biggr\},
                                                                   \EQN AMSS_lagr;a  \cr}
$$
where
$$
\Psi_t(s^t)=\Psi_{t-1}(s^{t-1})+\gamma_t(s^t)
                                                                   \EQN AMSS_lagr;b
$$
and $\Psi_{-1}(s^{-1})=0$. The first-order condition with respect to $c_t(s^t)$ can be
expressed as
$$\EQNalign{
&u_c(s^t)-u_{\ell}(s^t)     \cr
&+ \Psi_t(s^t)\left\{ \left[
                       u_{cc}(s^t) - u_{c\ell}(s^{t})\right]z(s^{t})
                       + u_{c}(s^{t})\,z_c(s^{t}) \right\}  \cr
             &  - \gamma_t(s^t)\left[
                       u_{cc}(s^{t}) - u_{c\ell}(s^{t})\right]b_t(s^{t-1}) =0 \,,
                                                      \EQN AMSS_foc;a \cr
\noalign{\vskip.3cm}
\noalign{\noindent \rm and with respect to $b_t(s^t)$,}
\noalign{\vskip.3cm}
&E_{t} \left[\gamma_{t+1}(s^{t+1})\,u_c(s^{t+1})\right] = 0\,.
                                                      \EQN AMSS_foc;b \cr}
$$
\autoparens
If we substitute $z(s^t)$ from equation \Ep{AMSS_z} and
its derivative $z_c(s^t)$ into first-order condition \Ep{AMSS_foc;a},
we will find only two differences from the corresponding
condition \Ep{TS_barg} for the optimal allocation in an economy with
state-contingent government debt.
%\NFootnote{For reasons that
%will become clear as we proceed, we are not including
%a nonnegative lump-sum transfer $T_t(s^t)$ in the function
%$z(s^t)$ of equation \Ep{AMSS_foc;a}.  Including it would not lead to any material difference
%in expression \Ep{TS_barg}.}
First, the term involving $b_t(s^{t-1})$
in first-order condition \Ep{AMSS_foc;a} does not appear in
expression \Ep{TS_barg}.  Once again, this term reflects the constraint that
beginning-of-period government indebtedness must be the same across all
realizations of next period's state, a constraint that is not present
if government debt can be state contingent. Second, the Lagrange
multiplier $\Psi_t(s^t)$ in first-order condition \Ep{AMSS_foc;a} may
change over time in response to realizations of the state, while the
multiplier $\Phi$ in expression \Ep{TS_barg}  is time invariant.

Next, we are interested to learn
if the optimal allocation without state-contingent government debt
will eventually be characterized by an
expression similar to \Ep{TS_barg},
i.e., whether or not the Lagrange multiplier $\Psi_t(s^t)$ converges
to a constant, so that from there on, the absence of state-contingent
debt no longer binds.


\subsection{Future values of $\{g_t\}$ become deterministic}
Aiyagari et al.\ (2002) prove that if $\{g_t(s_t)\}$ has absorbing states
in the sense that $g_t = g_{t-1}$ almost surely for $t$ large enough,
then $\Psi_t(s^t)$ converges when $g_t(s_t)$ enters an absorbing state.
The optimal tail allocation for this economy without state-contingent
government debt coincides with the allocation of an economy with
state-contingent debt that would have occurred under the same
shocks, but for different initial debt. That is, the limiting random
variable $\Psi_\infty$ will then play the role of the single multiplier
in an economy with state-contingent debt because, as noted above,
the first-order condition \Ep{AMSS_foc;a} will then be the same
as expression \Ep{TS_barg}, where $\Phi=\Psi_\infty$. The value of $\Psi_\infty$
depends on the realization of the government expenditure path. If the
absorbing state is reached after many bad shocks (high values of
$g_t(s_t)$), the government will have accumulated high debt, and
convergence will occur to a contingent-debt economy with
high initial debt and therefore a high value of the multiplier
$\Phi$.

This particular result about convergence
can be stated in more general terms, i.e., $\Psi_t(s^t)$ can
be shown to converge if the future path of government expenditures
eventually  becomes deterministic, for example, if government expenditures eventually  become
 constant. Once uncertainty about future government expenditures ceases, the
government can thereafter  attain the Ramsey allocation
with one-period risk-free bonds, as described at the beginning of
this chapter. In the present setup, this becomes
apparent from examining first-order condition \Ep{AMSS_foc;b}
when there is no uncertainty: next period's nonstochastic
marginal utility of consumption must be multiplied
by a nonstochastic multiplier $\gamma_{t+1}=0$ in order for
that first-order condition to be satisfied under certainty.
The zero value of all future multipliers $\{\gamma_t\}$
implies convergence of $\Psi_t(s^t)=\Psi_\infty$, and we are
back to our earlier logic where expression \Ep{TS_barg}
with $\Phi_t=\Psi_\infty$
characterizes the optimal tail allocation for an
economy without state-contingent government debt when there
is no uncertainty.

%\endchapter    NOT OK

\subsection{Stochastic $\{g_t\}$ but special preferences}
To study whether $\Psi_t(s^t)$ can converge when
$g_t(s_t)$ remains stochastic forever,  it is helpful to
substitute expression \Ep{AMSS_lagr;b} into first-order
condition \Ep{AMSS_foc;b}
$$
E_{t} \left\{\left[\Psi_{t+1}(s^{t+1})-\Psi_t(s^t)\right]\,u_c(s^{t+1})\right\} = 0 ,
$$
which can be rewritten as
$$\EQNalign{
\Psi_t(s^t)&=E_{t} \left[\Psi_{t+1}(s^{t+1})
{u_c(s^{t+1})\over E_{t}u_c(s^{t+1})}\right] \cr
\noalign{\vskip.3cm}
&=E_{t} \Psi_{t+1}(s^{t+1})
+ {{\rm cov}_{t} \left(\Psi_{t+1}(s^{t+1}), \,u_c(s^{t+1})\right)
                     \over E_{t}u_c(s^{t+1})}   \,.  \hskip1cm    \EQN AMSS_Psi   \cr}
$$
\vskip.3cm

Aiyagari et al.\ (2002) present a convergence result for a special
class of preferences that make the covariance term in equation
\Ep{AMSS_Psi} identically equal to zero. The  household's utility is
assumed to be linear in consumption and additively separable from
the utility of leisure. (See the preference specification in
 subsection \use{sec:Ex3revist}.) Thus, the marginal utility of consumption is
constant and expression \Ep{AMSS_Psi} reduces to
$$
\Psi_t(s^t)=E_{t} \Psi_{t+1}(s^{t+1}).
$$
The stochastic process $\Psi_t(s^t)$ is evidently a nonnegative
martingale. As described in equation \Ep{AMSS_lagr;b},
$\Psi_t(s^t)$ fluctuates over time in response to realizations
of the multiplier $\gamma_t(s^t)$ that can be either positive
or negative; $\gamma_t(s^t)$ measures the marginal impact of news about
the present value of government expenditures on the maximum
utility attained by the planner. The cumulative multiplier
$\Psi_t(s^t)$ remains strictly
positive so long as the government must resort to distortionary
taxation either in the current period or for some realization of the
state in a future period.

By a theorem of Doob (1953, p.\ 324), a nonnegative martingale like  $\Psi_t(s^t)$ converges almost surely.\NFootnote{For a
discussion of the martingale convergence theorem, see the appendix
to chapter \use{selfinsure}.} If the process for government
expenditures is sufficiently stochastic, e.g., when $g_t(s_t)$ is
stationary with a strictly positive variance, then Aiyagari et al.
(2002) prove that $\Psi_t(s^t)$ converges almost surely to zero.
When setting $\Psi_\infty=\gamma_\infty=0$ in first-order
condition \Ep{AMSS_foc;a}, it follows that the optimal tax policy
must eventually lead to a first-best allocation with
$u_c(s^t)=u_{\ell}(s^t)$, i.e., $\tau^n_\infty=0$. This implies
that government assets converge to a level always sufficient to
support government expenditures from interest earnings alone. Unspent interest earnings on government-owned assets
are returned to the households as positive lump-sum transfers. Such transfers
  occur whenever government expenditures fall below
their maximum possible level.

A proof that $\Psi_t(s^t)$  converges to zero and that government
assets eventually  become large enough to finance all future government
expenditures % under the stated preferences
 can be  constructed along
lines that supported analogous outcomes in our chapter
\use{selfinsure} analysis of  self-insurance with incomplete markets.
Like the analysis there, we can  appeal to a martingale convergence theorem
and use an argument based on contradictions to rule out
convergence to any number other than zero. To establish a
contradiction in the present setting, suppose that $\Psi_t(s^t)$
does not converge to zero but rather to a strictly positive limit,
$\Psi_\infty > 0$. According to our argument above, the optimal
tail allocation for this economy without state-contingent
government debt will then coincide with the allocation of an
economy that has state-contingent debt and a particular initial
debt level. It follows that these two economies should have
identical labor tax rates supporting that optimal tail allocation.
But Aiyagari et al. (2002) show that a government that follows
such a tax policy and has access only to risk-free bonds to absorb
stochastic surpluses and deficits will with positive probability
 either see its debt grow without bound or watch its assets grow without
bound, two outcomes that are  inconsistent with an optimal
allocation. The heuristic explanation is as follows. The
government in an economy with state-contingent debt uses these
debt instruments as an ``insurance policy'' to smooth taxes
across realizations of the state. The government's lack of access to  such an
``insurance policy'' when only risk-free bonds are available
means that implementing those very same tax rates, unresponsive as
they are to realizations of the state, would  expose the
government to a positive probability of seeing either its debt
level or its asset level drift off to infinity. But that
contradicts a supposition that such a tax policy would be optimal
in an economy without state-contingent debt. First, it is
impossible for government debt to grow without bound, because
households would not be willing to lend to a government that
violates its natural borrowing limit. Second, it is not optimal
for the government to accumulate assets without bound, because
welfare could then be increased by cutting tax rates in some
periods and thereby reducing the deadweight loss of
taxation.\NFootnote{Aiyagari et al. (2002, lemma 3) suggest that
unbounded growth of government-owned assets constitutes a
contradiction because it violates a lower bound on debt, or an
``asset limit.'' But we question this argument, since a government
can trivially avoid violating any asset limit by making positive
lump-sum transfers to the households. A correct proof should
instead be based on the existence of welfare improvements
associated with cutting distortionary taxes instead of making any
such lump-sum transfers to households.} Therefore, we conclude
that $\Psi_t(s^t)$ cannot converge to a nonnegative limit other
than zero.

For more general preferences and ample randomness in
government expenditures, Aiyagari et al. (2002) cannot
characterize the limiting dynamics of $\Psi_t(s^t)$ except
to rule out convergence to a strictly positive number.
So at least two
interesting possibilities remain: $\Psi(s^t)$ may converge
to zero or it may have a nondegenerate distribution in the
limit.

\subsection{Example 3 revisited: $g_t=0$ for $t\not= T$, and $g_T$ is stochastic}\label{sec:Ex3revist}%
To illustrate differences in optimal tax policy between economies
with and without state-contingent government debt, we revisit our
third example %above of government expenditures that was taken from
%Lucas and Stokey's (1983) analysis of an economy with state-contingent debt.
%Let us examine how the optimal policy changes if
when now the government has
access only to risk-free bonds.\NFootnote{Our first  two   examples
above involve no uncertainty, so the issue of state-contingent debt
does not arise. Hence the optimal tax policy is unaltered in those
two examples.} We assume that the household's utility function is quasi-linear:\index{quasi-linear utility}%
$$
u\left(c_t(s^t), \ell_t(s^t)\right) = c_t(s^t) + H\left(\ell_t(s^t)\right),
$$
where $H_\ell>0$, $H_{\ell \ell}<0$ and $H_{\ell \ell \ell}>0$.
We assume that $H_\ell(0)=\infty$ and $H_\ell(1)<1$ to guarantee
that the first-best allocation without distortionary taxation
has an interior solution for leisure.
Given these preferences, the first-order condition \Ep{AMSS_foc;a}
with respect to consumption simplifies to
$$u_c(s^t)-u_{\ell}(s^t) + \Psi_t(s^t) \, u_{c}(s^t)\,z_c(s^{t}) =0 \,,
$$
which for our quasi-linear preference specification becomes
$$\EQNalign{
&\left[1+\Psi_t(s^t)\right] \left\{1-H_\ell\!\left(1-c_t(s^t)-g_t(s_t)\right)\right\}\cr
\noalign{\vskip.3cm}
& \;\;=-\Psi_t(s^t)\, H_{\ell \ell}\!\left(1-c_t(s^t)-g_t(s_t)\right)
               \left[c_t(s^t)+g_t(s_t)\right]. \hskip1cm \EQN AMSS_example \cr}
$$

As in our earlier analysis of this example, we assume that
$g_T=g>0$ with probability $\alpha$ and $g_T=0$ with probability
$1-\alpha$. We also retain our assumption that the government
starts with no assets or debt, $b_0(s^{-1})=0$, so that
the multiplier on constraint \Ep{AMSS_st;a} is strictly positive,
$\gamma_0(s^0)= \Psi_0(s^0) > 0$. Since no additional information
about future government expenditures is revealed in periods $t<T$,
it follows that the multiplier $\Psi_t(s^t) = \Psi_0(s^0) \equiv \Psi_0 >0$
for $t<T$. Given the multiplier $\Psi_0$, the optimal
consumption level for $t<T$, denoted $c_0$, satisfies the following
version of first-order condition \Ep{AMSS_example}:
$$\left[1+\Psi_0\right] \left\{1-H_\ell\!\left(1-c_0\right)\right\}
=-\Psi_0\, H_{\ell \ell}\!\left(1-c_0\right)\,c_0\,. \EQN AMSS_c_0
$$
In period $T$, there are two possible values of $g_T$, and hence
the stochastic multiplier $\gamma_T(s^T)$ can take two possible
values, one negative value and one positive value, according to
first-order condition \Ep{AMSS_foc;b}; $\gamma_T(s^T)$ is negative
if $g_T=0$ because that represents good news that should cause the
multiplier $\Psi_T(s^T)$ to fall. In fact, the multiplier
$\Psi_T(s^T)$ falls all the way to zero if $g_T=0$ because the
government would then never again have to resort to distortionary
taxation. And any tax revenues raised in earlier periods and
carried over as government-owned assets would then also be handed
back to the households as a lump-sum transfer. If, on the other
hand, $g_T=g>0$, then $\gamma_T(s^T)\equiv \gamma_T$ is strictly
positive and the optimal consumption level for $t>T$, denoted
$\tilde c$, would satisfy the following version of first-order
condition \Ep{AMSS_example}
$$
\left[1+\Psi_0+\gamma_T\right]
              \left\{1-H_\ell\!\left(1-\tilde c\right)\right\}
=-\left[\Psi_0+\gamma_T\right]\,
H_{\ell \ell}\!\left(1-\tilde c\right)\,\tilde c\,.
                                    \hskip.1cm         \EQN AMSS_c_1
$$
In response to $\gamma_T>0$, the multiplicative factors within
square brackets have increased on both sides of equation \Ep{AMSS_c_1}
but proportionately more so on the right side. Because both
equations \Ep{AMSS_c_0} and \Ep{AMSS_c_1} must hold with equality at
the optimal allocation, it follows that the change from $c_0$
to $\tilde c$ has to be such that
$\left\{1-H_\ell\!\left(1- c\right)\right\}$ increases proportionately
more than $-\{H_{\ell \ell}\!\left(1-c\right)\, c\}$. Since
the former expression is decreasing in $c$ and the latter expression
is increasing in $c$, we can then conclude that $\tilde c < c_0$
and hence that the implied labor tax rate is raised for all periods $t>T$
if government expenditures turn out to be strictly positive in period $T$.

It is obvious from this example that a government with access only
to risk-free bonds cannot smooth tax rates across  different
realizations of the state. Recall that the optimal tax policy with
state-contingent debt prescribed a constant tax rate for all
$t\not=T$ regardless of the realization of $g_T$. Note also that,
as discussed above, the multiplier $\Psi_t(s^t)$ in the economy
without state-contingent debt does converge when the future path
of government expenditures becomes deterministic in period $T$. In
our example, $\Psi_t(s^t)$ converges  either to zero or to
$(\Psi_0+\gamma_T)>0$, depending on the realization of government
expenditures. Starting from period $T$, the optimal tail
allocation coincides then with the allocation of an economy with
state-contingent debt that would have occurred under the same
shocks, but for different initial debt, either a zero debt level
associated with $\Phi=0$, if $g_T=0$, or the positive
debt level that would correspond to $\Phi = \Psi_0+\gamma_T$, if
$g_T=g>0$.

\auth{Schmitt-Grohe, Stephanie} \auth{Uribe, Martin} \auth{Siu, Henry} %
Schmitt-Grohe and Uribe (2004a) and Siu (2004) analyze optimal
monetary and fiscal policies in economies in which the government
can issue only {\it nominal\/} risk-free debt.  Unanticipated
inflation makes risk-free nominal debt state contingent in real
terms
 and  provides a motive for the government to make inflation
vary. Schmitt-Grohe and Uribe and Siu both focus on how
price stickiness would affect the
 government's use of fluctuations in
inflation as an indirect way of introducing state-contingent
debt.  They find that  even a very small amount of
price stickiness causes the volatility of
the optimal inflation rate to become very small. Thus, the
government abstains from the indirect channel for synthesizing
state-contingent debt. The authors relate their finding to the
aspect of Aiyagari's et al.'s calculations for an economy with no
state-contingent debt, mentioned in  footnote 10 of this chapter, that the Ramsey
allocation in their economy without state-contingent debt closely
approximates that for the economy with complete markets.



\section{Nominal debt  as
state-contingent real debt}\label{sec:nom_into_real}%
We now turn to a monetary economy of Lucas and Stokey (1983) that, under particular preferences,
Chari, Christiano, and Kehoe (1996)  used to study the optimality of the
Friedman rule and whether  equilibrium price level adjustments can transform nominal
non-state-contingent debt into state-contingent real debt.
In particular,  Chari, Christiano and Kehoe restricted preferences to satisfy  Assumption 1 below in order to  show optimality of the Friedman rule.
They stated no additional conditions to reach their conclusion that, under an appropriate policy,  non-state-contingent nominal government debt can be transformed into
   state-contingent real debt.  However, we find that their statements about the equivalence of allocations under these two debt structures  require stronger assumptions  because of a potential  sign-switching problem with  optimal debt across state realizations at a point in time. To obtain Chari, Christiano and Kehoe's conclusion,  we add our  Assumption 2 below. \auth{Chari, V.V.}\auth{Christiano, Lawrence J.}\auth{Kehoe, Patrick J.}%
\auth{Lucas, Robert E., Jr.}\auth{Stokey, Nancy L.}%
\index{Friedman rule}%

Our strategy is to follow Chari, Christiano, and Kehoe by first, in subsection \use{sec:opt_nonmonetary},  finding a Ramsey plan and an associated Ramsey allocation for a nonmonetary
economy with state-contingent government debt; then, in subsection \use{sec:map_mon_into_nonmonet},  stating  conditions on fundamentals that suffice to allow that  same allocation to  be supported by a Ramsey plan for
a monetary economy without real state-contingent government debt but only nominal non-state-contingent debt.

A key outcome here is that the Ramsey equilibrium in the monetary economy makes
the price level fluctuate in response to  shocks to government expenditures. The price level adjusts  to deliver history-contingent returns to
holders of government debt required to support the Ramsey allocation.      %{sec:price_indeterminacy}
This structure includes many, if not most, components of  a `fiscal theory of the price level'.\NFootnote{We admit that some writers could legitimately
 beg to differ here because  we have
   chosen to express the fiscal theory of the price level within the straightjacket of a rational expectations competitive equilibrium in an Arrow-Debreu complete markets economy.
   Some would argue that the fiscal theory of the price level can dispense with auxiliary assumptions like rational expectations and complete markets.} That it does so in a coherent way  can help us describe a set of contentious
  issues associated with some expositions of fiscal  theories of the price level.   We take up this theme in section \use{sec:CKK_FTPL}, and again in
chapter \use{fiscalmonetary}.
\index{fiscal theory of price level}%

\subsection{Setup and main ideas}
The production technology is as in \Ep{TSs_tech} except that there
are now two distinct goods being produced; a `cash good' and a `credit good'.
Let $c_{1t}(s^t)$ and $c_{2t}(s^t)$ denote the household's consumption of
the cash good and the credit good, respectively, while government consumption
$g_t(s_t)$ is  made up of credit goods only. The feasibility condition
becomes
$$
c_{1t}(s^t) + c_{2t}(s^t) + g_t(s_t) = n_t(s^t).             \EQN TSs_tech_money
$$
The household's preferences are ordered by
$$ \sum_{t=0}^\infty \sum_{s^t}
       \beta^t \, \pi_t(s^t) \, u[  c_{1t}(s^t), c_{2t}(s^t), \ell_t(s^t) ], \quad \beta \in (0,1)   \EQN TSs_pref_money
$$
where $u$ is  increasing, strictly concave, and
twice continuously differentiable in all of its arguments.
Moreover, the utility function satisfies the Inada conditions
$$ \lim_{c_1 \downarrow 0} u_1(s^t) = \lim_{c_2 \downarrow 0} u_2(s^t) =
                 \lim_{\ell \downarrow 0} u_3(s^t) = +\infty,
$$
where $u_j$ is the derivative of the utility function with respect
to its $j$th argument.

In period $t$, households trade money, assets, and goods in  particular ways.
At the start of period $t$, after observing the current state $s_t$,
households trade money and assets in a centralized securities market.
The assets are one-period, state-noncontingent, nominal discount bonds. Let
$M_{t+1}(s^t)$ and $B_{t+1}(s^t)$ denote the money and the nominal
bonds held at the end of the securities market trading. Let $R_t(s^t)$
denote the gross nominal interest rate, so  $1/R_t(s^t)$ is
the purchase price of bonds.  After securities trading, each household
splits into a worker and a shopper. The shopper must use the money to
purchase cash goods. To purchase credit goods, the shopper issues one-period nominal
claims that  are  to be settled in the securities market in the next period.
The worker is paid in cash at the end of the period. He adds this cash to any cash unexpended during shopping and carries it into the next period.

The household's budget constraint in the securities market is
$$\EQNalign{
&M_{t+1}(s^t) + B_{t+1}(s^t) / R_t(s^t) = B_t(s^{t-1}) \cr
&\hskip1cm  + M_t(s^{t-1}) - P_{t-1}(s^{t-1}) c_{1t-1}(s^{t-1})
- P_{t-1}(s^{t-1}) c_{2t-1}(s^{t-1}) \cr
&\hskip1cm + P_{t-1}(s^{t-1}) [1-\tau^n_{t-1}(s^{t-1})] n_{t-1}(s^{t-1}),   \EQN bc_security_money \cr}
$$
where $P$ is the nominal price of goods. The left side of \Ep{bc_security_money}
is the nominal value of assets held at the end of securities market trading.
The first term on the right side is the value of nominal debt bought in the
preceding period. The next two terms on the right side are the shopper's unspent
cash. The fourth term is the payment for credit goods, and the last term is
after-tax labor earnings. Purchases of cash goods must satisfy a cash-in-advance
constraint:
$$
P_t(s^t) c_{1t}(s^t) \leq M_{t+1}(s^t).                         \EQN CIA_money
$$

Let $\lambda_t(s^t)$ and $\mu_t(s^t)$ be the Lagrange multipliers on constraints
\Ep{bc_security_money} and \Ep{CIA_money}, respectively. The first-order
conditions for the household's %optimization
problem are
$$\EQNalign{
c_{1t}(s^t)\rm{:}& \ \ \beta^t \pi_t(s^t) u_1(s^t) = P_t(s^t) [ \mu_t(s^t)
                          + \sum_{s^{t+1}\vert s^t} \lambda_{t+1}(s^{t+1}) ] ,
  \hskip2cm                                            \EQN money_foc;a \cr
c_{2t}(s^t)\rm{:}& \ \ \beta^t \pi_t(s^t) u_2(s^t) =
                   P_t(s^t) \sum_{s^{t+1}\vert s^t} \lambda_{t+1}(s^{t+1}) ,
                                                            \EQN money_foc;b \cr
n_t(s^t)\rm{:}& \ \ \beta^t \pi_t(s^t) u_3(s^t) =
    P_t(s^t) [1-\tau^n_t(s^t)]  \sum_{s^{t+1}\vert s^t} \lambda_{t+1}(s^{t+1}) ,
                                                            \EQN money_foc;c \cr
M_{t+1}(s^t)\rm{:}& \ \ \lambda_t(s^t) = \mu_t(s^t)
                             +   \sum_{s^{t+1}\vert s^t} \lambda_{t+1}(s^{t+1}),
                                                            \EQN money_foc;d \cr
B_{t+1}(s^t)\rm{:}& \ \ \lambda_t(s^t)/R_t(s^t)  =
                                \sum_{s^{t+1}\vert s^t} \lambda_{t+1}(s^{t+1}).
                                                            \EQN money_foc;e  \cr}
$$
After substituting \Ep{money_foc;d} into \Ep{money_foc;a},
\Ep{money_foc;e} into \Ep{money_foc;b}, and \Ep{money_foc;e}
into \Ep{money_foc;c}, respectively, the following conditions emerge
$$\EQNalign{
& \beta^t \pi_t(s^t) u_1(s^t) = P_t(s^t) \lambda_t(s^t) ,
  \hskip2cm                                            \EQN money2_foc;a \cr
& \beta^t \pi_t(s^t) u_2(s^t) = P_t(s^t) \lambda_t(s^t) / R_t(s^t) ,
                                                       \EQN money2_foc;b \cr
& \beta^t \pi_t(s^t) u_3(s^t) = P_t(s^t) [1-\tau^n_t(s^t)]
                                         \lambda_t(s^t) / R_t(s^t) .
                                                       \EQN money2_foc;c \cr}
$$
By \Ep{money2_foc}, marginal rates of substitution between goods and
leisure satisfy
$$
{u_1(s^t) \over R_t(s^t) u_3(s^t)} = {u_2(s^t) \over u_3(s^t)}
= { 1 \over 1-\tau^n_t(s^t) }.
$$
Hence, the marginal rate of substitution
between the cash and credit good  equals the nominal interest rate,
$$
{u_1(s^t) \over u_2(s^t)} = R_t(s^t).
$$

Another expression for the nominal interest rate is obtained by solving
for $\lambda_t(s^t)$ from \Ep{money2_foc;a} and a corresponding expression
for $\lambda_{t+1}(s^{t+1})$, which are then substituted into \Ep{money_foc;e} to get
$$
R_t(s^t) = {u_1(s^t) / P_t(s^t) \over
 \beta \sum_{s_{t+1}} \pi_{t+1}(s^{t+1}\vert s^t) u_1(s^{t+1})/P_{t+1}(s^{t+1})}.
                                                       \EQN money_interest
$$
In an equilibrium, the nominal interest rate must satisfy $R_t(s^t)\geq1$ because
otherwise, households could make infinite profits by buying money and selling bonds.
This inequality captures the much discussed zero lower bound on the net nominal interest rate
$R_t(s^t) - 1$. \index{zero lower bound}%

Money is injected  into and withdrawn from the economy through government open-market
operations in the securities market. The constraint on open market operations is
$$\EQNalign{
&M_{t+1}(s^t) - M_t(s^{t-1}) + B_{t+1}(s^t) / R_t(s^t) = B_t(s^{t-1}) \cr
&+ P_{t-1}(s^{t-1}) g_{t-1}(s_{t-1})
 - P_{t-1}(s^{t-1}) \tau^n_{t-1}(s^{t-1}) n_{t-1}(s^{t-1}). \hskip1cm
                                                  \EQN gov_security_money \cr}
$$
The terms on the left side of this equation are the assets sold by
the government. The first term on the right is the payment on debt
incurred in the preceding period, the second term is the payment for
government consumption, and the third term is tax receipts. Recall that
government consumption consists solely of credit goods.

To ensure that the Ramsey problem is nontrivial in this environment, we
follow Chari et al.\ (1996) and assume that households hold no nominal
assets at the beginning of period 0,
$M_0(s^{-1})=B_0(s^{-1})=0$.\NFootnote{As in the case of our earlier restriction on the
capital tax in period~0, the initial condition here restricts the government's
access to nondistorting ways of financing its expenditures. In a
monetary economy, if the initial stock of nominal assets held by
households is positive, then welfare is maximized by increasing the initial
price level to infinity. If the initial stock is negative, then welfare
is maximized by setting the initial price level so low that those initial
assets are sufficient to finance the government's entire stream of expenditures
without ever having to levy distorting taxes.}
Hence, the government
constraint \Ep{gov_security_money} in period 0 becomes
$$
M_{1}(s^0)  + B_{1}(s^0) / R_0(s^0) = 0. \EQN gov_security0_money
$$
Starting with \Ep{gov_security_money}, and then recursively eliminating
debt issued in a preceding period, by using corresponding versions of
\Ep{gov_security_money}, continuing all the way back to time~0 when
\Ep{gov_security0_money} applies, we arrive at the following expression for how
nominal, state-noncontingent debt evolves over time,
$$\EQNalign{
B_{t+1}(s^t) &= \sum_{j=0}^{t-1} ( \prod_{k=j}^{t}  R_k(s^k) ) P_j(s^j)
[ g_j(s_j) - \tau^n_j(s^j) n_j(s^j) ]  \cr
&+ \sum_{j=1}^{t-1} ( \prod_{k=j}^{t}  R_k(s^k) ) [1-R_{j-1}(s^{j-1})] M_j(s^{j-1}) \cr
&-R_t(s^t) M_{t+1}(s^t).
                                                    \EQN gov_securityPV_money  \cr}
$$

%%\noindent
Chari et al. (1996) maintain the following assumption:
\medskip
\noindent{\sc Assumption 1:}  The utility function is of the form
$u(c_1, c_2, \ell) = V( f(c_1, c_2), \ell )$
where $f$ is homothetic.
\medskip
\noindent
Under this assumption,  Chari et al. (1996) demonstrate
that the Friedman rule is optimal, i.e., the Ramsey equilibrium has
$R_t(s^t)=1$ for all $s^t$.
%\noindent
Despite the need to use some distorting taxes in the Ramsey equilibrium, it is
not optimal to levy an inflation tax. Moreover, the Ramsey solution
in our monetary economy with nominal, state-noncontingent debt yields an
allocation that is identical to that in a frictionless nonmonetary economy in
which the government issues (real) state-contingent debt. This finding enables us
in the next section first to  solve the Ramsey problem for the nonmonetary  economy, and then
to verify that the  Ramsey allocation for the nonmonetary economy  is also the Ramsey allocation for the monetary economy.

The reason that identical Ramsey allocations can prevail for the two economies is that the nominal, state-noncontingent debt in
the monetary economy can in effect be transformed into state-contingent real debt by
systematically varying the price level. In bad times associated  with high government
expenditures, it is optimal to raise the price level so that  real debt payments can be made to be relatively small.
In good times with low government expenditures, it is optimal to lower the price level to make  real
debt payments become  relatively large.

The above description of an optimal policy potentially identifies a technical  problem that could potentially  invalidate
the finding of Chari et al. (1996). Specifically, while monetary policy can be used
to vary the size of real indebtedness at the beginning of a period, it can never alter
 the {\it  sign \/} of the asset position  inherited from  last period. Hence,
monetary policy cannot replicate a state-contingent debt policy in the nonmonetary economy
if there is any period when the nonmonetary economy has positive state-contingent indebtedness
under some realizations of the state and negative under others. Therefore, we must add the
following  assumption on primitives that will suffice to justify  the finding of
Chari et al. (1996).
\medskip
\noindent{\sc Assumption 2:}  The Ramsey equilibrium in the nonmonetary economy with
state-contingent debt has strictly positive indebtedness, $b_{t}(s^{t})>0$, in all
periods $t\geq1$ after all histories $s^{t}$.
\medskip
\noindent




\subsection{Optimal taxation in a nonmonetary economy}\label{sec:opt_nonmonetary}%
Here we  solve the Ramsey problem in a frictionless nonmonetary
economy. Then in the next section, we will show how,  under a particular monetary policy,  the same optimal
allocation can also be supported as an equilibrium
of the monetary economy.

To have a complete set of tax instruments in the
nonmonetary economy, we introduce a value-added tax on good 1,
$\tau^{c1}_t(s^t)$; it becomes a subsidy when negative.
The government's budget constraint at time $t$ after history $s^t$
becomes
$$\EQNalign{
&\sum_{s_{t+1}} q^t_{t+1}(s_{t+1},s^t) b_{t+1}(s_{t+1}\vert s^t)
= b_t(s_t \vert  s^{t-1}) \cr
&\hskip1cm + g_t(s_t) - \tau^n_t(s^t) n_t(s^t) - \tau^{c1}_t(s^t) c_{1t}(s^t),
                                                  \EQN gov_bc_nonmoney \cr}
$$
where as earlier, $b_{t+1}(s_{t+1} \vert  s^{t})$ is a
state-contingent asset  traded in period $t$ at the
price $q^t_{t+1}(s_{t+1},s^t)$, in terms of time $t$ goods (not including any value-added tax on good 1).


The representative household maximizes expression \Ep{TSs_pref_money} subject
to its present-value budget constraint:
$$\EQNalign{
\sum_{t=0}^\infty\ \sum_{s^t} q^0_t(s^t)
     &\left\{\left[1+\tau^{c1}_t(s^t)\right]c_{1t}(s^t) + c_{2t}(s^t)\right\}   \cr
&= \sum_{t=0}^\infty\ \sum_{s^t} q^0_t(s^t) \left[1-\tau^n_t(s^t)\right] n_t(s^t).
                                                     \EQN bc_nonmonetary  \cr}
$$
The first-order conditions for this problem imply
$$\EQNalign{
q^0_t(s^t) =& \beta^t \pi(s^t) { u_2(s^t) \over u_2(s^0) },
                                                       \EQN nonmonetary2_foc;a \cr
1+\tau^{c1}_t(s^t) =& { u_1(s^t) \over u_2(s^t) },
                                                        \EQN nonmonetary2_foc;b \cr
1-\tau^{n}_t(s^t)  =& { u_3(s^t) \over u_2(s^t) }.
                                                        \EQN nonmonetary2_foc;c \cr}
$$
After using these expressions to eliminate prices and taxes
from the household's present-value budget constraint
\Ep{bc_nonmonetary}, the implementability condition in the
Ramsey problem becomes
$$ \sum_{t=0}^\infty\  \sum_{s^t} \beta^t \pi_t(s^t)
         \left[u_1(s^t) c_{1t}(s^t) + u_2(s^t) c_{2t}(s^t)
          - u_3(s^t) n_t(s^t)\right] = 0.              \EQN TSs_cham15_nonmonetary
$$
The first-order conditions of the Ramsey problem with
respect to consumption goods $c_{it}(s^t)$, $i=1,2$, are
$$
(1+\Phi) u_i(s^t) + \Phi \left[\sum_{j=1}^2 u_{ji}(s^t) c_{jt}(s^t)
        -  u_{3i}(s^t) n_t(s^t) \right] = \theta_t(s^t),    \hskip.02cm
                                                    \EQN TSs_foc_nonmonetary
$$
where $\Phi$ and $\theta_t(s^t)$ are Lagrange multipliers
on implementability condition \Ep{TSs_cham15_nonmonetary} and
resource constraint \Ep{TSs_tech_money}, respectively. After
dividing by $u_i(s^t)$ and noting that by Assumption 1,
$u_{3i}(s^t)/u_i(s^t)=V_{12}(s^t)/V_1(s^t)$, we have
$$
(1+\Phi) + \Phi \left[\sum_{j=1}^2
                { u_{ji}(s^t) c_{jt}(s^t) \over u_i(s^t) }
        -  { V_{12}(s^t) \over V_1(s^t) } n_t(s^t) \right]
 = { \theta_t(s^t) \over u_i(s^t) },    \hskip.02cm
                                                    \EQN TSs_foc2_nonmonetary
$$
for $i=1,2$.
Next, note that a utility function that satisfies Assumption 1
implies that the value of the summation on the left side of
expression \Ep{TSs_foc2_nonmonetary} is the same for $i=1$ and
$i=2$.\NFootnote{Recall that homotheticity, as in Assumption 1,
implies that for any constant $\kappa>0$,
$$
  { u_1\left[\kappa c_{1t}(s^t), \kappa c_{2t}(s^t), n_t(s^t)\right]
    \over
    u_2\left[\kappa c_{1t}(s^t), \kappa c_{2t}(s^t), n_t(s^t)\right] }
= { u_1\left[c_{1t}(s^t), c_{2t}(s^t), n_t(s^t)\right]
    \over
    u_2\left[c_{1t}(s^t), c_{2t}(s^t), n_t(s^t)\right] }.
$$
Differentiating this expression with respect to $\kappa$ and
evaluating at $\kappa=1$ gives
$$
\sum_{j=1}^2 { u_{j1}(s^t) c_{jt}(s^t) \over u_1(s^t) }
= \sum_{j=1}^2 { u_{j2}(s^t) c_{jt}(s^t) \over u_2(s^t) }.
$$}
Thus, the value of the entire left side of expression
\Ep{TSs_foc2_nonmonetary} is the same for $i=1$ and $i=2$
and consequently, so must the value of the right side be, so that
it follows that $u_1(s^t)= u_2(s^t)$. From
expression \Ep{nonmonetary2_foc;b}, we then conclude that the two
consumption goods are taxed at the same rates, i.e.,
$\tau^{c1}_t(s^t) = 0$.\NFootnote{Given that we use tax instruments
$\tau^{c1}_t(s^t)$ and  $\tau^{n}_t(s^t)$ to control the
tax wedges between $c_{1t}(s^t)$, $c_{2t}(s^t)$ and $\ell_t(s^t)$,
equal taxation of the two goods implies
$\tau^{c1}_t(s^t) = 0$, and hence, the optimal tax wedge between
goods consumption and leisure is attained with the labor tax.
If we instead had formulated the problem with two separate
value-added taxes on the two goods, $\tau^{c1}_t(s^t)$ and
$\tau^{c2}_t(s^t)$, but no labor tax, the equal taxation
result for goods would have meant
$\tau^{c1}_t(s^t) = \tau^{c2}_t(s^t)$,
where the level of that common value-added tax rate would have
served the same role as the present labor tax, i.e.,
to establish the optimal tax wedge between goods consumption
and leisure.}

Given $\tau^{c1}_t(s^t) = 0$, the government debt satisfies
an expression analogous to  expression \Ep{TS_bc2_LS},
namely,
$$\EQNalign{
 b_t(s_t \vert s^{t-1}) &= \tau^n_t(s^t) n_t(s^t) - g_t(s_t)   \cr
&\hskip1cm +  \sum_{j=1}^\infty \sum_{s^{t+j}\vert s^t} q^t_{t+j}(s^{t+j})
    \left[\tau^n_{t+j}(s^{t+j}) n_{t+j}(s^{t+j}) - g_{t+j}(s_{t+j})\right]
\hskip.3cm\cr
&= \sum_{j=0}^{\infty} \sum_{s^{t+j}\vert s^t} \beta^j \pi_{t+j}(s^{t+j}\vert s^t)
{u_2(s^{t+j})\over u_2(s^{t})}
\left\{\Biggl[1-{u_3(s^{t+j})\over u_2(s^{t+j})}\right]               \cr
%& \hskip1cm \cdot
& \cdot
\left[c_{1t+j}(s^{t+j})+c_{2t+j}(s^{t+j})+
      g_{t+j}(s_{t+j})\right] - g_{t+j}(s_{t+j}) \Biggr\}.
                                               \EQN TS_bc2_nonmonetary \cr}
$$





\subsection{Optimal policy in a corresponding  monetary economy}\label{sec:map_mon_into_nonmonet}%
We want to verify our conjecture  that the optimal allocation of the nonmonetary
economy with state-contingent debt can be attained in the monetary
economy with state-noncontingent nominal debt under a  monetary policy that  implements
the Friedman rule, i.e., $R_t(s^t)=1$ for all $s^t$.

At the beginning of any period $t>0$ and history $s^t$, given
outstanding nominal assets $\{M_t(s^{t-1}), B_t(s^{t-1})\}$
and last period's price level $P_{t-1}(s^{t-1})$, the government
targets a current price level equal to
$$\EQNalign{
P_t(s^t) &=  \left\{ P_{t-1}(s^{t-1}) \left[ g_{t-1}(s_{t-1})
       - \tau^n_{t-1}(s^{t-1}) n_{t-1}(s^{t-1}) \right]  \right. \cr
& \hskip.3cm \left.
+ M_t(s^{t-1}) + B_t(s^{t-1}) \right\} / b_t(s_t \vert s^{t-1}),
                                               \EQN price_target \cr}
$$
which is evidently a function of the beginning-of-period indebtedness of the
corresponding nonmonetary economy, $b_t(s_t \vert s^{t-1})>0$.
The government attains the target by conducting open-market
operations to make the money supply become
$$
M_{t+1}(s^t) = P_t(s^t) c_{1t}(s^t),      \EQN money_target
$$
and the resulting nominal debt level $B_{t+1}(s^t)$ is given by
\Ep{gov_security_money}, or equivalently, by
\Ep{gov_securityPV_money}. Note that, if our conjecture $R_t(s^t)=1$
is correct, policy rule \Ep{price_target} and government budget
constraint \Ep{gov_security_money} imply
$$
{M_{t+1}(s^t) + B_{t+1}(s^t) \over P_t(s^t) } = b_t(s_t \vert s^{t-1}).
                                           \EQN asset_target
$$

We now proceed as if our conjecture is correct, and verify that
the gross nominal interest rate  given by expression \Ep{money_interest}
would indeed  equal to unity under the government policy we have described.
The key steps in verifying that the policy attains $R_t(s^t)=1$ are:
$$\EQNalign{
R_t(s^t) &= {u_2(s^t) / P_t(s^t) \over
 \beta \sum_{s_{t+1}} \pi_{t+1}(s^{t+1}\vert s^t) u_2(s^{t+1})/P_{t+1}(s^{t+1})} \cr
\noalign{\vskip.2cm}
& = {1 / P_t(s^t) \over
 \sum_{s_{t+1}} q_{t+1}^t(s_{t+1},s^t) /P_{t+1}(s^{t+1})} \cr
\noalign{\vskip.2cm}
& = {\displaystyle 1 / P_t(s^t) \over
\displaystyle \sum_{s_{t+1}} q_{t+1}^t(s_{t+1},s^t) { \displaystyle
 b_{t+1}(s_{t+1} \vert s^t)
\over \displaystyle
P_t(s^t) \left[ g_t(s_t) - \tau^n_t(s^t) \right]
+ M_{t+1}(s^{t}) + B_{t+1}(s^{t}) } } \cr
\noalign{\vskip.2cm}
& = { g_t(s_t) - \tau^n_t(s^t) +b_t(s_t \vert s^{t-1}) \over
 \sum_{s_{t+1}} q_{t+1}^t(s_{t+1},s^t) b_{t+1}(s_{t+1} \vert s^t) }
= 1.\cr }
$$
Under our conjecture, the first equality above substitutes
$u_2(s^t)=u_1(s^t)$ into expression \Ep{money_interest}. The
second equality uses expression \Ep{nonmonetary2_foc;a} for the
price of state-contingent claims in the nonmonetary economy.
The third equality uses policy rule \Ep{price_target} for the
price level $P_{t+1}(s^{t+1})$, while the fourth equality
invokes expression \Ep{asset_target}. The last equality follows
from government budget constraint \Ep{gov_bc_nonmoney} in the
nonmonetary economy and hence, we have confirmed that our
conjecture $R_t(s^t)=1$ is correct.

Next, even though it is redundant, it is instructive to verify that
the government budget constraint of the nonmonetary economy
implies the budget constraint of the monetary economy. Starting with
government budget constraint \Ep{gov_bc_nonmoney} in the nonmonetary
economy with $\tau^{c1}_t(s^t)=0$, we use
expression \Ep{nonmonetary2_foc;a} for the price of state-contingent
claims (where $u_2(s^t)=u_1(s^t)$), and eliminate
all state-contingent debt by using policy rule \Ep{price_target},
$$\EQNalign{
& \sum_{s_{t+1}} \beta \pi_{t+1}(s^{t+1}\vert s^t)
{u_1(s^{t+1})\over u_1(s^{t})} \cr
\noalign{\vskip.2cm}
& \hskip1cm \cdot {P_t(s^t) \over P_{t+1}(s^{t+1}) }
\left\{ \left[ g_t(s_t) - \tau^n_t(s^t) n_t(s^t) \right]
+ {M_{t+1}(s^t) + B_{t+1}(s^t) \over P_t(s^t) } \right\}             \cr
\noalign{\vskip.2cm}
&= {P_{t-1}(s^{t-1}) \over P_{t}(s^{t}) }
\left\{ \left[ g_{t-1}(s_{t-1})
- \tau^n_{t-1}(s^{t-1}) n_{t-1}(s^{t-1}) \right]
+ {M_{t}(s^{t-1}) + B_{t}(s^{t-1}) \over P_{t-1}(s^{t-1}) } \right\}  \cr
\noalign{\vskip.2cm}
&\hskip1cm + g_t(s_t) - \tau^n_t(s^t) n_t(s^t). \cr}
$$
Invoking equilibrium expression \Ep{money_interest} for the nominal
interest rate, this expression can be written
$$\EQNalign{
& \hskip.3cm \left[R_t(s^t)\right]^{-1}
\left\{ \left[ g_t(s_t) - \tau^n_t(s^t) n_t(s^t) \right]
+ {M_{t+1}(s^t) + B_{t+1}(s^t) \over P_t(s^t) } \right\}             \cr
\noalign{\vskip.2cm}
&= {P_{t-1}(s^{t-1}) \over P_{t}(s^{t}) }
\left\{ \left[ g_{t-1}(s_{t-1})
- \tau^n_{t-1}(s^{t-1}) n_{t-1}(s^{t-1}) \right]
+ {M_{t}(s^{t-1}) + B_{t}(s^{t-1}) \over P_{t-1}(s^{t-1}) } \right\}  \cr
\noalign{\vskip.2cm}
&\hskip1cm + g_t(s_t) - \tau^n_t(s^t) n_t(s^t). \cr}
$$
Since we have confirmed that $R_t(s^t)=1$ under the
postulated monetary policy, this expression can then be
simplified to become government budget
constraint \Ep{gov_security_money} in the monetary economy.




\section{Relation to fiscal theories of the price level}\label{sec:CKK_FTPL}%
In chapter \use{fiscalmonetary}, we take up monetary-fiscal theories of inflation, including one
that has been christened a `fiscal theory of the price level'.  The model of the previous section
includes  components that combine to give rise to all of the forces active in that  theory. As emphasized by Niepelt (2004), accounts of that theory
are too often at best incomplete  because they leave implicit  aspects of an underlying general equilibrium  model. For that reason,
we find it enlightening
to interpret statements about the fiscal theory of the price level within the context of a coherent general equilibrium model like that of Chari, Christiano, and Kehoe (1996).
\auth{Niepelt, Dirk}
\auth{Chari, V.V.}
\auth{Christiano, Lawrence J.}
\auth{Kehoe, Patrick J.}
\auth{Cochrane, John H.}

\subsection{Budget constraint versus asset pricing equation}\label{sec:CKK_valuationequation}%
To take a prominent example,  Cochrane (2005) asserts that under a fiat money system there exists  no government budget
constraint, only a {\it valuation equation\/} for nominal government debt.   To express
Cochrane's point of view within  the economy of
Chari, Christiano, and Kehoe (1996), solve for $b_t(s_t \vert s^{t-1})$ from policy rule
\Ep{price_target} and substitute the outcome  into \Ep{TS_bc2_nonmonetary} to get
$$\EQNalign{
&{ M_t(s^{t-1}) + B_t(s^{t-1}) + P_{t-1}(s^{t-1})
\left[ g_{t-1}(s_{t-1}) - \tau^n_{t-1}(s^{t-1}) \right]
 \over P_t(s^t) }                                          \cr
 &= \tau^n_t(s^t) n_t(s^t) - g_t(s_t)   \cr
&+  \sum_{j=1}^\infty \sum_{s^{t+j}\vert s^t} q^t_{t+j}(s^{t+j})
    \left[\tau^n_{t+j}(s^{t+j}) n_{t+j}(s^{t+j}) - g_{t+j}(s_{t+j})\right].
             \hskip.5cm                        \EQN TS_bc2_Cochrane \cr}
$$
The left side is the stock of    nominal liabilities of
the government outstanding at the beginning of period $t$ divided by the
equilibrium price level $P_t(s^t)$. The right side is the real value of future government  surpluses.  Cochrane notes  the resemblance
of this equation to an asset pricing equation and interprets it as a fiscal theory of the price level by asserting that  news about the right side of \Ep{TS_bc2_Cochrane} will cause immediate adjustments in the price level that he regards as `revaluations' of the government's
outstanding stock of government debt.

 Niepelt (2004) objects to isolating
  equation \Ep{TS_bc2_Cochrane} and interpreting it in this way.  Niepelt's perspective can be expressed neatly within the  Chari, Christiano, and Kehoe model.
  Here it is the   stochastic process for the price level that  is  an equilibrium object that adjusts to create a stochastic process of realized
   return on nominal government debt that does the same job as state-contingent real government debt.   Within an equilibrium, individuals are never  `surprised' or `disappointed'
   by rate of return realizations on nominal government debt.  As the economy passes from one Arrow-Debreu node to its successor, Cochrane's favorite equation \Ep{TS_bc2_Cochrane} of
   course prevails
   (along with other equations such as the household's intertemporal optimization conditions that help
   determine a continuation equilibrium), but it is  potentially   misleading to promote
   equation   by itself \Ep{TS_bc2_Cochrane} to the status of a complete theory of the price level.



\subsection{Disappearance of quantity theory?}\label{sec:CKK_quantitytheory}%
 Actually,  a version of the quantity theory  coexists along with equation \Ep{TS_bc2_Cochrane} in the Chari, Christiano, and Kehoe model.  To see this, note that the government can set the initial price level $P_0(s^0)$ to any positive number by executing an
 appropriate time~$0$ open market operation.\NFootnote{The argument of this subsection treats time $0$ in a peculiar way because no endogenous variables
  inherited from the past  impede independently manipulating time $0$ (and all subsequent) nominal quantities.  This special treatment of time $0$ also characterizes many  other presentations of the
 quantity theory of money as a `pure units change' experiment that  multiplies nominal quantities at all dates and all histories by the same positive scalar.  Commenting on a paper by Robert Townsend at the Minneapolis Federal Reserve Bank in 1985,  Ramon Marimon
 asked `when is time $0$?', thereby anticipating doubts expressed by Niepelt (2004).}
Thus, we can regard the time~$0$ government budget constraint \Ep{gov_security0_money} as a constraint on
a time~$0$ open market operation.  The government can set  the nominal money supply $M_1(s^0)>0$ to an arbitrary
positive number subject to the constraint $B_1(s^0)=-M_1(s^0)$  that holds under the Friedman
rule.  The government issues money to purchase nominally denominated bonds subject to $B_1(s^0)=-M_1(s^0)$, as given
 by government budget
constraint \Ep{gov_security0_money} under the Friedman rule.  The household is willing to issue these bonds in exchange for money.
Presuming that the cash-in-advance constraint is satisfied with equality, the price level and money supply
then conform to $P_0(s^0)c_{10}(s^0) = M_1(s^0)$, which is a version of the quantity theory equation for the price level at time $0$.\NFootnote{Under the
Friedman rule with a zero nominal interest rate, the household would
 be indifferent between holding excess balances of money above and beyond
cash-in-advance constraint \Ep{CIA_money} or holding of nominal
government bonds. Likewise, the government would be indifferent about whether to
issue  nominal indebtedness in the form of nominal bonds or money,
because both liabilities carry the same cost to the government, either
in the form of interest payments on bonds or open-market repurchases of
money to deliver a deflation that amounts to the same real return on
money as on bonds. However, while the
composition of nominal government liabilities is indeterminate
under the Friedman rule, the total amount of such liabilities and
hence, the price level is determinate.\label{Friedman_indet}}
This equation provides the basis for
a sharp statement of the quantity theory: with fiscal policy being held constant, a time~$0$ open market operation that increases $M_1(s^0)$ leads to a proportionate increase
in the price level at all histories and dates while leaving the equilibrium allocation and real rates of return  unaltered.


\subsection{Price level indeterminacy under interest rate peg}\label{sec:SW_interestpeg}%
 Sargent and Wallace (1975) established
that an interest rate peg  leads to price level indeterminacy in an {\it ad hoc\/} macroeconomic model.
We demonstrate how the same result arises in two different forms in the
Chari, Christiano, and Kehoe model. First, using  an argument similar to that of the
quantity theory in section \use{sec:CKK_quantitytheory}, we show that a proportionate
increase in the price level at all histories and dates is consistent with a
given interest rate peg.
Second, we extend the indeterminacy result to an economy
with ongoing sunspot uncertainty.
%; something that may or may not have
%been intended by Sargent and Wallace (1975), but seems to capture how
%their analysis has come to be understood.
\auth{Sargent, Thomas J.}
\auth{Wallace, Neil}

We simplify our analysis  by setting all distortionary taxes,
government consumption, and government bond issues equal to zero.  We
 assume that the government can levy a real lump sum tax $\tau^h_t(s^t)$
on the household at time $t$ after history $s^t$.  %, as in chapter
%\use{linappro}.
A negative value of $\tau^h_t(s^t)$ means a lump sum
transfer to the household. As with other taxes, the lump sum tax is
payable in money and paid to the government in the securities market;
when $\tau^h_t(s^t)$ is negative, the household receives the lump sum
transfer as money in the securities market. The new version of government
budget constraint \Ep{gov_security_money} becomes
$$
M_{t+1}(s^t) - M_t(s^{t-1}) = - P_t(s^t) \tau^h_t(s^t),
                                               \EQN gov_security_money2
$$
where now the government uses the revenues from lump sum taxation to
alter the money supply in the securities market. The new version of
government constraint \Ep{gov_security0_money} in period $0$ is
$$
M_{1}(s^0)  = - P_0(s^0) \tau^h_0(s^0). \EQN gov_security0_money2
$$

Under an interest rate peg, the government stands
ready to accommodate any money demand that arises at the pegged
interest rate.
What will be the resulting equilibrium price level? Note
from interest rate expression \Ep{money_interest} that any
proportionate increase in the price level at all histories and dates
is consistent with a given interest rate peg. This means that the price level
is indeterminate under an interest rate peg.

%Besides that interest rates are the same when scaling all prices
%by the same constant,
We now turn to a form of  price level indeterminacy
that can be driven by ongoing sunspot uncertainty. For a given interest
rate peg $R_t(s^t)$ and a current price level $P_t(s^t)$, interest rate
expression \Ep{money_interest} implies a unique magnitude for the
denominator of that equation, but not for the individual price levels
$P_{t+1}(s^{t+1})$ after each history $s^{t+1}$ in the next
period. There exist equilibria in which next period's price level depend on a sunspot, so long as restriction \Ep{money_interest} is
satisfied with respect to the expected inverse of next period's
price level, weighted by the marginal utilities in different
states next period. Hence, when the government stands ready to
accommodate any such system of expectations, an interest rate peg
is associated with price level indeterminacy.



\subsection{Monetary or fiscal theory of the price level?}

Sargent and Wallace (1975) show that the price level becomes
determinate under a money supply rule, and the same holds true
in the Chari, Christiano, and Kehoe model. Specifically, when the
equilibrium nominal interest is strictly positive, it follows
that cash-in-advance constraint \Ep{CIA_money} holds
with equality and thus, the price level is given by
$P_t(s^t)=M_{t+1}(s^t)/c_{1t}(s^t)$.
Price level determinacy also prevails under the Friedman rule
with a zero nominal interest rate, as discussed in
footnote \use{Friedman_indet}. Thus, in this economy the government's ability to control
 the money supply  gives it the ability to control
the price level.

While the analysis of Sargent and Wallace (1975) is typically   viewed as describing  a monetary theory
of the price level,  proponents of the fiscal theory of the price
level might reply that it is really
fiscal policy as summarized by  right side of government budget constraint
\Ep{gov_security_money2} that determines the price level.
If we ignore  period $0$, we can confirm  that assertion.
The argument goes as follows.
Instead of thinking in terms of the quantity of money needed to
support a price level $P_t(s^t)$, the government proceeds by
calculating the implied real value of households' money balances
carried over from last period, $M_t(s^{t-1})/P_t(s^t)$, and if
that quantity is higher (lower) than the quantity of cash goods
that will be transacted in the present period, $c_{1t}(s^t)$, the
government sets a lump sum tax (transfer) equal to the difference,
$$
\tau_t^h(s^t) = {M_t(s^{t-1}) \over P_t(s^t) } - c_{1t}(s^t).
                                                     \EQN SW_FTPL
$$
After using cash-in-advance constraint \Ep{CIA_money} at equality
to eliminate the last term in expression \Ep{SW_FTPL},
this is just government budget constraint \Ep{gov_security_money2}.
Hence, for the same reason that a money supply rule achieved price
level determinacy so would the prescribed lump sum tax associated with  the required fiscal policy.

So far, so good for the fiscal theory of the price level. But in the present economy, the fiscal theory
runs into trouble for  period $0$. Since we have
assumed that there are no outstanding money balances at the
beginning of period $0$, the fiscal policy according
to expression \Ep{SW_FTPL} becomes
$$
\tau_0^h(s^0) = - c_{10}(s^0).                      \EQN SW_FTPL0
$$
That is, since the government supplies all  money balances
in period $0$, the fiscal policy prescribes
a lump sum transfer (negative tax) equal to the quantity of cash
goods that will be transacted in period $0$. But what will  the
equilibrium price level be? Because expression \Ep{SW_FTPL0} lacks
the first term on the right side of expression \Ep{SW_FTPL} in
all subsequent periods, there is effectively no anchor for the price
level in period $0$. We conclude that, in contrast to an admissible
exogenous process for the nominal money supply, an admissible
exogenous process for real lump sum taxes and transfers is
associated with price level indeterminacy because any positive
number $P_0(s^0)$ can serve as the equilibrium price level in
period $0$.\NFootnote{A monetary model with a cash-in-advance
constraint like \Ep{CIA_money},
is well suited to deliver a conclusion that control of the money
supply can lead to price level determinacy. Because by assumption,
money must be used in exchanges and therefore, cash-in-advance
constraint \Ep{CIA_money} takes on the appearance of a
quantity-theory-of-money equation. But this support for a
particular currency's value vanishes if there are substitutes
that can provide the same transaction services, like a foreign
country's currency, or if money is valued because of
dynamic inefficiency in an overlapping generations model, as
analyzed in chapter \use{ogmodels}. When revisiting the fiscal
theory of the price level in sections \use{sec:10Mdoctrines}
and \use{sec:FTPL_KW_SW}, we will return to these issues in the
context of  a two-country
(two-currency) overlapping generations model.}


A reader might regard this way of disarming a fiscal theory
of the price level as  a knife-edged result because any positive
initial money balances, $M_0(s^{-1})>0$, would render the price level
determinate. Perhaps, but this special example motivates us
to return to the fiscal theory of the price level in
chapter \use{fiscalmonetary}, and section \use{sec:BassettoFTPL}
in particular, where we discuss Bassetto's (2002)
 reformulation of that theory.
In criticizing the valuation-equation view of the government
budget constraint in section \use{sec:CKK_valuationequation},
Bassetto argues that the fiscal theory of the price level should
be cast as a game where government strategies are specified for
arbitrary outcomes, not just equilibrium outcomes. Such a complete specification of a policy scheme
is crucial for determining whether the scheme implements a unique
equilibrium, or whether instead  it leaves room for multiple equilibria indexed by systems of
private sector expectations.
\auth{Bassetto, Marco}

%%{\bf (2b)} ... make contact with Schmitt-Groh\`e and Uribe
%%in ``Price level determinacy and monetary policy under a
%%balanced-budget requirement'' (JME 2000), as well as philosophical
%%reflections by Woodford.




\index{optimal taxation!human capital} \index{human capital}
\auth{Jones, Larry E.} \auth{Manuelli, Rodolfo} \auth{Rossi, Peter
E.}
\section{Zero tax on human capital}
Returning to a nonstochastic nonmonetary model, Jones, Manuelli, and Rossi (1997)
show that the optimality of a limiting zero tax also applies to labor
income in a model with human capital, $h_t$, so long as the technology
for accumulating human capital displays constant returns to scale in
the stock of human capital and goods used (not including raw labor).

We postulate the following human capital technology,
$$
h_{t+1} = (1-\delta_h) h_t + H(x_{ht}, h_t, n_{ht}),   \EQN T_h
$$
where $\delta_h \in (0,1)$ is the rate at which human capital
depreciates. The function $H$ describes how new human capital
is created with the input of a market good $x_{ht}$, the stock
of human capital $h_t$, and raw labor $n_{ht}$. Human capital
is in turn used to produce ``efficiency units'' of labor $e_t$,
$$
e_t = M(x_{mt}, h_t, n_{mt}),                          \EQN T_e
$$
where $x_{mt}$ and $n_{mt}$ are the market good and raw labor
used in the process. We assume that both $H$ and $M$ are
homogeneous of degree one in market goods ($x_{jt}, j=h,m$)
and human capital ($h_t$), and twice continuously differentiable
with strictly decreasing (but everywhere positive) marginal
products of all factors.

The number of  efficiency units of labor $e_t$ replaces our earlier argument
for labor in the production function,
$F(k_t, e_t)$. The household's preferences are still
described by  expression \Ep{cham1}, with leisure
$\ell_t = 1 - n_{ht} - n_{mt}$. The economy's
aggregate resource constraint is
$$\EQNalign{ & c_t + g_t + k_{t+1} + x_{mt} + x_{ht}  \cr
  &= F\left[k_t, M(x_{mt}, h_t, n_{mt}) \right] + (1-\delta) k_t. \EQN cham2h
                                                                        \cr}$$

The household's present-value budget constraint is
$$\EQNalign{
\sum_{t=0}^\infty q^0_t (1+\tau^c_t) c_t &=
\sum_{t=0}^\infty q^0_t
         \left[(1-\tau^n_t) w_t e_t - (1+\tau^m_t) x_{mt} - x_{ht} \right]  \cr
& + \left[(1-\tau^k_{0}) r_{0} + 1-\delta \right] k_0 + b_0,
                                                              \EQN T_bcPV2h \cr}
$$
where we have added $\tau^c_t$ and $\tau^m_t$ to the set of tax instruments,
to enhance the government's ability to control various margins. Substitute
equation
 \Ep{T_e} into equation
 \Ep{T_bcPV2h}, and let $\lambda$ be the Lagrange multiplier on
this budget constraint, while $\alpha_t$ denotes the Lagrange multiplier on
equation
 \Ep{T_h}. The household's first-order conditions are then
$$\EQNalign{
c_t\rm{:}&\ \ \ \ \beta^t u_c(t)- \lambda q^0_t (1+\tau^c_t) = 0,
                                                             \EQN T_foch;a \cr
n_{mt}\rm{:}&\ \ \ -\beta^t u_{\ell}(t) + \lambda q^0_t (1-\tau^n_t)
                                w_t M_n(t) = 0,              \EQN T_foch;b \cr
n_{ht}\rm{:}&\ \ \ -\beta^t u_{\ell}(t) + \alpha_t H_n(t) = 0,    \EQN T_foch;c \cr
x_{mt}\rm{:}&\ \ \ \lambda q^0_t \left[ (1-\tau^n_t) w_t M_x(t)
                                   - (1+\tau^m_t) \right] = 0, \EQN T_foch;d \cr
x_{ht}\rm{:}&\ \ \ -\lambda q^0_t + \alpha_t H_x(t) = 0,            \EQN T_foch;e \cr
h_{t+1}\rm{:}&\ \ \ -\alpha_t + \lambda q^0_{t+1} (1-\tau^n_{t+1}) w_{t+1} M_h(t+1)\cr
        &\qquad \quad
        + \alpha_{t+1} \left[1-\delta_h + H_h(t+1)\right] = 0. \EQN T_foch;f \cr}
$$
Substituting equation \Ep{T_foch;e} into equation \Ep{T_foch;f} yields
\offparens
$$\EQNalign{
{q^0_t \over H_x(t) } = q^0_{t+1} \bigl[&
       {1-\delta_h + H_h(t+1) \over H_x(t+1) } \cr
        &+ (1-\tau^n_{t+1}) w_{t+1} M_h(t+1)    \bigr].         \EQN T_fochH \cr}
$$
\autoparens


We now use the household's first-order conditions to simplify the
sum on the right side of the present-value constraint \Ep{T_bcPV2h}.
First, note that
homogeneity of $H$ implies that equation \Ep{T_h} can be written as
$$
h_{t+1} = (1-\delta_h) h_t + H_x(t) x_{ht} + H_h(t) h_t.
$$
Solve for $x_{ht}$ with this expression,
use $M$ from equation \Ep{T_e} for  $e_t$, and substitute into
the sum on the right side of equation \Ep{T_bcPV2h}, which then becomes
$$\EQNalign{
\sum_{t=0}^\infty q^0_t
  & \Bigl\{ (1-\tau^n_t) w_t M_x(t) x_{mt} + (1-\tau^n_t) w_t M_h(t) h_t
\Bigr. \cr
  &\Bigl.  - (1+\tau^m_t) x_{mt} -
       { h_{t+1} -[1-\delta_h + H_h(t)] h_t \over  H_x(t) }  \Bigr\}.    \cr}
$$
Here we have also invoked the homogeneity of $M$. First-order condition
 \Ep{T_foch;d} implies that the term multiplying $x_{mt}$ is zero,
$[(1-\tau^n_t) w_t M_x(t) - (1+\tau^m_t)]=0$. After rearranging,
we are left with
$$\EQNalign{
&\left[ { 1-\delta_h + H_h(0) \over  H_x(0) }
                         + (1-\tau^n_0) w_0 M_h(0) \right] h_0
- \sum_{t=1}^\infty h_t \Biggl\{ {q^0_{t-1} \over H_x(t-1) }  \cr &
\hskip1cm
   -  q^0_t \left[ { 1-\delta_h + H_h(t) \over  H_x(t) }
                         + (1-\tau^n_t) w_t M_h(t) \right] \Biggr\}.  \EQN T_simple \cr}
$$
However, the term in braces is zero by first-order condition \Ep{T_fochH},
so the sum on the right side of equation \Ep{T_bcPV2h} simplifies to the very
first term in this expression.

Following our standard scheme of constructing the Ramsey plan,
a few more manipulations of the household's first-order conditions
are needed to solve for prices and taxes in terms of the allocation.
We first assume that $\tau^c_0 = \tau^k_0 = \tau^n_0 = \tau^m_0 = 0$.
If the numeraire is $q^0_0=1$, then condition \Ep{T_foch;a} implies
$$
 q^0_t = \beta^t {u_c(t) \over u_c(0)}
                                {1 \over 1+\tau^c_t }.    \EQN T_foch2;a
$$
From equations \Ep{T_foch;b} and \Ep{T_foch2;a} and $w_t=F_e(t)$, we obtain
$$
(1+\tau^c_t)  {u_\ell(t) \over u_c(t)}
        = (1-\tau^n_t) F_e(t) M_n(t),                      \EQN T_foch2;b
$$
and, by equations \Ep{T_foch;c}, \Ep{T_foch;e}, and \Ep{T_foch2;a},
$$
(1+\tau^c_t) {u_\ell(t) \over u_c(t)}
        = {H_n(t) \over H_x(t) },                          \EQN T_foch2;c
$$
and equation \Ep{T_foch;d} with $w_t=F_e(t)$ yields
$$
1+\tau^m_t = (1-\tau^n_t) F_e(t) M_x(t).                   \EQN T_foch2;d
$$
For a given allocation, expressions \Ep{T_foch2} allow us to
recover prices and taxes in a recursive fashion: \Ep{T_foch2;c} defines
$\tau^c_t$ and  \Ep{T_foch2;a} can be used to compute $q^0_t$,
\Ep{T_foch2;b} sets $\tau^n_t$, and \Ep{T_foch2;d} pins down
$\tau^m_t$.

Only one task  remains to complete our strategy of determining
prices and taxes  that achieve any allocation.
The additional condition \Ep{T_fochH}
characterizes the household's intertemporal choice of human capital,
which imposes still another constraint on the price $q^0_t$ and the
tax $\tau^n_t$. Our determination of $\tau^n_t$ in equation
 \Ep{T_foch2;b} can be
thought of as manipulating the margin that the household faces in its
static choice of supplying effective labor $e_t$, but the tax rate also
affects the household's dynamic choice of human capital $h_t$. Thus,
in the Ramsey problem, we will have
to impose the extra constraint that the allocation is
consistent with the same $\tau^n_t$ entering both equations
 \Ep{T_foch2;b} and
 \Ep{T_fochH}. To find an expression for this extra constraint, solve
for $(1-\tau^n_t)$ from equation
 \Ep{T_foch2;b} and a lagged version of equation \Ep{T_fochH},
which are then set equal to each other. We eliminate the price $q^0_t$ by
using equations \Ep{T_foch2;a} and \Ep{T_foch2;c}, and  the final constraint
becomes
$$\EQNalign{
u_\ell(t-1) H_n(t) = & \beta u_\ell(t) H_n(t-1) \cr
&\cdot \left[ 1-\delta_h + H_h(t) + H_n(t) { M_h(t) \over M_n(t)} \right].  \EQN T_fochH2 \cr}
$$

Proceeding to step 2 in constructing the Ramsey plan, we use condition
\Ep{T_foch2;a} to eliminate $q^0_t(1+\tau^c_t)$ in the household's
budget constraint \Ep{T_bcPV2h}. After also invoking the simplified expression
\Ep{T_simple} for the sum on the right side of \Ep{T_bcPV2h}, %the household's
%``adjusted budget constraint''
the implementability condition can be written as
$$ \sum_{t=0}^\infty \beta^t u_c(t) c_t - \tilde A = 0, \EQN cham15h $$
where $\tilde A$ is given by
$$\EQNalign{
 \tilde A &= \tilde A(c_0, n_{m0}, n_{h0}, x_{m0}, x_{h0})   \cr
          &= u_c(0)
   \Biggl\{ \left[ { 1-\delta_h + H_h(0) \over  H_x(0) }
                         + F_e(0) M_h(0) \right] h_0         \cr
          & \hskip2cm   + [F_k(0) + 1-\delta_k] k_0 + b_0 \Biggr\}.     \cr}
$$
In step 3, we define
$$ V(c_t, n_{mt}, n_{ht}, \Phi) = u(c_t,1-n_{mt}-n_{ht})
    + \Phi u_c(t) c_t ,
                                                   \EQN cham17h
$$
and formulate a Lagrangian,
$$\EQNalign{
 J = & \sum_{t=0}^\infty \beta^t \biggl\{ V(c_t, n_{mt}, n_{ht}, \Phi)   \cr
&  + \theta_t \Bigl\{ F\left[k_t, M(x_{mt}, h_t, n_{mt}) \right]  + (1-\delta) k_t \cr
&  \hskip1cm           - c_t - g_t - k_{t+1} - x_{mt} - x_{ht} \Bigr\}            \cr
& + \nu_t \left[(1-\delta_h) h_t + H(x_{ht}, h_t, n_{ht}) - h_{t+1}\right]
 \biggr\} -\Phi \tilde A . \EQN chamlagh   \cr}
$$
This formulation would correspond to the Ramsey problem if it were not
for the missing constraint \Ep{T_fochH2}.
Following Jones, Manuelli, and Rossi (1997), we will solve for the
first-order conditions associated with equation
 \Ep{chamlagh}, and when it is evaluated
at a steady state, we can verify that constraint \Ep{T_fochH2} is
satisfied even though it has not been imposed.
Thus, if both the problem in expression
 \Ep{chamlagh} and the proper Ramsey problem
with constraint \Ep{T_fochH2} converge to a unique steady state, they will
converge to the same steady state.

The first-order conditions for equation \Ep{chamlagh} evaluated at the steady
state are
$$\EQNalign{
c\rm{:}&\ \ \ \ V_c = \theta
          \EQN T_SS;a \cr
n_{m}\rm{:}&\ \ \ V_{n_m} = - \theta F_e M_n      \EQN T_SS;b \cr
n_{h}\rm{:}&\ \ \ V_{n_h} = -\nu H_n \EQN T_SS;c \cr
x_{m}\rm{:}&\ \ \ 1 = F_e M_x                                    \EQN T_SS;d \cr
x_{h}\rm{:}&\ \ \ \theta = \nu H_x \EQN T_SS;e \cr
h\rm{:}&\ \ \ 1 = \beta \left(1-\delta_h + H_h
+ {\theta \over \nu} F_e M_h \right) \EQN T_SS;f \cr
k\rm{:}&\ \ \ 1 = \beta ( 1-\delta_k + F_k ).           \EQN T_SS;g \cr}
$$
Note that $V_{n_m}=V_{n_h}$, so by conditions \Ep{T_SS;b} and \Ep{T_SS;c},
$$
{\theta \over \nu} = {H_n \over F_e M_n},                        \EQN T_SSratio
$$
which we substitute into equation \Ep{T_SS;g},
$$
1 = \beta \left(1-\delta_h + H_h + H_n {M_h \over M_n}\right).   \EQN T_SSh
$$
Condition \Ep{T_SSh} coincides with constraint \Ep{T_fochH2}, evaluated in
a steady state. In other words, we have confirmed that the problem
\Ep{chamlagh} and the proper Ramsey problem with constraint \Ep{T_fochH2}
share the same steady state, under the maintained assumption that both
problems converge to a unique steady state.

What is the optimal $\tau^n$? The substitution of equation  \Ep{T_SS;e} into
equation \Ep{T_SSratio} yields
$$
H_x = {H_n \over F_e M_n}.                        \EQN T_SSratio2
$$
The household's first-order conditions \Ep{T_foch2;b} and
\Ep{T_foch2;c} imply in a steady state that
$$
(1-\tau^n) H_x = {H_n \over F_e M_n}.             \EQN T_SSratio22
$$
It follows immediately from equations \Ep{T_SSratio2} and \Ep{T_SSratio22} that
$\tau^n=0$. Given $\tau^n=0$, conditions \Ep{T_foch2;d} and \Ep{T_SS;d} imply
$\tau^m=0$. We conclude that in the present model neither labor nor
capital should be taxed in the limit.

\section{Should all taxes be zero?}
The optimal steady-state tax policy of the model in the previous section
is to set $\tau^k=\tau^n=\tau^m=0$. However, in general, this implies
$\tau^c\not=0$. To see this point, use  equation
 \Ep{T_foch2;b} and  $\tau^n=0$ to get
$$
1+\tau^c = {u_c \over u_\ell} F_e M_n.                      \EQN T_labor1
$$
From equations \Ep{T_SS;a} and \Ep{T_SS;b}
$$
F_e M_n = -{V_{n_m} \over V_c} = {u_\ell + \Phi u_{c\ell} c \over
              u_c + \Phi (u_c + u_{cc} c)  }.               \EQN T_labor2
$$
Hence,
$$
1+\tau^c = {u_c u_\ell + \Phi u_c u_{c\ell} c \over
       u_c u_\ell + \Phi (u_c u_\ell + u_{cc} u_\ell c)  }.  \EQN T_labor3
$$
As discussed earlier, a first-best solution without distortionary taxation
has $\Phi=0$, so $\tau^c$ should trivially be set equal to zero.
In a second-best solution, $\Phi>0$ and we get $\tau^c=0$ if and only
if
$$
u_c u_{c\ell} c =u_c u_\ell + u_{cc} u_\ell c,                \EQN T_labor4
$$
which is in general not satisfied. However, Jones, Manuelli, and
Rossi (1997) point out one interesting class of utility functions that is
consistent with equation \Ep{T_labor4}:
$$u(c, \ell)
=\cases{ {\displaystyle c^{1-\sigma} \over
          \displaystyle 1-\sigma} v(\ell) &if $\sigma>0, \sigma\not=1$ \cr
\noalign{\vskip.3cm}
         {\rm ln}(c) + v(\ell)                 &if $\sigma=1$.\cr}
$$
If a steady state exists, the optimal solution for these preferences
is eventually to set all taxes equal to zero. It follows that the
optimal plan involves collecting tax revenues in excess of expenditures
in the initial periods. When the government has amassed claims against
the private sector so large that the interest earnings suffice to
finance $g$, all taxes are set equal to zero. Since the steady-state
interest rate is $R=\beta^{-1}$, we can use the government's budget
constraint \Ep{T_gov} to find the corresponding value of
government indebtedness
$$
b= {\beta \over \beta -1} g < 0.
$$

%SW% \auth{Judd, Kenneth L.}
\auth{Jones, Larry E.} \auth{Manuelli, Rodolfo} \auth{Rossi, Peter E.}
\section{Concluding remarks}
Perhaps the most startling finding of this chapter is that
the optimal steady-state tax on physical capital in a
nonstochastic economy is equal to {\it zero}. The conclusion
follows immediately from time-additively separable utility,
a standard constant-returns-to-scale production technology,
competitive markets, and a complete set of flat-rate taxes.
%SW% The result
%SW% that capital should not be taxed in the steady state is
%SW% robust to whether or not the government must balance its budget
%SW% in each period and to any redistributional concerns
%SW% arising from a social welfare function. As a stark illustration,
%SW% Judd's (1985b)
%SW% example demonstrates that the
%SW% result holds when the government is constrained to run a
%SW% balanced budget and when it cares only about the workers
%SW% who are exogenously constrained to not hold any assets.
%SW% Thus, the capital owners who are assumed not to work
%SW% will be exempt from taxation in the steady state, and the
%SW% government will finance its expenditures solely by levying
%SW% wage taxes on the group of agents that it cares about.
%SW%
It is instructive to consider Jones, Manuelli, and Rossi's (1997)
extension of the no-tax result to labor income, or more precisely
human capital.  They ask rhetorically, Is physical capital special?
We are inclined to answer yes to this question for the following reason.
The zero tax on human capital is derived in a model where
the production of both human capital and ``efficiency units''
of labor show constant returns to scale in the stock of human
capital and the use of final goods but not raw labor
which otherwise enters as an input in the production functions.
These assumptions explain why the stream of future labor income
in the
household's present-value budget constraint in equation \Ep{T_bcPV2h}
is reduced to the first term in equation \Ep{T_simple}, which is
the value of the household's human capital at time 0.
Thus, the functional forms have made raw labor
disappear as an object for taxation in future periods.
Or in the words of Jones, Manuelli, and Rossi (1997, pp.\ 103 and 99),
``Our zero tax results are
driven by zero profit conditions. Zero profits follow from
the assumption of linearity in the accumulation technologies.
Since the activity `capital income' and the activity
`labor income' display constant returns to scale in
reproducible factors, their `profits' cannot enter the
budget constraint in equilibrium.'' But for alternative production
functions that make the endowment of raw labor reappear,
the optimal labor tax
would not be zero. It is for this reason that we think physical
capital is special: because the zero-tax result arises with
the minimal assumptions of the standard neoclassical growth
model, while the zero-tax result on labor income requires
that raw labor vanish from the agent's present-value
budget constraints.\NFootnote{One
special case of Jones, Manuelli, and Rossi's (1997)
framework with its zero-tax result for labor
is Lucas's (1988) endogenous growth model studied in chapter \use{growth}.
 Recall our alternative interpretation of that model as one without
any nonreproducible raw labor but just two reproducible
factors: physical and human capital. No wonder that raw
labor in Lucas's model does not affect the optimal labor
tax, since the model can equally well be thought of as an
economy without raw labor.}

\auth{Chari, V.V.} \auth{Christiano, Lawrence J.} \auth{Kehoe, Patrick J.}
\auth{Domeij, David} \auth{Heathcote, Jonathan}
The weaknesses of our optimal steady-state tax analysis
are that it says nothing about how long it takes to reach
the zero tax on capital income and how taxes and
any redistributive transfers are set during the transition
period. These questions have to be studied numerically,
as was done by Chari, Christiano, and Kehoe (1994),
though their paper does not involve any redistributional
concerns because of the assumption of a representative agent.
Domeij and Heathcote (2000)
construct a model with heterogeneous agents and incomplete
insurance markets to study the welfare implications
of eliminating capital income taxation.
Using earnings and wealth data from the United States,
they calibrate a stochastic process for labor earnings
that implies a wealth distribution of asset holdings
resembling the empirical one. Setting initial tax rates
equal to estimates of present taxes in the United States,
they study the effects of an unexpected policy reform
that sets the capital tax permanently equal to zero and
raises the labor tax to maintain long-run budget balance.
They find that a majority of households prefers
the  status quo to the tax reform because of the
distributional implications.

This example illustrates
the importance of a well-designed tax and transfer policy in the
transition to a new steady state.  In addition, as shown by Aiyagari (1995),
the optimal capital tax in a heterogeneous-agent model with incomplete
insurance markets is actually positive, even in the long run.
A positive capital tax is used to counter the tendency of
such an economy to overaccumulate capital because of too much
precautionary saving. We say  more about these heterogeneous-agent
models in chapter \use{incomplete}.\auth{Aiyagari,  Rao}\index{redistribution}%

\auth{Kocherlakota, Narayana R.} \auth{Golosov, Mikhail}  \auth{Tsyvinski, Aleh} %
Golosov, Kocherlakota,  and Tsyvinski  (2003) pursue another way
of disrupting the connection between stationary values of the two
key Euler equations that underlie Chamley and Judd's
zero-tax-on-capital outcome.  They put the Ramsey planner in a
private information environment in which it cannot observe the
hidden skill levels of  different households.  That impels the
planner to
design the tax system as an optimal dynamic
incentive mechanism that trades off current and continuation
values in an optimal way. We discuss such mechanisms for coping
with private information in chapter
\use{socialinsurance}. Because the information problem alters the
planner's Euler equation for the household's consumption, Chamley
and Judd's result does not hold for this environment.

An  assumption maintained throughout the chapter has been that
the government can commit to future tax rates when solving
the Ramsey problem at time 0. As noted earlier,
taxing the capital stock at time 0 amounts to lump-sum
taxation and therefore disposes of
distortionary taxation. It follows that
a government without a commitment technology would be tempted
in future periods to renege on its promises and levy a
confiscatory tax on capital. An interesting question arises:
 can the incentive to maintain a good reputation replace
a commitment
technology? That is, can a  promised policy be sustained in
an equilibrium because the government  wants to preserve its
reputation? Reputation involves history dependence and incentives and
will be studied in chapter \use{fiscalmonetary}.



%\section{Exercises}
\showchaptIDfalse
\showsectIDfalse
\section{Exercises}
\showchaptIDtrue
\showsectIDtrue
\noindent
{\it Exercise \the\chapternum.1} \quad {\bf A small open economy}
(Razin and Sadka, 1995)
\medskip
\noindent
Consider the nonstochastic model with capital and labor in this
chapter, but assume that
the economy is a small open economy that cannot affect the
international rental rate on capital, $r^*_t$.
Domestic firms can rent any amount of capital at this price, and
the households and the government can choose to go short or long in
the international capital market at this rental price. There is
no labor mobility across countries.
We retain the assumption that the government levies a tax $\tau^n_t$ on
each household's labor income, but households no longer have
to pay taxes on their capital income. Instead, the
government levies a tax $\hat \tau^k_t$ on domestic firms' rental
payments to capital regardless of the capital's origin (domestic
or foreign). Thus, a
domestic firm faces a total cost of $(1+\hat \tau^k_t)r^*_t$ on a
unit of capital rented in period $t$.

\medskip
\noindent{\bf a.} Solve for the optimal capital tax $\hat \tau^k_t$.

\medskip
\noindent{\bf b.} Compare the optimal tax policy of this small open
economy to that of the closed economy of this chapter.


\medskip
\noindent {\it Exercise \the\chapternum.2} \quad {\bf Consumption taxes}
\medskip
\noindent
Consider the nonstochastic model with capital and labor in this
chapter, but instead of labor and capital taxation assume that the
government sets labor and consumption taxes, $\{\tau^n_t, \tau^c_t\}$.
Thus, the household's present-value budget constraint is now given by
$$
\sum_{t=0}^\infty q^0_t (1+\tau^c_t) c_t =
\sum_{t=0}^\infty q^0_t (1-\tau^n_t) w_t n_t
   + \left[r_{0} + 1-\delta \right] k_0 + b_0.
$$

\medskip
\noindent{\bf a.} Solve for the Ramsey plan.

\medskip
\noindent{\bf  b.} Suppose that the solution to the Ramsey problem converges
to a steady state.
Characterize the optimal limiting sequence of consumption taxes.

\medskip
\noindent{\bf  c.} In the case of capital taxation, we imposed an
exogenous upper bound on $\tau^k_0$. Explain why a similar exogenous
restriction on $\tau^c_0$ is needed to ensure an interesting
Ramsey problem. ({\it Hint:} Explore the implications of setting
$\tau^c_t=\tau^c$ and $\tau^n_t=-\tau^c$ for all $t\geq 0$,
where $\tau^c$ is a large positive number.)


\medskip
\noindent {\it Exercise \the\chapternum.3} \quad {\bf
Specific utility function} (Chamley, 1986)
\medskip
\noindent
Consider the nonstochastic model with capital and labor in this
chapter, and assume that the period utility function in equation \Ep{cham1}
is given by
$$
u(c_t,\ell_t) = {c_t^{1-\sigma} \over 1-\sigma} + v(\ell_t),
$$
where $\sigma>0$. When $\sigma$ is equal to $1$, the term
$c_t^{1-\sigma} / (1-\sigma)$ is replaced by $\log (c_t)$.

\medskip
\noindent{\bf  a.} Show that the optimal tax policy in this economy is to set
capital taxes equal to zero in period 2 and from there on, i.e.,
$\tau^k_t=0$ for $t\geq 2$. ({\it Hint:} Given the preference specification,
evaluate and compare equations \Ep{T_focRq} and \Ep{cham19;a}.)
%%using expressions \Ep{cham14;a} and \Ep{cham17}.)

\medskip
\noindent{\bf  b.} Suppose there is uncertainty in the economy, as
in the stochastic model with capital and labor in this chapter.
Derive the optimal {\it ex ante capital tax rate\/} for $t\geq 2$.

\medskip
\noindent {\it Exercise \the\chapternum.4} \quad {\bf
Two labor inputs} (Jones, Manuelli, and Rossi, 1997)
\medskip
\noindent
Consider the nonstochastic model with capital and labor in this
chapter, but assume that there are two labor inputs, $n_{1t}$ and
$n_{2t}$, entering the production function, $F(k_t,n_{1t},n_{2t})$.
The household's period utility function is still given by $u(c_t,\ell_t)$
where leisure is now equal to
$$
\ell_t = 1 - n_{1t} - n_{2t}.
$$
Let $\tau^n_{it}$ be the flat-rate tax at time $t$ on wage earnings
from labor $n_{it}$, for $i=1,2$, and $\tau^k_t$ denotes the
tax on earnings from capital.

\medskip
\noindent{\bf  a.} Solve for the Ramsey plan. What is the relationship
between the optimal tax rates $\tau^n_{1t}$ and $\tau^n_{2t}$ for
$t\geq 1$? Explain why your answer is different for period $t=0$.
As an example, assume that $k$ and $n_1$ are complements
while $k$ and $n_2$ are substitutes.

\medskip

We now assume that the period utility function is given by
$u(c_t,\ell_{1t},\ell_{2t})$ where
$$
\ell_{1t} = 1-n_{1t},\qquad \hbox{\rm and} \qquad \ell_{2t} = 1-n_{2t}.
$$
Further, the government is now constrained to
set the same tax rate on both types of labor, i.e.,
$\tau^n_{1t}=\tau^n_{2t}$ for all $t\geq 0$.

\medskip
\noindent{\bf  b.} Solve for the Ramsey plan. ({\it Hint:} Using the household's
first-order conditions, we see that the restriction
$\tau^n_{1t}=\tau^n_{2t}$ can be incorporated into the Ramsey
problem by adding the constraint
$u_{\ell_1}(t) F_{n_2}(t) = u_{\ell_2}(t) F_{n_1}(t)$.)

\medskip
\noindent{\bf  c.} Suppose that the solution to the Ramsey problem converges
to a steady state where the constraint that the two labor taxes should
be equal is binding. Show that the limiting capital tax is not zero
unless $F_{n_1} F_{n_2k} = F_{n_2} F_{n_1k}$.


%%%AAAAA


\medskip
\noindent {\it Exercise \the\chapternum.5} \quad {\bf
Another specific utility function}

\medskip
\noindent
   Consider the following optimal taxation problem.
There is no uncertainty.  There is one good that is produced by
labor $x_t$ of the representative household, and that can be divided
among private consumption $c_t$ and government consumption
$g_t$ subject to
$$ c_t + g_t = 1-x_t. \eqno(0) $$
The good is produced by zero-profit competitive
firms that pay the worker a pretax wage of $1$ per unit of $1-x_t$ (i.e.,
the wage is tied down by the linear technology).
A representative consumer maximizes
$$ \sum_{t=0}^\infty \beta^t u(c_t, x_t)  \eqno(1) $$
subject to the sequence of budget constraints
$$ c_t + q_t b_{t+1} \leq (1-\tau_t) (1-x_t) + b_t \eqno(2) $$
where $c_t$ is consumption, $x_t$ is leisure, $q_t$ is the price of
consumption at $t+1$ in units of time $t$ consumption, and
$b_t$ is a stock of one-period IOUs owned by the household and falling
due at time $t$.  Here $\tau_t$ is a flat-rate
tax on the household's labor supply $1-x_t$.   Assume that $u(c,x) = c- .5(1-x)^2$.

\medskip
\noindent{\bf a.}  Argue that in a competitive equilibrium,
$q_t = \beta$ and $x_t=\tau_t$.
\medskip
\noindent{\bf b.}  Argue that in a competitive equilibrium with
$b_0=0$ and $\lim_{t\rightarrow \infty} \beta^t b_t =0$,
 the sequence of budget constraints  (2) imply
the following single  intertemporal constraint:
$$ \sum_{t=0}^\infty  \beta^t \left(c_t - (1-x_t)(1- \tau_t)\right) =0 .$$
\medskip
Given an exogenous sequence of government purchases $\{g_t\}_{t=0}^\infty$,
a government wants to maximize (1) subject both
to the budget
constraint
$$  \sum_{t=0}^\infty \beta^t \left( g_t - \tau_t (1-x_t) \right) =0
  \eqno(3) $$
and to the household's first-order condition
$$ x_t = \tau_t. \eqno(4) $$


\medskip
\noindent{\bf c.}  Consider the following government expenditure
process defined for $t \geq 0$:

%$$ q_t^0(s^t) =\cases{ \beta^t, & if $s^t = \tilde s^t$; \cr
%                       0, & otherwise ;\cr}  $$
$$g_t =\cases{0, & if $t$ is even;\cr
             .2, & if $t$ is odd; \cr} $$
Solve the Ramsey plan.   Show that the optimal tax rate
is given by
$$ \tau_t = \overline \tau
%  ANSWER = {\mu_0 \over 1-2\mu_0} \quad
\hskip.5cm \forall t\geq0 . $$
Please compute the value for
$\overline \tau $ when $\beta=.95$.
\medskip

\noindent{\bf d.}  Consider the following government expenditure
process defined for $t \geq 0$:
$$g_t =\cases{.2, & if $t$ is even;\cr
              0, & if $t$ is odd; \cr} $$
Show that $\tau_t =   \overline \tau \ \ \forall t \geq 0$.
Compute $\overline \tau$ and comment on whether it is larger or
smaller than the value you computed in part (c).

\medskip
\noindent{\bf e.}  Interpret your results in parts c and d
in terms of ``tax-smoothing.''

\medskip
\noindent{\bf f.}
Under what circumstances, if any, would $\overline \tau =0$?
\vfil\eject

\medskip
\noindent {\it Exercise \the\chapternum.6} \quad {\bf
Yet another specific utility function}

\medskip
\noindent
   Consider an economy with a representative household
with preferences over streams of consumption $c_t$ and labor supply
$n_t$  that are ordered by
$$ \sum_{t=0}^\infty \beta^t (c_t - u_1 n_t - .5 u_2 n_t^2) ,
\quad \beta \in (0,1) \eqno(1) $$
where $u_1, u_2 >0$.
The household operates a linear technology
$$ y_t = n_t  , \eqno(2)$$
where $y_t$ is output.  There is no uncertainty.
 There is a government that finances an exogenous
stream of government purchases $\{g_t\}$ by a flat rate tax $\tau_t$
on labor.    The feasibility condition for the economy is
$$ y_t = c_t + g_t. \eqno(3) $$

At time $0$ there are complete markets in dated consumption goods.
Let $q_t$ be the price of a unit of consumption at date $t$ in terms
of date $0$ consumption. The budget constraints for the household
and the government, respectively,  are
$$ \sum_{t=0}^\infty q_t [(1-\tau_t) n_t - c_t ] =0 \eqno(4) $$
$$ \sum_{t=0}^\infty q_t (\tau_t n_t - g_t ) =0. \eqno(5) $$

\medskip
\noindent{\bf Part I.}  Call a tax rate process $\{\tau_t\} $
budget feasible if it satisfies (5).
\medskip
\noindent{\bf a.}   Define a competitive
equilibrium with taxes.

\medskip
\noindent{\bf Part II.}  A Ramsey planner chooses a competitive
equilibrium to maximize  (1).
\medskip
\noindent{\bf b.}  Formulate   the Ramsey problem.
Get as far as you can in solving it for the Ramsey plan, i.e.,
 compute the competitive equilibrium price system and
tax policy under the Ramsey plan.
How does the Ramsey plan pertain to ``tax smoothing?''
\medskip
\noindent{\bf c.}  Consider two possible government expenditure sequences:
Sequence A:  $\{g_t\} = \{0, g, 0 , g, %%%\hfil\break
0, g,  \ldots\}$. Sequence B:
$\{g_t\}= \{\beta g, 0, \beta g, 0, \beta g, 0, \ldots \}$.   Please
tell how the   Ramsey equilibrium tax rates and interest rates differ
across the two equilibria associated with sequence A and sequence B.





\medskip
\noindent {\it Exercise \the\chapternum.7} \quad {\bf
Comparison of tax systems}

\medskip
\noindent
Consider an economy with a representative household that
orders consumption, leisure streams $\{c_t, \ell_t\}_{t=0}^\infty$ according
to
$$ \sum_{t=0}^\infty \beta^t u(c_t, \ell_t), \qquad \beta \in (0,1) $$
where $u$ is  increasing, strictly concave, and
twice continuously differentiable in $c$ and $\ell$.
The household is endowed with one unit of time that can be used for leisure
$\ell_t$ and labor $n_t$; $\ell_t+n_t=1$.

A single good is produced with labor $n_t$ and capital $k_t$ as inputs.
The output can be consumed by households, used by the government, or used
to augment the capital stock. The technology is described by
$$c_t + g_t + k_{t+1} = F(k_t,n_t) + (1-\delta) k_t, $$
where $\delta \in (0,1)$ is the rate at which capital depreciates, and
$g_t\geq0$ is an exogenous amount of government
purchases in period $t$. The production function
$F(k,n)$ exhibits constant returns to scale.

The government finances its purchases by levying
two flat-rate, time varying taxes $\{\tau^n_t, \,\tau^a_t\}_{t=0}^\infty$.
$\tau^n_t$ is a tax on labor earnings and the tax revenue from
this source in period $t$ is equal to $\tau^n_t w^n_t n_t$, where $w_t$ is
the wage rate. $\tau^a_t$ is
a tax on capital earnings and the asset value of the capital stock net of
depreciation. That is, the tax revenue from this source in period
$t$ is equal to $\tau^a_t (r_t + 1-\delta) k_t$, where $r_t$ is the rental
rate on capital. We assume that the tax rates in period $0$ cannot be
chosen by the government but must be set equal to zero,
$\tau^n_0 = \tau^a_0 = 0$.
The government can trade  one-period bonds. We assume that
there is no outstanding government debt at time $0$.

\medskip
\noindent{\bf  a.} Formulate the Ramsey problem, and characterize the optimal
government policy using the primal approach to taxation.

\medskip
\noindent{\bf  b.} Show that if there exists a steady state Ramsey
allocation, the limiting tax rate $\tau^a_\infty$ is zero.

\medskip
\noindent
Consider another economy with identical preferences, endowment,
technology and government expenditures but where labor taxation is
forbidden. Instead of a labor tax this economy must
use a consumption tax $\tilde \tau^c_t$. (We use a tilde
to distinguish outcomes in this economy as compared to the previous
economy.) Hence, this economy's tax revenues in period $t$
are equal to $\tilde \tau^c_t \tilde c_t
+ \tilde \tau^a_t (\tilde r_t + 1-\delta) \tilde k_t$.
We assume that the tax rates in period $0$ cannot be
chosen by the government but must be set equal to zero,
$\tilde \tau^c_0 = \tilde \tau^a_0 = 0$. And as before,
the government can trade in one-period bonds and there
is no outstanding government debt at time $0$.

\medskip
\noindent{\bf  c.} Formulate the Ramsey problem, and characterize the optimal
government policy using the primal approach to taxation.
\medskip

\noindent
Let the allocation and tax rates that solve the Ramsey problem
in question {\bf a}
be given by $\Omega \equiv \{c_t, \ell_t, n_t, k_{t+1},
\tau^n_t, \,\tau^a_t\}_{t=0}^\infty$. And let the allocation
and tax rates that solve the Ramsey problem in question {\bf c}
be given by  $\tilde \Omega \equiv
\{\tilde c_t, \tilde \ell_t, \tilde n_t,
\tilde k_{t+1},\tilde \tau^c_t,$ $\tilde \tau^a_t\}_{t=0}^\infty$.

\medskip
\noindent{\bf  d.} Make a careful argument for how the allocation
$\{c_t, \ell_t, n_t, k_{t+1}\}_{t=0}^\infty$ compares
to the allocation $\{\tilde c_t, \tilde \ell_t, \tilde n_t,
\tilde k_{t+1}\}_{t=0}^\infty$.

\medskip
\noindent{\bf  e.} Find expressions for the tax rates
$\{\tilde \tau^c_t, \,\tilde \tau^a_t\}_{t=1}^\infty$
solely in terms of
$\{\tau^n_t, \,\tau^a_t\}_{t=1}^\infty$.


\medskip
\noindent{\bf  f.} Write down the government's present
value budget constraint in the first economy which holds with
equality for the allocation and tax rates as given by
$\Omega$. Can you manipulate this expression so that
you arrive at the government's present value budget constraint in
the second economy by only using
your characterization of $\tilde \Omega$
in terms of $\Omega$ in questions {\bf d} and {\bf e}?



\medskip
\noindent {\it Exercise \the\chapternum.8}\quad {\bf
Taxes on capital, labor, and consumption, I}

\medskip
\noindent
Consider the nonstochastic version of the model with capital and labor in this
chapter, but now assume that the
government sets labor, capital, and consumption taxes, $\{\tau^n_t, \tau^k_t, \tau^c_t\}$.
Thus, the household's present-value budget constraint is
$$
\sum_{t=0}^\infty q^0_t (1+\tau^c_t) c_t =
\sum_{t=0}^\infty q^0_t (1-\tau^n_t) w_t n_t
   + \left[r_{0}(1-\tau^k_t) + 1-\delta \right] k_0 + b_0.
$$

\medskip
\noindent{\bf a.} Suppose that there are upper bounds on $\tau^k_0$ and $\tau^c_0$. Solve for the Ramsey plan.  Is the Ramsey tax system
unique?

\medskip
\noindent{\bf  b.} Take an arbitrary (non-optimal) tax policy that sets $\tau^c_t = 0, \tau^n_t = \hat \tau^n, \tau^k_t = \hat \tau^k$
for all $t \geq 0$.  Let the equilibrium allocation under this policy be $\{\hat c_t, \hat n_t, \hat k_{t+1}\}_{t=0}^\infty$.
Show that you can support the same $(\hat \cdot)$ allocation with a  different tax policy, namely one that sets  $\tau^k_t =0 $ for all $t$ and time varying
taxes $\tau^c_t, \tau^n_t$ for $t \geq 0$.    Find expressions for the time varying taxes $\tau^c_t, \tau^n_t$ as functions of the original $(\hat \cdot)$
tax policy and the $(\hat \cdot)$  equilibrium allocation.

\medskip
\noindent{\bf  c.} Interpret the time-varying taxes that you computed in part {\bf b} in terms of the intertemporal distortions
on the final goods $(c_t, \ell_t)$  that are in effect  induced
by a constant tax on capital. (By `final goods' we refer to the goods that appear in the representative household's utility function.)


\medskip
\noindent{\bf d.} Interpret the outcomes in parts {\bf b} and {\bf c} in terms of the following advice that comes from the asymptotic properties
of the Ramsey plan. Lesson 1: don't distort intertemporal margins in the limit.  Lesson 2: distort intratemporal margins in the same way each period.



\medskip

\medskip
\noindent {\it Exercise \the\chapternum.9}\quad {\bf
Taxes on capital, labor, and consumption, II}

\medskip
\noindent
Consider the nonstochastic version of the  model with capital and labor in this
chapter, but now assume that the
government sets labor, capital, and consumption taxes, $\{\tau^n_t, \tau^k_t, \tau^c_t\}$.
The household's present-value budget constraint is
$$
\sum_{t=0}^\infty q^0_t (1+\tau^c_t) c_t =
\sum_{t=0}^\infty q^0_t (1-\tau^n_t) w_t n_t
   + \left[r_{0}(1-\tau^k_t) + 1-\delta \right] k_0 + b_0.
$$

\medskip
\noindent{\bf a.} Suppose that there are upper bounds on $\tau^k_0$ and $\tau^c_0$. Solve for the Ramsey plan.  Is the Ramsey tax system
unique?

\medskip
\noindent{\bf  b.} Take an arbitrary (non-optimal) tax policy that involves $\tau^c_t = 0, \tau^n_t = \hat \tau^n, \tau^k_t = \hat \tau^k$
for all $t \geq 0$.  Let the equilibrium allocation under this policy be $\{\hat c_t, \hat n_t, \hat k_{t+1}\}_{t=0}^\infty$.


\medskip
\noindent{\bf  c.} Freeze the capital tax policy $\tau^k_t = \hat \tau^k$ for all $t \geq 0$.  But now allow $\{\tau^c_t, \tau^n_t\}_{t=0}^\infty$
to vary through time.  Can you find a tax policy within this class that solves the Ramsey problem?


\medskip
\noindent{\bf d.} Interpret the outcome in part  {\bf c} in terms of the following advice that comes from the asymptotic properties
of the Ramsey plan. Lesson 1: don't distort intertemporal margins in the limit.  Lesson 2: distort intratemporal margins in the same way each period.

\vfil\eject

\medskip
\noindent {\it Exercise \the\chapternum.10}\quad {\bf
Lucas and Stokey (1983) model}

\medskip
\noindent Consider the following version of Lucas and Stokey's (1983) model
of optimal taxation with complete markets.  A stochastic state $s_t$ at time $t$ determines
an exogenous shock to government purchases $g_t(s_t)$. The history of events
$s^t$ indexes history-contingent commodities:
 $c_t(s^t)$, $\ell_t(s^t)$, and $n_t(s^t)$ are the
household's consumption, leisure, and labor at time $t$ given history
$s^t$. There is no capital. % Let $u_c(s^t)$
% and so on denote the values
%of the indicated objects at time $t$ for history $s^t$,
%evaluated at an allocation to be understood from the
%context.

A representative household's preferences are ordered by
$$
\sum_{t=0}^\infty\ \ \sum_{s^t} \beta^t \pi_t(s^t) u[c_t(s^t), \ell_t(s^t)] ,
                                                             \eqno(1) $$
where $\ell_t \in [0,1]$.
The household has quasi-linear utility function
$$ u(c,\ell) = c + H(\ell) \eqno(2) $$
where $H' >0, H''<0$, and $H'''$ is well defined.
The production function is linear in the only input labor, so the feasibility
condition at $t,s^t$ is
$$c_t(s^t) + g_t(s^t) = 1 - \ell_t
 $$

Given history $s^t$ at time $t$, the government finances its exogenous
purchase $g_t(s_t)$ and any debt obligation by levying a flat-rate tax on earnings
from labor at rate $\tau^n_t(s^t)$, and by issuing
state-contingent debt. Let $b_{t+1}(s_{t+1}|s^t)$ be government indebtedness to
the private sector at the beginning of period $t+1$ if event $s_{t+1}$ is
realized. This state-contingent asset is traded in period $t$ at the price
$p_t(s_{t+1}|s^t)$, in terms of time $t$ goods. The government's budget
constraint is
$$\eqalign{
 g_t(s_t) =  & \tau^n_t(s^t) w_t(s^t) n_t(s^t)                   \cr
             & + \sum_{s_{t+1}} p_t(s_{t+1} | s^t) b_{t+1}(s_{t+1} | s^t)
               - b_t(s_t | s^{t-1}),
                                                              \cr}\eqno(3)
$$
where  $w_t(s^t)$ is the competitive equilibrium  wage rate for labor.

\medskip

\noindent{\bf a.}  Formulate the Ramsey problem using the `primal approach'.  Get as far as you can in solving it.

\medskip
\noindent{\bf b.}  Describe how $c_t, \ell_t$, and $\tau^n_t$ behave as  functions of $g_t$.

\medskip
\noindent{\bf c.}  Suppose that government expenditures $g_t$ are drawn from the following stochastic process:
$g_t = 0$ for $t= 0, \ldots , T-1$; for all  $t \geq T, g_t = .5$ with probability $\alpha \in (0,1)$, and for all $t \geq T, g_t = 0 $ with probability
$(1-\alpha)$.  All uncertainty about $g_t$ for $t \geq T$ is resolved at time $T$.  Describe the government's optimal tax-debt strategy as it unfolds as
time passes and chance occurs.





\medskip
\noindent {\it Exercise \the\chapternum.11} \quad {\bf
Another Lucas-Stokey economy}

\medskip
\noindent
Consider the following economy {\it without capital}.  There is an exogenous state $s_t$ governed by an $S$-state Markov chain with initial
distribution over states $\pi_0$ and transition matrix $P$.  Let $s^t = s_t, s_{t-1}, \ldots, s_0$.   There is one good in the economy
produced by the linear technology
$$  y_t(s^t) = n_t (s^t),$$
where $n_t = 1 - l_t$ is the amount of labor supplied by a representative household and $y_t$ is total output. The consumer is endowed with one
unit of leisure each period and can choose $l_t \in [0, 1]$.  A government
consumes an exogenous stream of government expenditures $g_t(s^t)$, while the representative household consumes the endogenous stream of
consumption $c_t(s^t)$.  Feasible allocations satisfy
$$ y_t(s^t) = c_t(s^t) + g_t(s^t) $$
and $c_t(s^t) \geq 0$.
The  representative household orders consumption-leisure plans $\{c_t(s^t), l_t(s^t)\}$ according to
$$ \sum_{t=0}^\infty \sum_{s^t} \beta^t \Pi_t(s^t) [ c_t(s^t) + H (l_t(s^t)) ]  \leqno(1) $$
where $\Pi_t(s^t)$ is the joint probability distribution over history $s^t$ induced by  $P, \pi_0$, and $H(l)$ satisfies
$H'> 0, H''< 0, H(0) > 1, H(1) > 0 $.
\medskip

\noindent We ask you to analyze a competitive equilibrium with distorting taxes.  In particular, the government finances its expenditures
by levying a flat rate tax $\tau_t^c(s^t)$ on time $t$, history $s^t$ consumption, perhaps also  issuing history-contingent government debt.   There are no other taxes.
Let $q_t^0(s^t)$ be
the (Arrow-Debreu) price of one unit of consumption at $t, s^t$.   The (Arrow-Debreu version of the) government's budget constraint is
$$ \sum_{t=0}^\infty \sum_{s^t} q_t^0(s^t) g_t(s^t) + q_0^0(s^0) b_0 = \sum_{t=0}^\infty \sum_{s^t} q_t^0(s^t) \tau_t^c(s^t) c_t(s^t) , \leqno(2) $$
where $b_0$ is initial government debt at the beginning of time $0$ measured in units of time $0$ consumption goods.

\medskip

\noindent Please consider the following special case of the economy.  For $t \neq 2$, government expenditures are constant at
level $g=g_L = .1$.  But at $t=2$, government expenditures are $g_H = .2 $ with probability $.5$ and again $g_L=.1$ with probability $.5$.

\medskip
\noindent{\bf a.} (``Finding the state is an art'')  \quad Please define a state space and corresponding
 Markov chain for this economy.  Please  completely specify the state
space $S$, the initial distribution $\pi_0$, and the transition matrix $P$. {\bf Hint:}  Try defining the state as a $(t, g)$ pair.
Try getting by with these   5 states:\hfil\break
$(0, g_L), (1, g_L), (2, g_L), (2, g_H), (t \geq 3, g_L)$, say, and call them states $1, 2, 3, 4, 5$, respectively.  Then
take the data supplied and create $\pi_0$ and $P$.

\medskip
\noindent{\bf b.} For the Markov chain that you created in part {\bf a}, please compute unconditional probabilities over  histories  at dates $t \geq 1$,
i.e., please compute $\Pi_t(s^t), t \geq 1$.

\medskip
\noindent {\bf c.} Please define an Arrow-Debreu style competitive equilbrium with distorting taxes for this environment, with all trades at time $0$.

\medskip
\noindent{\bf d.} Please define an  Arrow-securities style competitive equilibrium with distorting taxes, with trades each period of one-period ahead
Arrow securities.

\medskip
\noindent {\bf e.}  Temporarily suppose that $b_0 = 0$.  Consider a government policy that always runs a balanced budget budget, i.e.,
that  sets $g_t(s^t) = \tau_t^c(s^t) c_t(s^t)$ for all $t, s^t$.  Describe  outcomes (Arrow securities prices and the allocation) in a competitive
equilibrium with sequential trading of Arrow securities.

\medskip
\noindent{\bf f.}  Now suppose that $b_0 >0$.  Consider a government policy that always runs a balanced budget budget, i.e.,
that  sets $\tau_t^c(s^t) c_t(s^t)$ at a level that guarantees that total government debt owed at time $t\geq 1$ equals $b_0$ for all $t, s^t$.  (Here the
government always runs what the IMF calls a  balanced budget `net-of-interest'.  In doing this, it always rolls over its one-period debt $b_t = b_0$.)
Please find a formula for the rate of return paid on government debt
and describe precisely the quantities of  history-contingent bonds issued by the government at each $t, s^t$.  Please describe how the tax rate
$\tau_t^c(s^t)$ depends on $b_0$.

\medskip


\noindent{\bf g.}  Continuing to assume that  $b_0  >0$, now consider another government fiscal policy.  Here the government
sets a constant tax rate $\bar \tau^c = \tau_t^c(s^t)$ for all $t, s^t$.  The tax rate is set to satisfy the time $0$ Arrow-Debreu
budget constraint (2).  Please tell how to compute $\bar \tau^c$. Please describe the Arrow securities that the government
issues or purchases at each state. (Please carefully take into account how you have defined states in part {\bf a}.)

\medskip
\noindent{\bf h.} Using the labeling of states described in part {\bf a},
please describe the payouts on the government securities for the following two histories  for this economy:
$$ (1, 2, 3, 5, 5, 5, 5, \ldots) $$
and
$$ (1, 2, 4, 5, 5, 5, 5, \ldots).$$

\medskip
\noindent{\bf i.}  Please formulate and solve a Ramsey problem for this economy, assuming  that $\tau_t^c(s^t)$ is the only tax that the government
can impose.  Please state the Ramsey problem carefully and describe in detail an algorithm for computing all of the objects that comprise a Ramsey plan.

\medskip
\noindent{\bf j.}  Please compare the Ramsey plan that you computed in part {\bf i} with the arbitrary policies studied in parts {\bf f} and {\bf g}.

\medskip

\noindent{\bf k.} Define a ``continuation of a Ramsey plan.''  For this economy, is a continuation of a Ramsey plan a Ramsey plan? Please explain.

\medskip
\noindent {\it Exercise \the\chapternum.12} \quad {\bf
Yet another Lucas-Stokey economy}

\medskip

\noindent Consider the following economy without capital.  There is an exogenous state $s_t$ governed by an $S$-state Markov chain with initial
distribution over states $\pi_0$ and transition matrix $P$.  Let $s^t = s_t, s_{t-1}, \ldots, s_0$.   There is one good in the economy
produced by the linear technology
$$  y_t(s^t) = n_t (s^t),$$
where $n_t = 1 - l_t$ is the amount of labor supplied by a representative household and $y_t$ is total output. The consumer is endowed with one
unit of leisure each period and can choose $l_t \in [0, 1]$. A government
consumes an exogenous stream of government expenditures $g_t(s^t)$, while the representative household consumes the endogenous stream of
consumption $c_t(s^t)$.  Feasible allocations satisfy
$$ y_t(s^t) = c_t(s^t) + g_t(s^t) $$
and $c_t(s^t) \geq 0$.
The  representative household orders consumption-leisure plans $\{c_t(s^t), l_t(s^t)\}$ according to
$$ \sum_{t=0}^\infty \sum_{s^t} \beta^t \Pi_t(s^t) [ u(c_t(s^t)) + l_t(s^t) ]  \leqno(1) $$
where $\Pi_t(s^t)$ is the joint probability distribution over history $s^t$ induced by  $P, \pi_0$, and $u(c)$ satisfies
$u'> 0, u''< 0, u'(0) = +\infty $.
\medskip

\noindent We ask you to analyze a competitive equilibrium with distorting taxes.  In particular, the government finances its expenditures
by levying a flat rate tax $\tau_t^n(s^t)$ on time $t$, history $s^t$ labor, perhaps also  issuing history-contingent government debt.  Let $q_t^0(s^t)$ be
the (Arrow-Debreu) price of one unit of consumption at $t, s^t$.  There are no other taxes.  The(Arrow-Debreu version of the) government's budget constraint is
$$ \sum_{t=0}^\infty \sum_{s^t} q_t^0(s^t) g_t(s^t) + q_0^0(s^0) b_0  = \sum_{t=0}^\infty \sum_{s^t} q_t^0(s^t) \tau_t^n(s^t) w_t(s^t) n_t(s^t), \leqno(2) $$
where $b_0$ is initial government debt at the beginning of time $0$ and $w_t(s^t)$ is a pre-tax wage rate rate paid to labor.

\medskip

\noindent Please consider the following special case of the economy.  For $t \neq \{2,3\}$, government expenditures are constant at
level $g=g_L = .1$.  But at $t=2$ and $t=3$ government expenditures are $g_H = .2 $ with probability $.5$ and  $g_L=.1$ with probability $.5$.


\medskip
\noindent{\bf a.} (``Finding the state is an art'')  \quad Please define a state space and corresponding
 Markov chain for this economy.  Please  completely specify the state
space $S$, the initial distribution $\pi_0$, and the transition matrix $P$.  {\bf Hint:}  Try defining the state as a $(t, g)$ pair.
Try getting by with these   5 states:\hfil\break
$(0, g_L), (1, g_L), (2, g_L), (2, g_H),t \geq 3, g_L)$, say, and call them states $1, 2, 3, 4, 5$, respectively.  Then
take the data supplied and create $\pi_0$ and $P$.

\medskip
\noindent{\bf b.} For the Markov chain that you created in part {\bf a}, please compute unconditional probabilities over  histories  at dates $t \geq 1$,
i.e., please compute $\Pi_t(s^t), t \geq 1$.

\medskip
\noindent {\bf c.} Please define an Arrow-Debreu style competitive equilbrium with distorting taxes for this environment, with all trades at time $0$.

\medskip
\noindent{\bf d.} Please define an  Arrow-securities style competitive equilibrium with distorting taxes, with trades each period of one-period ahead
Arrow securities.

\medskip
\noindent {\bf e.}  Temporarily suppose that $b_0 = 0$.  Consider a government policy that always runs a balanced budget budget, i.e.,
that  sets $g_t(s^t) = \tau_t^n(s^t) w_t(s^t) n_t(s^t)$ for all $t, s^t$.  Describe  outcomes (Arrow securities prices and the allocation) in a competitive
equilibrium with sequential trading of Arrow securities.

\medskip
\noindent{\bf f.}  Now suppose that $b_0 >0$.  Consider a government policy that always runs a balanced budget budget, i.e.,
that  sets $\tau_t^n(s^t) w_t(s^t) n_t(s^t)$ at a level that guarantees that total government debt owed at time $t\geq 1$ equals $b_0$ for all $t, s^t$.  (Here the
government always runs what the IMF calls a  balanced budget `net-of-interest'.  In doing this, it always rolls over its one-period debt $b_t = b_0$.)
Please find a formula for the rate of return paid on government debt
and describe precisely the quantities of  history-contingent bonds issued by the government at each $t, s^t$.  Please describe how the tax rate
$\tau_t^n(s^t)$ depends on $b_0$.

\medskip


\noindent{\bf g.}  Continuing to assume that  $b_0  >0$, now consider another government fiscal policy.  Here the government
sets a constant tax rate $\bar \tau^n = \tau_t^n(s^t)$ for all $t, s^t$.  The tax rate is set to satisfy the time $0$ Arrow-Debreu
budget constraint (2).  Please tell how to compute $\bar \tau^n$. Please describe the Arrow securities that the government
issues or purchases at each state. (Please carefully take into account how you have defined states in part {\bf a}.)

\medskip
\noindent{\bf h.} Please describe the payouts on the government securities for all possible histories for this economy.

\medskip
\noindent{\bf i.}  Please formulate and solve a Ramsey problem for this economy, assuming  that $\tau_t^n(s^t)$ is the only tax that the government
can impose.  Please state the Ramsey problem carefully and describe in detail an algorithm for computing all of the objects that comprise a Ramsey plan.

\medskip
\noindent{\bf j.}  Please compare the Ramsey plan that you computed in part {\bf i} with the arbitrary policies studied in parts {\bf f} and {\bf g}.

\medskip

\noindent{\bf k.} Define a ``continuation of a Ramsey plan.''  For this economy, is a continuation of a Ramsey plan a Ramsey plan? Please explain.


\medskip
\noindent {\it Exercise \the\chapternum.13} \quad {\bf
Positive initial debt}

\medskip
\noindent Please describe outcomes in  modified versions of examples 1, 2, and 3 of section \use{sec:labor_tax_smoothing} in which $b_0 >0$ rather than $b_0 = 0$.
