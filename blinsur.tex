
\input grafinp3
\input psfig
%\showchaptIDtrue
%\def\@chaptID{4.}
%\eqnotracetrue
%\hbox{}
\footnum=0
\chapter{Optimal Unemployment Insurance\label{uninsur1}}

\section{History-dependent unemployment insurance}   %% schemes}
This chapter applies the recursive contract machinery studied in
chapters \use{socialinsurance}, \use{socialinsurance2},
and \use{credible} in contexts that are simple enough that we
can go a long way toward computing  optimal contracts by hand.
The contracts encode history dependence by mapping an initial
promised value and a random time $t$ observation into a time $t$ consumption allocation
and a continuation value to bring into next period.  We use recursive
contracts to study good ways of providing consumption insurance  when incentive problems
 come from the insurance authority's
inability to observe the effort that an unemployed person exerts
 searching for a job.  We begin by studying a setup
of Shavell and Weiss (1979) and Hopenhayn and Nicolini (1997)
that focuses on a single isolated spell of unemployment followed by permanent employment.  Later we take up settings of Wang and
Williamson (1996) and Zhao (2001)
with alternating spells of employment and
unemployment in which the planner has limited information
about a  worker's effort while he is on the job, in addition to not
observing his search effort while he is unemployed.
Here history dependence manifests itself in
an optimal contract with intertemporal tie-ins
across these spells.  Zhao uses her model to rationalize unemployment compensation
that replaces a fraction of a worker's earnings on his or her previous job.

\auth{Shavell, Stephen}\auth{Weiss, Laurence}
\auth{Hopenhayn, Hugo A.}  \auth{Nicolini, Juan Pablo}
\auth{Zhao, Rui}
\index{unemployment!compensation}
\section{A one-spell model}
This section describes a model of optimal unemployment
compensation along the lines of Shavell and Weiss (1979) and
Hopenhayn and Nicolini (1997). We shall use the techniques of
Hopenhayn and Nicolini to analyze a model closer to Shavell and
Weiss's. An unemployed worker orders stochastic processes of
consumption and  search effort $\{c_t , a_t\}_{t=0}^\infty$
according to
$$ E \sum_{t=0}^\infty \beta^t \left[ u(c_t) - a_t \right] \EQN hugo1 $$
where $\beta \in (0,1)$ and
$u(c)$ is strictly increasing, twice differentiable,
and strictly concave.  We assume that $u(0)$ is well defined.
  We require
that $c_t \geq 0$ and $ a_t \geq 0$.
    All jobs are alike and pay wage
$w >0$ units of the consumption good each period forever. An unemployed
worker searches with effort $a$ and with probability
$p(a)$ receives a permanent job at the beginning
of the next period. Once
a worker has found a job, he is beyond the grasp
of the unemployment insurance agency.\NFootnote{This is Shavell and
Weiss's assumption, but not Hopenhayn and Nicolini's. Hopenhayn
and Nicolini allow the unemployment insurance agency to
impose history-dependent taxes on previously unemployed workers.
Since there is no incentive problem after the worker has found
a job, it is optimal for the agency to provide an employed worker with
a constant level of consumption, and hence, the agency imposes
a permanent per-period history-dependent tax on a previously
unemployed worker. See exercise  \the\chapternum.2.}
%% a permanent per-period history-dependent tax on previously
%%unemployed workers.}
Furthermore, $a=0$ when the worker is
employed.  The probability
of finding a job  is $p(a)$ where $p$ is an increasing
and strictly concave and twice differentiable function of $a$, satisfying
$p(a)   \in [0,1]$ for  $a \geq 0$, $p(0)=0$.
The consumption good is nonstorable.
 The unemployed worker has no savings and cannot borrow or lend.
The  insurance agency is the unemployed worker's only source of consumption
smoothing over time and across states.


\subsection{The autarky problem}
As a benchmark, we first study the fate of the unemployed worker
who has no access to unemployment insurance.  Because employment
is an absorbing state for the worker, we work backward from that
state.
  Let $V^e$ be the expected sum of discounted one-period utilities of an employed worker.
Once the worker is employed, $a=0$, making  his period utility
be $u(c)-a = u(w)$ forever.
Therefore,
$$  V^e = {u(w) \over (1-\beta)}  .  \EQN hugo2  $$
Now let $V^u$ be the expected present value of utility for an
unemployed worker who chooses the current period pair $(c,a)$
optimally.  The Bellman
equation for $V^u$  is
$$ V^u = \max_{a \geq 0} \biggl\{ u(0) - a + \beta \left[
   p(a) V^e + (1-p(a)) V^u \right] \biggr\} .  \EQN hugo3 $$
The first-order condition for this problem is
$$ \beta p'(a) \left[V^e - V^u \right] \leq 1\,,  \EQN hugo4   $$
with equality if $a>0$. Since there is no state variable in this
infinite horizon problem, there is a time-invariant optimal
search intensity $a$ and an associated value of being unemployed $V^u$.
Let $V_{\rm aut} = V^u$ denote the solution of Bellman equation \Ep{hugo3}.

 Equations \Ep{hugo3} and \Ep{hugo4} form the basis for
an iterative algorithm for computing $V^u = V_{\rm aut}$.  Let $V^u_j$ be
the estimate of $V_{\rm aut}$ at the $j$th iteration.   Use this value
in equation \Ep{hugo4} and solve
for an estimate of effort  $a_j$.  Use this value in a version of equation
\Ep{hugo3} with $V^u_j$ on the right side
to compute $V^u_{j+1}$.  Iterate to convergence.

\subsection{Unemployment insurance with full information}
As another benchmark, we study the provision of insurance with
full information.  An insurance agency can observe and control
the unemployed person's consumption and search effort.   The
agency wants to design an unemployment insurance contract to give
the unemployed worker expected discounted utility $V > V_{\rm aut}$.
The planner wants to deliver value $V$ in the most efficient way,
meaning the way that minimizes expected
 discounted cost, using $\beta$ as the discount factor.
We formulate the optimal insurance problem
recursively.  Let $C(V)$ be the expected discounted cost of giving
the worker expected discounted utility
$V$. The cost function is strictly convex because
a higher $V$ implies a lower marginal utility of the worker;
that is, additional expected ``utils'' can be awarded to the worker
only at an increasing marginal cost in terms of the consumption good.
Given $V$, the planner
assigns first-period pair   $(c,a)$ and promised
continuation value $V^u$, should  the worker  be unlucky
and not find a job; $(c, a, V^u)$ will all be
chosen to be functions of $V$ and to
satisfy the Bellman equation
$$ C(V) = \min_{c, a, V^u} \biggl\{ c  + \beta [1 - p(a)] C(V^u) \biggr\} ,
\EQN hugo5$$
where the minimization is subject to the promise-keeping constraint
$$ V \leq u(c) - a + \beta
\left\{ p(a) V^e + [1-p(a)] V^u \right\}. \EQN hugo6 $$
Here $V^e$ is given by equation \Ep{hugo2}, which reflects the
assumption that once the worker is employed, he is beyond the
reach of the unemployment insurance agency.
The right side of  Bellman equation \Ep{hugo5} is attained by
policy functions
$c=c(V), a=a(V)$, and $V^u=V^u(V)$.
The promise-keeping constraint,
 equation \Ep{hugo6},
asserts that the 3-tuple $(c, a, V^u)$ attains
at least $V$.   Let $\theta$ be the Lagrange multiplier
on constraint \Ep{hugo6}.  At an interior solution, the first-order
conditions with
respect to $c, a$, and $V^u$, respectively, are
$$\EQNalign{ \theta & = {1 \over u'(c)}\,,  \EQN hugo7;a \cr
             C(V^u) & = \theta \left[ {1 \over \beta p'(a)} -
                           (V^e - V^u) \right]\,,  \EQN hugo7;b  \cr
             C'(V^u) & = \theta\,.  \EQN hugo7;c \cr} $$

The envelope condition   $C'(V) = \theta$ and equation
\Ep{hugo7;c} imply that $C'(V^u) =C'(V)$.  Strict convexity of $C$ then
implies that $V^u =V$. Applied repeatedly over time,
$V^u=V$ makes
the continuation value remain constant during the entire
spell of unemployment.    Equation
\Ep{hugo7;a} determines $c$, and equation \Ep{hugo7;b} determines
$a$, both as functions of the promised $V$.  That $V^u = V$ then
implies that $c$ and $a$ are held constant during the unemployment
spell.   Thus, the unemployed worker's consumption $c$ and search effort $a$ are both ``fully smoothed''
 during the unemployment
spell.  But
the worker's consumption is not  smoothed across states of
employment and unemployment unless $V=V^e$.

\subsection{The incentive problem}
 The preceding efficient insurance scheme requires that the insurance agency
control both $c$ and $a$.  It will not do for the insurance agency
simply to announce $c$ and then allow the worker to choose $a$.
Here is why.
The  agency delivers a value $V^u$  higher than
the autarky value $V_{\rm aut}$ by doing two things.   It
{\it increases\/}
the unemployed worker's consumption $c$ and {\it decreases\/} his search
effort $a$. But the prescribed
search effort is {\it higher\/} than what the worker would choose
if he were to be guaranteed consumption level $c$ while he
remains unemployed.
This follows from equations \Ep{hugo7;a} and \Ep{hugo7;b} and the
fact that the insurance scheme is costly, $C(V^u)>0$, which imply
$[ \beta p'(a) ]^{-1} > (V^e - V^u)$.
 But look at the worker's
first-order condition \Ep{hugo4} under autarky.
It implies that if search effort $a>0$, then
$[\beta p'(a)]^{-1} = [V^e - V^u]$, which is inconsistent
with the preceding inequality
$[ \beta p'(a) ]^{-1} > (V^e - V^u)$ that prevails when $a >0$ under
the social
insurance arrangement.
If he were free to choose $a$, the worker would therefore want to
fulfill \Ep{hugo4}, either at equality so long as $a >0$, or by setting
$a=0$ otherwise.  Starting from the  $a$ associated with
the social insurance scheme,
he  would establish the desired equality
in \Ep{hugo4} by {\it lowering\/} $a$, thereby decreasing
the term $[ \beta p'(a) ]^{-1}$ (which also lowers $(V^e - V^u)$
when the value of being
unemployed $V^u$ increases]). If an equality can be established before
$a$ reaches zero, this would be the worker's preferred search effort;
otherwise the worker would find it optimal to accept the insurance
payment, set $a=0$,  and  never work again.
Thus, since the worker does not take the
cost of the insurance scheme into account, he would choose a search
effort below the socially optimal one. The efficient contract
exploits the agency's ability to control {\it both\/} the unemployed
worker's consumption {\it and\/} his search effort.

\subsection{Unemployment insurance with asymmetric information}
Following Shavell and Weiss (1979) and Hopenhayn and Nicolini
(1997), now assume that  the unemployment insurance agency cannot
observe or enforce $a$, though it can observe and control $c$.
The worker is free to choose $a$, which puts expression \Ep{hugo4}, the worker's first-order condition under autarky,
back in the picture.\NFootnote{We are assuming that the worker's
best response to the unemployment insurance arrangement is
completely characterized by the first-order condition  \Ep{hugo4},
an instance of the so-called ``first-order'' approach to incentive problems.}
 Given any contract, the individual will choose search effort according to
the first-order condition \Ep{hugo4}.  This fact leads the insurance agency
to design the unemployment insurance contract to respect this restriction.
Thus, the recursive contract design problem is now to minimize the right side of equation
\Ep{hugo5} subject to expression \Ep{hugo6} and the incentive constraint \Ep{hugo4}.

Since the restrictions \Ep{hugo4} and \Ep{hugo6} are not linear
and generally do not define a convex set, it becomes difficult
to provide conditions under which the solution to the dynamic
programming problem results in a convex function $C(V)$. As
discussed in Appendix A of chapter \use{socialinsurance},
this complication can be handled by convexifying
the constraint set through the introduction of lotteries.
However, a common finding is that optimal plans do not involve
lotteries, because convexity of the constraint set is a sufficient
but not necessary condition for convexity of the cost function.
Following Hopenhayn and Nicolini (1997), we therefore proceed
under the assumption that $C(V)$ is strictly convex in order to
characterize the optimal solution.

Let $\eta$ be the multiplier on constraint \Ep{hugo4}, while
$\theta$ continues to denote the multiplier on constraint \Ep{hugo6}.
But now we replace the weak inequality in \Ep{hugo6} by an equality.
The unemployment insurance agency cannot award a higher utility than
$V$ because that might violate an incentive-compatibility constraint
for exerting the proper search effort in earlier periods.
At an interior solution, the first-order conditions with
respect to $c, a$, and $V^u$, respectively, are\NFootnote{Hopenhayn and Nicolini
let the insurance agency also choose $V^e$, the continuation
value from $V,$ if the worker finds a job.   This approach reflects
their assumption that the agency can tax a previously unemployed
worker after he becomes employed. See exercise  \the\chapternum.2.}
$$\EQNalign{ \theta & = {1 \over u'(c)}\,,  \EQN hugo8;a \cr
 C(V^u)  & = \theta \left[ {1 \over \beta p'(a)} - (V^e - V^u) \right]
            \,-\, \eta {p''(a) \over p'(a)} (V^e - V^u)                  \cr
         & = \,- \eta {p''(a) \over p'(a)} (V^e - V^u) \,,  \EQN hugo8;b \cr
 C'(V^u) & = \theta \,-\, \eta {p'(a) \over 1-p(a)}\, ,  \EQN hugo8;c \cr}
$$
where the second equality in equation \Ep{hugo8;b} follows from strict equality
of the incentive constraint \Ep{hugo4} when $a>0$. As long as the
insurance scheme is associated with costs, so that $C(V^u)>0$, first-order
condition \Ep{hugo8;b} implies that the multiplier $\eta$ is strictly
positive. The first-order condition \Ep{hugo8;c} and the
envelope condition $C'(V) = \theta$ together allow us to conclude that
$C'(V^u) < C'(V)$. Convexity of $C$ then implies that $V^u < V$.
After we have also used equation \Ep{hugo8;a}, it follows that
in order to provide  the proper incentives, the consumption
of the unemployed worker must decrease as the duration of the unemployment
spell lengthens. It also follows from \Ep{hugo4} at equality that
search effort $a$ rises as $V^u$ falls, i.e., it rises with the duration
of unemployment.

  The duration dependence of benefits is  designed to provide
incentives to search.  To see this, from \Ep{hugo8;c}, notice how
the conclusion that consumption falls with the duration of
unemployment depends on the assumption that more search effort
raises the prospect of finding a job, i.e., that $p'(a) > 0$. If
$p'(a) =0$, then \Ep{hugo8;c} and the strict convexity of $C$ imply that
$V^u =V$. Thus, when $p'(a) =0$, there is no reason for the
planner to make consumption fall with the duration of
unemployment.

%Hopenhayn and Nicolini (1996) work with first-order conditions
%for a version of this problem, and coax much qualitative information
%from them.  We refer the reader to them
%for analytical results, and turn our attention
%to formulating a numerical strategy for solving the problem.
%%%%%%%%%%%%%%%%%%%%%%%%%  insert of Eva-Mike stuff

%%%%%%%%%%%%%%
%$$\grafone{hugoreplnew.eps,height=2.5in}{{\bf Figure 15.5} Top panel:
%  replacement
%ratio $c/w$ as a function of duration of unemployment in Shavell-Weiss
%model. Bottom panel: effort $a$ as function of duration.}$$
%%%%%%%%%%%%%%%%

\midfigure{hugoreplnewf}
\centerline{\epsfxsize=3truein\epsffile{hugoreplnew.eps}}
\caption{Top panel:  replacement ratio $c/w$ as a function of duration of
unemployment in the Shavell-Weiss model. Bottom panel: effort $a$ as a function of duration.}
\infiglist{hugoreplnewf}
\endfigure

%\mtlb{hugo.m}\mtlb{hugo1a.m}\mtlb{hugofoc1.m}\mtlb{valhugo.m}
\mtlb{hugo.m} \mtlb{hugo1a.m} \mtlb{hugofoc1.m} \mtlb{hugofoc1.m}
\mtlb{valhugo.m}
\subsection{Computed example}
For parameters chosen by Hopenhayn and Nicolini, Figure \Fg{hugoreplnewf} %15.5
displays the
replacement ratio $c / w$ as a function of  the duration of the unemployment
spell.\NFootnote{This figure was computed using the Matlab programs
{\tt hugo.m}, {\tt hugo1a.m}, {\tt hugofoc1.m}, {\tt valhugo.m}.
These are available in the subdirectory {\tt hugo}, which
contains a readme file.  These programs were composed
by various members of Economics 233 at Stanford in 1998,
especially Eva Nagypal, Laura Veldkamp, and Chao Wei.}
 This schedule was computed by finding the optimal policy functions
$$ \eqalign{ V^u_{t+1} & = f(V^u_t) \cr
             c_t & = g(V^u_t). \cr} $$
and iterating on them, starting from some initial $V^u_0 > V_{\rm aut}$,
where $V_{\rm aut}$ is the autarky level for an unemployed worker.
Notice how the replacement ratio declines with duration.
Figure \Fg{hugoreplnewf} %15.5
sets $V^u_0$ at 16,942, a number that
has to be interpreted in the context of Hopenhayn and Nicolini's
parameter settings.

We computed these numbers using the parametric version studied by Hopenhayn
and Nicolini.\NFootnote{In  section \use{HNalgorithm}, %of chapter \use{practical},
we described a computational strategy
of iterating to convergence on the Bellman equation \Ep{hugo5}, subject
to expressions \Ep{hugo6} at equality, and \Ep{hugo4}.}
  Hopenhayn and Nicolini chose parameterizations and parameters as follows:
They interpreted one  period as one week, which led them
to set $\beta=.999$.  They took  $u(c) = {c^{(1-\sigma)} \over 1 - \sigma}$
and set
$\sigma=.5$.  They set the wage $w=100$ and
specified the hazard function to be  $p(a) = 1 - \exp(-ra)$, with $r$ chosen
to give a hazard rate $ p(a^*) = .1$, where
$a^*$ is the optimal search  effort under autarky. To compute the numbers
in Figure \Fg{hugoreplnewf} we used
these same settings.

\subsection{Computational details}
Exercise {\it \the\chapternum.1\/} asks the reader  to solve the Bellman equation
numerically.
In doing so, it is useful to note that there
%%is a natural upper bound to the set
%%of continuation values $V^u$.  To  compute it,
are natural lower and upper bounds to the set
of continuation values $V^u$. The lower bound is
the expected lifetime utility in autarky,
$V_{\rm aut}$. To compute the upper bound,
represent condition \Ep{hugo4} as
$$ V^u \geq V^e  - [\beta  p'(a)]^{-1},$$
with equality if $ a > 0$.
If there is zero search effort, then $V^u \geq V^e -[\beta p'(0)]^{-1}$.  %% >
Therefore, to rule out zero search effort we require
$$ V^u < V^e - [\beta p'(0)]^{-1} .$$                                     %% \leq
(Remember that $p''(a) < 0$.)  This step gives  our upper bound
for $V^u$.

To formulate the Bellman equation numerically,
we suggest using the constraints to eliminate $c$ and $a$ as choice
variables, thereby reducing the Bellman equation to
a minimization over the one choice variable $V^u$.
First express the promise-keeping constraint \Ep{hugo6} as
$u(c) = V + a - \beta \{p(a) V^e +[1-p(a)] V^u \}$.                       %% \geq
That is, consumption is equal to
$$ c = u^{-1}\left(
     V+a -\beta [p(a)V^e + (1-p(a))V^u] \right). \EQN hugo21$$
%%For the preceding utility function,
%%whenever the right side of this
%%inequality is negative, then this  promise-keeping constraint  is
%%not binding and can be satisfied with $c=0$.   This observation
%%allows us to write
%%$$ c = u^{-1}\left( \max\left\{0,
%%     V+a -\beta [p(a)V^e + (1-p(a))V^u] \right\} \right). \EQN hugo21$$
Similarly, solving the inequality \Ep{hugo4} for $a$ and using the
assumed  functional
form for $p(a)$  leads to
$$ a = \max\left\{0, {\log[r \beta (V^e - V^u)] \over r } \right\}.
             \EQN hugo22 $$
Formulas \Ep{hugo21} and \Ep{hugo22} express $(c,a)$ as functions
of $V$ and  the continuation value $V^u$.  Using these functions
allows us to write the Bellman equation in $C(V)$ as
$$ C(V)  = \min_{V^u} \left\{ c + \beta [1 - p(a)] C(V^u) \right\} \EQN hugo23 $$
where $c$ and $a$ are given by equations \Ep{hugo21} and \Ep{hugo22}.


\subsection{Interpretations}
The substantial downward slope in the replacement ratio in Figure \Fg{hugoreplnewf} % Figure 15.5
comes entirely  from the incentive constraints facing the
planner.   We saw earlier that without private information, the
planner would smooth consumption
%%across unemployment states
over the unemployment spell by
keeping the  replacement ratio constant. In the situation depicted in
Figure \Fg{hugoreplnewf}, the planner can't observe the worker's search effort
and therefore makes   the replacement ratio  fall and search
effort rise as the duration of unemployment increases, especially
early in an unemployment spell.   There is a ``carrot-and-stick''
aspect to the replacement rate and search effort  schedules: the
``carrot'' occurs in the forms of high compensation and low search
effort early in an unemployment spell.  The ``stick''
occurs in the low compensation and high effort  later in
the spell.   We shall see  this carrot-and-stick feature in some of the
credible government policies analyzed in chapters \use{credible}, \use{chang}, and
\use{wldtrade}.
\index{carrot and stick}

The planner offers declining benefits and asks for increased search
effort as the duration of an unemployment spell rises in order to provide
unemployed workers with proper incentives, not to punish an unlucky worker
who has been unemployed for a long time.  The planner believes that a
worker who has been unemployed a long time is unlucky, not that he has
done anything wrong (i.e., not lived up to the contract).  Indeed, the
contract is designed to induce the unemployed workers to search in
the way the planner expects.  The falling consumption and rising
search effort of the unlucky ones with long unemployment spells are
simply the prices that have to be paid for the common good of providing
proper incentives.


\subsection{Extension: an on-the-job tax}
 Hopenhayn and Nicolini allow the planner to tax the worker {\it after\/}
he becomes employed, and they let the tax depend on the duration of unemployment.
Giving the planner this additional instrument substantially decreases the
rate at which the replacement ratio falls during a spell of unemployment.
Instead, the planner makes use of a more powerful tool: a {\it permanent\/}
bonus or tax after the worker becomes employed.  Because it endures, this
tax or bonus is especially potent when the discount factor is high.  In
exercise {\it \the\chapternum.2\/}, we ask the reader to set up the functional equation for
Hopenhayn and Nicolini's model.

\subsection{Extension: intermittent unemployment spells}
In Hopenhayn and Nicolini's model, employment is an absorbing
state and there are no incentive problems after a job is found.
There are not multiple spells of unemployment.   Wang and
Williamson (1996) built a model in which there can be multiple
unemployment spells, and in which there is also an incentive
problem on the job.  As in Hopenhayn and Nicolini's model, search
effort affects the probability of finding  a job.  In addition,
while on a job, effort affects the probability that the job ends
and that the worker becomes unemployed again.  Each job pays the
same wage.   In Wang and Williamson's setup, the promised value
keeps track of the duration and number of spells of employment as
well as of the number and duration of spells of unemployment.  One
contract transcends employment and unemployment.

\section{A multiple-spell model with lifetime contracts}
  Rui Zhao (2001) modifies and extends features of Wang and Williamson's
model.  In her model, effort on the job affects output as well as the
probability that the job will end.  In Zhao's model, jobs randomly end,
recurrently returning a worker to the state of unemployment.  The
probability that a job ends depends directly or indirectly
on the effort that workers expend on the job.  A planner observes
the worker's output and employment status, but never his effort,
and  wants to insure the worker.  Using recursive methods,
Zhao designs  a history-dependent assignment of unemployment benefits,
if unemployed, and wages, if employed, that balance a planner's
desire to insure the worker with the need to provide incentives to
supply effort in work and search.  The planner uses history dependence
to tie compensation while unemployed (or employed) to earlier outcomes
that partially inform the planner about the workers' efforts while
employed (or unemployed).  These intertemporal tie-ins give rise to
what Zhao interprets broadly as a ``replacement rate'' feature that
we seem to observe in unemployment compensation systems.

\subsection{The setup}
In a special case of Zhao's model, there are two effort levels.
Where $a \in \{a_L, a_H\}$ is a worker's effort   and $\overline y_i >
\overline y_{i-1}$, an employed worker produces $y_t \in \left[ \overline y_1, \cdots,
\overline y_n \right]$ with probability
$$ {\rm Prob}(y_t = \overline y_i) = p(\overline y_i;a ). $$
Zhao assumes:
\medskip
\noindent{\sc Assumption 1:}  $p(\overline y_i;a)$ satisfies the
{\it monotone likelihood ratio\/} property:
${p(\overline y�_i;a_H)\over p(\overline y_i;a_L)}$ increases as $\overline y_i$ increases.
\medskip
At the end of each period, jobs end with probability
$\pi_{eu}$.  Zhao embraces one of two alternative assumptions
about
the job separation
rate $\pi_{eu}$, allowing it  to depend on either current output $y$ or
current work effort $a$.  She assumes:
\medskip
\noindent{\sc Assumption 2:} Either  $\pi_{eu} (y)$ decreases with $y$
or $\pi_{eu}(a)$ decreases with $a$.
\medskip


Unemployed workers produce nothing and search for a job subject to the
following  assumption about the job finding rate $\pi_{ue}(a)$:

\medskip
\noindent{\sc Assumption 3:} $\pi_{ue}(a)$ increases with $a$.
\medskip
The worker's one-period utility function
is $U(c,a) = u(c) - \phi(a)$ where
$u(\cdot)$ is continuously differentiable, strictly increasing and
strictly concave, and $\phi(a)$ is continuous, strictly increasing,
and strictly convex.  The worker orders random $\{c_t,a_t\}_{t=0}^\infty$
 sequences
according to
$$ E \sum_{t=0}^\infty \beta^t U(c_t, a_t) , \quad \beta \in (0,1) .
\EQN zhao2 $$

We shall regard a planner as being
 a coalition of firms united
with an unemployment insurance agency.  The planner is risk neutral and can
borrow and lend at a constant risk-free gross one-period interest rate
of $R = \beta^{-1}$.

Let the worker's employment state be $s_t \in S = \{e,u\}$
where $e$ denotes employed, $u$ unemployed.  The worker's
output at $t$ is
$$ z_t = \cases{ 0 & if $s_t = u$, \cr
                 y_t & if $s_t =e $. \cr}
$$
For $t \geq 1$,  the time $t$ component of the publicly observed information
is
$$ x_t =(z_{t-1}, s_t) , $$
and $x_0 = s_0$.
At time $t$, the planner observes
the history $x^t$ and the worker observes $(x^t, a^t)$.

The transition probability for $x_{t+1} \equiv (z_t, s_{t+1})$
 can be factored as follows:
$$ \pi(x_{t+1} | s_t, a_t) = \pi_z(z_t;s_t,a_t) \pi_s(s_{t+1};z_t, s_t, a_t)
  \EQN zhao5 $$
where $\pi_z$ is the distribution of output conditioned on
the state and the action, and $\pi_s$ encodes the transition probabilities
of employment status conditional on output, current employment
status, and effort.   In particular, Zhao assumes that
$$ \eqalign{
\pi_s(u;0, u, a) & = 1 - \pi_{ue}(a) \cr
\pi_s(e;0, u, a) & = \pi_{ue}(a) \cr
\pi_s(u;y,e, a) & = \pi_{eu}(y, a) \cr
\pi_s(e;y,e, a) & = 1- \pi_{eu}(y, a) . \cr   } \EQN zhao6 $$


\subsection{A recursive lifetime contract}
%%%%%%%%%%%%%%%%%%%%%%%%%%%%%%%%%%%%%%%%%%%%%%
%  Let $v$ be  a promised value of the worker's
%  expected discounted utility \Ep{zhao2}.
%For a given  $v$,
%let $w(z,s')$ be the continuation value of promised utility
%\Ep{zhao2} for next period when today's output is $z$ and
%tomorrow's unemployment state is $s'$.  At the beginning
%of next period, $(z,s')$ will be the  labor market outcome
%most recently observed by the planner.
%Let $W = \{W_s\}_{s\in \{u,e\}}$ be two compact sets of continuation
%values, one set for $s =u$ and another for $s=e$.
%For each $(v,s)$,  a {\it recursive contract\/} specifies
%an output-contingent  consumption level $c(z)$  today,
%a recommended effort level $a$, and continuation values
%$w(z,s')$ to be used to reset $v$ tomorrow.
%
%Zhao imposes the following constraints on the contract:
%
%\medskip
%$$ \sum_z \pi_z(z;s,a) \left(u(c(z)) + \beta \sum_{s'} \pi_s(s';z,s,a)
%   w(z,s') \right) - \phi(a) \geq v  \EQN zhao7 $$
%and for all $s, a$,
%$$ \eqalign{ &
%\sum_z \pi_z(z;s,a) \left(u(c(z)) + \beta \sum_{s'} \pi_s(s';z,s,a)
%   w(z,s') \right) - \phi(a) \geq \cr
%& \sum_z \pi_z(z;s,\tilde a) \left(u(c(z)) + \beta \sum_{s'} \pi_s(s';z,s,\tilde a)
%   w(z,s') \right) - \phi(\tilde a) \ \ \forall \tilde a. \cr } \EQN zhao8  $$
%Constraint \Ep{zhao7} entails {\it promise keeping\/},
%while \Ep{zhao8} are the incentive-compatibility
%or ``effort-inducing'' constraints.
%In addition, a contract has to satisfy
%$\underline c \leq c(z) \leq \overline c $ for all $z$ and
%$w(z,s') \in W_{s'}$ for all $(z,s')$.
%A contract is said to be {\it incentive compatible\/} if it satisfies
%the  incentive compatibility constraints
%\Ep{zhao8}.\NFootnote{We assume two-sided
%commitment to the contract and therefore ignore  the participation
%constraints that Zhao imposes on the contract.    She requires that
% continuation
%values $w(z,s')$ be at least as great as the  autarky values
%$V_{s',{\rm aut}}$   for each $(z,s')$.}
%
%\medskip
%\specsec{Definition:} A recursive contract $(c(z),  w(z,s'), a)$
%is said to be {\it feasible with respect to $W$\/} for
%a given $(v,s)$ pair if it is incentive compatible in state
%$s$, delivers promised value $v$, and $w(z,s') \in W_{s'}$ for
%all $(z,s')$.
%\medskip
%Let $C(v,s)$ be the minimum cost to the planner of delivering promised
%value  $v$ to a worker in employment state $s$.
%We can represent the
%Bellman equation for $C(v,s)$ in terms of the following two-part
%optimization:
%$$\EQNalign{ \Psi(v,s,a) &  = \min_{c(z), w(z,s')}
%  \Biggl\{ \sum_z \pi_z(z;s,a) \bigl(-z + c(z)
%    \cr & +
% \beta \sum_{s'} \pi_s(s';z,s,a) C(w(z,s'),s') \bigr) \Bigr\} \EQN zhao9;a \cr
%  C(v,s) & = \min_{a \in [a_L, a_H]} \Psi(v,s, a).  \EQN zhao9;b \cr} $$
%%%%%%%%%%%%%%%%%%%%%%%%%%%%%%%%%%%%%%%%%%%%%%%%%%%%%%%%%%%%%%%%%%%%%%%%%%%%%%%%

Consider a worker with beginning-of-period employment status $s$ and
promised value $v$.
For given  $(v,s)$,
let $w(z,s')$ be the continuation value of promised utility
\Ep{zhao2} for next period when today's output is $z$ and
tomorrow's employment state is $s'$.  At the beginning
of next period, $(z,s')$ will be the  labor market outcome
most recently observed by the planner.
Let $W = \{W_s\}_{s\in \{u,e\}}$ be two compact sets of continuation
values, one set for $s =u$ and another for $s=e$.
For each $(v,s)$,  a {\it recursive contract\/} specifies
a recommended effort level $a$ today,
an output-contingent  consumption level $c(z)$  today,
and continuation values
$w(z,s')$ to be used to reset $v$ tomorrow.

For each $(v,s)$, the contract $(a, c(z), w(z,s'))$ must satisfy:
%%\medskip
$$ \sum_z \pi_z(z;s,a) \left(u(c(z)) + \beta \sum_{s'} \pi_s(s';z,s,a)
   w(z,s') \right) - \phi(a) = v  \EQN zhao7 $$                      %%% \geq
and                                                                  %%% for all $s, a$,
$$ \eqalign{ &
\sum_z \pi_z(z;s,a) \left(u(c(z)) + \beta \sum_{s'} \pi_s(s';z,s,a)
   w(z,s') \right) - \phi(a) \geq \cr
& \sum_z \pi_z(z;s,\tilde a) \left(u(c(z)) + \beta \sum_{s'} \pi_s(s';z,s,\tilde a)
   w(z,s') \right) - \phi(\tilde a) \ \ \forall \tilde a. \cr } \EQN zhao8  $$
Constraint \Ep{zhao7} entails {\it promise keeping\/},
while \Ep{zhao8} are the incentive-compatibility
or ``effort-inducing'' constraints.
In addition, a contract has to satisfy
$\underline c \leq c(z) \leq \overline c $ for all $z$ and
$w(z,s') \in W_{s'}$ for all $(z,s')$.
A contract is said to be {\it incentive compatible\/} if it satisfies
the  incentive compatibility constraints
\Ep{zhao8}.\NFootnote{We assume two-sided
commitment to the contract and therefore ignore  the participation
constraints that Zhao imposes on the contract.    She requires that
 continuation
values $w(z,s')$ be at least as great as the  autarky values
$V_{s',{\rm aut}}$   for each $(z,s')$.}

\medskip
\noindent{\sc Definition:} A recursive contract $(a, c(z),  w(z,s'))$
is said to be {\it feasible with respect to $W$\/} for
a given $(v,s)$ pair if it is incentive compatible in state
$s$, delivers promised value $v$, and $w(z,s') \in W_{s'}$ for
all $(z,s')$.
\medskip
Let $C(v,s)$ be the minimum cost to the planner of delivering promised
value  $v$ to a worker in employment state $s$.
We can represent the
Bellman equation for $C(v,s)$ in terms of the following two-part
optimization:
$$\EQNalign{ \Psi(v,s,a) &  = \min_{c(z), w(z,s')}
  \Biggl\{ \sum_z \pi_z(z;s,a) \Bigl(-z + c(z)
    \cr &
 \hskip2cm +\beta \sum_{s'} \pi_s(s';z,s,a) C(w(z,s'),s') \Bigr) \Biggr\}
                                                       \hskip.5cm \EQN zhao9;a \cr
\noalign{\noindent \rm subject to constraints \Ep{zhao7} and \Ep{zhao8}, and} \cr
  C(v,s) & = \min_{a \in [a_L, a_H]} \Psi(v,s, a).  \EQN zhao9;b \cr} $$
The function $\Psi(v,s,a)$ assumes that the worker  exerts effort level
$a$.  Later, we shall typically assume that parameters are such that
$C(v,s)=\Psi(v,s,a_H)$, so that the planner finds it optimal always
to induce high effort.
Put a Lagrange multiplier
$\lambda(v,s,a)$ on the promise-keeping constraint \Ep{zhao7} and
another multiplier
$\nu(v,s,a)$ on the  effort-inducing constraint \Ep{zhao8} given $a$,
 and form
the Lagrangian:
$$ \eqalign{L = & \sum_z \pi_z(z;s,a) \Biggl\{ -z + c(z)
    + \beta \sum_{s'} \pi_s(s';z,s,a) C(w(z,s'),s') \cr
  & - \lambda(v,s,a)
\left[ u(c(z)) + \beta \sum_{s'} \pi_s(s';z,s,a)
   w(z,s') ) - \phi(a) - v \right] \cr
 & - \nu(v,s,a) \Biggl[ u(c(z)) + \beta \sum_{s'} \pi_s(s';z,s,a)
   w(z,s')  - \phi(a)    \cr
 & - {\pi_z(z;s,\tilde a) \over \pi_z(z;s,a)}
   \left( u(c(z)) + \beta \sum_{s'} \pi_s(s';z,s,\tilde a)w(z,s')  -
 \phi(\tilde a)   \right) \Biggr] \Biggr\} , \cr }$$
where $\tilde a \in \{a_L, a_H\}$ and $\tilde a \not= a$.
First-order conditions for $c(z)$ and $w(z,s')$, respectively, are
$$\EQNalign{ {1 \over u'(c(z)) }  = &
  \lambda(v,s,a) + \nu(v,s,a) \left( 1 -
  {\pi_z(z;s,\tilde a) \over \pi_z(z;s,a) } \right)  \EQN zhao12;a \cr
\noalign{\vskip.2cm}
  C_v(w(z,s'),s')  = & \lambda(v,s,a) \cr & +
  \nu(v,s,a) \left[ 1 - {\pi_z(z;s,\tilde a) \over
                    \pi_z(z;s,a) }
       {\pi_s(s';s,z,\tilde a)\over \pi_s(s';z,s,a)} \right] .\hskip.5cm
 \EQN zhao12;b \cr }$$
The envelope conditions are
$$ \EQNalign{ \Psi_v(v,s,a) & = \lambda(v,s,a) \EQN zhao13;a \cr
        C_v(v,s) & = \Psi_v(v,s, a^*) \EQN zhao13;b \cr}$$
where $a^*$ is the planner's optimal choice of $a$.

To deduce the dynamics of compensation,
Zhao's strategy is to study the first-order conditions
\Ep{zhao12} and envelope conditions \Ep{zhao13} under two  cases,
$s=u$ and $s=e$.







\subsection{Compensation dynamics when unemployed}
  In the unemployed state ($s=u$),
the first-order conditions become
$$\EQNalign{ & {1 \over u'(c)} = \lambda(v,u,a) \EQN zhao14;a \cr
   & C_v(w(0,u),u) = \lambda(v,u,a) + \nu(v,u,a)
 \left[ 1 - {1 - \pi_{ue}(\tilde a)  \over 1 - \pi_{ue}(a)} \right]
  \EQN zhao14;b \cr
 & C_v(w(0,e),e) = \lambda(v,u,a) + \nu(v,u,a) \left[1 -
  {\pi_{ue} (\tilde a) \over \pi_{ue}(a) }\right].  \EQN zhao14;c \cr}  $$
The effort-inducing constraint \Ep{zhao8} can be rearranged to become
$$ \beta (\pi_{ue}(a) - \pi_{ue}(\tilde a)) (w(0,e) - w(0,u))
  \geq \phi(a) - \phi (\tilde a) .$$
Like Hopenhayn and Nicolini,  Zhao describes how
compensation and effort depend on the duration
of unemployment:
\medskip
\noindent{\sc Proposition:} To induce high search effort, unemployment
benefits must {\it fall\/} over an unemployment spell.
\medskip
\noindent{\sc Proof:}  When search effort is high, the effort-inducing
constraint binds.  By assumption 3,
$$ {1 - \pi_{ue}(a_L) \over  1 - \pi_{ue}(a_H)}
  > 1 > {\pi_{ue}(a_L) \over \pi_{ue}(a_H) }. $$
These inequalities and the first-order condition \Ep{zhao14} then imply
$$ C_v(w(0,e),e) > \Psi_v(v,u,a_H) > C_v(w(0,u),u). \EQN zhao15 $$
Let $c_u(t), v_u(t)$, respectively,
 be consumption and the continuation value for an
unemployed worker.  Equations \Ep{zhao14} and the envelope conditions imply
$${1 \over u'(c_u(t))} = \Psi_v(v_u(t), u, a_H) > C_v(v_u(t+1), u)
    = {1 \over u'(c_u(t+1))}. \EQN zhao16 $$
Concavity of $u$ then implies that $c_u(t) > c_u(t+1)$.
In addition, notice that
$$ C_v(w(0,u),u) - C_v(v,u)
  = \nu(v,u,a_H) \left(1-{1 - \pi_{ue}(a_L) \over 1 - \pi_{ue}(a_H)} \right),
 \EQN zhao16 $$
which follows from the first-order conditions \Ep{zhao14} and the envelope
conditions.  Equation \Ep{zhao16} implies that continuation
values fall with the duration of unemployment.\qed

\subsection{Compensation dynamics while employed}

When the worker is employed, for each promised value $v$,
the contract specifies
output-contingent consumption and continuation values
a $c(y), w(y,s')$.  When $s=e$,
the first-order conditions \Ep{zhao12} become
$$ \EQNalign{  {1 \over u'(c(y))} &= \lambda(v,e,a)
  + \nu(v,e,a) \left(1 - {p(y;\tilde a) \over p(y;a)} \right)
\EQN zhao18;a \cr
  C_v(w(y,u),u)  &= \lambda(v,e,a)   + \nu(v,e,a)
   \left(1 - {p(y;\tilde a) \over p(y;a)} {\pi_{eu}(y,\tilde a)
   \over \pi_{eu}(y,a)} \right) \EQN zhao18;b \cr
 C_v(w(y,e),e) &= \lambda(v,e,a)  + \nu(v,e,a)
   \left(1 - {p(y;\tilde a) \over p(y;a)} {1-\pi_{eu}(y,\tilde a)
   \over 1- \pi_{eu}(y,a)} \right). \hskip1.5cm \EQN zhao18;b \cr }$$
Zhao uses these first-order conditions to characterize
how compensation depends on  output:
\medskip
\noindent{\sc Proposition:}  To induce high work effort, wages and continuation
values increase with current output.
\medskip
\noindent{\sc Proof:}  For any $y > \tilde y$, let
$d ={p(\tilde y;a_L) \over p(\tilde y; a_H)} -
    {p(y;a_L) \over p(y;a_H)}. $
Assumption 1 about $p(y;a)$
implies that $d >0$. The first-order
conditions \Ep{zhao18} imply that
$$\EQNalign{& {1 \over u'(c(y))} - {1 \over u'(c(\tilde y))}
  = \nu (v,e,a) d >0  , \EQN zhao19;a \cr
&  C_v(w(y,u),u) - C_v(w(\tilde y,u),u) \propto \nu(v,e,a) d > 0 ,\EQN zhao19;b
   \cr
&  C_v(w(y,e),e) - C_v(w(\tilde y,e),e) \propto \nu(v,e,a) d > 0 .\EQN zhao19;c
  \cr} $$
Concavity of $u$ and convexity of $C$ give the result. \qed
\medskip

In the following proposition, Zhao shows how continuation values
at the start of unemployment   spells should depend on the history
of the worker's
outcomes during previous employment and unemployment spells.
\medskip
\noindent{\sc Proposition:}  If the job separation rate depends on current
output, then the replacement rate immediately after a worker loses a job
is 100\%.  If the job separation rate depends on work effort, then
the replacement rate is less than 100\%.
\medskip
\noindent{\sc Proof:}  If the job separation rate depends on {\it output\/},
the first-order conditions \Ep{zhao18} imply
$$ {1 \over u'(c(y))} = C_v(w(y,u),u) = C_v(w(y,e),e)). \EQN zhao20 $$
This is because $\pi_{eu}(y,\tilde a) = \pi_{eu}(y,a)$ when the job
separation rate depends on output.  Let $c_e(t), c_u(t)$ be consumption
of employed and unemployed workers, and let $v_e(t), v_u(t)$ be
the assigned promised values at $t$.  Then
$$ {1 \over u'(c_e(t))} = C_v(v_{u,t+1},u) = {1 \over c_u(t+1)} $$
where the first equality follows from  \Ep{zhao20}  and the
second from the envelope condition.
If the job separation rate depends on {\it work effort\/}, then
the first-order conditions \Ep{zhao18} imply
$${1 \over u'(c(y))} - C_v(w(y,u),u) = \nu(v,e,a) {p(y;a_L) \over p(y;a_H)}
  \left(  {\pi_{eu}(a_L) \over \pi_{eu}(a_H)} -1 \right). \EQN zhao21 $$
Assumption 2 implies that the right side of \Ep{zhao21} is positive,
which implies that
  $${1 \over u'(c_e(t))} > C_v(v_u(t+1),u) = {1\over u'(c_u(t+1))}. $$
\qed

\subsection{Summary}
 A worker in Zhao's model enters a lifetime contract that makes
compensation respond to the history of outputs on the current and
past jobs, as well as on the durations of all previous spells of
unemployment.\NFootnote{We have analyzed a version of Zhao's model
in which the worker is committed to obey the contract.  Zhao
incorporates an enforcement problem in her model by allowing
the worker to accept an outside option each period.}
Her model has the outcome that compensation at the beginning
of an unemployment spell varies directly with the compensation
attained on the previous job.
%%  A single contract
%%governs the employment relation and the unemployment
%%insurance arrangement.
   This aspect of her model offers a possible explanation for why
unemployment insurance systems often feature a ``replacement rate''
that gives more unemployment insurance payments to workers
who had higher wages in their prior jobs.


\section{Concluding remarks}

  The models that we have studied in this chapter isolate the worker
from capital markets so that the worker
cannot   transfer consumption across time
or states except by adhering to the contract offered by the planner.
If the worker in the models of this chapter
were allowed to save or issue a risk-free asset bearing
a gross one-period rate of return approaching $\beta^{-1}$, it would
interfere substantially with
the planner's ability to provide incentives by manipulating
 the worker's continuation value in response to observed current outcomes.
In particular, forces identical to those analyzed
 in the Cole and Kocherlakota setup that we analyzed at length
in chapter \use{socialinsurance} would circumscribe the planner's
ability to supply insurance.
In the context of unemployment insurance models like that of this chapter,
  this   point has been studied in detail
in papers by Ivan Werning (2002) and Kocherlakota (2004).
\auth{Werning, Ivan} \auth{Kocherlakota, Narayana R.} \auth{Cole,
Harold L.}

Pavoni and Violante (2007) substantially extended models like those in this chapter to perform
positive and normative analysis of a sequence of government programs  that try efficiently to provide insurance, training, and  proper incentives
for unemployed and undertrained workers to reenter employment.
\auth{Pavoni, Nicola}%
\auth{Violante, Giovanni L.}%

%\endchapter
%\section{Exercises}
\showchaptIDfalse
\showsectIDfalse
\section{Exercises}
\showchaptIDtrue
\showsectIDtrue
\medskip
\noindent{\it Exercise \the\chapternum.1}\quad {\bf Optimal unemployment compensation}

\medskip
\noindent{\bf a.}   Write a program
to compute the autarky solution, and use it to  reproduce
Hopenhayn and Nicolini's calibration of  $r$, as described in text.

\medskip
\noindent {\bf b.}  Use your calibration from part a.
  Write a program to compute the optimum value function $C(V)$ for the
insurance design problem with incomplete information. Use the program to
form versions of Hopenhayn and Nicolini's table 1, column 4 for three
different initial values of $V$, chosen by you to belong to the set
$(V_{\rm   aut}, V^e)$.

\medskip
\noindent{\it Exercise  \the\chapternum.2}  \quad {\bf Taxation after employment}
\medskip
\noindent Show how the functional equation
\Ep{hugo5}, \Ep{hugo6} would be modified if the planner
were permitted to tax workers after   they became employed.

\medskip
\noindent{\it Exercise \the\chapternum.3}\quad {\bf Optimal unemployment compensation with
unobservable wage offers}
\medskip
\noindent
Consider an unemployed person with preferences given by
$$
    E \sum_{t=0}^\infty \beta^t u(c_t)  \,,
$$
where $\beta \in (0,1)$ is a subjective discount factor, $c_t \geq 0$ is
consumption at time $t$, and the utility function
$u(c)$ is strictly increasing, twice differentiable,
and strictly concave. Each period the worker draws one offer $w$
from a uniform wage distribution on the domain
$[w_{\scriptscriptstyle L}, w_{\scriptscriptstyle H}]$
with $0 \leq w_{\scriptscriptstyle L} < w_{\scriptscriptstyle H} < \infty$.
Let the cumulative density function be denoted
$F(x)={\rm Prob}\{w\le x\}$, and denote its density by $f$,
which is constant on the domain
$[w_{\scriptscriptstyle L}, w_{\scriptscriptstyle H}]$.
After the worker has accepted a wage offer $w$, he receives the wage $w$
per period forever.  He is then beyond the grasp of the unemployment
insurance agency. During the unemployment spell, any consumption smoothing
has to be done through the unemployment insurance agency because the worker
holds no assets and cannot borrow or lend.
\medskip
\noindent {\bf a.}   Characterize the worker's optimal reservation wage
when he is entitled to a time-invariant unemployment compensation $b$ of
indefinite duration.

\medskip
\noindent {\bf b.}  Characterize the optimal unemployment compensation
scheme under full information. That is, we assume that the insurance
agency can observe and control the unemployed worker's consumption
and reservation wage.

\medskip
\noindent {\bf c.}  Characterize the optimal unemployment compensation
scheme under asymmetric information where the insurance agency cannot
observe wage offers, though it can observe and control the unemployed
worker's consumption. Discuss the optimal time profile of the
unemployed worker's consumption level.

\medskip
\noindent{\it Exercise \the\chapternum.4} \quad {\bf Full unemployment insurance}
\medskip
\noindent
An unemployed worker orders stochastic processes of
consumption, search effort $\{c_t , a_t\}_{t=0}^\infty$
according to
$$ E \sum_{t=0}^\infty \beta^t \left[ u(c_t) - a_t \right]  $$
where $\beta \in (0,1)$ and
$u(c)$ is strictly increasing, twice differentiable,
and strictly concave.  It is required
that $c_t \geq 0$ and $ a_t \geq 0$.
    All jobs are alike and pay wage
$w >0$ units of the consumption good each period forever. After
a worker has found a job, the unemployment insurance agency can
tax the employed worker at a rate $\tau$ consumption goods per period.
The unemployment agency can make $\tau$ depend on the worker's
unemployment history.
The probability
of finding a job  is $p(a)$, where $p$ is an increasing
and strictly concave and twice differentiable function of $a$, satisfying
$p(a)   \in [0,1]$ for  $a \geq 0$, $p(0)=0$.
The consumption good is nonstorable.
The unemployed person cannot borrow or lend and holds no assets.
If the unemployed worker is to do any consumption smoothing,
it has to be through the unemployment insurance agency.
The insurance agency can observe the worker's search effort and can control
his consumption.    An employed worker's consumption is $w -\tau$ per period.

\medskip
\noindent{\bf a.}  Let $V_{\rm aut} $ be the value of an unemployed
 worker's expected discounted utility when he has no access to unemployment
insurance.  An unemployment insurance agency wants to insure unemployed
workers and to deliver expected discounted utility
$V > V_{\rm aut}$  at minimum expected discounted cost $C(V)$.
The insurance agency also uses the discount factor $\beta$.
The insurance agency controls $c,a,\tau$, where $c$ is consumption
of an unemployed worker.  The worker pays the tax
$\tau$ only after he becomes employed.
Formulate the Bellman equation for $C(V)$.
\medskip
%\noindent {\bf Answer:}
%$$ C(V) = \min_{c, a, V^u, V^e } \left\{c + \beta[1-p(a)] C(V^u)
%     - \beta p(a) \left[ {\tau \over 1 - \beta} \right]\right\} $$
%where the minimization is subject to the promise keeping constraints
%$$ \eqalign{ V & \leq u(c) - a + \beta \left[ p(a) V^e +
%  (1-p(1)) V^u\right] \cr
%   V^e & = {u(w-\tau) \over 1 -\beta} .   \cr} $$
%
%\medskip
%\noindent{\bf b.}  Prove that the optimal policy of the insurance agency
%is a policy that satisfies  $c = w -\tau$.
%
%\medskip
%{\bf Answer:}  I have this on yellow sheets.


\medskip
\noindent {\it Exercise \the\chapternum.5}
\quad { \bf Two effort levels}
\medskip
\noindent
 An unemployment insurance agency wants to
insure unemployed workers in the most efficient way. An unemployed
worker receives no income and
chooses a sequence of search intensities $a_t \in \{0, a\}$
to  maximize the utility functional
$$ E_0 \sum_{t=0}^\infty \beta^t \left\{ u(c_t) - a_t \right\}, \quad \beta
\in (0,1)  \leqno (1) $$
where $u(c)$ is an increasing,  strictly concave,
 and twice continuously   differentiable
function of consumption of a single good. There are two values
of the search intensity, $0$ and $a$.  The probability
of finding a job at the beginning of period $t+1$ is
$$\pi(a_t) = \cases{\pi(a), &   if  $a_t =a $; \cr
                   \pi(0) < \pi(a), & if $a_t =0 $, \cr} \leqno(2) $$
where we assume that $a > 0$.   Note that the worker exerts search
effort in period $t$ and possibly receives a job at the beginning
of period $t+1$.  Once the worker finds a job, he receives a fixed
wage $w$ forever, sets $a=0$,  and has continuation utility $V_e
={u(w) \over 1-\beta} $. The consumption good is not storable and
workers can neither borrow nor lend. The unemployment agency can
borrow and lend at a constant one-period risk-free gross  interest
rate of $R = \beta^{-1}$. The unemployment agency cannot observe
the worker's effort  level.

\medskip
\noindent{\bf Subproblem A}
\medskip
\noindent{\bf a.}  Let $V$ be the value of (1) that the
unemployment agency has promised an unemployed
 worker at the start of a period (before he has made his search decision).
Let $C(V)$ be  the minimum cost to the unemployment insurance agency
of delivering promised value $V$. Assume that
the unemployment insurance agency wants the unemployed worker
to set $a_t = a$ for as long as he is unemployed (i.e., it wants to promote
high search effort).
Formulate a Bellman equation for $C(V)$, being careful to
specify any promise-keeping and incentive constraints.
(Assume that there are no participation constraints: the unemployed
worker must participate in the program.)

\medskip
\noindent{\bf b.} Show that if the incentive constraint binds,
then the  unemployment agency offers the worker benefits that
decline as the duration of unemployment grows.
\medskip
\noindent{\bf c.}  Now alter assumption (2) so that $\pi(a) = \pi(0)$.
Do benefits still decline with increases in the duration of unemployment?
Explain.
%\vfil\eject
\medskip
\noindent{\bf Subproblem B}
\medskip
\noindent {\bf d.}  Now assume that the unemployment insurance agency
can tax the worker after he has found a job, so that his continuation
utility upon entering a state of employment is
${u(w-\tau) \over 1 -\beta}$, where $\tau$ is a tax that is permitted
to depend on the duration of the unemployment spell.
Defining $V$ as above, formulate the Bellman equation for $C(V)$.
\medskip
\noindent{\bf e.}  Show how the tax $\tau$ responds to the duration
of unemployment.


\medskip
\noindent {\it Exercise \the\chapternum.6}
\quad { \bf Partially observed search effort}
\medskip
\noindent

\noindent Consider the following modification of a model of Hopenhayn and Nicolini.  An insurance agency wants to insure an infinitely lived unemployed
worker against the risk that he will not find a job.  With probability $p(a)$, an unemployed worker who searches with effort $a$ this period
will find a job that earns wage $w$ in consumption units per period. That job will start next period, last forever, and the worker will never quit it.    With probability $1-p(a)$ he will find himself unemployed again at the beginning of next period.  We assume that $p(a)$ is an increasing and strictly concave and twice differentiable function of $a$ with $p(a) \in [0,1]$
for $a\geq 0$ and $p(0)=0$.  The insurance agency is the worker's only source of consumption (there is no storage or saving available to the worker).
The worker values consumption according to a twice continuously differentiable and strictly concave utility function $u(c)$ where $u(0)$ is finite.
While unemployed, the worker's utility is $u(c) -a$; when he is employed it is $u(w)$ (no effort $a$ need be applied when he is working).


With exogenous probability $d \in (0,1)$ the insurance agency observes the search effort  of a worker who searched last period but did not find a job.
With probability $1-d$, the insurance agency  does not observe the last-period search intensity of an unemployed worker who was not successful in finding
a job  period.


Let $V$ be the expected discounted utility of an unemployed worker who is searching for work this period.  Let $C(V)$ be the minimum cost to the unemployment
insurance agency of delivering $V$ to the unemployed worker.

\medskip
\noindent{\bf a.}  Formulate a Bellman equation for $C(V)$.

\medskip
\noindent{\bf b.}  Get as far as you can in analyzing how the unemployment compensation contract offered to the worker depends on the duration
of unemployment and the history of observed search efforts that are detected by the UI agency.

\medskip

\noindent{\it Hint:} you might want to allow the continuation value when unemployed to depend on last period's search effort when it is observed.


