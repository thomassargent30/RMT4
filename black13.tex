
\input grafinp3
%\input grafinput8
\input psfig
%\eqnotracetrue

%\showchaptIDtrue
%\def\@chaptID{13.}

%\eqnotracetrue

\def\toone{{t+1}}
\def\ttwo{{t+2}}
\def\tthree{{t+3}}
\def\Tone{{T+1}}
\def\TTT{{T-1}}
\def\rtr{{\rm tr}}
\footnum=0
\chapter{Self-Insurance\label{selfinsure}}

\section{Introduction}

  This chapter describes a version of what is sometimes
called a savings problem (e.g., Chamberlain and Wilson, 2000).
A consumer wants to maximize the expected discounted sum of a
concave function of one-period consumption rates, as in
chapter \use{recurge}.  However, the consumer is cut off from
all insurance markets and almost all asset markets.  The
consumer can  purchase only nonnegative amounts of a single
risk-free asset.   The absence of insurance opportunities
induces the consumer to use variations over time in his asset holdings to acquire
``self-insurance.''
\auth{Chamberlain, Gary} \auth{Wilson, Charles}  \index{optimal savings problem}

  This model is interesting to us partly as a benchmark to compare
with the complete markets model of  chapter \use{recurge} and some of the
recursive contracts models of chapters \use{socialinsurance} and \use{socialinsurance2}, where
information and enforcement problems restrict allocations relative to
chapter \use{recurge}, but nevertheless permit more insurance than is
allowed in this chapter.  A version of the single-agent model
of this chapter will also be an  important component of the incomplete
markets models  of chapter \use{incomplete}.  Finally, the chapter
provides our first encounter with the powerful supermartingale convergence theorem.

To highlight the effects of uncertainty and borrowing constraints, we shall
study versions of the savings problem under alternative assumptions about the
stringency of the borrowing constraint and  about
whether the household's endowment stream is known or uncertain.

\section{The consumer's environment}

  An agent orders consumption streams according to
$$ E_0 \sum_{t=0}^\infty \beta^t u(c_t), \EQN selfobj$$
where $\beta \in (0,1)$, and $u(c)$ is a strictly increasing, strictly concave,
twice continuously differentiable function of the consumption
of a single good $c$.  The agent is endowed with
 an infinite random sequence $\{y_t\}_{t=0}^\infty$ of the good.
Each period, the endowment takes one of a finite
number of values, indexed by $s \in {\bf S}$. In particular,
the set of possible endowments is
$\overline y_1 < \overline y_2 < \cdots < \overline y_S$.
Elements of the sequence of endowments are independently and identically
distributed with ${\rm Prob} (y = \overline y_s) = \Pi_s, \Pi_s \geq 0$,
and $\sum_{s \in {\bf S}} \Pi_s = 1$.
 There are no
insurance markets.

 The
agent can hold nonnegative   amounts of a single risk-free
asset that has a  net rate of return $r$, where $(1+r) \beta =   1$.
 Let $a_t \geq 0$ be the agent's assets at the beginning
of period $t$, including the current realization of the income
process.
(Later we shall use an alternative  notation by
defining $b_t = - a_t + y_t$ as the {\it debt\/} of the consumer at the
beginning of period $t$, {\it excluding\/} the time $t$ endowment.)
 We assume that $a_0 =y_0$ is drawn from the
time-invariant endowment  distribution $\{\Pi_s\}$.
(This is equivalent to assuming that $b_0=0$ in the alternative notation.)
  The agent faces the sequence of budget constraints
$$
a_{t+1} = (1+r) (a_t - c_t) + y_{t+1} \,,                    \EQN self_bc
$$
where $0 \leq c_t \leq a_t$,  with $a_{0}$  given.
That $c_t \leq a_t$ expresses the constraint that holdings
of the asset at the end of the period
(which evidently  equal ${a_{t+1} - y_{t+1} \over 1+r})$
%(which equal
%$a_t - y_t$)
 must    be nonnegative.  The very important constraint $c_t \geq 0$
is either imposed
or comes from an Inada condition $\lim_{c \downarrow  0} u'(c) =
+\infty$.

The Bellman equation for an agent with $a>0$ is
\offparens
$$\EQNalign{
V(a) \;=\;& \max_{c} \, \left\{ u(c) \,+\, \sum_{s=1}^S
           \beta \, \Pi_s V\Bigl[(1+r)(a-c) + \overline y_s \Bigr] \right\} \hskip1cm \
             \EQN self_V \cr
          & \hbox{\rm subject to} \ \qquad 0\leq c\leq a\,,                     \cr}
$$
\autoparens
where $\overline y_s$ is the income realization in state $s\in {\bf S}$.
The value function $V(a)$ inherits some properties from $u(c)$;
in particular, $V(a)$ is increasing, strictly concave, and differentiable.

 \index{self-insurance}
``Self-insurance'' occurs when the agent uses savings to insure
himself against income fluctuations. On the one hand, in response
to low income realizations, an agent can draw down his savings and
avoid temporary large drops in consumption. On the other hand, the
agent can partly save high income realizations in anticipation of
poor outcomes in the future. We are interested in the long-run
properties of an optimal ``self-insurance'' scheme. Will the
agent's future consumption settle down around some level $\bar
c$?\NFootnote{As will occur in the model of social insurance
without commitment, to be analyzed in chapter
\use{socialinsurance}.}  Or will the agent eventually become
 impoverished?\NFootnote{As
in the case of social insurance with asymmetric information,
to be analyzed in chapter \use{socialinsurance}.} Following
the analysis of Chamberlain and Wilson (2000) and Sotomayor (1984),
we will show that neither of these outcomes occurs: consumption will diverge
to infinity!\auth{Chamberlain, Gary} \auth{Wilson, Charles}
\auth{Sotomayor, Marilda A. de Oliveira}

Before analyzing it under uncertainty, we'll briefly consider the savings
problem under a certain endowment sequence.  With a nonrandom endowment that
does not grow perpetually, consumption {\it does\/} converge.

\section{Nonstochastic endowment}

Without uncertainty, the question of insurance is moot.
However, it is instructive to study the optimal consumption decisions
of an agent with an uneven income stream who faces a borrowing constraint.
We break our analysis of the nonstochastic case into two parts,
depending on the stringency of the borrowing constraint.  We begin with
the least stringent possible borrowing constraint, namely, the natural
borrowing constraint on one-period Arrow securities, which are
risk free in the current context.  After that, we'll arbitrarily
tighten  the borrowing constraint to arrive at the no-borrowing
condition $a_{t+1} \geq y_{t+1}$
imposed in the statement of the problem in the previous section.
With the natural borrowing constraint, the outcome is that the agent completely
smooths consumption, having a constant consumption rate over time.  With the more
stringent no-borrowing constraint, in general the outcome will be different.  Here consumption
will be a monotonic increasing sequence with jumps in consumption at times when the no-borrowing constraint binds.

  For convenience, we temporarily use our alternative notation.  We
let $b_t$ be the   amount of one-period {\it debt\/} that
the consumer {\it owes\/} at time $t$;
$b_t$ is related to $a_t$ by
$$ a_t = - b_t + y_t ,$$
with $b_0 = 0$.
Here $-b_t$ is the consumer's asset position {\it before\/} the realization
of his  time $t$ endowment.
In this notation, the time $t$ budget constraint
 \Ep{self_bc} becomes
$$ c_t + b_t \leq \beta b_{t+1} + y_t \EQN self_bc1 $$
where in terms of $b_{t+1}$, we would express
a no-borrowing constraint ($a_{t+1}  \geq y_{t+1}$)
as $$b_{t+1} \leq 0. \EQN noborr $$

The no-borrowing constraint \Ep{noborr} is evidently more  stringent
than the natural borrowing constraint on one-period Arrow securities
that we imposed in chapter \use{recurge}.
Under an Inada condition on  $u(c)$ at $c=0$,  or alternatively when
$c_t \geq 0$ is imposed,
the natural borrowing constraint in this nonstochastic
case is found by
solving \Ep{self_bc1} forward with $c_t \equiv 0$:
$$ b_t  \leq \sum_{j=0}^\infty \beta^j y_{t+j} \equiv \overline b_t .
\EQN borrowcon $$
The right side is the present value of the endowment, which is the maximal amount that  it is feasible
to  repay at time $t$
when $c_t \geq 0$.

Solve \Ep{self_bc1}  forward and impose the initial condition $b_0=0$
and the terminal condition $ \lim_{T \rightarrow \infty} \beta^{T+1} b_{T+1} = 0 $ to get
$$  \sum_{t=0}^\infty \beta^t c_t \leq
 \sum_{t=0}^\infty \beta^t y_t .
\EQN constraint0 $$
When $c_t \geq 0$,
under the natural borrowing constraints, this
is the only restriction that the budget constraints \Ep{self_bc1}
impose on the $\{c_t\}$ sequence.
   The first-order necessary conditions  for  maximizing
\Ep{selfobj} subject to \Ep{constraint0} are
$$ u'(c_t) \geq u'(c_{t+1}), \quad = \ {\rm if} \ b_{t+1} < \overline b_{t+1}, \ \ t \geq 0.
\EQN lfonc3 $$
It is possible to satisfy these first-order conditions
by setting $c_t = \overline c$ for all $t \geq 0$, where $\overline c$
is the constant consumption level chosen to satisfy  \Ep{constraint0}
at equality:
$$ { \overline c \over 1 - \beta} = \sum_{t=0}^\infty \beta^t
  y_{t} .   \EQN consflat $$
At equality, the sequence of budget constraints \Ep{self_bc1} implies
that debt $b_t$ satisfies the following equation:
$$ \sum_{j=0}^\infty \beta^j c_{t+j} + b_t = \sum_{j=0}^\infty \beta^j y_{t+j}.   \EQN debt_eqn_2 $$
Under the particular consumption-smoothing policy \Ep{consflat}, $b_t$ is given by
$$\EQNalign{
b_t = \beta^{-t} \sum_{j=0}^{t-1} \beta^j \left(\overline c -y_j\right)
    &= \beta^{-t} \left({\overline c \over 1-\beta} - {\beta^t \overline c \over 1-\beta}
       - \sum_{j=0}^{t-1} \beta^j y_j \right)  \cr
    &=
\sum_{j=0}^\infty \beta^j y_{t+j} - {\overline c \over 1 - \beta} }$$
where the last equality invokes \Ep{consflat}. This expression for $b_t$
is evidently less than or equal to $\overline b_t$ for all
$t \geq 0$.
Thus, under the natural borrowing constraints,
we have constant consumption for $t \geq 0$, i.e., perfect consumption
smoothing over time.

The natural debt limits allow $b_t$ to be positive, provided that
it is not too large. Next we shall study the more severe ad hoc
debt limit that requires $- b_t \geq 0$, so that the consumer can
{\it lend\/}, but not borrow.  This restriction will limit consumption smoothing
for households whose incomes are growing, and  who therefore are
naturally borrowers.\NFootnote{See exercise {\it \the\chapternum.1\/} for how income growth
and shrinkage impinge on consumption in the presence of an ad hoc
borrowing constraint.}

\subsection{An ad hoc borrowing constraint: nonnegative assets}\label{sec:ad_hoc_borrowing}%
We continue to assume a known endowment sequence but
 now impose a no-borrowing constraint
$(1+r)^{-1} b_{t+1}  \leq 0  \
 \forall t \geq 0$.  To facilitate the transition to our subsequent
analysis of the problem under uncertainty, we work in terms of a
definition of assets that include this period's income, $a_t =
-b_t+y_t$.\NFootnote{When $\{y_t\}$ is an i.i.d.\ process, working with
$a_t$ rather than $b_t$  makes it possible to formulate the
consumer's Bellman equation in terms of the single state variable
$a_t$, rather than the pair $b_t, y_t$. We'll exploit this idea
again in chapter \use{incomplete}.} Let $(c_t^*, a_t^*)$ denote an
optimal path. First-order necessary conditions for an optimum are
$$ u'(c_t^*) \geq u'(c_{t+1}^*),  \quad = \ {\rm if} \ c_t^* < a_t^*
 \EQN lfonc2 $$
for $t\geq 0$.
Along an optimal path for $t \geq 1$, it must be true that either
\medskip
\itemitem{(a) } $c^*_{t-1} = c^*_t$; or
\itemitem{(b) } $c^*_{t-1} < c^*_t$ and $c^*_{t-1} = a^*_{t-1}$,
and hence $a^*_{t} = y_t$.
\medskip
\noindent
Condition (b) states that the no-borrowing constraint binds only when
the consumer desires to shift consumption from the future to the present.
He will desire to do that only when his endowment is
growing.

According to conditions a and b, $c_{t-1}$ can never exceed $c_t$.
The reason is that a declining consumption sequence can be
improved by cutting a marginal unit of
consumption at time $t-1$ with a utility loss of $u'(c_{t-1})$
and increasing consumption at time $t$ by the saving plus
interest with a discounted utility gain of
$\beta (1+r) u'(c_t) = u'(c_t) >  u'(c_{t-1})$, where the
inequality follows from the strict concavity of $u(c)$ and
$c_{t-1}>c_t$. A symmetric argument rules out
$c_{t-1}<c_t$ as long as the nonnegativity constraint on savings
is not binding; that is, an agent would choose to cut his savings
to make $c_{t-1}$ equal to $c_t$ as in condition a. Therefore,
consumption increases from one period to another as in condition b
only for a constrained agent with zero savings,
$a^*_{t-1}-c^*_{t-1}=0$. It follows that next period's assets
are then equal to next period's income, $a^*_t = y_t$.

Solving the
budget constraint \Ep{self_bc} at equality
forward for $a_t$ and rearranging gives
$$ \sum_{j=0}^\infty \beta^j c_{t+j} = a_t + \sum_{j=1}^\infty
  \beta^j y_{t+j}.   \EQN lbudget5 $$
At dates $t\geq 1$ for which $a_t = y_t$,  so that the no-borrowing constraint
was binding at time $t-1$,  \Ep{lbudget5} becomes
$$\sum_{j=0}^\infty \beta^j c_{t+j} = \sum_{j=0}^\infty \beta^j y_{t+j}.
\EQN lbudget6 $$
Equations \Ep{lbudget5} and \Ep{lbudget6} contain important information
about the  optimal solution.
Equation \Ep{lbudget5} holds for all dates  $t \geq 1$  at which
the consumer arrives
with positive net assets $a_t - y_t >0$.  Equation
\Ep{lbudget6} holds for those dates $t$ at which net assets
or savings
$a_t - y_t$
are zero, i.e., when the no-borrowing constraint was
binding at $t-1$.   If the no-borrowing constraint is binding
only finitely often, then after the {\it last\/} date $\overline t -1$
at which it was binding, \Ep{lbudget6} and the Euler equation
\Ep{lfonc2} imply that consumption will thereafter be constant
at a rate $\tilde c$ that satisfies
${\tilde c \over 1 -\beta } = \sum_{j=0}^\infty \beta^j
y_{{\overline t}+j}.$

In more detail,
suppose that an agent arrives in period $t$ with zero savings and
knows that the borrowing constraint will never bind again. He
would then find it optimal to choose the highest sustainable
{\it constant\/} consumption.  This is given by the annuity return on
the present value of the tail of the income process starting from period $t$,
$$
x_t\;\equiv\; {r \over 1+r} \sum_{j=t}^\infty (1+r)^{t-j} y_{j}\,.
\EQN self_x $$
In the optimization problem under certainty, the
impact of the borrowing
constraint will not vanish until the date at which
the annuity return on  the present value of the tail (or remainder) of the
income process is maximized. We state this in the following proposition.

\medskip
\noindent{\sc Proposition 1:} Given a borrowing constraint and a nonstochastic
endowment stream, the limit of
the nondecreasing optimal consumption path is
$$
\bar c \;\equiv\; \lim_{t \rightarrow \infty} c^*_t
\;=\; \sup_t x_t \;\equiv\;\bar x \,.              \EQN self_barc
$$
\medskip
\noindent{\sc Proof:}
We will first show that $\bar c \leq \bar x$. Suppose to the contrary
that $\bar c > \bar x$. Then conditions a and b imply that
there is a $t$ such that $a^*_t=y_t$ and $c^*_{j}>x_t$ for all
$j\geq t$. Therefore, there is a $\tau$ sufficiently large that
$$
\offparens
0 \;<\; \sum_{j=t}^\tau (1+r)^{t-j}
\bigl(c^*_{j} \,-\, y_{j} \bigr) \;=\;
(1+r)^{t-\tau} \bigl( c^*_{\tau} \,-\, a^*_{\tau} \bigr)\,,
$$
\autoparens
where the equality uses $a^*_t=y_t$ and successive iterations on
budget constraint \Ep{self_bc}. The implication that
$c^*_{\tau} > a^*_{\tau}$
constitutes a contradiction because it violates the constraint that
savings are nonnegative in optimization problem \Ep{self_V}.

To show that $\bar c \geq \bar x$, suppose to the contrary that
$\bar c < \bar x$. Then there is an $x_t$ such that $c^*_{j} < x_t$
for all $j \geq t$, and hence
$$ \eqalign{
 & \sum_{j=t}^\infty (1+r)^{t-j} c^*_{j} \;<\;
\sum_{j=t}^{\infty} (1+r)^{t-j} x_{t} \;=\;
\sum_{j=t}^{\infty} (1+r)^{t-j} y_{j} \cr &  \;\leq\;
a^*_t \,+\, \sum_{j=t+1}^\infty (1+r)^{t-j} y_{j} \,, \cr}
$$
where the last weak inequality uses $a^*_t \geq y_t$.
Therefore, there is an $\epsilon > 0$ and $\hat \tau > t$ such
that for all $\tau > \hat \tau$,
$$
\sum_{j=t}^\tau (1+r)^{t-j} c^*_{j} \;<\;
a^*_t \,+\, \sum_{j=t+1}^{\tau} (1+r)^{t-j} y_{j} \,-\, \epsilon \,,
$$
and after invoking budget constraint \Ep{self_bc} repeatedly,
$$
(1+r)^{t-\tau} c^*_{\tau} \;<\;
(1+r)^{t-\tau} a^*_{\tau} \,-\, \epsilon \,,
$$
or, equivalently,
$$
 c^*_{\tau} \;<\; a^*_{\tau} \,-\, (1+r)^{\tau -t} \epsilon \,.
$$
We can then construct an alternative feasible consumption sequence
$\{c^{\epsilon}_j\}$ such that $c^{\epsilon}_j=c^*_j$ for $j \not= \hat \tau$
and $c^{\epsilon}_j=c^*_j + \epsilon$ for $j=\hat \tau$. The fact
that this alternative sequence yields higher utility establishes the
contradiction. \qed
\medskip

  More generally, we know that at each date $t\geq 1 $
for which the no-borrowing
constraint is binding at date $t-1$, consumption will increase to   satisfy
\Ep{lbudget6}.   The time series of consumption will thus be a
discrete time
step function whose jump dates $\overline t$
 coincide with the dates at which
$x_t$ attains new highs:
$$ \overline t = \{t: x_t > x_s, s <  t\}. $$
If there is a finite last date $\overline t$,
optimal consumption is a monotone bounded sequence that converges to a
finite limit.


In summary, we  have shown that under certainty, the optimal consumption
sequence converges to a finite limit as long  as the discounted value
of future income is bounded across all starting dates $t$.
Surprisingly enough, that result is
overturned when there is uncertainty. But first, consider a simple example of
a nonstochastic endowment process.

\subsection{Example: periodic endowment process}

Suppose that the endowment oscillates between one unit of the consumption good in even periods
and zero units in odd periods. The annuity return on this
endowment process is equal to
$$ \EQNalign{ x_t \Big|_{t\ {\rm even}} &=
    {r \over 1+r} \sum_{j=0}^\infty (1+r)^{-2j}
   = ( 1 -\beta) \sum_{j=0}^\infty \beta^{2j}
                      = {1 \over 1 + \beta }\,,  \hskip1cm \ \EQN self_exev;a \cr
x_t \Big|_{t\ {\rm odd}} &=
    {1 \over 1+r} \, x_t \Big|_{t\ {\rm even}}
   = { \beta \over 1 + \beta}\,.
 \EQN self_exev;b \cr}
$$
According to Proposition 1, the limit of the optimal consumption path
is then $\bar c = (1+\beta)^{-1}$. That is, as soon as the agent
reaches the first even period in life, he sets consumption equal
to $\bar c$ forevermore. The associated
beginning-of-period assets $a_t$ fluctuates between
$(1+\beta)^{-1}$ and 1.

The exercises at the end of this chapter contain more
examples.

\section{Quadratic preferences}

It is useful briefly to consider the linear quadratic permanent
income model as a benchmark for the results to come.  Assume as
before that $\beta(1+r)=1$ and that the household's budget
constraint at $t$ is \Ep{self_bc1}. Rather than the no-borrowing
constraint \Ep{noborr}, we impose that\NFootnote{The natural borrowing
limit assumes that consumption is nonnegative, while the model with
quadratic preferences permits consumption to be negative.  When
consumption can be negative, there seems to be no natural lower
bound to the amount of debt that could be repaid, since more
payments can always be wrung out of the consumer.  Thus, with
quadratic preferences and the associated possibility of negative consumption, we have to rethink the sense of a borrowing
constraint.  The restriction \Ep{borrowlimit2} allows negative
consumption but limits the rate at which debt is allowed to grow in
a way designed to rule out a  Ponzi scheme that would have the consumer
always consume bliss consumption by accumulating debt without
limit.}
 $$E_0\left( \lim_{t\rightarrow \infty}
 \beta^t b_t^2   \right)
= 0. \EQN borrowlimit2 $$  This constrains
the asymptotic rate at which debt can grow.
Subject to this constraint, solving \Ep{self_bc1}
forward yields
$$ b_t = \sum_{j=0}^\infty \beta^j (y_{t+j} - c_{t+j}).
\EQN fr1 $$

 We alter the preference specification
above to make $u(c_t) $ a quadratic function
$-.5 (c_t - \gamma)^2$, where $\gamma >0$ is a bliss
consumption level. Marginal utility is
linear in consumption: $u'(c) = \gamma - c$.
We put no bounds on $c$; in particular, we allow
consumption to be negative.  We allow $\{y_t\}$
to be an arbitrary stationary stochastic process.

The weakness of constraint
\Ep{borrowlimit2} allows the household's first-order
condition to prevail with equality at all $t \geq 0$:
$u'(c_t) = E_t u'(c_{t+1})$.  The linearity of
marginal utility in turn implies
$$ E_t c_{t+1} = c_t,   \EQN mmart $$
which states that $c_t$ is a martingale.
Combining \Ep{mmart} with \Ep{fr1} and taking
expectations  conditional on time $t$ information
gives
$ b_t = E_t \sum_{j=0}^\infty \beta^j y_{t+j}
  - {1 \over 1-\beta} c_t $
or
$$ c_t = {r \over 1+r} \left[-b_t +
E_t \sum_{j=0}^\infty \left({1 \over 1+r}\right)^j y_{t+j} \right].
\EQN fr2 $$
Equation  \Ep{fr2} is a version of the permanent
income hypothesis and tells the consumer to set
his current consumption equal to the annuity return on his nonhuman ($-b_t$) and
human   wealth
($E_t \sum_{j=0}^\infty \left({1 \over 1+r}\right)^j y_{t+j}
$).   We can substitute this consumption
rule into \Ep{self_bc1} and rearrange to get
$$ b_{t+1} = b_t + r E_t \sum_{j=0}^\infty
  \left({1 \over 1+r}\right)^j y_{t+j}  -(1+r)y_t. \EQN fr3 $$
Equations \Ep{fr2} and \Ep{fr3}  imply that
under the optimal  policy,
$c_t, b_t$ both have unit roots and that
they are cointegrated.\NFootnote{See section \use{sec:LQmodel}, especially page \use{cointegrationtag}.}
  \index{cointegration}%
If we define permanent income at $t$ as
$ y_{pt} \equiv {\frac{r}{1+r}} E_t \sum_{j=0}^\infty (1+r)^{-j} y_{t+j}$, then \Ep{fr3}
can be represented as
$$
b_{t+1} = b_t + (1+r)[y_{pt} - y_t ] \EQN fr33, $$
which states that the time $t$ increment in the  consumer's debt equals $(1+r)$ times the difference between his time $t$ permanent income and his current income.


  Consumption rule \Ep{fr2} has the remarkable  feature
of \idx{certainty equivalence}:  consumption $c_t$ depends only
on the first moment of the discounted value of the endowment
sequence.  In particular, the conditional variance  of the present
value of the endowment does
not matter.\NFootnote{This property of the consumption rule reflects
the workings of the type of certainty equivalence that we
discussed in chapter \use{dplinear}.} Under rule \Ep{fr2},
consumption is a martingale and the consumer's assets
$b_t$ are a unit root process.  Neither consumption nor
assets converge, though at each point in time, the consumer
expects his consumption not to drift in its average value.

The next section shows that
these outcomes  change dramatically when we alter
the specification of the utility function to rule out
negative consumption.

\section{Stochastic endowment process: i.i.d.\ case}

With uncertain endowments, the first-order condition for the
optimization problem  \Ep{self_V} is
\offparens
$$
u'(c) \;\geq\; \sum_{s=1}^S \beta (1+r) \Pi_s
            V'\Bigl[(1+r)(a-c) + \overline y_s \Bigr]\,,              \EQN self_foc1
$$
\autoparens
with equality if the nonnegativity constraint on savings is
not binding. The \idx{Benveniste-Scheinkman formula} implies $u'(c)=V'(a)$,
so the first-order condition can also be written as
$$
V'(a) \;\geq\; \sum_{s=1}^S \beta (1+r) \Pi_s
            V'(a^\prime_s)\,,                             \EQN self_foc2
$$
where $a^\prime_s$ is next period's assets if today's income shock
is $\overline y_s$. (Recall again that $V(a)$ is increasing, strictly concave, and differentiable.)
 Since $\beta^{-1}=(1+r)$, $V'(a)$ is a nonnegative
supermartingale.
By a theorem of Doob (1953, p. 324),\NFootnote{See the appendix
of this chapter for a statement of the theorem.}
$V'(a)$ must then converge almost surely.\NFootnote{See footnote \use{ftsupermart} on page \use{supermart_first} for an earlier encounter with
this force.} Here is how to think about this result.  Think about generating a   realization of the entire endowment sequence $\{y_t\}$ via a computer simulation.
  For each such sample realization of $\{y_t\}$, the sequence  $\{V'(a_t)\}$ converges to a particular number $\lim_{t\rightarrow \infty} V'(a_t)$.
  The particular number converged to can vary across realizations, possibly tracing out a nontrivial limiting distribution of $V'(a)$.
  But in our case, the distribution of the limiting value of $V'(a)$ turns out to be degenerate and concentrated at a single value.
\index{supermartingale convergence theorem}%
\index{martingale!convergence theorem}%
In particular, the limiting value of $V'(a)$ must be zero because of the following
argument. % Suppose to the contrary that $V'(a)$ converges    for some realization
%to a strictly positive limit. That implies that $a$ converges
%to a finite positive value. But this implication is
%immediately contradicted when we take into account the budget constraint \Ep{self_bc} together with  properties of  the optimal policy
%function, which makes  $c$ an increasing  function of $a$:
%%here assets are equal to the value of the past period's savings
%%including interest, and a stochastic income $y_s$.
%randomness
%of $y$ and therefore of $c$ contradicts a finite limit for $a$. Instead,
%$V'(a)$ must converge to zero, implying that assets diverge
%to infinity.
%(We return to this result in
%chapter \use{incomplete}.)   %% on incomplete market models.)
Suppose to the contrary that $V'(a)$ converges to a strictly positive limit. That  implies that $a$ converges to a finite positive value. But this implication is  contradicted by budget constraint \Ep{self_bc}, %(13.2)
 which states that  assets are equal to the value of the past period's savings including interest plus a stochastic income $y_s$. The random nature of $y_s$ contradicts a finite limit for $a$. Instead, $V'(a)$
  must converge to zero, implying that assets converge to infinity. (We return to this result in
chapter \use{incomplete}.)

Although assets diverge to infinity, they do not increase
monotonically. Since assets are used for self-insurance,
 {\it low\/} income realizations are associated with
{\it reductions\/} in assets.
To show this outcome,
 suppose to the contrary that even the lowest income realization
$\overline y_1$ is associated with nondecreasing assets; that is,
$(1+r)(a-c)+\overline y_1 \geq a$. Then we have
\offparens
$$\EQNalign{
V'\Bigl[(1+r)(a-c) + \overline y_1 \Bigr] &\;\leq\;
V'(a) \cr
&\;=\; \sum_{s=1}^S \Pi_s
V'\Bigl[(1+r)(a-c) + \overline y_s \Bigr] \,, \hskip1cm \      \EQN self_contradict \cr}
$$
\autoparens
where the last equality  is first-order condition \Ep{self_foc2}
when the nonnegativity constraint on savings is not binding and when
$\beta^{-1}=(1+r)$.
Since $V'[(1+r)(a-c) + \overline y_s ] \leq
V'[(1+r)(a-c) + \overline y_1 ]$ for all $s\in {\bf S}$, expression
\Ep{self_contradict} implies that the derivatives of $V$ evaluated
at different asset values are equal to each other, an implication
that is contradicted by the strict concavity of $V$.

The fact that assets diverge to infinity means that the individual's
consumption also diverges to infinity. After invoking the
\idx{Benveniste-Scheinkman formula}, first-order condition \Ep{self_foc1}
can be rewritten as
$$
u'(c) \;\geq\; \sum_{s=1}^S \beta (1+r) \Pi_s u'(c^\prime_s)
      \;=\; \sum_{s=1}^S  \Pi_s u'(c^\prime_s)\,,           \EQN self_foc3
$$
where $c^\prime_s$ is next period's consumption if the income
shock is $\overline y_s$, and the last equality uses $(1+r)=\beta^{-1}$.
It is important to recognize that the individual will never find
it optimal to choose a time-invariant consumption level for the
indefinite future. Suppose to the contrary that the individual
at time $t$ were to choose a constant consumption level for all
future periods. The maximum constant consumption level that would
be sustainable under all conceivable future income realizations
is the annuity return on  his current assets $a_t$ and a stream of
future incomes all equal to the lowest income realization. But
whenever there is a future period with a higher income realization,
we can use an argument similar to our section \use{sec:ad_hoc_borrowing} construction of the sequence
$\{c^\epsilon_j\}$ in the case of certainty to show that
the initial time-invariant consumption level does not maximize
the agent's utility. It follows that future consumption
will vary with income realizations and  that consumption cannot
converge to a finite limit with an i.i.d.\ endowment process.
Hence, from  the martingale convergence
theorem, the nonnegative supermartingale $u'(c)$ in \Ep{self_foc3}
must converge to zero, since any strictly positive limit would
imply that consumption
converges to a finite limit, which cannot be.




\section{Stochastic endowment process: general case}

The result that consumption diverges to infinity with an i.i.d.\
endowment process is extended by
Chamberlain and Wilson (2000) to an arbitrary stationary
stochastic endowment
process that is sufficiently stochastic. Let $I_t$ denote the information
set at time $t$. Then the general version of first-order condition
\Ep{self_foc3} becomes
\offparens
$$
u'(c_t) \;\geq\; E\Bigl[u'(c_{t+1}) \Big| I_t\Bigr],           \EQN self_foc4
$$
\autoparens
where $E(\cdot|I_t)$ is the expectation operator conditioned upon
information set $I_t$. Assuming a bounded utility function,
Chamberlain and Wilson prove the following result, where $x_t$ is defined
in \Ep{self_x}:

\medskip
\noindent{\sc Proposition 2:}
If there is an $\epsilon >0$ such that for any $\alpha\in {\bbR}^+$
\offparens
$$
P\Bigl( \alpha \leq x_t \leq \alpha + \epsilon \Big| I_t \Bigl) < 1- \epsilon$$
for all $I_t$ and $t\geq0$, then $P(\lim_{t\to\infty} c_t = \infty)=1$.
\medskip

Without providing a proof here, it is useful to make a connection to
the nonstochastic case in Proposition 1. Under certainty, the limiting
value of the consumption path is given by the highest annuity return on  the
endowment process across all starting dates $t$; $\bar c = \sup_t x_t$.
Under uncertainty, Proposition 2 says that the consumption path will
never converge to any finite limit if the annuity return on  the endowment
process is sufficiently stochastic. Instead, the optimal consumption path
will  converge to infinity. This stark difference between the case of
certainty and uncertainty is quite remarkable.\NFootnote{In exercise
{\it \the\chapternum.3,\/} you will be asked to prove that the divergence of consumption
to $+\infty$ also occurs under a stochastic counterpart to the
natural borrowing limits. These are less  stringent than the no-borrowing
condition.}

\section{Intuition}

Imagine that you perturb any constant endowment stream by adding the slightest
i.i.d.\ component.  Our two propositions then say that  the optimal consumption
path changes from being a constant to becoming a stochastic process
that goes to infinity.  Beyond appealing to martingale convergence theorems,
 Chamberlain and Wilson (2000, p.\ 381) comment on the difficulty
of developing economic intuition for this startling finding:

{\narrower\smallskip\noindent
Unfortunately, the line of argument used in the proof does not provide
a very convincing economic explanation. Clearly the strict concavity
of the utility function must play a role. (The result does not hold if,
for instance, $u$ is a linear function over a sufficiently large domain
and $(x_t)$ is bounded.) But to simply attribute the result to risk
aversion on the grounds that uncertain future returns will cause
risk-averse consumers to save more, given any initial asset level, is
not a completely satisfactory explanation either. In fact, it is a bit
misleading. First, that argument only explains why expected accumulated
assets would tend to be larger in the limit. It does not really explain
why consumption should grow without bound. Second, over any finite time
horizon, the argument is not even necessarily correct. \smallskip}

\noindent
Given a finite horizon, Chamberlain and Wilson proceed to discuss
how mean-preserving spreads of future income leave current consumption
unaffected when the agent's utility function is quadratic over a
sufficiently large domain.


We believe that the economic intuition is to be found in the strict
concavity of the utility function {\it and} the assumption that the marginal
utility of consumption must remain positive for any arbitrarily high
consumption level. This rules out quadratic utility,
for example.  To advance this explanation, we first focus on
utility functions whose marginal utility of consumption is
strictly convex, i.e., $u'''>0$ if the function is thrice differentiable.
Then, Jensen's
inequality implies $\sum_s \Pi_s u'(c_s) >u'( \sum_s \Pi_s c_s)$;
first-order condition \Ep{self_foc3} then implies
$$
c \;<\; \sum_{s=1}^S  \Pi_s c^\prime_s\,,          \EQN self_cons
$$
where the strict inequality follows from our earlier argument
that future consumption levels will not be constant but will
vary with income realizations.
In other words, when the marginal utility of consumption is strictly
convex, a given absolute
decline in consumption is not only more costly in utility than a
gain from an identical absolute increase in consumption, but the former
is also associated with a larger rise in {\it marginal\/} utility as compared
to the drop in {\it marginal\/} utility of the latter.
To set
today's marginal utility of consumption equal to next period's
expected marginal utility of consumption, the consumer
must therefore balance future states with expected declines in
consumption against  appropriately
higher expected increases in consumption
for other states. Of course, when next period arrives  and the
consumer chooses optimal consumption
(which is then on average higher than
last period's consumption), the same argument applies again.
That is, the process exhibits a ``ratchet effect'' by which consumption
tends toward ever higher levels. Moreover, this on-average increasing
consumption sequence cannot converge to a finite limit because of
our earlier argument based on an agent's desire to exhaust all
his resources while respecting his budget constraint.

 This argument for the optimality of unbounded consumption growth
applies to utility functions whose marginal utility of consumption is
strictly convex. But even utility functions that do not have convex marginal
utility globally  must ultimately conform to a similar condition over
long enough intervals of the positive real line, because otherwise
those utility functions would eventually violate the assumptions of a
strictly positive, strictly diminishing marginal utility
of consumption, $u'>0$ and $u''<0$. Chamberlain and Wilson's
reference to a quadratic utility function illustrates the problem
of how otherwise the marginal utility of consumption will  turn negative
at large consumption levels. Thus,
our understanding of the remarkable result in Proposition 2 is
aided by considering the inexorable ratchet effect on consumption
implied by the first-order condition for the agent's
optimal intertemporal choice.


\section{Endogenous labor supply}\label{sec:self_insure_labor}%
Contributions by
Marcet, Obiols-Homs, and Weil (2007) and Zhu (2009) studied how  adding an endogenous labor supply
decision would affect outcomes in the presence of a  precautionary savings motive.  A key
insight is that a wealth effect on labor
supply can instruct an infinitely lived household to accumulate sufficient wealth to retire  and
thereby isolate itself thereafter from non-financial income risk.  That force can allow both assets and consumption to converge
while work converges to zero.   The Marcet, Obiols-Homs, and Weil (2007) and Zhu (2009) work provides valuable additional intuition
about the structure of the divergence of assets and consumption in the Chamberlain-Wilson model.

\auth{Marcet, Albert}%
\auth{Obiols-Homs, Francesc}%
\auth{Weil,  Philippe}%
\auth{Zhu, Shenghao}%
At each date $t \geq 0$, a household chooses $c_t \geq 0$ and
$h_t \in [0, 1]$, where $c_t$ is the consumption of a single good and
$h_t$ is leisure.
The household orders a stochastic process $\{c_t, h_t\}_{t=0}^\infty$ according to
the utility functional
$$ E_0 \sum_{t=0}^\infty \beta^t u(c_t, h_t), \quad 0 < \beta < +\infty . \EQN CWM_1 $$
Following Zhu (2009), we adopt the assumptions that
$u(c,h)$ is  (A1), twice continuously differentiable; (A2) strictly increasing and strictly concave in $c$ and $h$,
 $\lim_{c \rightarrow 0} u_1(c,h) = +\infty \forall h \in [0,1]$, and
 $\lim_{h \rightarrow 0} u_2(c,h) = + \infty \forall c \geq 0$; and
 (A3) $u(c,h) \in [0, M], M >0$.


The consumer can hold nonnegative amounts of a single asset that bears a constant
net rate of interest $r > 0$. We assume that $\beta R =1$, where $R = (1+r)$.
The consumer receives labor income $(1-h_t) e_t w$ at time $t$,
where $w$ is a fixed wage and $e_t$ is the  time $t$ realization of
a productivity shock.  The productivity or labor efficiency shock $e_t$ follows
a discrete state Markov process on the state space $E = \bmatrix{\bar e_1 & \ldots & \bar e_n}$ where $[0 < \bar e_1 < \cdots < \bar e_n] $;
 the transition density $\pi(e' | e)$ satisfies $\sum_{e'} \pi(e'|e ) = 1, \pi(e' | e) >0 \ \forall \ (e, e') \in E \times E$.
The consumer's time $t$ budget constraint is $ c_t + A_{t+1} = R A_t + (1-h_t) e_t w $
or
$$ c_t +  h_t e_t w =  R A_t + e_t w - A_{t+1}.  \EQN CWM_budget2 $$
We regard the left side as the consumer's total time $t$ expenditures on consumption and leisure.

The household chooses $\{c_t, h_t,  A_{t+1}\}_{t=0}^\infty$ to maximize \Ep{CWM_1} subject to $A_0 \geq 0$ given,
\Ep{CWM_budget2} for all $t \geq 0$, and the information assumption that  $A_t, e_t$ are known at time $t$.
\auth{Foley, Duncan K.}%
\auth{Hellwig, Martin F.}%
Following Foley and Hellwig (1975) and Zhu (2009), we solve the problem in two steps. Step 1 solves an intratemporal
problem and step 2 solves an intertemporal problem.
\medskip
\noindent{\bf Step 1:} Form the indirect utility function $J(Y,e)$ defined as
$$ J(Y,e) = \max_{c,h} u(c,h) \EQN Jindirect $$
subject to
$$ c + h e w = Y, \quad 0 \leq h \leq 1, c \geq 0 . \EQN Jconstraint $$
Here $Y= R A + we - A'$ stands for total current  expenditures on consumption an leisure.
The first-order necessary condition for the problem on the right side of \Ep{Jindirect} is
$$ {\frac{u_2(c,h)}{u_1(c,h)}} \geq e w , \quad = {\rm if \ } h < 1 . \EQN Jfonc $$
The right side of \Ep{Jindirect} is attained by policy functions $c=c(Y,e), h=h(Y,e)$.
Under assumptions A1-A3, Zhu (2009) shows that  the indirect utility function
$J(Y,e)$  is bounded; strictly increasing and strictly concave in $Y$; and
continuously differentiable in $Y$ with
$$ J_1(Y, e) = u_1(c(Y,e),h(Y,e))  \quad \forall Y \in (0, +\infty) . \EQN JYderiv $$

\medskip
\noindent{\bf Step 2:}  Solve the intertemporal maximization problem
$$ V(A_0, e_0) = \max_{Y_t \geq 0} E_0 \sum_{t=0}^\infty \beta^t J(Y_t, e_t) \EQN CWM_intertemporal $$
subject to the sequence of constraints
$$ Y_t + A_{t+1} = R A_t + e_t w, \quad t \geq 0 \EQN Vconstraint_intertemporal $$
and given $A_0 \geq 0$. The  Bellman equation  associated with this problem is
$$ V(A,e) = \max_{A' \in \Gamma(A,e)} \Bigl\{ J(RA + ew - A', e) + \beta E \bigl[ V(A',e') | e \bigr] \Bigr\} \EQN Bellman_CWM $$
where
$$ \Gamma(A,e) = \{ A' : 0 \leq A' \leq RA + e w \} .$$
An optimum policy $Y = Y(A,e)$ and the implied  $A'= A(A,e)$ attain
$$V(A,e) = J(Y(A,e),e) + \beta \sum_{e'} V\bigl(A(A,e),e'\bigr) \pi(e'|e) .$$
Under assumptions A1-A3, Zhu shows that $V(A,e)$ is continuous, strictly increasing,  strictly concave in $A$,   continuously differentiable,
and that
$$ V_1(A,e) = R J_1(Y(A,e),e) \quad \forall A \in [0,+\infty] \EQN VA_deriv .$$
Zhu shows that $A(A,e)$ is continuous and weakly increasing in $A$ and that $Y(A,e)$ is strictly increasing in $A$.
Furthermore, the Inada conditions A2 imply that
$\lim_{Y \rightarrow 0} J_1(Y, e) = +\infty \ \forall e \in E$. Assembling earlier results also tells us
that
$$ V_1(A, h) = R J_1(A,e) = R u_1(c,h). \EQN V_1_formula $$


\medskip
\noindent{\bf Outcome with endogenous labor supply:}
The first-order necessary condition for asset choice is
$$ V_1(A,e) \geq \beta R E [ V_1(A',e') |e ], \quad = {\rm if} \ A' > 0 . \EQN supermart_zhu$$
We assume that $\beta R =1 $ as we have done throughout this chapter. When $\beta R =1$,
inequality \Ep{supermart_zhu} asserts that $V_1(A,h)$ and therefore $u_1(c,h)$ is a
nonnegative supermartingale.  Applying Doob's supermartingale convergence theorem  implies that
$\lim_{t\rightarrow +\infty} V_1(A_t, e_t) =\lim_{t \rightarrow +\infty} R u_1(c_t, h_t)  $ exists and is almost surely finite.
This result leaves open two possibilities: Either

\medskip
\noindent{\bf 1.} $V_1(A_t, e_t) = R u_1(c_t, h_t) \rightarrow 0$ while $h_t$ remains uniformly bounded away from  one infinitely often with the
outcomes that
$\lim_{t\rightarrow \infty } A_t = +\infty $ and $\lim_{t\rightarrow \infty} c_t = + \infty$; or
\medskip
\noindent{\bf 2.}  $h_t \rightarrow 1 , A_t \rightarrow \bar A, c_t \rightarrow r \bar A$ almost surely.
\medskip

Case 1 is a version of our earlier result with exogenous and perpetually random non-financial income.  Here  a precautionary savings motive
causes assets and consumption both to diverge to $+\infty$.  The possibility of case 2 inspired
Marcet, Obiols-Homs, and Weil (2007) and Zhu (2009) to make  the labor supply decision  endogenous. What drives case 2 is a wealth effect that causes the household eventually
to withdraw all labor from the market and thereafter consume leisure 100\% of his time. That shuts down his effective exposure to the random labor productivity process and  extinguishes subsequent randomness in his income process.

Whether case 1 or 2 prevails depends on the shape of the consumer's utility function $u(c,h)$.  Zhu (2009)
considers the following two assumptions that tilt things toward case 2. Assumption A4 asserts
that $u_{12} u_1 - u_{11}u_2 >0 $ and $u_{12} u_2 - u_{22} u_1 >0$.  This assumption makes
$c$ and $h$ both be normal goods, and also implies that ${\frac{u_2}{u_1}}$ is {\it increasing} in $c$ and {\it decreasing} in $h$. It also
implies that $c(Y,e)$ and $h(Y,e)$ are both increasing in $Y$.  Zhu also makes the stronger assumption
A4' that $u_{12} >0$, which makes $c$ and $h$ be complements and implies A4.
Zhu (2009) establishes  the following:

\medskip
\noindent {\sc Proposition:}   Under assumptions A1-A4', (a) $c(A,Y)$ and $h(A,Y)$ are both continuous and increasing in $A$;
(b) $h(A,e)=1 \quad \forall e$ when $A$ is sufficiently large.
\medskip

\noindent Zhu shows how this proposition is the heart of an argument that generates sufficient conditions for case 2 to prevail. In this way,
he constructs circumstances that disarm the divergence outcomes of Chamberlain and Wilson.\NFootnote{Zhu also provides
examples of preferences that push things toward case 1.  An example is  preferences of a type used by
Greenwood, Hercowitz, and Huffman (1988): $u(c,h) = U(c-G(1-h))$, where $U' >0, U'' < 0, G' >0, G'' >0$ with $U$ bounded above.  Here
the marginal rate of substitution between $c$ and $h$ depends only on $h$ and labor supplied is independent of the intertemporal consumption-savings
choice.}
\auth{Greenwood, Jeremy}%
\auth{Hercowitz, Zvi}%
\auth{Huffman, Gregory W.}%


\section{Concluding remarks}

  This chapter has maintained the assumption that $\beta (1+r) =1$,
which is a very important ingredient in delivering the divergence
toward infinity of the agent's asset and consumption level.
  Chamberlain and Wilson (1984)
study a much more general version of the model where they relax this condition.
\auth{Chamberlain, Gary} \auth{Wilson, Charles}

To build
some incomplete markets models, chapter \use{incomplete} will put together continua of agents
facing generalizations of the savings problems.  The models of that chapter will
determine the interest rate $1+r$ as an equilibrium object. In
these models, to define a stationary equilibrium, we want the
sequence of distributions of each agent's asset holdings to
converge to a well-defined invariant distribution with finite
first and second moments.  For there to exist a stationary
equilibrium without aggregate uncertainty, the findings of the
present chapter would lead us to anticipate that the equilibrium
interest rate in those models must fall short of $\beta^{-1}$. In
a production economy with physical capital, that result implies
that the marginal product of capital will be less than the one
that would prevail in a complete markets world when the stationary
interest rate would be given by $\beta^{-1}$. In other words, an
incomplete markets economy is characterized by an overaccumulation
of capital that drives the interest rate below $\beta^{-1}$, which
serves to choke off  the desire to accumulate an infinite amount of
assets that agents would have had  if the interest rate had been equal
to $\beta^{-1}$.

Chapters \use{socialinsurance} and \use{socialinsurance2} will consider several models in
which the condition $\beta (1+r)=1$ is maintained. There the assumption
will be that a social planner has access to risk-free loans outside
the economy and seeks to maximize agents' welfare subject to enforcement
and/or information problems.  The environment is once again assumed to be
stationary without aggregate uncertainty, so in the absence of enforcement
and information problems the social planner would just redistribute the
economy's resources in each period without any intertemporal trade
with the outside world.  But when agents are free to leave the economy
with their endowment streams and forever live in autarky, optimality
 prescribes that the  planner amass sufficient outside claims
so that each agent is granted a constant consumption stream in the limit, at
a level that weakly dominates autarky for all realizations of an agent's
endowment.  In the case of asymmetric information,  where the  planner
can  induce agents to tell the truth only by manipulating promises of future
utilities, we obtain a conclusion that is diametrically opposite to the
self-insurance outcome of the present chapter.  Instead of consumption
approaching infinity in the limit, the optimal solution has all agents'
consumption approaching its lower bound.

\appendix{A}{Supermartingale convergence theorem}\label{app:Ach16}%
This appendix states the supermartingale convergence theorem.
Let the elements of the 3-tuple
$(\Omega,  {\cal F}, P)$ denote a sample space, a collection
of events, and a probability measure, respectively.
Let $t \in T$ index time, where $T$ denotes the nonnegative
integers.
Let ${\cal F}_t$ denote an increasing sequence of $\sigma$-fields
of ${\cal F}$ sets.
Suppose that
\medskip
\noindent(i)  $Z_t$ is measurable with respect to ${\cal F}_t$;
\medskip
\noindent(ii)  $E | Z_t   | < + \infty $;
\medskip
\noindent(iii) $ E(Z_t | {\cal F}_s) = Z_s$ almost surely for
all $s < t; s, t \in T$.
\medskip
\noindent Then $\{Z_t, t \in T\}$ is said to be a {\it martingale\/} with
respect to ${\cal F}_t$.
 If (iii) is replaced by
$E(Z_t | {\cal F}_s ) \geq Z_s$ almost surely, then
$\{Z_t\}$ is said to be a {\it submartingale\/}.
 If (iii) is replaced by
$E(Z_t | {\cal F}_s ) \leq Z_s$ almost surely, then
$\{Z_t\}$ is said to be a {\it supermartingale\/}.

\medskip
We have the following important theorem.
\medskip
\noindent{\sc Supermartingale Convergence Theorem:}  Let
$\{Z_t,   {\cal F}_t\}$ be a nonnegative supermartingale.
%%% with $|Z_t|$ bounded.
Then there exists a random variable
$Z$ such that $\lim Z_t = Z$ almost surely and
$E |Z| < + \infty$, i.e., $Z_t$ converges almost surely
to a finite limit.
%%% $E |Z| \leq \lim  {\rm inf}_{t \rightarrow \infty}  E |Z_t| < + \infty$.




%\section{Exercises}
\showchaptIDfalse
\showsectIDfalse
\section{Exercises}
\showchaptIDtrue
\showsectIDtrue
\medskip

\noindent{\it Exercise  \the\chapternum.1} \quad
 A consumer has preferences over sequences of a single
consumption good that are ordered by
$\sum_{t=0}^\infty \beta^t u(c_t)$, where $\beta \in (0,1)$ and
$u(\cdot)$ is strictly increasing, twice continuously differentiable,
strictly concave, and satisfies the Inada condition
$\lim_{c \downarrow 0} u'(c) = +\infty$. The one good is not storable.
   The consumer
has an endowment sequence of the one good $y_t = \lambda^t, t \geq 0$,
where $|\lambda \beta | < 1$.    The consumer can borrow
or lend at a constant and exogenous risk-free net interest rate of
$r$ that satisfies $(1+r)\beta =1$.   The consumer's budget constraint
at time $t$ is
$$ b_t + c_t \leq y_t + (1+r)^{-1} b_{t+1} $$
for all $t \geq 0$, where $b_t$ is the {\it debt\/} (if positive) or
{\it assets\/} (if negative)  due at $t$,
and the consumer has initial debt $b_0=0$.

\medskip
\noindent {\bf Part   I.}  In this part, assume that the consumer is subject to
the ad hoc borrowing constraint $b_t \leq 0 \ \forall t \geq 1$.
Thus, the consumer can lend but not borrow.

\medskip
\noindent {\bf a.}
Assume that $\lambda < 1$.  Compute the household's optimal plan for
$\{c_t,  b_{t+1}\}_{t=0}^\infty$.
\medskip
\noindent {\bf b.}  Assume that $\lambda >1$.  Compute the household's optimal
plan
$\{c_t,  b_{t+1}\}_{t=0}^\infty$.

%{\bf Answer:}  In part a, the borrowing constraint doesn't bind.
%The optimal $c_t$ is constant at ${1 \over 1-\beta \lambda}$.
%In part  b, the borrowing constraint always binds. The optimal plan is
%to consume the endowment each period.

\medskip
\noindent{\bf Part II.}
In this part, assume that the consumer is subject to
the natural borrowing constraint associated with the given endowment
sequence.


\medskip
\noindent{\bf c.} Compute the natural borrowing limits for all
$t\geq 0$.
\medskip
\noindent{\bf d.}
Assume that $\lambda < 1$.  Compute the household's optimal plan for
$\{c_t,  b_{t+1}\}_{t=0}^\infty$.
\medskip
\noindent {\bf e.}  Assume that $\lambda >1$.  Compute the household's optimal
plan
$\{c_t,  b_{t+1}\}_{t=0}^\infty$.

\medskip
%{\bf Answer:  In parts, (d) and (e), the household sets constant
%consumption at ${1 \over 1 - \lambda \beta}$.  In case d, the household
%saves, while in case e, the household perpetually borrows.

\medskip
\noindent{\it Exercise \the\chapternum.2} \quad
The household has preferences over stochastic processes of a single consumption
good that are ordered by
$ E_0 \sum_{t=0}^\infty \beta^t \ln(c_t)$, where $\beta \in (0,1)$
and $E_0$ is the mathematical
expectation with respect to the distribution of the  consumption sequence
of a single nonstorable good, conditional
on the value of the time $0$ endowment.
 The consumer's  endowment is the following
stochastic process: at times $t=0,1$, the household's endowment is drawn
from the distribution
${\rm Prob}(y_t = 2)  = \pi$,
${\rm Prob}(y_t = 1)  =1 -  \pi$, where $\pi \in (0,1)$.  At all times
$t\geq 2$, $y_t =y_{t-1}$.
At each date $t \geq 0$, the household can  lend, but not borrow, at an
exogenous and constant risk-free one-period net interest rate of
$r$ that satisfies $(1+r)\beta =1$.
The consumer's budget constraint at $t$ is
$a_{t+1} = (1+r) (a_t -c_t) + y_{t+1}$, subject to the initial
condition $a_0 = y_0$.
One-period assets carried $(a_t -c_t)$ over into period
$t+1$ from $t$ must be nonnegative, so that the no-borrowing
constraint is $a_t \geq c_t$. At time $t=0$, after $y_0$ is realized,
the consumer devises an optimal consumption plan.
\medskip
\noindent{\bf a.} Draw a tree that portrays the possible paths for
the endowment sequence from date $0$ onward.
\medskip
\noindent{\bf b.}  Assume that $y_0 = 2$.  Compute the consumer's optimal
consumption and lending plan.
\medskip
\noindent{\bf c.}  Assume that $y_0 =1$.  Compute the consumer's optimal
consumption and lending plan.
\medskip
\noindent{\bf d.}  Under the two assumptions on the initial condition
for $y_0$ in the preceding two questions,
 compute the asymptotic distribution of the marginal
utility of consumption $u'(c_t)$ (which in this case is the distribution
of $u'(c_t) = V_t^\prime(a_t)$
for $t \geq 2$),   where $V_t(a)$ is the consumer's value function
at date $t$).

\medskip
\noindent{\bf e.} Discuss whether your   results in part d conform to
Chamberlain and Wilson's application of the supermartingale convergence
theorem.

\medskip
\noindent{\it Exercise  \the\chapternum.3}  \quad
Consider the stochastic version of the savings problem
under the following {\it natural borrowing constraints\/}.   At each
date $t \geq 0$, the consumer can issue risk-free one-period debt
up to an amount   that it is feasible for him to
repay almost surely, given the nonnegativity   constraint on consumption
$c_t \geq 0$ for all $t \geq 0$.

\medskip
\noindent{\bf a.} Verify that the  natural debt limit
is $(1+r)^{-1} b_{t+1} \leq {\overline y_1 \over r} $.
\medskip
\noindent{\bf b.}  Show that the natural debt limit can also be expressed
as $a_{t+1} - y_{t+1} \geq - {(1+r) \overline y_1 \over r}$ for all
$t \geq 0$.
\medskip
\noindent{\bf c.}  Assume that $y_t$ is an i.i.d.\ process with nontrivial
distribution $\{\Pi_s\}$,  in the sense that at least two distinct endowments
occur with positive probabilities.   Prove that optimal consumption
diverges to $+ \infty$ under the natural borrowing limits.
\medskip
\noindent{\bf d.}  For identical realizations of the endowment sequence,
get as far as you can in comparing what would be the sequences
of optimal consumption under the natural and  ad hoc borrowing
constraints.

\medskip
\noindent{\it Exercise  \the\chapternum.4}  \quad {\bf Trade?}
\medskip
\noindent  A  pure endowment  economy consists of two
households with identical preferences  but different endowments.
A household of type $i$ has preferences that are ordered by
$$ E_0 \sum_{t=0}^\infty \beta^t u(c_{it})  , \quad
\beta \in (0,1) \leqno(1)$$
where $c_{it}$ is time $t$ consumption
of a single consumption good,  $u(c_{it}) = u_1 c_{it} -
 .5 u_2 c_{it}^2$, where
$u_1, u_2 >0$,  and $E_0$ denotes the mathematical
expectation conditioned on time $0$ information.  The household of type
$1$
has a stochastic  endowment  $y_{1t}$ of the good
governed by
$$ y_{1t+1} = y_{1t} + \sigma \epsilon_{t+1} \leqno(2) $$
where $\sigma >0$ and $\epsilon_{t+1} $ is an i.i.d.\ process
Gaussian process with mean $0$ and variance $1$.
The household of type 2 has endowment
$$ y_{2t+1} = y_{2t} - \sigma \epsilon_{t+1} \leqno(3) $$
where $\epsilon_{t+1}$ is the {\it same\/} random process as in
(2).
At time $t$, $y_{it}$ is realized before consumption at $t$ is chosen.
Assume that at time $0$, $y_{10}  = y_{20}$ and that $y_{10}$ is substantially
less than the bliss point $u_1/u_2$.  To make the computation easier,
please assume that there is no disposal of resources.

\medskip
\noindent{\bf Part I.} In this part, please
assume that there are complete markets in
history- and date-contingent claims.
\medskip
\noindent{\bf a.} Define   a competitive equilibrium, being
careful to specify all of the objects of which a competitive
equilibrium is composed.
\medskip
\noindent{\bf b.}  Define a Pareto problem for a fictitious planner
who attaches equal weight to the two households.  Find the consumption
allocation that solves the Pareto (or planning) problem.
\medskip
\noindent{\bf c.}  Compute a competitive equilibrium.

\medskip

\noindent{\bf Part II.}  Now assume that markets are incomplete.
There is only one traded asset: a one-period risk-free
bond that both households can either purchase or issue.
The gross rate of  return on the asset between date $t$ and
date $t+1$ is $R_t$.  Household
$i$'s budget constraint at time $t$ is
$$  c_{it} + R_t^{-1} b_{it+1}  = y_{it} + b_{it} \leqno(4) $$
where $b_{it} $ is the  value in terms of time $t$ consumption
goods of household's $i$ holdings of one-period risk-free bonds.
We require that
a consumers's holdings of bonds are subject to the restriction
$$ \lim_{t\rightarrow +\infty}  \beta^t  u'(c_{it}) E b_{it+1} = 0. \leqno(5) $$
Assume that $b_{10} = b_{20} = 0$. An  incomplete markets
competitive equilibrium is a gross interest rate sequence
$\{R_t\}$, sequences of bond holdings $\{b_{it}\}$
 for $i=1,2$,  and
feasible allocations $\{c_{it}\}, i=1,2$ such that given $\{R_t\}$,
household
$i=1,2$ is  maximizing (1) subject to the sequence of budget constraints
(4) and the given initial levels of $b_{10},b_{20}$.


\medskip
\noindent{\bf d.}  A friend of yours recommends
the guess-and-verify method and offers the following guess
about the equilibrium.    He conjectures that there are no gains to trade:
in equilibrium, each household simply consumes its endowment.
Please verify  or falsify this guess.  If you verify it, please
give  formulas for the equilibrium $\{R_t\}$  and the stocks
of bonds held by each household  at each date.


\medskip
\noindent{\it Exercise  \the\chapternum.5}  \quad {\bf Trade??}
\medskip


\noindent A consumer orders consumption streams according to
$$ E_0 \sum_{t=0}^\infty \beta^t {c_t^{1-\gamma} \over 1-\gamma},
\quad \beta \in (0,1) \leqno(1) $$
where $\gamma > 1$ and $E_0$ is the mathematical expectation
conditional on time $0$ information.
The consumer can borrow or lend a one-period risk-free security that
bears a fixed gross rate of return of $R = \beta^{-1}$.   The consumer's
budget constraint at time $t$ is
$$ c_t + R^{-1} b_{t+1} = y_t + b_t \leqno(2) $$
where $b_t$ is the level of the asset  that the consumer brings into
period $t$.    The household is subject to a ``natural'' borrowing limit.
The household's initial asset level is $b_0 = 0$ and his
endowment sequence $y_t$ follows the process
$$ y_{t+1} = y_t \exp(\sigma_y \varepsilon_{t+1} + \mu) \leqno(3) $$
where $\varepsilon_{t+1}$ is an i.i.d.\ Gaussian process with mean
zero and variance $1$, $\mu = .5 \gamma \sigma_y^2$, and $\sigma_y >0 $.
The consumer chooses a process $\{c_t,  b_{t+1} \}_{t=0}^\infty$
to maximize (1) subject to (2), (3), and  the natural borrowing limit.


\medskip
\noindent{\bf a.}  Give a closed-form expression for the consumer's
optimal consumption and asset accumulation plan.
\medskip
\noindent{\it Hint 1:}  If $\log x$ is  ${\cal N}(\mu,
\sigma^2)$, then $E x = \exp(\mu+\sigma^2/2)$.
\medskip
\medskip\noindent
{\it Hint 2:}  You could start by trying
to verify the following guess:
the optimal policy has $b_{t+1} = 0$ for all $t\geq 0$.
\medskip

\noindent{\bf b.} Discuss the solution  that you obtained in part {\bf a}
in terms of Friedman's permanent income hypothesis.

\medskip
\noindent{\bf c.}  Does the household engage in precautionary
savings?
