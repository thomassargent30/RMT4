
\input grafinp3
%\input grafinput8
\input psfig

\showchaptIDtrue
\def\@chaptID{11.}

%\eqnotracetrue

%\hbox{}

\def\toone{{t+1}}
\def\ttwo{{t+2}}
\def\tthree{{t+3}}
\def\Tone{{T+1}}
\def\TTT{{T-1}}
\def\rtr{{\rm tr}}
\footnum=0
\chapter{Economic Growth\label{growth}}

\section{Introduction}

This chapter describes basic nonstochastic models of sustained
economic growth.   We begin by describing a
 benchmark exogenous  growth model where sustained growth
is driven by exogenous growth in labor productivity.
Then we turn our attention to several   endogenous growth
 models where sustained growth of labor   productivity
is somehow {\it chosen\/} by the households in the economy.
We describe several models that differ in whether the
equilibrium market economy matches what a benevolent planner
would choose.   Where the market outcome doesn't match
the planner's outcome, there can be room   for welfare-improving
government interventions. The objective of the chapter is to
shed light on the mechanisms at work in different models.
We try to facilitate comparison by using the same production
function for most of our discussion while changing the meaning
of one of its arguments.

Paul Romer's
\auth{Romer, Paul M.}%
work has been an impetus to the revived interest
in the theory of economic growth. In the spirit of Arrow's (1962)
\auth{Arrow, Kenneth J.}%
model of learning by doing, Romer (1986) presents an endogenous
growth model where the accumulation of capital (or knowledge) is
associated with a positive externality on the available technology.
The aggregate of all agents' holdings of capital
is positively related to the level of technology, which
in turn interacts with individual agents' savings decisions
and thereby determines the economy's growth rate.
Thus, the households in this economy are {\it choosing\/} how
fast the economy is growing, but they do so in an unintentional way.
The competitive equilibrium growth rate falls short of the
socially optimal one.

Another approach to generating endogenous growth is to assume
that all production factors are reproducible. Following Uzawa (1965),
\auth{Uzawa, Hirofumi}%
Lucas (1988)
\auth{Lucas, Robert E., Jr.}%
formulates a model with accumulation of both physical
and human capital. The joint accumulation of all inputs
ensures that growth will not come to a halt
even though each individual factor in the final-good
production function is subject to
diminishing returns. In the absence of externalities,
the growth rate in the competitive equilibrium
coincides in this model with the social optimum.

Romer (1987) constructs a model where agents can choose to engage
in research that produces technological improvements.
Each invention represents a technology for
producing a new type of intermediate input that can be used in
the production of final goods without affecting the marginal product
of existing intermediate inputs. The introduction of new inputs
enables the economy to experience sustained growth even though each
intermediate
input taken separately is subject to diminishing returns.
In a decentralized equilibrium, private agents will expend
resources on research  only if they are granted property rights over
their inventions. Under the assumption of infinitely lived patents,
Romer solves for a monopolistically competitive equilibrium that
exhibits the classic tension between static and dynamic
efficiency. Patents and the associated market power are necessary for
there to be research and new inventions in a decentralized equilibrium,
while the efficient production of existing intermediate inputs would
require marginal-cost pricing, that is, the abolition of granted
patents. The monopolistically competitive equilibrium is characterized
by a smaller
supply of each intermediate input and a lower growth rate than would be
socially optimal.

Finally, we revisit the question of when nonreproducible factors may not
pose an obstacle to growth. Rebelo (1991)
\auth{Rebelo, Sergio}%
shows that even if there are nonreproducible factors in fixed supply in
a neoclassical growth model, sustained growth is possible if there is
a ``core'' of capital goods that is produced without the direct or
indirect use of the nonreproducible factors.  Because of
the ever-increasing relative scarcity of a nonreproducible factor,
Rebelo finds that its price increases over time relative to a
reproducible factor.
Romer (1990) assumes that research requires the input of labor
and not only goods as in his earlier model (1987). Now, if labor is in fixed
supply and workers' innate productivity is constant, it follows immediately
that growth must asymptotically come to an halt.  To make sustained growth
feasible, we can take a cue from our earlier discussion.
One modeling strategy would be to introduce an externality that enhances
researchers' productivity, and an alternative approach would be to assume
that researchers can accumulate human capital. Romer adopts the first type
of assumption, and we find it instructive to focus
on its role in overcoming a barrier to growth that nonreproducible labor
would otherwise pose.

\section{The economy}

The economy has a constant population of a large number of identical
agents who order consumption streams $\{c_t\}_{t=0}^\infty$ according to
$$ \sum_{t=0}^\infty \beta^t u(c_t), \  {\rm with} \ \ \beta\in(0,1)
\;\;{\rm and}\;\; u(c)={c^{1-\sigma} -1 \over 1-\sigma} \;\;\; {\rm for}\;\;
\sigma\in[0, \infty), \EQN grow1 $$
and $\sigma=1$ is taken to be logarithmic utility.\NFootnote{By virtue of
L'H\^opital's rule, the limit of
$(c^{1-\sigma} -1)/(1-\sigma)$ is $\log(c)$ as $\sigma$ goes to $1$.}
Lowercase letters for quantities, such as $c_t$ for consumption, are
used to denote individual variables, and  uppercase letters stand
for aggregate quantities.

For most part of our discussion of economic growth, the production
function takes the form
$$
F(K_t,X_t) \,=\, X_t f(\hat K_t), \quad {\rm where} \;
                               \hat K_t\equiv {K_t \over X_t}. \EQN grow2
$$
That is, the production function $F(K,X)$ exhibits constant returns
to scale in its two arguments, which
via \idx{Euler's theorem} on linearly homogeneous functions implies
$$ F(K,X) = F_1(K,X) K + F_2(K,X) X,           \EQN grow3
$$
where $F_i(K,X)$ is the derivative with respect to the $i$th argument
(and $F_{ii}(K,X)$ will be used to denote the second derivative with
respect to the $i$th argument). The input $K_t$ is physical capital with
a rate of depreciation equal to $\delta$. New capital can be created by
transforming one unit of output into one unit of capital.
Past investments are reversible. It follows that the relative price of capital
in terms of the consumption good must always be equal to $1$. The
second argument $X_t$ captures the contribution of labor.
Its precise meaning will differ among the various setups that we
will examine.

We assume that the production function satisfies standard assumptions
of positive but diminishing marginal products,
$$
F_i(K,X)>0, \quad F_{ii}(K,X)<0, \quad {\rm for}\; i=1,2;
$$
and the \idx{Inada condition}s,
$$\EQNalign{
\lim_{K\to 0}F_1(K,X) \,&=\, \lim_{X\to 0}F_2(K,X) =\infty, \cr
\lim_{K\to \infty}F_1(K,X) \,&=\, \lim_{X\to \infty}F_2(K,X) =0, \cr}
$$
which imply
$$
\lim_{\hat K\to 0}f'(\hat K) = \infty,\quad
\lim_{\hat K\to \infty}f'(\hat K) =0.                       \EQN grow_inada
$$
We will also make use of the mathematical fact that a linearly homogeneous
function $F(K,X)$ has first derivatives $F_i(K,X)$ homogeneous of degree 0;
thus, the first derivatives are only functions of the ratio $\hat K$. In
particular, we have
$$\EQNalign{
&F_1(K,X) = {\partial \, X f\left({K / X}\right) \over \partial \, K }
         = f'(\hat K),                                  \EQN grow4;a   \cr
\noalign{\vskip.3cm}
&F_2(K,X) = {\partial \, X f\left({K / X}\right) \over \partial \, X }
         = f(\hat K) - f'(\hat K) \hat K.                \EQN grow4;b  \cr}
$$

\subsection{Balanced growth path}

We seek additional technological assumptions to
generate market outcomes with steady-state growth of consumption at a
constant rate $1+\mu=c_{t+1}/c_t$. The literature uses the term
``\idx{balanced growth path}'' to denote a situation
where all endogenous variables grow at constant (but possibly different)
rates. Along such a steady-state growth path (and
during any transition toward the steady state), the return to
physical capital must be such that households are willing to hold the
economy's capital stock.

In a competitive equilibrium where firms
rent capital from the agents, the rental payment $r_t$ is equal to the
marginal product of capital,
$$
r_t = F_1(K_t,X_t) = f'(\hat K_t).                             \EQN grow_r
$$
Households
maximize utility given by equation \Ep{grow1} subject to the sequence of
 budget constraints
$$
c_t + k_{t+1} = r_t k_t + (1-\delta) k_t + \chi_t,         \EQN grow_bc
$$
where $\chi_t$ stands for labor-related budget terms. The
first-order condition with respect to $k_{t+1}$ is
$$
u'(c_t) = \beta u'(c_{t+1})\, (r_{t+1} + 1 - \delta).        \EQN grow_foc
$$
After using equations \Ep{grow1} and \Ep{grow_r} in equation \Ep{grow_foc}, we arrive
at the following equilibrium condition:
$$
\left({c_{t+1} \over c_t}\right)^\sigma \,=\, \beta [f'(\hat K_{t+1}) +1 - \delta].
                                                            \EQN growth
$$
We see that a constant consumption growth rate on the left  side is
sustained in an equilibrium
by a constant rate of return on the right  side. It was
also for this reason that we chose the class of utility functions in
equation \Ep{grow1} that exhibits a
constant \idx{intertemporal elasticity of substitution}.
These preferences allow for
balanced growth paths.\NFootnote{To ensure well-defined maximization
problems, a maintained assumption throughout the
chapter is that parameters are such that any derived consumption
growth rate $1+\mu$ yields finite lifetime utility; i.e., the
implicit restriction on parameter values is that
$\beta (1+\mu)^{1-\sigma}<1$. To see that this condition is needed,
substitute the consumption
sequence $\{c_t\}_{t=0}^\infty =\{(1+\mu)^t c_0\}_{t=0}^\infty$
into equation \Ep{grow1}.}

Equation \Ep{growth} makes clear that capital accumulation alone cannot sustain
steady-state consumption growth when the labor input $X_t$ is constant over time,
$X_t=L$. Given the second Inada condition in equations
 \Ep{grow_inada}, the limit
of the right  side of equation \Ep{growth} is $\beta(1-\delta)$
when $\hat K$ approaches
infinity. The steady state with a constant labor input must therefore
be a constant consumption level and a capital-labor ratio $\hat K^{\star}$ given by
$$
f'(\hat K^{\star}) = \beta^{-1} - (1-\delta).                   \EQN growth_0
$$
In chapter \use{dplinear}
 we derived a closed-form solution for the transition dynamics
toward such a steady state in the case of logarithmic utility, a Cobb-Douglas
production function, and $\delta=1$.

\index{growth!exogenous}
\section{Exogenous growth}

As in Solow's (1956) classic article,
the simplest way to ensure steady-state consumption growth is to
postulate exogenous labor-augmenting technological change at the constant
rate $1+\mu\geq 1$,
$$
X_t = A_t L, \qquad {\rm with}\;\; A_{t} = (1+\mu) A_{t-1},
$$
where $L$ is a fixed stock of labor.
Our conjecture is then that both consumption and physical capital will
grow at that same rate $1+\mu$ along a balanced growth path. The same
growth rate of $K_t$ and $A_t$ implies
that the ratio $\hat K$ and therefore the marginal product of capital
remain constant in the steady state. A time-invariant rate of return is in
turn consistent with households
choosing a constant growth rate of consumption,
given the assumption of isoelastic preferences.

Evaluating equation
 \Ep{growth} at a steady state, the optimal ratio $\hat K^{\star}$ is
given by
$$
\left(1+\mu\right)^\sigma \,=\, \beta [f'(\hat K^{\star}) +1 - \delta].
                                                            \EQN growth_1
$$
While the steady-state consumption growth rate is exogenously
given by $1+\mu$, the endogenous steady-state ratio $\hat K^{\star}$
is such that the implied rate of return on capital induces the agents to
choose a consumption growth rate of $1+\mu$. As can be seen,
a higher degree of patience (a larger $\beta$), a higher willingness
intertemporally to substitute (a lower $\sigma$), and a more durable
capital stock (a lower $\delta$) each
yield a higher ratio $\hat K^{\star}$, and
therefore more output (and consumption) at a point in time, but the
growth rate remains fixed at the rate of exogenous labor-augmenting
technological change.
It is straightforward to verify that the competitive equilibrium
outcome is Pareto optimal, since the private return to capital
coincides with the social return.

Physical capital is compensated according to equation \Ep{grow_r}, and labor
is also paid its marginal product in a competitive equilibrium,
$$
w_t = F_2(K_t,X_t) \, {{\rm d}\, X_t \over {\rm d}\,L\hfill}
    = F_2(K_t,X_t) \, A_t.                                     \EQN grow_w
$$
So, by equation \Ep{grow3}, we have
$$
r_t K_t + w_t L = F(K_t,A_t L).
$$
Factor payments are equal to total production, which is the
standard result of a competitive equilibrium with
constant-returns-to-scale technologies. However, it is interesting to
note that if $A_t$ were a separate production factor, there could not
exist a competitive equilibrium, since factor payments based on marginal
products would exceed total production. In other words, the dilemma
would then be that the production function
$F(K_t,A_t L)$ exhibits increasing returns to
scale in the three ``inputs'' $K_t$, $A_t$, and $L$, which is not
compatible with the existence of a competitive equilibrium. This
problem is to be kept in mind as we now turn to one
way to endogenize economic growth.
\index{growth!externality}

\section{Externality from spillovers}

Inspired by Arrow's (1962)
\auth{Arrow, Kenneth J.}%
paper on learning by doing, Romer (1986)
\auth{Romer, Paul M.}%
suggests that economic growth can be endogenized by assuming that
technology grows because of aggregate spillovers coming from
firms' production activities. The
problem alluded to in the previous section, that a competitive
equilibrium fails to exist in the presence of increasing returns
to scale, is avoided by letting technological advancement be
external to firms.\NFootnote{Arrow (1962) focuses on learning from
experience that is assumed to get embodied in capital goods,
while Romer (1986) postulates spillover effects of
firms' investments in knowledge.
In both analyses, the productivity of a given firm is a function
of an aggregate state variable, either the economy's stock of
physical capital or stock of knowledge.}
As an illustration, we assume that firms
face a fixed  labor productivity that is proportional to the current
economy-wide average of physical capital per worker.\NFootnote{This
specific formulation of spillovers is analyzed in a rarely cited
paper by Frankel (1962).}
In particular,
$$
X_t = \bar K_t L, \qquad {\rm where}\;\; \bar K_{t} = {K_t \over L}.
$$
The competitive rental rate of capital is still given by equation
 \Ep{grow_r},
but we now trivially have $\hat K_t=1$, so equilibrium condition \Ep{growth}
becomes
$$
\left({c_{t+1} \over c_t}\right)^\sigma \,=\, \beta [f'(1) +1 - \delta].
                                                            \EQN growth_2
$$
Note first that this economy has no transition
dynamics toward a steady state. Regardless of the initial capital stock,
equation \Ep{growth_2} determines a time-invariant growth rate. To
ensure a positive growth rate, we require the parameter restriction
$\beta [f'(1) +1 - \delta]\geq 1$. A second critical property of the
model is that the economy's growth rate
is now a function of preference and technology parameters.

The competitive equilibrium is no  longer Pareto optimal, since the
private return on capital falls short of the social rate of return,
with the latter return given by
$$
{{\rm d}\, F\left(K_t, {\displaystyle K_t \over \displaystyle L} L \right)
                          \over {\rm d}\, K_t}
= F_1(K_t,K_t) + F_2(K_t,K_t) = f(1),                      \EQN grow_sp
$$
where the last equality follows from equations
 \Ep{grow4}. This higher social rate of
return enters a planner's first-order condition, which then
also implies a higher optimal consumption growth rate,
$$
\left({c_{t+1} \over c_t}\right)^\sigma \,=\, \beta [f(1) +1 - \delta].
                                                            \EQN growth_2sp
$$

Let us reconsider the suboptimality of the decentralized
competitive equilibrium. Since the agents and the planner share
the same objective of maximizing utility, we are left with
exploring differences in their constraints. For a given sequence of
the spillover $\{\bar K_t\}_{t=0}^\infty$,
the production function $F(k_t,\bar K_t l_t)$
exhibits constant returns to scale in $k_t$ and $l_t$. So, once again,
factor payments in a competitive equilibrium will be equal to total
output, and optimal firm size is indeterminate. Therefore, we can
consider a representative agent with one unit of labor endowment who
runs his own production technology, taking the spillover effect as given. His
resource constraint becomes
$$
c_t + k_{t+1} = F(k_t,\bar K_t) + (1-\delta) k_t
              = \bar K_t f\left({k_t\over\bar K_t}\right) + (1-\delta) k_t,
$$
and the private gross rate of return on capital is equal to
$f'(k_t/\bar K_t) +1-\delta$.
After invoking the equilibrium condition $k_t=\bar K_t$, we arrive at
the competitive equilibrium return on capital $f'(1) +1-\delta$ that
appears in equation \Ep{growth_2}. In contrast, the
planner maximizes utility subject to a resource constraint where
the spillover effect is internalized,
$$
C_t + K_{t+1} = F\left(K_t, {K_t \over L} L\right) + (1-\delta) K_t
              = [f(1) + 1 - \delta] K_t.
$$
%\vfill\eject
\index{growth!reproducible factors}
\section{All factors reproducible}
\subsection{One-sector model}
An alternative approach to generating endogenous growth is to assume that
all factors of production are producible. Remaining within a one-sector
economy, we now assume that human capital $X_t$ can be produced in the
same way as physical capital but rates of depreciation might differ. Let
$\delta_X$ and $\delta_K$ be the rates of depreciation of human capital
and physical capital, respectively.

The competitive equilibrium wage is equal to the marginal product of human
capital
$$
w_t = F_2(K_t,X_t).                                            \EQN grow_w2
$$
Households maximize utility subject to budget constraint \Ep{grow_bc} where
the term $\chi_t$ is now given by
$$
\chi_t = w_t x_t + (1-\delta_X) x_t - x_{t+1}.
$$
The first-order condition with respect to human capital becomes
$$
u'(c_t) = \beta u'(c_{t+1})\, (w_{t+1} + 1 - \delta_X).        \EQN grow_foc2
$$
Since both equations
 \Ep{grow_foc} and \Ep{grow_foc2} must hold, the rates of return
on the two assets have to obey
$$
F_1(K_{t+1},X_{t+1}) - \delta_K = F_2(K_{t+1},X_{t+1}) - \delta_X,
$$
and after invoking equations \Ep{grow4},
$$
f(\hat K_{t+1}) - (1+\hat K_{t+1}) f'(\hat K_{t+1}) \;=\; \delta_X -\delta_K,
                                                               \EQN grow6
$$
which uniquely determines a time-invariant competitive equilibrium ratio
$\hat K^{\star}$, as
a function solely of depreciation rates and parameters of the production
function.\NFootnote{The left side of equation
 \Ep{grow6} is strictly increasing, since
the derivative with respect to $\hat K$ is $-(1 +\hat K) f''(\hat K)>0$.
 Thus, there can only be one solution to equation \Ep{grow6} and existence
is guaranteed because the left  side ranges from minus infinity to
plus infinity.  The limit of the left  side when $\hat K$ approaches
zero is
$f(0) -\lim_{\hat K\to 0}f'(\hat K)$, which is equal to minus infinity by
equations \Ep{grow_inada} and the fact that $f(0)=0$.  (Barro and
Sala-i-Martin (1995)
\auth{Barro, Robert J.}\auth{Sala-i-Martin, Xavier}%
show that the Inada conditions and constant returns to scale imply that all
production factors are essential, i.e., $f(0)=0$.) To establish that the
left side of equation \Ep{grow6} approaches plus infinity when $\hat K$ goes
to infinity, we can define the function $g$ as $F(K,X)=K g(\hat X)$
where $\hat X\equiv X/K$ and derive an alternative expression for the
left  side of equation \Ep{grow6}, $(1+\hat X) g'(\hat X) - g(\hat X)$,
for which we take the limit when $\hat X$ goes to zero.}

After solving for $f'(\hat K^{\star})$ from equation \Ep{grow6} and
substituting into equation \Ep{growth}, we arrive at an expression for the
equilibrium growth rate
$$
\left({c_{t+1} \over c_t}\right)^\sigma \,=\,
      \beta \left[{f(\hat K^{\star}) \over 1+\hat K^{\star}}  +1
        - {\delta_X + \hat K^{\star} \delta_K \over 1+ \hat K^{\star}} \right].
                                                            \EQN growth_3
$$
As in the previous model with an externality, the economy here is
void of any transition dynamics toward a steady state.
But this implication  now critically hinges on investments being
reversible so that the initial stocks of physical capital and
human capital are inconsequential. In contrast to the previous model,
the present competitive equilibrium is Pareto optimal because there is
no longer any discrepancy between private and
social rates of return.\NFootnote{It is instructive to compare the present
model with two producible factors, $F(K,X)$, to the previous setup
with one producible factor and an externality, $\tilde F(K,X)$ with
$X=\bar K L$.
Suppose the present technology is such that $\hat K^{\star}=1$ and
$\delta_K=\delta_X$, and the two different setups are equally
productive;
i.e., we assume that $F(K,X)=\tilde F(2K,2X)$, which implies
$f(\hat K)=2\tilde f(\hat K)$. We can then verify that the present
competitive equilibrium growth rate in equation \Ep{growth_3} is the same as the
planner's solution for the previous setup in equation \Ep{growth_2sp}.}

The problem of optimal taxation with commitment (see chapter \use{optax}) is
studied for this model of endogenous growth by Jones, Manuelli, and
Rossi (1993),
 who adopt the assumption of irreversible investments.
\auth{Jones, Larry E.} \auth{Manuelli, Rodolfo} \auth{Rossi, Peter E.}
\subsection{Two-sector model}

Following Uzawa (1965),
\auth{Uzawa, Hirofumi}%
 Lucas (1988)
\auth{Lucas, Robert E., Jr.}%
explores endogenous growth in a two-sector model with all factors
being producible. The resource constraint in the goods sector is
$$
C_t + K_{t+1} = K_t^\alpha (\phi_t X_t)^{1-\alpha} + (1-\delta) K_t,
                                                        \EQN grow_two;a
$$
and the linear technology for accumulating additional \idx{human capital} is
$$
X_{t+1} - X_t = A (1-\phi_t) X_t,                       \EQN grow_two;b
$$
where $\phi_t\in[0,1]$ is the fraction of human capital employed in the
goods sector, and $(1-\phi_t)$ is devoted to human capital accumulation.
(Lucas provides an alternative interpretation that we will discuss
later.)

We seek a balanced growth path where consumption, physical capital, and
human capital grow at constant rates (but not necessarily the same ones) and
the fraction $\phi$ stays constant over time. Let $1+\mu$ be the growth
rate of consumption, and equilibrium condition \Ep{growth} becomes
$$
\left( 1+\mu \right)^\sigma \,=\, \beta \left(\alpha K_{t}^{\alpha-1}
               \left[\phi X_{t}\right]^{1-\alpha} +1 - \delta \right).
                                                            \EQN growth_4
$$
That is, along the balanced growth path, the marginal product of physical
capital must be constant. With the assumed Cobb-Douglas technology, the
marginal product of capital is proportional to the average product, so
that by  dividing equation
\Ep{grow_two;a} through by $K_t$ and applying equation \Ep{growth_4}
we obtain
$$
{C_t \over K_t} + {K_{t+1} \over K_t} =
{(1+\mu)^\sigma \beta^{-1} - (1-\alpha)(1-\delta) \over \alpha}. \EQN grow7
$$
By definition of a balanced growth path, $K_{t+1}/K_t$ is constant, so
equation \Ep{grow7} implies that $C_t/K_t$ is constant; that is, the
capital stock must grow at the same rate as consumption.

Substituting $K_t=(1+\mu)K_{t-1}$ into equation \Ep{growth_4},
$$
\left( 1+\mu \right)^\sigma - \beta (1 - \delta) =
       \beta \alpha \left[(1+\mu)K_{t-1}\right]^{\alpha-1}
               \left[\phi X_{t}\right]^{1-\alpha},
$$
and dividing by the similarly rearranged equation
\Ep{growth_4} for period $t-1$, we
arrive at
$$
1=(1+\mu)^{\alpha-1} \left[{X_t \over X_{t-1}}\right]^{1-\alpha},
$$
which directly implies that human capital must also grow at the rate
$1+\mu$ along a balanced growth path. Moreover, by equation \Ep{grow_two;b},
the growth rate is
$$
1+\mu = 1 + A(1-\phi),                                    \EQN growth_4A
$$
so it remains to determine the steady-state value of $\phi$.

The equilibrium value of $\phi$ has to be such that a unit of human capital
receives the same factor payment in both sectors; that is,
 the marginal products
of human capital must be the same,
$$
p_t A = (1-\alpha) K_t^\alpha \left[\phi X_t\right]^{-\alpha},
$$
where $p_t$ is the relative price of human capital in terms of the composite
consumption/capital good. Since the ratio $K_t/X_t$ is constant along a
balanced growth path, it follows that the price $p_t$ must also be constant
over time. Finally, the remaining equilibrium condition is that the rates of
return on human and physical capital be equal,
$$
{p_{t}(1+A) \over p_{t-1}} = \alpha K_{t}^{\alpha-1}
               \left[\phi X_{t}\right]^{1-\alpha} +1 - \delta ,
$$
and after invoking a constant steady-state price of human capital and
equilibrium condition \Ep{growth_4}, we obtain
$$
1+\mu = [\beta (1+A)]^{1/\sigma}.                            \EQN growth_4B
$$
Thus, the growth rate is positive as long as $\beta (1+A)\geq 1$,
but feasibility requires also that solution \Ep{growth_4B} fall below
$1+A$, which is the maximum growth rate of human capital in equation
\Ep{grow_two;b}.
This parameter restriction, $[\beta (1+A)]^{1/\sigma} < (1+A)$,
also ensures that the growth rate in
equation \Ep{growth_4B} yields finite lifetime utility.

As in the one-sector model, there is no discrepancy between private and
social rates of return, so the competitive equilibrium is Pareto optimal.
Lucas (1988) does allow for an externality (in the spirit of our earlier
section) where the economy-wide average
of human capital per worker enters the production function in the goods
sector, but, as he notes, the externality is not
needed to generate endogenous growth.

Lucas provides an alternative interpretation of the technologies
in equations \Ep{grow_two}. Each worker is assumed to be endowed with one unit of
time. The time spent in the goods sector is denoted $\phi_t$, which is
multiplied by the
agent's human capital $x_t$ to arrive at the efficiency units of
labor supplied.
The remaining time is spent in the education sector
with a constant marginal
productivity of $A x_t$ additional units of human capital acquired.
Even though Lucas's interpretation does introduce a
nonreproducible factor in the form of a time endowment, the multiplicative
specification makes the model identical to an economy with only
two factors that are both reproducible. One section ahead we will study a
setup with a nonreproducible factor that has some nontrivial implications.

\index{growth!research and monopolistic competition}
\index{monopolistic competition}
\section{Research and monopolistic competition}
Building on Dixit and Stiglitz's (1977)
\auth{Dixit, Avinash K.}\auth{Stiglitz, Joseph E.}%
formulation of the demand for
differentiated goods and the extension to differentiated inputs in
production by Ethier (1982),
\auth{Ethier, Wilfred J.}%
Romer (1987)
\auth{Romer, Paul M.}%
studied an economy with an aggregate resource constraint of the following type:
$$
C_t + \int_0^{A_{t+1}} Z_{t+1}(i) \, {\rm d}i + (A_{t+1}-A_t) \kappa =
L^{1-\alpha} \, \int_{0}^{A_t} Z_t (i)^\alpha \, {\rm d}i,     \EQN grow8
$$
where one unit of the intermediate input $Z_{t+1}(i)$ can be produced from
one unit of output at time $t$, and $Z_{t+1}(i)$ is used in production in
the following period $t+1$. The continuous range of inputs at time $t$,
$i\in [0, A_t]$, can be augmented for next period's production function at
the constant marginal cost $\kappa$.

In the allocations that we are about to study, the quantity of an intermediate
input will be the same across all existing types, $Z_t(i)=Z_t$ for
$i\in [0, A_t]$. The resource constraint \Ep{grow8} can then be written as
$$
C_t + A_{t+1} Z_{t+1} + (A_{t+1}-A_t) \kappa =
L^{1-\alpha} \, A_t Z_t^\alpha.                                \EQN grow8a
$$
If $A_t$ were constant over time, say, let $A_t=1$ for all $t$, we would
just have a parametric example of an economy yielding a no-growth steady
state given by equation \Ep{growth_0} with $\delta=1$. Hence, growth can only be
sustained
by allocating resources to a continuous expansion of the range of inputs.
But this approach poses a barrier to the existence of
a competitive equilibrium, since
the production relationship $L^{1-\alpha} \, A_t Z_{t}^\alpha$ exhibits
increasing returns to scale in its three ``inputs.'' Following Judd's (1985a)
\auth{Judd, Kenneth L.}%
treatment of patents in a dynamic setting of Dixit and Stiglitz's (1977)
model of monopolistic competition, Romer (1987) assumes that an inventor
of a new intermediate input obtains an infinitely lived patent on
that design. As the sole supplier of an input, the inventor can recoup the
investment cost $\kappa$ by setting a price of the input above its marginal
cost.
%\vfill\eject

\subsection{Monopolistic competition outcome}
The final-goods sector is still assumed to be characterized by perfect
competition because it exhibits constant returns to scale in the
labor input $L$ and the existing continuous range of intermediate inputs $Z_t(i)$.
Thus, a competitive outcome prescribes that each input is paid its
marginal product,
$$\EQNalign{
w_t &= (1-\alpha) L^{-\alpha} \,
        \int_{0}^{A_t} Z_{t}(i)^\alpha \, {\rm d}i,     \EQN grow_wR \cr
p_t(i) &= \alpha L^{1-\alpha} \, Z_{t}(i)^{\alpha-1},     \EQN grow_pR \cr}
$$
where $p_t(i)$ is the price of intermediate input $i$ at time $t$ in
terms of the final good.


Let $1+R_m$ be the steady-state interest rate along the balanced growth
path that we are seeking. In order to find the equilibrium invention rate
of new inputs, we first compute the profits from producing and selling
an existing input $i$. The profit at time $t$ is equal to
$$
\pi_t(i) = \left[p_t(i) - (1+R_m)\right]Z_t(i),          \EQN grow_profit
$$
where the cost of supplying one unit of the input $i$ is one unit of
the final good acquired in the previous period; that is, the cost is the
intertemporal price $1+R_m$. The first-order condition of maximizing
the profit in
equation \Ep{grow_profit} is the familiar expression that the monopoly price
$p_t(i)$ should be set as a markup above marginal cost, $1+R_m$,
and the markup is inversely related to the absolute value of the
demand elasticity of input $i$,
$|\epsilon_t(i)|$:
$$\EQNalign{
p_t(i) &= {1+R_m \over 1+ \epsilon_t(i)^{-1}},
                                                          \EQN grow_markup \cr
\noalign{\vskip.3cm}
\epsilon_t(i) &= \left[{{\rm \partial}\, p_t(i) \over {\rm \partial}\, Z_t(i)}
                {Z_t(i) \over p_t(i)} \right]^{-1} <0.                  \cr}
$$
The constant marginal cost, $1+R_m$, and
the constant-elasticity demand curve \Ep{grow_pR},
$\epsilon_t(i) = -(1-\alpha)^{-1}$,
yield a time-invariant monopoly price, which, substituted into
demand curve \Ep{grow_pR}, results in a time-invariant equilibrium
quantity of input $i$:
$$\EQNalign{
p_t(i) &= {1+R_m \over \alpha},                        \EQN grow9;a   \cr
\noalign{\vskip.3cm}
Z_t(i) &= \left( {\alpha^2 \over 1+R_m} \right)^{1/(1-\alpha)} L \equiv Z_m .
                                                       \EQN grow9;b   \cr}
$$
By substituting equation \Ep{grow9} into equation
 \Ep{grow_profit}, we obtain an input
producer's steady-state profit flow,
$$
\pi_t(i)= (1-\alpha) \alpha^{1/(1-\alpha)}
\left( {\alpha \over 1+R_m} \right)^{\alpha /(1-\alpha)} L \equiv \Omega_m(R_m) .
                                                            \EQN grow_profit2
$$

In an equilibrium with free entry,
the cost $\kappa$ of inventing a new input must be equal
to the discounted stream of future profits associated with being the sole
supplier of that input,
$$
\sum_{t=1}^\infty (1+R_m)^{-t} \, \Omega_m(R_m) = {\Omega_m(R_m) \over R_m};
                                                            \EQN grow_PVprofit
$$
that is,
$$
R_m \kappa = \Omega_m(R_m).                                 \EQN grow10
$$
The profit function $\Omega_m(R)$ is positive, strictly decreasing in $R$,
and convex, as depicted in Figure \Fg{growth1f}. It follows that there
exists a unique intersection between $\Omega(R)$ and $R \kappa$ that
determines $R_m$. Using the corresponding version of equilibrium condition
\Ep{growth}, the computed interest rate $R_m$ characterizes a balanced growth
path with
$$
\left({c_{t+1} \over c_t}\right)^\sigma = \beta (1+R_m), \EQN grow11
$$
as long as $1+R_m\geq \beta^{-1}$; that is, the technology must be
sufficiently productive relative to the agents' degree of
impatience.\NFootnote{If the computed value $1+R_m$ falls short of $\beta^{-1}$,
the technology
does not present sufficient private incentives for new inventions, so
the range of intermediate inputs stays constant over time, and the
equilibrium interest rate equals $\beta^{-1}$.}
It is straightforward to verify that the range of inputs must grow at
the same rate as consumption in a steady state. After substituting the constant
quantity $Z_m$ into resource constraint \Ep{grow8a} and dividing by $A_t$,
we see that a constant $A_{t+1}/A_t$ implies that $C_t/A_t$ stays constant;
that is, the range of inputs must grow at the same rate as consumption.

%%%%%%%%%%%%%%
%\grafone{grow1.ps,height=2.5in}{{\bf Figure 11.1} Interest rates in
%a version of Romer's (1987) model of research and monopolistic competition.
%The dotted line is the linear relationship $\kappa R$, while the solid and
%dashed curves depict $\Omega_m(R)$ and $\Omega_s(R)$, respectively. The
%intersection between $\kappa R$ and $\Omega_m(R)$ [$\Omega_s(R)$] determines
%the interest rate along a balanced growth path for the laissez-faire
%economy (planner allocation), as long as
%$R\geq \beta^{-1}-1$. The parameterization is $\alpha=0.9$,
%$\kappa=0.3$, and $L=1$.}
%%%%%%%%%%%%%%%%%%%

\midfigure{growth1f}
\centerline{\epsfxsize=3truein\epsffile{grow1.ps}}
\caption{Interest rates in a version of Romer's (1987) model of research and
monopolistic competition.  The dotted line is the linear relationship $\kappa R$,
while the solid and dashed curves depict $\Omega_m(R)$ and $\Omega_s(R)$,
respectively. The intersection between $\kappa R$ and $\Omega_m(R)$ [$\Omega_s(R)$]
determines the interest rate along a balanced growth path for the laissez-faire
economy (planner allocation), as long as $R\geq \beta^{-1}-1$.  The
parameterization is $\alpha=0.9$, $\kappa=0.3$, and $L=1$.}
\infiglist{grow1f}
\endfigure

\medskip

Note that the solution to equation \Ep{grow10} exhibits positive
\idx{scale effects} where a larger labor force $L$ implies a higher
interest rate and therefore a higher growth rate in equation
\Ep{grow11}. The reason is that a larger economy enables input
producers to profit from a larger sales volume in equation \Ep{grow9;b},
which spurs more inventions until the discounted stream of profits
of an input is driven down to the invention cost $\kappa$ by means of
the higher equilibrium interest rate. In other words, it is less
costly for a larger economy to expand its range of inputs because
the cost of an additional input is smaller in per capita terms.

\subsection{Planner solution}
Let $1+R_s$ be the social rate of interest along an optimal balanced
growth path. We analyze the planner problem in two steps.
First, we establish that the socially optimal supply of an input $i$ is
the same across all existing inputs and constant over time.
Second, we derive $1+R_s$ and the implied optimal growth rate of
consumption.

For a given social interest rate $1+R_s$ and a range of inputs $[0,A_t]$,
the planner would choose the quantities of intermediate inputs that
maximize
$$
L^{1-\alpha} \, \int_{0}^{A_t} Z_{t}(i)^\alpha \, {\rm d}i
- (1+R_s) \,  \int_{0}^{A_t} Z_{t}(i) \, {\rm d}i,
$$
with the following first-order condition with respect to $Z_t(i)$:
$$
Z_t(i) = \left( {\alpha \over 1+R_s} \right)^{1/(1-\alpha)} L \equiv Z_s.
                                                       \EQN grow12
$$
Thus, the quantity of an intermediate input is the same across all
inputs and constant over time. Hence, the planner's problem is
simplified to one where utility function \Ep{grow1}
is maximized subject to resource constraint \Ep{grow8a} with quantities
of intermediate inputs given by equation \Ep{grow12}. The first-order
condition with respect to $A_{t+1}$ is then
$$
\left({c_{t+1} \over c_t}\right)^\sigma =
          \beta { L^{1-\alpha} Z_s^\alpha + \kappa
          \over Z_s + \kappa } = \beta (1+R_s),            \EQN grow13
$$
where the last equality merely invokes the definition of $1+R_s$
as the social marginal rate of
intertemporal substitution,  $\beta^{-1}(c_{t+1}/c_t)^\sigma$.
After substituting equation
\Ep{grow 12} into equation \Ep{grow13} and rearranging the
last equality, we obtain
$$
R_s \kappa = (1-\alpha)
\left( {\alpha \over 1+R_s} \right)^{\alpha /(1-\alpha)} L \equiv \Omega_s(R_s) .
                                                            \EQN grow14
$$
The solution to this equation, $1+R_s$, is depicted in Figure \Fg{growth1f}, and
existence is guaranteed in the same way as in the case of $1+R_m$.

We conclude that the social rate of return $1+R_s$ and
therefore the optimal growth rate exceed the laissez-faire
outcome, since the function $\Omega_s(R)$ lies above the function
$\Omega_m(R)$,
$$
\Omega_m(R) = \alpha^{1/(1-\alpha)} \Omega_s(R).           \EQN grow15
$$
We can also show that the laissez-faire supply of an input falls
short of the socially optimal one:
$$
Z_m < Z_s \qquad \Longleftrightarrow \qquad \alpha {1+R_s \over 1+R_m} <
1. \EQN grow16
$$
To establish condition \Ep{grow16}, divide equation \Ep{grow9;b} by equation \Ep{grow12}.
%  Lars: I used the simplified argument. OK?
%To establish that the interest rates satisfy that last inequality,
%divide \Ep{grow10} by \Ep{grow14} and utilize \Ep{grow15},
%$$
%{R_m \over R_s} =
%\alpha^{1/(1-\alpha)} { \Omega_s(R_m) \over \Omega_s(R_s)} =
%\alpha^{1/(1-\alpha)} \left({ 1+R_s \over 1+R_m}\right)^{\alpha/(1-\alpha)}.
%$$
%Since $R_m<R_s$, it follows that
%$$
%{1+R_m \over 1+R_s} >
%\alpha^{1/(1-\alpha)} \left[{ 1+R_s \over 1+R_m}\right]^{\alpha/(1-\alpha)},
%$$
%which after rearrangement yields \Ep{grow16}.
 Thus, the laissez-faire equilibrium
is characterized by a smaller supply of each intermediate input
and a lower growth rate than would be socially optimal. These
inefficiencies reflect the fact that suppliers of intermediate inputs do not
internalize the full contribution of their inventions, and so their
monopolistic pricing results in less than socially efficient quantities
of inputs.
\index{growth!nonreproducible factors}


\section{Growth in spite of nonreproducible factors}

\subsection{``Core'' of capital goods produced without nonreproducible inputs}
It is not necessary that all factors be producible in order to experience
sustained growth through factor accumulation in the neoclassical framework.
Instead, Rebelo (1991)
\auth{Rebelo, Sergio}%
shows that the critical requirement for perpetual
growth is the existence of a ``core'' of capital goods that is produced
with constant returns technologies and without the direct or indirect use
of nonreproducible factors. Here we will study the simplest version of
his model with a single capital good that is produced without any input
of the economy's constant labor endowment. Jones and Manuelli (1990)
\auth{Jones, Larry E.}\auth{Manuelli, Rodolfo}%
provide a general discussion of convex models of economic growth
and highlight the crucial feature that the rate of return to
accumulated capital must remain bounded above the inverse of the
subjective discount factor in spite of any nonreproducible factors
in production.

Rebelo (1991) analyzes the competitive equilibrium for the following technology:
$$\EQNalign{
C_t     &= L^{1-\alpha} \, \left(\phi_t K_t\right)^\alpha ,  \EQN grow17;a \cr
I_t     &= A (1-\phi_t) K_t,                                   \EQN grow17;b \cr
K_{t+1} &= (1-\delta) K_t + I_t,                               \EQN grow17;c \cr}
$$
where $\phi_t\in[0,1]$ is the fraction of capital employed in the consumption
goods sector and $(1-\phi_t)$ is employed in the linear technology producing
investment goods $I_t$. In a competitive equilibrium, the rental price of
capital $r_t$ (in terms of consumption goods)
is equal to the marginal product of capital, which then has to be the same
across the two sectors (as long as they both are operating):
$$
r_t = \alpha L^{1-\alpha} \, \left(\phi_t K_t\right)^{\alpha-1} = p_t A, \EQN grow18
$$
where $p_t$ is the relative price of capital in terms of consumption goods.

Along a steady-state growth path with a constant $\phi$, we can compute the
growth rate of capital by substituting equation
 \Ep{grow17;b} into equation \Ep{grow17;c} and
dividing by $K_t$,
$$
{K_{t+1} \over K_t} = (1-\delta) + A(1-\phi) \equiv 1+\rho(\phi).  \EQN grow19
$$
Given the growth rate of capital, $1+\rho(\phi)$, it is straightforward to
compute other rates of change:
$$\EQNalign{
{p_{t+1} \over p_t} &= [1+\rho(\phi)]^{\alpha-1},               \EQN grow20;a \cr
\noalign{\vskip.3cm}
{C_{t+1} \over C_t} &= {p_{t+1}I_{t+1} \over p_t I_t}
= {p_{t+1}K_{t+1} \over p_t K_t} = [1+\rho(\phi)]^\alpha.     \EQN grow20;b \cr}
$$
Since the values of investment goods and the capital stock in terms of
consumption goods grow at the same rate as consumption,
$[1+\rho(\phi)]^\alpha$, this common rate is also
the steady-state growth rate of the economy's net income, measured as
$C_t + p_t I_t - \delta p_t K_t$.

Agents maximize utility given by condition \Ep{grow1} subject to budget constraint
\Ep{grow_bc} modified to incorporate the relative price $p_t$,
$$
c_t + p_t k_{t+1} = r_t k_t + (1-\delta) p_t k_t + \chi_t.         \EQN grow_21
$$
The first-order condition with respect to capital is
$$
\left( {c_{t+1} \over c_t} \right)^\sigma
      = \beta {(1-\delta)p_{t+1} + r_{t+1} \over p_t}.             \EQN grow22
$$
After substituting $r_{t+1}=p_{t+1}A$ from equation \Ep{grow18}
and steady-state rates of change from equation \Ep{grow20} into equation
\Ep{grow22}, we arrive at the following equilibrium condition:
$$
\left[1+\rho(\phi)\right]^{1-\alpha (1-\sigma)} = \beta (1-\delta +A). \EQN growth_5
$$
Thus, the growth rate of capital and therefore the growth rate of consumption
are positive as long as
$$
\beta (1-\delta +A)\geq 1.                                          \EQN grow23;a
$$
Moreover, the maintained
assumption of this chapter that parameters are such that derived growth rates
yield finite lifetime utility, $\beta (c_{t+1}/c_t)^{1-\sigma} <1$, imposes
here the parameter restriction
$\beta[\beta(1-\delta +A)]^{\alpha (1-\sigma)/[1-\alpha (1-\sigma)]}<1$, which
can be simplified to read
$$
\beta (1-\delta +A)^{\alpha (1-\sigma)}<1.                        \EQN grow23;b
$$
Given that conditions \Ep{grow23} are satisfied, there is a unique equilibrium
value of $\phi$ because the left side of equation \Ep{growth_5} is
monotonically decreasing in $\phi\in[0,1]$ and it is strictly greater (smaller)
than the right side for $\phi=0$ ($\phi=1$). The outcome is socially
efficient because private and social rates of return are the same as in the
previous models with all factors reproducible.
\index{growth!research and monopolistic competition}
\index{growth!externality}
\subsection{Research labor enjoying an externality}
Romer's (1987)
\auth{Romer, Paul M.}%
 model includes labor as a fixed nonreproducible factor,
but similar to the last section, an important assumption is that this
nonreproducible factor is not used in the production of inventions that
expand the input variety (which constitutes a kind of reproducible
capital in that model).  In his sequel, Romer (1990) assumes that
the input variety $A_t$ is expanded through the effort of researchers
rather than the resource cost $\kappa$ in terms of final goods.  Suppose
that we specify this new invention technology as
$$
A_{t+1} - A_t = \eta (1-\phi_t) L,
$$
where $(1-\phi_t)$ is the fraction of the labor force employed in the research
sector (and $\phi_t$ is working in the final-goods sector). After dividing by
$A_t$, it becomes clear that this formulation cannot support sustained growth,
since new inventions bounded from above by $\eta L$ must become a smaller
fraction of any growing range $A_t$. Romer solves this problem by assuming
that researchers' productivity grows with the range of inputs (i.e., an
externality as discussed previously):
$$
A_{t+1} - A_t = \eta A_t (1-\phi_t) L,
$$
so the growth rate of $A_t$ is
$$
{A_{t+1} \over A_t} = 1 + \eta (1-\phi_t) L.                   \EQN grow24
$$

When seeking a balanced growth path with a constant $\phi$, we can use
the earlier derivations, since the optimization problem of
monopolistic input producers is the same as before.
After replacing $L$ in
equations
\Ep{grow9;b} and \Ep{grow_profit2} by $\phi L$, the steady-state supply
of an input and the profit flow of an input producer are
$$\EQNalign{
Z_m &= \left( {\alpha^2 \over 1+R_m} \right)^{1/(1-\alpha)} \phi L, \EQN grow25;a  \cr
\noalign{\vskip.3cm}
\Omega_m(R_m) &= (1-\alpha) \alpha^{1/(1-\alpha)}
\left( {\alpha \over 1+R_m} \right)^{\alpha /(1-\alpha)} \phi L.    \EQN grow25;b  \cr}
$$

In an equilibrium, agents must be indifferent between earning the wage in the
final-goods sector equal to the marginal product of labor and being a researcher
who expands the range of inputs by $\eta A_t$ and receives the associated
discounted stream of profits in equation \Ep{grow_PVprofit}:
$$
(1-\alpha) \left(\phi L\right)^{-\alpha} A_t Z_m^\alpha
           = \eta A_t {\Omega_m(R_m) \over R_m}.
$$
The substitution of equation \Ep{grow25} into this expression yields
$$
\phi = {R_m \over \alpha \eta L},                                    \EQN grow26
$$
which, used in equation
\Ep{grow24}, determines the growth rate of the input range,
$$
{A_{t+1} \over A_t} = 1 + \eta L - {R_m \over \alpha}.                 \EQN grow27
$$
Thus, the maximum feasible growth rate in equation \Ep{grow24}, that is,
 $1+\eta L$ with $\phi=0$,
requires an interest rate $R_m=0$,
 while the growth vanishes as $R_m$ approaches
$\alpha \eta L$.

As previously, we can show that both consumption and the input range must
grow at the same rate along a balanced growth path. It then remains to determine
which consumption growth rate given by equation
 \Ep{grow27}, is supported by
Euler equation \Ep{grow11}:
$$
1 + \eta L - {R_m \over \alpha} = \left[\beta (1+R_m)\right]^{1/\sigma}. \EQN grow28
$$
The left side of equation \Ep{grow28} is monotonically decreasing
in $R_m$, and the right side is increasing.
It is also trivially true that the left  side is strictly greater than the
right  side for $R_m=0$. Thus, a unique solution exists as long as the
technology is sufficiently productive, in the sense that
$\beta(1+\alpha\eta L)>1$. This parameter restriction
ensures that the left side of equation \Ep{grow28} is
strictly less than the right side
at the interest rate $R_m=\alpha \eta L$ corresponding to a situation with
zero growth, since no labor is allocated to the research sector, $\phi=1$.

Equation \Ep{grow28} shows that this alternative model of research shares the
scale implications described earlier; that is,
 a larger economy in terms of $L$ has
a higher equilibrium interest rate and therefore a higher growth rate. It
can also be shown that the laissez-faire outcome continues to produce a smaller
quantity of each input and to yield a lower growth rate than what is socially optimal.
An additional source of underinvestment is now that agents who invent new inputs
do not take into account that their inventions will increase the productivity
of all future researchers.




\section{Concluding remarks}

This chapter has focused on the mechanical workings of endogenous
growth models, with only limited reference to the motivation behind
assumptions.
For example, we have examined how externalities
might enter models to overcome the onset of diminishing returns from
nonreproducible factors without referring too much to the authors'
interpretation of those externalities. The formalism of
models is of course silent on why the assumptions are made, but
the conceptual ideas behind the models contain valuable insights.
In the last setup, Paul Romer
argues that input designs represent excludable factors
in the monopolists' production of inputs but the input variety $A$ is
also an aggregate stock of knowledge that enters as
a nonexcludable factor in the production
of new inventions. That is, the patent holder of an input type
has the sole right to produce and sell that particular input, but she
cannot stop inventors from studying the input design and learning
knowledge that helps to invent new inputs.
This multiple use of an input design hints at the nonrival nature
of ideas and technology (i.e., a nonrival object has the property that
its use by one person in no way limits its use by another).
Romer (1990, p. S75) emphasizes this fundamental nature of technology
and its implication; ``If a nonrival good has productive value, then
output cannot be a constant-returns-to-scale function of all its
inputs taken together. The standard replication argument used to
justify homogeneity of degree one does not apply because it is not
necessary to replicate nonrival inputs.'' Thus, an endogenous
growth model that is driven by technological change must be
one where the advancement enters the economy as an externality or the
assumption of perfect competition must be abandoned. Besides
technological change, an alternative approach in the endogenous
growth literature is to assume that all production factors are
reproducible, or that there is a ``core'' of capital goods produced without
the direct or indirect use of nonreproducible factors.

As we have seen,
much of the effort in the endogenous growth literature is geared toward finding
the proper technology specification. Even though growth is an endogenous outcome
in these models, its manifestation  ultimately hinges on technology assumptions.
In the case of the last setup, as pointed out by Romer (1990, p. S84),
\auth{Romer, Paul M.}%
``Linearity in $A$ is what makes unbounded growth possible, and in this sense,
unbounded growth is more like an assumption than a result of the model.''
It follows that various implications of the analyses stand and fall with the
assumptions on technology. For example, the preceding
model of research and monopolistic
competition implies that the laissez-faire economy grows at a slower rate
than the social optimum, but Benassy (1998)
\auth{Benassy, Jean-Pascal}%
 shows how this result can be overturned
if the production function for final goods on the right side
of equation
\Ep{grow8} is multiplied by the input range raised to some power $\nu$,
$A_t^\nu$. It then becomes possible that the laissez-faire growth rate exceeds
the socially optimal rate because the new production function disentangles input
producers' market power, determined by the parameter $\alpha$, and the economy's
returns to specialization, which is here also related to the parameter $\nu$.

\auth{Segerstrom, Paul S.} \auth{Anant, T.C.A.} \auth{Dinopoulos, Elias}
\auth{Grossman, Gene M.} \auth{Helpman, Elhanan} \auth{Aghion, Philippe}
\auth{Howitt, Peter}
Segerstrom, Anant, and Dinopoulos (1990), Grossman and Helpman (1991), and
Aghion and Howitt (1992)
provide early attempts to explore endogenous growth arising from technologies
that allow for product improvements and therefore product obsolescence.
These models open the possibility that the laissez-faire growth rate is
excessive because of a {\it business-stealing\/} effect, where agents fail to
internalize the fact that their inventions exert a negative effect on incumbent
producers. As in the models of research by Romer (1987, 1990)
\auth{Romer, Paul M.}%
 covered in this chapter, these other technologies exhibit scale effects, so
that increases in the resources devoted to research imply faster economic growth.
Charles  Jones (1995), Young (1998), and Segerstrom (1998) criticize this feature
and propose assumptions on technology that do not give rise to scale effects.
\auth{Jones, Charles I.} \auth{Young, Alwyn} \auth{Segerstrom, Paul S.}

%\section{Exercises}
\showchaptIDfalse
\showsectIDfalse
\section{Exercises}
\showchaptIDtrue
\showsectIDtrue
\medskip
\noindent{\it Exercise \the\chapternum.1} \quad {\bf Government spending and investment},
donated by Rodolfo Manuelli
\medskip\noindent
Consider the following economy. There is a representative agent
who has preferences given by
$$ \sum_{t=0}^\infty \beta^t u(c_t), $$
where the function $u$ is differentiable, increasing, and strictly
concave.  The technology in this economy is given by
$$ \eqalign{& c_t + x_t + g_t \leq f(k_t, g_t),    \cr
   & k_{t+1} \leq (1-\delta)k_t + x_t, \cr
   & (c_t, k_{t+1}, x_t) \geq (0, 0, 0), \cr}  $$
and the initial condition $k_0 > 0$, given. Here
$k_t$ and $g_t$ are capital per worker and government spending per worker.
  The function $f$ is
assumed to be strictly concave, increasing in each argument,
twice differentiable,
and such that the partial derivative with respect to both arguments converge
to zero as the quantity of them grows without bound.
\medskip

\noindent{\bf a.} Describe a set of equations that
characterize an interior solution to the planner's
problem when the planner can choose the
sequence of government spending.
\medskip

\noindent{\bf b.} Describe the steady state for the ``general'' specification
of this economy.  If necessary, make assumptions to guarantee
that such a steady state exists.
\medskip

\noindent{\bf c.} Go as far as you can describing how the steady-state
levels of capital per worker and government spending per worker change
as a function of the discount factor.
\medskip

\noindent{\bf d.} Assume that the technology level can vary.  More precisely,
assume that the production function is given by
 $f(k,g,z) = zk^\alpha g^\eta$, where $0<\alpha<1$, $0<\eta<1$, and
$\alpha+\eta<1$.  Go as far as you can describing how the investment/GDP
 ratio and the government spending/GDP ratio vary with the technology level
$z$ at the steady state.

\medskip
\noindent{\it Exercise \the\chapternum.2}  \quad {\bf Productivity and employment}, donated
by Rodolfo Manuelli
\medskip\noindent
Consider a basic growth  economy with one modification.
Instead of assuming that the labor supply is fixed at $1$, we
include leisure in the utility function. %% as well as in the production function.
To simplify, we consider the total endowment of time to be $1$.
  With this modification, preferences and technology are given by
\medskip
$$
 \sum_{t=0}^\infty \beta^t u(c_t, 1-n_t) , $$
$$c_t + x_t + g_t \leq zf(k_t, n_t),  $$
$$k_{t+1} \leq (1-\delta)k_t + x_t. $$
\medskip\noindent
In this setting, $n_t$ is the number of hours worked by
the representative household at time $t$.  The rest of
the time, $1-n_t$, is consumed as leisure.  The
functions $u$ and $f$ are assumed to be strictly increasing in
each argument, concave, and twice differentiable.  In addition,
$f$ is such that the marginal product of capital converges to
zero as the capital stock goes to infinity for any given value of labor, $n$.
\medskip

\noindent{\bf a.} Describe the steady state of this economy.  If necessary,
make additional assumptions to guarantee that it exists and is unique.
If you make additional assumptions, go as far as you can giving an
economic interpretation of them.
\medskip

\noindent{\bf b.} Assume that $f(k,n) = k^\alpha n^{1-\alpha}$
and {\ninepoint $u(c,1-n) = [c^\mu (1-n)^{1-\mu}]^{1-\sigma}/(1-\sigma)$.}
%endninepoint
What is the effect of changes in the technology (say increases in $z$)
on employment and output per capita?
\medskip

\noindent{\bf c.} Consider next an increase in $g$.  Are there
conditions under which an increase in $g$ will result in an
increase in the steady-state $k/n$ ratio?  How about an increase
in the steady-state level of output per capita?  Go as far as you
can giving an economic interpretation of these conditions.  (Try to
do this for general $f(k,n)$ functions with the appropriate
convexity assumptions, but if this proves too hard, use the
Cobb-Douglas specification.)

\medskip
\noindent{\it Exercise  \the\chapternum.3} \quad {\bf Vintage capital and cycles},
 donated by Rodolfo Manuelli
\medskip\noindent
Consider a standard one-sector optimal growth model with only one
difference:
If $k_{t+1}$ new units of capital are built at time $t$, these units remain
fully productive (i.e., they do not depreciate) until time $t+2$, at which
point they disappear.  Thus, the technology is given by
$$ c_t + k_{t+1} \leq zf(k_t + k_{t-1}). $$

\noindent{\bf a.} Formulate the optimal growth problem.
\medskip

\noindent{\bf b.} Show that, under standard conditions, a steady
state exists and is unique.
\medskip

\noindent{\bf c.} A researcher claims that with the unusual depreciation
pattern, it is possible that the economy displays cycles.  By
this he means that, instead of a steady state, the economy will
converge to a period two sequence like
($c^o, c^e, c^o, c^e, \ldots$) and
 ($k^o, k^e, k^o, k^e, \ldots$), where $c^o$ ($k^o$)
 indicates consumption (investment) in odd periods, and $c^e$ ($k^e$)
 indicates consumption (investment) in even periods.
Go as far as you can determining whether this can happen.
  If it is possible, try to provide an example.

\medskip
\noindent{\it Exercise \the\chapternum.4}\quad {\bf Excess capacity}, donated by
Rodolfo Manuelli

\medskip\noindent
In the standard growth model,
 there is no room for varying the rate of utilization of capital.
  In this problem, you will explore how the nature of the solution is changed
 when variable rates of capital utilization are allowed.
\medskip
\noindent
As in the standard model, there is a representative agent with
preferences given by
$$\sum_{t=0}^\infty \beta^t u(c_t),   \quad 0 < \beta < 1.$$
It is assumed that $u$ is strictly increasing, concave, and twice
differentiable.  Output depends on the actual number of machines used at
 time $t$, $\kappa_t$.  Thus, the aggregate resource constraint is
$$c_t + x_t \leq zf(\kappa_t),$$
where the function $f$ is strictly increasing,
 concave, and twice differentiable.  In addition, $f$ is
such that the marginal product of capital converges to zero
 as the stock goes to infinity.
Capital that is not used does not depreciate.  Thus, capital
accumulation satisfies
$$k_{t+1} \leq (1-\delta)\kappa_t + (k_t - \kappa_t) + x_t,$$
where we require that the number of machines used, $\kappa_t$,
is no greater than the number of machines available, $k_t$, or
$k_t \geq \kappa_t$.
This specification captures the idea that if some machines are
not used, $k_t - \kappa_t>0$, %% in our notation,
they do not depreciate.
\medskip

\noindent{\bf a.} Describe the planner's problem and analyze, as
thoroughly as you can, the first-order conditions.  Discuss your results.
\medskip

\noindent{\bf b.} Describe the steady state of this economy.  If necessary,
make additional assumptions to guarantee that it exists and is unique.  If
you make additional assumptions, go as far as you can giving an economic
interpretation of them.
\medskip

\noindent{\bf c.} What is the optimal level of capacity utilization in
this economy in the steady state?
\medskip

\noindent{\bf d.} Is this model consistent with the view that cross-country
 differences in output per capita are associated with
differences in capacity utilization?


\medskip
\noindent{\it Exercise \the\chapternum.5 } \quad  {\bf Heterogeneity and growth}, donated
by Rodolfo Manuelli
\medskip
\noindent
Consider an economy populated by a large number of households indexed by $i$.
The utility function of household $i$ is
$$ \sum_{t=0}^\infty \beta^t u_i(c_{it}),  $$
where $0<\beta<1$, and $u_i$ is differentiable, increasing and strictly concave.
Note that although we allow the utility function to be ``household specific,''
all households share the same discount factor.  All households are endowed with
one unit of labor that is supplied inelastically.

\medskip
Assume that in this economy capital markets are perfect and that households start
with initial capital given by $k_{i0}>0$.  Let total capital in the economy at
time $t$ be denoted $k_t$ and assume that total labor is normalized to $1$.
\medskip
Assume that there is a large number of firms that produce output using capital
and labor.  Each firm has a production function given by $F(k,n)$ which is
increasing, differentiable, concave, and homogeneous of degree $1$.  Firms
maximize the present discounted value of profits.  Assume that initial
ownership of firms is uniformly distributed across households.
\medskip

\noindent{\bf a.} Define a competitive equilibrium.
\medskip

\noindent{\bf b.} Discuss (i) and (ii) and justify your answer.  Be as formal as you can.
\medskip
\item{(i)} Economist A argues that the steady state of this economy
is unique and independent of the $u_i$ functions, while B says that without
knowledge of the $u_i$ functions it is impossible to calculate the steady-state
interest rate.
\medskip
\item{(ii)}  Economist A says that if $k_0$ is the steady-state aggregate
stock of capital, then the pattern of ``consumption inequality'' will mirror
exactly the pattern of ``initial capital inequality'' (i.e., $k_{i0}$), even
though capital markets are perfect.  Economist B argues that for all $k_0$,
in the long run, per capita consumption will be the same for all households.
\medskip


\medskip

\noindent{\bf c.} Assume that the economy is at the steady state.  Describe the
effects of the following three policies.
\medskip

\item{(i)}   At time zero, capital is redistributed across households
(i.e., some people must surrender capital %get taxed
and others get their capital).
\medskip

\item{(ii)}  Half of the households are required to pay a lump-sum tax.
The proceeds of the tax are used to finance a transfer program to the other
half of the population.
\medskip

\item{(iii)} Two-thirds of the households are required to pay a lump-sum tax.
The proceeds of the tax are used to finance the purchase of a public good,
say $g$, which does not enter in either preferences or technology.
%\vfil\eject
\medskip
\noindent{\it Exercise \the\chapternum.6} \quad {\bf Taxes and growth},
 donated by Rodolfo Manuelli

\medskip
\noindent
Consider a simple two-planner economy.  The first planner picks ``tax rates,''
$\tau_t$,
%(we denote the tax rate by $\tau$)
and makes transfers to the representative
agent, $v_t$.
% (which we denote $v_t$).
The second planner takes the tax rates and
the transfers as given.  That is, even though we know the connection between
tax rates and transfers, the second planner does not, he or she takes the sequence
of tax rates and transfers as given and beyond his or her control when solving for
the optimal allocation.  Thus, the problem faced by the second planner (the only
one we will analyze for now) is
$$ \max \sum_{t=0}^\infty \beta^t u(c_t)  $$
subject to
$$\eqalign{& c_t + x_t + g_t - v_t \leq (1-\tau_t) f(k_t), \cr
& k_{t+1} \leq (1-\delta)k_t + x_t,  \cr
& (c_t, k_{t+1}, x_t) \geq (0,0,0),  \cr}  $$
and the initial condition $k_0>0$, given.
The functions $u$ and $f$ are assumed to be strictly increasing,
concave, and twice differentiable.  In addition,
$f$ is such that the marginal product of capital converges to
zero as the capital stock goes to infinity.

%The functions $u$ and $f$ have the same properties as in the general problem
%described in secction 3 (problem (P.1)).
\medskip

\noindent{\bf a.} Assume that $0<\tau_t = \tau<1$, that is, the tax rate is constant.
Assume that $v_t = \tau f(k_t)$ (remember that we know this,
but the planner takes
$v_t$ as given at the time he or she maximizes).  Show that there exists a steady
state, and that for any initial condition $k_0>0$ the economy converges to the
steady state.
\medskip

\noindent{\bf b.} Assume now that the economy has reached
the steady state you analyzed
in a.  The first planner decides to change the tax
rate to $0<\tau' < \tau$.
(Of course, the first planner and we know that this
 will result in a change in
$v_t$; however, the second planner, the one that maximizes,
acts as if $v_t$
is a given sequence that
is independent of his or her decisions.)  Describe the new
steady state as well as the dynamic path followed by the economy to reach this new
steady state.  Be as precise as you can about consumption, investment and output.
\medskip

\noindent{\bf c.} Consider now a competitive economy in which households, but not firms,
pay income tax at rate $\tau_t$ on both labor and capital
income.  In addition, each household receives a transfer, $v_t$, that it takes
to be given and independent of its own actions.  Let the aggregate per capita
capital stock be $k_t$.  Then, balanced budget on the part of the government
implies $v_t = \tau_t(r_t k_t + w_t, n_t)$, where $r_t$ and $w_t$ are the
rental prices of capital and labor, respectively.  Assume that the production
function is $F(k,n)$, with $F$ homogeneous of degree $1$, concave, and ``nice.''
Go as far as you can describing the impact of the change described in b on
the equilibrium interest rate.
