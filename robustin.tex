%% robustin.tex
%% borrowed from inputs2 to reform table of contents
%% in robustness book
%%




%%%%  \input Txsf1gs
%%%%%% (3)  table of contents



%%++ 9 pt fonts:
% family sc: small capitals [9]
\def\sc{\fam\scfam\tensc}
\newfam\scfam
\font\tensc=cmcsc10
\font\ninesc=cmcsc9
\font\eightsc=cmcsc8
\font\sevensc=cmcsc8 at 7pt
\font\sixsc=cmcsc8 at 6pt



\font\twelvesc=cmcsc10 at 12pt
\font\elevensc=cmcsc10 at 10.95pt



%%++ 8 pt fonts:
\def\eightfonts{%
   \global\font\eightrm=cmr8
   \global\font\eighti=cmmi8
   \global\font\eightsy=cmsy8
   \global\font\eightex=cmex10
   \global\font\eightbf=cmbx8
   \global\font\eightsl=cmsl8
   \global\font\eighttt=cmtt8
   \global\font\eightit=cmti8
   \skewchar\eighti='177
   \skewchar\eightsy='60
   \hyphenchar\eighttt=-1
   \moreeightfonts                               % any custom fonts
   \gdef\eightfonts{\relax}}

\def\moreeightfonts{\relax}                      % User customization hook


%       Switch to 8 point type
\message{8pt,}
\def\eightpoint{\eightfonts               % load  8pt fonts if needed
   \def\rm{\fam0\eightrm}%
   \textfont0=\eightrm\scriptfont0=\sevenrm\scriptscriptfont0=\fiverm
   \textfont1=\eighti\scriptfont1=\seveni\scriptscriptfont1=\fivei
   \textfont2=\eightsy\scriptfont2=\sevensy\scriptscriptfont2=\fivesy
   \textfont3=\eightex\scriptfont3=\eightex\scriptscriptfont3=\eightex
   \textfont\itfam=\eightit\def\it{\fam\itfam\eightit}%
   \textfont\slfam=\eightsl\def\sl{\fam\slfam\eightsl}%
   \textfont\ttfam=\eighttt\def\tt{\fam\ttfam\eighttt}%
   \textfont\bffam=\eightbf
   \scriptfont\bffam=\sevenbf
   \scriptscriptfont\bffam=\fivebf\def\bf{\fam\bffam\eightbf}%
   \def\mib{\relax}%
   \def\scr{\relax}%
   \tt\ttglue=.5emplus.25emminus.15em
   \normalbaselineskip=11pt
   \setbox\strutbox=\hbox{\vrule height 8pt depth 3pt width 0pt}%
   \normalbaselines\rm\singlespaced}%

\font\chapfont=cmbx10  at 14pt
\font\chapsfont=cmti10 at 12pt
\font\secfont=cmbx10 at 12pt
\font\subsecfont=cmti10 at 12pt
\font\headlinefont=cmti10 at 10pt % font for headlines
\font\foliofont=cmti10 at 9pt    % font for page numbers
\font\titlesmall=cmti10 at 20pt   % for title and half-title pages
\font\titlebig=cmti10 at 24pt     %
%\font\partfont=cmti10 at 16pt     % for parts
\edef\Tbf{\chapfont}    % those are the corresponding macros in TeXsis
\edef\tbf{\secfont}
\edef\Tbfs{\chapsfont}

%%%%% modify footnote size  from txsmacs.tex  June 4, 2004
%
\def\FootFont{\eightpoint\rm}
%\newcount\footnum \footnum=0
%\let\footnotemark=\empty                % for # marks and such
%
%\def\NFootnote{% consecutively numbered footnotes
%  \advance\footnum by 1
%  \xdef\lab@l{\the\footnum}%
%  \Footnote{\footnotemark\the\footnum}}
%
%%%%% end of NFootnote modification




\TOCmargin=1cm


% kludge to modify table of contents entries


\def\fakeaddTOC#1#2#3{\relax}
\def\realaddTOC#1#2#3{\relax}

\catcode`\@=11                                  % let's use @ as a letter here


% redefine TOCitem to change the entries written to the Table of Contents
%     The chapter entries are pretty much the same, except that small font
%     starts after them.  The section and subsection entries are just the
%     name of the section (no page numbers, no leaders, no separate line)


\newif\ifwriteTOC \writeTOCtrue


\def\TOCitem#1#2#3{%          process a Table of Contents item
   \ifcase#1%\bigskip
     \vskip16pt plus6pt minus2pt\goodbreak                  % chapters: \bigskip ,
   \begingroup                                  %
     \twelvepoint
     \advance\leftskip by #1\parindent          % indent sub-items
     \raggedright\tolerance=1700                % don't justify
     \advance\rightskip by \TOCmargin           % right margin comes in, but
     \parfillskip=-\TOCmargin                   % page number at edge of page
     \hangindent=1.41\parindent\hangafter=1     % hanging indentation
     \noindent #2%\hskip 0pt plus 10pt           % the text
     \hfil%\leaddots                                  % leaders to edge
     \hbox to 2em{\hss #3}%                     % the page number
     \vskip 0pt                                 % end paragraph
   \endgroup                                    %
     \medskip\tenpoint\rightskip=3pc\noindent%                % smaller font
     \or                                        % sections:
     #2\                                    % the text alone
     \or                                        % subsections:
     \ifwriteTOC #2\                      % if switch off, text alone
      \else \writeTOCtrue\fi                  % else nothing, switch back on
\fi
   }

%\font\partfont=cmr10 at 18pt
%\font\partfont=cmti10 at 22pt
\font\partfont=cmti10 at 20pt  % changed May 14, 2012
\def\topblankline{\hbox{\vrule height10.675pt depth5.325pt width0pt}\nobreak}

% some section or subsection titles should not be followed by a period:
% introduce the following control sequence at the end of the title
% preceded by \string
\def\ignoreperiod#1{\if#1.\relax\else#1\fi}


%%%%% (4) redefine macros for list of illustrations

\long\def\FIGLitem#1#2#3\@endFIGLitem#4{% process a Figure or Table list item
   %\medskip                             % some space between items
   \begingroup                          %
 %%    \raggedright\tolerance=1700        % don't justify
     \ifx\TOCmargin\undefined\skip0=\parindent % sans \TOCmargin use \parindent
     \else\skip0=\TOCmargin\fi          % but use \TOCmargin if it's there
     \advance\rightskip by \skip0       % right margin comes in, but
     \parfillskip=-\skip0               % page number at edge of page
     \hangindent=1.41\parindent\hangafter=1 % hanging indentation
     \noindent \ifshowsectID #1\ \fi #2 % show section numbers
        #3 \hskip 0pt plus 10pt         % text of ``caption''
     %\leaddots                          % leaders to edge
     \hbox to 2em{\hss\linkto{page.#4}{#4}}%  page number (could be a link)
     \vskip 0pt                         % end paragraph
   \endgroup
   }



%\def\FIGLitem#1#2#3#4{% process a Figure or Table list item
%   %\medskip                                     % some space between items
%   \begingroup                                  %
%     \raggedright\tolerance=1700                % don't justify
%     \advance\rightskip by \TOCmargin           % right margin comes in, but
%     \parfillskip=-\TOCmargin                   % page number at edge of page
%     \hangindent=2\parindent\hangafter=1     % hanging indentation
%     \noindent \ifshowsectID #1\ \fi #2         % show section numbers
%        #3 \hskip 0pt plus 10pt                 % text of ``caption''
%     \hfil %     \leaddots                                  % leaders to edge
%     \hbox to 2em{\hss #4}%                     % the page number
%     \vskip 0pt                                 % end paragraph
%   \endgroup                                    %
%   }




%%%%%%%%%%% (7) macro to make the single pages with "Part I", "Part II"


\newif\ifpartpage
\partpagefalse
\def\makepart#1#2{%
\nosechead{} \global\edef\HeadText{}\ifodd\pageno\relax
\else\advance\pageno by 1\fi
\nopagenumbers
\normalbottom
\let\ChapterTitle=\relax \let\SectionTitle=\relax
\partpagetrue
\centerline{
\hbox{\vrule height45pt depth0pt width0pt}
\vbox{\partfont\noindent #1\hfil\break \vskip4pt \noindent #2}}
\TOCwrite{\vskip0pt\penalty -200\topblankline} % decrease the penalty to push a part's TOC entry to the next page (FRV)
%\addTOC{0}{{\bf #1: #2}}{} \partpagefalse\hbox{}\newpage\pagenumbers\hbox{}\newpage}
\addTOC{0}{{\bf #1: #2}}{} \partpagefalse\hbox{}\newpage\hbox{}\newpage\pagenumbers}
%\addTOC{0}{{\bf #1: #2}}{} \partpagefalse\hbox{}\newpage\pagenumbers}
\addTOC{0}{{\fam \bffam \twelvebf #1: #2}}{}


%%%%% (2)  headline macros  (copied from small change format file inputs2.tex, 12/2/03)

% modify TeXsis macros to have the chapter name on even pages
% and section name on odd pages


% define a parallel set of macros for the left headline
\def\setLHeadline#1{\@setLHeadline#1\n\endlist}   % set the \HeadText
\def\@setLHeadline#1\n#2\endlist{% #1 is everything up to first \n
      \global\edef\LHeadText{#1}%                % No: just use #1
}


\newif\ifrightnom   \rightnomtrue
\def\HeadLine{%
   \edef\firstm{{\XA\iffalse\firstmark\fi}}%    % chapternumber of \firstmark
   \edef\topm{{\XA\iffalse\topmark\fi}}%        % chapternumber of \topmark
%   \ifRunningHeads                              % print running heads?
%     \def\He@dText{\HeadText}%                  % define head text
%   \else\def\He@dText{\relax}\fi                % or nothing
\ifnum\sectionnum=0\relax\def\LHeadText{\HeadText}% use chapter title
\else\setLHeadline{\SectionTitle}\fi%
   \ifbookpagenumbers                           % book-like numbering
      \ifodd\pageno\rightnomtrue                %   odd numbers right
      \else\rightnomfalse\fi                    %   even numbers left
   \else\rightnomtrue\fi                        % or always right
   \ifx\topm\firstm                             % Marks the same?
     \ifrightnom                                % number on right?
{\noindent{\headlinefont\hss\LHeadText}\hss\llap{\PageNumber}}% display section title (FRV)
     \else%                                      % or on left
{\noindent\rlap{\PageNumber}\hss\headlinefont
%\ifnum\chapternum=0\relax \else
%  \ifshowchaptID Chapter \the\chapternum:\
%\fi%
%\fi%
\HeadText\hss}%%   display chapter title (FRV)
      \fi%                                       % for \ifrightnom
   \else \hfill \fi}%                           % NOTHING ON FIRST PAGE

%%%% 8.  dsection{}{}


\def\dsection#1#2{%        create a new section of a document (2d arg added FRV)
  \ignorespaces
   \vskip\sectionskip                           % make some space
   \goodbreak\pagecheck\sectionminspace         % new page if needed
   \global\advance\sectionnum by \@ne           % increment section counter
%
%  Section ID:
%
   \edef\lab@l{\@chaptID\SectionStyle{\the\sectionnum}}% For \label
   \ifshowsectID                                % show section number?
     \global\edef\@sectID{\SectionStyle{\the\sectionnum}.}% save for later
     \global\edef\@fullID{\lab@l.\space\space}% % what we will use here
     \r@set                                     %   and reset counters
   \else\gdef\@fullID{}\fi                      % otherwise  section ID is empty
   \everysection%                                % user customization here
%
%  Print the Section title:
%
   \ifx\tbf\undefined\def\tbf{\bf}\fi           % default \tbf is \bf
   \vbox{%                                      % keep heading in \vbox
     {\raggedright\pretolerance=2000\hyphenpenalty=2000 % no hyphenation
     \setbox0=\hbox{\noindent\tbf\@fullID}%     % find width of ID
     \hangindent=\wd0 \hangafter=1              % ... for hanging indent
     \noindent\unhbox0{\tbf{#1}}\relax          % number and title
     %\medskip}}%                                % FRV
     }}%                      % FRV
   \nobreak                                     %
%
%  Table of Contents and Running Headlines:
%
   \begingroup                                  % group for \contents, etc.
     \def\label##1{}%                           % disable \label
   \def\@arg{#2}\ifx\@arg\empty                 % Is there a 2d arg?     (FRV)
     \global\edef\SectionTitle{#1}%             % no, use title for head (FRV)
   \else                                        % there was, so...       (FRV)
     \global\edef\SectionTitle{#2}%             % use it for running head(FRV)
   \fi%                                         %                        (FRV)
     \def\n{}\def\nl{}\def\mib{}%               % turn off \n, etc
     \ifnum\chapternum=0\setHeadline{#1}\fi     % no chap. number -> set headine
     \emsg{Section \@fullID #1}%                % announce in log file
     \def\@quote{\string\@quote\relax}%         % in case of \quoteon
     \addTOC{1}{\TOCsID{\lab@l.}#1}{\folio}% % Table of Contents entry
   \endgroup                                    % end group
   \nobreak                                     %
  \vskip\baselineskip%                          % FRV
   \nobreak                                     %
   \s@ction%                                    % checkenv, etc..
   \aftersection%                               % user can customize
}
