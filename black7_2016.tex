\input grafinp3
%\input grafinput8
\input psfig
%\eqnotracetrue

%\input form1

\input psfig
%\input grafinput8

%\showchaptIDtrue
%\def\@chaptID{7.}

%\hbox{}
\footnum=0
\chapter{Equilibrium with Complete Markets\label{recurge}}

\section{Time $0$ versus sequential trading}
 This chapter describes competitive   equilibria
of a pure exchange infinite horizon economy with stochastic
endowments.  These are  useful for studying risk sharing,
asset pricing, and consumption.       We describe two systems of markets:  an {\it Arrow-Debreu\/} structure with complete
markets in dated contingent claims all traded at time $0$,
\auth{Arrow, Kenneth J.}\auth{Debreu, Gerard}%
 and a
sequential-trading structure with complete one-period {\it Arrow
securities\/}.
 \index{Arrow securities}%
These two entail different assets  and timings of
trades, but have identical consumption allocations.
Both are referred to as complete markets
economies.  They allow more comprehensive sharing of risks
than do the incomplete markets economies to be
studied in chapters \use{selfinsure} and \use{incomplete}, or the economies
with imperfect enforcement or imperfect information, studied  in chapters
\use{socialinsurance} and \use{socialinsurance2}.
\index{complete markets}




\section{The physical setting: preferences and endowments}
%%\subsection{Preferences and endowments}
In each period $t\geq 0$, there is a realization of a stochastic
event $s_t \in S$. Let the history of events up and until time $t$
be denoted $s^t = [s_0, s_1, \ldots, s_t]$. The unconditional
probability of observing a particular sequence of events $s^t$ is
given by a probability measure $\pi_t(s^t)$. For $t > \tau$, we write the probability
of observing $s^t$ conditional on the realization of $s^\tau$as $\pi_t(s^t\vert s^\tau)$.
In this chapter, we shall assume that trading occurs after
observing $s_0$,
which we capture by setting $\pi_0(s_0)=1$ for the initially
given value of $s_0$.\NFootnote{Most of our formulas carry over to
the case where trading occurs before $s_0$ has been realized;
just postulate a nondegenerate probability
distribution $\pi_0(s_0)$ over the initial state.}
%so that the appropriate distribution of $s^t$ is
%conditional on $s_0$.\NFootnote{Most of our formulas carry over to
%the case where trading occurs before $s_0$ has been realized; just
%replace the probability measure $\pi_t(s^t\vert s_0)$ by
%$\pi_t(s^t)$.}

 In section \use{sec:Markovs} we shall follow much of the
 literatures in macroeconomics and econometrics and assume that
 $\pi_t(s^t)$ is induced by  a Markov process. We wait to impose
 that special assumption until section \use{sec:Markovs} because some important findings do not
 require making that assumption.




There are $I$ consumers named $i=1, \ldots , I$. Agent $i$
owns a stochastic endowment of one good
$y_t^i(s^t)$ that depends on the
history $s^t$.
The history $s^t$ is publicly observable.
Household $i$
purchases a history-dependent  consumption plan $c^i =
 \{c_t^i(s^t)\}_{t=0}^\infty$ and
orders these
consumption streams by\NFootnote{Exercises  \the\chapternum.13 -  \the\chapternum.17
study examples in which we replace \Ep{eq0} with
$$ U_i(c^i) =
   \sum_{t=0}^\infty \sum_{s^t} \beta^t u_i[c_t^i(s^t)]
   \pi_t^i(s^t), $$
where $\pi^i(s^t)$ is a personal probability distribution specific to consumer $i$.  Blume and Easley
(2006) studied such settings and focused  particularly on whose  beliefs ultimately influence  tails
of allocations and prices. We discuss related consequences of heterogenous beliefs and   heterogenous discounting in
appendices to this chapter.
%Throughout most of this chapter, we adopt the assumption, routinely
%employed in much of macroeconomics,
%that all agents share  probabilities.
}
\auth{Blume, Lawrence}%
\auth{Easley, David}%
$$ U_i(c^i) =
   \sum_{t=0}^\infty \sum_{s^t} \beta^t u_i[c_t^i(s^t)]
   \pi_t(s^t),
 \EQN eq0 $$
 where $0 < \beta < 1$.
The right side is equal to $ E_0 \sum_{t=0}^\infty \beta^t
u_i(c_t^i) $, where $E_0$ is the mathematical expectation operator,
conditioned on $s_0$. Here $u_i(c)$ is an increasing, twice
continuously differentiable, strictly concave function of
consumption $c\geq 0$  of one good. The utility function satisfies
the Inada condition\NFootnote{This Inada condition implies that each
agent chooses strictly positive consumption for every
date-history pair. Those interior solutions enable us to confine our
analysis to Euler equations that hold with equality and also guarantee that
`natural debt limits' don't bind  in economies with
sequential trading of Arrow securities.}
%%\NFootnote{One role of this Inada
%%condition is  to make
%%the consumption of each agent strictly positive in every
%%date-history pair.  A related role is to deliver a state-by-state borrowing limit
%%to impose in economies with sequential trading of Arrow securities.}
%justify the  restrictions that we impose
%on the quantity
%of Arrow  securities that can be issued.}
$$ \lim_{c \downarrow 0} u'_i(c) = +\infty.$$

%%TOM: I have percentaged out the following paragraph because it might be too
%%much to ask from the students at this point. (In any case, we should delete
%%the reference to `common utility function,' after our generalization above.
%%The preceding footnote might suffice where we announce what will be discussed
%%in the appendices of this chapter.

%Notice that in assuming \Ep{eq0}, we are imposing  identical preference orderings
%across all individuals $i$ that can be represented in terms of discounted
%expected utility
%with  common $\beta$, common utility function $u(\cdot)$, and common probability
%distributions $\pi_t(s^t)$. As we proceed through this chapter,
%watch for results that would evaporate if we were
%instead to allow $\beta, u(\cdot)$, or $\pi_t(s^t)$ to depend on $i$.

A {\it feasible allocation} satisfies
$$\sum_i c_t^i(s^t) \leq \sum_i y_t^i(s^t)  \EQN eq4 $$
for all $t$ and for all $s^t$.


\section{Alternative trading arrangements}
For a two-event stochastic process $s_t \in S = \{0, 1\}$, the
trees in Figures \Fg{tree1f} and \Fg{tree2f} % 7.1 and 7.2
 give
two portraits of how histories $s^t$ unfold.  From the
perspective of time $0$ given $s_0=0$, Figure \Fg{tree1f} portrays
all prospective histories possible up to time
$3$. Figure \Fg{tree2f} portrays  a {\it particular\/} history that it is
known the economy has indeed followed up to   time $2$, together
with the  two possible one-period continuations into period $3$
that can occur after that history.


\midfigure{tree1f}
\centerline{\epsfxsize=3truein\epsffile{tree1.eps}} \caption{The
Arrow-Debreu commodity space for a two-state Markov chain. At time
$0$, there are trades in time $t=3$ goods for each of the eight
nodes that signify histories that can possibly be reached starting from
the node at time $0$.} \infiglist{tree1f}
\endfigure



In this chapter we shall study two distinct trading arrangements
that correspond, respectively,  to the two views of the economy in
Figures \Fg{tree1f} and \Fg{tree2f}. %7.1 and 7.2.
One is
what we shall call the Arrow-Debreu structure.  Here  markets meet
at time $0$ to  trade claims to consumption at all times $t >0$
and that are contingent on all possible histories up to $t$,
$s^t$. In that economy, at time $0$ and for all $t \geq 1$, households trade claims on
the time $t$ consumption good {\it at all nodes\/} $s^t$. After
time $0$, no further trades occur.
     The other economy has {\it sequential\/} trading of
only one-period-ahead state-contingent claims. Here trades of one-period ahead state-contingent claims occur
at each date $t \geq 0$.   Trades for history
$s^{t+1}$--contingent date $t+1$ goods occur only at the {\it
particular\/}   date $t$ history $s^t$ that has been reached at
$t$, as in Figure \Fg{tree2f}.   %.
 It turns out that  these two trading
arrangements support identical equilibrium allocations. Those
allocations share the notable property of  being functions only of
the {\it aggregate\/} endowment realization $\sum_{i=1}^I y_t^i(s^t)$ and time-invariant
parameters describing the initial distribution of wealth.






\subsection{History dependence}
Before trading, the situation of household $i$ at time $t$ depends on the history $s^t$.
  A natural measure of household
$i$'s luck in life is  $\{y^i_0(s_0), y^i_1(s^1), \ldots,$
$ y^i_t(s^t)\}$, which evidently in general depends on the history $s^t$. A
question that will occupy us in this chapter and in chapters
\use{incomplete} and
\use{socialinsurance} is whether, after trading, the household's
consumption allocation at time $t$ is also history dependent.  Remarkably, in
the complete markets models of this chapter, the consumption
allocation  at time $t$  depends only on the aggregate
endowment realization at time $t$ and some time-invariant parameters that describe the time $0$ {\it initial\/} distribution of wealth. The market incompleteness of chapter
\use{incomplete} and the  information and enforcement frictions of
chapter \use{socialinsurance} will break that result and put
history dependence into equilibrium allocations.



\topfigure{tree2f}
\centerline{\epsfxsize=3truein\epsffile{tree2.eps}} \caption{The
commodity space with Arrow securities.  At date $t=2$, there are
trades in time $3$ goods for only those time $t=3$ nodes that can
be reached from the realized time $t=2$ history $(0,0, 1)$.}
\infiglist{tree2f}
\endfigure

%\midinsert
%$$ \grafone{tree2.eps,height=.9in}{{\bf Figure 7.2}  The commodity space
%with Arrow securities. At date $t=2$, there are   trades in time $3$
%goods for only those time $t=3$ nodes that can be  reached from the
%realized time $t=2$ history $(0,0, 1)$.}
%$$
%\endinsert
\section{Pareto problem}
As a benchmark against which to measure allocations attained by a
market  economy, we seek efficient allocations. An allocation is
said to be efficient if it is Pareto optimal: it has the property
that any reallocation  that makes one household strictly better
off also makes one or more other households worse off. We can find
efficient allocations by posing a Pareto problem for a fictitious
social planner. The  planner attaches nonnegative {\it Pareto
weights\/} $\lambda_i, i=1, \ldots, I$ to the consumers' utilities and
chooses allocations $c^i, i=1, \ldots, I$ to maximize
$$ W = \sum_{i=1}^I \lambda_i U_i(c^i)  \EQN planner1  $$
subject to \Ep{eq4}.
We call an allocation {\it efficient\/}
if it solves this problem for some set of
nonnegative $\lambda_i$'s.
Let $\theta_t(s^t)$ be a nonnegative Lagrange multiplier on the
feasibility constraint \Ep{eq4}
for time  $t$ and  history  $s^t$,  and form
the Lagrangian
$$ L =
 \sum_{t=0}^\infty \sum_{s^t}
   \left\{  \sum_{i=1}^I \lambda_i \beta^t u_i(c_t^i(s^t)) \pi_t(s^t)
     + \theta_t(s^t) \sum_{i=1}^I [ y_t^i(s^t) - c_t^i(s^t) ]  \right\} . $$
The first-order condition for maximizing $L$
with respect to $c_t^i(s^t)$ is
$$ \beta^t u'_i(c_t^i(s^t))\pi_t(s^t) = \lambda_i^{-1}  \theta_t(s^t)
\EQN planner2 $$
for each $i, t, s^t$.  Taking the ratio of \Ep{planner2} for  consumers
$i$ and $1$, respectively,
gives
$$ {u'_i(c^i_t(s^t)) \over u'_1(c^1_t(s^t))} = {\lambda_1  \over \lambda_i}  \EQN foncPareto101 $$
which implies
$$ c_t^i(s^t) = u^{\prime -1}_i
    \left(\lambda_i^{-1} \lambda_1 u'_1(c_t^1(s^t)) \right). \EQN planner3
$$
Substituting \Ep{planner3} into feasibility condition \Ep{eq4} at
equality gives
$$ \sum_i u^{\prime -1}_i
    \left(\lambda_i^{-1} \lambda_1 u'_1(c_t^1(s^t)) \right)
  = \sum_i y_t^i(s^t)  . \EQN planner4 $$
Equation \Ep{planner4} is one equation in the one unknown  $c_t^1(s^t)$. The right
side of \Ep{planner4} is the realized aggregate endowment, so the
left side is  a function only of the aggregate endowment. Thus, given $\{\lambda_i\}_{i=1}^I$,
$c_t^1(s^t)$ depends only on the current realization of the
aggregate endowment and not separately  either on  the date $t$ or on the specific history $s^t$
leading up to that aggregate endowment or  the cross-section distribution of individual
endowments realized at $t$. Equation \Ep{planner3} then implies that for all $i$,
$c_t^i(s^t)$ depends only on the aggregate endowment realization.
We thus have:
\medskip
\noindent{\sc Proposition 1:} An efficient allocation is a function of
the realized aggregate endowment and  does not depend separately  on either the specific history $s^t$
leading up to that aggregate endowment or on the cross-section distribution of individual
endowments realized at $t$: $c_t^i(s^t)=c_\tau^i(\tilde
s^{\tau})$ for $s^t$ and $\tilde s^{\tau}$ such that $\sum_j
y_t^j(s^t) = \sum_j y_\tau^j(\tilde s^{\tau})$.
\medskip

 To compute the optimal allocation, first solve  \Ep{planner4}
for $c_t^1(s^t)$, then solve \Ep{planner3} for $c_t^i(s^t)$.
Note  from \Ep{planner3} that only the ratios of the Pareto weights matter,
so that we are free to normalize the weights, e.g., to impose
$\sum_i \lambda_i=1$.



\subsection{Time invariance of Pareto weights}
  Through  equations \Ep{planner3} and \Ep{planner4},
the allocation $c_t^i(s^t)$ assigned to consumer $i$
depends in a time-invariant way on the aggregate endowment
$\sum_j y_t^j(s^t)$.  Consumer $i$'s share of the aggregate endowment varies
directly with his Pareto weight $\lambda_i$.   In chapter
\use{socialinsurance}, we shall see that the constancy through
time of the Pareto weights
 $\{\lambda_j\}_{j=1}^I$ is a telltale sign that there are no
enforcement- or information-related  incentive problems
in this economy. In chapter \use{socialinsurance}, when we inject those imperfections  into the environment, the time invariance of the Pareto
weights evaporates.
\index{Pareto weights!time invariance}%
%%%%%%%%%%%%%%%%%%%%%%%%%%%%%%%%%%%%%%%%%%%%


\section{Time $0$ trading: Arrow-Debreu securities}
We now describe how an optimal allocation can be attained by a
competitive equilibrium with the Arrow-Debreu timing. Households
trade dated history-contingent claims to consumption. There is a
complete set of securities. Trades occur at time $0$, after $s_0$
has been realized. At $t=0$, households can exchange claims on
time $t$ consumption, contingent on history $s^t$ at price
$q_t^0(s^t)$, measured in some unit of account. The superscript $0$ refers to the date at which
trades occur, while the subscript $t$ refers to the date that
deliveries are to be made.  The household's budget constraint is
$$ \sum_{t=0}^\infty \sum_{s^t}  q_t^0(s^t)  c_t^i(s^t)   \leq
   \sum_{t=0}^\infty \sum_{s^t}  q_t^0(s^t) y_t^i(s^t).   \EQN eq2 $$
The household's problem is to choose  $c^i$
to maximize expression \Ep{eq0} subject to inequality  \Ep{eq2}.
%Here $q_t^0(s^t)$
%is the price of time $t$ consumption contingent on
%history $s^t$ at $t$ in terms of an abstract unit of account
%or  numeraire.

Underlying the {\it single\/}  budget constraint \Ep{eq2} is the
fact that multilateral   trades are possible through a clearing
operation that keeps track of net claims.\NFootnote{In
 the language of modern payments systems, this is a system
with net settlements, not gross settlements, of trades.}
All trades occur at time $0$.  After time $0$, trades that
were agreed to at time $0$ are executed, but no more trades occur.


Attach a Lagrange multiplier $\mu_i$ to each household's  budget constraint  \Ep{eq2}.
We obtain the
first-order conditions for the household's problem:
$$  {\partial U_i(c^i) \over \partial c_t^i(s^t)} = \mu_i q_t^0(s^t), \EQN eq3b $$
for all $i, t, s^t$.
The left side is the derivative of total utility with respect to
the time $t$, history $s^t$ component of consumption.  Each
household has its own Lagrange multiplier $\mu_i$ that is independent of time. With specification \Ep{eq0} of the utility functional,
we have
$$ {\partial U_i(c^i) \over \partial c_t^i(s^t)} =   \beta^t
 u'_i[c_t^i(s^t)] \pi_t(s^t) .  \EQN eq3a $$
This expression  implies that equation \Ep{eq3b} can be written
$$ \beta^t u'_i[c_t^i(s^t)] \pi_t(s^t) = \mu_i q_t^0(s^t). \EQN eq3 $$


We use the following  definitions:
\medskip
\noindent{\sc Definitions:}   A {\it price system} is a sequence of
functions $\{q_t^0(s^t)\}_{t=0}^\infty$.  An {\it allocation} is
a list of sequences of functions $c^i=\{c_t^i(s^t)\}_{t=0}^\infty$,
 one for each $i$.
\medskip
\noindent{\sc Definition:} A {\it competitive equilibrium} is
a feasible allocation and a price system such that, given
the price system,  the allocation solves
each household's problem.
\medskip
Notice that equation \Ep{eq3} implies
$$ {u'_i[c_t^i(s^t)] \over
 u'_j[c_t^j(s^t)] }
 = {\mu_i\over \mu_j}          \EQN eq5 $$
for all pairs $(i,j)$.
Thus, ratios of marginal utilities between pairs of agents are
constant across all histories and dates.


An equilibrium allocation solves equations  \Ep{eq4},  \Ep{eq2},
and \Ep{eq5}. Note that equation \Ep{eq5} implies that
$$ c_t^i(s^t) = u^{\prime -1}_i  \left\{u'_1[c_t^1(s^t)] {\mu_i \over \mu_1}
 \right\}. \EQN eq7 $$
Substituting this into equation \Ep{eq4} at equality gives
$$ \sum_i
u^{\prime -1}_i  \left\{u'_1[c_t^1(s^t)] {\mu_i \over \mu_1}\right\} =
       \sum_i y_t^i(s^t) .  \EQN eq8 $$
The right side of equation \Ep{eq8} is the current realization of
the aggregate endowment. %It does not {\it per se} depend on the
%specific history leading up this outcome;
Therefore, the left
side, and  so $c_t^1(s^t)$,  must also depend only on the current
aggregate endowment, as well as on the ratios
$\{ {\frac{\mu_i}{\mu_1}} \}_{i=2}^I$.  It follows from equation \Ep{eq7} that the
equilibrium allocation $c_t^i(s^t)$ for each $i$ depends only on
the economy's aggregate endowment as well as on
$\{ {\frac{\mu_j}{\mu_1}} \}_{j=2}^I$. We summarize this analysis  in
the following proposition:
\medskip
\noindent{\sc Proposition 2:}  The competitive equilibrium allocation
is a function of the realized aggregate endowment and does not depend
on time $t$ or the specific history or on
the cross section distribution of endowments:
$c_t^i(s^t)=c_\tau^i(\tilde s^{\tau})$ for all histories $s^t$ and $\tilde
s^{\tau}$ such that $\sum_j y_t^j(s^t) = \sum_j y_\tau^j(\tilde
s^{\tau})$.


%is not history dependent; $c^i_t(s^t)=\bar c^i(s_t)$.
\medskip


\index{history dependence}
%\medskip
%The lack of history dependence of the equilibrium allocation is noteworthy.
%In chapters \use{incomplete} and \use{socialinsurance}, we shall analyze
%settings in which impediments to trade that emerge from information
%and enforcement problems cause allocations to be history dependent.






\subsection{Equilibrium pricing function}
  Suppose that $c^i$, $i=1, \ldots, I$  is an equilibrium allocation.
Then the   marginal condition \Ep{eq3b} or \Ep{eq3} can be regarded as determining
the price system $q_t^0(s^t)$ as a function of the equilibrium allocation assigned
to household $i$, for any $i$. But to exploit this fact in computation, we need a way first to compute
an equilibrium allocation without simultaneously  computing prices.  As we shall see soon, solving the planning problem provides a convenient way
to do that.


  Because the units of the
price system are arbitrary, one of the prices
can be  normalized
at any positive value.  We shall set $q^0_0(s_0)= 1$,
putting the price system in units of time $0$
  goods. This choice implies that
 $\mu_i  =  u'_i[c_0^i(s_0)] $
for all $i$.
%one of the multipliers
%can be  normalized
%at any positive value.  We shall set $\mu_1 =  u'[c_0^1(s_0)]$, so that
%$q_0^0(s_0) = 1$, putting the price system in units of time $0$
%  goods.\NFootnote{This choice
%also implies that $\mu_i  =  u'[c_0^i(s_0)] $
%for all $i$.}



\subsection{Optimality of equilibrium allocation}
  A competitive  equilibrium allocation   is a particular Pareto
optimal allocation, one that sets the Pareto weights $\lambda_i =
\mu_i^{-1}$.  These weights are unique up to multiplication by a positive scalar. Furthermore, at a
competitive equilibrium allocation, the {\it shadow prices\/}
$\theta_t(s^t)$ for the associated planning problem equal the
prices $q_t^0(s^t)$ for goods to be delivered at date $t$
contingent on history $s^t$ associated with the Arrow-Debreu
competitive equilibrium. That  allocations for the planning
problem and the competitive equilibrium are identical reflects the
two fundamental theorems of welfare economics (see Mas-Colell,
Whinston, and Green (1995)).
The first welfare theorem states that a  competitive equilibrium allocation is efficient.
The second welfare theorem states that any efficient allocation can be supported by a competitive equilibrium
with an appropriate initial distribution of wealth.
 \auth{Mas-Colell, Andreu}
\auth{Whinston, Michael D.}\auth{Green, Jerry R.}


\subsection{Interpretation of trading arrangement}
  In the  competitive   equilibrium, all trades occur at $t=0$ in
one market.  Deliveries occur after $t=0$, but no more trades.
  A vast clearing or credit system operates at $t=0$.
It ensures that condition \Ep{eq2} holds for each household
$i$.   A symptom of the once-and-for-all and net-clearing trading arrangement is
that each household faces one  budget constraint that accounts
for  trades across all dates and histories.


  In section \use{sec:arrowsecurities}, we describe another trading arrangement with
more trading dates but fewer securities at each date. %As a prelude
%to that section, we describe some asset-pricing implications
%embedded in the model with time $0$ trading. But first we study
%three examples of equilibrium allocations when markets are
%complete.


\subsection{Equilibrium computation}
To compute an equilibrium, we have somehow to determine ratios of
the Lagrange multipliers, $\mu_i / \mu_1$, $i=1, \ldots, I $,
that appear in equations \Ep{eq7} and \Ep{eq8}.  The following {\it
Negishi algorithm\/} accomplishes this.\NFootnote{See Negishi
(1960).}
\medskip
\noindent{\bf 1.}  Fix a positive value for one $\mu_i$, say $\mu_1$,
throughout the algorithm.  Guess  some positive values for the
remaining  $\mu_i$'s. Then solve  equations \Ep{eq7} and \Ep{eq8} for
a candidate consumption allocation $c^i, i=1, \ldots, I$.

\medskip
\noindent{\bf 2.}  Use \Ep{eq3} for any household $i$ to solve for the
price system $q_t^0(s^t)$.
\medskip
\noindent{\bf 3.}  For $i =1, \ldots, I$, check the budget constraint
\Ep{eq2}.  For those $i$'s for which the cost of consumption
exceeds the value of their endowment, raise $\mu_i$, while for
those $i$'s for which the reverse inequality holds, lower $\mu_i$.
\medskip \noindent{\bf 4.} Iterate to convergence on steps 1-3.
\medskip
Multiplying all of the $\mu_i$'s by a positive scalar
simply  changes the units of the price system. That is why we
are free to normalize as we have in step 1.

In general, the equilibrium price system and distribution of wealth are mutually determined. Along with the equilibrium allocation, they solve
a vast system of simultaneous equations.  The Negishi algorithm provides one way to solve those equations.
In applications, it can be complicated to implement.  Therefore, in order to simplify things,
  most of the examples and exercises in this chapter specialize  preferences
in a way that eliminates the dependence of equilibrium prices on the distribution of wealth.







\section{Simpler computational algorithm}
The preference specification in the following example enables us to avoid iterating on Pareto weights as in
the Negishi algorithm.
\subsection{Example 1: risk sharing}\label{sec:ex1_risksharing}%
\noindent Suppose that the agents have identical preference orderings where
the one-period utility function is of the constant
relative risk-aversion (CRRA) form
$$ u(c) = (1-\gamma)^{-1} c^{1-\gamma} , \ \gamma > 0.$$
Then equation \Ep{eq5} implies
$$ [c_t^i(s^t)]^{-\gamma}
   = [c_t^j(s^t)]^{-\gamma} {\mu_i \over \mu_j} $$
or
$$ c_t^i(s^t) = c_t^j(s^t) \left({\mu_i \over \mu_j}\right)^{-{1 \over  \gamma}}.
    \EQN consmooth $$
Equation \Ep{consmooth} states that time $t$ elements
of consumption allocations  to
distinct agents are
constant fractions of one another. With a power utility function,
 it says that  individual
consumption is perfectly correlated with the aggregate endowment
or aggregate consumption.\NFootnote{Equation \Ep{consmooth}
implies that conditional on the history $s^t$, time $t$ consumption $c_t^i(s^t)$
 is independent of the household's
individual endowment at $t, s^t$, $y_t^i(s^t)$.  Mace (1991), Cochrane (1991),
and Townsend (1994) have tested and rejected versions of this
conditional independence hypothesis. In chapter
\use{socialinsurance}, we study how particular impediments to
trade explain these rejections. } \auth{Cochrane, John
H.} \auth{Mace, Barbara} \auth{Townsend, Robert M.}

The fractions  of the aggregate endowment assigned to each
individual are independent of the realization of $s^t$.    Thus,
there is extensive cross-history and cross-time consumption
sharing. The constant-fractions-of-consumption characterization
comes from two aspects of the theory: (1) complete markets
and (2) a  homothetic one-period utility function.


\subsection{Implications for equilibrium computation}\label{sec:compeasy}%
\noindent Equation \Ep{consmooth} and the pricing formula
\Ep{eq3} imply that an equilibrium price vector satisfies
$$ q_t^0(s^t) = \mu_i^{-1} \alpha_i^{-\gamma} \beta^t (\overline y_t(s^t))^{-\gamma} \pi_t(s^t) , \EQN price_precompute$$
%$$ q_t^0(s^t) = \mu_i^{-1} \beta^t \left(\alpha_i \overline y_t(s^t)\right)^{-\gamma} \pi_t(s^t)  $$
where $c_t^i(s_t) = \alpha_i \overline y_t(s^t)$, $\overline y_t(s^t) = \sum_i y_t^i(s^t)$, and $\alpha_i$ is consumer $i$'s
fixed consumption share of the aggregate endowment.
We are free to normalize the price system by setting $\mu_i \alpha_i^{-\gamma}$ for one consumer to an arbitrary positive number.

The homothetic CRRA preference specification that leads to equation \Ep{price_precompute} allows us to compute an equilibrium
using the following steps:
\medskip
\item{1.} Use \Ep{price_precompute} to compute an equilibrium price system.

\medskip
\item{2.} Use this price system and consumer $i$'s budget constraint to compute
$$ \alpha_i = {\frac {\sum_{t=0}^\infty \sum_{s^t} q_t^0(s^t) y_t^i(s^t) } {\sum_{t=0}^\infty \sum_{s^t} q_t^0(s^t) \bar y_t(s^t)  } } .$$
Thus, consumer $i$'s fixed consumption share $\alpha_i$ equals its share of aggregate wealth evaluated at the competitive  equilibrium price vector.

%\section{Further Examples}


\subsection{Example 2: no aggregate uncertainty}\label{sec:example2GG}%
In this example, the endowment structure is sufficiently simple that we can compute an equilibrium without assuming a homothetic
one-period utility function. Let the stochastic event $s_t$ take values on the unit interval
$[0,1]$.  There are two households, with $ y_t^1(s^t) =s_t$ and
$y_t^2(s^t) =1-s_t$.  Note that the aggregate endowment is
constant, $\sum_i y_t^i(s^t) =1$.  Then equation \Ep{eq8} implies
that $c_t^1(s^t)$ is constant over time and across histories, and
equation \Ep{eq7} implies that $c_t^2(s^t)$ is also constant.
%  We use a guess-and-verify
%method to find the equilibrium.
%Guess that
Thus, the equilibrium allocation satisfies
$c_t^i(s^t) = \bar c^i$ for all $t$ and $s^t$, for $i=1,2$.  Then
from equation \Ep{eq3},
$$ q_t^0(s^t) = \beta^t \pi_t(s^t) {u'_i(\bar c^i) \over \mu_i}, \EQN ADprice $$
for all $t$ and $s^t$, for $i=1,2$.  Household $i$'s budget constraint
implies
 $$ {u'_i(\bar c^i) \over \mu_i} \sum_{t=0}^\infty \sum_{s^t} \beta^t \pi_t(s^t)
\left[\bar c^i - y_t^i(s^t)\right] = 0 .$$ Solving this equation
for $\bar c^i$ gives
$$ \bar c^i =( 1 -\beta) \sum_{t=0}^\infty \sum_{s^t} \beta^t \pi_t(s^t) y_t^i(s^t) .
      \EQN eq20 $$
Summing equation  \Ep{eq20} verifies that $\bar c^1 + \bar c^2
=1$.\NFootnote{If we let $\beta^{-1} = 1 +r$, where $r$ is
interpreted as the risk-free rate of interest, then note that
\Ep{eq20} can be expressed as
$$ \bar c^i =\left({r \over 1+r}\right) E_0 \sum_{t=0}^\infty
     (1+r)^{-t}  y_t^i(s^t) .
$$%%      \EQN eq20new $$
Hence, equation \Ep{eq20} is a version of Friedman's permanent income
model,  which asserts that a household with zero financial assets
consumes the annuity value of its human wealth defined as the
expected discounted value of its labor income (which for present
purposes we take to be $ y^i_t(s^t)$).  In the present
example, the household completely  smooths its consumption across
time and
  histories,
something that the household in Friedman's model typically cannot
do. See chapter \use{selfinsure}.}


\subsection{Example 3: periodic endowment processes}
 Consider the special case of the previous example in which
 $s_t$ is deterministic and alternates between the values 1 and 0;
$s_0=1$, $s_t=0$ for $t$ odd, and $s_t=1$ for $t$ even.
%is a two-state Markov chain taking the two values
%$\{0,1\}$, with transition probabilities
%$\pi(1 | 0) = \pi(0 | 1) = 1$,
%$\pi(0 | 0) = \pi(1 | 1) = 0$. Suppose that
%$\pi_0(1) =1$, so that the initial value of the shock $s_0$ and therefore
%of the endowment $y_0^1$ is $1$.
Thus, the endowment processes
are perfectly predictable sequences $(1, 0, 1, \ldots)$ for
the first  agent and $(0, 1, 0, \ldots)$ for the second agent.
Let $\tilde s^t$ be the history of $(1, 0, 1, \ldots)$ up to
$t$.  Evidently, $\pi_t(\tilde s^t) =1$, and the probability
assigned to all other histories up to $t$ is zero.
The equilibrium price system is then
$$ q_t^0(s^t) =\cases{ \beta^t, & if $s^t = \tilde s^t$; \cr
                       0, & otherwise;\cr}  $$
when using the time $0$ good as numeraire, $q^0_0(\tilde s_0)=1$.
From equation \Ep{eq20}, we have
$$ \EQNalign{ \bar c^1 &=( 1 -\beta) \sum_{j=0}^\infty \beta^{2j}
                      = {1 \over 1 + \beta },   \EQN alloc;a \cr
              \bar c^2 & =( 1 -\beta) \beta \sum_{j=0}^\infty \beta^{2j}
                      = { \beta \over 1 + \beta}.  \EQN alloca;b \cr} $$
Consumer 1 consumes more every period because he is richer by virtue
of receiving his endowment earlier.

\subsection{Example 4}\label{sec:example4}%
In this example, we assume that the one-period utility function is $\frac{c^{1-\gamma}}{1-\gamma}$.
There are two consumers named $i=1,2$.  Their endowments are $y_t^1 = y_t^2 = .5 $ for $t=0,1$ and
$y_t^1 = s_t $ and $y_t^2 = 1-s_t$ for $t \geq 2$.  The state space $s_t = \{0,1\}$ and $s_t$ is governed by
a  Markov chain with  probability $\pi(s_0 = 1) = 1$ for the initial state and  time-varying transition probabilities  $\pi_1(s_1 =1 | s_0 =1) = 1 , \pi_2(s_2=1|s_1=1) = \pi_2(s_2=0|s_1 = 1) = .5,  \pi_t(s_t=1 | s_{t-1} = 1) =1, \pi_t(s_t =0 | s_{t-1}=0) = 1$
for $t > 2$.  This specification implies that $\pi_t(1, 1,  \ldots,1, 1, 1) = .5$ and $\pi_t( 0, 0,  \ldots, 0, 1 ,1 )=.5$
for all $t > 2$.

We can apply the method of subsection \use{sec:compeasy} to compute an equilibrium.  The aggregate endowment is $\overline y_t(s^t)=1$
for all $t$ and all $s^t$.  Therefore, an equilibrium price vector is
$q_1^0(1,1) = \beta, q^0_2(0,1,1)= q^0_2(1,1,1)= .5 \beta^2$ and $q_t^0(1, 1, \ldots, 1, 1) = q_t^0(0, 0, \ldots, 1, 1) = .5 \beta^t$
for $t > 2$.  Use these prices to compute the value of agent $i$'s endowment: $\sum_t \sum_{s^t} q_t^0(s^t) y_t^i(s^t)
=\sum_t \beta^t .5 [.5 + .5 + 0 + \ldots + 0] +  \sum_t \beta^t .5 [.5 + .5 + 1 + \ldots +1] = 2 \sum_t \beta^t .5 [.5 + .5 + \ldots + .5] = .5 \sum_t \beta^t = {.5 \over 1 - \beta}$. %\frac{.5}{1-\beta}$.
  Consumer $i$'s budget constraint is satisfied when he consumes a constant
consumption of $.5$ each period in each state: $c_t^i(s^t) = .5$ for all $t$ for all $s^t$.

In subsection \use{sec:example4cont}, we shall use the equilibrium allocation from the Arrow-Debreu economy in this example
to synthesize an equilibrium in an economy with sequential trading.





\section{Primer on asset pricing}
Many asset-pricing models       assume complete markets and price
an asset by breaking it into a sequence of history-contingent
claims, evaluating each  component of that sequence with the
relevant ``state price deflator'' $q_t^0(s^t)$, then adding up
those values. The  asset is  {\it redundant\/}, in the
sense that it offers a bundle of history-contingent dated claims,
each component of which has already  been priced by the market.
While we shall devote chapters \use{assetpricing1} and \use{assetpricing2} entirely  to
asset-pricing theories, it is useful to give some pricing formulas
at this  point because they help illustrate  the complete market
competitive structure.


\subsection{Pricing redundant assets}
  Let $\{d_t(s^t)\}_{t=0}^\infty$ be a stream of claims
on time $t$, history $s^t$ consumption, where $d_t(s^t)$ is a measurable
function of $s^t$.  The price of   an asset entitling
the owner to this stream
must be
$$ p^0_0(s_0)  = \sum_{t=0}^\infty  \sum_{s^t} q_t^0(s^t) d_t(s^t) . \EQN asset1
$$
If this equation did not hold, someone could make unbounded profits
by synthesizing this asset  through
purchases or sales  of  history-contingent dated commodities
and then either buying or selling the asset.
We shall elaborate  this arbitrage argument
below and later in  chapter \use{assetpricing1} on asset pricing.




\subsection{Riskless consol}
  As an example, consider the price of a {\it riskless consol}, that is,
an asset offering to pay one unit of consumption for sure
each period.  Then  $d_t(s^t) = 1$ for all $t$ and $s^t$, and the price
of this asset is
$$ \sum_{t=0}^\infty  \sum_{s^t} q_t^0(s^t) . \EQN consol1 $$


\subsection{Riskless strips}
  As another example, consider a  sequence of  {\it strips} of \index{strips}%
payoffs on the riskless consol.  The time $t$ strip is just the payoff
process $d_\tau = 1 \ {\rm if} \ \tau=t \geq 0$,
 and $0 \ {\rm otherwise}$.  Thus,
the owner of the strip is entitled  to the time $t$ coupon only.
The value of the time $t$ strip at time $0$ is evidently
$$ \sum_{s^t} q_t^0(s^t) .$$
Compare this to the price of the consol \Ep{consol1}.  We can think of the $t$-period riskless strip as a
$t$-period zero-coupon bond. See appendix  \use{sec:backus} of chapter \use{assetpricing2} for an
account of a  closely related model of yields on such bonds.


\subsection{Tail assets}
Return to the stream of dividends $\{d_t(s^t)\}_{t \geq 0}$
generated by the asset   priced in equation \Ep{asset1}. For $\tau
\geq 1$, suppose that we strip off the first $\tau-1$ periods of
the dividend and want  the time $0$ value of the remaining dividend
stream $\{d_t(s^t)\}_{t \geq   \tau}$.   Specifically, we seek
the value of this  asset for a particular possible realization of $s^\tau$. Let
$p^0_\tau(s^\tau)$ be the time $0$ price of an asset that entitles
the owner to dividend stream  $\{d_t(s^t)\}_{t \geq   \tau}$ if
history $s^\tau$ is realized,
$$ p^0_\tau(s^\tau)  = \sum_{t \geq \tau} \;
\sum_{s^t \vert s^\tau} q_t^0(s^t) d_t(s^t) ,
                                                                    \EQN asset2
$$
where the summation over $s^t \vert s^\tau$ means that we sum
over all possible subsequent histories $\tilde s^t$ such that
$\tilde s^\tau=s^\tau$.
When the units of the price are time $0$, state $s_0$ goods, the normalization is $q_0^0(s_0)=1$.
To convert the price into units of time $\tau$, history $s^\tau$ consumption
goods, divide by $q^0_\tau(s^\tau)$ to get
$$ p^\tau_\tau(s^\tau)  \equiv {p^0_\tau(s^\tau) \over q^0_\tau(s^\tau)}
  =   \sum_{t \geq \tau} \; \sum_{s^t \vert s^\tau}
{q_t^0(s^t) \over q_\tau^0(s^\tau)} d_t(s^t) .          \EQN asset3
$$
Notice that\NFootnote{Because the marginal conditions hold for all
consumers, this condition holds for all $i$.}
$$ \eqalign{  q^\tau_t(s^t) \equiv
{q_t^0(s^t) \over q_\tau^0(s^\tau)} & = {\beta^t u'_i[c_t^i(s^t)] \pi_t(s^t)
               \over \beta^\tau u'_i[c_\tau^i(s^\tau)] \pi_\tau(s^\tau) } \cr
        & = \beta^{t - \tau} {u'_i[c_t^i(s^t)] \over u'_i[c_\tau^i(s^\tau)]}
          \pi_t(s^t|s^\tau). \cr}
    \EQN qdef $$
Here
$q^\tau_t(s^t)$ is the price of one unit of consumption delivered
at time $t$, history $s^t$ in terms of the date $\tau$, history $s^\tau$
consumption good; $\pi_t(s^t|s^\tau)$ is the
probability of history $s^t$ conditional on history
$s^\tau$ at date $\tau$.  %%%, as given by \Ep{cecm_pi}.
Thus, the price at time  $\tau$, history $s^\tau$   for the ``tail asset'' is
$$ p^\tau_\tau(s^\tau)
  =   \sum_{t \geq \tau} \; \sum_{s^t \vert s^\tau}
q_t^\tau(s^t)  d_t(s^t) .                              \EQN asset4 $$


When we want to create a time series of, say, equity prices,
we use the ``tail asset'' pricing formula \Ep{asset4}.  An equity purchased
at time $\tau$ entitles the owner to the dividends from time
$\tau$ forward.   Our formula \Ep{asset4} expresses
the asset price in terms of prices with time $\tau$, history
$s^\tau$ good as numeraire.





\subsection{One-period returns}
The one-period version of equation \Ep{qdef} is
$$q^\tau_{\tau+1}(s^{\tau +1}) = \beta {u'_i[c_{\tau+1}^i(s^{\tau+1})]
               \over u'_i[c_\tau^i(s^\tau)] } \pi_{\tau+1}(s^{\tau+1} \vert s^\tau). $$
The right side is the one-period {\it pricing kernel} at time $\tau$.
If we want to  find the  price  at time $\tau$ at history $s^\tau$
of a claim to a random payoff $\omega(s_{\tau+1})$, we use
$$ p^\tau_{\tau}(s^\tau) = \sum_{s_{\tau+1}} q^\tau_{\tau+1}(s^{\tau+1})
      \omega(s_{\tau+1}) $$
or
$$ p^\tau_{\tau}(s^\tau) = E_\tau \left[\beta {u'(c_{\tau+1}) \over u'(c_{\tau})}
       \omega(s_{\tau +1}) \right], \EQN oneperiodp1 $$
where $E_\tau$ is the conditional expectation operator.
We have deleted the $i$ subscripts on utility function,
and the $i$ superscripts on consumption, with the
understanding that equation \Ep{oneperiodp1} is true for any consumer $i$;
we have also suppressed the  dependence of $c_{\tau}$ on $s^\tau$,
which is implicit.


Let $R_{\tau+1} \equiv \omega(s_{\tau+1}) /p^\tau_{\tau}(s^\tau) $
 be the one-period gross {\it return} on the asset.
Then for any asset, equation \Ep{oneperiodp1} implies
$$ 1 = E_\tau \left[\beta{u'(c_{\tau+1}) \over u'(c_\tau)} R_{\tau+1} \right]
     \equiv E_\tau \left[ m_{\tau+1} R_{\tau+1} \right]  .
    \EQN oneperiodr1 $$
The term $m_{\tau+1} \equiv \beta u'(c_{\tau+1}) / u'(c_\tau)$
functions as a {\it stochastic discount factor}.    Like
$R_{\tau+1}$, it is a random \index{stochastic discount factor}%
variable  measurable with respect to $s_{\tau+1}$, given
$s^{\tau}$.
    Equation \Ep{oneperiodr1} is a restriction on
the conditional moments of returns and $m_{t+1}$.  Applying
the law of iterated expectations to equation \Ep{oneperiodr1} gives
the unconditional moments restriction
$$ 1 = E \left[\beta{u'(c_{\tau+1}) \over u'(c_\tau)} R_{\tau+1} \right]
     \equiv E \left[ m_{\tau+1} R_{\tau+1} \right]  .
    \EQN oneperiodr2 $$
In chapters \use{assetpricing1} and \use{assetpricing2} we shall many more instances of this
equation.


In the next section, we display  another market structure in which
  the one-period pricing kernel $q^t_{t+1}(s^{t+1})$    also plays
a decisive role.  This structure uses the
celebrated one-period ``Arrow  securities,'' the sequential
trading of which  substitutes perfectly for the comprehensive trading
of long horizon claims at  time $0$.

\index{Arrow securities}

\section{Sequential trading}\label{sec:arrowsecurities}%
This section describes an alternative
market structure that preserves both the  equilibrium  allocation
 and the  key one-period asset-pricing formula
\Ep{oneperiodp1}.


\auth{Arrow, Kenneth J.}

\subsection{Arrow securities}
We build on an insight of Arrow (1964) that one-period securities
are enough to implement complete markets, provided that new
one-period  markets are reopened for trading each period and provided that time $t$, history $s^t$ wealth is properly assigned to each agent.
 Thus, at
each date  $t \geq 0$, but only at the history $s^t$ actually realized,  trades occur in a set of claims   to
one-period-ahead state-contingent consumption. We describe a
competitive equilibrium of this sequential-trading economy.
  With a full array of these one-period-ahead claims,
the sequential-trading arrangement attains the same allocation
as the competitive  equilibrium that we described earlier.
 \index{Arrow securities}

\auth{Arrow, Kenneth J.}


\subsection{Financial wealth as an endogenous state variable}
A key step in constructing a sequential-trading arrangement is to
identify a variable to serve as the state in a value function for
the household at date $t$. We find this state by taking an
equilibrium allocation and price system for the (Arrow-Debreu)
time $0$ trading structure and applying a guess-and-verify method.
We begin by asking the following question. In the competitive
equilibrium where all trading takes place at time $0$, what is the implied continuation wealth of household $i$ at time $t$ after history $s^t$?
The answer is obtained by summing up the value of the household's
holdings of claims to current and future consumption as of time $t$ and
history $s^t$. Since history $s^t$ has been realized, we discard all
claims contingent on time $t$ histories $\tilde s^t \neq s^t$ that
were not realized. Hence, the implied wealth is simply determined
by the trades that were undertaken by household $i$ at the outset of a
time $0$ trading equilibrium, when the household can be thought of as
having sold off the entire endowment stream on the right side of budget
constraint \Ep{eq2} in order to acquire the contingent consumption claims
on the left side of budget constraint \Ep{eq2}.

The differences in a sequential-trading arrangement are that (Arrow)
one-period securities are traded period by period, and that households
retain the ownership to their endowment processes throughout time.
Hence, from the perspective of a sequential-trading arrangement, the
wealth of household $i$ at a point in time can be decomposed into two
components of {\it financial wealth} and {\it non-financial wealth},
respectively.\NFootnote{In some applications,  financial wealth
is also called `non-human wealth' and non-financial wealth is called
`human wealth'.}
Financial wealth at time $t$ after history $s^t$ is the household's
beginning-of-period holdings of Arrow securities that are contingent on
the current state $s_t$ being realized,
while the household's current and future endowment constitute non-financial
wealth. Given Arrow's (1964) insight that the two trading arrangements
yield an identical equilibrium allocation, a household's financial wealth
in a sequential trading equilibrium should be equal to the above implied
continuation wealth in a time $0$ trading equilibrium, except for reducing
the latter by the value of the household's current and future endowment
(non-financial wealth). Thus, the
financial wealth of household $i$ at time $t$ after history $s^t$,
expressed in terms of the date $t$, history $s^t$ consumption good is
$$
\Upsilon^i_t(s^t) =
\sum_{\tau=t}^\infty \; \sum_{s^\tau \vert s^t}
    q_\tau^t(s^\tau) \left[ c^i_\tau(s^\tau) - y_\tau^i(s^\tau) \right]. \EQN cecm_wealth$$
Notice that budget constraint \Ep{eq2} at equality implies that
each household starts out with zero financial wealth at time $0$,
$\Upsilon^i_0(s^0)=0$ for all $i$. At $t >0$, financial wealth
$\Upsilon^i_t(s^t)$ typically differs from zero for household $i$,
but it sums to zero across $i$,
$$
\sum_{i=1}^{I}  \Upsilon^i_t(s^t) = 0, \qquad \forall t, s^t,
$$
which follows from feasibility constraint \Ep{eq4} at equality.
That is, the traded Arrow securities that make up financial
wealth are in zero net supply -- positive holdings of some households
constitute indebtedness of other households who have issued those
securities.




%%%%%%%%%%%%%%

%A key step in constructing a sequential-trading arrangement is to
%identify a variable to serve as the state in a value function for
%the household at date $t$. We find this state by taking an
%equilibrium allocation and price system for the (Arrow-Debreu)
%time $0$ trading structure and applying a guess-and-verify method.
%We begin by asking the following question. In the competitive
%equilibrium where all trading takes place at time $0$, what is the implied
%continuation wealth of household $i$ at time
%$t$ after history $s^t$, but before adding in its time $t$, history $s^t$ endowment
%$y_t^i(s^t)$? To answer this question, in period $t$, conditional on history
%$s^t$, we sum up the value of the household's purchased claims to
%current and future goods net of its outstanding liabilities. Since
%history $s^t$ has been realized, we discard all claims and liabilities
%contingent on time $t$ histories $\tilde s^t \neq s^t$ that were not realized.
%Household $i$'s net claim to delivery of goods in a future period $\tau\geqt$
%contingent on history $\tilde s^\tau$ whose time $t$ partial history
%$\tilde s^t=s^t$ is  $[c^i_\tau(\tilde s^\tau)-y_\tau^i(\tilde s^\tau)]$.
%Thus, the household's financial wealth, or the value of all its
%current and future net claims, expressed in terms of the date $t$,
%history $s^t$ consumption good is
%$$
%\Upsilon^i_t(s^t) =
%\sum_{\tau=t}^\infty \; \sum_{s^\tau \vert s^t}
%    q_\tau^t(s^\tau) \left[ c^i_\tau(s^\tau) - y_\tau^i(s^\tau) \right].
%\EQN cecm_wealth$$
%Notice that feasibility
%constraint \Ep{eq4} at equality implies that
%$$
%\sum_{i=1}^{I}  \Upsilon^i_t(s^t) = 0, \qquad \forall t, s^t.
%$$


%\subsection{Financial and non-financial wealth}
%Define $\Upsilon^i_t(s^t)$ as {\it financial wealth}
%and $\sum_{\tau=t}^\infty \; \sum_{s^\tau \vert s^t}
%    q_\tau^t(s^\tau)  y_\tau^i(s^\tau)$ as {\it non-financial wealth}.\NFootnote{In
%some applications,  financial wealth
%is also called `non-human wealth' and non-financial wealth is called `human wealth'.}
%In terms of these concepts,
%     \Ep{cecm_wealth} implies
%     $$
%\Upsilon^i_t(s^t) +
%\sum_{\tau=t}^\infty \; \sum_{s^\tau \vert s^t}
%    q_\tau^t(s^\tau)  y_\tau^i(s^\tau) = \sum_{\tau=t}^\infty \;
%\sum_{s^\tau \vert s^t}
%    q_\tau^t(s^\tau)  c^i_\tau(s^\tau)  , \EQN cecm_wealth$$
%which states that at each time and each history, the sum of financial and
%non-financial wealth
%equals the present value of current and future consumption claims.
%At time $0$,  we have set $ \Upsilon^i_t(s^0)=0$ for all $i$. At $t >0$,
%financial wealth
%$\Upsilon^i_t(s^t)$ typically differs from zero for  individual $i$,
%but it sums to zero across $i$.



%\index{wealth!financial}
%\index{wealth!non-financial}






\subsection{Reopening markets}

Formula \Ep{qdef}  takes the form of a pricing
function for a complete markets economy with date- and history-contingent
commodities whose markets can be regarded as having been reopened at date $\tau$, history
$s^\tau$, starting from   wealth levels implied by the tails of each household's
endowment and consumption streams for a complete markets economy that originally convened at $t=0$.
 We leave it as an exercise to the
reader to prove the following proposition.


\medskip
\noindent{\sc Proposition 3:} Start from the distribution of time
$t$, history $s^t$ wealth that is implicit in a time $0$ Arrow-Debreu
equilibrium. If markets are reopened at date $t$ after history
$s^t$, no trades  occur. That is,  given the  price system
\Ep{qdef}, all households choose to continue the  tails of their
original consumption plans.
\medskip
\index{no-trade result}






\index{Ponzi schemes!ruling out} \index{debt limit!natural}
\subsection{Debt limits}

In moving from  the Arrow-Debreu economy to one with sequential
trading, we propose to  match  the time $t$, history $s^{t}$ wealth of
the household  in the sequential economy with the equilibrium  tail wealth
$\Upsilon^i_t(s^t)$ from the Arrow-Debreu economy computed in equation
\Ep{cecm_wealth}. But first we have to say something about debt
limits, a feature that was only implicit in the time $0$ budget constraint \Ep{eq2} in the Arrow-Debreu economy.
%because we imposd \Ep{eq2}.
In moving to the sequential formulation, we    restrict asset trades to prevent Ponzi schemes.  We
want the weakest possible restrictions.  We
synthesize restrictions that work by starting from the equilibrium
allocation of the Arrow-Debreu economy (with time $0$ markets), and
find some state-by-state debt limits that support
the equilibrium allocation that emerged from the Arrow-Debreu economy under a sequential trading arrangement.
 Often we'll refer to these weakest possible
debt limits as the ``\idx{natural debt limit}s.'' These limits come
from the common sense requirement that it has to be {\it
feasible\/} for the consumer to repay his state contingent debt in
every possible state.
%If we assume that $c_t^i$ must be
%nonnegative or if an Inada condition at $c=0$ keeps $c \geq 0$,
Together with  our assumption that $c_t^i(s^t)$ must be nonnegative, that
feasibility requirement leads to the natural debt limits.
%Throughout this section we assume either the additional restriction
%$c_t^i \geq 0$  or else impose an Inada condition  on the
%one-period utility function:
%$\lim_{c \downarrow 0 }u'(c) = +\infty$.  Either of these leads us
%to restrict attention to nonnegative consumption
%processes.




Let $q_\tau^t(s^\tau)$ be the Arrow-Debreu price, denominated  in
units of the date $t$, history
 $s^t$  consumption good.
Consider the value of the tail of  agent $i$'s endowment sequence
at time $t$ in history $s^t$:
$$ A^i_t(s^t) =
   \sum_{\tau=t}^\infty \; \sum_{s^\tau \vert s^t}
    q^t_\tau(s^\tau) y_\tau^i(s^\tau).   \EQN cecm_debt1$$
 We call
$A^i_t(s^t)$ the {\it natural debt limit} at time $t$ and history
$s^t$. It is the  maximal value that agent $i$   can
repay starting from that period, assuming that his consumption is
zero always.
%\NFootnote{These debt limits require
%either an Inada condition on preferences that implies that
%$c_t^i \geq 0$ or an arbitrary restriction that $c_t^i >0$.}
With sequential trading, we shall require that household $i$ at time $t-1$ and
history $s^{t-1}$ cannot promise to pay more than $A^i_t(s^t)$
conditional on the realization of $s_t$ tomorrow, because it will
not be feasible to repay more. Household $i$ at
time $t-1$ faces one such borrowing constraint for each possible
realization of $s_t$  tomorrow.


\subsection{Sequential trading}
There is a sequence of markets in one-period-ahead
state-contingent
 claims.    At each date $t \geq 0$,
households trade claims to date $t+1$ consumption, whose payment
is  contingent
on the realization of $s_{t+1}$.   Let $\tilde a_t^i(s^t)$ denote the
claims to time $t$ consumption, other than its time $t$  endowment $y_t^i(s^t)$,
that household $i$ brings into time $t$ in history $s^t$. Suppose that
$ \tilde Q_t(s_{t+1} | s^t)$
is a {\it pricing kernel}  to be interpreted as follows:
$\tilde Q_t(s_{t+1} | s^t)$ is the price of one unit of time $t+1$ consumption,
contingent on the realization $s_{t+1}$ at $t+1$, when the
history at $t$ is $s^t$.
 %We are guessing
%that such a pricing kernel exists.
% and depends neither  on $t$ nor the distribution
%of wealth.
The household faces a sequence of budget constraints
for $t \geq 0$, where the time $t$, history $s^t$ budget constraint is
$$ \tilde c_t^i(s^t) + \sum_{s_{t+1}} \tilde a^i_{t+1}(s_{t+1},s^t) \tilde Q_t(s_{t+1} | s^t)
     \leq  y_t^i(s^t) + \tilde a_t^i(s^t) . \EQN eq10   $$
%Here it is understood that $\theta^i_{t+1}=\theta^i_{t+1}(s_{t+1})$,
%which depends on the realization of $s_{t+1}$ and the amount
%$\theta^i_{t+1}(s_{t+1})$ purchased at $t$.
At time $t$, a household chooses $\tilde c^i_t(s^t)$
and $\{\tilde a^i_{t+1}(s_{t+1},s^t)\}$, where
$\{\tilde a^i_{t+1}(s_{t+1},s^t)\}$  is a vector of claims on time $t+1$ consumption,
there being one element of the vector for each  value of the time $t+1$ realization of
$s_{t+1}$.    To rule out Ponzi schemes,
we impose the state-by-state borrowing
constraints
$$ -  \tilde a^i_{t+1}(s^{t+1}) \leq  A^i_{t+1}(s^{t+1}),        \EQN cecm_debt2$$
where $A^i_{t+1}(s^{t+1})$ is computed in equation \Ep{cecm_debt1}.


Let $\eta^i_t(s^t)$ and $\nu^i_t(s^t;s_{t+1})$ be  nonnegative
Lagrange multipliers on the budget constraint \Ep{eq10} and the
borrowing constraint \Ep{cecm_debt2}, respectively, for time $t$
and history $s^t$. Form the Lagrangian  \offparens
$$ \eqalign{ L^i =
 &\sum_{t=0}^\infty \sum_{s^t}
   \Bigr\{  \beta^t u_i(\tilde c_t^i(s^t)) \pi_t(s^t)  \cr
 & +\eta^i_t(s^t) \Bigl [y_t^i(s^t) + \tilde a_t^i(s^t) - \tilde c_t^i(s^t)
       - \sum_{s_{t+1}} \tilde a^i_{t+1}(s_{t+1},s^t) \tilde Q_t(s_{t+1} | s^t) \Bigr] \cr
 & +\sum_{s_{t+1}} \nu^i_t(s^t;s_{t+1}) \left[ A^i_{t+1}(s^{t+1}) + \tilde a^i_{t+1}(s^{t+1}) \right]
                                                             \Bigl\}, \cr} $$
for a given initial wealth $\tilde a^i_0(s_0)$.
The first-order conditions for maximizing $L^i$ with respect to
$\tilde c^i_t(s^t)$ and $\{\tilde a^i_{t+1}(s_{t+1},s^t)\}_{s_{t+1}}$ are
$$\EQNalign{
\beta^t u'_i(\tilde c_t^i(s^t)) \pi_t(s^t) -\eta^i_t(s^t) &=0\,,\EQN seqFOC;a \cr
\noalign{\vskip.2cm}
-\eta^i_t(s^t) \tilde Q_t(s_{t+1} | s^t) + \nu^i_t(s^t;s_{t+1})
+ \eta^i_{t+1}(s_{t+1},s^t) & =0\,, \hskip1cm \EQN seqFOC;b \cr}
$$
for all $s_{t+1}$, $t$, $s^t$. In the optimal solution to this
problem, the natural debt limit \Ep{cecm_debt2} will not be
binding, and hence the Lagrange multipliers $\nu^i_t(s^t;s_{t+1})$
 all equal  zero for the following reason: if there were any
history $s^{t+1}$ leading to a binding natural debt limit, the
household would from then on have to set consumption equal to zero
in order to honor its debt. Because the household's utility
function satisfies the Inada condition
$\lim_{c \downarrow 0} u'_i(c) = +\infty$, that would mean that all
future marginal utilities would be infinite. Thus, it would be easy
to find alternative affordable allocations that yield higher
expected utility by postponing earlier consumption to periods
after such a binding constraint. %, i.e., alternative preferable
%allocations where the natural debt limits no longer bind.

After
setting $\nu^i_t(s^t;s_{t+1})=0$ in equation \Ep{seqFOC;b}, the
first-order conditions imply the following restrictions on the
optimally chosen consumption allocation,
$$ \tilde Q_t(s_{t+1} | s^t) = \beta
{ u'_i(\tilde c_{t+1}^i(s^{t+1})) \over u'_i(\tilde c_t^i(s^t)) }
\pi_t(s^{t+1} | s^t),                                \EQN seqFOCc
$$
for all $s_{t+1}$, $t$, $s^t$.




 \medskip
\noindent{\sc Definition:}  A {\it distribution of wealth} is
a vector $\vec {\tilde a}_t(s^t) = \{\tilde a_t^i(s^t)\}_{i=1}^I$ satisfying
$\sum_i \tilde a_t^i(s^t) = 0$.
\medskip


\noindent{\sc Definition:} A {\it   competitive equilibrium with sequential trading of one-period Arrow securities} is
an initial distribution of wealth $\vec {\tilde a}_0(s_0)$, a collection of borrowing limits $\{A_t^i(s^t)\}$ satisfying
\Ep{cecm_debt1} for all $i$, for all $t$, and for all $s^t$,
a feasible allocation $\{\tilde c^i\}_{i=1}^I$, and pricing
kernels $\tilde Q_t(s_{t+1} | s^t)$ such
that

\noindent{\bf (a)} for all $i$, given
 $\tilde a^i_0(s_0)$, the borrowing limits $\{A_t^i(s^t)$,  and the pricing kernels, the consumption allocation
$\tilde c^i$ solves the household's problem for all $i$;

\noindent{\bf (b)} for all realizations of $\{s^t\}_{t=0}^\infty$, the
households' consumption
allocations and implied  portfolios
$\{\tilde c^i_t(s^t), \{\tilde a^i_{t+1}(s_{t+1},s^t)\}_{s_{t+1}}\}_i$
satisfy
$\sum_i \tilde c^i_t(s^t) = \sum_i y_t^i(s^t)$ and
$\sum_i \tilde a_{t+1}^i(s_{t+1},s^t) = 0$
for all $s_{t+1}$.
%for all $i$,  $\{\theta_t^i\}_{t=0}^\infty$ satisfies
%$\theta_{t+1}^i = g(\theta_t^i,s_t,s_{t+1})$; and
%(c) for all $t$,  $\sum_i h(\theta_{t}^i,s_t) = \sum_i y^i(s_t)$.
\medskip
\noindent This definition leaves open the initial
distribution of wealth. We'll say more about the initial distribution of wealth soon.  %The Arrow-Debreu equilibrium with
%complete markets at time $0$ in effect pinned down a {\it
%particular\/} distribution of wealth.
\index{competitive
equilibrium!sequential trading}


\subsection{Equivalence of allocations}
By making an appropriate guess about the form of the pricing
kernels, it is easy to show that a competitive equilibrium
allocation of the complete markets model with time $0$ trading is
also an allocation for a competitive equilibrium with sequential trading of one-period
Arrow securities, one
with a particular initial distribution of wealth.  Thus, take
$q_t^0(s^t)$ as given from the Arrow-Debreu equilibrium and
suppose that the pricing kernel $\tilde Q_t(s_{t+1} | s^t)$ makes
the following recursion true:
$$\EQNalign{
 q^0_{t+1}(s^{t+1}) &= \tilde Q_t(s_{t+1} | s^t) q_t^0(s^t), \cr
\noalign{\hbox{\rm or}}
\tilde Q_t(s_{t+1} | s^t) &= q^t_{t+1}(s^{t+1}),       \EQN eq16 \cr}
$$
where recall that $q_{t+1}^t(s^{t+1}) ={\frac{q_{t+1}^0(s^{t+1})}{q_t^0 (s^t)}}$.


Let
$\{c_t^i(s^t)\}$ be a competitive equilibrium allocation
in the Arrow-Debreu economy.
If equation \Ep{eq16} is satisfied, that allocation is also a
sequential-trading competitive equilibrium allocation.  To show this fact,
take the household's first-order conditions \Ep{eq3} for the Arrow-Debreu
economy from two successive periods and divide one by
the other to get
$$ {\beta u'_i[c_{t+1}^i(s^{t+1})] \pi(s^{t+1} | s^t) \over
       u'_i[c_t^i(s^t)]} = {q^0_{t+1}(s^{t+1}) \over q_t^0(s^t) }
   = \tilde Q_t(s_{t+1}\vert s^t).  \EQN eq17 $$
If the pricing kernel satisfies equation \Ep{eq16}, this equation
is equivalent with the first-order condition \Ep{seqFOCc} for the sequential-trading
competitive equilibrium economy. It remains for us
to choose the initial wealth of the sequential-trading equilibrium so
that the sequential-trading competitive equilibrium duplicates the
Arrow-Debreu competitive equilibrium allocation.


We conjecture that the initial wealth vector
$\vec {\tilde a}_0(s_0)$ of the sequential-trading economy should be chosen
to be the zero vector. This is a natural conjecture, because
it means that each household must rely on its own
endowment stream to finance consumption, in the same way that households are
constrained to finance their history-contingent purchases
for the infinite future at time $0$ in the Arrow-Debreu economy.
To prove that the conjecture is correct, we must show that the zero
initial wealth vector enables household $i$ to finance $\{c^i_t(s^t)\}$
and leaves no room to increase consumption in any period
 after any history.


The proof proceeds by guessing that, at time $t\geq 0$ and history
$s^t$, household $i$ chooses a portfolio given
by $\tilde a^i_{t+1}(s_{t+1},s^t)=\Upsilon^i_{t+1}(s^{t+1})$ for all
$s_{t+1}$.  The value of this  portfolio expressed
in terms of the date $t$, history $s^t$ consumption good is
$$\EQNalign{
  \sum_{s_{t+1}} \tilde a^i_{t+1}(s_{t+1},s^t)  \tilde Q_t(s_{t+1} | s^t)
& = \sum_{s^{t+1}\vert s^t} \Upsilon^i_{t+1}(s^{t+1}) q^t_{t+1}(s^{t+1}) \cr
&= \sum_{\tau=t+1}^\infty \; \sum_{s^\tau \vert s^t}
    q_\tau^t(s^\tau) \left[ c^i_\tau(s^\tau) -
y_\tau^i(s^\tau) \right] ,\hskip1cm
                                                   \EQN cecm_Wcost \cr}
$$
where we have invoked expressions \Ep{cecm_wealth} and
\Ep{eq16}.\NFootnote{We have also used the following identities,
$$
q^{t+1}_\tau(s^\tau) q^t_{t+1}(s^{t+1})
=  {q_\tau^0(s^\tau) \over q_{t+1}^0(s^{t+1})}
   {q_{t+1}^0(s^{t+1}) \over q_t^0(s^t)}
= q^t_\tau(s^\tau) \ \hbox{\rm for} \;\tau>t.
$$}
To demonstrate that household $i$ can afford this portfolio strategy, we now
use budget constraint \Ep{eq10} to compute the implied consumption plan
$\{\tilde c^i_\tau(s^\tau)\}$. First, in the initial period $t=0$ with
$\tilde a^i_0(s_0)=0$, the substitution of equation \Ep{cecm_Wcost} into
budget constraint \Ep{eq10}
at equality yields
$$ \tilde c_0^i(s_0) + \sum_{t=1}^\infty \; \sum_{s^t}
    q_t^0(s^t) \left[ c^i_t(s^t) - y^i_t(s^t) \right]
   =  y^i_t(s_0) + 0\,.  $$
This expression together with budget constraint \Ep{eq2} at
equality imply $\tilde c_0^i(s_0) = c_0^i(s_0)$. In other words,
the proposed  portfolio is affordable in period $0$ and the
associated consumption plan is the same as in the competitive
equilibrium of the Arrow-Debreu economy. In all consecutive future
periods $t > 0$ and histories $s^t$, we replace $\tilde
a^i_t(s^t)$ in constraint \Ep{eq10} by $\Upsilon^i_t(s^t)$, and
after noticing that the value of the asset portfolio in
\Ep{cecm_Wcost} can be written as
$$
\sum_{s_{t+1}} \tilde a^i_{t+1}(s_{t+1},s^t) \tilde Q_t(s_{t+1} | s^t)
= \Upsilon^i_t(s^t) - \left[ c^i_t(s^t) - y^i_t(s^t) \right],
\EQN cecm_Wcost2
$$
it follows immediately from \Ep{eq10} that $\tilde c^i_t(s^t)= c^i_t(s^t)$
for all periods and histories.


We have shown that the proposed portfolio strategy attains the same
consumption plan
as in the competitive equilibrium of the Arrow-Debreu economy, but what
precludes household $i$ from further increasing current consumption by
reducing some component of the asset portfolio? The answer lies
in the debt limit restrictions to which the household must adhere. In
particular,
if the household wants to ensure that consumption plan
$\{c^i_\tau(s^\tau)\}$
can be attained starting next period in all possible future states,
the household should subtract the value of this commitment to future
consumption
from the natural debt limit in \Ep{cecm_debt1}.
 Thus, the household is facing a
state-by-state borrowing constraint that is more
 restrictive than restriction \Ep{cecm_debt2}:
for any $s^{t+1}$,
$$  -  \tilde a^i_{t+1}(s^{t+1}) \leq  A^i_{t+1}(s^{t+1} )
- \sum_{\tau=t+1}^\infty \; \sum_{s^\tau \vert s^{t+1}}
  q_\tau^{t+1}(s^\tau) c^i_\tau(s^\tau)
  = - \Upsilon^i_{t+1}(s^{t+1}),$$
or
$$   \tilde a^i_{t+1}(s^{t+1})  \geq  \Upsilon^i_{t+1}(s^{t+1}). $$
Hence, household $i$ does not want to increase consumption at time $t$ by
reducing next period's wealth below $\Upsilon^i_{t+1}(s^{t+1})$ because
that would jeopardize attaining the preferred consumption plan
that satisfies first-order conditions \Ep{seqFOCc} for all future periods and
histories.








\section{Recursive competitive equilibrium}\label{sec:Markovs}%
We have established that  equilibrium allocations are the same
in the Arrow-Debreu economy with  complete markets in dated
contingent claims all traded at time 0 and in a sequential-trading
economy with a complete set of one-period Arrow securities. This finding
holds for arbitrary individual endowment processes
$\{y^i_t(s^t)\}_i$ that are measurable functions of the history of
events $s^t$, which in turn are governed by some arbitrary
probability measure $\pi_t(s^t)$. At this level of generality,
 the pricing kernels $\tilde Q_t(s_{t+1} | s^t)$ and the
wealth distributions $\vec {\tilde a}_t(s^t)$ in the
sequential-trading economy both depend on the history $s^t$, so both
 are time-varying functions of all past events
$\{s_\tau\}_{\tau=0}^t$. This can make it  difficult to
formulate an economic model that can be used to confront empirical
observations. We want  a framework in which economic outcomes
are functions of a limited number of ``state variables'' that
summarize the effects of past events and current information. This
 leads us to make the following specialization of the
exogenous forcing processes that facilitates a recursive formulation
of the sequential-trading equilibrium.


\subsection{Endowments governed by a Markov process}
Let $\pi(s'|s)$ be a Markov chain with given initial distribution
$\pi_0(s)$ and state space $s\in S$.   That is, ${\rm
Prob}(s_{t+1}  = s' \vert
 s_t = s) =\pi(s'|s)$ and ${\rm Prob}  (s_0 = s) = \pi_0(s)$.
As we saw in chapter \use{timeseries},
 the chain induces a sequence of probability
measures $\pi_t(s^t)$  on histories $s^t$
via the recursions
$$ \pi_t(s^t) = \pi(s_t | s_{t-1}) \pi(s_{t-1} | s_{t-2}) \ldots
       \pi(s_1 | s_0)   \pi_0(s_0) . \EQN eq1  $$
%We call $s_t$ the {\it state\/} at $t$ and $s^t$ the {\it history\/} at $t$.
In this chapter, we have assumed that trading occurs after $s_0$ has been observed,
which we capture by setting $\pi_0(s_0)=1$ for the initially
given value of $s_0$.
%Formula \Ep{eq1} is the unconditional probability of $s^t$ when
%$s_0$ has not been realized.  In this chapter we have assumed that
%trading occurs after $s_0$ has been observed, so we have used
%the distribution of $s^t$ that is conditional
%on $s_0$,
%$$ \pi_t(s^t|s_0) = \pi(s_t | s_{t-1}) \pi(s_{t-1} | s_{t-2}) \ldots
%       \pi(s_1 | s_0)   . \EQN eq1a  $$

Because of the Markov property, the conditional probability
$\pi_t(s^t\vert s^\tau)$ for $t>\tau$ depends only on the state
$s_\tau$ at time $\tau$ and does not depend on the history
before $\tau$,
$$\pi_t(s^t|s^\tau) = \pi(s_t\vert s_{t-1}) \pi(s_{t-1} \vert s_{t-2})
          \ldots\pi(s_{\tau+1} \vert s_\tau).\EQN cecm_pi$$


Next, we assume that households' endowments in period $t$ are
time invariant measurable functions of $s_t$,
$y^i_t(s^t)=y^i(s_t)$ for each $i$. %This assumption means that
%each household's endowment follows a Markov process, since $s_t$
%itself is governed by a Markov process.
Of course, all of our
previous results continue to hold, but the Markov assumption
for $s_t$ imparts further structure to the equilibrium.


\index{history dependence!lack of}
\subsection{Equilibrium outcomes inherit the Markov property}
Proposition 2  asserted a particular kind of history independence
of the equilibrium allocation that prevails under any
stochastic process for the endowments. In particular, each
individual's consumption is  a function only of the current
realization of the aggregate endowment and does not depend on the
specific history leading to that outcome.\NFootnote{Of course, the equilibrium allocation also depends on the distribution
of $\{y_t^i(s^t)\}$ {\it processes} across agents $i$, as reflected in the relative values of the Lagrange multipliers
$\mu_i$.} Now, under %the
%assumption that  all time $t$ endowments depend only on the time $t$ value of a Markov state $s_t$,
%it follows immediately from equations \Ep{eq7} and \Ep{eq8} that
%the equilibrium allocation is  a function only of the current
%our
%state $s_t$,
our present assumption that $y_t^i(s^t)= y^i(s_t)$ for each $i$, it follows immediately that
$$
c^i_t(s^t) = \bar c^i(s_t).     \EQN Markovc
$$

Substituting \Ep{cecm_pi} and \Ep{Markovc} into
\Ep{seqFOCc} shows that the pricing kernel in the sequential-trading
equilibrium is   a function only of the current state,
$$
\tilde Q_t(s_{t+1} | s^t) = \beta
{ u'_i(\bar c^i(s_{t+1})) \over u'_i(\bar c^i(s_t)) }
\pi(s_{t+1} | s_t) \equiv Q(s_{t+1} | s_t).   \EQN MarkovQ
$$
After similar substitutions with respect to equation \Ep{qdef},
we can also
establish history independence of the relative prices in the
Arrow-Debreu economy:




\medskip
\noindent{\sc Proposition 4:}  If time $t$  endowments are a function of a Markov
state $s_t$, the Arrow-Debreu equilibrium price of date-$t\geq 0$,
history $s^t$
consumption goods expressed in terms of date $\tau$ ($0\leq \tau \leq t$),
history $s^\tau$
consumption goods is not history dependent:
$q^\tau_t(s^t) = q^{j}_{k}(\tilde s^{k})$ for $j, k \geq 0$
such that $t-\tau=k-j$ and
$[s_\tau, s_{\tau+1}, \ldots , s_t] =
 [\tilde s_{j}, \tilde s_{j+1}, \ldots , \tilde s_{k}]$.
\medskip


Using this proposition, we can verify that both the natural debt
limits \Ep{cecm_debt1} and households' wealth levels \Ep{cecm_wealth}
exhibit history independence,
$$\EQNalign{
 A^i_t(s^t) &= \bar A^i(s_t) \,,              \EQN MarkovA \cr
 \Upsilon^i_t(s^t) &= \bar \Upsilon^i(s_t)\,. \EQN MarkovU \cr}
$$
The finding concerning wealth levels \Ep{MarkovU} conveys a  useful
insight into how the sequential-trading competitive equilibrium
attains the first-best outcome in which no idiosyncratic risk is
borne by individual households. In particular, each household
enters every period with a wealth level that is independent of
past realizations of his endowment. That is, his past trades have
fully insured him against the idiosyncratic outcomes of his
endowment. And from that very same insurance motive, the household
now enters the present period with a wealth level that is a
function of the current state $s_t$. It is a state-contingent
wealth level that was chosen by the household in the previous
period $t-1$, and this wealth will be just sufficient to
continue  a trading strategy previously designed to insure against future
idiosyncratic risks. The optimal holding of wealth is  a function
 of $s_t$ alone because the current state $s_t$ determines the
current endowment and the current pricing kernel  and contains all information
relevant for predicting
future realizations of the household's endowment process as well as future prices. It can
be shown that a household especially wants  higher wealth levels for
those states next period that  either  make his next period
endowment low or more generally  signal poor future prospects for its endowment into
the more distant future. Of course, individuals' desires are tempered by differences in the economy's
aggregate endowment across states (as reflected in equilibrium
asset prices). Aggregate shocks cannot be diversified away but
must  be borne somehow  by all of the households. The pricing kernel
$Q(s_t\vert s_{t-1})$ and the assumed clearing of all markets
set into action  an ``invisible hand'' that coordinates households'
transactions at time $t-1$  in such a way  that only aggregate
risk and  no idiosyncratic risk is borne by the households.




\subsection{Recursive formulation of optimization and equilibrium}
The fact  that the pricing kernel $Q(s'|s)$ and the endowment $y^i(s)$
are functions of a Markov process $s$ motivates us to seek a
recursive formulation of the household's optimization problem.
Household $i$'s state at time $t$ is its wealth $a^i_t$ and the
current realization $s_t$. We seek a pair of optimal policy
functions $h^i(a,s)$, $g^i(a,s,s')$ such that the household's
optimal decisions are
$$\EQNalign{ c_t^i  & = h^i(a^i_t, s_t),   \EQN policy1;a \cr
         a_{t+1}^i(s_{t+1}) & = g^i(a_t^i, s_t, s_{t+1}). \EQN policy1;b
  \cr} $$


Let $v^i(a,s)$ be the optimal value of household $i$'s problem
starting from state $(a, s)$;     $v^i(a,s)$ is the maximum
expected discounted utility  that household $i$ with current wealth $a$
can attain in state $s$.
The Bellman equation for the household's problem is
$$ v^i(a, s) = \max_{c, \hat a(s')} \left\{ u_i(c) + \beta
\sum_{s'} v^i[\hat a(s'),s'] \pi (s' | s) \right\} \EQN eq11 $$
where the maximization is subject to the following version
of constraint \Ep{eq10}:
$$ c + \sum_{s'} \hat a(s') Q(s' | s)
     \leq  y^i(s) + a  \EQN eq12   $$
and also
$$\EQNalign{c & \geq 0, \EQN eeq12;a \cr
           -   \hat a(s') & \leq \bar A^i(s'), \hskip.5cm \forall s'.   \EQN eeq12;b \cr}$$
Let the optimum decision rules be
$$\EQNalign{ c & =  h^i(a,s), \EQN policy2;a \cr
            \hat a(s') & = g^i(a,s,s').  \EQN policy2;b \cr}$$
Note that the solution of the Bellman equation implicitly depends
on $Q(\cdot \vert \cdot)$ because it appears in the constraint
\Ep{eq12}. In particular, use the first-order conditions for  the
problem on the right of equation \Ep{eq11} and the
\idx{Benveniste-Scheinkman formula} and rearrange to get
$$ Q(s_{t+1} | s_t ) = {\beta u'_i(c_{t+1}^i) \pi(s_{t+1} | s_t)
                 \over u'_i(c_t^i) }, \EQN eq14 $$
where it is understood that $c_t^i = h^i(a_t^i,s_t)$
and
$c_{t+1}^i = h^i(a_{t+1}^i(s_{t+1}), s_{t+1})
= h^i(g^i(a_t^i,s_t,s_{t+1}), s_{t+1})$.










% \medskip
%\specsec{Definition:}  A {\it distribution of wealth} is
%a vector $\vec a_t = \{a_t^i\}_{i=1}^I$ satisfying $\sum_i a_t^i = 0$.
%\medskip


\medskip\noindent{\sc Definition:} A {\it  recursive competitive equilibrium} is
an initial distribution of wealth $\vec a_0$, a set of borrowing limits $\{\bar A^i(s)\}_{i=1}^I$,
a pricing kernel $Q(s' | s)$, sets of value functions $\{v^i(a,s)\}_{i=1}^I$, and
decision rules $\{h^i(a,s), g^i(a,s,s')\}_{i=1}^I$ such
that

\noindent{\bf (a)} The state-by-state borrowing constraints satisfy the recursion
$$ \bar A^i(s) = y^i(s) + \sum_{s'} Q(s'|s) \bar A^i(s'|s).  \EQN borrowingconA $$
\medskip
\noindent{\bf (b)} For all $i$, given
 $a^i_0$, $\bar A^i(s)$,  and the pricing kernel, the value functions and decision rules
solve the household's problem;
\medskip
\noindent{\bf (c)} For all realizations of $\{s_t\}_{t=0}^\infty$, the consumption and asset
portfolios $\{\{c^i_t,$
$\{\hat a^i_{t+1}(s')\}_{s'}\}_i\}_t$ implied by the
decision rules satisfy $\sum_i c^i_t = \sum_i y^i(s_t)$ and
$\sum_i \hat a_{t+1}^i(s') = 0$
for all $t$ and $s'$.
%for all $i$,  $\{\theta_t^i\}_{t=0}^\infty$ satisfies
%$\theta_{t+1}^i = g(\theta_t^i,s_t,s_{t+1})$; and
%(c) for all $t$,  $\sum_i h(\theta_{t}^i,s_t) = \sum_i y^i(s_t)$.
\medskip
\noindent \index{recursive!competitive equilibrium}


We shall use the recursive competitive equilibrium concept
extensively in our discussion of asset pricing in chapter \use{assetpricing1}.

\subsection{Computing an equilibrium with sequential trading of  Arrow-securities}\label{sec:example4cont}%
We use example 4 from subsection \use{sec:example4} to illustrate the following algorithm for computing an
equilibrium in an economy with sequential trading of a complete set of Arrow securities:

\medskip
\item {1.}  Compute an equilibrium of the Arrow-Debreu economy with time $0$ trading.
\medskip
\item {2.} Set the equilibrium allocation for the sequential trading economy to  the equilibrium allocation from Arrow-Debreu
time $0$ trading economy.
\medskip
\item {3.} Compute equilibrium prices from  formula \Ep{eq14} for a Markov economy
or the corresponding formula \Ep{eq17} for a non-Markov economy.
\medskip
\item{4.} Compute debt limits from \Ep{borrowingconA}.
\medskip
\item {5.}  Compute portfolios of one-period Arrow securities  by first computing implied time $t$, history $s^t$ wealth
$\Upsilon^i_t(s^t) $ from \Ep{cecm_wealth} evaluated at the Arrow-Debreu equilibrium prices,
then set $a_t^i(s_t) = \Upsilon^i_t(s^t)$.

\medskip

Applying this procedure to example 4 from section \use{sec:example4} gives us
the price system $Q_0(s_1=1| s_0=1) =\beta, Q_0(s_1=0|s_0=1) =0, Q_1(s_2=1|s_1=1) = .5 \beta, Q_1(s_2=0|s_1=0) = .5 \beta$  and
$Q_t(s_{t+1} = 1| s_t =1) = Q_t(s_{t+1} = 0| s_t =0) = \beta $ for $t \geq 2$.
Also, $\Upsilon^i_t(s^t) = 0$ for $i=1,2$ and $t=0,1$.  For $t \geq 2$,
$\Upsilon_t^1 (s_t =1)= \sum_{\tau \geq t} \beta^{\tau - t}[.5 -1] = {-.5 \over 1-\beta}$ and
$\Upsilon_t^2(s_t=1) = \sum_{\tau \geq t} \beta^{\tau - t}[.5 -0] = {.5 \over 1-\beta}$.
Therefore, in period $1$, the first  consumer trades Arrow securities in amounts $a_2^1(s_2=1) = {-.5 \over 1-\beta},
a_2^1(s_2 = 0) = {.5 \over 1-\beta}$, while the second consumer trades  Arrow securities in amounts $a_2^1(s_2=1) = {.5 \over 1-\beta},
a_2^1(s_2 = 0) = {-.5 \over 1-\beta}$   After period $2$, the consumers perpetually roll over their debts or assets of either
${.5 \over 1-\beta}$ or ${-.5 \over 1 - \beta}$.


\section{$j$-step pricing kernel}\label{sec:multi-step}%
We are sometimes interested in the price at time $t$ of a claim to
one unit of consumption at date $\tau >t $ contingent on the
time $\tau$ state being $s_\tau$, {\it regardless\/} of the
particular history by which $s_\tau$ is reached at $\tau$.  We let
$Q_j(s' | s)$ denote the $j$-step pricing kernel to be interpreted
as follows: $Q_j(s' | s)$ gives the price of one unit of
consumption $j$ periods ahead, contingent on the state in that
future period being $s'$, given that the current state is $s$. For
example, $j=1$ corresponds to the one-step pricing kernel $Q(s' |
s)$.


With markets in all possible $j$-step-ahead contingent claims,
the counterpart to constraint \Ep{eq10}, the household's budget
constraint at time $t$, is
$$ c_t^i + \sum_{j=1}^\infty \sum_{ s_{t+j}} Q_j( s_{t+j} | s_t)
   z^i_{t,j}( s_{t+j}) \leq  y^i(s_t) + a_t^i. \EQN cecm_eq10   $$
Here $z^i_{t,j}( s_{t+j})$ is household $i$'s holdings at
the end of period $t$ of contingent claims that pay one unit of
the consumption good $j$ periods ahead at date $t+j$, contingent on
the state at date $t+j$ being $ s_{t+j}$. The household's
wealth in the next period  depends on the chosen asset portfolio
and the realization of $s_{t+1}$,
$$
a^i_{t+1}(s_{t+1}) = z^i_{t,1}(s_{t+1})
   + \sum_{j=2}^\infty \sum_{ s_{t+j}}  Q_{j-1}( s_{t+j} | s_{t+1})
           z^i_{t,j}( s_{t+j}) .
$$
The realization of $s_{t+1}$ determines which element of the vector of
one-period-ahead claims $\{z^i_{t,1}( s_{t+1})\}$  pays off at
time $t+1$,
and also the capital gains and losses inflicted on the holdings of longer horizon
claims implied by equilibrium prices $Q_j( s_{t+j+1} | s_{t+1})$.


With respect to $z^i_{t,j}( s_{t+j})$ for $j>1$,
use the first-order condition for the problem on the right of \Ep{eq11}
and the Benveniste-Scheinkman formula and rearrange to get
$$ Q_j( s_{t+j} | s_t ) = \sum_{s_{t+1}}
   {\beta u'_i[c^i_{t+1}(s_{t+1})] \pi(s_{t+1} | s_t) \over u'_i(c_t^i) }
   Q_{j-1}( s_{t+j} | s_{t+1} ).             \EQN Qrecurs $$
This expression, evaluated at the competitive equilibrium
consumption allocation, characterizes two adjacent pricing
kernels.\NFootnote{According to expression \Ep{Markovc}, the
equilibrium consumption allocation is not history dependent, so
that $(c_t^i, \{c^i_{t+1}(s_{t+1})\}_{s_{t+1}})= (\bar c^i(s_t),
\{\bar c^i(s_{t+1})\}_{s_{t+1}})$. Because marginal conditions
hold for all households, the characterization of pricing kernels
in \Ep{Qrecurs} holds for any $i$.} Together with first-order
condition \Ep{eq14}, formula \Ep{Qrecurs}  implies that the
kernels $Q_j, j = 2, 3, \ldots,$ can be computed recursively:
$$ Q_j( s_{t+j} | s_t) = \sum_{s_{t+1}} Q_1(s_{t+1} | s_t)
                      Q_{j-1}( s_{t+j} | s_{t+1}). \EQN Qrecurs2 $$




\subsection{Arbitrage-free pricing}
It is useful briefly to describe how arbitrage free pricing theory
\index{arbitrage!no-arbitrage principle}%
 deduces restrictions on asset
prices by manipulating  budget sets with redundant
assets.
%\NFootnote{Arbitrage pricing theory was advocated by
%Stephen Ross (1976).} \auth{Ross, Stephen}
We now present an
arbitrage argument as an alternative way of deriving restriction
\Ep{Qrecurs2} that was established above by using households'
first-order conditions evaluated at the equilibrium consumption
allocation. In addition to $j$-step-ahead contingent claims, we
illustrate the arbitrage-free pricing theory by augmenting the trading
opportunities in our Arrow securities economy by letting the
consumer also trade  an ex-dividend Lucas tree. Because markets
are already complete, these additional assets are redundant. They
have to be priced in a way that leaves the budget set
unaltered.\NFootnote{That the additional assets are redundant
follows from the fact that trading Arrow securities is sufficient
to complete markets.}


  Assume that at time $t$,  in addition to purchasing
a quantity $z_{t,j}(s_{t+j})$   of $j$-step-ahead claims paying
one unit of consumption at time $t+j$ if the state takes value
$s_{t+j}$ at time $t+j$, the consumer also purchases  $N_{t}$
units of  a stock or  Lucas tree. Let the ex-dividend price of the
tree at time $t$  \index{arbitrage}%
 be $p(s_t)$.  Next period, the
tree pays a dividend $d(s_{t+1})$ depending on the
 state $s_{t+1}$. Ownership of the $N_{t}$ units
of the tree  at the beginning of $t+1$ entitles the
consumer to a claim on  $N_{t}[p(s_{t+1}) + d(s_{t+1})]$
 units of time $t+1$ consumption.\NFootnote{We calculate the price of this asset
using a different method in
chapter \use{assetpricing1}.}
As before, let $a_t$ be the wealth  of the consumer, apart from
his endowment, $y(s_t)$.  In this setting, the augmented version of
constraint \Ep{cecm_eq10}, the consumer's budget constraint, is
$$c_t + \sum_{j=1}^\infty \sum_{s_{t+j}} Q_j(s_{t+j} | s_t)
      z_{t,j}(s_{t+j}) + p(s_t) N_{t} \leq a_t  + y(s_t)
\EQN arbit1;a $$
and
$$\eqalign{ a_{t+1}(s_{t+1}) & = z_{t,1}(s_{t+1}) + \left[p(s_{t+1})
+ d(s_{t+1})\right] N_{t} \cr
   + & \sum_{j=2}^\infty \sum_{s_{t+j}} Q_{j-1}(s_{t+j} | s_{t+1})
       z_{t,j}(s_{t+j}). \cr} \EQN arbit1;b $$
Multiply equation  \Ep{arbit1;b} by $Q_1(s_{t+1} | s_t)$, sum
over $s_{t+1}$, solve for \hfil\break
 $\sum_{s_{t+1}} Q_1(s_{t+1} | s_t) z_{t,1}(s_t)$,
and substitute this expression in \Ep{arbit1;a} to get
$$\EQNalign{ c_t +& \left\{p(s_t) - \sum_{s_{t+1}} Q_1(s_{t+1} | s_t)
      [ p(s_{t+1}) + d(s_{t+1})]   \right\} N_{t} \cr
      + & \sum_{j=2}^{\infty} \sum_{s_{t+j}} \left\{
     Q_j(s_{t+j} | s_t) - \sum_{s_{t+1}} Q_{j-1}(s_{t+j} | s_{t+1})
      Q_1(s_{t+1} | s_t) \right\} z_{t,j}(s_{t+j}) \hskip1cm \cr
     + & \sum_{s_{t+1}} Q_1(s_{t+1} | s_t ) a_{t+1}(s_{t+1})
                                 \leq a_t + y(s_t). \EQN arbit2\cr}$$
If the two terms in braces are not zero, the consumer can attain
unbounded consumption and future wealth by purchasing or selling
either the stock (if the first term in braces is not zero) or a
state-contingent  claim (if any of the terms in the second set of
braces is not zero).   Therefore, so long as the utility function
has no satiation point, in any equilibrium, the terms in the
braces must be zero. Thus, we have the  arbitrage-free pricing
formulas \index{arbitrage!no-arbitrage principle}%
$$\EQNalign{ p(s_t) &=  \sum_{s_{t+1}} Q_1(s_{t+1} | s_t)
      [ p(s_{t+1}) + d(s_{t+1})],    \EQN arbit3;a \cr
     Q_j(s_{t+j} | s_t) & = \sum_{s_{t+1}} Q_{j-1}(s_{t+j} | s_{t+1})
      Q_1(s_{t+1} | s_t) . \EQN arbit3;b \cr } $$
These are called {\it arbitrage-free pricing formulas\/} because if they were
violated,  there would exist an {\it arbitrage}.  An
\idx{arbitrage} is defined as a risk-free transaction that earns
positive profits.



\section{Term structure of yields on risk-free claims}\label{sec:term_structure_LA}%
Assume that we are in the setting of section \use{sec:Markovs}.
Recall formula \Ep{eq14} for the price $Q_\tau(s_{t+\tau} | s_t)$ of a $\tau$-period ahead Arrow security in Markov state
$s_t$.  At time $t$ in Markov state $s_t$, the price of a risk-free claim on  one unit of consumption at time $t+\tau$ is\NFootnote{The one-period gross {\it return\/} on this risk-free claim on consumption  at time $t+\tau$ is risky and equals
$$ R_\tau(s_{t+1},s_t) = {\frac{p_{\tau-1}(s_{t+1}) - p_\tau(s_t)}{p_\tau(s_t)}} .$$}
$$ p_\tau(s_t) = \sum_{s_{t+\tau} } Q_\tau(s_{t+\tau} | s_t) . $$
At time $t$, the  ``yield to maturity'' or just ``yield'' of a sure claim to one unit of time $t+j$ consumption is
the scalar $\rho_\tau(s_t)$ that satisfies \index{yield!to maturity}
$$ p_\tau(s_t) = \exp( -\tau \rho_\tau(s_t)) $$
or
$$ \rho_\tau(s_t) = - {\frac{\log p_\tau(s_t)}{\tau}}.  \EQN eq:yield $$
The vector  $\rho_\tau(s_t)_{\tau=1}^T$ is  called the ``term structure of interest rates.''
Theories of  the term structure of interest rates to be outlined in chapters \use{assetpricing1} and \use{assetpricing2}
seek to explain the evolution of $\rho_\tau(s_t)_{\tau=1}^T$ over time.

\subsection{Constructing yields}

Often prices $p_\tau(s_t)$ are not observed directly because claims to risk-free promises to goods $t+j$ periods ahead are not traded.  But when
enough distinct {\it bundles\/} of such claims are traded, there are ways to estimate the $p_\tau(s_t)$'s by appropriately unbundling a set of bond prices.
  Here is brief account of how.
Suppose that at time $t$ observations are available on the prices of a number $I$  bonds. The $i$th bond is a claim
to a stream of risk-free claims to $d_{i,t+\tau}$ units of time $t+\tau$ consumption for dates $t+ \tau$ where $\tau = 1, 2, \ldots, T$.
Some components of  $d_{i,t+\tau}$ can be zero.  According to our theory,  the observed price $v_{i,t}$ of  bond $i$ at $t$ satisfies
$$ v_{i,t}(s_t) = \sum_{\tau =1}^T d_{i,t+\tau} p_\tau(s_t) . $$  Represent this equation for our collection of bonds as $i = 1, \ldots , I$
as
$$  \bmatrix{ v_{1,t} \cr v_{1,t+2} \cr \vdots \cr v_{I,t} } =
     \bmatrix{ d_{1,t+1} & d_{1,t+2} & \cdots & d_{1,t+\tau} \cr
      d_{2,t+1} & d_{2,t+2} & \cdots & d_{2,t+\tau} \cr
      \vdots & \vdots & \vdots & \vdots \cr
       d_{I,t+1} & d_{I,t+2} & \cdots & d_{I,t+\tau} }
      \bmatrix{ p_1(s_t) \cr p_2(s_t) \cr \vdots \cr p_I(s_t) }   $$
or
$$ V_t = D_t P_t . $$
If we observe $V_t$ and $D_t$, we can recover the prices $P_t$ by applying an appropriate inverse or generalized inverse to each side of this matrix
equation.
If $I =T$ and $D_t$ is of full rank, we use
$$ P_t = D_t^{-1} V_t ,$$
while if $ I > T$ we use the least squares formula
$$ \hat P_t = (D_t' D_t)^{-1} D_t' V_t \EQN ls101 $$
and if $T > I$ we use the formula
$$ \hat P_t = D_t' (D_t D_t')^{-1} V_t . \EQN ls102 $$
We use  formula \Ep{ls101} when there are too many securities and formula \Ep{ls102}  when there are too few securities  relative to the primitive securities whose prices $P_t$ we want
to infer. After we have constructed $P_t$ or $\hat P_t$, we construct yields from equation \Ep{eq:yield}. \index{inverse!generalized}%
\index{least squares}%







\section{Recursive version of Pareto problem}\label{sec:RecursivePareto}%
At the  outset of this chapter, we characterized Pareto
optimal allocations. By formulating a Pareto problem  recursively, this section
gives a preview of things to come in chapters \use{socialinsurance}
and \use{credible}.
For this purpose, we
consider a special case of the section \use{sec:example2GG} example 2 of an
economy with a constant aggregate
endowment and two types of household with $y_t^1 = s_t$, $y_t^2 = 1-s_t$.
We now assume that the $s_t$ process
is i.i.d., so that $\pi_t(s^t) = \pi(s_t)
\pi(s_{t-1}) \cdots \pi(s_0)$.   Also, let's assume
that $s_t$ has a discrete distribution
so that
$s_t \in [\overline s_1, \ldots, \overline s_S]$
with probabilities $\Pi_i = {\rm Prob}(s_t = \overline s_i)$
where $\overline s_{i+1} > \overline s_i$ and $\overline s_1 \geq 0$
and $\overline s_S \leq 1$.




In our recursive formulation, each period a planner
delivers a pair of previously promised discounted
utility streams by assigning a state-contingent
consumption allocation today and a pair of state-contingent
promised discounted utility streams starting tomorrow.
Both the state-contingent consumption  today and the promised
discounted utility tomorrow are functions of the initial
promised discounted utility levels.


  Define $v$ as  the expected discounted utility of
a type 1 person
and $P(v)$ as the maximal expected discounted utility
that can be offered to a type 2 person, given that
a type 1 person is offered at least $v$. Each of these expected values
is to be evaluated before the realization of the state at the initial
date.

The Pareto problem is to choose stochastic processes $\{c_t^1(s^t), c_t^2(s^t)\}_{t=0}^\infty$ to
 maximize
$P(v)$ subject to  $c_t^1(s^t) + c_t^2(s^t) =1$ and the utility constraint
$\sum_{t=0}^\infty \sum_{s^t} \beta^t u_1(c_t^1(s^t)) \pi_t(s^t) \geq v.$
In terms of the competitive equilibrium
allocation calculated for  the section \use{sec:example2GG} example 2  economy above,
let $\overline c= \overline c^1$ be the  constant consumption
allocated to a type 1 person and $1-\overline
c=\overline c^2$ be the constant consumption
allocated to a type 2 person.
Since we have shown that the competitive equilibrium allocation is
a Pareto optimal allocation, we already know one point on the
Pareto frontier $P(v)$. In particular, when a type 1 person is promised
$v= u_1(\overline c) / (1 - \beta) $, a type 2 person attains
life-time utility
$P(v) =  u_2(1-\overline c) / (1-\beta)$.
%%%Evidently, $v= {u(\overline c) \over 1 - \beta} $,
%%%$P(v) =  {u(1-\overline c)\over 1-\beta}$, and
%%%$P'(v) = -{u'(1-\overline c) \over u'(\overline c)}$.

We can express the discounted values $v$  and $P(v)$ recursively\NFootnote{This is our first example
of a `dynamic program squared'.  We call it that because the state variable $v$ that appears in the Bellman equation
for $P(v)$ itself satisfies another Bellman equation.}
as
$$  v = \sum_{s=1}^S \left[ u_1(c_s)   +\beta w_s \right]  \Pi_i $$
and
$$ P(v) = \sum_{s=1}^S \left[ u_2(1- c_s) + \beta P(w_s) \right] \Pi_s, $$
where $c_s$ is consumption of the type 1 person in state $s$,
$w_s$ is the continuation value assigned to the type
$1$ person in state $s$; and $1-c_s$ and $P(w_s)$ are the consumption
and the continuation value, respectively, assigned to a type 2 person in state $s$.
Assume that  the continuation values $w_s \in V$, where $V$ is a set
of admissible discounted values of utility.   In this section,
 we assume
that $V = [u_1(\epsilon) / (1-\beta),\,  u_1(1) / (1 -\beta)]$ where
$\epsilon\in(0,1)$ is an arbitrarily small number.
%%%$V = [0, {u(1) \over 1 -\beta}]$. In effect, a consumption
%%%plan offers a household a state-contingent consumption vector in
%%%period $i$ and  a state-contingent vector of continuation values
%%%$w_i$ in state $i$, with each $w_i$ itself being a present value
%%%of one-period future utilities.

In effect, before the realization of the current state, a Pareto optimal
allocation offers the type 1 person a state-contingent vector of
consumption $c_s$ in state $s$ and
a state-contingent vector of continuation values
$w_s$ in state $s$, with each $w_s$ itself being a present value
of one-period future utilities.
In terms of the pair of values
$(v, P(v))$,
 we can express the Pareto problem recursively as
$$ P(v) = \max_{\{c_s, w_s \}_{s=1}^S }
  \sum_{s=1}^S [ u_2(1-c_s) + \beta P(w_s)] \Pi_s  \EQN paret1 $$
where the maximization is subject to
$$ \sum_{s=1}^S [u_1(c_s) + \beta w_s ] \Pi_s \geq v \EQN paret2 $$
where $c_s \in [0,1]$ and $w_s \in V$.


To solve the Pareto problem, form the
Lagrangian
$$ L = \sum_{s=1}^S \Pi_s [u_2(1-c_s) + \beta P(w_s)
   + \theta (u_1(c_s) + \beta w_s) ] - \theta v $$
where $\theta $ is a Lagrange multiplier on constraint \Ep{paret2}.
First-order conditions
with respect to $c_s$ and $w_s$, respectively, are
$$\EQNalign{ - u'_2(1-c_s) + \theta u'_1(c_s) & = 0, \EQN paret3;a \cr
        P'(w_s) + \theta & = 0. \EQN paret3;b \cr} $$
The envelope condition is
$ P'(v) = -\theta$.  Thus, \Ep{paret3;b} becomes
$  P'(w_s) = P'(v)   $.  But $P(v)$ happens to be strictly concave,
so  this equality implies
$ w_s = v$.  Therefore,  any solution of the Pareto problem leaves
the continuation value $w_s$ independent  of  the state $s$.
Equation
\Ep{paret3;a} implies that
$${u'_2(1-c_s) \over u'_1(c_s)} = -P' (v) .\EQN paret4 $$
Since the right side of \Ep{paret4} is independent of $s$, so is
the left side, and therefore $c$ is independent of $s$.
  And since $v$ is constant over time (because $w_s = v$ for all $s$),
it follows that $c$ is constant over time.


 Notice from
\Ep{paret4} that $P'(v)$ serves as a   relative Pareto weight on
the type 1 person. The recursive formulation brings out that,
because $P'(w_s) = P'(v)$,
 the relative  Pareto weight remains constant over time and is independent
of the  realization of $s_t$.    The planner imposes
complete risk sharing.


In chapter \use{socialinsurance}, we shall encounter recursive formulations
again. Impediments to risk sharing that occur in the form
either of enforcement or of information constraints will
impel the planner sometimes to make continuation values respond
to the current realization of shocks to endowments or preferences.




\section{Concluding remarks}
  The  framework in this chapter serves much of  macroeconomics
either
as foundation or straw man (``benchmark'' is a kinder phrase than
``straw man'').
   It is the foundation of
extensive literatures on asset pricing and risk sharing.
We describe the literature on asset pricing in more
detail in chapters \use{assetpricing1} and \use{assetpricing2}.
The model also serves as benchmark, or point of departure,
for a variety of models designed to confront observations
that seem inconsistent with complete markets. In particular,
for models   with exogenously imposed   incomplete markets,
see chapters \use{selfinsure} on precautionary saving and
\use{incomplete} on  incomplete markets. For
models with endogenous incomplete markets,
see chapters \use{socialinsurance} and \use{socialinsurance2} on
enforcement and information problems. For
models of money, see
chapters \use{fiscalmonetary} and \use{townsend}.
 To take monetary theory as an example, complete
markets models assign no role to  money because they
contain an efficient multilateral trading mechanism, with such
extensive  netting of claims that no additional asset is
required to facilitate bilateral exchanges.   Any modern model of
money introduces frictions that impede complete markets.     Some
monetary models (e.g., the cash-in-advance model of Lucas, 1981)
impose minimal impediments to complete markets in ways that  preserve many
of the asset-pricing implications of complete markets models while
also activating classical monetary doctrines like the quantity
theory of money. The shopping time model of chapter
\use{fiscalmonetary}
 is constructed in a similar
spirit.  Other monetary  models, such as the Townsend turnpike
model of chapter \use{townsend} or the Kiyotaki-Wright search
model of chapter \use{search2}, impose more  extensive frictions
on multilateral exchanges and leave the complete markets model
farther behind.
  Before leaving the complete markets model, we'll put it
to work in several of the following chapters.
% \use{ogmodels}, \use{ricardian}, and \use{assetpricing}.
\auth{Lucas, Robert E., Jr.} \auth{Kiyotaki,
Nobuhiro} \auth{Wright, Randall} \auth{Townsend, Robert M.}

%
%
%\auth{Alvarez, Fernando} \auth{Jermann, Urban} \auth{Lustig,
%Hanno}\auth{Hansen, Lars P.}\auth{Tallarini, Thomas}
%%\vfil\eject
%\appendix{A}{Consumption strips and the cost of business cycles}
%%\subsection{Consumption strips}
%  This appendix briefly describes  the ideas of
%Alvarez and Jermann (2003); Lustig (2000); Hansen, Sargent, and Tallarini (1999);
%and Tallarini (2000).   Their purpose
%is to   link measures of the cost of business cycles with
%a risk premium for some assets.
%To this end, consider an endowment economy with a representative
%consumer  endowed with a consumption process
%$c_t = c(s_t)$, where $s_t$ is  Markov with transition
%probabilities $\pi(s'|s)$.      Alvarez and Jermann
%define a one-period  consumption strip as a claim to the random
%payoff $c_t$, sold  at date $t-1$. The price in terms of
%time $t-1$ consumption of this one-period consumption strip is
%$$ a_{t-1} = E_{t-1} m_t c_t ,\EQN alvjer1 $$
%where  $m_t$ is the one-period stochastic discount factor
%$$ m_t = {\beta u'(c_t) \over u'(c_{t-1})} . \EQN alvjer2 $$
%Using the definition of a conditional covariance,
%equation \Ep{alvjer1} implies
%$$ a_{t-1} = E_{t-1} m_t E_{t-1} c_t + {\rm cov}_{t-1} (c_t, m_t),
%          \EQN alvjer3 $$
%where ${\rm cov}_{t-1} (c_t, m_t) <0$.
%Note that the price of  a one-period claim on $E_{t-1} c_t $ is simply
%$$ \tilde a_{t-1} = E_{t-1} m_t E_{t-1} c_t, \EQN  alvjer4 $$
%so that the negative covariance in equation \Ep{alvjer3} is a discount
%due to risk in the price  of the risky claim on $c_t$ relative
%to the risk-free claim on a payout with the same mean.
%Define the multiplicative risk premium on the consumption strip
%as  $(1+\mu_{t-1}) \equiv \tilde a_t / a_t $, which evidently
%equals
% $$  1 + \mu_{t-1} = {E_{t-1} m_t E_{t-1} c_t \over
%                      E_{t-1} m_t c_t }. \EQN alvjer5 $$
%
%
%
%\auth{Alvarez, Fernando} \auth{Jermann, Urban} \auth{Lustig,
%Hanno}
%\subsection{Link to business cycle costs}
%The cost of business cycle as defined in chapter \use{practical}
%does not link immediately to an asset-pricing calculation because
%it is inframarginal.   Alvarez and Jermann (2003) and Hansen,
%Sargent, and Tallarini (1999) were interested in coaxing attitudes
%about \auth{Tallarini, Thomas}\auth{Hansen, Lars P.}%
% the cost of
%business  cycles from asset prices. Alvarez and Jermann designed a
%notion of the  marginal   costs of business cycles to match asset
%pricing. With the timing conventions of Lustig (2000),  their
%concept of marginal cost corresponds to the  risk premium in
%one-period consumption strips.
%
%
%  Alvarez and Jermann (2003) and Lustig (2000) define
%the total costs of business cycles in terms of a
%stochastic process of adjustments to consumption $\Omega_{t-1}$
%constructed to satisfy
%$$ E_0 \sum_{t=0}^\infty \beta^t u[(1+\Omega_{t-1})c_t ]
%     = E_0 \sum_{t=0}^\infty \beta^t u(E_{t-1} c_t) .$$
%The idea is to compensate the consumer for the
%one-period-ahead risk in consumption that he faces.
%
%
%The time $t$ component of the marginal
% cost of business cycles  is defined as follows through
%a variational argument, taking the endowment as a benchmark.
%Let $\alpha \in (0,1)$ be a parameter to index consumption
%processes.  Define
%$\Omega_{t-1}(\alpha)$ implicitly by means of
%$$ E_{t-1} u\{[1+\Omega_{t-1}(\alpha)]c_t\} =
%    E_{t-1} u[\alpha E_{t-1} c_t + (1-\alpha) c_t].  \EQN alvjer6 $$
%Differentiate equation \Ep{alvjer6} with respect to $\alpha$ and evaluate
%at $\alpha=0$ to get
%$$ \Omega'_{t-1}(0) = {E_{t-1} u'(c_t) (E_{t-1} c_t - c_{t})
%                \over E_{t-1} c_t u'(c_t) }  .$$
%Multiply both numerator and denominator of the right side
%by $\beta / u'(c_{t-1})$ to get
%$$ \Omega'_{t-1}(0) = {E_{t-1} m_t (E_{t-1} c_t - c_t )
%                     \over E_{t-1} m_t c_t } ,\EQN alvjer7 $$
%where we use $\Omega_{t-1}(0)=0$.
%Rearranging gives
%$$ 1 + \Omega'_{t-1}(0)  = {E_{t-1} m_t E_{t-1} c_t \over
%                            E_{t-1} m_t c_t }. \EQN alvjer8 $$
%Comparing equation \Ep{alvjer8} with \Ep{alvjer5} shows that
%the marginal cost of business cycles equals the multiplicative
%risk premium on the one-period consumption strip. Thus,
%in this economy, the marginal cost of business cycles
%can be coaxed from asset market data.


\vfill\eject

\noindent
{\bf Appendices: Departures from key assumptions}

\medskip

\noindent
The  finding that the equilibrium consumption of a household
depends in a time-invariant way on the aggregate endowment hinges
 on this chapter's assumptions that there are  complete markets and  that
consumers  share the same  discount factor and the same probabilities.
In appendix \use{app:compAA}, we show that when households have heterogeneous
discount factors,  equilibrium allocations eventually entail  efficient
 immiserization: at equilibrium prices,  less patient
households choose to consume almost nothing
as time approaches infinity.
In appendix \use{app:compB}, we show that when households have heterogeneous beliefs,  as  time passes  equilibrium allocations assign   less
and less  to those
with less accurate beliefs.
Consumers' asset choices accompany these outcomes in ways that reflect their disparate probability assessments.
Appendix \use{sec:Beker} shows that if markets are incomplete, there no longer prevails an inextricable
link between  either the degree of patience or the
accuracy of beliefs and  which consumers continue to consume  asymptotically. Instead, many things can happen.



\appendix{A}{Heterogenous discounting\label{app:compAA}}

We now modify \Ep{eq0} to allow for heterogenous discount
factors.  Let $\beta_i$ denote
the subjective discount factor of household $i$, where
$1 > \beta_1\geq \beta_2 \ldots \geq \beta_I > 0$.
Counterparts to first-order conditions \Ep{foncPareto101} for the
Pareto problem become
$$ \left[ \beta_i \over \beta_1 \right]^t
{u_i^\prime(c^i_t(s^t)) \over u_1^\prime(c^1_t(s^t))}
                   = {\lambda_1  \over \lambda_i}.  \EQN discount_0 $$
These imply that ratios of marginal utilities in an efficient allocation are no longer time invariant.
 Specifically, let $\lambda^t_i$ be the
Pareto weight attached to consumer $i$'s utility in a
remainder Pareto problem at date $t$. It is instructive to
normalize  the weight assigned to the
most patient household at the value assigned in the initial time $t=0$
Pareto problem, i.e., $\lambda^t_1 = \lambda_1$ for all $t\geq0$.
From expression \Ep{discount_0} we can then infer that the continuation
Pareto weight on household $i$'s utility satisfies
$$
\lambda^t_i =  \left[ \beta_i \over \beta_1 \right]^t \lambda_i,
$$
where $\lambda_i$ is the weight on household $i$'s utility in the
initial Pareto problem.
%Since  $\lambda^t_i$ for all $i  \neq 1$,  converges to
%zero, the most patient $i=1$ agent  eventually consumes everything.
For a consumer $i$ with  $\beta_i < \beta_1$, $\lambda^t_i$ converges to
zero.  The most patient type of household eventually
consumes everything.


\appendix{B}{Heterogenous beliefs \label{app:compB}}
\auth{Debreu, Gerard}%
 Instead of our highly restrictive specification  % time-separable, discounted expected utility preferences
 \Ep{eq0}, namely,
 $$U_i(c^i) =
   \sum_{t=0}^\infty \sum_{s^t} \beta^t u_i[c_t^i(s^t)]
   \pi_t(s^t),$$
% $ U^i(c^i) = \sum_{t=0}^\infty \sum_{s^t} \beta^t u(c^i(s^t)) \pi_t(s^t)$,
we could follow
  Debreu (1954, 1959)  and impute  more  general preferences to
consumer $i$.    We %Debreu
could just posit $I$  distinct   preference orderings   $U_i(c^i)$.
% over consumption plans $\{c_t^i(s^t)\}_{t\geq 0}$. instead of the special
% time-separable, discounted expected utility form of \Ep{eq0}.
The functions
$U_i(c^i)$ could   directly specify attitudes about  tradeoffs among consumption goods at different times and histories.
There is no need to say anything directly about how likely the $s^\infty$ paths are from  anyone's  point of view, including the planner's.
To the Pareto planner, all that matters are  the preferences $U_i(c^i)$ of the $I$ consumers and the aggregate resources available at different dates and histories.
The outcome of a Pareto problem is an allocation  across people
for all times and all histories.  %In the broader construction of Debreu,
The theory   can be silent about probabilities over histories $\{s^\infty\}$.



\index{rational expectations}%
We break that silence %about   which
%history $\{s^\infty\}$ is likely  actually to  occur
when we assign  probabilities $\pi_t(s^t)$ over histories $s^t$ to someone either inside the model or outside the model.
In the body of this chapter, we  assigned the same probabilities to all agents inside the model, to an outside observer (``the economist''),
and presumably to ``nature''.  In particular, when we  adopted our restrictive expected utility specification \Ep{eq0}, we
 imputed a common probability specification $\{\pi_t(s^t)\}$   to all consumers $i=1, \ldots, I$.  Beyond that,
 at various points in the body of the  chapter and  in  chapters \use{assetpricing1} and \use{assetpricing2} about asset pricing,
 we   go on to impute the same common probability model shared by agents inside the model  both to outside economists and to   `nature', alternatively known as the
actual history-generating mechanism.  Imputing common beliefs to everyone inside a model, to nature,
and to the author of the model too, is to embrace the  ``rational expectations hypothesis.''
\index{rational expectations!communism}%
\index{communism!rational expectations}%
Widely used in macroeconomics, finance, and public finance, this ``communism of probability models'' assumption imposes substantially more structure than did Debreu.
%By doing that, the rational expectations assumption acquires empirical power.
By equating probability distributions
 across all agents inside a model,  nature, and  an outside econometrician, the rational expectations hypothesis eliminates parameters that would
 be required to describe heterogeneous beliefs. In doing this it combines and confounds  potentially distinct roles that actual and  perceived probabilities  $\pi_t(s^t)$ play in influencing not only actual outcomes but also   individuals'
preferences, and through them, Pareto optimal allocations and competitive equilibrium prices.
To indicate some of these conceptually distinct roles, it is useful temporarily to abandon the assumption of common beliefs and to follow
Blume and Easley (2006) by considering
a setting with disparate beliefs.  This will  allow us to evaluate  Milton Friedman's (1953, p. 22) conjecture
that, by redistributing wealth towards people with more accurate beliefs, market forces eventually make outcomes resemble
 a rational expectations equilibrium. \index{rational expectations}%


Thus, suppose that we modify  the utility functional \Ep{eq0} for consumer $i$
to allow for different probabilities
$$ U_i(c^i) = \sum_{t=0}^\infty \sum_{s^t} \beta^t u_i(c^i_t(s^t))
\pi_t^i(s^t) .  \EQN heterogbel1 $$
With preferences altered in this way, assume that the planner
maximizes the same welfare function $W$ defined in \Ep{planner1}.    Then  counterparts to first-order conditions \Ep{foncPareto101} become
$$ {\frac{\pi_t^i(s^t)}{\pi_t^1(s^t)}} \;
{\frac{u_i^\prime(c^i_t(s^t))}{u_1^\prime(c^1_t(s^t))}}
 = {\frac{\lambda_1}{\lambda_i}} \; .
                                                    \EQN heterogbel2 $$
The likelihood ratios ${\frac{\pi_t^i(s^t)}{\pi_t^1(s^t)}}$, which are identically unity when there are common beliefs, now influence optimal allocations.
For a given vector of time $0$  Pareto weights $\{\lambda_i\}_{i=1}^I$, the Pareto planner assigns relatively more consumption to
people who place higher probabilities on a date-history pair.


It is useful once again to let $\lambda^t_i(s^t)$ be the
Pareto weight attached to consumer $i$'s utility in a
remainder or continuation Pareto problem as of date $t$, history $s^t$.  A continuation Pareto
planner chooses a continuation allocation
$\{c_{\tau}^i(s^\tau);$ for all $\tau\geq t,s^\tau | s^t\}$,
$i = 1, \ldots , I$  to maximize
$$ W_t(s^t) = \sum_i \lambda^t_i(s^t)
\sum_{\tau = t}^\infty \sum_{s^\tau | s^t}
\beta^t u_i(c_{\tau}^i(s^\tau)) \pi_{\tau}^i(s^\tau | s^t) .$$
Under heterogenous
beliefs, the allocations chosen by every  continuation  planner  equal  continuations of the  original allocation provided that
 continuation Pareto weights satisfy
$$
\lambda^t_i(s^t) =  \left[ \pi_t^i(s^t) \over \pi_t^1(s^t) \right]
                    \lambda_i.
$$

\index{Friedman, Milton}%

\subsection{Example: one type's beliefs are closer to the truth}
\label{sec:heterogbel_example}%

\noindent
The  Pareto problem above does not involve a {\it true} probability specification. But a true
stochastic process must be specified if we want to compute
an {\it actual} distribution of consumption governed by
\Ep{heterogbel2}.
We now describe an  example that illustrates the principle that
 the consumer whose probability specification is closest to nature's
will eventually consume the entire aggregate endowment.

 Suppose that $s_t$ is
truly independent and identically distributed and that
${\rm Prob}(s_t = s) = \pi(s)>0$ for all $s\in S$,
and $\sum_s \pi(s)=1$.
Consumer $i$ also believes that $s_t$ is
independent and identically distributed but with
subjective probabilities ${\rm Prob}(s_t = s) = \pi^i(s)>0$
for all $s\in S$, and $\sum_s \pi^i(s)=1$.
The \idx{entropy} of the actual distribution $\pi$
relative to consumer $i$'s distribution $\pi^i$ is
$$ {\rm ent}(\pi, \pi^i) = \sum_s \pi(s)
\log \left({\frac{\pi(s)}{\pi^i(s)}}\right).
$$
Relative entropy is a nonnegative function that attains
a minimum value of 0 when $\pi^i=\pi$.\NFootnote{To show
that relative entropy is a nonnegative function, note
that the negative of ${\rm ent}(\pi, \pi^i)$ is bounded
from above by zero:
$$ -{\rm ent}(\pi, \pi^i) = \sum_s \pi(s) \log
\left({\frac{\pi^i(s)}{\pi(s)}}\right)
\leq \log \left(\sum_s \pi(s) {\frac{\pi^i(s)}{\pi(s)}}\right)
= \;\log \left(\sum_s \pi^i(s)\right) =0,
$$
where the weak inequality follows from Jensen's inequality
and concavity of the natural log function.
Equality holds for $\pi^i=\pi$, and by strict concavity
of the  log function, ${\rm ent}(\pi, \pi^i) >0$
for all $\pi^i \not= \pi$.}
Relative entropy indicates how far one probability distribution
is from another.\NFootnote{Relative entropy also governs statistical properties of likelihood ratio tests for
discriminating one statistical model from another in large samples.} \index{entropy}%

Let the beliefs
of consumer 1 be closer to the truth than those of
other consumers, as measured by  \idx{relative entropy}, so that
${\rm ent}(\pi, \pi^i) > {\rm ent}(\pi, \pi^1) \geq 0$ for all
$i>1$.
Under our assumption of i.i.d.\ uncertainty, first-order
conditions \Ep{heterogbel2} can be written
$$ {\frac{u_i^\prime(c^i_t(s^t))}{u_1^\prime(c^1_t(s^t))}}
 = {\frac{\lambda_1}{\lambda_i}} \;
   {\frac{\prod_{\tau=0}^t \pi^1(s_{\tau})}
        {\prod_{\tau=0}^t \pi^i(s_{\tau})}}
 = {\frac{\lambda_1}{\lambda_i}} \;
{\frac{\prod_{\tau=0}^t {\frac{\displaystyle \pi(s_{\tau})}
                            {\displaystyle \pi^i(s_{\tau})}}}
     {\prod_{\tau=0}^t {\frac{\displaystyle \pi(s_{\tau})}
                            {\displaystyle \pi^1(s_{\tau})}}}}\,,
                                                    \EQN heterogbel10
$$
where the second equality is obtained after multiplying and dividing
by $\pi_t(s^t)=\prod_{\tau=0}^t \pi(s_{\tau})$. After taking
logarithms and then dividing by $t$, expression \Ep{heterogbel10}
becomes
$$ {\frac{1}{t}}\,
\log\left[{\frac{u_i^\prime(c^i_t(s^t))}{u_1^\prime(c^1_t(s^t))}}\right]
 = {\frac{1}{t}}\, \log\left({\frac{\lambda_1}{\lambda_i}}\right) \; +
{\frac{1}{t}}\sum_{\tau=0}^t
\left[\log\left({\frac{\pi(s_{\tau})}{\pi^i(s_{\tau})}}\right)
- \log\left({\frac{\pi(s_{\tau})}{\pi^1(s_{\tau})}}\right)\right].
$$
Since the log likelihood ratios
$\log\left({\frac{\pi(s_{\tau})}{\pi^i(s_{\tau})}}\right)$ are
independent and identically distributed random variables, by
the law of large numbers
$$ {\frac{1}{t}}\sum_{\tau=0}^t
\log\left({\frac{\pi(s_{\tau})}{\pi^i(s_{\tau})}}\right)
\rightarrow
\sum_s \pi(s) \log \left({\frac{\pi(s)}{\pi^i(s)}}\right)
= {\rm ent}(\pi, \pi^i).
$$
Consequently,
$$ {\frac{1}{t}}\,
\log\left[{\frac{u_i^\prime(c^i_t(s^t))}{u_1^\prime(c^1_t(s^t))}}\right]
\rightarrow
{\rm ent}(\pi, \pi^i) - {\rm ent}(\pi, \pi^1) >0,
                                                    \EQN heterogbel11
$$
so the strictly positive right side implies that the ratio of marginal
utilities diverges to $+\infty$. Since the marginal utility in the
denominator is bounded below because of the finite aggregate endowment,
we conclude that the marginal utility of consumer $i$ goes to
infinity and hence $\lim_t c^i_t(s^t)\rightarrow 0$.

This example is a special case of
Blume and Easley's (2006) more general proposition that in the limit as $t \rightarrow +\infty$, the  consumer with beliefs closest to the truth, as measured by relative entropy, receives the entire allocation.
\auth{Blume, Lawrence}%
\auth{Easley, David}%









%%%%%%%%%%%%%%%%%%%%%%%%%%%%%%%%%%%%%%%%%%%%%%%%%%%%%%

%\subsection{Example 1: one type knows the truth}
%\label{sec:heterogbel_example1}%

%\noindent
%This example illustrates that if one person's probability specification matches nature's while everyone else's doesn't, then  asymptotically
%the ratio of the knowledgeable person's consumption to that of a less knowledgeable person  diverges to $+\infty$.


%Suppose  that $S = \{0, 1\},  I=2$, and that  agent $i$ believes that $s_t$ %is independently and identically distributed
%with ${\rm Prob}(s_t = 0) = \alpha_i$.  Then for a history  $s_t, s_{t-1}, %\ldots, s_1$ with $s_\tau = 1$ $t-n$ times and
%$s_\tau = 0$ $n$ times, we have
%$$ L_t^2 (s^t) \equiv {\frac{\pi_t^2(s^t)}{\pi_t^1(s^t)}}  = \left({\frac{1-\alpha_2}{1-\alpha_1}}\right)^{t-n}
%\left({\frac{\alpha_2}{\alpha_1}}\right)^n . \EQN LRatexam1 $$
%The  likelihood ratio  $L_t^i (s^t)$ is a random variable.
%Suppose that $\{s_t\}_{t=1}^\infty$ is truly independently and identically %distributed with ${\rm Prob}(s_t = 0) = \tilde \alpha$.  To compute
%the {\it actual} distribution of consumption governed by \Ep{heterogbel2}, %we  want  the likelihood ratio \Ep{LRatexam1} under the true distribution.
%For the empirical  distribution, ${\frac{n}{t}}$ is the fraction of $0$'s in a sample of length $t$.  A Law of Large Numbers asserts that
%$$ \lim_{t \rightarrow \infty} {\frac{n}{t}} = \tilde \alpha .$$
%So
%$$ \lim_{t\rightarrow \infty} L_t^2 = \lim_{t \rightarrow \infty}\left({\frac{1-\alpha_2}{1-\alpha_1}}\right)^{(1-\tilde \alpha) t}
%\left({\frac{\alpha_2}{\alpha_1}}\right)^{\tilde \alpha t} .$$


%Now assume that $\alpha_2 = \tilde \alpha, \alpha_1 \neq \tilde \alpha$, so %that consumer 2 knows the true distribution, while consumer 1 does not.
%Then for large $t$ we have the good approximation
%$$ L_t^2 = \left({\frac{1-\tilde \alpha}{1-\alpha_1}}\right)^{(1-\tilde \alpha) t}
%\left({\frac{\tilde \alpha}{\alpha_1}}\right)^{\tilde \alpha t} , $$
%which implies that
%$$ {\frac{1}{t}} \log L_t^2 = (1-\tilde \alpha) \log \left( {\frac{1-\tilde %\alpha}{1 - \alpha_1}} \right)
%  + \tilde \alpha \log \left({\frac{\tilde \alpha}{\alpha_1}}\right)  . \EQN LRatexam2$$
%The right side of \Ep{LRatexam2} is the expected log likelihood ratio evaluated under the true distribution. This object is
%the \idx{entropy} of the actual distribution parameterized by $\tilde \alpha$ relative to consumer
%1's distribution, call it ${\rm ent}(\tilde \alpha, \alpha_1)$.  Given $\tilde \alpha \in [0,1]$, \idx{relative entropy} is a nonnegative function
%of $\alpha_1$ on the interval $\alpha_1 \in [0,1]$; it attains its minimum %value of $0$ at the value $\alpha_1 = \tilde \alpha$. Relative
%entropy indicates how far one probability distribution is from another.\NFootnote{Relative entropy also governs statistical properties of likelihood ratio tests for discriminating one statistical model from another.} \index{entropy}%

%From equation \Ep{ LRatexam2}, we conclude that
%$$ L_t^2 = e^{{\rm ent}(\tilde \alpha,  \alpha_1) t} , \EQN LRatexam3 $$
%which evidently diverges to $+ \infty$ if  ${\rm ent}(\tilde \alpha,  \alpha_1) >0$, i.e., whenever $\alpha_1 \neq \tilde \alpha$.
%To create an example, for our utility function,  set $\gamma=1$  so that
% $u(c) = \log (c)$.  Then
% $$ c_t^2(s^t) = \left( {\frac{\lambda_2}{\lambda_1}} \right)L_t^2  c_t^1(s^t) , \EQN LRatexam4 $$
%so that when entropy is positive, the ratio of person $2$'s consumption to %person 1's consumption diverges to $+\infty$ so long as
%$ \left( {\frac{\lambda_2}{\lambda_1}} \right) > 0$.

%\subsection{Example 2: one type is closer to the truth}

%This example illustrates that if one consumer's probability specification is wrong but still `better' than those of other consumers, then
% asymptotically
%the ratio of the consumption of the person with the better approximation   %to that of the person with the worse approximation diverges to $+\infty$.

%Modify the preceding example  to assume that $\alpha_2 \neq \tilde \alpha$ %and $\alpha_1 \neq \tilde \alpha$, so that both consumers
%have beliefs that differ from the true data generating process.  Proceeding %as before, we deduce that under the true process
%$$ {\frac{1}{t}} \log L_t^2 =   (1-\tilde \alpha) \log \left( {\frac{1- \alpha_2}{1 - \alpha_1}} \right)
%  + \tilde \alpha \log \left({\frac{ \alpha_2}{\alpha_1}}\right)  $$ which %can be rewritten as
%$$ \eqalign{ {\frac{1}{t}} \log L_t^2 & =   (1- \tilde \alpha) \log \left( %{\frac{1-  \tilde \alpha}{1 -   \alpha_1}} \right)
%  + \tilde \alpha \log \left({\frac{\tilde  \alpha}{ \alpha_1}}\right) \cr
%    & - (1- \tilde \alpha) \log \left( {\frac{1- \tilde \alpha}{1 -  \alpha_2}} \right)
%  - \tilde \alpha \log \left({\frac{ \tilde \alpha}{ \alpha_2}}\right) }  $$
%  or
%$$   {\frac{1}{t}} L_t^2 = {\rm ent}(\tilde \alpha, \alpha_1) - {\rm ent} (\tilde \alpha, \alpha_2) .$$
%It follows that
%$$ L_t^2 = e^{\left[ {\rm ent}(\tilde \alpha, \alpha_1) - {\rm ent} (\tilde %\alpha, \alpha_2) \right] t}, $$
%so that \Ep{LRatexam4} now implies that if ${\rm ent}(\tilde \alpha, \alpha_1) - {\rm ent} (\tilde \alpha, \alpha_2) >0 $, then the ratio of person $2$'s consumption to person 1's diverges to $+\infty$.

%This example is a special case of
%Blume and Easley's (2006) more general proposition that in the limit  those  %consumers with beliefs closest to the truth, as measured by relative entropy, acquire the entire allocation.
%\index{Blume, Lawrence}%
%\auth{Easley, David}%


%%%%%%%%%%%%%%%%%%%%%%%%%%%%%%%%%%%%%%%%%%%%%%%%%%%%%%%%%%%%%%%%%%



\subsection{Equilibrium prices reflect beliefs}

Do competitive equilibrium prices ``accurately reflect available information?''  If ``accurately'' means ``embed correct probability assessments,''
the theory presented in this appendix answers ``no, not at first, but yes asymptotically''.  The ``yes asymptotically'' answer formalizes
Milton Friedman's assertion that competition and survival of the fittest will eventually align the subjective beliefs reflected in competitive equilibrium prices of
risky securities with the objective probabilities that generate the data.


\subsection{Mispricing?}

In our example, what drives the divergence outcome is that the consumer with the less accurate beliefs ``pays too much'' when buying  insurance
and ``accepts too little'' when selling insurance.  The inexorable working of the law of large numbers eventually transfers more and more wealth to  consumers with  more
accurate beliefs.

\subsection{Learning}

While we have presented simple examples in which agents don't learn about probabilities, the same basic force  continues to drive outcomes
when consumers can learn.
Thus, Blume and Easley (2006) presented richer examples with heterogeneous beliefs across agents who
 update  using Bayes' rule.
   Entropy continues to play the same key role in governing  tails of Pareto allocations.
 Blume and Easley's  analysis covers cases  in which the sole source of heterogeneity is that different Bayesian
agents have different priors.  They construct examples in which    agents with either
 a looser  or a less accurate prior receive  equilibrium allocations that approach zero asymptotically. Relative entropies play a key role
 in their analysis.



\subsection{Role of complete markets}
In the body of this chapter, we showed that   Pareto optimal  consumption allocations are  competitive equilibrium allocations
for two alternative trading structures with complete markets, one with  trading of many securities only at time $0$, and another with trading each
period $t \geq 0$ of far fewer one-period securities.  In studying these structures, we maintained the
homogeneous beliefs  of preference specification \Ep{eq0}.  Equivalence of  Pareto optimal allocations to competitive equilibrium allocations
 also applies to the heterogenous beliefs setting of this appendix.

The assertions about limiting allocations that we have made in  this appendix  all come from manipulating  first-order condition \Ep{heterogbel2} for our  Pareto problem. These assertions about  outcomes in complete markets economies don't carry over to
incomplete market economies, for example, of the type to be analyzed in chapter \use{incomplete}.   Indeed, there exist examples of incomplete markets
economies in which the consumption of the consumer with {\it less\/} accurate beliefs grows over time.\NFootnote{See
Blume and Easley (2006) and
Cogley, Sargent, and Tsyrennikov (2013).}

\auth{Cogley, Timothy}%
\auth{Sargent,  Thomas J.}%
\auth{Tsyrennikov, Viktor}%
\auth{Blume, Lawrence}%
\auth{Easley, David}%





\appendix{C}{Incomplete markets\label{sec:Beker}}
Beker and Chattopadhyay (2010) analyze infinite horizon
economies with two consumers, one good, and incomplete markets.
So long as an equilibrium remains effectively constrained
by market incompleteness, Beker and Chattopadhyay prove that either (a) the consumption
of both consumers is arbitrarily close to zero infinitely often,
or (b) the consumption of one consumer
converges to zero. The result prevails  whether or not
beliefs are  heterogeneous. Moreover, a consumer whose consumption eventually
vanishes can be marginally more patient or have  more
accurate beliefs than another consumer whose consumption remains positive.  These outcomes stand in contrast to
those  with complete markets, as illustrated
by an  example to be presented in section \use{sec:Beker_example}.

\auth{Beker, Pablo}%
\auth{Chattopadhyay, Subir}%

To attain an outcome in which both consumers' consumptions remain positive,
Beker and Chattopadhyay  show that it is sufficient to assume
that individuals' endowments are uniformly
positive and governed by a Markov process. Imposition  of a uniform bound on the
value of a consumer's debt prevents a
consumer's consumption from vanishing:  even the most unfortunate run of
low endowments will eventually be interrupted
by  lucky draws of higher endowments that
had not earlier been fully mortgaged to support past
consumption.\NFootnote{Finding that nobody's consumption asymptotically vanishes in
an incomplete-market economy with  Markov endowments is reassuring for the way we compute
stationary equilibria in the class of so-called Bewley models of
chapter \use{incomplete}.}

A possible limitation of by Beker and
Chattopadhyay's  analysis  is that they assumed that equilibrium outcomes are such that
consumers' Euler
equations always hold with equality. But   in
an incomplete-market economy  in which  Ponzi schemes
are ruled out by imposing ad hoc borrowing limits, a consumer's Euler equation will occasionally hold as a strict inequality (see chapter \use{selfinsure}).

To assure that Euler equations always hold with equality in  an
equilibrium, we recommend  loosening borrowing limits as much as possible by
imposing {\it natural debt limits}. As discussed in chapter
\use{incomplete}, natural debt limits in an incomplete-market
economy depend  on worst-case scenarios, i.e., they require  that  if
a worst possible state were to last forever, a consumer
would be able to  service  debt set at his
natural debt limit by setting  consumption identically to zero forever. Under
our maintained assumption that preferences satisfy an Inada
condition that asserts that the marginal utility of consumption becomes
infinite when consumption goes to zero, it follows that a consumer
will always choose debt to be  strictly less than
his natural debt limit, making the Euler equation
 hold with equality always.



\subsection{An example economy}

There are two types of consumers in equal numbers, so we refer to
one representative consumer for each type, indexed by $i=1,2$.
Agent $i$ has preferences ordered by
$$ U_i(c^i) =
   \sum_{t=0}^\infty \sum_{s^t} \beta_i^t \,u_i[c_t^i(s^t)]\,
   \pi_t^i(s^t),  \EQN Beker0 $$
where  $u_1(c)=c^{1-\gamma}/(1-\gamma)$,
for $\gamma\in(0,1)$ and $\gamma> 1$, and  $u_2(c) = \ln (c)$.
For all $t\geq0$, and all $s^t$ the true
probabilities $\pi_t(s^t)>0$ and the subjective probabilities
$\pi_t^i(s^t)>0$ for all $i$.
The aggregate endowment is always strictly positive and stochastic:
for all $t\geq 0$ and all $s^t$, $Y_t(s^t)>0$; and there exist
$s^{t+1}|s^t$ and $\tilde s^{t+1}|s^t$ such that
$Y_{t+1}(s^{t+1}) \neq Y_{t+1}(\tilde s^{t+1})$.
Agent $2$ receives the entire aggregate
endowment  in the initial period $0$,
but nothing there afterwards: $(y_0^1(s^0), y_0^2(s^0)) = (0, Y_0(s^0))$ and
$(y_t^1(s^t), y_t^2(s^t)) = (Y_t(s^t), 0)$ for all $t> 0$
and all $s^t$.



Agents sequentially trade  a single one-period asset
available in zero net supply having exogenous gross
payoffs $\mu_{t+1}(s^{t+1})>0$ for all $t\ge0$ and all $s^{t+1}$.
One unit of the asset acquired in period $t$ at history $s^t$ bears
 gross payoff $\mu_{t+1}(s_{t+1}, s^t)$ at $t+1$  when
state $s_{t+1}$ is realized. Let $b^i_{t+1}(s^t)$ denote the quantity of
the single asset purchased (or sold, if the quantity is negative)
by consumer $i$ at an equilibrium price $p_t(s^t)$.
The one-period budget constraint of consumer $i$ is
$$ c^i_t(s^t) + p_t(s^t) b^i_{t+1}(s^t) \leq a^i_t(s^t),  \EQN Beker1
$$
where $a^i_t(s^t)$ is consumer $i$'s accumulated wealth at the beginning
of period $t$, history $s^t$, including his endowment (this differs from our definition of $a^i_t(s^t)$
in the body of this chapter
that excluded current period's endowment), i.e.,
$$ a^i_t(s^t) = \mu_t(s^t) b^i_t(s^{t-1}) + y^i_t(s^t).  \EQN Beker2
$$
Agent $i$'s choice of asset $b^i_{t+1}(s^t)$ satisfies an Euler equation
$$
p_t(s^t) = \beta_i {\sum_{s^{t+1} | s^t} \pi^i_{t+1}(s^{t+1}| s^t)
\mu_{t+1}(s^{t+1}) [c^i_{t+1}(s^{t+1})]^{-\gamma} \over
[c^i_t(s^t)]^{-\gamma} },                                 \EQN Beker3
$$
with preference parameter $\gamma>0$ for consumer $1$ and $\gamma=1$ for
consumer $2$.

Agent $2$'s logarithmic utility function and the fact that
his sole endowment is received at time $0$
imply that consumer $2$'s decision rules
are \NFootnote{We guess, and then verify, that consumer $2$'s
value function takes the form
$v_t(a^2_t(s^t),s^t) = \ln[a^2_t(s^t)]/(1-\beta_2) + \hat v_t(s^t)$,
i.e., the value function can be decomposed into two additively
separable terms. The consumer can exert control over wealth in the first
term, while the second term is a function of prices that are taken as
given by the consumer. The Bellman equation, with our guess of the value
function, becomes
$$\EQNalign{
v_t(a^2_t(s^t),s^t) &= \max_{b^2_{t+1}(s^t)} \Biggl\{
\ln\left[a^2_t(s^t) - p_t(s^t) b^2_{t+1}(s^t)\right]            \cr
& + \beta_2  \sum_{s^{t+1}|s^t}  \pi_{t+1}^2(s^{t+1}|s^t)
\left(  { \ln\left[\mu_{t+1}(s^{t+1}) b^2_{t+1}(s^{t})\right] \over 1-\beta_2 }
        + \hat v_{t+1}(s^{t+1}) \right) \Biggr\},      \hskip.5cm (\star)   \cr}
$$
where the budget constraint \Ep{Beker1} and the law of motion for wealth
\Ep{Beker2} are substituted into the utility function and the next
period's value function, respectively. Taking a first-order condition
with respect to $b^2_{t+1}(s^t)$ and rearranging, yields
asset decision rule \Ep{Beker4;b}, which after substituted into
budget constraint \Ep{Beker1}, yields consumption decision rule
\Ep{Beker4;a}. Next, we substitute this optimal choice into the right
side of $\rm{(\star)}$, and rearrange,
$$\EQNalign{
v_t(a^2_t(s^t),s^t) =& { \ln[a^2_t(s^t)] \over 1-\beta_2 }
+ \ln(1-\beta_2) + \beta_2  \sum_{s^{t+1}|s^t}  \pi_{t+1}^2(s^{t+1}|s^t) \cr
& \cdot \left( {1 \over 1 - \beta_2} \ln\left[ \beta_2 {\mu_{t+1}(s^{t+1}) \over p_t(s^t)}\right] + \hat v_{t+1}( s^{t+1}) \right) \equiv
{ \ln[a^2_t(s^t)] \over 1-\beta_2 } +  \hat v_t(s^t),  \cr}
$$
which is the form of value function that we had conjectured. Another way
to demonstrate the optimality of decision rules \Ep{Beker4}, is to consider
the corresponding finite-horizon problem and recursively, compute
time-dependent value functions as well as decision rules. As
the remainder problem goes to infinity, the decision rules can be shown to
converge to \Ep{Beker4}.}
$$ \EQNalign{ c^2_t(s^t) &=( 1 -\beta_2) a^2_t(s^t),   \EQN Beker4;a \cr
              b^2_{t+1}(s^t) &= \beta_2 a^2_t(s^t) / p_t(s^t),
                                                        \EQN Beker4;b \cr} $$
so consumer $2$ consumes a constant fraction $(1-\beta_2)$ of his
beginning-of-period wealth and saves the rest.  Substituting decision rule \Ep{Beker4;a}
for next period's consumption $c^2_{t+1}(s^{t+1})$ in consumer $2$'s
Euler equation \Ep{Beker3} and  invoking the law of motion for
wealth in \Ep{Beker2} gives
$$ \EQNalign{
p_t(s^t) &= \beta_2 {\sum_{s^{t+1} | s^t} \pi^2_{t+1}(s^{t+1}| s^t)
\mu_{t+1}(s^{t+1}) [(1 -\beta_2) \mu_{t+1}(s^{t+1}) b^2_{t+1}(s^t)]^{-1}
         \over [c^2_t(s^t)]^{-1} }                                \cr
  &= \beta_2 { [(1 -\beta_2) b^2_{t+1}(s^t)]^{-1}
             \;\sum_{s^{t+1} | s^t} \pi^2_{t+1}(s^{t+1}| s^t)
         \over [c^2_t(s^t)]^{-1} }                                 \cr
  &= \beta_2 {\mu_{t+1}(\tilde s^{t+1}) [c^2_{t+1}(\tilde s^{t+1})]^{-1}
         \over [c^2_t(s^t)]^{-1} },                  \EQN Beker5  \cr}
$$
for any $\tilde s^{t+1} | s^t$. Equality \Ep{Beker5} implies that
the product
%is constant for all
%$s_{t+1}$, i.e.,
 $\mu_{t+1}(s^{t+1}) [c^2_{t+1}(s^{t+1})]^{-1}$ of two $s^{t+1}$-measurable random variables
equals an $s^t$-measurable random variable. %Therefore, we are free to represent
%its future value by the product that will be realized after
%any history $\tilde s^{t+1} |s^t$.

Market clearing dictates that
$$ \EQNalign{ c^1_t(s^t)     &= Y_t(s^t) - c^2_t(s^t),  \EQN Beker6;a \cr
              b^1_{t+1}(s^t) &= - b^2_{t+1}(s^t),       \EQN Beker6;b \cr}
$$
for all $t\geq 0$ and all $s^t$. For $t >0$,  we can use
consumer $2$'s consumption decision rule \Ep{Beker4;a} and the law of
motion for wealth in \Ep{Beker2} to express consumption market
clearing condition \Ep{Beker6;a} as
$$
c^1_t(s^t)  = Y_t(s^t) - (1 -\beta_2) \mu_t(s^t) b^2_t(s^{t-1}),
                                                        \EQN Beker7
$$
for all $t>0$ and all $s^t$.

To characterize an equilibrium, it is useful to
compute
%the following variable for consumer $i$ with preference
%parameter $\gamma$,
$$
\xi^i_t(\tilde s^{t}) =
{\mu_{t}(\tilde s^{t}) [c^i_{t}(\tilde s^{t})]^{-\gamma}
\over   \sum_{s^{t} | s^{t-1}} \pi^i_{t}(s^{t}| s^{t-1})
\mu_{t}(s^{t}) [c^i_{t}(s^{t})]^{-\gamma} },       \EQN Beker8
$$
for $\tilde s^{t} | s^{t-1}$.
The random variable $\xi^i_t(\tilde s^{t})$ has conditional mean one.
For consumer $2$, we showed in \Ep{Beker5} that the
product $\mu_{t}(s^{t}) [c^2_{t}(s^{t})]^{-1}$ is constant across
realizations of the stochastic event $s_{t}$ and hence,
$\xi^2_t(\tilde s^{t})=1$. For consumer $1$, we can use
market-clearing condition \Ep{Beker7} to deduce
$$
\xi^1_t(\tilde s^{t}) =
{\mu_{t}(\tilde s^{t}) [Y_{t}(\tilde s^{t})
          - (1 -\beta_2) \mu_{t}(\tilde s^{t}) b^2_{t}(s^{t-1})]^{-\gamma}
\over   \sum_{s^{t} | s^{t-1}} \pi^1_{t}(s^{t}| s^{t-1})
\mu_{t}(s^{t}) [Y_{t}(s^{t})
          - (1 -\beta_2) \mu_{t}(s^{t}) b^2_{t}(s^{t-1})]^{-\gamma} }
                                                            \EQN Beker9
$$
for $\tilde s^{t} | s^{t-1}$.


\subsection{Asset payoff correlated with i.i.d.\ aggregate
            endowment}\label{sec:Beker_example}%
\noindent
We now assume that asset payoffs are perfectly correlated with the
aggregate endowment, $\mu_t(s^t)=Y_t(s^t)$ that in turn is governed by
an i.i.d.\ process, $Y_t(s^t) = Y(s_t)$ with probability
$\pi_t(s^t)=\pi(s_t)>0$, for $s_t\in S$. Agent $i$ knows that the
stochastic outcomes are governed by an i.i.d.\ process; his subjective
probabilities are $\pi^i(s)>0$ for $s\in S$.

Substituting $\mu_t(s^t)=Y_t(s^t)$ into \Ep{Beker9} yields
$$ \EQNalign{
\xi^1_t(\tilde s^{t}) &= {[Y_{t}(\tilde s^{t})]^{1-\gamma}
                     [1 - (1 -\beta_2) b^2_{t}(s^{t-1})]^{-\gamma}
\over   \sum_{s^{t} | s^{t-1}} \pi^1_{t}(s^{t}| s^{t-1})
[Y_{t}(s^{t})]^{1-\gamma} [1 - (1 -\beta_2) b^2_{t}(s^{t-1})]^{-\gamma} } \cr
&= {[Y(\tilde s_t)]^{1-\gamma}
\over   \sum_{s\in S} \pi^1(s)
[Y(s)]^{1-\gamma} }\equiv \xi^1(\tilde s_{t}),    \EQN Beker10 \cr}
$$
where the second equality  invokes the assumption of i.i.d.\ aggregate
endowment.

Substituting $\mu_t(s^t)=Y_t(s^t)$ into Euler equation \Ep{Beker3} and
equating the left sides for consumers $1$ and $2$, thereby eliminating $p_t(s^t)$, yields
$$ \EQNalign{
&\beta_1 {\sum_{s^{t+1} | s^t} \pi^1_{t+1}(s^{t+1}| s^t)
Y_{t+1}(s^{t+1}) [c^1_{t+1}(s^{t+1})]^{-\gamma} \over
[c^1_t(s^t)]^{-\gamma} }                              \cr
& \hskip5cm =
\beta_2 {Y_{t+1}(\tilde s^{t+1}) [c^2_{t+1}(\tilde s^{t+1})]^{-1}
         \over [c^2_t(s^t)]^{-1} },        \hskip1cm  \EQN Beker11 \cr}
$$
where the expression for consumer $2$ is taken from \Ep{Beker5}. Recall
that we can pick any $\tilde s^{t+1}|s^t$, because the product
$\mu_{t+1}(\tilde s^{t+1}) [c^2_{t+1}(\tilde s^{t+1})]^{-1}$ is
constant across realizations of $\tilde s^{t+1}| s^t$. Now
let $\tilde s_{t+1}$ denote the particular event that is
 realized at time $t+1$ so that the realized history
becomes $\tilde s^{t+1}$.
Next, multiply and divide the left side of \Ep{Beker11} by
$Y_{t+1}(\tilde s^{t+1}) [c^1_{t+1}(\tilde s^{t+1})]^{-\gamma}$,
$$
\beta_1 \, {1 \over \xi^1_{t+1}(\tilde s^{t+1}) } \,
 {[c^1_{t+1}(\tilde s^{t+1})]^{-\gamma}
\over [c^1_t(s^t)]^{-\gamma} }
=
\beta_2 {[c^2_{t+1}(\tilde s^{t+1})]^{-1}
         \over [c^2_t(s^t)]^{-1} },              %%% \EQN Beker12
$$
then rearrange to get
$$ \EQNalign{
{ [c^1_{t+1}(\tilde s^{t+1})]^{\gamma} \over c^2_{t+1}(\tilde s^{t+1}) }
&= {\beta_1 \over \beta_2}\, {1 \over \xi^1_{t+1}(\tilde s^{t+1}) }\,
{ [c^1_{t}(\tilde s^{t})]^{\gamma} \over c^2_{t}(\tilde s^{t}) } \cr
&= \left[{\beta_1 \over \beta_2}\right]^{t+1}
 {1 \over \prod_{j=1}^{t+1} \xi^1(\tilde s_j) }\,
{ [c^1_{0}(\tilde s^{0})]^{\gamma} \over c^2_{0}(\tilde s^{0}) },
                                                         \EQN Beker13  \cr }
$$
where the second equality is obtained by iterating. We have also
invoked the assumption of an i.i.d.\ stochastic process so that
$\xi^1_t(\tilde s^{t}) = \xi^1(\tilde s_t)$, as defined in \Ep{Beker10}.

Given a preference parameter $\gamma\neq 1$ for consumer $1$,
we now show that $\prod_{j=1}^{T} \xi^1(\tilde s_j) \rightarrow 0$ as $T$
goes to infinity for $\{\pi^1(s); s\in S\}$ sufficiently close
to $\{\pi(s); s\in S\}$. It suffices to show that
$\lim_{T\to\infty} {1 \over T}$
$ \sum_{j=1}^{T} \log( \xi^1(\tilde s_j)) <0$,
i.e.,
$$
\left[
\lim_{T\to\infty} {1 \over T} \sum_{j=1}^{T}
\log\left( [Y(\tilde s_j)]^{1-\gamma}  \right) \right]
- \log\left( \sum_{s\in S} \pi^1(s) [Y(s)]^{1-\gamma} \right) <0.  \EQN Beker14
$$
Since the aggregate endowment is an i.i.d.\ process, by the law of
large numbers,
$$
{1 \over T} \sum_{j=1}^{T} \log\left( [Y(\tilde s_j)]^{1-\gamma}  \right)
\rightarrow
\sum_{s\in S} \pi(s) \log \left( [Y(s)]^{1-\gamma} \right),     %%%\EQN Beker15
$$
and so, by Jensen's inequality, there exists $\epsilon >0$ such that
$$
\left[
\lim_{T\to\infty} {1 \over T} \sum_{j=1}^{T}
\log\left( [Y(\tilde s_j)]^{1-\gamma}  \right) \right]
- \log\left( \sum_{s\in S} \pi(s) [Y(s)]^{1-\gamma} \right) < -\epsilon,
                                                               %%% \EQN Beker16
$$
where by continuity, the inequality  in \Ep{Beker14} also holds for
$\{\pi^1(s); s\in S\}$ sufficiently close to $\{\pi(s); s\in S\}$.

It now follows from \Ep{Beker13} that if both consumers have the same
discount factor, $\beta_1=\beta_2$, and if both  hold the same true beliefs
$\pi^1(s)=\pi^2(s)=\pi(s)$ for all $s\in S$, then the ratio of consumer
$1$'s consumption to consumer $2$'s consumption diverges to $+ \infty$.
Given that feasibility (market clearing) imposes an upper bound on
consumer $1$'s consumption, $c^1_t(s^t) \leq Y_t(s^t)$, we conclude that
$c^2_T(s^T) \rightarrow 0$ as $T$ goes to infinity, i.e.,
consumer $2$'s consumption vanishes with probability one. In contrast, in a complete
market economy, the consumption allocation would be invariant to
calender time and just depend on the aggregate endowment
realization (and a set of time-invariant Pareto weights).

Actually, our finding in this example does not depend
on the beliefs held by consumer $2$, since his subjective
probabilities $\{\pi^2(s); s\in S\}$ are absent from equilibrium
expression \Ep{Beker13}. For the sake of the argument, suppose
that consumer $2$ holds the true beliefs but that consumer $1$ has incorrect
beliefs. As shown above, for
$\{\pi^1(s); s\in S\}$ sufficiently close to $\{\pi(s); s\in S\}$,
consumer $2$'s consumption  eventually vanishes.
This  differs from the outcome in the complete market
economy that we studied in section \use{sec:heterogbel_example},
where the consumer with the true beliefs would eventually consume the
entire aggregate endowment.

Using  the same line of reasoning, another implication of \Ep{Beker13}
is that even if consumer $1$ is marginally less patient than consumer $2$,
$\beta_1 < \beta_2$, and allowing for the possibility that
consumer $1$ also has marginally incorrect beliefs, it still follows
that consumer $2$'s consumption goes to zero as time goes
to infinity. The consumption of a more patient consumer with more accurate
beliefs could not vanish
 in a complete market economy.

%Beker and Chattopadhyay (2010) use this example to illustrate one of
%the possible outcomes in an infinite-horizon endowment economy populated
%by two types of consumers and where markets are incomplete. Their
%general finding is that either (i) the two consumers' one-period ahead
%intertemporal marginal rates
%of substitution are equalized in the limit or (ii) the ratio of marginal
%rates of substitution displays one period ahead conditional variability
%forever and then either (a) both consumers will experience consumption
%arbitrarily close to zero infinitely often, or (b) one of the two consumers
%will vanish eventually, as in the example. The finding applies
%regardless of whether beliefs are homogeneous or heterogenous.




\subsection{Beneficial market incompleteness}

Modifying the preceding example, we now assume that the preference parameter
of consumer $1$
is $\gamma=1$, i.e., both consumers have a logarithmic utility function,
and that  discount factors are identical, $\beta_1 = \beta_2 = \beta$.
According to \Ep{Beker10}, $\xi^1(\tilde s)=1$ for all $\tilde s\in S$
and hence, the consumption ratio $c^1_t(\tilde s^t) / c^2_t(\tilde s^t)$
in \Ep{Beker13} remains constant over time. In this
incomplete markets economy, it turns out that the consumption allocation
does not depend on whether consumers' beliefs are correct or incorrect. Furthermore,
the allocation equals  the allocation that would prevail in
a complete markets economy with correct beliefs.
This outcome motivates our section title `beneficial' market
incompleteness;  we use quotation marks to acknowledge
a paternalistic view that the welfare of consumers who maximize expected
utility are to be evaluated from the perspective  of true
probabilities.\NFootnote{One qualification to  calling it `beneficial'
market incompleteness is that an consumer with subjective probabilities
that differ from the true probabilities would at any point in time
prefer to make other choices if there were complete rather than
incomplete markets and thus, the consumer is constantly under the subjective
perception of being deprived of preferable options that he could have
chosen under complete markets. One countercriticism in favor of the
paternalistic view is that with utility functions defined over actual
consumption at each point in time, what matters are the histories of
consumption that will eventually be realized. That is, there is no loss
of utility {\it per se} from a deluded consumer holding dismal subjective
beliefs about future consequences of having had to make choices from a
restricted set of options. But rather, market incompleteness is
beneficial by compelling the consumer period-by-period to choose a better
allocation (in terms of expected utility under the true probability
distribution).}
% ...along some equilibrium realizations,
%deluded choices made under incorrect beliefs could possibly result in
%higher ex-post lifetime utility (in terms of period-$0$ discounted
%utility), if initial realizations of events were deemed more
%likely under the consumer's subjective beliefs as compared to the true
%true probabilities. Admitted, the consumer
%with less correct beliefs in a complete market economy will eventually
%vanish, but the distant future matters less under discounting.}

First, we compute the allocation of a complete markets economy in which everyone has
correct beliefs. Referring to our sections
\use{sec:ex1_risksharing} and \use{sec:compeasy} analysis
of an economy with a common utility function of the constant relative
risk-aversion (CRRA) form, an equilibrium allocation assigns
 consumers  constant fractions of the aggregate endowment
at all times and histories.   Those fractions equal  consumers' shares of the aggregate
present-value wealth evaluated at the competitive equilibrium price
vector. Let $\lambda_i$ denote the consumption share of consumer $i$.
Using the good in period $0$ as  numeraire,  Arrow-Debreu
security prices are
$$
q^0_t(s^t) = \beta^t \pi_t(s^t) { [\lambda_i Y_t(s^t)]^{-1} \over
             [\lambda_i Y_0(s^0)]^{-1} }
           = \beta^t \pi_t(s^t) { Y_0(s^0) \over Y_t(s^t) }.  \EQN Beker17
$$
The value of consumer $1$'s wealth is
$\sum_{t=1}^\infty \sum_{s^t} q^0_t(s^t) Y_t(s^t)$ and consumer $2$'s
wealth is $Y_0(s^0)$, so the share $\lambda_2$ of consumer $2$ is
$$
\lambda_2 = { \displaystyle Y_0(s^0) \over \displaystyle
Y_0(s^0) + \sum_{t=1}^\infty \sum_{s^t} q^0_t(s^t) Y_t(s^t) }
          = {\displaystyle Y_0(s^0) \over \displaystyle
Y_0(s^0) + {\displaystyle \beta Y_0(s^0) \over \displaystyle 1-\beta }}
= 1-\beta,                                                    \EQN Beker18
$$
where the second equality invokes the equilibrium expression
for prices \Ep{Beker17}. The corresponding consumption share of
consumer $1$ is $\lambda_1= \beta$.

Second, we verify that  regardless of whether
consumers' beliefs are correct, the  allocation equals the equilibrium allocation for the incomplete markets economy. We have already verified
equality for $t=0$
because the  allocation $c^2_0(s^0)=\lambda_2 Y_0(s^0)
= (1-\beta) Y_0(s^0)$ is indeed the consumption of consumer $2$ in
an incomplete market economy, as given by \Ep{Beker4;a}, where
consumer $2$ consumes a fraction $(1-\beta)$ of his initially
accumulated wealth $a^2_0(s^0)=Y_0(s^0)$. To confirm that it is the equilibrium allocation
for all future periods, we conjecture that consumption shares
$(\lambda_1, \lambda_2)$ constitute an incomplete market equilibrium,
then compute the implied equilibrium prices $p_t(s^t)$ from
Euler equations. Next, given those prices, we compute consumers' choices
of consumption and verify that they coincide with the conjectured
consumption shares $(\lambda_1, \lambda_2)$. We do this
with consumer $2$'s Euler equation \Ep{Beker5},
$$
p_t(s^t) = \beta {Y_{t+1}(\tilde s^{t+1}) [\lambda_2 Y_{t+1}(\tilde s^{t+1})]^{-1}
         \over [\lambda_2 Y_t(s^t)]^{-1} } = \beta Y_t(s^t).   \EQN Beker19
$$
Given these prices,
%% $p_t(s^t)=\beta Y_t(s^t)$, for all $t\geq 0$ and all $s^t$,
we use consumer $2$'s decision rule in \Ep{Beker4;b} to compute his
asset choice at time $0$, $b^2_1(s^0) = \beta Y_0(s^0)/p_0(s^0) = 1$,
i.e., consumer $2$ purchases one unit of the asset with asset payoffs
next period equal to $\mu_1(s^1) = Y_1(s^1)$. It follows that
consumer $2$'s beginning-of-period wealth in period $1$ is
$a^2_1(s^1)=Y_1(s^1)$,
so we can apply the same  reasoning to period $1$:
according to decision rules \Ep{Beker4}, consumer $2$ consumes a fraction
$(1-\beta)$ of $a^2_1(s^1)$ and saves the rest by purchasing
assets $b^2_2(s^1)= \beta a^2_1(s^1)/p_1(s^1)=1$. This continues
{\it ad infinitum}, where in each period, consumer $2$ lends
a fraction $\beta$ of his accumulated wealth to consumer $1$ in exchange
for consumer $1$'s entire endowment  next period. In this way we  verify
our conjecture that the incomplete market economy has an equilibrium
allocation with constant consumption shares
$(\lambda_1, \lambda_2)=(\beta, 1-\beta)$. Because our argument has
not involved any mathematical expectations, consumers' beliefs can be either correct or
incorrect.\NFootnote{Why don't consumers' beliefs matter in the incomplete
markets economy? Actually, our assertion that no expectations were
involved in the above reasoning is subject to  a qualification.
That qualification is best seen by replacing consumer $2$'s Euler
equation in the above reasoning with that of consumer $1$, as given
by \Ep{Beker3}. This switch reinserts expectations into our
reasoning, but as before for consumer $2$ in \Ep{Beker5}, these
expectations now also vanish for consumer $1$ when the product
$\mu_{t+1}(s^{t+1}) [c^1_{t+1}(s^{t+1})]^{-1}$ is  constant across realizations of next period's stochastic
event $s_{t+1}$.}

What features of the example explain why the equilibrium
allocation of the incomplete market economy, with or without
correct beliefs, equals that of the complete market economy
with correct beliefs? Starting with the case of correct beliefs, we
know from section \use{sec:ex1_risksharing}
that  a common utility
function of the constant relative risk-aversion (CRRA) form
implies that an efficient allocation prescribes that each
 consumer consumes a constant share of the aggregate endowment
in all periods and all states. So it seems to be important
that the exogenously specified payoffs of the single asset in the
incomplete market economy are perfectly correlated with the
aggregate endowment.

What  role is played by the exogenous package  of
Arrow securities implicit in the single asset in the
incomplete market economy?  How can we be certain that this mix
can support the same  allocation that would prevail in  a complete
market economy? The answer is that consumer $1$ owns the entire
aggregate endowment after period $0$ and hence, the efficient
risk sharing  characterized here by having each distinct consumer
consume a constant fraction of the aggregate endowment can be achieved
by trading Arrow securities for each possible
state next period in proportion to the realizations of the
aggregate endowment in those states.  The  trades implicit in
  the  bundle
associated with the single asset in the incomplete market economy accomplish exactly this.



























%\section{Exercises}
\showchaptIDfalse \showsectIDfalse
\section{Exercises}
\showchaptIDtrue \showsectIDtrue
\medskip
\medskip
\noindent{\it Exercise \the\chapternum.1}\quad {\bf Existence of
representative consumer}
\medskip
\noindent Suppose households 1 and 2 have one-period utility
functions $u(c^1)$ and $w(c^2)$, respectively, where $u$ and $w$
are both increasing, strictly concave, twice differentiable
functions of a scalar consumption rate.  Consider the Pareto
problem:
 $$v_\theta(c) = \max_{\{c^1, c^2\}} \left[ \theta u(c^1)
   + (1-\theta) w(c^2) \right]$$ subject to
the constraint $c^1 + c^2 = c$.  Show that the solution of this
problem has the form of a concave utility function $v_\theta(c)$,
which depends on the Pareto weight $\theta$. Show that
$v_\theta'(c) = \theta u'(c^1) = (1-\theta) w'(c^2)$.

The function $v_\theta(c)$ is the utility function of the {\it
representative consumer}.  Such a representative consumer always
lurks within a complete markets competitive equilibrium even with
heterogeneous preferences. At a competitive equilibrium, the
marginal utilities of the representative agent and each and every
agent are proportional.
\vfil\eject
\medskip
\noindent {\it Exercise \the\chapternum.2} \quad  {\bf Term
structure of interest rates}
\medskip
\noindent Consider an economy with a single consumer.  There is
one good in the economy, which arrives in the form of an exogenous
endowment obeying\NFootnote{Such a specification was adopted by Mehra
and Prescott (1985).} \auth{Prescott, Edward C.} \auth{Mehra,
Rajnish}%
$$ y_{t+1}  = \lambda_{t+1} y_t,$$
where $y_t$ is the endowment at time $t$ and $\{\lambda_{t+1}\}$
is  governed by a two-state Markov chain with transition matrix
$$ P = \left[\matrix{ p_{11} & 1 - p_{11} \cr
              1 - p_{22} & p_{22} \cr} \right],$$
and initial distribution $\pi_\lambda = \left[\matrix{\pi_0 & 1 -
\pi_0\cr}\right]$.  The value of $\lambda_t$ is given by $\bar
\lambda_1 = .98$ in state 1 and $\bar \lambda_2 = 1.03$ in state
2.
 Assume that the history of $y_s, \lambda_s$ up to $t$ is
observed at time $t$.  The consumer has endowment process
$\{y_t\}$ and has preferences over consumption streams that are
ordered by
$$  E_0 \sum_{t=0}^\infty \beta^t u(c_t)  $$
where $\beta \in (0,1)$ and $u(c) = {c^{1-\gamma} \over 1 -
\gamma}$, where $\gamma \geq 1$.

\medskip
\noindent{\bf a.} Define a competitive equilibrium, being careful
to name all of the objects of which it consists.
\medskip
\noindent{\bf b.}  Tell how to compute a competitive equilibrium.
\medskip
\noindent  For the remainder of this problem, suppose that
$p_{11} = .8$, $p_{22} = .85$, $\pi_0 = .5$, $\beta = .96$, and
$\gamma =2$.
  Suppose that the economy begins
with $\lambda_0 = .98$ and $y_0=1$.
\medskip
\noindent{\bf c.} Compute the (unconditional) average growth rate
of consumption, computed before having observed $\lambda_0$.
\medskip
\noindent{\bf d.} Compute the time $0$ prices of three risk-free
discount bonds, in particular, those  promising to pay one unit of
time $j$ consumption for $j=0, 1, 2$, respectively.
\medskip
\noindent{\bf e.}
  Compute the time $0$ prices of three  bonds,
in particular, ones  promising to pay one unit of time $j$
consumption contingent on $\lambda_j = \bar \lambda_1$ for
$j=0,1,2$, respectively.

\medskip
\noindent{\bf f.}
  Compute the time $0$ prices of three  bonds,
in particular, ones  promising to pay one unit of time $j$
consumption contingent on $\lambda_j = \bar \lambda_2$ for
$j=0,1,2$, respectively.
\medskip
\noindent{\bf g.} Compare the prices that you computed in parts d,
e, and f.



\medskip
\noindent{\it Exercise \the\chapternum.3} \ \  An economy consists
of two infinitely lived consumers named $i=1,2$. There is one
nonstorable consumption good. Consumer $i$ consumes $c^i_t$ at
time $t$. Consumer $i$ ranks consumption streams by
$$ \sum_{t=0}^\infty \beta^t u(c^i_t), $$
where $\beta\in (0,1)$ and $u(c)$ is increasing, strictly concave,
and twice continuously differentiable.   Consumer $1$ is  endowed
with a stream of the consumption good $y^i_t = 1, 0, 0, 1, 0, 0,
1, \ldots$.  Consumer $2$ is endowed with a stream of the
consumption good   $0, 1, 1, 0, 1, 1, 0, \ldots$.  Assume that
there are complete markets with time $0$ trading.

\medskip
\noindent{\bf a.}  Define a competitive equilibrium.
\medskip
\noindent{\bf b.}  Compute a competitive equilibrium.
\medskip
\noindent{\bf c.}  Suppose that one of the consumers markets a
derivative  asset that promises to pay  .05 units of consumption
each period.    What would the price of that asset be?

\medskip
\noindent{\it Exercise \the\chapternum.4} \ \  Consider a pure
endowment economy with a single representative consumer; $\{c_t,
d_t\}_{t=0}^\infty$ are the consumption and endowment processes,
respectively. Feasible allocations satisfy
$$ c_t \leq d_t .$$
The endowment process is described by\NFootnote{See Mehra and
Prescott (1985).}
$$ d_{t+1} =\lambda_{t+1} d_t .  $$
The growth rate $\lambda_{t+1}$ is described by a two-state
Markov process with transition probabilities
$$ P_{ij} = {\rm Prob} (\lambda_{t+1} = \bar \lambda_j \vert
    \lambda_t = \bar \lambda_i ). $$
Assume that
$$ P = \left[\matrix{ .8 & .2 \cr
                      .1 & .9 \cr} \right], $$
and that
$$ \bar \lambda = \left[\matrix{.97 \cr
                                1.03 \cr}\right].$$
In addition, $\lambda_0 = .97$ and $d_0 = 1$  are both known at
date $0$.  The consumer has preferences over consumption ordered
by
$$ E_0 \sum_{t=0}^\infty \beta^t {c_t^{1-\gamma} \over 1 - \gamma},$$
where $E_0$ is the mathematical expectation operator, conditioned
on information known at time $0$, $\gamma = 2, \beta = .95$.
\medskip


\noindent{\bf Part I}
\medskip
\noindent  At time $0$, after $d_0$ and $\lambda_0$ are known,
there are complete markets in date- and history-contingent claims.
The market prices are denominated in units of time $0$ consumption
goods.

\medskip
\noindent{\bf a.}  Define a competitive equilibrium, being careful
to specify all  the objects composing an equilibrium.

\medskip
\noindent{\bf b.}  Compute the equilibrium price of a claim to one
unit of consumption at date $5$, denominated in units of time $0$
consumption, contingent on the following history of growth rates:
$(\lambda_1, \lambda_2, \ldots, \lambda_5) =( .97, .97, 1.03, .97,
1.03)$.  Please give a numerical answer.

\medskip
\noindent{\bf c.}  Compute the equilibrium price of a claim to one
unit of consumption at date $5$, denominated in units of time $0$
consumption, contingent on the following history of growth rates:
$(\lambda_1, \lambda_2, \ldots, \lambda_5) = (1.03,  1.03, 1.03,
1.03, .97)$.



\medskip
\noindent{\bf d.}  Give a formula for the price at time $0$ of a
claim on the entire endowment sequence.

\medskip
\noindent{\bf e.}  Give a formula for the price at time $0$ of a
claim on consumption in period 5,   contingent on the growth rate
$\lambda_5$ being $.97$ (regardless of the intervening growth
rates).

\vfil\eject

\medskip
\noindent{\bf Part II}

\medskip
\noindent Now assume a different market structure.  Assume that at
each date $t\geq 0$ there  is a complete set of one-period forward
Arrow securities.
\medskip
\noindent{\bf f.}  Define a (recursive) competitive equilibrium
with Arrow securities, being careful to define all of the objects
that compose such an equilibrium.
\medskip
\noindent{\bf g.} For the representative consumer in this
economy, for each state compute the ``natural debt limits'' that
constrain state-contingent borrowing.
\medskip
\noindent{\bf h.} Compute a competitive equilibrium with Arrow
securities. In particular,  compute both the pricing kernel and
the allocation.
\medskip
\noindent{\bf i.}  An entrepreneur       enters this economy and
proposes to issue a new security each period,  namely, a risk-free
two-period bond.  Such a bond issued in period $t$ promises to pay
one unit of  consumption at time $t+1$ for sure. Find the price of
this new security in period $t$, contingent on $\lambda_t$.







\medskip
\noindent{\it Exercise \the\chapternum.5} \quad
\medskip
\noindent  An economy consists of two consumers, named $i=1,2$.
The economy exists in discrete time  for periods $t \geq 0$.
There is one good in the economy, which is not storable and
arrives in the form of an endowment stream owned by each consumer.
The endowments to consumers $i=1,2$ are
$$\eqalign{ y_t^1 & = s_t \cr
            y_t^2 & = 1 \cr }  $$
where $s_t$ is a random variable governed by a two-state Markov
chain with values $s_t = \bar s_1 = 0$ or  $s_t  = \bar s_2 = 1 $.
The Markov chain  has time invariant transition probabilities
denoted by $\pi(s_{t+1} =  s' | s_t = s) = \pi(s'|s)$, and the
probability distribution over the initial  state is $\pi_0(s)$.
The {\it aggregate endowment} at $t$ is $Y(s_t) = y_t^1 + y_t^2$.


   Let $c^i$ denote the stochastic
process of consumption for agent $i$.
  Household $i$ orders consumption streams according to
$$ U(c^i)= \sum_{t=0}^\infty \sum_{s^t}  \beta^t \ln [c_t^i(s^t)]
 \pi_t (s^t),  $$
where $\pi_t(s^t)$ is the probability of the history $s^t = (s_0,
s_1, \ldots , s_t)$.
\medskip
\noindent{\bf a.}  Give a formula for $\pi_t(s^t)$.
\medskip
\noindent{\bf b.}  Let $\theta \in (0,1)$ be a Pareto weight on
household $1$.  Consider the     planning problem
$$ \max_{c^1, c^2} \left\{ \theta \ln(c^1) + (1-\theta) \ln(c^2)
  \right\}  $$
where the maximization is subject to
$$ c_t^1(s^t) + c_t^2(s^t) \leq Y(s_t) . $$
Solve the Pareto problem, taking $\theta$ as a parameter.
\medskip

\noindent{\bf c.}   Define a {\it competitive equilibrium} with
history-dependent Arrow-Debreu  securities traded once and for all
at time $0$.  Be careful to define all of the objects that compose
a competitive equilibrium.
\medskip
\noindent{\bf d.}   Compute the competitive equilibrium price
system (i.e., find the prices of all of the Arrow-Debreu
securities).
\medskip
\noindent{\bf e.}    Tell the relationship between the solutions
(indexed by $\theta$) of the Pareto problem  and the competitive
equilibrium allocation.  If you wish, refer to the two welfare
theorems.
\medskip
\noindent{\bf f.}  Briefly tell how you can compute the
competitive equilibrium price system {\it before} you have figured
out the competitive equilibrium allocation.
\medskip
\noindent {\bf g.}   Now define a recursive  competitive
equilibrium with trading every period in one-period Arrow
securities only.   Describe all of the objects of which such an
equilibrium is composed. (Please denominate the prices of
one-period time $t+1$ state-contingent
 Arrow securities in units of time $t$ consumption.)
   Define
the ``natural borrowing limits'' for each consumer in each state.
Tell how to compute these natural borrowing limits.

\medskip
\noindent{\bf h.} Tell how to  compute the    prices of
one-period Arrow securities.  How many prices are there (i.e., how
many numbers do you have to compute)? Compute all of these prices
in the special case that $\beta = .95$ and $\pi(s_j | s_i) =
P_{ij}$ where $P = \left[\matrix{ .8 & .2 \cr
   .3 & .7 \cr}\right]$.
\medskip
\noindent{\bf i.}  Within the one-period Arrow securities economy,
  a new asset is introduced.     One of the households
decides to market a one-period-ahead riskless claim to one unit
of consumption (a one-period real bill). Compute the equilibrium
prices of this security when $s_t = 0$ and when    $s_t =1$.
Justify your formula for these prices in terms of first
principles.

\medskip
\noindent{\bf j.}  Within the one-period Arrow securities
equilibrium,  a new asset is introduced.     One of the households
decides to market a two-period-ahead riskless claim to one unit of
consumption (a two-period real bill). Compute the equilibrium
prices of this security when $s_t = 0$ and when    $s_t =1$.
\medskip
\noindent{\bf k.}
  Within the one-period Arrow securities
equilibrium,  a new asset is introduced.     One of the households
decides at time $t$ to market five-period-ahead claims to
consumption at $t+5$ contingent on  the value of $s_{t+5}$.
Compute the equilibrium prices of these securities when $s_t = 0$
and $s_t =1$  and $s_{t+5} = 0$ and $s_{t+5} = 1$.
\medskip






\noindent{\it Exercise \the\chapternum.6} \quad {\bf Optimal
taxation}
\medskip
\noindent
 The government of a small country
must finance an exogenous stream of government purchases
$\{g_t\}_{t=0}^\infty$.  Assume that $g_t$ is described by a
discrete-state Markov chain with transition matrix $P$ and initial
distribution $\pi_0$.   Let $\pi_t(g^t)$ denote the probability of
the history $g^t= g_t, g_{t-1} , \ldots , g_0$, conditioned on
$g_0$.  The state of the economy is completely described by the
history $g^t$. There are complete markets in date-history claims
to goods. At time $0$, after $g_0$ has been realized, the
government can purchase or sell  claims to time $t$  goods
contingent on the history $g^t$ at a price $p_t^0(g^t) = \beta^t
\pi_t(g^t)$, where $\beta \in (0,1)$. The date-state  prices are
exogenous to the small country. The government finances its
expenditures by raising history-contingent tax revenues of $R_t =
R_t(g^t) $ at time $t$.  The present
 value of its expenditures
must not exceed  the present  value of its revenues.

Raising revenues by taxation is distorting.  The government
confronts a dead weight loss function $W(R_t)$ that measures the
distortion at time $t$. Assume that $W$ is an increasing, twice
differentiable, strictly convex function that satisfies $W(0) =0,
W'(0) = 0 , W'(R) > 0$ for $R >0$ and $W''(R) > 0$ for $R \geq 0$.
The government devises a state-contingent taxation and borrowing
plan to minimize
$$ E_0 \sum_{t=0}^\infty  \beta^t W(R_t), \eqno(1) $$
where $E_0$ is the mathematical expectation conditioned on $g_0$.


Suppose that $g_t$ takes two possible values, $\bar g_1 =.2$
(peace) and $\bar g_2 = 1$ (war) and that   $P = \left[\matrix{ .8
& .2 \cr
                                 .5 & .5 \cr}\right]. $
Suppose that $g_0 = .2.$   Finally, suppose that $W(R) = .5 R^2$.

\medskip
\noindent{\bf a.} Please write out (1) long hand, i.e., write out
an explicit expression for the mathematical expectation $E_0$ in
terms of a summation over the appropriate probability
distribution.
\medskip
\noindent{\bf b.}   Compute the optimal tax and borrowing plan. In
particular,
 give analytic expressions for
$R_t =R_t(g^t)$ for all $t$ and all $g^t$.
\medskip
\noindent{\bf c.}  There is    an equivalent market setting in
which the government can buy and sell one-period Arrow securities
each period.   Find the price of one-period Arrow securities at
time $t$, denominated in units of the time $t$ good.
\medskip
\noindent{\bf d.} Let $B_t(g_t)$  be the one-period Arrow
securities at $t$ that the government issued for state $g_t$ at
time $t-1$. For $t >0$,
 compute
$B_t(g_t)$  for $g_t =\bar g_1$ and $g_t = \bar g_2$.

\medskip
\noindent{\bf e.} Use your answers to parts b and d to describe
the government's optimal policy for taxing and borrowing.


\medskip

%\vfil\eject


\medskip
\noindent{\it Exercise \the\chapternum.7} \quad {\bf A competitive
equilibrium}

\medskip
\noindent
 An endowment  economy consists
of two type of consumers.  Consumers of type 1 order consumption
streams of the one good according to
$$ \sum_{t=0}^\infty \beta^t c^1_t $$
and consumers of type 2 order consumption streams according to
$$ \sum_{t=0}^\infty \beta^t \ln(c^2_t) $$
where $c_t^i\geq 0$ is the consumption of a type $i$ consumer and
$\beta \in (0,1)$ is  a common discount factor. The consumption
good is tradable but nonstorable. There are equal numbers of the
two types of consumer. The consumer of type 1 is endowed with the
consumption sequence
$$ y_t^1 = \mu >0 \quad \forall t \geq 0 $$
where $\mu >0$. The consumer of type 2 is endowed with the
consumption sequence
$$ y_t^2 = \cases{ 0 & if  $t\geq 0$ is even   \cr
                   \alpha & if $t\geq 0 $ is odd \cr} $$
where $\alpha = \mu(1+\beta^{-1})$.

\medskip
\noindent{\bf a.} Define a competitive equilibrium with time $0$
trading. Be careful to include definitions of all of the objects
of which a competitive equilibrium is composed.
\medskip
\noindent{\bf b.}  Compute a competitive equilibrium allocation
with time $0$ trading.

\medskip
\noindent{\bf c.}  Compute the time $0$ wealths of the two types
of consumers using the competitive equilibrium prices.
\medskip
\noindent{\bf d.}  Define
 a competitive equilibrium with sequential
trading of Arrow securities.
\medskip
\noindent{\bf e.} Compute
 a competitive equilibrium with sequential
trading of Arrow securities.

%\vfil\eject
\medskip

\noindent{\it Exercise \the\chapternum.8} \quad {\bf Corners}
\medskip
\noindent
 A pure endowment  economy consists
of two type of consumers.  Consumers of type 1 order consumption
streams of the one good according to
$$ \sum_{t=0}^\infty \beta^t c^1_t $$
and consumers of type 2 order consumption streams according to
$$ \sum_{t=0}^\infty \beta^t \ln(c^2_t) $$
where $c_t^i\geq 0$ is the consumption of a type $i$ consumer and
$\beta \in (0,1)$ is  a common discount factor. Please note the
nonnegativity constraint on consumption of each person (the force
of this is that $c^i_t$ is {\it consumption\/}, not {\it
production\/}).   The consumption good is tradable but
nonstorable. There are equal numbers of the two types of consumer.
The consumer of type 1 is endowed with the consumption sequence
$$ y_t^1 = \mu >0 \quad \forall t \geq 0 $$
where $\mu >0$. The consumer of type 2 is endowed with the
consumption sequence
$$ y_t^2 = \cases{ 0 & if  $t\geq 0$ is even   \cr
                   \alpha & if $t\geq 0 $ is odd \cr} $$
where $$\alpha = \mu(1+\beta^{-1}).\eqno(1) $$

\medskip
\noindent{\bf a.} Define a competitive equilibrium with time $0$
trading. Be careful to include definitions of all of the objects
of which a competitive equilibrium is composed.
\medskip
\noindent{\bf b.}  Compute a competitive equilibrium allocation
with time $0$ trading. Compute the equilibrium price system.
Please also compute the sequence of one-period gross interest
rates.  Do they differ between odd and even periods?

\medskip
\noindent{\bf c.}  Compute the time $0$ wealths of the two types
of consumers using the competitive equilibrium prices.
\medskip
\noindent{\bf d.} Now consider an economy identical to the
preceding one except in one respect.  The endowment of consumer 1
continues to be 1 each period, but we assume that the endowment of
consumer 2 is larger (though it continues to be zero in every even
period). In particular, we alter  the assumption about endowments
in condition (1) to the new condition
$$\alpha > \mu(1+\beta^{-1}). $$
Compute the competitive equilibrium allocation and price system
for this economy.
\medskip
\noindent{\bf e.}  Compute the sequence of one-period interest
rates implicit in the equilibrium price system that you computed
in part d.  Are interest rates higher or lower than those you
computed in part b?
\index{equivalent martingale measure}
\medskip
\noindent{\it Exercise \the\chapternum.9}\quad {\bf Equivalent
martingale measure}
\medskip\noindent
Let $\{d_t(s_t)\}_{t=0}^\infty$ be   a stream of payouts. Suppose
that there are complete markets.  From \Ep{eq3} and
\Ep{asset1}, the price at time $0$ of a claim on this stream of
dividends is
$$ a_0 = \sum_{t=0}\sum_{s^t} \beta^t {u'(c_t^i(s^t))\over \mu_i} \pi_t(s^t) d_t(s_t). $$
Show that this $a_0$ can also be represented as
$$\eqalignno{ a_0  & = \sum_t b_t \sum_{s^t} d_t(s_t) \tilde \pi_t(s^t) &(1) \cr
        & = \tilde E_0 \sum_{t=0}^\infty b_t d_t(s_t) \cr} $$
where $\tilde E$ is the mathematical expectation with respect to
the twisted measure $\tilde \pi_t(s^t)$ defined by
$$ \eqalign{ \tilde \pi_t (s^t) & = b_t^{-1} \beta^t {u'(c_t^i(s^t))\over \mu_i} \pi_t(s^t)   \cr
      b_t & = \sum_{s^t} \beta^t {u'(c_t^i(s^t))\over \mu_i}\pi_t(s^t) . \cr}$$
Prove that $\tilde \pi_t(s^t)$ is a probability measure.
Interpret $b_t$ itself as a price of particular asset. Note:
$\tilde \pi_t(s^t)$ is called an {\it equivalent martingale
measure\/}. \index{martingale!equivalent measure}%
 See chapters
\use{assetpricing1} and \use{assetpricing2}.
%\vfil\eject
\medskip
\noindent{\it Exercise \the\chapternum.10}\quad {\bf
Harrison-Kreps prices}
\medskip\noindent
Show that the asset price  in (1) of the previous exercise can also be
represented as
$$\eqalign{ a_0 & = \sum_{t=0}^\infty \sum_{s^t}\beta^t  p_t^0(s^t)  d_t(s^t)\pi_t(s^t) \cr
              & = E_0 \sum_{t=0}^\infty \beta^t p_t^0 d_t \cr} $$
where $p_t^0(s^t)= q_t^0(s^t) / [\beta^t \pi_t(s^t)]$.

\medskip
\noindent{\it Exercise \the\chapternum.11}\quad {\bf Early
resolution of uncertainty}
\medskip\noindent
An economy consists of two households named $i=1,2$.   Each
household evaluates streams of a single consumption good according
to $ \sum_{t=0}^\infty \sum_{s^t} \beta^t u[c_t^i(s^t)]
\pi_t(s^t)
 $.  Here $u(c)$ is an increasing,
twice continuously differentiable, strictly concave function of
consumption $c$  of one good. The utility function satisfies the
Inada condition $ \lim_{c \downarrow 0} u'(c) = +\infty.$ A
feasible allocation satisfies $\sum_i c_t^i(s^t) \leq \sum_i
y^i(s_t) $.  The households' endowments of the one nonstorable
good are both functions of a state variable $s_t \in {\bf S} =
\left\{0, 1, 2\right\}$; $s_t$ is described by a time invariant
Markov chain with initial distribution $\pi_0 = \left[\matrix{0 &
1 & 0 \cr}\right]' $ and transition density defined by the
stochastic matrix
$$P = \left[\matrix{ 1 & 0 & 0 \cr .5 & 0 & .5 \cr 0 & 0 & 1\cr}\right].$$  The endowments of the two households
are
$$\eqalign{ y_t^1 & = s_t /2 \cr
            y_t^2 & = 1-s_t/2 .\cr} $$
 \medskip
\noindent{\bf a.}  Define a competitive equilibrium with Arrow
securities.
\medskip
\noindent{\bf b.}  Compute a competitive equilibrium with Arrow
securities.
\medskip
\noindent {\bf c.} By hand, simulate the economy.  In particular,
for every possible realization of the histories $s^t$, describe
time series of $c_t^1, c_t^2$ and the wealth levels $a_t^i$
of the households.  (Note: Usually this would be an impossible
task by hand, but this problem has been set up to make the task
manageable.)

\medskip
\noindent{\it Exercise \the\chapternum.12}\quad donated by Pierre-Olivier
Weill
\medskip
\noindent An economy is populated by a continuum of  infinitely
lived consumers of  types $j \in \{0,1\}$, with a
measure one of each. There is one nonstorable consumption good
 arriving in the form of an endowment stream owned by each
consumer. Specifically, the endowments are
%
%\begin{eqnarray*}
%y_t^0(s_t)&=& (1-s_t) \bar y^0 \medskip
%y_t^1(s_t)&=& s_t \bar y^1,
%\end{eqnarray*}

$$\eqalign{
y_t^0(s_t)&= (1-s_t) \bar y^0 \cr
y_t^1(s_t)&= s_t \bar y^1,\cr}$$
\noindent where $s_t$ is a two-state time-invariant Markov chain
 valued in $\{0,1\}$ and $\bar y^0 < \bar y^1$. The initial
state is $s_0=1$. Transition probabilities are denoted $\pi(s' |
s)$ for  $(s,s') \in \{0,1\}^2$, where $'$ denotes a next period value.
 The aggregate endowment is
 $y_t(s_t) \equiv (1-s_t) \bar y^0 + s_t \bar y^1$. Thus,
  this economy fluctuates stochastically between recessions
$y_t(0)= \bar y^0$ and booms $y_t(1) = \bar y^0$. In a recession,
the aggregate endowment is owned by type $0$ consumers, while in a
boom it is owned by a type $1$ consumers. A consumer orders
consumption streams according to:
$$
U(c^j) = \sum_{t=0}^{\infty} \sum_{s^t} \beta^t \pi(s^t | s_0)
  {c_t^j(s^t)^{1-\gamma}\over {1-\gamma}},
$$
\noindent where $s^t=(s_t,s_{t-1},\ldots,s_0)$ is the history of
the state up to time $t$, $\beta \in (0,1)$ is the discount
factor, and $\gamma >0$ is the coefficient of relative risk
aversion.\medskip

\noindent {\bf a.} Define a competitive equilibrium with time $0$
trading. Compute the price system $\{q_t^0(s^t)\}_{t=0}^{\infty}$
and the equilibrium allocation $\{c^j(s^t)\}_{t=0}^{\infty}$, for
$j \in \{0,1\}$.\medskip

\noindent {\bf b.} Find a utility function $\bar U(c) = E_0\bigl(
\sum_{t=0}^{\infty} \beta^t u(c_t) \bigr)$ such that the price
system $q_t^0(s^t)$ and the aggregate endowment $y_t(s_t)$ is an
equilibrium allocation of the single-agent economy $\bigl( \bar U
,\{y_t(s_t)\}_{t=0}^{\infty} \bigr)$. How does your answer depend
on the initial distribution of endowments $y_t^j(s_t)$ among the
two types $j \in \{0,1\}$? How would you defend the
representative agent assumption in this economy?\medskip

\noindent {\bf c.} Describe the equilibrium allocation under the
following three market structures: (i) at each node $s^t$,
agents can  trade only claims on their entire endowment streams;
(ii) at each node $s^t$, there is a complete set of one-period
ahead Arrow securities; and (iii) at each node $s^t$, agents can
only trade two risk-free assets, namely, a one-period zero-coupon bond
that pays one unit of consumption for sure at $t+1$ and a
two-period zero-coupon bond that pays one unit of the consumption good
for sure at $t+2$. How would you modify your answer in the absence
of aggregate uncertainty?\medskip

\noindent {\bf d.} Assume that $\pi(1|0)=1$, $\pi(0|1)=1$, and as
before $s_0=1$. Compute the allocation in an equilibrium with
time $0$ trading. Does the type $j=1$ agent always consume the
largest share of the aggregate endowment? How does it depend on
parameter values? Provide economic intuition for your results.\medskip

\noindent {\bf e.} Assume that $\pi(1|0)=1$ and $\pi(0|1)=1$.
Remember that $s_0=1$. Assume that at $t=1$ agent $j=0$ is given
the option to default on her financial obligation. For example, in
the time $0$ trading economy, these obligations are deliveries of
goods. Upon default, it is assumed that the agent is excluded from
the market and has to consume her endowment forever. Will the
agent ever exercise her option to default?



\medskip
\noindent{\it Exercise \the\chapternum.13} \quad {\bf Diverse beliefs, I}
\medskip
\noindent
 A pure endowment economy is populated by two consumers. Consumer $i$ has preferences over history-contingent
consumption sequences $\{c_t^i(s^t)\}$ that are ordered by
$$ \sum_{t=0}^\infty \sum_{s^t} \beta^t u(c_t^i(s^t)) \pi_t^i (s^t) ,$$
where $u(c) = \ln(c)$  and where $\pi_t^i(s^t)$ is a density that consumer $i$ assigns to history $s^t$. The state space is time invariant. In particular,
$s_t \in S = \{0, .5,  1\}$ for all $t \geq 0$.
Only two histories are possible for $t=0, 1, 2, \ldots$:
$$ \eqalign{ & {\rm history \ 1}: .5, 1, 1, 1, 1, \ldots \cr
             & {\rm history \ 2}: .5, 0, 0, 0, 0, \ldots \cr } $$
Consumer $1$ assigns probability $1/3$ to  history 1  and probability $2/3$ to history 2, while consumer 2 assigns probability
$2/3$ to history 1 and probability $1/3$ to history 2.  Nature assigns equal probabilities to the two histories.
The endowments of the two consumers are:

$$ \eqalign{ y_t^1 & = s_t \cr
          y_t^2 & = 1-s_t . \cr} $$

\medskip
\noindent{\bf a.} Define a competitive  equilibrium with sequential trading of a complete set of one-period Arrow securities.

\medskip
\noindent {\bf b.}  Compute a competitive equilibrium with sequential trading of a complete set of one-period Arrow securities.

\medskip
\noindent{\bf c.} Is the equilibrium allocation Pareto optimal?



\medskip
\noindent{\it Exercise \the\chapternum.14} \quad {\bf Diverse beliefs, II}
\medskip
\noindent
Consider the following $I$ person  pure endowment economy.  There is a state variable $s_t \in S$ for all $t\geq 0$.
Let $s^t$ denote a history of $s$ from $0$ to $t$.  The time $t$  aggregate endowment is a function of the history, so $Y_t = Y_t(s^t)$.  Agent $i$ attaches a personal probability of $\pi_t^i(s^t)$ to history $s^t$. The history $s^t$ is observed by all
$I$ people at time $t$. Assume that  for all $i$, $\pi_t^i(s_t) > 0$ if and only if  $\pi_t^1(s_t) > 0$ (so the consumers agree about which histories have positive probability).
   Consumer $i$ ranks consumption plans $c_t^i(s^t)$ that are measurable functions of histories
via the expected utility functional
$$ \sum_{t=0}^\infty  \sum_{s^t}\beta^t \ln (c_t^i(s^t)) \pi_t^i(s^t) \leqno(1) $$
The ownership structure of the economy is not yet determined.

\medskip
\noindent A planner puts positive Pareto weights $\lambda_i >0$ on consumers $i=1, \ldots, I$ and solves a time $0$ Pareto
problem that respects each consumer's preferences as represented by (1).

\medskip
\noindent {\bf a.}  Show how to solve for a Pareto optimal allocation.  Display an expression for $c^i_t(s^t)$ as a function
of $Y_t(s^t)$ and other pertinent variables.

\medskip
\noindent {\bf b.} Under what circumstances does the Pareto plan imply  complete risk-sharing among the $I$ consumers?

\medskip
\noindent {\bf c.}  Under what circumstances does the Pareto plan imply an allocation that is not history dependent? By `not history
dependent', we mean
that  $Y_t(s^t) = Y_t(\tilde s^t)$  would imply the same allocation at time $t$?
\medskip
\noindent {\bf d.}  For a given set of Pareto weights, find an associated equilibrium price vector and an initial distribution
of wealth among the $I$ consumers that makes the Pareto allocation be the  allocation associated with  a competitive
equilibrium with time $0$ trading of history-contingent claims on consumption.

\medskip
\noindent {\bf e.}  Find a formula for the equilibrium price vector in terms of equilibrium
quantities and the beliefs of consumers.

\medskip
\noindent {\bf f.} Suppose that $I=2$.  Show that as $\lambda_2 / \lambda_1 \rightarrow + \infty$, the planner would distribute
initial wealth in a way that makes consumer $2$'s beliefs more and more influential in determining  equilibrium prices.


\medskip
\noindent{\it Exercise \the\chapternum.15} \quad {\bf Diverse beliefs, III}
\medskip
\noindent An economy consists of two consumers named $i=1,2$.   Each
consumer evaluates streams of a single nonstorable consumption good according
to $$ \sum_{t=0}^\infty \sum_{s^t} \beta^t \ln [c_t^i(s^t)]
\pi_t^i(s^t).
 $$  Here $\pi_t^i(s^t)$ is consumer $i$'s subjective probability over history $s^t$. %and  $u(c)$ is an increasing,
%twice continuously differentiable, strictly concave function of
%consumption $c$  of one good. The utility function satisfies the
%Inada condition $ \lim_{c \downarrow 0} u'(c) = +\infty.$
 A
feasible allocation satisfies $\sum_i c_t^i(s^t) \leq \sum_i
y^i(s_t) $ for all $t \geq 0$ and for all $s^t$.  The consumers' endowments of the one good are  functions of a state variable $s_t \in {\bf S} =
\left\{0, 1, 2\right\}$.  In truth,  $s_t$ is described by a time invariant
Markov chain with initial distribution $\pi_0 = \left[\matrix{0 &
1 & 0 \cr}\right]' $ and transition density defined by the
stochastic matrix
$$P = \left[\matrix{ 1 & 0 & 0 \cr .5 & 0 & .5 \cr 0 & 0 & 1\cr}\right]$$
where $P_{ij} = {\rm Prob}[s_{t+1} = j-1| s_t = i-1]$ for $i=1,2,3$ and $j=1,2,3$.  The endowments of the two consumers
are
$$\eqalign{ y_t^1 & = s_t /2 \cr
            y_t^2 & = 1-s_t/2 .\cr} $$
In part I, both consumers know the true probabilities over histories $s^t$ (i.e., they
know both $\pi_0$ and $P$).  In part II, the two consumers have different subjective probabilities.

\vfil\eject

\medskip
\noindent{\bf Part I:}
\medskip
\noindent Assume that both consumers know $(\pi_0, P)$, so that
$\pi_t^1(s^t) = \pi_t^2(s^t)$ for all $t\geq 0$ for all $s^t$.


\medskip
\noindent{\bf a.}  Show how to deduce $\pi_t^i(s^t)$ from
$(\pi_0, P)$.

 \medskip
\noindent{\bf b.}  Define a competitive equilibrium with sequential trading of Arrow
securities.
\medskip
\noindent{\bf c.}  Compute a competitive equilibrium with sequential trading of Arrow
securities.
\medskip
\noindent {\bf d.} By hand, simulate the economy.  In particular,
for every possible realization of the histories $s^t$, describe
time series of $c_t^1, c_t^2$ and the wealth levels for the two consumers.

\medskip
\noindent{\bf Part II:}
\medskip

\noindent Now assume that while consumer 1 knows $(\pi_0, P)$, consumer 2 knows $\pi_0$ but thinks
that $P$ is
$$\hat P= \left[\matrix{ 1 & 0 & 0 \cr .4 & 0 & .6 \cr 0 & 0 & 1\cr}\right].$$
\medskip
\noindent {\bf e.} Deduce $\pi_t^2(s^t)$ from $(\pi_0, \hat P)$ for all $t \geq 0$ for all $s^t$.
\medskip
\noindent{\bf f.} Formulate and solve a Pareto problem for this economy.
\medskip
\noindent{\bf g.} Define an equilibrium with time $0$ trading of a complete set of
Arrow-Debreu history-contingent securities.
\medskip
\noindent{\bf h.} Compute  an equilibrium with time $0$ trading of a complete set of
Arrow-Debreu history-contingent securities.

\medskip
\noindent{\bf i.} Compute an equilibrium with sequential trading of Arrow securities.  For every
possible realization of $s^t$ for all $t\geq 0$, please describe time series of $c_t^1, c_t^2$ and the wealth levels for the two consumers.


%%%%% HHHH new question Fall 2008


\medskip
\noindent{\it Exercise \the\chapternum.16} \quad {\bf Diverse beliefs, IV}
\medskip
\noindent
 A pure exchange economy is populated by two consumers. Consumer $i$ has preferences over history-contingent
consumption sequences $\{c_t^i(s^t)\}$ that are ordered by
$$ \sum_{t=0}^\infty \sum_{s^t} \beta^t u(c_t^i(s^t)) \pi_t^i (s^t) ,$$
where $u(c) = \ln(c)$, $\beta \in (0,1)$,   and  $\pi_t^i(s^t)$ is a density that consumer $i$ assigns to history $s^t$. The state space is time invariant. In particular,
$s_t \in S = \{0, .5,  1\}$ for all $t \geq 0$.
Only two histories are possible for $t=0, 1, 2, \ldots$:
$$ \eqalign{ & {\rm history \ 1}: .5, 1, 1, 1, 1, \ldots \cr
             & {\rm history \ 2}: .5, 0, 0, 0, 0, \ldots \cr } $$
Consumer $1$ assigns probability $1$ to  history 1  and probability $0$ to history 2, while consumer 2 assigns probability
$0$ to history 1 and probability $1$ to history 2.  Nature assigns equal probabilities to the two histories.
The endowments of the two consumers are:

$$ \eqalign{ y_t^1 & = s_t \cr
          y_t^2 & = 1-s_t . \cr} $$

\medskip
\noindent{\bf a.} Formulate and solve a  Pareto problem for this economy.

\medskip
\noindent{\bf b.} Define a competitive  equilibrium with sequential trading of a complete set of one-period Arrow securities.

\medskip
\noindent {\bf c.} Does a competitive equilibrium with sequential trading of a complete set of one-period Arrow securities
exist for this economy?  If it does, compute it.  If it does not, explain why.



\medskip
\noindent{\it Exercise \the\chapternum.17} \quad {\bf Diverse beliefs, V}
\medskip
\noindent
 A pure exchange economy is populated by two consumers. Consumer $i$ has preferences over history-contingent
consumption sequences $\{c_t^i(s^t)\}$ that are ordered by
$$ \sum_{t=0}^\infty \sum_{s^t} \beta^t u(c_t^i(s^t)) \pi_t^i (s^t) ,$$
where $u(c) = \ln(c)$, $\beta \in (0,1)$,   and $\pi_t^i(s^t)$ is a density that consumer $i$ assigns to history $s^t$. The state space is time invariant. In particular,
$s_t \in S = \{0, .5,  1\}$ for all $t \geq 0$.
Only two histories are possible for $t=0, 1, 2, \ldots$:
$$ \eqalign{ & {\rm history \ 1}: .5, 1, 1, 1, 1, \ldots \cr
             & {\rm history \ 2}: .5, 0, 0, 0, 0, \ldots \cr } $$
Consumer $1$ assigns probability $1$ to  history 1  and probability $0$ to history 2, while consumer 2 assigns probability
$0$ to history 1 and probability $1$ to history 2.  Nature assigns equal probabilities to the two histories.
The endowments of the two consumers are:

$$ \eqalign{ y_t^1 & = 1- s_t \cr
          y_t^2 & = s_t . \cr} $$

\medskip
\noindent{\bf a.} Formulate and solve a  Pareto problem for this economy.

\medskip
\noindent{\bf b.} Define a competitive  equilibrium with sequential trading of a complete set of one-period Arrow securities.

\medskip
\noindent {\bf c.} Does a competitive equilibrium with sequential trading of a complete set of one-period Arrow securities
exist for this economy?  If it does, compute it.  If it does not, explain why.


\medskip
\noindent{\it Exercise \the\chapternum.18} \quad {\bf Risk-free bonds}

\medskip
\noindent
An economy consists of a single representative consumer who  ranks streams of a single nonstorable consumption good according
to $ \sum_{t=0}^\infty \sum_{s^t} \beta^t \ln [c_t(s^t)]
 \pi_t(s^t)$.  Here $\pi_t(s^t)$ is the  subjective probability that the consumer attaches to  a history  $s^t$ of a Markov state $s_t$, where
 $s_t \in \{1, 2\}$. Assume that the subjective probability $\pi_t(s^t)$  equals the objective probability.  %and  $u(c)$ is an increasing,
%twice continuously differentiable, strictly concave function of
%consumption $c$  of one good. The utility function satisfies the
%Inada condition $ \lim_{c \downarrow 0} u'(c) = +\infty.$
  Feasibility for this pure endowment economy is expressed by the condition $c_t \leq y_t$, where $y_t$ is the endowment at time
  $t$.  The endowment is exogenous and governed by
$$ y_{t+1} = \lambda_{t+1} \lambda_t \cdots \lambda_1 y_0$$
 for $t \geq 0$ where  $y_0 > 0$.  Here $\lambda_t$ is a function of the Markov state $s_t$.  Assume that
$\lambda_t = 1$ when $s_t=1$ and $\lambda _t = 1 + \zeta$ when $s_t =2 $, where $\zeta >0$. States  $s^t =[s_t, \ldots, s_0]$ are known
at time $t$, but future states are not.    The state   $s_t$ is described by a time invariant
Markov chain with initial probability distribution $\pi_0 = \left[\matrix{
1 & 0 \cr}\right]' $ and transition density defined by the
stochastic matrix
$$P = \left[\matrix{ P_{11} & P_{12}\cr
                      P_{21} & P_{22} }\right]$$
where $P_{ij} = {\rm Prob}[s_{t+1} = j| s_t = i]$ for $i=1,2$ and $j=1,2$. Assume that $P_{ij} \geq 0$ for all  pairs $(i,j)$.





\medskip
\noindent{\bf a.}  Show how to deduce $\pi_t(s^t)$ from
$(\pi_0, P)$.

 \medskip
\noindent{\bf b.}  Define a competitive equilibrium with sequential trading of Arrow
securities.
\medskip
\noindent{\bf c.}  Compute a competitive equilibrium with sequential trading of Arrow
securities.
\medskip
\noindent {\bf d.} Let $p_t^b$ be the time $t$ price of a risk-free claim to one unit of consumption at time $t+1$.
Define a competitive equilibrium with sequential trading of  risk-free claims to consumption one period ahead.
\medskip
\noindent {\bf e.} Let $R_t = (p_t^b)^{-1}$ be the  one-period risk-free gross interest rate.
Give a formula for $R_t$  and tell how it depends on the history $s^t$.
\medskip
\noindent {\bf f.}  Suppose that $\beta = .95, \zeta = .02$ and $P =  \left[\matrix{ 1 &  0 \cr
                      .5 & .5 }\right]$.  Please compute $R_t$ when $s_t =1$.  Then compute $R_t$ when
                      $s_t =2$.
\medskip
\noindent{\bf g.} What parts of your answers depend on assuming that the subjective probability $\pi_t(s^t)$
equals the objective probability?

\medskip

\noindent{\it Exercise \the\chapternum.19} \quad {\bf Entropy}

\medskip

\noindent
Let $\tilde f(s)$ and $f(s)$ be two alternative probability density functions for a random variable $s$.
Define the relative entropy of $f$ with respect to $\tilde f$ by
$$ {\rm ent} (f, \tilde f) = \int \log \Biggl( { \tilde f(s) \over f(s)  }\Biggr) \tilde f(s) d s = \int \log \Biggl( { \tilde f(s) \over f(s)  }\Biggr) \Biggl( { \tilde f(s) \over f(s)  }\Biggr)
f(s) d s .$$

\medskip
\noindent Prove that ${\rm ent} (f, \tilde f) \geq 0$ and that ${\rm ent} (\tilde f, \tilde f) = 0$.

\medskip

\noindent {\bf Hint 1:}  Define $m(s)  = \Bigl( { \tilde f(s) \over f(s)  }\Bigr)$ and express entropy
as
$$ \int m(s) \log m(s) f(s) ds .$$

\medskip

\noindent {\bf Hint 2:}  Verify that $m \log m \geq m - 1 $.

\medskip

\noindent {\bf Hint 3:} Verify that $\int m (s) f(s) = 1$.

\medskip

\noindent{\it Exercise \the\chapternum.20} \quad {\bf Heterogeneous  beliefs again}

\medskip

\noindent Consider an example of the heterogeneous beliefs economy described in Appendix \use{app:compB}  of this  chapter.
Suppose  that $S = \{0, 1\},  I=2$, and that  agent $i$ believes that $s_t$ is independently and identically distributed
with ${\rm Prob}(s_t = 0) = \alpha_i$. Assume that $u_i(c^i_t) = \log c^i_t$ for both consumers. The aggregate endowment is $y_t(s^t) = 1 - .5 s_t$. Feasibility requires
$c_t^1 + c_t^2 \leq y_t(s^t)$.  A time $0$ Pareto planner puts weight $\lambda \in (0,1)$ on consumer 1 and weight $(1-\lambda)$ on consumer 2.
Suppose that $\{s_t\}_{t=1}^\infty$ is truly independently and identically distributed with ${\rm Prob}(s_t = 0) = \tilde \alpha$.
Assume that $I=2, \alpha_2 = \tilde \alpha, \alpha_1 \neq \tilde \alpha$, so that consumer 2 knows the true distribution, while consumer 1 does not.

\medskip

\noindent{\bf a.}  Let $L^2_t(s^t) = {\pi_t^2(s^t) \over \pi_t^1(s^t)}$.  Show that a Pareto optimal allocation satisfies
$$ c_t^2(s^t) = \Biggl[ { (1-\lambda) L^2(s^t) \over \lambda + (1-\lambda) L_t^2(s^t) } \Biggr] y_t(s^t) .$$

\medskip
\noindent{\bf b.}  Find a formula for the price $\tilde Q(s_{t+1} | s^t)$ of Arrow securities.

\medskip
\noindent{\bf c.} State assumptions under which you can show that
$$ \tilde Q_t(s_{t+1} | s^t) \rightarrow  \beta \Biggl( {y_t(s_t) \over y_{t+1}(s_{t+1})} \Biggr) \pi_t^2(s_{t+1}| s^t), $$
as $t \rightarrow +\infty$ in some sense. Describe the sense in which what you claim is true.

\medskip
\noindent{\bf d.} Use your findings from part {\bf c} to argue that the ``tail'' of the economy resembles a representative
consumer economy.

\medskip

\noindent{\it Exercise \the\chapternum.21} \quad {\bf Recursive version of Pareto problem}

\medskip


\noindent {\bf a.}  Please reread section  \use{sec:RecursivePareto}.

\medskip

\noindent {\bf b.}  Assume exactly the same setting as in section  \use{sec:RecursivePareto}, except now assume that consumer  $1$ believes probabilities $\Pi_s^1$ and that consumer two  believes
probabilities $\Pi_s^2$.

\medskip

\noindent {\bf c.}  Formulate a recursive version of the Pareto problem in terms of the Bellman equation
$$  P(v) = \max_{\{c_s, w_s\}} \sum_{i=s}^S [ u_2(1-c_s) + \beta P(w_s) ] \Pi_s^2  $$
where the maximization is subject to
$$ \sum_{s=1}^S [u_1(c_s) + \beta w_s ] \Pi_s^1 \geq v  $$
where $c_s$ is consumption assigned to consumer 1, $1-c_s$ is consumption assigned to consumer 2,  $v$ is the promised value assigned to consumer 1,
$w_s$ is a continuation  value assigned to
consumer 1,
$P(v)$ is the value attained by consumer 2, and $P(w_s)$ is the continuation value assigned to consumer 2.

\medskip

\noindent{\bf d.}  Assuming that $P(v)$ is concave and differentiable, find first-order conditions for $c_s, w_s$ for $s = 1, \ldots, S$.

\medskip


\noindent{\bf e.}  Argue that these first-order conditions imply that for states $i$ such that ${\Pi_s^1 \over \Pi_s^2} > 1$, the Pareto planner
gives larger values of  {\it both\/} $c_s^1$ and $w_s^1$ than he or she does in states for which  ${\Pi_s^1 \over \Pi_s^2} < 1$.


\medskip
\noindent {\bf f.}  How do the outcomes described in item {\bf e} relate to the asymptotic outcomes described in Appendix \use{app:compB}  of this  chapter?

\medskip

\noindent{\it Exercise \the\chapternum.22} \quad {\bf Recovering probabilities from prices}
\medskip
\noindent Consider a representative agent economy with an exogenous endowment of a single consumption good governed by a two-state Markov chain
with state $s_t \in \{ 1, 2\}$ having transition  matrix $P$ with typical element $P_{ij} = {\rm Prob} (s_{t+1} = j | s_t = i)$.
The representative consumer's endowment $y_t$ is a time invariant function  $y(s_t)> 0$ of the Markov state. So is his consumption.
The representative consumer orders consumption streams by
$$ E_0 \sum_{t=0}^\infty \beta^t u(c_t), $$
where $\beta \in (0,1)$ and $u(c)= {\frac{1}{1-\gamma}}c^{1-\gamma}$ for $\gamma > 0$.
At each date $t \geq 0$, there are complete markets in $h$-step ahead Arrow securities of horizons $h=1,2$.
Recall formula \Ep{Qrecurs2} for the $h$-step ahead Arrow securities prices, which  for convenience we repeat here
$$ Q_h( s_{t+h} | s_t) = \sum_{s_{t+1}} Q_1(s_{t+1} | s_t)
                      Q_{h-1}( s_{t+h} | s_{t+1}). $$

\medskip
\noindent{\bf a.} With an abuse of notation, arrange the one-period Arrow security prices into a $2 \times 2$ matrix $Q$ defined
according to
$$ Q_{ij} = Q(s_{t+1} = j| s_t = i). $$
Then show that formula \Ep{Qrecurs2} can be represented as
$$ Q^h = Q Q^{h-1} $$
for $h \geq 2$.

\medskip
\noindent {\bf b.}  Suppose that you observe a complete set of one and two period ahead Arrow securities prices for {\it one\/} and only one state $i$.  That is,
you know the $i$th row of both $Q$ and $Q^2$.  Show how you can recover both rows of $Q$ from this information.

\medskip
\noindent{\bf c.}  Recall  that consumption of a representative consumer is a time-invariant function of the Markov state.
Find a formula for each element of $Q$  in terms of $\beta, \gamma$, the consumption rates in the two states, and  $P$.

\medskip
\noindent{\bf d.} Prove that if you  know all elements of $Q$, you can infer $\beta$ and all elements of $P$ even if you don't know  $\gamma$.


\medskip

\noindent{\it Exercise \the\chapternum.23} \quad {\bf Recovering probabilities from prices, II}
\medskip
\noindent Consider a representative agent economy with an exogenous endowment of a single consumption good whose geometric growth rate is
governed by  a two-state Markov chain
with state $s_t \in \{ 1, 2\}$ having transition  matrix $P$ with typical element $P_{ij} = {\rm Prob} (s_{t+1} = j | s_t = i)$.
The growth rate of the consumer's endowment $y_t$ is a time invariant function  $\lambda(s_t)> 0$ of the Markov state.
The endowment obeys $y_{t+1} = \lambda_{t+1} y_t$ for $t \geq 0$.
The representative consumer orders consumption streams by
$$ E_0 \sum_{t=0}^\infty \beta^t u(c_t), $$
where $\beta \in (0,1)$ and $u(c)= {\frac{1}{1-\gamma}}c^{1-\gamma}$ for $\gamma > 0$.
There are complete markets in one and two period Arrow securities.


\medskip
\noindent{\bf a.} With an abuse of notation, arrange the one-period Arrow security prices into a $2 \times 2$ matrix $Q$ defined
according to
$$ Q_{ij} = Q(s_{t+1} = j| s_t = i). $$
Then show that formula \Ep{Qrecurs2} can be represented as
$$ Q^h = Q Q^{h-1} $$
for $h \geq 2$.


\medskip
\noindent {\bf b.}  Suppose that you observe all one and two period  Arrow securities prices for a single state $i$.  That is,
you know the $i$th row of both $Q$ and $Q^2$.  Show how you can recover both rows of $Q$ from this information.


\medskip
\noindent{\bf c.}  Find a formula for each element of $Q$  in terms of $\beta, \gamma$, and $P$.

\medskip
\noindent{\bf d.}   Prove that if you  know all elements of $Q$, you {\it cannot} infer $\beta$ and all elements of $P$ even if you know  $\gamma$.


\medskip

\noindent{\it Exercise \the\chapternum.24} \quad {\bf Recovering probabilities from prices, III}
\medskip
\noindent Consider a representative agent economy with an exogenous endowment of a single consumption good governed by a two-state Markov chain
with state $s_t \in \{ 1, 2\}$ having transition  matrix $P$ with typical element $P_{ij} = {\rm Prob} (s_{t+1} = j | s_t = i)$.
The representative consumer's endowment $y_t$ is a time invariant function  $y(s_t)> 0$ of the Markov state. So is his consumption $c_t = c(s_t)$.
The representative consumer orders consumption streams by
$$ E_0 \sum_{t=0}^\infty \beta^t u(c_t), $$
where $\beta \in (0,1)$,  $u(c)= {\frac{1}{1-\gamma}}c^{1-\gamma}$ for $\gamma > 0$, and $E_0$ denotes a mathematical expectation conditioned on date
$0$ information, in this case the time $0$ value  $s_0$ of the state.
At each date $t \geq 0$, there are complete markets in one-step ahead Arrow securities.
There are also markets in $h$-period risk-free pure discount  bonds.  A risk-free pure discount $h$-period bond issued at $t$ promises to pay one unit of
consumption at date $t+h$ for sure.  At time $t$, the equilibrium price of a sure claim on one unit of  consumption
at time $t+h$ is $p^t_{t+h}$.

\medskip
\noindent{\bf a.} Please present a formula for a stochastic discount factor for pricing $h$-period ahead risky claims for this economy.
Is this stochastic discount factor unique?

\medskip
\noindent {\bf b.}  Please find formulas for the equilibrium  price $p^t_{t+h}$  of $h$-period  risk-free pure discount bonds for $h \geq 1$.


\medskip
\noindent{\bf c.}  At a single date  $t$, you observe the state of  the economy $s_t$. You also observe $p^t_{t+h}$ for $h=1,2,3,4$.
Your friend tells you that from these four observations you can infer $\beta, P_{11}, P_{22}$, and $\left({\frac{c(1)}{c(2)}}\right)^{-\gamma}$ for this economy.
Do you agree?  Provide an argument.






\eqnotracefalse
