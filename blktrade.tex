
\input grafinp3
\input psfig
%\showchaptIDtrue
%\def\@chaptID{4.}
%%\eqnotracetrue
%\hbox{}

\chapter{Two Topics in International Trade\label{wldtrade}}
\footnum=0
\section{Two dynamic contracting problems}

This chapter studies two models in which recursive contracts are
used to overcome incentive problems commonly thought to occur in
international trade.  The first is Andrew Atkeson's model of
lending in the context of a dynamic setting that contains both a
moral hazard problem due to asymmetric information {\it and\/} an
enforcement problem due to borrowers' option to disregard the
contract. It is a considerable technical achievement that Atkeson
managed to include both of these elements in his contract design
problem.  But this substantial technical accomplishment is not
just showing off. As we shall see, {\it both\/} the moral hazard
{\it and\/} the self-enforcement requirement for the contract are
required in order to explain the feature of observed repayments
that Atkeson was after: that the occurrence of especially low
output realizations prompt the contract to call for net repayments
from the borrower to the lender, exactly the occasions when an
unhampered insurance scheme would have lenders extend credit to
borrowers.

The  second is Bond and Park's model of a recursive contract that induces two countries
to adopt free trade when they begin with a pair of
 promised values that implicitly determine the distribution
of eventual welfare gains from trade liberalization.
%%%a Pareto noncomparable initial condition.
The new
policy is accomplished by a gradual relaxation of tariffs, accompanied by
trade concessions.  Bond and Park's model of gradualism is all about the
dynamics of promised values that are used optimally to manage participation
constraints.
 \auth{Atkeson, Andrew} \auth{Bond, Eric W.} \auth{Park,
Jee-Hyeong} \index{lending with moral hazard}
\section{Lending with moral hazard and difficult enforcement}
Andrew Atkeson (1991) designed a model to explain how, in defiance
of the pattern predicted by complete markets models, low output
realizations in various countries in the mid-1980s prompted
international lenders to ask those countries for net repayments.  A complete
markets model would have net flows to a borrower during
periods of bad endowment shocks.  Atkeson's idea was that
information and enforcement problems could produce the
observed  outcome.  Thus,  Atkeson's model combines two
features of the  models we have seen in chapter \use{socialinsurance}:
incentive problems from private information and participation
constraints coming from enforcement problems.


Atkeson showed that the optimal contract  handles enforcement and information problems
through the shape of the repayment schedule, thereby indirectly manipulating continuation values.
  Continuation values respond only by updating
a single state variable, a measure of resources available to the
borrower, that appears in the optimum value function, which in turn
is affected only through the repayment schedule.  Once this state
variable is taken into account, promised values do not appear as
independently manipulated state variables.\NFootnote{To understand
how Atkeson achieves this outcome, the reader should also digest
the approach described in chapter \use{credible}.}

Atkeson's model brings together several features.  He studies
a ``borrower'' who by himself is situated like a planner
in a stochastic growth model, with the only vehicle for saving
being a stochastic investment technology.  Atkeson adds the
possibility that the planner can also borrow
subject to both participation and information constraints.

A borrower lives for $t=0,1,2,\ldots $.  He begins life with $Q_{0}$
units of a single good.  At each date $t\geq 0$, the borrower has access
to an investment technology.  If $I_t\geq 0$ units of the good are
invested at  $t$, $Y_{t+1}=f(I_{t},\varepsilon_{t+1})$
units of time $t+1$  goods are available, where $\varepsilon_{t+1}$ is
an i.i.d. random variable.
  Let  $g(Y_{t+1}, I_{t})$ be the
probability density of  $Y_{t+1}$  conditioned on  $I_{t}$. It is
assumed that increased investment shifts the distribution of returns
toward higher returns.

  The borrower has preferences over consumption streams ordered by
$$  (1-\delta)E_0 \sum_{t=0}^\infty \delta^t u(c_t) \EQN atkes1 $$
where $\delta \in (0,1)$ and $u(\cdot)$ is increasing, strictly concave,
twice continuously differentiable, and $u'(0) = +\infty$.

Atkeson used various technical conditions to render his model
tractable. He assumed that for each investment $I$, $g(Y, I)$
 has finite support  $\left(Y_{1},\ldots, Y_{n}\right)$, with
$Y_n > Y_{n-1} > \ldots >Y_1$.  He assumed that  $g(Y_{i}, I)>0$
  for all values of  $I$ and all states $Y_{i}$, making it
impossible precisely to infer $I$  from $Y$.
He further assumed that the distribution $g\left( Y,  I\right)$
is given by the convex combination of two underlying distributions
$g_0(Y)$ and $g_1(Y)$ as follows:
$$g( Y, I) =\lambda(I) g_{0}(Y)
+[1-\lambda( I) ] g_{1}( Y),  \EQN assumpt5 $$
where $g_{0}(Y_i)/g_1(Y_i) $  is monotone  and increasing
in  $i$,  $0\leq \lambda(I) \leq 1$, $\lambda^{\prime}
\left( I\right) >0$,  and \ $\lambda^{\prime \prime }\left( I\right) \leq
0 $  for all  $I$.   Note that
$$
g_I(Y,I) = \lambda'(I) [g_0(Y) - g_1(Y)],  \EQN assumpt5_deriv
$$
where $g_I$ denotes the derivative with respect to $I$. Moreover,
the assumption that increased investment shifts the distribution
of returns toward higher returns implies
$$
\sum_i Y_i \left[g_0(Y_i) - g_1(Y_i) \right] >0.       \EQN assumpt5b
$$

We shall consider the borrower's choices in three environments: (1)
autarky, (2) lending from risk-neutral lenders under complete observability
of the borrower's choices and complete enforcement,  and (3) lending
under incomplete observability and  limited enforcement. Environment
3 is Atkeson's.
%For expositional simplicity, we will initially proceed as if $Y$ is
%a continuous variable.
 We  can use environments 1 and 2 to construct
bounds on the value function for performing computations described
in an appendix.

\subsection{Autarky}

Suppose that there are no lenders.  Thus, the ``borrower'' is
just an isolated household endowed with the technology. The
household chooses $(c_{t},I_{t})$ to maximize expression
\Ep{atkes1} subject to
$$\eqalign{ c_{t}+I_{t} & \leq Q_{t}  \cr
 Q_{t+1} & =Y_{t+1} .\cr} $$
The optimal  value function $U(Q)$  for this problem  satisfies
the Bellman equation
$$U(Q)=\max_{Q\geq I \geq 0}\Bigl\{ (1-\delta)\,
u(Q-I)+\delta \sum_{Q'} U\bigl(Q^{\prime}\bigr)g(Q^{\prime},I)
    \Bigr\}.
 \EQN bellmanaut $$
The first-order condition for $I$ is
$$ -(1-\delta) u'(Q-I) + \delta \sum_{Q'} U(Q') g_I(Q',I)
                                      \leq 0, \hskip.5cm = 0 \;\hbox{\rm if}\; I>0.
   \EQN foncautark $$
%%for $0<I<Q$.
This first-order condition implicitly defines a rule for
accumulating capital under autarky.

\index{complete markets}
\subsection{Investment with full insurance}
We now consider an environment in which in addition
to investing $I$ in the technology, the borrower
can issue \idx{Arrow securities} at a vector of prices $q(Y',I)$, where
we let $'$ denote next period's values, and $d(Y')$ the
quantity of one-period  Arrow securities issued by the borrower;
$d(Y')$ is the number of units of next period's consumption
good that the borrower promises to deliver.
Lenders observe the level of investment $I$, and so the pricing kernel
$q(Y',I)$ depends explicitly on $I$.  Thus, for  a promise to pay
one unit of output next period contingent on
next-period output realization
$Y'$, for each level of $I$, the borrower faces a different price.
(As we shall soon see,
in Atkeson's model  lenders cannot observe $I$, making it impossible
to condition the price on $I$.)
We shall assume that the Arrow securities are priced by
 risk-neutral investors who also have
one-period discount factor $\delta$.
This implies that the price
of Arrow securities is given by
$$ q(Y',I) = \delta g(Y', I),  \EQN atnew4 $$
which in turn implies that the
gross one-period risk-free interest rate is $\delta^{-1}$.

In a complete markets world where there is no problem with information
or enforcement, the borrower's optimal investment decision is not
a function of the borrower's own holdings of the good. Instead, the
optimal investment level maximizes the project's
present value when evaluated at the prices for Arrow securities:
$$\max_{I \geq 0}\Bigl\{ -I+ \sum_{Y'} Y' q(Y',I)\Bigr\}, \EQN atnew_invest
$$
and after imposing expression \Ep{atnew4}
$$\max_{I \geq 0}\Bigl\{ -I+ \delta \sum_{Y'} Y' g(Y',I)\Bigr\}.
$$
Hence, when the prices for Arrow securities are determined by
risk-neutral investors, the optimal investment level maximizes
the project's expected payoffs discounted at the
risk-free interest rate $\delta^{-1}$. The first-order condition
for $I$ is
$$ \sum_{Y'} Y' g_I(Y',I)
                  \leq \delta^{-1}, \hskip.5cm = \delta^{-1}\;\hbox{\rm if}\; I>0;
   \EQN atnew5
$$
and after invoking equation \Ep{assumpt5_deriv}
$$
\lambda'(I) \sum_{Y'} Y'
 \left[g_0(Y') - g_1(Y') \right]  \leq \delta^{-1},
                          \hskip.5cm = \delta^{-1}\;\hbox{\rm if}\; I>0.
$$
This condition uniquely determines the investment level $I$, since
the left side is decreasing in $I$ and must eventually approach
zero because of the upper bound on $\lambda(I)$.


As in chapter \use{recurge}, we formulate the
 borrower's budget constraints
recursively as
$$\EQNalign{ c & - \sum_{Y'} q(Y',I^*) d(Y')  + I^*   \leq Q \EQN atnew1;a \cr
             Q' & = Y' - d(Y'), \EQN atnew1;b \cr} $$
where $I^*$ is the solution to investment problem \Ep{atnew_invest}.
Let $W(Q)$ be the optimal value for a borrower with goods $Q$.
The borrower's  Bellman  equation is
$$\eqalign{ W(Q)   = \max_{c, d(Y')} \bigg\{ &(1-\delta) u(c) + \delta
     \sum_{Y'} W[Y' - d(Y')] g(Y',I^*)   \cr
     & + \mu[ Q - c +\sum_{Y'} q(Y',I^*) d(Y')  - I^*] \biggr\},  \cr}
   \EQN atnew2 $$
where $\mu$ is a Lagrange multiplier on expression \Ep{atnew1;a}.
First-order conditions with respect to
$c, d(Y')$, respectively, are
$$ \EQNalign{ c\rm{:}  & \ \ (1-\delta) u'(c) - \mu= 0, \EQN atnew3;a \cr
  d(Y')\rm{:} &   \ \ - \delta W'[Y' - d(Y')] g(Y',I^*) + \mu q(Y',I^*)  = 0.
       \EQN atnew3;b \cr } $$
By substituting \Ep{atnew4} and \Ep{atnew3;a} into first-order condition
\Ep{atnew3;b}, we obtain
$$
  - W'[Y' - d(Y')] + (1-\delta) u'(c) = 0,
$$
and after invoking the Benveniste-Scheinkman
condition, $W'(Q')=(1-\delta)u'(c')$, we arrive at the consumption-smoothing
result $c' = c$.  This in turn implies, via the status of
$Q$ as the state variable in the Bellman
equation, that $Q' = Q = Q_0$.  Thus, the
solution has $I$ constant over time at a level $I^*$ determined
by equation \Ep{atnew5}, and $c$ and the functions $d(Y')$
satisfying
$$\EQNalign{ c  + I^* & =  Q_0 + \sum_{Y'} q(Y',I^*) d(Y')   \EQN atnew6;a \cr
               d(Y') & =  Y' - Q_0. \EQN atnew6;b \cr} $$
The borrower borrows a constant $\sum_{Y'} q(Y',I^*) d(Y')$ each period,
invests the same  $I^*$ each period,  and makes high repayments when
$Y'$ is high and low repayments when $Y'$ is low.  This
is the standard full-insurance solution.

We now turn to Atkeson's  setting where  the borrower does
better than under autarky but worse than with the
loan contract under perfect enforcement and observable
investment.  Atkeson found a contract
with value $V(Q)$ for which $U(Q) \leq V(Q) \leq W(Q)$.
We shall want to compute $W(Q)$ and $U(Q)$ in order to
compute the value of the borrower
under the more    restricted contract.




%%%%%%%%%%%%%%%%%%%%%%%%%%%%%%%%%%%%%%%%%%%%%%%%%%%%%%%%%%%%%%
%As in chapter \use{recurge}, we formulate the
% borrower's budget constraints
%recursively as
%$$\EQNalign{ c & - \sum_{Y'} q(Y',I) d(Y')  + I   \leq Q \EQN atnew1;a \cr
%             Q' & = Y' - d(Y'). \EQN atnew1;b \cr} $$
%Let $W(Q)$ be the optimal value for a borrower with goods $Q$.
%The borrower's  Bellman  equation is
%$$\eqalign{ W(Q)  & = \max_{c, I, d(Y')} \bigg\{ (1-\delta) u(c) + \delta
%     \sum_{Y'} W[Y' - d(Y')] g(Y',I)   \cr
%     & + \lambda [ Q - c +\sum_{Y'} q(Y',I) d(Y')  - I] \biggr\},  \cr}
%   \EQN atnew2 $$
%where $\lambda$ is a Lagrange multiplier on expression \Ep{atnew1;a}.
%First-order conditions with respect to
%$c, I, d(Y')$, respectively, are
%$$ \EQNalign{ c\rm{:}  & \ \ (1-\delta) u'(c) - \lambda = 0, \EQN atnew3;a \cr
%            I\rm{:} &  \ \ \delta \sum_{Y'}
%    W[Y' - d(Y')] g_I(Y', I) +  \cr & \quad \quad
%    \lambda q_I(Y',I) d(Y') - \lambda
%    = 0, \EQN atnew3;b \cr
%  d(Y')\rm{:} &   \ \ - \delta W'[Y' - d(Y')] g(Y',I) + \lambda q(Y',I)  = 0.
%       \EQN atnew3;c \cr } $$
%Letting risk-neutral lenders determine the price
%of Arrow securities implies that
%$$ q(Y',I) = \delta g(Y', I),  \EQN atnew4 $$
%which in turn implies that the
%gross one-period risk-free interest rate is $\delta^{-1}$.
%At these prices for Arrow securities, it is profitable to invest in the
%stochastic technology until the expected rate of return on the marginal
%unit of investment is driven down to $\delta^{-1}$:
%$$\EQNalign{
%&\sum_{Y'}
%    [Y'-d(Y')] g_I(Y',I)  = \delta^{-1},             \EQN atnew5 \cr
%\noalign{\hbox{\rm and after invoking equation \Ep{assumpt5}}}
%&\lambda'(I) \sum_{Y'} [Y' - d(Y')]
% \left(g_0(Y') - g_1(Y') \right)  = \delta^{-1}. \cr}
%$$
%This condition uniquely determines the investment level $I$, since
%the left side is decreasing in $I$ and must eventually approach
%zero because of the upper bound on $\lambda(I)$. (The investment level
%is strictly positive as long as the left-hand side exceeds $\delta^{-1}$
%when $I=0$.)
%
%The first-order condition \Ep{atnew3;c} and the Benveniste-Scheinkman
%condition, $W'(Q')=(1-\delta)u'(c')$, imply the consumption-smoothing
%result $c' = c$.  This in turn implies, via the status of
%$Q$ as the state variable in the Bellman
%equation, that $Q' = Q$.  Thus, the
%solution has $I$ constant over time at a level determined
%by equation \Ep{atnew5}, and $c$ and the functions $d(Y')$
%satisfying
%$$\EQNalign{ c  + I & =  Q + \sum_{Y'} q(Y',I) d(Y')   \EQN atnew6;a \cr
%               d(Y') & =  Y' - Q \EQN atnew6;b \cr} $$
%The borrower borrows a constant $\sum_{Y'} q(Y',I) d(Y')$ each period,
%invests the same  $I$ each period,  and makes high repayments when
%$Y'$ is high and low repayments when $Y'$ is low.  This
%is the standard full-insurance solution.
%%%%%%%%%%%%%%%%%%%%%%%%%%%%%%%%%%%%%%%%%%%%%%%%%%%%%%%%%%%%%%%%%%%%%%%%%%%%%%%%%%%%%


%%%%
\subsection{Limited commitment and unobserved investment}

Atkeson designed an optimal recursive contract that copes with two
impediments to risk sharing:
(1) moral hazard, that is,  hidden action:
  the lender cannot observe the borrower's action
$I_{t}$ that affects the probability distribution of returns $Y_{t+1}$; and
(2) one-sided limited commitment:  the borrower is free to default on the
contract and can choose to revert to autarky at any state.

Each period, the borrower confronts a
two-period-lived, risk-neutral lender who
is endowed with $M >0$ in each period of his life. Each lender
can lend or borrow at a risk-free gross interest rate of $\delta^{-1}$ and
must earn an expected return of at least $\delta^{-1}$ if
he is to lend to the borrower.  The lender is also willing to
{\it borrow\/} at this same expected rate of return.
  The lender
can lend up to $M$ units of consumption to the borrower in the first period
of his life, and could {\it repay} (if the borrower lends)
up to $M$ units of consumption
in the second period of his life.
The lender lends  $b_{t}\leq M$  units to the borrower and
gets a state-contingent repayment $d(Y_{t+1})$, where $-M \leq d(Y_{t+1})$,
 in the second period of his life.
That the repayment is state contingent lets the
lender insure the borrower.


A lender is willing to make a one-period
loan to the borrower, but  only
 if the loan contract ensures repayment.
The borrower will fulfill the contract
only if he wants.
The lender observes  $Q$,  but observes
neither  $C$  nor  $I$. Next period, the lender can
observe $Y_{t+1}$. He bases the repayment
on that observation.



Where $c_t + I_t - b_t = Q_t$,
Atkeson's optimal recursive contract takes the form
$$\EQNalign{d_{t+1}& =d\left(Y_{t+1},Q_{t}\right) \EQN contract1;a \cr
   Q_{t+1} & =Y_{t+1}-d_{t+1} \EQN contract1;b \cr
   b_t & = b(Q_t). \EQN contract1;c \cr}$$
The repayment schedule  $d(Y_{t+1},Q_{t})$ depends
only on observables and is designed to recognize the
limited commitment and moral hazard problems.

 Notice how $Q_t$ is the only state variable in the contract.  Atkeson
uses the apparatus of Abreu, Pearce, and Stacchetti (1990),
discussed in chapter \use{credible}, to show that the state can be
taken to be $Q_t$, and that it is not necessary to keep track of the
history of past $Q$'s.  Atkeson obtains the following \idx{Bellman equation}.
Let $V(Q)$  be the optimum value of a borrower in state $Q$ under
the optimal contract.  Let $A=(c,I,b,d(Y'))$, all to be chosen as functions
of $Q$.  The Bellman equation is
%{\eightpoint
$$
V(Q)=\max_{A}\Bigl\{ (1-\delta
)\;u\,(\,c\,)+\delta \sum _{Y'} V\;[Y^{\prime
}-d\;(Y^{\prime },Q)]\,g\,(Y^{\prime }, I)  \Bigr\} \EQN bellman1;a $$
subject to
%%%{\ninepoint
$$\EQNalign{c & +I-b  \leq Q,\;\;   b\leq M, \ -d(Y^{\prime },Q)
\leq M, \;\; c\geq 0, \;\; I\geq 0 \hskip1cm  \EQN bellman1;b \cr
b&\leq \delta \sum_{Y'} d(Y^{\prime })\;g\,(Y^{\prime }, I\,)
             \EQN bellman1;c \cr
V&\,[Y^{\prime }-d\,(Y^{\prime })] \geq U\;(Y^{\prime }) \EQN bellman1;d
\cr}$$
$$ I  =\argmax_{\tilde I\,\epsilon [ 0,Q+b ] } \Bigl\{
\bigl( 1-\delta \bigr) \;u\;\bigl( Q+b-\tilde I\bigr) +\delta
\sum_{Y'} V\,\bigl[ Y^{\prime }-d\,( Y^{\prime},Q) \bigr]
\,g\,( Y^{\prime }, \tilde I)
     \Bigr\}.  \EQN bellman1;e $$
%%%}%endninepoint
\medskip
Condition \Ep{bellman1;b} is feasibility.  Condition
\Ep{bellman1;c} is a rationality constraint
for lenders: it requires that the gross return from lending to the borrower
be at least as great as the alternative
yield available to lenders, namely, the risk-free gross interest rate
 $\delta^{-1}$.
Condition \Ep{bellman1;d}
says that in every state tomorrow, the borrower must want
to comply with the contract; thus the value of affirming
the contract (the left side) must be at least as great as the value of
autarky.  Condition \Ep{bellman1;e} states that the borrower chooses $I$ to
maximize his expected utility under the contract.


There are many value functions $V(Q)$ and associated
contracts $b(Q), d(Y',Q)$ that satisfy conditions \Ep{bellman1}. Because
we want the optimal contract, we want the $V(Q)$ that is the
largest (hopefully, pointwise).     The usual
strategy of iterating on the Bellman equation, starting
from an arbitrary guess  $V^0(Q)$, say, $0$, will not work in this
case because high candidate continuation values $V(Q')$ are
needed to support good current-period outcomes.
   But a modified version of the usual iterative strategy does work,
which is to make sure that we start with a large enough initial
guess at the continuation value function $V^0(Q')$.
 Atkeson (1988, 1991) verified that the optimal contract can be constructed
by iterating to convergence on conditions \Ep{bellman1}, provided
that the iterations begin from a large enough initial value
function $V^0(Q)$.  (See the appendix for a computational exercise
using Atkeson's iterative strategy.)
 He adapted ideas from  Abreu, Pearce, and Stacchetti (1990)
to show this result.\NFootnote{See chapter \use{credible}
for some work with the
Abreu, Pearce, and Stacchetti structure, and for  how, with history dependence,
 dynamic programming principles direct attention to
{\it sets} of continuation value functions.  The need to handle
a set of continuation values appropriately is why Atkeson must
initiate his iterations from a sufficiently  high initial
value function.}  In the next subsection, we shall form a Lagrangian in which the role of continuation values is explicitly accounted for.

\vskip.5cm
\medskip
\noindent
{\bf Binding participation constraint}
\smallskip
\noindent  Atkeson motivated his work as an effort to explain why
countries often experience capital outflows in the very-low-income
periods in which they would be borrowing
{\it more} in a complete markets  setting.
  The optimal contract associated with conditions \Ep{bellman1}
 has the feature that Atkeson sought: the  borrower
makes net repayments $d_t > b_t $ in states with low
output realizations.

Atkeson establishes this property using the following argument.
First, to permit him to capture the borrower's best response with
a first-order condition, he assumes the following conditions about the
outcomes:\NFootnote{The first assumption makes the lender prefer that
the borrower would make larger rather than smaller investments.
See Rogerson (1985b) for conditions needed to validate the first-order
approach to incentive problems.}
\medskip
\noindent{\sc Assumptions:}   For the optimum contract
$$ \sum_i d_i \bigl[g_0(Y_i) - g_1(Y_i)\bigr]  \geq 0. \EQN assumpt6 $$
This makes the value of repayments increasing in investment. In addition,
assume that the borrower's constrained optimal investment level is interior.
\medskip
Atkeson assumes conditions \Ep{assumpt6} and
\Ep{assumpt5}  to justify using the first-order condition
for the right side of equation \Ep{bellman1;e} to characterize
 the investment decision.
The first-order condition for investment is
 $$ - (1-\delta) u'(Q+b-I) + \delta \sum_i
                       V(Y_i -d_i) g_I(Y_i,I) = 0.$$

\subsection{Optimal capital outflows under distress}

To deduce a key property of the repayment schedule, we will follow
Atkeson by introducing a continuation value $\tilde V$ as an
additional choice variable in a programming  problem that represents
a form of the contract design problem.
Atkeson shows how \Ep{bellman1} can be viewed as the outcome
of a more elementary programming problem in which
the contract designer chooses  the continuation value
function from a set of permissible values.\NFootnote{See Atkeson (1991)
and chapter \use{credible}.}
Following Atkeson, let $U_d(Y_i) \equiv \tilde V(Y_i - d(Y_i))$
where $\tilde V(Y_i - d(Y_i))$ is a continuation value function to be
chosen by the author of the contract.   Atkeson shows that
we can regard the contract author as choosing a continuation
value function along with the elements of $A$, but that
in the end it will be optimal for him to choose the continuation
values  to satisfy
the Bellman equation \Ep{bellman1;a}.

We follow Atkeson and regard the $U_d(Y_i)$'s  as choice variables.
 They must
satisfy $U_d(Y_i) \leq V(Y_i - d_i)$, where $V(Y_i-d_i)$ satisfies
the Bellman equation \Ep{bellman1}.
\noindent
Form the Lagrangian
$$\eqalign{ J(A, U_d, \mu)  = &  (1-\delta) u(c) + \delta \sum_i U_d(Y_i) g(Y_i,I) \cr
            & + \mu_1(Q + b - c -I) \cr
        & + \mu_2 \bigl[\delta \sum_i d_i g(Y_i,I) - b\bigr] \cr
        & + \delta \sum_i \mu_3(Y_i) g(Y_i, I) \bigl[U_d(Y_i)
                - U(Y_i)\bigr] \cr
        & + \mu_4 \bigl[ - (1-\delta) u'(Q+b-I) + \delta \sum_i
                       U_d(Y_i) g_I(Y_i,I)\bigr] \cr
        & + \delta \sum_i \mu_5(Y_i) g(Y_i,I) \bigl[ V(Y_i -d_i)
                  - U_d(Y_i) \bigr], \cr} \EQN lagrange $$
where the $\mu_j$'s are nonnegative Lagrange multipliers. To investigate
the consequences of a binding participation constraint,
rearrange the first-order condition with respect to  $U_d(Y_i)$
to get
$$ 1 + \mu_4 {g_I(Y_i,I) \over g(Y_i,I) } =
    \mu_5(Y_i) - \mu_3(Y_i) , \EQN fonc2 $$
where $g_I/g = \lambda'(I)\bigl[{g_0(Y_i) - g_1(Y_i) \over g(Y,I)}\bigr]$,
which is negative for low $Y_i$ and positive for high $Y_i$.
All the  multipliers are nonnegative. Then evidently  when
the left side of equation \Ep{fonc2} is {\it negative}, we must have
$\mu_3(Y_i) >0$, so that condition \Ep{bellman1;d} is binding
and $U_d(Y_i) = U(Y_i)$.   Therefore,
$V(Y_i - d_i)  = U(Y_i)$ for states with $\mu_3(Y_i) >0$.
Atkeson uses this finding to show that in states  $Y_i$ where
$\mu_3(Y_i) >0$, new loans $b'$ cannot exceed repayments
$d_i=d(Y_i)$.   This conclusion follows from the following argument.
The optimality condition \Ep{bellman1;e} implies that $V(Q)$ will
satisfy
$$ V(Q) = \max_{I \in [0, Q+b]} u(Q+b-I) + \delta \sum_{Y'}
  V(Y' - d(Y')) g(Y',I).   \EQN andynew1 $$
Using the participation constraint \Ep{bellman1;d} on the right side
of \Ep{andynew1} implies
$$ V(Q) \geq \max_{I \in [0, Q+b]} \left\{u(Q+b-I) + \delta \sum_{Y'}
   U(Y_i') g(Y',I) \right\}  \equiv U(Q+b) \EQN andynew2 $$
  where $U$ is the value function for the autarky problem
\Ep{bellmanaut}.
In states in which $\mu_3 >0$, we know that, first,
$V(Q) = U(Y)$, and, second,  that by \Ep{andynew2} $V(Q) \geq U(Y + (b-d))$.
But we also know that $U$ is increasing.  Therefore,
we must have that $(b-d) \leq 0$, for otherwise
 $U$ being increasing
induces a contradiction.
We conclude that
 for those low-$Y_i$ states for which $\mu_3>0$,
$b \leq d(Y_i)$, meaning that there are no capital inflows for
these states.\NFootnote{This argument
highlights the important role of limited enforcement in producing
capital outflows at low output realizations.}

%Thus, in low output realization states
%where condition \Ep{bellman1;d} is binding, the borrower experiences a
%``capital outflow.''
Capital outflows in bad times
provide good incentives because they occur only at output
realizations so low that they are more likely  to occur when
the borrower has undertaken too little investment.   Their role is to
provide incentives for the borrower to invest enough
to make it unlikely that those low-output states will occur.
The occurrence of
 capital outflows at low outputs is not
called for by
the complete markets contract \Ep{atnew6;b}. On the contrary,
the complete markets
contract provides
a ``capital inflow'' to the lender in low-output states.
That the pair of functions
 $b_t = b(Q_t)$, $d_t = d(Y_t, Q_{t-1})$ forming the optimal
contract  specifies repayments in those distressed  states is how the
contract provides incentives for the borrower to make
investment decisions that reduce the likelihood that combinations of
$(Y_t, Q_t, Q_{t-1})$ will occur that
trigger capital outflows under distress.

 We remind the reader of the remarkable feature of Atkeson's
contract that the repayment schedule and the state
variable $Q$ ``do all the work.''  Atkeson's contract
manages to encode all history dependence in an extremely
economical fashion.  In the end, there is no need, as occurred
in the problems that we studied in chapter \use{socialinsurance},
to add a promised value as an independent state variable.
%
%result in an invest decision $I_t = I(Q_t)$ and a law of motion for the state
%$$ Q_{t+1} = Y_{t+1} - d(Y_{t+1},Q_t), $$
%where $ Y_{t+1} \sim g[Y_{t+1}, I(Q_t)]$.   We are interested
%in net repayments at $t$, $d(Y_t, Q_{t-1}) - b(Q_t)$, and how they
%compare with \Ep{atnew6;b}.  The computations indicate that
%$\ldots$.
%\medskip
%\noindent{\bf Computational results to be added}
%\medskip


% INSERT BOND-PARK ANALYSIS
%\showchaptIDtrue

 \auth{Bond, Eric W.} \auth{Park, Jee-Hyeong}
\section{Gradualism in trade policy}

We now describe a version of Bond and Park's (2002) analysis
of gradualism in bilateral agreements to liberalize international trade.
Bond and Park cite examples in which a large country extracts a possibly
rising sequence of transfers  from a small country in exchange for a gradual
lowering of tariffs in the large country.  Bond and Park interpret
gradualism in terms of the history-dependent policies that vary
the continuation value of the large country in a way that induces it gradually
to reduce its distortions from tariffs while still gaining from a move toward
free trade.  They interpret the transfers as trade concessions.\NFootnote{Bond
and Park say that in practice, the trade concessions take the form of
reforms of policies in the small country about protecting intellectual
property, protecting rights of foreign investors, and managing the domestic
economy.  They do not claim explicitly to model these features.}

We begin by laying out a simple general equilibrium model of trade
between two countries.\NFootnote{Bond and Park (2002) work in terms of
a partial equilibrium model that differs in details but shares the
spirit of our model.} The outcome of this theorizing will be a
pair of indirect utility functions $r_L$ and $r_S$ that give the
welfare of a large and small country, respectively, both as
functions of a tariff $t_L$ that the large country imposes on the
small country, and a transfer $e_S$ that the small country
voluntarily offers to the large country.

\subsection{Closed-economy model}

First, we describe a one-country model.  The country consists of a fixed
number of identical households.  A typical household has preferences
$$
 u(c,\ell) = c + \ell - 0.5\,\ell^{\,2},
                                                    \EQN M1utility
$$
where $c$ and $\ell$ are consumption of a single consumption good and
leisure, respectively. The household is endowed with a quantity $\bar y$
of the consumption good and one unit of time that can be used for either
leisure or work,
$$
1= \ell + n_{1} + n_{2},   \EQN M1endow
$$
where $n_{j}$ is the labor input in the production of intermediate good $x_{j}$,
for $j=1,2$. The two intermediate goods can be combined to produce additional
units of the final consumption good.  The technology is as follows:
$$\EQNalign{
x_{1} &= n_{1} , & M1tech;a \cr
x_{2} &= \gamma \, n_{2}, \hskip1cm \gamma\in [0,1],& M1tech;b \cr
y    &= 2\, \min\{x_{1},\, x_{2}\},               & M1tech;c    \cr
c    &= y + \bar y , & M1tech;d \cr}
$$
where consumption $c$ is the sum of production $y$ and the endowment
$\bar y$.

Because of the Leontief production function for the final consumption good,
a closed economy will produce
the same quantity of each intermediate good. For a given production
parameter $\gamma$, let $\tilde \chi(\gamma)$ be the
identical amount of each intermediate good that would be produced per unit
of labor input.
That is, a fraction $\tilde \chi(\gamma)$ of one unit of labor input would
be spent on producing $\tilde \chi(\gamma)$ units of intermediate good 1
and another fraction
$\tilde \chi(\gamma)/\gamma$ of the labor input would be devoted
to producing the same amount of intermediate good 2:
$$
\tilde \chi(\gamma) +  {\tilde \chi(\gamma) \over \gamma} = 1
\hskip.5cm \Longrightarrow \hskip.5cm \tilde \chi(\gamma) = {\gamma \over 1+\gamma}.
\EQN fernan1 $$
The linear technology implies a competitively determined wage at
which all output is paid out as labor compensation.
The optimal choice of leisure makes
the marginal utility of consumption from an extra unit of labor input
equal to
the marginal utility of an extra unit of leisure:
$
2\, \min\{\tilde \chi(\gamma),\, \tilde \chi(\gamma)\}
=\,{d \hfill \over d \ell} \Bigl[\ell - 0.5\,\ell^{\,2}\Bigr] . $
Substituting for $\tilde \chi(\gamma)$ from  \Ep{fernan1} gives
$
{2\, \gamma \over 1+\gamma} = 1-\ell $, which can be rearranged to
become
$$\ell={\cal L}(\gamma) =\, {1-\gamma \over 1+\gamma} .   \EQN M1ell
$$
It follows that per capita, the
equilibrium quantity of each intermediate good
is given by
$$
x_1 = x_2 = \chi(\gamma) \,\equiv\, \tilde \chi(\gamma) [1-{\cal
L}(\gamma)] \,=\, {2\, \gamma^2 \over (1+\gamma)^2}. \EQN M1x
$$

\vskip.5cm
\medskip
\noindent
{\bf Two countries under autarky}
\smallskip
\noindent  Suppose that there are two countries named $L$ and $S$ (denoting
large and small).   Country
$L$ consists of $N\geq1$ identical consumers, while country $S$ consists
of one household. All households have the same preferences \Ep{M1utility},
but technologies differ across countries. Specifically, country $L$ has
production parameter $\gamma=1$ while country $S$ has $\gamma = \gamma_S <1$.


Under no trade or {\it autarky\/}, each country is a closed economy
whose allocations are given by \Ep{M1ell}, \Ep{M1x}, and \Ep{M1tech}.
%Evaluating these
%expressions, we obtain
%$$\EQNalign{
%\{\ell_L,n_{1L},n_{2L},c_L\}&=\{0,\,0.5,\, 0.5,\,  \bar y +1\},  \cr
%\{\ell_S,n_{1S},n_{2S},c_S\}&=\{\ell(\gamma_S),\, x(\gamma_S), \,
%(\gamma_S)/\gamma_S,\, \bar y + 2\, x(\gamma_S)\} \cr}.  $$
Evaluating these
expressions, we obtain
$$\EQNalign{
\{ \ell_L,n_{1L},n_{2L},c_L\}&=\{0,\,0.5,\, 0.5,\,  \bar y +1\},  \cr
\{\ell_S,n_{1S},n_{2S},c_S\}&=\{{\cal L}(\gamma_S),\, \chi(\gamma_S), \,
\chi(\gamma_S)/\gamma_S,\, \bar y + 2\, \chi(\gamma_S)\} . \cr}
$$
%for country $L$ and $S$, respectively.
The relative price between the two intermediate goods is $1$ in country
$L$ while for country $S$, intermediate good 2 trades at a price
$\gamma_S^{-1}$ in terms of intermediate good 1. The difference in
relative prices across countries implies gains from
trade.

\subsection{A Ricardian model of  two countries under free trade}

Under free trade, country $L$ is large enough to meet both
countries' demands for intermediate good 2 at a relative price of $1$
and hence country $S$ will specialize in the production of
intermediate good 1 with $n_{1S}=1$. To find the time $n_{1L}$
that a worker in country $L$ devotes to the production of
intermediate good 1, note
that the world demand at a relative price of $1$ is equal to
$0.5(N+1)$ and, after imposing market clearing, that
$$\EQNalign{
N \, n_{1L} \,+\, 1 \,&=\, 0.5\,(N+1) \cr
n_{1L} \, & =\, {N-1 \over 2N}.}
$$
%The free-trade allocation becomes
%$$\EQNalign{
%\{\ell_L,n_{1L},n_{2L},c_L\}&=\{0,\,(N-1)/(2N),\, (N+1)/(2N),\,
%\bar y +1\},  \cr
%\{\ell_S,n_{1S},n_{2S},c_S\}&=\{0,\, 1, \, 0,\, \bar y + 1\} \cr}.
%$$
The free-trade allocation becomes
$$\EQNalign{
\{\ell_L,n_{1L},n_{2L},c_L\}&=\{0,\,(N-1)/(2N),\, (N+1)/(2N),\,
  \bar y +1\},  \cr
\{\ell_S,n_{1S},n_{2S},c_S\}&=\{0,\, 1, \, 0,\, \bar y + 1\}. \cr}
$$
%for country $L$ and $S$, respectively.
Notice that the welfare of a household in country $L$ is the same as
under autarky because we have
$\ell_L=0$, $c_L=\bar y +1$. The invariance of country $L$'s allocation
to opening trade is an immediate
implication of the fact that the equilibrium prices under free trade
are the same as those in country $L$ under autarky. Only
country $S$ stands to gain from free trade.


\subsection{Trade with a tariff}
Although country $L$ has nothing to gain from free trade, it can gain
from trade if it is accompanied by a distortion to the terms of
trade that is implemented through
a tariff on country $L$'s imports.
Thus, assume that country $L$ imposes a tariff of $t_L \geq 0$
 on all imports into $L$.  For any quantity of intermediate or final goods
imported into country $L$, country $L$ collects  a fraction $t_L$ of those
goods by levying the tariff. A necessary condition for the existence of
an equilibrium with trade is that the tariff does not exceed $(1-\gamma_S)$,
because otherwise country $S$ would choose to produce intermediate
good 2 rather than import it from country $L$.

Given that $t_L\leq 1- \gamma_S$,
we can find the equilibrium with trade
as follows. From the perspective of country $S$,
$(1-t_L)$ acts like the production parameter $\gamma$,
i.e., it determines the cost of obtaining one unit of intermediate good 2
in terms of foregone production of intermediate good 1. Under
autarky that price was $\gamma^{-1}$; with trade and a tariff $t_L$,
that price becomes
 $(1-t_L)^{-1}$. For country $S$, we can therefore draw upon
the analysis of a closed economy and just replace $\gamma$ by $1-t_L$.
The allocation with trade for country $S$ becomes
$$
\{\ell_S,n_{1S},n_{2S},c_S\}=\{{\cal L}(1-t_L),\, 1-{\cal L}(1-t_L), \,
0,\, \bar y + 2\, \chi (1-t_L)\}.                \EQN Salloc
$$
In contrast to the equilibrium under autarky, country $S$ now allocates
all labor input $1-{\cal L}(1-t_L)$ to the production of intermediate
good 1 but retains only a quantity $\chi(1-t_L)$ of total production
for its own use, and exports the rest $\chi(1-t_L)/(1-t_L)$ to country $L$.
After paying tariffs, country $S$ purchases
an amount $\chi(1-t_L)$ of intermediate good 2 from country $L$. Since
this quantity of intermediate good 2 exactly equals the amount
of intermediate good 1 retained in country $S$, production of
the final consumption good given by \Ep{M1tech;c} equals
$2\, \chi(1-t_L)$.

Country $L$ receives a quantity $\chi(1-t_L)/(1-t_L)$ of intermediate good 1
from country $S$, partly  as tariff revenue $t_L\, \chi(1-t_L)/(1-t_L)$ and
partly as payments
for its exports of intermediate good 2, $\chi(1-t_L)$. In response to the
inflow of intermediate good 1, an aggregate quantity of labor equal to
$\chi(1-t_L) + 0.5\, t_L\, \chi(1-t_L)/(1-t_L)$ is reallocated
in country $L$ from the production of intermediate good 1 to the
production of intermediate good 2.  This allows country $L$ to
meet the demand for intermediate good 2 from country $S$ and at the
same time increase its own use of each
intermediate good by $0.5\, t_L\, \chi(1-t_L)/(1-t_L)$. The per capita
trade allocation for country $L$ becomes
$$\EQNalign{
\{\ell_L,n_{1L},n_{2L},&c_L\}=
\Biggl\{0,\, 0.5-{(1-0.5t_L)\,\chi(1-t_L) \over (1-t_L)N},\, \cr
 &0.5+{(1-0.5t_L)\,\chi(1-t_L) \over (1-t_L)N},\,
\bar y + 1 + t_L{\chi(1-t_L)\over (1-t_L)N} \Biggr\}.         \hskip1cm    & Lalloc \cr}
$$

\subsection{Welfare and Nash tariff}

For a given tariff $t_L\leq 1-\gamma_S$, we can compute the welfare
levels in a trade equilibrium.
Let $u_S(t_L)$ and $u_L(t_L)$ be the indirect utility of
country $S$ and country $L$, respectively, when the tariff is $t_L$.
After substituting the equilibrium allocation \Ep{Salloc} and \Ep{Lalloc}
into the utility function of \Ep{M1utility}, we obtain
$$\eqalign{
u_S(t_L) &= u(c_S,\ell_S)                                       \cr
&= \bar y + 2 \, \chi(1-t_L) + {\cal L}(1-t_L) - 0.5\,{\cal L}
(1-t_L)^{\,2},   \cr
\noalign{\vskip.2cm}
u_L(t_L) &= N\,u(c_L,\ell_L)
= N\,(\bar y + 1) + t_L{\chi(1-t_L)\over 1-t_L}, \cr}   \EQN Lars1000
$$
where we multiply the utility function of the representative agent in
country $L$ by $N$ because we are aggregating over all agents in a
country. We now invoke equilibrium expressions \Ep{M1ell} and \Ep{M1x},
and take derivatives with respect to $t_L$.
As expected, the welfare of country $S$ decreases with the tariff
while the welfare of country $L$ is a strictly concave function that
initially increases with the tariff:
$$\EQNalign{
{d u_S(t_L) \over d t_L \hfill}&= -{4\,(1-t_L) \over (2-t_L)^3}<0\,,
                                                           &Utilderiv;a  \cr
\noalign{\vskip.2cm}
{d\, u_L(t_L) \over d \, t_L \hfill}&= {2\,(2-3t_L) \over (2-t_L)^3}
                   \cases{ >0 &for $t_L<2/3$ \cr \leq 0 &for $t_L\geq2/3$ \cr}
 \, &Utilderiv;b \cr
\noalign{\hbox{\rm and}}
{d \, ^2 u_L(t_L) \over d\, t_L^2 \hfill} &= -{12 t_L \over (2-t_L)^4}\leq 0\,,
                                                           &Utilderiv;c \cr}
$$
where it is understood that the expressions are evaluated for
$t_L\leq 1-\gamma_S$.


The tariff enables country $L$ to reap some of the benefits from trade.
In our model, country $L$ prefers a tariff
$t_L$ that maximizes its tariff revenues.

\medskip
\noindent{\sc Definition:} In a one-period
 {\it Nash equilibrium\/}, the government
of   country $L$ imposes a tariff rate that satisfies
$$ t_L^N = \min\Bigl\{\argmax_{t_L} u_L(t_L),\; 1-\gamma_S\Bigr\}.  \EQN Nashtariff $$
\medskip
\noindent
From expression
\Ep{Utilderiv;b}, we have $ t_L^N = \min\{2/3,\, 1-\gamma_S\}$.

\medskip
\noindent{\sc Remark:} At the Nash tariff, country $S$ gains from trade
if $2/3 < 1-\gamma_S$.  Country $S$ gets no gains from trade
if $1 -\gamma_S \leq 2/3$.
\medskip
Measure world welfare by
$u_W(t_L) \equiv u_S(t_L) + u_L(t_L)$. This measure of world
welfare satisfies
$$\EQNalign{
{d\, u_W(t_L) \over d\, t_L \hfill}&= -{2\,t_L \over (2-t_L)^3}\leq 0\,,
                                                           &Utilworld;a \cr
\noalign{\hbox{\rm and}}
{d\, ^2 u_W(t_L) \over d\, t_L^2 \hfill} &= -{4\,(1+t_L) \over (2-t_L)^4}<0\,.
                                                           &Utilworld;b \cr}
$$
We summarize our findings:
\medskip
\noindent{\sc Proposition 1:}    World welfare $u_W(t_L)$ is  strictly concave,
is decreasing in $t_L \geq 0$,  and is maximized by
setting $t_L =0$.   %%%%Thus, $u_W(t_L)$ is maximized at $t_L=0$.
But   $u_L(t_L)$ is strictly concave in $t_L$ and
is maximized at $t_L^N >0$.  Therefore, $u_L(t_L^N) > u_L(0)$.
\medskip
\noindent A consequence of this proposition is that
country $L$ prefers the Nash equilibrium to free trade, but country
$S$ prefers free trade.  To induce
country $L$
to accept free trade, country $S$ will have to transfer resources
to it.  We now study how country $S$ can do that efficiently in an
intertemporal version of the model.


 \auth{Bond, Eric W.} \auth{Park, Jee-Hyeong}

\subsection{Trade concessions}
To get a model in the spirit of Bond and Park (2002),
we now assume that the two countries can make trade concessions
that take the form of a direct transfer of the consumption good
between them.
 We augment
 utility functions
$u_L, u_S$
 of the form \Ep{M1utility} with these transfers to obtain the
payoff functions
$$ \EQNalign{r_L(t_L, e_S) &= u_L(t_L) + e_S \EQN bppayoff1;a \cr
             r_S(t_L, e_S) &= u_S(t_L) - e_S , \EQN bppayoff1;b \cr} $$
where $t_L\geq 0$ is  a tariff on the imports of
country $L$, $e_S\geq 0$ is a transfer from country $S$ to country
$L$. These definitions make sense because the indirect utility
functions \Ep{Lars1000} are linear in consumption of the final
consumption good, so that by transferring the final consumption good, the
small country transfers utility.
The transfers $e_S$ are to be voluntary and must be nonnegative (i.e., the
country cannot extract transfers from the large country).
We have already seen that
$u_L(t_L)$ is strictly concave and  twice continuously differentiable
with $u_L'(0) >0$ and that  $u_W(t_L) \equiv u_S(t_L) + u_L(t_L)$ is
strictly concave and twice continuously differentiable with
$u_W'(0) =0$.
We call {\it free trade\/} a situation in which $t_L=0$.
We let $(t_L^N, e_S^N)$ be the Nash equilibrium tariff rate  and transfer
for a one-period, simultaneous-move  game in which the two countries have
payoffs
\Ep{bppayoff1;a} and \Ep{bppayoff1;b}. Under Proposition 1,
$t_L^N  >0, e_S^N =0$.
 Also, $u_L(t_L^N) > u_L(0)$ and $u_S(0) > u_S(t_L^N)$,
so that country $S$ gains and country $L$ loses in
moving from the Nash equilibrium
 to free trade with $e_S =0$.

\subsection{A repeated tariff game}

We now suppose that the economy repeats itself infinitely.
for $t \geq 0$.
Denote the pair of time $t$ actions of the
two countries by $\rho_t= (t_{Lt},  e_{St})$.  For $t \geq 1$, denote
the history of actions up to time $t-1$
as $\rho^{t-1} = [\rho_{t-1}, \ldots, \rho_0]$.
A policy $\sigma_{S}$ for country $S$ is an initial $e_{S0}$ and for $t\geq 1$
a sequence of functions expressing $e_{St}= \sigma_{St}(\rho^{t-1})$.
A policy $\sigma_{L}$
for country $L$ is an initial $t_{L0}$ and for $t\geq 1$
a sequence of functions expressing $t_{Lt}= \sigma_{Lt}(\rho^{t-1})$.
Let $\sigma$ denote the pair of policies $(\sigma_{L},  \sigma_S)$.
The {\it policy\/} or {\it strategy profile\/}
 $\sigma$ induces time $t$ payoff
$r_i(\sigma_t)$ for country
$i$ at time $t$, where $\sigma_t$ is the time $t$ component of
$\sigma$.
We measure country
$i$'s present discounted   value by
$$ v_i(\sigma) = \sum_{t=0}^\infty \beta^t r_i(\sigma_t) \EQN trade20 $$
where $\sigma$ affects $r_i$ through its effect on $c_i$.
Define $\sigma|_{\rho^{t-1}}$ as the continuation of $\sigma$ starting at
$t$  after
history  $\rho^{t-1}$.
Define the continuation value of $i$ at time $t$ as
$$ v_{it} = v_i(\sigma|_{\rho^{t-1}}) =
 \sum_{j=0}^\infty \beta^j r_i(\sigma_j|_{\rho^{t-1}}). $$

We use the following standard definition:
\medskip
\noindent{\sc Definition:}
A {\it subgame perfect equilibrium\/} is a strategy profile $\sigma$
such that for all $t\geq 0$ and all histories $\rho^t$, country
$L$ maximizes its continuation value starting from $t$, given
$\sigma_S$, and country $S$  maximizes its continuation    value
starting from $t$, given $\sigma_{L}$.
\medskip

It is easy to verify that a strategy that forever repeats the
static Nash equilibrium outcome $(t_L, e_S)= (t_L^N, 0)$ is
a subgame perfect equilibrium.


\subsection{Time-invariant transfers}

We first study circumstances under which
there exists a time-invariant transfer
$e_S >0$ that will induce country $L$ to move  to free trade.


Let $v_i^N = {u_i(t_L^N) \over 1 -\beta}$ be the present discounted value
of country $i$ when the static Nash equilibrium is repeated forever.
If both countries are to prefer free trade with a time-invariant
transfer level $e_S>0$,
the following two participation constraints
must hold:
$$\EQNalign{ v_L & \equiv {u_L(0) + e_S \over 1 -\beta} \geq u_L(t_L^N)
   + e_S + \beta v_L^N
\EQN tradepL \cr
             v_S & \equiv{ u_S(0) - e_S \over 1 -\beta} \geq  u_S(0)
 + \beta v_S^N.
 \EQN tradepS \cr}$$
The timing here articulates what it means for $L$ and $S$ to
choose simultaneously: when $L$ defects from $(0, e_S)$, $L$
retains the transfer $e_S$ for that period. Symmetrically, if $S$
defects, it enjoys the zero tariff  for that one period.   These
temporary gains provide the temptations to defect. Inequalities
\Ep{tradepL} and \Ep{tradepS} say that countries $L$ and $S$ both get
higher continuation values from remaining in free trade with the
transfer $e_S$ than they get in the repeated static Nash
equilibrium.  Inequalities \Ep{tradepL} and \Ep{tradepS}  invite
us to study strategies that have each country respond to any
departure from what it had expected the other country to do this
period by forever after choosing the Nash equilibrium actions
$t_L= t_L^N$ for country $L$ and  $e_S=0$ for country $S$. Thus,
the response to any deviation from anticipated behavior is to
revert to the repeated static
 Nash equilibrium, itself a subgame perfect
equilibrium.\NFootnote{In chapter \use{credible},
we study  the consequences of reverting to a subgame perfect equilibrium
that gives {\it worse\/} payoffs to both $S$ and $L$ and how
the worst subgame perfect equilibrium
payoffs and strategies  can be constructed.}

Inequality \Ep{tradepL} (the participation constraint for $L$)
 and the definition of $v_L^N$ can be  rearranged
to get
$$ e_S \geq {u_L(t_L^N) - u_L(0) \over \beta }.  \EQN tradees1$$
Time-invariant transfers $e_S$ that satisfy inequality \Ep{tradees1} are
sufficient to induce $L$ to abandon the Nash equilibrium and
set its tariff to zero.
The {\it minimum\/} time-invariant transfer that will induce $L$
to accept free trade is then
$$ e_{S{\rm min}} = {u_L(t_L^N) - u_L(0) \over \beta }.  \EQN tradees2$$

  Inequality \Ep{tradepS} (the participation constraint for $S$)
 and the definition of $v_S^N$ yield
$$ e_S \leq \beta (u_S(0) - u_S(t_L^N)),  \EQN tradees3 $$
which restricts the time-invariant transfer that $S$ is willing
to make to move to free trade by setting $t_L =0$.
  Evidently, the {\it largest\/}
time-invariant transfer
that $S$ is willing to pay is
$$ e_{S { \rm max}} = \beta (u_S(0) - u_S(t_L^N)).  \EQN tradees4 $$

 If we substitute $e_S = e_{S {\rm min}}$ into the definition
of $v_L$ in \Ep{tradepL}, we find that the {\it lowest\/}
continuation value $v_L$ for country $L$ that can be supported
by a stationary transfer is
$$ v_L^* = \beta^{-1} (v_L^N - u_L(0)) . \EQN VLmin $$
If we substitute $e_S = e_{S {\rm max}}$ into the definition
of $v_L$ we can conclude that the {\it highest\/} $v_L$ that can be sustained
by a stationary transfer is
$$ v_L^{**} = {u_L(0) + \beta (u_S(0) - u_S(t_L^N)) \over 1 -\beta} .
  \EQN VLmax $$

For there to exist a time-invariant transfer $e_S$ that induces both countries
to accept free trade, we require that
$v_L^* <v_L^{**}$ so that $[v_L^*,v_L^{**}]$ is nonempty.
  For a class of world economies differing only in their
discount factors, we can compute
a discount factor $\beta$ that makes $v_L^* = v_L^{**}$. This
is the critical value for the discount factor below which
the interval $[v_L^*, v_L^{**}]$ is empty. Thus,
equating the right sides of \Ep{VLmin} and \Ep{VLmax} and solving
for $\beta$
gives the critical value
$$ \beta_c \equiv \sqrt{{ u_L(t_L^N) - u_L(0) \over u_S(0) - u_S(t_t^N)}}.
  \EQN betac $$
We know that the numerator under the square root is positive
and that it is less than the denominator (because $S$ gains by
moving to free trade more than $L$ loses, i.e., $u_W(t_L)$ is maximized
at $t_L=0$).  Thus, \Ep{betac} has a solution
$\beta_c \in (0,1)$.   For $\beta > \beta_c$,
there is a nontrivial interval $[v_L^*, v_L^{**}]$.
For $\beta < \beta_c$, the interval is empty.

Now consider the utility possibility frontier
{\it without\/} the participation constraints
\Ep{tradepL}, \Ep{tradepS}, namely,
$$  v_S = {u_W(0) \over 1-\beta} - v_L. \EQN utilppf $$
  Then we have the following:
\medskip
\noindent{\sc Proposition 2:}
There is a critical value $\beta_c$
such
that for $\beta > \beta_c$, the interval
$[v_L^*, v_L^{**}]$ is nonempty.
 For $v_L \in [v_L^*, v_L^{**}]$, a pair
$(v_L, v_S)$ on the unconstrained utility possibility frontier
\Ep{utilppf} can be attained by a time-invariant  policy $(0, e_s)$ with
transfer $e_S >0$ from $S$ to $L$. The policy is supported by
a trigger strategy  profile that reverts forever to
$(t_L, 0)$ if expectations are ever disappointed.


\subsection{Gradualism: time-varying trade policies}

From now on, we assume that $\beta > \beta_c$, so that
$[v_L^*,v_L^{**}]$ is nonempty. We make this assumption because we want
to study settings in which the two countries eventually move to free trade
even if they don't start there.
  Notice %%%that $v_L^N < v_L^* $.
from expression \Ep{VLmin} that
$$\eqalign{ v_L^* &= \beta^{-1} \left( \left[u_L(t_L^N)
                         + \beta v_L^N \right] - u_L(0) \right) \cr
 &= v_L^N + \beta^{-1} \left( u_L(t_L^N) - u_L(0) \right) > v_L^N . \cr}
\EQN bopark_lbound
$$
Thus, even when $[v_L^*, v_L^{**}]$
is nonempty, there is an interval of continuation values $[v_L^N,
v_L^*)$ that cannot be sustained by a time-invariant transfer
scheme. Values $v_L >  v_L^{**}$ also fail to be sustainable by a
time-invariant transfer because the required $e_S$ is too high.
For initial values $v_L < v_L^*$ or $v_L > v_L^{**}$, Bond and
Park construct time-varying  tariff and transfer schemes that
sustain continuation value $v_L$. They proceed by designing a
recursive contract similar to ones constructed by Thomas and
Worrall (1988) and again by Kocherlakota (1996a). \auth{Worrall,
Tim} \auth{Thomas, Jonathan} \auth{Kocherlakota, Narayana R.}

   Let $v_L(\sigma), v_S(\sigma)$ be the discounted present
values delivered to countries
$L$ and $S$ under policy $\sigma$.
For a given initial promised value
$v_L$ for country $L$,  let $P(v_L)$ be the
maximal continuation value $v_S$  for country $S$,
 associated with a possibly time-varying trade policy.  The value
function $P(v_L)$ satisfies the functional equation
   $$  P(v_L) = \sup_{t_L, e_S, y} \left\{ u_S(t_L) - e_S
    + \beta P(y)  \right\}, \EQN bondpark1  $$
where the maximization is subject to $t_L \geq 0, e_S \geq 0$ and
$$ \EQNalign{ & u_L(t_L) + e_S + \beta y \geq v_L \EQN bondpark2;a \cr
    & u_L(t_L) + e_S + \beta y \geq u_L(t_L^N) + e_S + \beta v_L^N
   \EQN bondpark2;b \cr
  & u_S(t_L) - e_S + \beta P(y) \geq u_S(t_L) +\beta v_S^N  .
  \EQN  bondpark2;c \cr} $$
Here, $y$ is the continuation value for $L$, meaning next period's value
of $v_L$.
 Constraint \Ep{bondpark2;a} is the promise-keeping constraint, while
\Ep{bondpark2;b} and \Ep{bondpark2;c} are
the participation constraints for countries
$L$ and  $S$, respectively. The constraint set is convex and the objective
is concave, so $P(v_L)$ is concave (though not strictly concave,
an important qualification, as we shall see).


As with our study of Thomas and Worrall's  and Kocherlakota's model, we place nonnegative
multipliers $\theta$ on \Ep{bondpark2;a} and
$\mu_L, \mu_S$ on \Ep{bondpark2;b} and \Ep{bondpark2;c}, respectively,
form a Lagrangian, and obtain the following first-order necessary
conditions for
a saddlepoint:
$$ \EQNalign{t_L: \ \ &  u_S'(t_L) + (\theta+ \mu_L) u_L'(t_L) \leq
 0 , \ \ = 0 \ {\rm if} \ t_L > 0  \hskip1cm \EQN bondpark3;a \cr
  y: \  \ & P'(y) (1+\mu_S) + (\theta +\mu_L) =0 \EQN bondpark3;b \cr
 e_S: \ \  & - 1 +\theta - \mu_S \leq 0 , \ =0 \ {\rm if } \
  e_S > 0 . \EQN bondpark3;c \cr} $$
We analyze the consequences of these first-order conditions for
the optimal contract in three regions delineated by the
continuation values $v_L^*, v_L^{**}$.
%%values of the multipliers  $\mu_S, \mu_L$.

We break our analysis into two parts.  We begin by displaying particular
policies that attain initial values on the constrained Pareto
frontier.  Later, we show that there can be many  additional
policies that attain the same values, which as we shall see is
a consequence of a flat
interval in the   constrained Pareto frontier.

\subsection{Baseline policies}
%%\vskip.5cm
\medskip
\noindent
{\bf Region I:} $v_L \in [v_L^*, v_L^{**}]$ (neither PC binds)
\smallskip
\noindent When the initial value is in this interval, the continuation
value stays in this interval.
From the envelope property, %%Benveniste-Scheinkman formula,
$P'(v_L) = - \theta $.
If $v_L \in [v_L^*, v_L^{**}]$,
  neither participation constraint binds, and we have
$\mu_S=\mu_L=0$.   Then \Ep{bondpark3;b} implies
$$ P'(y) = P'(v_L).  $$
%By the concavity of $P$,
 This can be satisfied
by setting $y=v_L$.
Then $y=v_L$ and the always binding promise-keeping constraint
in \Ep{bondpark2;a}   imply that
$$v_L = y = {u_L(t_L) +  e_S \over 1 - \beta } \geq v_L^* > v_L^N \equiv
{u_L(t_L^N) \over 1 - \beta } ,   \EQN bopark1
$$
where the weak inequality states that $v_L$ trivially satisfies the
lower bound of region I, which in turn is strictly
greater than the Nash value $v_L^N$ according to expression \Ep{bopark_lbound}.
Because $u_L(t)$ is maximized
at $t_L^N$, the strict inequality in expression \Ep{bopark1}  holds
only if $e_S >0$.
  Then inequality \Ep{bondpark3;c}
and   $e_S >0$ imply
that $\theta = 1$.
Rewrite \Ep{bondpark3;a} as
$$u_W'(t_L)   \leq  0, \;\; =0 \ {\rm if}  \ t_L >0. $$
By Proposition 1, this implies that $t_L =0$.
We can solve for $e_S$  from
$$ v_L = {u_L(0) + e_S \over 1 -\beta} \EQN bondpark8 $$ and
then obtain $P(v_L)$ from ${u_S(0) - e_S \over 1-\beta}$.

Before turning to region II with $v_L > v_L^{**}$, we shall first
establish that there indeed exist such high continuation values for the
large country which cannot be sustained by a time-invariant transfer
scheme. This is done by showing that $P(v_L^{**}) > v_S^N$. That is,
there is scope for further increasing the continuation value of
the large country beyond $v_L^{**}$ before the associated continuation
value of the small country is reduced to $v_S^N$. The argument goes
as follows:
$$\eqalign{
P(v_L^{**}) &= {u_W(0) \over 1-\beta} - v_L^{**} \cr &=
{u_L(0)+u_S(0)\over 1-\beta}-{u_L(0) + \beta (u_S(0) - u_S(t_L^N))
\over 1 -\beta}\cr &= u_S(0) + \beta { u_S(t_L^N) \over 1 -\beta}
>
   u_S(t_L^N) + \beta { u_S(t_L^N) \over 1 -\beta} \equiv v_S^N, \cr }
                                                                \EQN bopark_ubound
$$
where the first equality uses the fact that the continuation value
$v_L^{**}$ lies on the unconstrained Pareto
frontier whose slope is $-1$ and the second equality invokes expression
\Ep{VLmax}. It then follows that $P(v_L^{**}) > v_S^N$.



%%%%%%%%%%%%%%%%%%%%%%%%%%%%%%%%
%Then $y=v_L$, \Ep{bondpark2;a}, and
%\Ep{bondpark2;b}   imply
%that $v_L = y = {u_L(t_L) +  e_S \over 1 - \beta }
%> v_L^N = {u_L(t_L^N) \over 1-\beta}$.  Because $u_L(t)$ is maximized
%at $t_L^N$, this equation  holds  only if $e_S >0$.
%  Then inequality \Ep{bondpark3;c}
%and   $e_S >0$ imply
%that $\theta = 1$.
%Rewrite \Ep{bondpark3;a} as
%$$u_W'(t_L)   \leq  0, \ =0 \ {\rm if}  \ t_L >0. $$
%This implies that $t_L =0$.
%We can solve for $e_S$  from
%$$ v_L = {u_L(0) + e_S \over 1 -\beta} \EQN bondpark8 $$ and
%then obtain $P(v_L)$ from ${u_S(0) - e_S \over 1-\beta}$.
%%%%%%%%%%%%%%%%%%%%%%%%%%%%%%%%%%%%%

%\subsection{Region II: $\mu_S >0$ ($PC_S$ binds)}

\vskip.5cm
\medskip
\noindent
{\bf Region II:} $v_L > v_L^{**}$ ($PC_S$ binds)
\smallskip
\noindent We shall verify that in region II, there is a solution
to the first-period first order necessary conditions
with $\mu_S >0$ and $e_S >0$.
When $v_L> v_L^{**}$, $\mu_S \geq 0$ and $\mu_L =0$.
When $\mu_S >0$, inequality \Ep{bondpark3;c} and
$e_S >0$  imply
$$
 \theta = 1 + \mu_S > 1.  \EQN bopark2
$$   Express \Ep{bondpark3;a} as
$$ u_W'(t_L)  +(\theta -1) u_L'(t_L) \leq 0, \;\; =0 \ {\rm if} \ t_L >0.
\EQN bondpark4 $$
Because $u_W'(0) =0$ and $u_L'(0)>0$, this   inequality can be satisfied
only if $t_L>0$.    Equation     \Ep{bondpark3;b}
implies that
$$P'(y) =- (1+\mu_S)^{-1} \theta = -1,
$$
where the second inequality invokes \Ep{bopark2}.
Therefore, $y \in [v_L^*, v_L^{**}]$, the region
of the Pareto frontier whose slope is $-1$ and
in which neither participation constraint
binds. We can solve for the required transfer
from $t=1$ onward from the following version
of \Ep{bondpark8}: $$y= {u_L(0) + e_S' \over 1 -\beta}, \EQN bondpark12 $$
where $e_S'$ denotes the value of $e_S$ for $t\geq 1$, because
once we move into \hbox{region I}, we stay there, having a time invariant
$e_S'>0$ with $t_L'=0$,  as our analysis
of region I indicated.       We can solve for $t_L, e_S$ for period
zero as follows.  For a given $\theta >1$,
solve the following equations for
$y, P(y), t_L, e_S, P(v_L)$:
$$\EQNalign{ & u_S'(t_L) + \theta u_L'(t_L) = 0 \EQN bondpark6;a \cr
     & v_L = u_L(t_L) + e_S + \beta y \EQN bondpark6;b   \cr
    & - e_S + \beta P(y) = \beta v_S^N \EQN bondpark6;c \cr
    & P(v_L)  = u_S(t_L) - e_S + \beta P(y) \EQN bondpark6;d \cr
   & y + P(y) = {u_W(0) \over 1 -\beta} .  \EQN bondpark6;e \cr}  $$
To find the maximized value $P(v_L)$, we must search
over solutions of \Ep{bondpark6}
for the $\theta >1$ that corresponds to the specified initial
continuation value $v_L$, (i.e., we are performing the
minimization over $\mu_S$  entailed in finding the
saddlepoint of the Lagrangian).

Using expression \Ep{bondpark6;c}, we can show that the transfer $e_S$
in period zero is also strictly positive,
$$
e_S = \beta [P(y) - v_S^N] \geq \beta \left[P(v_L^{**}) - v_S^N\right]> 0,
$$
where we have used the fact that $y \leq v_L^{**}$ and invoked the finding
in expression \Ep{bopark_ubound} that $P(v_L^{**}) > v_S^N$. Concerning
the relative size of $e_S$ at $t=0$ compared to the transfer $e_S'$ that
the small country pays in period $t=1$ and forever afterwards, we notice
that $e_S'$ is also subject to a participation constraint \Ep{bondpark2;c}
with the very same continuation value $P(y)$ (but where $u_S(t_L')=u_S(0)$).
Hence, we can express \Ep{bondpark2;c} for all periods $t\geq1$, given
a time-invariant continuation value $P(y)$ determined by
\Ep{bondpark6}, as
$$
e_S' \leq \beta [P(y) - v_S^N] = e_S,
$$
where the equality sign follows from \Ep{bondpark6;c}. We conclude
that the transfer is nonincreasing over time for our solution to an initial
continuation value in region II.

Thus, in region II, $t_L>0$ in period $0$, followed by $t_L'=0$ thereafter.
Moreover, the initial promised value to the large country $v_L > v_L^{**}$
is followed by a lower time-invariant continuation value $y \leq v_L^{**}$.
Subtracting \Ep{bondpark6;b} from \Ep{bondpark12} gives
$$ y= v_L + (u_L(0) - u_L(t_L)) + (e_S' - e_S).  $$
The contract sets the continuation value
 $y < v_L$ by making  $t_L >0$ (thereby making
$u_L(0) - u_L(t_L) <0$) and also possibly letting $e_S' - e_S <0$, so that
transfers can fall between periods  $0$ and $1$.
In region II, country $L$ induces $S$ to accept free trade by
a two-stage lowering of the tariff from the Nash level,
 so that $0 < t_L <t_L^N$ in period $0$,
with $t_L'=0$ for $t \geq 1$;      in return, it gets
period $0$ transfers of $e_S > 0$ and constant transfers
$e_S' > 0$ thereafter.


%%%%%%%%%%%%%%%%%%%%%%%%%%%%%%%%%%%%%%%%%%%%%%%%%%%%
%Thus, in Region II, $t_L>0$ in period $0$, followed by $t_L=0$ thereafter.
%Subtracting \Ep{bondpark6;b} from \Ep{bondpark12} gives
%$$ y= v_L + (u_L(0) - u_L(t_L)) + (e_S' - e_S).  $$
%The contract sets the continuation value
% $y < v_L$ by making  $t_L >0$ (thereby making
%$u_L(0) - u_L(t_L) <0$) and also possibly letting $e_S' - e_S >0$, so that
%transfers can rise between periods  zero and one.
%In region II, country $L$ induces $S$ to accept free trade by
%a two-stage lowering of the tariff from the Nash level,
% so that $0 < t_L <t_L^N$ in period zero,
%with $t_L=0$ for $t \geq 1$;      in return, it gets
%period zero transfers of $e_S \geq 0$ and constant transfers
%$e_S' > e_S$ thereafter.
%%%%%%%%%%%%%%%%%%%%%%%%%%%%%%%%%%%%%%%%%%%%%%%%%%%%%%

%%\vskip.5cm
%\vfil\eject
\medskip
\noindent
{\bf Region III:} $v_L \in  [v_L^N, v_L^{*})$ ($PC_L$ binds)
\smallskip
\noindent The analysis of region III is subtle.\NFootnote{The findings
of this  section  reproduce ones summarized in
Bond and Park's (2002) corollary to their Proposition 2.} It is natural
to expect that $\mu_S =0, \mu_L >0$ in  this region.
However, assuming that $\mu_L >0$ can be shown to lead
to a contradiction, implying that the pair
$v_L, P(v_L)$ both is and is not on the unconstrained
Pareto frontier.\NFootnote{Please show this in exercise
 {\it \the\chapternum.2\/}.}

  We can avoid the contradiction by assuming that $\mu_L =0$, so that
the participation constraint for country $L$ is barely binding.
We shall construct a solution to \Ep{bondpark3} and \Ep{bondpark2}
with period $0$ transfer $e_S>0$.
Note that \Ep{bondpark3;c} with $e_S >0$ implies $\theta =1$,
which from the envelope property $P'(v_L) = -\theta$ implies
that $(v_L, P(v_L))$ is actually on the {\it un}constrained  Pareto
frontier, a reflection of the participation constraint
for country $L$ barely binding.
With $\theta = 1$ and $\mu_L=0$, \Ep{bondpark3;a} implies
that $t_L=0$, which confirms $(v_L, P(v_L))$  being on the Pareto frontier.
We  can then solve the following equations
for $P(v_L), e_S, y, P(y)$:
$$ \EQNalign{ & P(v_L) + v_L  = {u_W(0) \over 1 -\beta} \EQN bondpark9;a \cr
              & v_L = u_L(0) + e_S + \beta y \EQN bondpark9;b \cr
              & u_L(0) + e_S + \beta y = u_L(t_L^N) + e_S + \beta v_L^N
                  \EQN bondpark9;c \cr
              & P(v_L) = u_S(0) - e_S + \beta P(y) \EQN bondpark9;d  \cr
              & P(y) + y =
                   {u_W(0) \over 1 -\beta}. \EQN bondpark9;e \cr  }$$
We shall soon see that these constitute only four linearly
independent equations.  Equations \Ep{bondpark9;a} and
\Ep{bondpark9;e}  impose that both $(v_L, P(v_L))$ and $(y, P(y))$
lie on the unconstrained Pareto frontier.  We can solve these
equations recursively.  First, solve for $y$ from \Ep{bondpark9;c}.
Then solve for $P(y)$ from \Ep{bondpark9;e}.  Next, solve for
$P(v_L)$ from \Ep{bondpark9;a}. Get $e_S$ from \Ep{bondpark9;b}.
Finally, equations \Ep{bondpark9;a}, \Ep{bondpark9;b}, and
\Ep{bondpark9;d} imply that equation \Ep{bondpark9;e} holds, which
establishes the reduced rank of the system of equations.

  We can use \Ep{bondpark12} to compute $e_S'$, the transfer from
 period  $1$ onward.   In particular,
$e_S'$ satisfies $ y = u_L(0) + e_S' + \beta y$.   Subtracting
\Ep{bondpark9;b} from this equation gives
$$ y - v_L = e_S' - e_S > 0.$$

Thus, when
$v_L < v_L^{*}$, country $S$ induces country $L$ immediately
to reduce its tariff to zero by paying transfers that rise between
period $0$  and period $1$ and that  thereafter remain constant.
That the initial tariff is zero means that we are immediately
on the unconstrained Pareto frontier.  It just takes time-varying
transfers to put us there.

%%%%%%%%%%%%
%$$ \grafone{bondpark1.eps,height=2.5in}{{\bf Figure XXX.1}
%The constrained Pareto frontier $v_S = P(v_L)$
%in the Bond-Park model.}$$
%%%%%%%%%%%%%%%%%%%%%%%%%%%

\midfigure{bondpark1f}
\centerline{\epsfxsize=3truein\epsffile{bondpark1.eps}}
\caption{The constrained Pareto frontier $v_S = P(v_L)$ in the Bond-Park model.}
\infiglist{bondpark1f}
\endfigure


\vskip.5cm
\medskip
\noindent
{\bf Interpretations}
\smallskip
\noindent For values of $v_L$ within
regions II and III, time-invariant transfers $e_S$ from country
$S$ to country $L$ are not capable of sustaining immediate and
enduring free trade.  But patterns of time-varying transfers
and tariff reductions are able to induce both countries to
move permanently to free trade after a one-period transition.
There is an asymmetry between regions II and III, revealed in
%figure XXXX.1
Figure \Fg{bondpark1f} and in our finding that $t_L=0$ in region III, so
that the move to free trade is immediate.
The asymmetry emerges from a difference in the quality of
instruments that the unconstrained country ($L$ in region II, $S$ in
region III) has to induce the constrained country eventually to accept free
trade by moving those instruments over time appropriately to
manipulate the continuation values of the constrained country
to gain its assent.  In region II, where $S$ is constrained,
all that $L$ can do is manipulate the time path of $t_L$,
a relatively inefficient instrument because it is a distorting
tax.  By lowering $t_L$ gradually, $L$ succeeds in raising
 the continuation
values of $S$ gradually, but at the cost of imposing a distorting
tax, thereby keeping $(v_L, P(v_L))$ inside the Pareto frontier.
In region III, where $L$ is constrained, $S$ has at its disposal a
nondistorting  instrument for  raising country $L$'s continuation
value by  increasing the transfer $e_S$ after period $0$.


The basic principle at work is   to  make the continuation
value rise for the country whose participation constraint
is binding.

\subsection{Multiplicity of payoffs and continuation values}

We now find more equilibrium policies
that support values in our three regions.
The unconstrained Pareto frontier is a straight line in the
space $(v_L, v_S)$ with a slope of $-1$:
$$ v_L + v_S = {u_W(0) \over 1-\beta} \equiv W.
$$
This reflects the fact that utility is perfectly transferable
between the two countries. As a result, there is a continuum
of ways to pick current payoffs $\{r_i;\, i=L,S\}$ and
continuation values $\{v_i';\, i=L,S\}$ that deliver
the promised values $v_L$ and $v_S$
to country $L$ and $S$, respectively. For example, each
country could receive a current payoff equal to the annuity
value of its promised value, $r_i = (1-\beta)v_i$, and retain its
promised value as a continuation value, $v_i' = v_i$.
That would clearly deliver the promised value to each country,
$$
r_i + \beta v_i' = (1-\beta) v_i + \beta v_i = v_i.
$$
Another example would reduce the prescribed current payoff to
country $S$ by $\triangle_S>0$ and increase the prescribed payoff
to country $L$ by the same amount. Continuation values $(v_S',v_L')$
would then have to be set such that
$$ \EQNalign{
& (1-\beta) v_S - \triangle_S + \beta v_S' = v_S, \cr
& (1-\beta) v_L + \triangle_S + \beta v_L' = v_L. \cr}
$$
Solving from these equations, we get
$$
\triangle_S = \beta (v_S' - v_S) = - \beta (v_L' - v_L).
$$
Here country $S$ is compensated for the reduction in current
payoff by an equivalent increase in the discounted
continuation value, while
country $L$ receives corresponding changes of
opposite signs.

Since the constrained Pareto frontier coincides with the
unconstrained Pareto frontier in regions I and III,
we would expect that the tariff games would also be  characterized
by multiplicities of payoffs and continuation values. We will now
examine how the participation constraints shape
the range of admissible equilibrium values.


\vfil\eject
\medskip
\noindent
{\bf Region I (revisited):} $v_L \in [v_L^*, v_L^{**}]$
\smallskip
\noindent From our earlier analysis, an equilibrium in region I satisfies
$$ \EQNalign{
& u_L(0) + e_S + \beta y = v_L,                  \EQN regionI;a \cr
& u_L(0) + e_S + \beta y \geq u_L(t_L^N) + e_S + \beta v_L^N,
                                                 \EQN regionI;b \cr
& u_S(0) - e_S + \beta (W - y) \geq u_S(0) +\beta v_S^N  ,
                                                 \EQN regionI;c \cr} $$
where we have invoked that $P(y)=W-y$ in regions I and III.
We  consider only $y \in [v_L^N, v_L^{**}]$ because our
earlier analysis ruled out any transitions from region I to region II.

Equation \Ep{regionI;a} determines the transfer and continuation
value needed to deliver the promised value $v_L$ to country $L$
under free trade:
$$
e_S + \beta y = v_L - u_L(0).
$$
The participation constraint for country $S$ requires that
inequality \Ep{regionI;c} be satisfied, which can be rewritten as
$$
e_S + \beta y \leq \beta (W - v_S^N) .      \EQN regionIS
$$
Since we are postulating that we are in region I with no
binding participation constraints, this condition
is indeed satisfied. Notice that incentive compatibility on
behalf of country $S$ does not impose any restrictions on the
mixture of transfer and continuation value
that deliver $e_S+\beta y$ to country $L$ beyond
our restriction above that $y \leq v_L^{**}$.

Turning to the participation constraint for country $L$,
we can rearrange inequality \Ep{regionI;b} to become
$$
y \geq \beta^{-1} \Bigl(u_L(t_L^N) + \beta v_L^N - u_L(0) \Bigr)
= \beta^{-1} \Bigl(v_L^N - u_L(0) \Bigr) = v_L^*.
$$
Thus, there cannot be a transition from region I to region III, a result
to be interpreted as follows.
We showed earlier that free trade is not incentive compatible with a time-invariant
transfer when the promised value of country $L$ lies in region III.
In other words, an initial promised value in region III cannot
by itself serve
as a continuation value to support free trade.
Now we are trying to attain free trade by offering country $L$
a continuation value in that very region III together with
a transfer that is even larger than the time-invariant transfer
considered earlier. (The transfer is larger than the
earlier time-invariant transfer because the initial
promised value $v_L$ is now assumed to lie in region I.)
Since that continuation value in region III was not incentive compatible for
country $L$ at a smaller transfer from country $S$, it will
certainly not be incentive compatible now when the transfer
is larger.

We conclude that there is a multiplicity of current payoffs and
continuation values in region I. Specifically, admissible equilibrium
continuation values are
$$
y \in \left[ v_L^*, \;
          \min \left\{\beta^{-1}\left(v_L - u_L(0)\right),\, v_L^{**}\right\}\right],
                                                    \EQN regionIrange
$$
where the upper bound incorporates our nonnegativity constraint
on transfers from country $S$ to country $L$, i.e., imposing
$e_S\geq 0$ in equation \Ep{regionI;a}.


\vskip.5cm
\medskip
\noindent
{\bf Region II (revisited):} $v_L > v_L^{**}$
\smallskip
\noindent From our earlier analysis, an equilibrium in region II satisfies:
$$ \EQNalign{
& u_L(t_L) + e_S + \beta y = v_L                   \EQN regionII;a \cr
& u_L(t_L) + e_S + \beta y \geq u_L(t_L^N) + e_S + \beta v_L^N
                                                 \EQN regionII;b \cr
& u_S(t_L) - e_S + \beta (W - y) = u_S(t_L) +\beta v_S^N  ,
                                                 \EQN regionII;c \cr} $$
where $0<t_L<t_L^N$ and we have used our earlier finding that the continuation
value $y$ will be in the region of the constrained Pareto frontier
whose slope is $-1$, i.e., $y \in [v_L^N, v_L^{**}]$ for which
$P(y)=W-y$.

Equation \Ep{regionII;c} determines the combination of the transfer and
continuation value received by country $L$:
$$
e_S + \beta y = \beta (W - v_S^N) .
$$
Once again, this participation constraint for
country $S$ does not impose any restrictions on the
relative composition of the transfer versus the continuation value
assigned to country $L$ (besides
our restriction above that $y \leq v_L^{**}$). For region II,
we have already shown that
the combined value of $e_S+\beta y$ is not sufficient to
support free trade, and that the necessary tariff in period 0 can
then be computed from equation \Ep{regionII;a}.

Finally, the participation constraint \Ep{regionII;b}
for country $L$
does impose a restriction on admissible equilibrium continuation
values $y$,
$$
y \geq \beta^{-1} \Bigl(u_L(t_L^N) + \beta v_L^N - u_L(t_L) \Bigr)
= \beta^{-1} \Bigl(v_L^N - u_L(t_L) \Bigr).
$$
Notice that this lower bound on admissible values of $y$ lies
inside region III,
$$
\beta^{-1}\Bigl(v^N_L -u_L(t_L)\Bigr) \cases{
> \beta^{-1} \Bigl(v^N_L -u_L(t^N_L)\Bigr) = v^N_L, &\cr
\noalign{\vskip.1cm}
< \beta^{-1} \Bigl(v^N_L -u_L(0)\Bigr) = v^*_L. &\cr}
$$
In contrast to our analysis of region I, a transition
into region III is possible when the initial promised value
belongs to region II. The reason is that the constrained
efficient tariff is then strictly positive in period $0$, which
relaxes the participation constraint for country $L$.
Hence, the range of admissible continuation values in region II
becomes
$$
y \in \left[ \beta^{-1}\left(v^N_L - u_L(t_L)\right),\; v_L^{**}\right].
$$


\vskip.5cm
\medskip
\noindent
{\bf Region III (revisited):} $v_L \in  [v_L^N, v_L^{*})$
\smallskip
\noindent The study of multiplicity of current payoffs and continuation
values in region III exactly parallels our analysis of region I.
The range of admissible continuation values is once again given
by \Ep{regionIrange}. The lower bound of $v^*_L$ is pinned down
by the participation constraint \Ep{regionI;b}
for country $L$ and
this implies an immediate transition out of region III into region I.

For the lowest possible promised value $v_L = v_L^N$, the range
of continuation values in \Ep{regionIrange} becomes degenerate,
with only one admissible value of $y=v^*_L$. From equation
\Ep{regionI;a}, we can verify that the pair
$(v_L,y)=(v^N_L,v^*_L)$ implies an equilibrium transfer that is
zero, $e_S=0$. For any other promised value in region III, $v_L\in
(v_L^N, v_L^{*}]$, there is a multiplicity of current payoffs and
continuation values. We can then pick a continuation value
$y>v^*_L$ that implies that the participation constraint
for country $L$ is not binding. Without any binding
participation constraints, it becomes apparent why our
analysis of multiplicity in region I is also valid for region III.


\section{Another model}
Fuchs and Lippi (2006) are  motivated by a vision  about the nature of monetary unions that was not well captured by work in the previous literature.
In particular, earlier work (1) assumed away commitment problems between members  that would occur within  an ongoing currency union, and (2)
modeled the consequences of abandoning a currency  as reversion to a worst case outcome of a repeated game played by independent monetary-fiscal
authorities.  The Fuchs-Lippi paper repairs both of these deficiencies by (1) imposing participation constraints each period for each member within
a currency union, and   (2) assuming that the consequence of a breakup is to move to the {\it best\/} outcome of the game played by independent monetary-fiscal authorities.

Here is the setup. Two countries have ideal levels of a policy setting (e.g., an interest rate) that are each hit by country-specific
idiosyncratic shocks.  The history of these shocks is common knowledge.  When not in a union, the countries play a repeated game. The best
equilibrium outcome is the point to which the countries  revert after a breakup. When in a union, the two countries play another repeated game.  The authors
model the benefit of being in the union as making it harder to effect a surprise change than it is outside it, thereby making it easier to abstain
from opportunistic monetary policy that, e.g., exploits the Phillips curve to get short run benefits in exchange for long-run costs.
\auth{Fuchs, William}%
\auth{Lippi, Francesco}%

The authors use `dynamic programming squared' to express  equilibrium strategies within the currency union game in terms of the current observed shock vector
and continuation values. The union  chooses a `public good', namely, the common policy each period.
It is a weighted average of the ideal points for the two
individual countries, with the weights being tilted a country whose participation constraint is binding that period.  The authors show that there
are three possible cases: (1) the shocks and initial continuation values are such that only country A's constraint is binding, in which case the policy tilts toward country A's ideal point;
(2) only country B's participation constraint is binding, in which case the policy tilts toward country B's ideal point; (3) the continuation values
and shocks are such that both countries' participation constraints are binding.  In case (3), the currency union breaks up.

  Depending on the specification of functional forms, preferences, and the joint distribution of shocks, case (3) may or may not be possible.  When it is, one can use the model to calculate waiting times to breakup of a union.  Many currency unions have broken up in the past, an observation
  that could be used to help reverse engineer parameter values -- something that the authors don't do.

  It is interesting to compare this model with an earlier risk-sharing model of Thomas and Worrall  and Kocherlakota that we studied in chapter \use{socialinsurance}. In that model, there was no case 3 and the analogue of the union, the relationship between the firm and the worker in Thomas and Worrall or between two consumers in Kocherlakota, lasts forever. One never observes defaults along the equilibrium path.  What is the source of the different  outcome
   in the Fuchs-Lippi model? The answer hinges on the part of the payoff structure of the Fuchs-Lippi model that
  captures the `public good' aspect of the monetary union policy choice. Both countries have to live with the same setting of a policy instrument
  and what one gains the other does not necessarily lose.  In the chapter \use{socialinsurance} model, each period when one person gets more, the other necessarily gets less, creating a symmetry in the participation constraints that prevents them from binding simultaneously.

\section{Concluding remarks}

Although the substantive application differs, mechanically
the models of this chapter work much like models that we studied in
chapters \use{stackel}, \use{socialinsurance}, and \use{credible}.  The key idea
is to cope with binding incentive constraints (in this case, participation
constraints), partly by changing the continuation values for those
agents whose incentive constraints  are binding.   For example, that creates
``intertemporal tie-ins'' that  Bond and Park
interpret as ``gradualism.''

\appendix{A}{Computations for Atkeson's model}

  It is instructive to compute  a numerical example of the optimal
contract for Atkeson's (1988) model.  Following Atkseson, we work with the
following numerical example.  Assume
$u(c) = 2 c^{.5}, \lambda(I) = \bigl({I \over Y_n + 2M}\bigr)^{.5},
g_i(Y_j)={\exp^{-\alpha_i Y_j} \over \sum_{k=1}^n \exp^{-\alpha_i Y_k}}$
with $n=5, Y_1 =100, Y_n =200, M=100, \alpha_1 = \alpha_2 =-.5, \delta =.9$.
 Here is  a version of  Atkeson's numerical algorithm:
\auth{Atkeson, Andrew}
\medskip
\noindent{1.}  First, solve the Bellman equation \Ep{bellmanaut} and
\Ep{foncautark}
for the autarky value $U(Q)$.
Use a polynomial for the value function.\NFootnote{We recommend the
Schumaker shape-preserving spline mentioned in chapter \use{practical} and
described by Judd (1998).}
\index{spline!shape preserving}
\medskip

\noindent{2.}  Solve the Bellman equation for the full-insurance
setting for the value function $W(Q)$ as follows.
First, solve equation \Ep{atnew5} for $I$.  Then solve equation
\Ep{atnew6;b} for $d(Y') = Y' - Q$ and compute $c = c(Q)$ from
\Ep{atnew6;a}.  Since $c$ is constant, $W(Q) =  u[c(Q)]$.
\medskip
Now, solve the Bellman equation for the contract with limited commitment
and unobserved action.
First, approximate $V(Q)$   by a polynomial, using
the method described in chapter \use{practical}.
Next, iterate on the Bellman equation, starting from
initial value function $V^0(Q) = W(Q)$ computed earlier. As
Atkeson shows, it  is
important to start with a value function {\it above} $V(Q)$.
We know that $W(Q) \geq V(Q)$.


Use the following steps:

\medskip
\item{1.} Let $V^j(Q)$ be the value function at the $j$th
iteration.  Let $d$ be the vector $\left[\matrix{d_1 & \ldots & d_n \cr}
\right]'$.   Define
$$ X(d) = \sum_i V^j (Y_i - d_i) [g_0(Y_i) - g_1(Y_i)]. \EQN foncI $$
   The first-order condition
for the borrower's problem \Ep{bellman1;e} is
$$ -(1-\delta) u'(Q+b-I) + \delta \lambda'(I) X \geq 0,
\quad = 0 \ {\rm if} \ I >0. $$
Given a candidate
continuation value function $V^j$, a value $Q$,
and    $b, d_1, \ldots, d_n$, solve the borrower's
first-order condition for a function
 $$ I = f(b, d_1, \ldots, d_n; Q).  $$
Evidently, when $X(d) <0$, $I=0$.  From equation \Ep{foncI} and
the particular example,
$$ I=f(b,d;Q) = {\delta^2 (Y_n + 2M) X(d)^2 \over 4(1-\delta)^2
       + \delta^2 (Y_n + 2M) X(d)^2 } (Q +b).  \EQN compute1  $$
Summarize this equation in a Matlab function.


\medskip
\item{2.}  Use equation \Ep{compute1} and the constraint \Ep{bellman1;c}
at equality to form
$$ b= \delta \sum_i d_i g[Y_i, f(b, d)]. $$
Solve this equation for a new function
$$ b= m(d) .\EQN compute2 $$
\medskip
\item{3.}  Write one step on the Bellman equation
as
$$\eqalign{ V^{j+1}(Q) = & \max_{d} \Biggl\{ (1-\delta)
     u\bigl[Q + m(d) -
f(m(d), d)\bigr] \cr
  & +  \delta \sum_i  V^j(Y_i - d_i) g\bigl[Y_i,
   f(m(d), d)\bigr] \cr
& - \sum_i \theta_i \Bigl[ \max\bigl(0,U(Y_i) -
    V^j(Y_i-d_i)\bigr) \Bigr] \cr
  & - \sum_i \eta_i \max[0, -d_i - M] - \eta_0 \max[0, m(d) -M] \Biggr\}, \cr}
                                        \EQN compute3$$
where $V^j(Q)$ is the value function at the $j$th iteration, and
$\theta_i >0, \eta_i$ are positive penalty parameters designed
to enforce the participation constraints \Ep{bellman1;d} and
the restrictions on the size of borrowing and repayments.
The idea is to set the $\theta_i$'s and $\eta_i$'s  large enough to
assure that $d$ is set so that constraint \Ep{bellman1;d} is satisfied
for all $i$.



%\section{Exercises}
\showchaptIDfalse
\showsectIDfalse
\section{Exercises}
\showchaptIDtrue
\showsectIDtrue
\medskip

\medskip
\noindent {\it Exercise \the\chapternum.1}
\medskip
\noindent Consider a version of Bond and Park's model with
$\gamma_S=.4$ and  the payoff functions  \Ep{bppayoff1;a} and
\Ep{bppayoff1;b}, with
$$\eqalign{ u_L(t_L)  &  = -.5 (t_L - .5 )^2  \cr
            u_W(t_L)  &  = - .5 t_L^2 ,    \cr}  $$
where $u_W(t_L) = u_L(t_L) + u_S(t_L)$.
\medskip
\noindent{\bf a.} Compute the cutoff value $\beta_c$ from \Ep{betac}.
For $\beta \in (\beta_c,1)$, compute
$v_L^*, v_L^{**}$.
\medskip
\noindent
{\bf b.}  Compute the constrained Pareto frontier.  ({\it Hint:}  In
region II, use \Ep{bondpark6} for a grid of values
$v_L$ satisfying $v_L > v_L^{**}$.)

\medskip
\noindent{\bf c.}  For a given $v_L \in (v_L^N, v^*)$,
compute $e_S, e_S', y$.


\medskip
\noindent{\it Exercise \the\chapternum.2} \quad
\medskip
\noindent Consider the Bond-Park model
analyzed above.   Assume that in region III,  $\mu_L >0, \mu_S =0$.  Show
that this leads to a contradiction.
