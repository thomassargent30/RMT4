%  revised by Tom, March 9, 2003


\input grafinp3
\input psfig
%\eqnotracetrue
\footnum=0
%\showchaptIDtrue


%\def\@chaptID{17.}
%\def\tone{{t+1}}
%
%\psdraft    % drop this when doing for real


%\hbox{}


\chapter{Fiscal-Monetary Theories of Inflation\label{fiscalmonetary}}

\section{The issues}
This chapter introduces some issues in monetary theory that mostly
revolve around coordinating monetary and fiscal policies. We start from
the observation that \index{complete markets!no role for money}
complete markets models have no role for inconvertible currency, and
therefore assign zero value to it.\NFootnote{In complete markets models,
money holdings would only serve as a store of value. The following
transversality condition would hold in a nonstochastic economy:
$$\lim_{T\to\infty} \prod_{t=0}^{T-1} R_t^{-1} {m_{T+1}\over p_T} = 0.
$$
The real return on money, $p_t/p_{t+1}$, would have
to equal the return $R_t$ on other assets, which, substituted into the
transversality condition, yields
$$\lim_{T\to\infty} \prod_{t=0}^{T-1} {p_{t+1} \over p_t}
{m_{T+1}\over p_T} = \lim_{T\to\infty} {m_{T+1}\over p_0} = 0.  $$
That is,  an inconvertible money (i.e., one for which
$\lim_{T\to\infty} m_{T+1} > 0$) must be valueless, $p_0=\infty$.}
  We  describe  one way to alter
a complete markets economy so that a positive value is assigned to an
inconvertible currency: we impose a transaction technology
with shopping time and real money balances as inputs.\NFootnote{See Bennett
McCallum (1983) for an early shopping time specification.}  We
\auth{McCallum, Bennett T.}
use the model to illustrate $10$ doctrines in monetary economics.
Most of these doctrines transcend many of the details of
the model. The important thing about the transactions
technology is that it makes demand for currency a decreasing
function of the rate of return on currency.  Our monetary
doctrines mainly emerge from manipulating that demand function
and the government's intertemporal budget constraint under
alternative assumptions about government monetary and fiscal
policy.\NFootnote{Many of the doctrines were originally developed
in setups differing in details from the one in this chapter.}


  After describing our $10$ doctrines, we  use the model to analyze
two important issues: the validity of Friedman's rule in the
presence of distorting taxation, and its sustainability in the
face of a \idx{time consistency} problem. Here, we use the methods
for solving an optimal taxation problem with commitment in chapter
\use{optax}, and for characterizing a credible government policy
in chapter \use{credible}.




\section{A shopping time monetary economy}
Consider an endowment economy with no uncertainty.
A representative household has one unit of time.
There is a single good  of constant amount $y>0$ each period $t \geq 0$.
The good can be divided between private consumption $\{c_{t}\}_{t=0}^\infty$
and government purchases $\{g_t\}_{t=0}^\infty$, subject to
$$ c_{t} + g_t = y.                   \EQN doc_resource$$
The preferences of the household are ordered by
$$ \sum_{t=0}^\infty \beta^t u(c_{t},\ell_t)  ,        \EQN doc_pref$$
where $\beta \in (0,1)$, $c_t\geq 0$ and $\ell_t\geq 0$ are
consumption and leisure at time $t$, respectively, and
$u_c$, $u_\ell > 0$, $u_{cc}$, $u_{\ell \ell} < 0$, and $u_{c \ell} \geq 0$.
With one unit of time per period, the household's
time constraint becomes
$$1\,=\, \ell_t + s_t.                \EQN doc_time
$$
We use $u_c(t)$ and so on to denote  time $t$ values of
the indicated objects, evaluated at an allocation to be understood from the
context.


To acquire the consumption good,
the household allocates time to shopping.
The amount of shopping time $s_t$ needed
to purchase a particular level of consumption $c_t$ is negatively related
to the household's holdings of real money balances $m_{t+1}/p_t$.
Specifically, the shopping or transaction technology is
\index{shopping technology}
$$
s_t\,=\,H\!\left(c_t,\, {m_{t+1} \over p_t} \right),             \EQN doc_shop
$$
where $H,\, H_c,\, H_{cc},\, H_{m/p,m/p}\geq 0$, $H_{m/p}, H_{c,m/p}\leq 0$.  A parametric example of this
transaction technology is
$$
H\!\left(c_t,\, {m_{t+1} \over p_t} \right) \,=\, {c_t \over m_{t+1}/p_t }
\epsilon,
       \EQN doc_shop1 $$
where $\epsilon >0$.  This corresponds to a transaction cost that would
arise in the frameworks of Baumol (1952) and Tobin (1956).
When a household spends money holdings for consumption purchases at a
constant rate $c_t$ per unit of time, $c_t (m_{t+1}/p_t)^{-1}$ is the
number of trips to the bank, and $\epsilon$ is the time cost per trip
to the bank.
\auth{Baumol, William J.} \auth{Tobin, James}

\subsection{Households}
The household maximizes expression \Ep{doc_pref} subject to
the transaction technology \Ep{doc_shop} and the sequence of budget
constraints
$$ c_t + {b_{t+1} \over R_t} + {m_{t+1} \over p_t} =
   y - \tau_t + b_t + {m_t \over p_t}.                 \EQN doc_bc
$$
Here, $m_{t+1}$ is nominal balances held between times $t$ and $t+1$;
$p_t$ is the price level; $b_t$ is the real value of one-period government
bond holdings that mature at the beginning
of period $t$, denominated in units of time $t$ consumption;
$\tau_t$ is a lump-sum tax at $t$; and $R_t$ is the real gross
rate of return on one-period bonds held from
$t$ to $t+1$.  Maximization of expression \Ep{doc_pref} is subject
to $m_{t+1} \geq 0$ for all $t \geq 0$,\NFootnote{Households cannot
issue money.}
no restriction on the sign of $b_{t+1}$ for all $t \geq 0$,
and given initial stocks $m_0,\, b_0$.


After consolidating two consecutive budget constraints given by
equation \Ep{doc_bc}, we arrive at
$$\EQNalign{
c_t + {c_{t+1} \over R_t} +
& \left(1 - {p_t \over p_{t+1}}\, {1\over R_t} \right) {m_{t+1} \over p_t}
+ {b_{t+2} \over R_t R_{t+1}} + {m_{t+2}/ p_{t+1} \over R_t}      \cr
\noalign{\vskip.2cm}
    &=  y - \tau_t + {y - \tau_{t+1} \over R_t} + b_t + {m_t \over p_t}.
                                                        \EQN doc_bc2 \cr}
$$
To ensure a bounded budget set, the expression in parentheses multiplying
nonnegative holdings of real balances must be greater than or equal to zero.
Thus, we have the arbitrage condition,
$$
1 - {p_t \over p_{t+1}} {1\over R_t} = 1- {R_{mt} \over R_t} =
{i_{t} \over 1+i_{t}} \geq 0,                \EQN doc_arbitrage
$$
where $R_{mt}\equiv p_t/p_{t+1}$ is the real gross return on money
held from $t$ to $t+1$, that is, the inverse of the inflation rate,
and $1+i_{t}\equiv R_t/R_{mt}$ is the gross nominal interest rate.
The real return on money $R_{mt}$ must be less than or equal to the
return on bonds $R_t$, because otherwise  agents would be able to make
arbitrarily large profits by choosing arbitrarily large money holdings
financed by issuing
bonds. In other words, the net nominal interest rate $i_{t}$
cannot be negative.


The Lagrangian for the household's optimization problem is
$$\EQNalign{
\sum_{t=0}^\infty \beta^t \Biggl\{ u(c_{t}, \ell_{t})  + \lambda_t
& \Bigl(y - \tau_t + b_t + {m_t \over p_t}  - c_t - {b_{t+1} \over R_t}
          - {m_{t+1} \over p_t} \Bigr)\Biggr.  \cr
&\Biggl. + \mu_t \Bigl[1 - \ell_t - H\!\Bigl(c_t,\, {m_{t+1}\over p_t}\Bigr)
                                                     \Bigr]\Biggr\}. \cr}
$$
At an interior solution, the first-order conditions with respect to
$c_{t}$, $\ell_{t}$, $b_{t+1}$, and $m_{t+1}$ are
$$\EQNalign{
u_c(t) - \lambda_{t} - \mu_t \,H_c(t)&= 0,            \EQN doc_foc1 \cr
u_{\ell}(t) - \mu_t &= 0,                             \EQN doc_foc2 \cr
- \lambda_{t} {1 \over R_t} + \beta \lambda_{t+1} &= 0,
                                                       \EQN doc_foc3 \cr
- \lambda_{t} {1\over p_t}- \mu_t \, H_{m/p}(t) {1\over p_t}
+  \beta \lambda_{t+1} {1 \over p_{t+1}}  &= 0.      \EQN doc_foc4 \cr }
$$
From equations \Ep{doc_foc1} and \Ep{doc_foc2},
$$
\lambda_{t} = u_c(t) - u_{\ell}(t) \, H_c(t).        \EQN doc_res1
$$
The Lagrange multiplier on the budget constraint is equal to the
marginal utility of consumption reduced by the marginal disutility of
having to shop for that increment in consumption. By substituting
equation \Ep{doc_res1} into equation \Ep{doc_foc3}, we obtain an
expression for the real interest rate,
$$
R_t \,=\; {1 \over \beta} \
{ u_c(t) - u_{\ell}(t) \, H_c(t)
\over
u_c(t+1) - u_{\ell}(t+1) \, H_c(t+1) }.                        \EQN doc_R
$$


The combination of equations
\Ep{doc_foc3} and \Ep{doc_foc4} yields
$$
{R_t - R_{mt} \over R_t}\, \lambda_t
  \,=\, -\mu_t \, H_{m/p}(t),        \EQN doc_res2
$$
which sets the cost equal to the benefit of the marginal unit of
real money balances
held from $t$ to $t+1$, all expressed in time $t$ utility.
The cost of holding money balances instead of bonds is lost interest
earnings $(R_t-R_{mt})$ discounted at the rate $R_t$ and expressed in
time $t$ utility when multiplied by the shadow price $\lambda_t$.
The benefit of an
additional unit of real money balances is the savings in shopping
time $-H_{m/p}(t)$ evaluated at the shadow price $\mu_t$.
By substituting equations \Ep{doc_foc2} and \Ep{doc_res1} into
equation \Ep{doc_res2}, we get
$$
\Bigl(1- {R_{mt} \over R_t}\Bigr)\,
\Bigl[ {u_c(t) \over u_{\ell}(t) }
      - H_c(t)     \Bigr]    + H_{m/p}(t)\,=\,0,
                                                    \EQN doc_Md
$$
with $u_c(t)$ and $u_{\ell}(t)$ evaluated at $\ell_t=1-H(c_t,\, m_{t+1}/p_t)$.
Equation \Ep{doc_Md} implicitly defines a money demand function
\index{money!demand function}
$$
{m_{t+1} \over p_t} \,=\, F(c_t, R_{mt}/R_t),             \EQN doc_Md2
$$
which is increasing in both of its arguments, as can be shown by
applying the implicit function rule to expression \Ep{doc_Md}.




\subsection{Government}
The government finances the purchase of the stream $\{g_t\}_{t=0}^\infty$ subject to the
sequence of budget constraints
$$ g_t = \tau_t + {B_{t+1} \over R_t} - B_t + {M_{t+1} - M_t \over p_t},
                                                              \EQN doc_gov $$
where $B_0$ and $M_0$ are given.  Here $B_t$ is government
indebtedness to the private sector, denominated in time $t$ goods,
maturing at the beginning of period $t$, and
$M_t$ is the stock of currency that
the government has issued as of the beginning of period $t$.




\subsection{Equilibrium}
We use the following definitions:
\medskip
\noindent{\sc Definition:} A {\it price system\/} is a pair of positive
sequences $\{R_t, p_t\}_{t=0}^\infty$.
%% \hfil\break \{p_t\}_{t=0}^\infty$.
\index{price system}
\medskip
\noindent{\sc Definition:}
 We take as  exogenous sequences $\{g_t, \tau_t\}_{t=0}^\infty$.  We
also take $B_0=b_0$ and $M_0=m_0> 0$ as  given. An {\it equilibrium\/} is
a price system,  a consumption sequence $\{c_t\}_{t=0}^\infty$, a sequence for
government indebtedness $\{B_t\}_{t=1}^\infty$, and a positive
sequence for the money supply $\{M_t\}_{t=1}^\infty$ for which the
following statements are true: (a) given the price system and taxes, the
household's optimum problem is solved with $b_t=B_t$ and $m_t = M_t$;
(b) the government's budget constraint is satisfied for all $t \geq 0$;
and (c) $c_t + g_t = y$.           \index{equilibrium}




\subsection{``Short run'' versus ``long run''}
 We shall study government policies designed to ascribe a definite
meaning to a distinction between outcomes in the ``short run'' (initial date)
and the ``long run'' (stationary equilibrium).
We assume
$$ \eqalign{ g_t &= g, \hskip.3cm \forall t \geq 0; \cr
             \tau_t &= \tau, \hskip.3cm \forall t \geq 1; \cr
             B_t &= B , \hskip.3cm  \forall t \geq 1. \cr} \EQN doc_stat1 $$
We permit $\tau_0 \neq \tau$ and $B_0 \neq B$.


   These settings of policy variables are designed to let us study
circumstances in which the economy is in a stationary
equilibrium for $t \geq 1$, but starts from some other
position at $t=0$.  We have enough free policy variables
to discuss two alternative meanings that the theoretical
literature has attached to the phrase ``open market operations.''
\index{open market operation}




\subsection{Stationary equilibrium}
   We seek an equilibrium for which
$$\eqalign{p_t/p_{t+1} &= R_m, \hskip.3cm \forall t \geq 0; \cr
            R_t &=R,\hskip.3cm \forall t \geq 0; \cr
            c_{t} &= c,\hskip.3cm \forall t \geq 0; \cr
            s_{t} &= s, \hskip.3cm \forall t \geq 0 .\cr} \EQN doc_stat2
$$
Substituting equations \Ep{doc_stat2} into equations \Ep{doc_R}
and \Ep{doc_Md2} yields
$$\eqalign{
R &\,=\, \beta^{-1},        \cr
{m_{t+1} \over p_t} &\,=% \, F(c, R_m/R) =
   f(R_m),   \cr} \EQN doc_stat3
$$
where we define $f(R_m) \equiv F(c,R_m/R)$ and
we have suppressed the constants $c$ and $R$ in the money
demand function  $f(R_m)$ in a stationary equilibrium.
Notice that $f'(R_m) \geq 0$, an inequality  that plays
an important role below.


Substituting equations \Ep{doc_stat1}, \Ep{doc_stat2}, and \Ep{doc_stat3}
into the government budget constraint \Ep{doc_gov}, using the
equilibrium condition $M_t = m_t$, and rearranging gives
$$ g - \tau+ B(R-1)/R = f(R_m)(1-R_m),\hskip.3cm \forall t \geq 1 .  \EQN doc_govLR
$$
Given the policy variables $(g,\tau,B)$, equation \Ep{doc_govLR}
determines the stationary rate of return on
currency $R_m$.  In \Ep{doc_govLR}, $g - \tau$ is the
net of interest deficit, sometimes called the operational
deficit; $g - \tau + B(R-1)/R$ is the gross of interest government
deficit; and $f(R_m) (1-R_m)$ is  the rate of seigniorage revenues
from printing currency.\NFootnote{The stationary value of seigniorage per period
is given by
$$
{M_{t+1} - M_t \over p_t} = {M_{t+1} \over p_t} - {M_t \over p_{t-1}}
{p_{t-1} \over p_t} = f(R_m) (1-R_m).$$} %
   The inflation tax rate is $(1-R_m)$ and
the quantity of real balances $f(R_m)$ is the base of the inflation tax.


\subsection{Initial date (time 0)}
Because $M_1/p_0 = f(R_m)$, the government budget constraint
at $t=0$ can be written
$$ M_0 / p_0  = f(R_m) - (g +B_0 -\tau_0) + B / R. \EQN doc_govSR $$




\subsection{Equilibrium determination}
Given the policy parameters $(g,\tau,\tau_0,B)$, the initial
stocks $B_0$ and $M_0$, and the equilibrium gross real interest
rate $R=\beta^{-1}$, equations \Ep{doc_govLR} and \Ep{doc_govSR}
determine $(R_m,p_0)$.
%%%\NFootnote{Notice that $B_0$ is a real stock of debt,
%%%and $M_0$ is a nominal stock.  Taking $B_0$ as exogenous is something
%%%that Woodford (1995) changed in his fiscal theory of the price
%%%level (see doctrine 9 below).}
The two equations are recursive: equation \Ep{doc_govLR} determines
$R_m$, then equation \Ep{doc_govSR} determines $p_0$.


It is useful to illustrate the determination of an equilibrium
with a parametric example. Let the utility function and the transaction
technology be given by
$$\eqalign{
u(c_{t},l_t) \,&=\, {c_t^{1-\delta} \over 1-\delta} +
                    {l_t^{1-\alpha} \over 1-\alpha},     \cr
H(c_t,\, m_{t+1}/p_t) \,&=\, {c_t \over 1 + m_{t+1}/p_t }, \cr}
$$
where the latter is a modified version of equation \Ep{doc_shop1}, so
that transactions can be carried out even in the absence of money.


For parameter values $(\beta, \delta, \alpha, c) =
(0.96, 0.7, 0.5, 0.4)$,
%Figure 17.1
Figure \Fg{mon1af} displays the
stationary gross of
interest deficit $g- \tau+B(R-1)/R$ and the stationary seigniorage
$f(R_m)(1-R_m)$;\NFootnote{For our parameterization in
Figure \Fg{mon1af},
%%Figure \Fg{mon1af}  %Figure 17.1
%%shows the stationary value of seigniorage per period,
%%$$
%%{M_{t+1} - M_t \over p_t} = {M_{t+1} \over p_t} - {M_t \over p_{t-1}}
%%{p_{t-1} \over p_t} = f(R_m) (1-R_m).
%%$$
 households choose to hold zero money balances
for $R_m < 0.15$, so at these rates there is no seigniorage
collected.  Seigniorage turns negative for $R_m>1$ because the government
is then continuously withdrawing money from circulation to raise the
real return on money above $1$.}
%
%Figure 17.2
Figure \Fg{mon1bf} shows $f(R_m) - (g +B_0 -\tau_0) + B / R$. %%%$M_0/p_0$.
 Stationary equilibrium is
determined as follows:  name constant values $\{g,\tau,B\}$ which
imply a stationary gross of interest
deficit $g-\tau+ B(R-1)/R$, then read an associated stationary
value $R_m$
from %Figure 17.1
Figure \Fg{mon1af} that satisfies equation \Ep{doc_govLR}; for this
value of $R_m$, find the value of $f(R_m) - (g +B_0 -\tau_0) + B / R$
in Figure \Fg{mon1bf} which is equal to $M_0/p_0$ by equation
\Ep{doc_govSR}. Thus, the initial price level $p_0$ is
determined because $M_0$ is given in period $0$.

%compute $f(R_m) - (g +B_0 -\tau_0) + B / R$, then
%read the associated equilibrium value $M_0/p_0$ from Figure \Fg{mon1bf}
%  %Figure 17.2
%that
%satisfies equation \Ep{doc_govSR}.



%%%%%%%%%%%%%%%%%%%
%\topinsert $$
%\grafone{mon1a.ps,height=2in}{{\bf Figure 17.1}
%Stationary seigniorage $f(R_m)(1-R_m)$
%as a function of the stationary rate of return on currency, $R_m$.
%An intersection of the stationary gross of interest
%deficit $g- \tau+B(R-1)/R$ with $f(R_m)(1-R_m)$ in this figure
%determines $R_m$.}$$
%%%%%%%%%%%%%%%%%%%%%%%%%%%


\midfigure{mon1af}
\centerline{\epsfxsize=2.8truein\epsffile{mon1a_new.ps}}
\caption{The stationary rate of return on currency, $R_m$, is determined by
the intersection between the stationary gross of
interest deficit $g- \tau+B(R-1)/R$ and the stationary seigniorage
$f(R_m)(1-R_m)$.}
%
%Stationary seigniorage $f(R_m)(1-R_m)$ as a function of the stationary
%rate of return on currency, $R_m$.  An intersection of the stationary gross of
%interest deficit $g- \tau+B(R-1)/R$ with $f(R_m)(1-R_m)$ in this figure
%determines $R_m$.}
\infiglist{mon1af}
\endfigure


%%%%%%%%%%%%%%%%%%%%%%%%
%$$\grafone{mon1b.ps,height=2in}{{\bf Figure 17.2}
%Real value of initial money balances $M_0/p_0$ as a function
%of the price level $p_0$. Given $R_m$, an
%intersection of $f(R_m) - (g +B_0 -\tau_0) + B / R$
%with $M_0/p_0$ in this figure determines $p_0$.}
%$$\endinsert
%%%%%%%%%%%%%%%%%%%%%%%%%%%


\midfigure{mon1bf}
\centerline{\epsfxsize=2.9truein\epsffile{mon1b_new.ps}}
\caption{Given $R_m$, the real value
of initial money balances $M_0/p_0$ is determined by
$f(R_m) - (g +B_0 -\tau_0) + B / R$. Thus, the price
level $p_0$ is determined because $M_0$ is given.}
%Real value of initial money balances $M_0/p_0$ as a function
%of the price level $p_0$. Given $R_m$, an intersection of
%$f(R_m) - (g +B_0 -\tau_0) + B / R$ with $M_0/p_0$ in this figure
%determines $p_0$.}
\infiglist{mon1bf}
\endfigure


\section{Ten monetary doctrines}\label{sec:10Mdoctrines}%
We now use equations \Ep{doc_govLR} and \Ep{doc_govSR} to explain
some important doctrines about money and government finance.
\index{quantity theory of money}%
\subsection{Quantity theory of money}
  The classic ``quantity theory of money'' experiment is to increase
$M_0$ by some factor $\lambda >1$ (a ``helicopter drop'' of money),
leaving all of the other parameters of the model fixed (including the
fiscal policy parameters ($\tau_0, \tau,g, B$)).  The effect is to
multiply the initial equilibrium price and money supply sequences
by $\lambda$ and to leave all other variables unaltered.
\index{deficit finance!as cause of inflation}
\subsection{Sustained deficits cause inflation}
The parameterization in Figures \Fg{mon1af} and \Fg{mon1bf} % 17.1 and 17.2
shows that there can be
multiple values of $R_m$ that solve equation \Ep{doc_govLR}.
As can be seen in Figure \Fg{mon1af}, %17.1,
 some values of the gross-of-interest deficit
$g - \tau + B(R-1)/R$ can be financed with either a low or high
rate of return on money. The tax rate on real money balances
is $(1-R_m)$ in a stationary equilibrium, so the higher $R_m$ that solves
equation \Ep{doc_govLR} is on the good side of a ``Laffer curve'' in
the inflation tax rate.
\index{Laffer curve}


   If there are multiple values of $R_m$ that solve equation
\Ep{doc_govLR}, we
shall always select the highest one for the purposes of doing our
comparative dynamic exercises.\NFootnote{In chapter \use{ogmodels}, we
studied the perfect-foresight dynamics of a closely related system and
saw that the stationary equilibrium selected here was {\it not\/} the
limit point of those dynamics.  Our selection of the higher rate of
return equilibrium can be defended by appealing to various forms of
``adaptive'' (nonrational) dynamics.  See Bruno and Fischer (1990),
Marcet and Sargent (1989), and Marimon and Sunder (1993).  Also,
see exercise {\it \the\chapternum.2.\/}} %%17.2.}
  The stationary equilibrium with the higher rate of
return on currency is associated with classical comparative dynamics:
an increase in the stationary gross-of-interest government budget
deficit causes a {\it decrease\/} in the rate of return on
currency (i.e., an increase in the inflation rate).  Notice how the
stationary equilibrium associated with the lower rate of return on
currency  has ``perverse'' comparative dynamics, from the point of view
of the classical doctrine that sustained government deficits
cause inflation.
\auth{Bruno, Michael}  \auth{Fischer, Stanley}
\auth{Marcet, Albert}  \auth{Sargent, Thomas J.}
\auth{Marimon, Ramon}  \auth{Sunder, Shyam}
\index{zero inflation policy}
\subsection{Fiscal prerequisites of zero inflation policy}
 Equation \Ep{doc_govLR} implies a restriction on fiscal policy that
is necessary and sufficient to sustain a zero inflation ($R_m = 1$)
equilibrium:
$$ g - \tau + B (R-1)/R =0 ,$$
or
$$ B = {R \over R-1} (\tau -g) = \sum_{t=0}^\infty R^{-t} (\tau -g).$$
This equation states that the real value of interest-bearing
government indebtedness equals the present value of the net-of-interest
 government {\it surplus}, with zero revenues being contributed by
an inflation tax. In this case, increased government debt implies a flow
of future government
surpluses, with complete abstention from the inflation tax.
\index{unpleasant monetarist arithmetic}
\subsection{Unpleasant monetarist arithmetic}
This doctrine describes the paradoxical effects of an open market operation
defined in the standard way that withholds from the monetary authority
the ability to alter taxes or expenditures.
   Consider an open market sale of bonds at time $0$, defined
as a {\it decrease\/} in $M_1$ accompanied by an {\it increase\/}
 in $B$, with all other government fiscal policy variables constant, including
($\tau_0, \tau$).  This policy can be analyzed
by increasing $B$ in equations \Ep{doc_govLR} and \Ep{doc_govSR}.
The effect of the policy
is to shift the permanent gross-of-interest deficit {\it upward\/}
by $(R-1)/R$ times the increase in $B$, which {\it decreases\/}
the real return on money $R_m$ in Figure \Fg{mon1af}. %17.1.
That is, the effect is unambiguously to
{\it increase\/} the stationary inflation rate (the inverse
of $R_m$). However, the effect on the initial price level $p_0$
can go either way,
depending on the slope of the revenue curve $f(R_m) (1-R_m)$;
the decrease in $R_m$ reduces the right-hand side of equation
\Ep{doc_govSR}, $f(R_m) - (g +B_0 -\tau_0) + B / R$, while the
increase in $B$ raises the value. Thus, the upward shift of the
curve in Figure \Fg{mon1bf} due to the higher value of $B$,
and the downward movement along that new curve due to the
lower equilibrium value of $R_m$, can cause $M_0/p_0$ to
move up or down,
that is, a decrease or an increase in the initial price
level $p_0$.
% Thus, the new equilibrium can
%move us to the left or the right along the curve $M_0 /p_0$ in
%Figure \Fg{mon1bf}, %17.2,
%that is, a decrease or an increase in the initial price
%level $p_0$.


\index{open market operation!one definition}
   The effect of a decrease in the money supply $M_1$ accomplished
through such an open market operation is at best temporarily to drive
the price level downward, at the cost of causing the inflation
rate to be permanently higher.  Sargent and Wallace (1981) called
this ``unpleasant monetarist arithmetic.''
\index{open market operation!another definition}


\subsection{An ``open market'' operation delivering neutrality}
We now alter the definition of open market operations for the purpose of
disarming  unpleasant monetarist arithmetic.   We supplement
 the fiscal powers of the monetary authority in a way that lets open
market operations have effects like those in the quantity theory experiment.
Let there be an initial equilibrium with policy values denoted by bars
over variables. Consider an open market sale or purchase defined as a
decrease in $M_1$ and simultaneous increases in $B$ and $\tau$ sufficient
to satisfy
$$ (1 -1/R)(\hat B -\bar B) = \hat \tau - \bar \tau , \EQN oe38$$
where variables with hats denote the new values of the corresponding
variables.  We assume that $\hat \tau_0 = \bar \tau_0$.


   As long as the tax rate from time $1$ on is adjusted
according to equation \Ep{oe38}, equation \Ep{doc_govLR} will be
satisfied at the initial value of $R_m$.  Equation \Ep{oe38}
imposes a requirement that the lump-sum tax $\tau$ be adjusted by
just enough to  service whatever additional interest payments are
associated with the alteration in $B$ resulting from the exchange
of $M_1$ for $B$.\NFootnote{This definition of an ``open market''
operation imputes unrealistic power to a monetary authority: on
earth, central banks don't set tax rates.} Under this definition
of an open market operation, reductions in $M_1$ achieved by
increases in $B$ and the taxes needed to service $B$ cause
proportionate decreases in the paths of the money supply and the
price level, and leave $R_m$ unaltered. In this way, we have salvaged a
version of
the pure quantity theory of money.
\index{optimal quantity of money}
\auth{Friedman, Milton}
\subsection{The ``optimum quantity'' of money}
Friedman's (1969) ideas about the optimum quantity of money can
be represented in Figures \Fg{mon1af} and \Fg{mon1bf}. %17.1 and 17.2.
Friedman noted that, given the stationary levels
of ($g,B$), the representative household prefers
stationary equilibria with higher rates of return on currency.
In particular, the higher the stationary level of real
balances, the better the household likes it.  By running
a sufficiently large gross-of-interest surplus, that is, a
negative value of  $g - \tau + B(R-1)/R$,
the government can attain any  value
 of $R_m \in (1,\beta^{-1})$.  Given $(g,B)$ and the target
value of $R_m$ in this interval, a tax rate $\tau$ can
be chosen to assure the required surplus.  The proceeds of the
tax are used to retire currency from circulation, thereby
generating a deflation that  makes the rate of return on
currency equal to the target value of  $R_m$.  According to Friedman, the optimal
policy is to satiate the system with real balances, insofar
as it is possible to do so.


The social value of real money balances in our model is that
they reduce households' shopping time. The optimum quantity
of money is the one that minimizes the time allocated to
shopping. For the sake of argument, suppose there is
a satiation point in real balances $\psi(c)$ for any consumption
level $c$, that is, $H_{m/p}(c,\, m_{t+1}/p_t)=0$ for
$m_{t+1}/p_t \geq \psi(c)$.
According to condition \Ep{doc_res2}, the government can
attain this optimal allocation only by choosing $R_m=R$, since
$\lambda_t, \, \mu_t >0$. (Utility is assumed to be strictly
increasing in both consumption and leisure.) Thus, welfare
is at a maximum when the economy is satiated with real balances.
For the transaction technology given by equation \Ep{doc_shop1},
the \idx{Friedman rule} can  be attained only approximately
because money demand is insatiable.


\index{legal restrictions}
\subsection{Legal restrictions to boost demand for currency}
If the government can somehow force households to increase their
real money balances to $\tilde f(R_m)>f(R_m)$, it can finance a
given stationary gross
of interest deficit $g - \tau + B(R-1)/R$ at a higher stationary
rate of return on currency $R_m$.  The increased demand for money
balances shifts the seigniorage curve in Figure \Fg{mon1af} %17.1
upward to
$\tilde f(R_m)(1-R_m)$, thereby increasing  the higher of the
two intersections of the curve $\tilde f(R_m)(1-R_m)$ with the
gross-of-interest deficit line in Figure \Fg{mon1af}. %17.1.
  By increasing the
base of the inflation tax, the rate $(1-R_m)$ of inflation
taxation can be diminished.
Examples of legal restrictions to increase the demand for government
issued currency include (a) restrictions on the rights of banks and
other intermediaries to issue bank notes or other close substitutes
for government issued currency,\NFootnote{In the U.S. Civil War, the U.S. Congress
taxed out of existence the notes that state-chartered banks had
issued, which before the war had been the country's paper currency.}
(b) arbitrary limitations on trading other assets that are close substitutes with
currency, and (c) reserve requirements.


   Governments intent on raising revenues through the inflation
tax have frequently resorted to legal restrictions and threats
designed to promote the demand for its currency.  In chapter
\use{townsend}, we shall study a version of Bryant and Wallace's
(1984) theory of some of those restrictions.  Sargent and Velde
(1995) describe  the sharp tools used to enforce such restrictions during ``the Terror'' during the French
Revolution.
\auth{Bryant, John}  \auth{Wallace, Neil} \auth{Velde, Fran\c
cois} \index{legal restrictions!Terror}


To assess the welfare effects of policies forcing households to hold higher
real balances, we must go beyond the incompletely articulated
transaction process underlying equation \Ep{doc_shop}.
We need an explicit model of how money facilitates transactions and
how the government interferes with markets to increase the demand for
real balances.  In such a model, there would be opposing effects on
social welfare. On the one hand, our discussion of the optimum quantity
of money says that a higher real return on money $R_m$ tends to improve
welfare. On the other hand, the imposition of legal restrictions aimed
at forcing households to hold higher real balances might elicit socially
wasteful activities devoted to evading those
restrictions.
\index{open market operation!one big one}
\auth{Lucas, Robert E., Jr.}  \auth{Wallace, Neil}
\auth{Paal, Beatrix}
\subsection{One big open market operation}
Lucas (1986) and Wallace (1989) describe  a  large open market
purchase of private indebtedness at time 0.  The purpose of the
operation is to provide the government with a portfolio of
interest-earning claims on the private sector, one that is
sufficient to permit it to run a gross-of-interest surplus.  The
government uses the surplus to reduce the money supply each
period, thereby engineering a deflation that raises the gross rate of
return on money above $1$. That is, the government uses its own
lending to reduce the gap in rates of return between its money and
higher-yield bonds. As we know from our discussion of the optimum
quantity of money, the increase in the real return on money $R_m$
will lead to higher
 welfare.\NFootnote{Beatrix Paal (2000) describes how the stabilization of
the second Hungarian hyperinflation had some features
of ``one big open market operation.''  After the stabilization
 the government lent the one-time seigniorage revenues gathered from
remonetizing the economy.  The severe hyperinflation
(about $4 \times 10^{24}$ in the previous year) had
reduced real balances of fiat currency virtually to zero.
Paal argues that the  fiscal aspects of the stabilization, dependent
as they were on those one-time seigniorage revenues, were foreseen and
shaped the dynamics of the preceding hyperinflation.}


To highlight the effects of the described open market policy, we
impose a nonnegative net-of-interest deficit, $g-\tau\geq 0$,
which prevents financing deflation by direct taxation.
The proposed operation is then to
increase $M_1$ and decrease $B$, with $B<0$ indicating
private indebtedness to the government. We generate a candidate
policy as follows:
Given values of ($g,\tau$), use equation \Ep{doc_govLR}
to pick a value of $B$ that solves equation
\Ep{doc_govLR} for a desired level of $R_m$, with
$1 < R_m \leq \beta^{-1}$.  Notice that
a negative level of $B$ will be required, since $g-\tau\geq 0$.
Substituting equation \Ep{doc_govSR} into equation \Ep{doc_govLR}
[by eliminating $f(R_m)$] and rearranging gives
$$ M_0/p_0 = \left({R-R_m\over 1-R_m}\right){B\over R} +
         \left({1\over 1-R_m}\right) (g-\tau) -
             (g+B_0-\tau_0) . \EQN oe39$$
The first term on the right side is positive,
while the remainder may be positive or negative.  The candidate
policy is  consistent with an equilibrium only if
$g,\tau,\tau_0,$ and $B_0$ assume values for which the
entire right side is positive.  In this case, there exists
a positive price level $p_0$ that solves equation \Ep{oe39}.


As an example,
assume that $g-\tau = 0$ and that $g + B_0 - \tau_0 =0$,
so that the government budget net of interest is balanced
from time $t=1$ onward.  Then we know that the right-hand side of
equation \Ep{oe39} is positive.  In this case it is feasible to operate
a scheme like this to support any return on currency
$1 < R_m < 1/\beta$. In the limit, when conducting an arbitrarily
large open market operation, the stationary return on money $R_m$ would
approach $1/\beta=R$ and, hence, $M_0/p_0$ in equation \Ep{oe39} would
approach zero. This means that the government is
engineering a hyperinflation in period $0$ that makes the
initial nominal money stock $M_0$ practically worthless.
But how is it that the government after such a hyperinflation
in period $0$, can support a stationary return on money of $R$
for the indefinite future? The explanation is as follows.
Since the hyperinflation in period $0$ has made the initial money holdings
almost worthless, the private sector's real balances at the end of
period $0$, $M_1/p_0$, come almost entirely from that period's
open-market operation. The government  injects that money stock
into the economy in exchange for interest-earning claims on the
private sector, $B/R \approx - M_1/p_0$. In future periods, the government
keeps those bond holdings constant while using the net interest earnings to
reduce the money supply in each future period. The government
passes  the interest earnings on to money holders by engineering a deflation
that yields a return on money equal to $R_m \approx R$.



\index{fiscal theory of inflation}
\subsection{A fiscal theory of the price level}
The preceding sections have illustrated what might be
called  a fiscal theory of {\it inflation}.   This theory
assumes that at time $t=0$ the government commits to a specific sequence of exogenous variables
ranging over $t \geq 0$. In particular, the government sets $g, \tau_0, \tau$, and $B$, while
$B_0$ and $M_0$ are inherited from the past. The  model then determines $R_m$ and $p_0$ via equations
\Ep{doc_govLR} and \Ep{doc_govSR}.  This system of equations determining equilibrium values is
recursive:
given $g, \tau$, and $B$, equation \Ep{doc_govLR} determines the rate of
return on currency $R_m$ (and therefore, in light of equation \Ep{doc_arbitrage}, inflation); then,  given $g, \tau, B$, and $R_m$, equation
\Ep{doc_govSR} determines $p_0$.  After $p_0$ is
determined, $M_1$ is determined from $M_1 / p_0 = f(R_m)$.
In this setting, the government  commits to a long-run gross-of-interest
 government deficit $g - \tau + B(R-1)/R$, and then the market
determines $p_0, R_m$.
\auth{Woodford, Michael} \auth{Sims, Christopher A.}%

  Woodford (1995) and Sims (1994)
 have converted  a version
of the same model into a fiscal theory of the {\it price level\/} by
altering  assumptions about the variables that the government sets.
 Rather than assuming that the government sets $B$, and thereby
the gross-of-interest government deficit, Woodford assumes
that $B$ is endogenous and that instead the government sets in advance
 a present value of seigniorage $f(R_m) (1-R_m)/(R-1)$.  This assumption
is equivalent to saying that the government commits  to fix either  the nominal interest rate or the gross
rate of inflation $R_m^{-1}$ (the nominal interest rate and $R_m$ are locked together by equation \Ep{doc_arbitrage}).
Woodford emphasizes that in the present setting, such a
nominal interest rate peg
\index{interest rate peg}
leaves the equilibrium price level process determinate.\NFootnote{Woodford (1995)
interprets this finding against the background of
a literature that occasionally asserted a different result,
namely, that interest rate pegging
led to price level indeterminacy because of the
associated  money supply endogeneity.
That other literature focused on the homogeneity properties of
conditions  \Ep{doc_R} and \Ep{doc_Md}:
the only ways in which the price level enters are as ratios to
the money supply or to the price level at another date.
This property suggested
that a policy regime that leaves
the money supply, as well as the price level, endogenous will
will determine neither.}
To illustrate Woodford's  argument in our setting,
 rearrange equation \Ep{doc_govLR} to
obtain
$$\eqalign{B/R
   &= {1\over R-1} \bigl[ (\tau-g) + f(R_m) (1-R_m) \bigr] \cr
   &= \sum^\infty_{t=1} R^{-t}(\tau-g) \,+\, f(R_m) {1-R_m \over R-1},\cr}
      \EQN Beqn  $$
which when substituted into equation \Ep{doc_govSR}  yields
$$\EQNalign{
&{M_0 \over p_0} \,+\, B_0 = \sum^\infty_{t=0} R^{-t} (\tau_t - g_t)
 \,+\, f(R_m) \Bigl( 1 + {1-R_m\over R-1}\Bigr)     \cr
& \quad = \sum^\infty_{t=0} R^{-t} (\tau_t - g_t) \,+\,
\sum^\infty_{t=1} R^{-t} f(R_m) (R-R_m) . \EQN doc_govPV \cr}
$$


In a stationary equilibrium, the real interest rate is equal to
$1/\beta$, so by multiplying the
nominal interest rate by $\beta$ we obtain the inverse of
the corresponding value for $R_m$.   Thus, pegging a nominal
rate is equivalent to pegging the inflation rate and
the steady-state flow of seigniorage $f(R_m)(1-R_m)$.
Woodford uses such
equations as follows: The government chooses $g, \tau, \tau_0$,
 and $R_m$ (or equivalently, $f(R_m)(1-R_m)$).  Then
equation \Ep{Beqn} determines $B$ as the present
value of the government surplus from time $1$ on, including seigniorage
revenues.   Equation \Ep{doc_govPV} then determines $p_0$.
Equation \Ep{doc_govPV}
says that the price level
is set to equate  the real value of {\it total\/} initial
government indebtedness to the present value of the
net-of-interest government surplus, including seigniorage revenues.
Finally, the endogenous quantity of real money balances is determined by the
demand function for money \Ep{doc_Md2},
$$
M_1/p_0 = f(R_m).  % F(y-g,\ R_m/R).
                       \EQN doc_demandM1
$$


Woodford uses this experiment to emphasize that without saying much
more, the mere presence of a ``quantity theory'' equation of the
form \Ep{doc_demandM1} does not imply the ``monetarist'' conclusion
that it is necessary to make the money supply exogenous in order
to determine the path of the  price level.


\auth{Buiter, Willem H.}
%\auth{Marimon, Ramon}
\auth{McCallum, Bennett T.}
    Several commentators
 have remarked that the Sims-Woodford
use of these equations puts the government on a different setting
than the private agents.\NFootnote{See Buiter (2002) %, Marimon (1998)
and McCallum (2001).} %
  Private agents' demand
curves are constructed by requiring their budget constraints
to hold for {\it all\/} hypothetical price processes, not just the
equilibrium  one.   However, under Woodford's assumptions
about what the government has already chosen
{\it regardless\/} of the $(p_0, R_m)$ it faces, the only way
an equilibrium can exist is if $p_0$ adjusts to make
equation \Ep{doc_govPV}  satisfied.  The government budget
constraint would not be satisfied unless $p_0$
adjusts to satisfy \Ep{doc_govPV}.


\auth{Sargent, Thomas J.} \auth{Wallace, Neil} By way of contrast,
in the fiscal theory of {\it inflation\/} described by Sargent and
Wallace (1981) and Sargent (2013), embodied in our description of
unpleasant monetarist arithmetic, the focus is on how the  one tax
rate that is assumed to be free to adjust, the inflation tax,
responds to fiscal conditions that the government inherits.  Sims
and Woodford forbid  the inflation tax from  adjusting, having set
it once and all for by pegging the nominal interest rate. They
thereby force other aspects of fiscal policy and the price system
to adjust. \index{exchange rate!indeterminacy}\auth{Kareken, John}
\auth{Wallace, Neil}
\subsection{Exchange rate indeterminacy}\label{Kareken-Wallace}%
  Kareken and Wallace's (1981)
%%%%and Helpman's (1981)
exchange rate indeterminacy result provides a good laboratory for
putting the fiscal theory of the price level to work. First, we
will describe a version of  Kareken and Wallace's result. Then, we
will show how it can be overturned by changing the assumptions
about policy to ones like Woodford's.


To describe the theory of exchange rate indeterminacy, we change
the preceding model so that there are two countries with identical
technologies and preferences. Let $y_i$ and $g_i$ be the endowment
of the good and government purchases for country $i=1,2$; where
$y_1+y_2=y$ and $g_1+g_2=g$. Under the assumption of complete
markets, equilibrium consumption $c_i$ in country $i$
is constant over time and $c_1+c_2=c$.


Each country issues currency. The government
of country $i$ has $M_{it+1}$  units of its currency
outstanding at the end of period $t$.  The price level in
terms of currency $i$ is $p_{it}$, and the exchange rate $e_t$ satisfies
the purchasing power parity condition
$p_{1t} = e_t p_{2t}$.   The household is indifferent
about which currency to use so long as both currencies
bear the same rate of return, and will not hold one with
an inferior rate of return.  This fact implies
that $p_{1t} / p_{1t+1}  =  p_{2t} / p_{2t+1}$, which
in turn implies that $e_{t+1} = e_t = e$.
Thus, the exchange rate is constant in a nonstochastic  equilibrium with
two currencies being valued.  We let
$M_{t+1} = M_{1t+1} + e M_{2t+1}$.
For simplicity, we assume that the money demand function is linear in the
transaction volume, $F(c,R_m/R) = c \hat F(R_m/R)$. It then follows that
the equilibrium condition in the world money market is
 $$ {M_{t+1} \over p_{1t}}  = f(R_m).    \EQN KW1  $$


In order to study stationary equilibria where all real variables remain
constant over time, we restrict attention to
identical monetary growth rates in the
two countries, $M_{it+1}/M_{it}=1+\epsilon$ for $i=1,2$.
We let $\tau_i$ and $B_i$ denote constant steady-state values
for lump-sum taxes, and real government
indebtedness for government $i$.
The budget constraint of government $i$ is
$$ \tau_i = g_i -  B_{i}{(1-R)\over R}
   -{M_{it+1} - M_{it} \over p_{it} } .  \EQN Budgi $$




  Here is a version of Kareken and Wallace's exchange rate
indeterminacy result:  Assume
that the governments of each country set $g_i$, $B_i$,
and $M_{it+1}=(1+\epsilon) M_{it}$, planning to adjust the
lump-sum tax $\tau_i$ to raise whatever revenues are needed to finance their
budgets. Then the constant monetary growth rate implies
$R_m = (1+\epsilon)^{-1}$ and equation \Ep{KW1} determines
the worldwide demand for real balances. But the exchange rate is not
determined under these policies. Specifically, the market clearing
condition for the money market at time 0 holds for {\it any} positive
$e$ with a price level $p_{10}$ given by
 $$ { M_{11} + e M_{21} \over p_{10}}  = f(R_m).    \EQN KW1_0  $$
For any such pair $(e, p_{10})$ that satisfies equation \Ep{KW1_0} with
an associated value for $p_{20}=p_{10}/e$,
governments' budgets are financed by setting lump-sum taxes
according to \Ep{Budgi}.   Kareken and Wallace
conclude that under such settings for government
policy variables, something more is needed to determine the exchange
rate.   With policy as specified here, the exchange rate
is indeterminate.\NFootnote{See Sargent and Velde (1990)
for an application of this theory to events surrounding
German monetary unification.}




\index{exchange rate!determinacy}
\subsection{Determinacy of the exchange rate retrieved}\label{Kareken-Wallace-fiscal}%
  A version of Woodford's assumptions about the variables
 that governments choose can render the exchange rate determinate.
Thus,  suppose that each
government sets a real level of seigniorage
$x_i = (M_{it+1} - M_{it} ) / p_{it}$ for all $t \geq 1$.
The budget constraint of government $i$ is then
$$ \tau_i = g_i -  B_{i}{(1-R)\over R}
   - x_i .  \EQN Budgi_2 $$
In order to study stationary equilibria where all real variables remain
constant over time, we allow for three cases with respect to
$x_1$ and $x_2$: they are both strictly positive, strictly negative, or
equal to zero.


To retrieve exchange rate determinacy,  we assume
that the governments of each country set $g_i$, $B_i$,
$x_i$, and $\tau_i$ so that budgets are financed according
to \Ep{Budgi_2}.
Hence, the endogenous inflation rate is pegged to deliver the
targeted levels of seigniorage,
$$ x_1 +   x_2 = f(R_m)(1-R_m).  \EQN Ret_money $$
The implied return on money $R_m$ determines the endogenous
monetary growth rates in a stationary equilibrium,
$$ R_m^{-1} =  {M_{it+1} \over M_{it}}\equiv  1+\epsilon ,
                 \hskip1cm \hbox{\rm for\ \ }i=1,2.   \EQN Money_growth$$
That is, nominal supplies of both monies grow at the rate of
inflation so that real money supplies remain constant over time.
The levels of those real money supplies satisfy the equilibrium
condition that the real value of net monetary growth is equal to
the real seigniorage chosen by the government,
$$ { \epsilon M_{it} \over p_{it} } = x_i,
                 \hskip1cm \hbox{\rm for\ \ }i=1,2.      \EQN Price_lev  $$
Equations \Ep{Price_lev} determine the price levels in the two countries
so long as the chosen amounts of seigniorage are not equal to zero,
which in turn determine a unique exchange rate,
$$
e = { p_{1t} \over p_{2t} } = { M_{1t} \over M_{2t} } { x_2 \over x_1 }
= { (1+\epsilon)^t M_{10} \over (1+\epsilon)^t M_{20} } { x_2 \over x_1 }
= { M_{10} \over M_{20} } { x_2 \over x_1 }.
$$
Thus,
with this  Sims-Woodford structure of government commitments (i.e., setting
of exogenous variables), the exchange rate is determinate. It is only the
third case of stationary equilibria with $x_1$ and $x_2$ equal to zero
where the exchange rate is indeterminate, because then there is no
relative measure of seigniorage levels that is needed to pin down the
denomination of the world {\it real} money supply for the purpose
of financing governments' budgets.


%%\note{If there were
%%outstanding government
%%debt denominated in nominal terms, exchange rate determinacy can be
%%retrieved even when $x_1$ and $x_2$ are zero.}
%%  ANY SUCH FOOTNOTE NEED TO BE ELABORATED UPON



\section{An example of exchange rate (in)determinacy}\label{sec:FTPL_KW_SW}%
As an illustration of the Kareken-Wallace exchange rate indeterminacy and
the Sims-Woodford fiscal theory of the price level, consider the following
version of the two-country environment in section \use{Kareken-Wallace}:
$$\EQNalign{
y_1 &= y_2 = y/2,   \EQN K_W1;a \cr
g_1 &= g_2 = 0,     \EQN K_W1;b \cr
B_1 &= B_2 = 0,     \EQN K_W1;c \cr
M_{10} &= M_{20},     \EQN K_W1;d \cr
{M_{1t+1} \over M_{1t}} &= {M_{2t+1} \over M_{2t}} = 1+\epsilon >1,
                     \hskip.5cm \forall t \geq 0. \EQN K_W1;e \cr}
$$
The governments in the two countries have no purchases to finance and
no bond holdings. The seigniorage raised by printing money is handed
over as lump-sum transfers to the households in each country, respectively.
The budget constraint of government $i$ is
$$ -\tau_i =  {M_{it+1} - M_{it} \over p_{it} } = x_i,  \EQN K_W2 $$
where the negative lump-sum tax, $-\tau_i$, is equal to the real value
of the country's seigniorage, $x_i$.

To operationalize the concept of exchange rate indeterminacy, we
assume that there is a `sunspot' variable that can take on three
values at the start of the economy.\NFootnote{Sunspots were
introduced by Cass and Shell (1983) to explain ``excess market
volatility.'' Sunspots represent extrinsic uncertainty not related
to the fundamentals of the economy.\index{sunspots}
\auth{Cass, David} \auth{Shell, Karl} } %
Each realization of the sunspot variable is associated with a
particular belief about the equilibrium value of the exchange rate
$e \in \{0, 1, \infty\}$ that will prevail in period $0$ and
forever thereafter. That is, depending on the sunspot realization,
all households will coordinate on one of the following three
beliefs about the equilibrium outcome in the world money market:
\itemitem{{\it i)} } the currency of country 2 is worthless
                               ($e=0$ and $p_{2t}=\infty, \forall t\geq0$);
\itemitem{{\it ii)} } the two currencies are traded one for one
                               ($e=1$ and $p_{1t}=p_{2t}, \forall t\geq0$);
\itemitem{{\it iii)} } the currency of country 1 is worthless
                               ($e=\infty$ and $p_{1t}=\infty,  \forall t\geq0$).

\noindent We assume that all households share the same belief
about the sunspot process, and that each sunspot realization is
perceived to occur with the same probability equal to $1/3$.

We also postulate that all households are risk-averse with
identical preferences and as stated in \Ep{K_W1;a} that they have
the same constant endowment stream. As initial conditions, the
representative household in country $i$ owns the
beginning-of-period money stock $M_{i0}$ of its country.


\subsection{Trading before sunspot realization}
The equilibrium allocation in this economy will depend on whether
or not households can trade before observing the sunspot
realization. In chapter \use{recurge}, we assumed that all trade
took place after any uncertainty had been resolved in the first
period. In our current setting, this would translate into
households trading after the sunspot realization, i.e., after the
agents have seen the sunspot and therefore after the coordination
of beliefs about the equilibrium value of the exchange rate. In
cases {\it i)} and {\it iii)}, this implies that the households in
the country with a valued currency will be better off because
their initial money holdings are valuable and they will receive
lump-sum transfers equal to their government's revenue from
seigniorage in each period. In case {\it ii)}, all households are
equally well off in the world economy because of identical budget
constraints.

Alternatively, we can assume that households can trade in markets
before the sunspot realization. In a complete market world, agents
would be able to trade in contingent claims with payoffs
conditional on the sunspot realization. Given the symmetries in
the environment with respect to preferences, endowment and
expected asset/transfer outcomes associated with the sunspot
process, the equilibrium allocation will be one of perfect pooling
with each household consuming $y/2$ in every period.\NFootnote{See
Lucas (1982) for a perfect pooling equilibrium in a two-country
world with two currencies. However, Lucas considers only {\it
intrinsic\/} uncertainty arising from stochastic endowment
streams.
\auth{Lucas, Robert E., Jr.}} %
Hence, the
households will use security markets to pool the risks associated
with the sunspot process. Given the {\it ex ante} symmetry in
possible sunspot realizations, it follows that equilibrium
contingent-claim prices will be such that a household in country $i$
can afford to trade half of its initial money holdings, $M_{i0}/2$,
and half of the entitlement to its future stream of lump-sum
transfers, $x_i/2$,
in exchange for the corresponding quantities from a household in
the other country. As a result, these diversified portfolios enable each
household to finance a smooth consumption stream equal to $y/2$
in every period regardless of the sunspot realization.

We have constructed a rational expectations equilibrium where the
equilibrium exchange rate is influenced by a sunspot process. But
even though the exchange rate can take on three different values
in this example, the households are insulated from any real
effects because of their trades in complete markets prior to the
sunspot realization. In this world, each government is assumed
 to print more of its currency each period at the net rate
$\epsilon>0$ and hand over the newly printed money to its
households as lump-sum transfers. The households in turn have
entered into contingent-claim contracts that oblige them to hand
over half of this newly printed currency to a household in the
other country, while receiving half of that other household's
government transfer. Given a sunspot realization that is
associated with either case {\it i)} or case {\it iii)} above, it
follows that these deliveries of newly printed currencies between
households are valuable in one direction but not in the other
direction.



\subsection{Fiscal theory of the price level}\label{sec:FTPL_take1}%
How can a fiscal theory of the price level overcome this
indeterminacy of the exchange rate? In the spirit of section
\use{Kareken-Wallace-fiscal}, suppose that each government sets a
real level of seigniorage given by
$$
x_1 = x_2 = 0.5 \cdot f\!\left({1\over 1+\epsilon}\right)
              \left[1 - {1\over 1+\epsilon}\right].
$$
From equations \Ep{Ret_money} and \Ep{Money_growth}, we
see that the governments  split the total world seigniorage
associated with a gross money growth rate equal to $1+\epsilon$.
Given such policies, both governments can satisfy their budget
constraints only if the equilibrium exchange rate is indeed $e=1$.
Hence, the fiscal theory of the price level here would  claim that
case {\it ii)} is the only viable rational expectations
equilibrium. In the words of Kocherlakota and Phelan (1999), ``the
fiscal theory of the price level is, at its core, a device for
selecting equilibria from the continuum which can exist in
monetary models.''

 Kocherlakota and Phelan (1999) are
 skeptical about this recommendation for selecting an equilibrium.
The fiscal theory proposes to rule out other  equilibria by
specifying government policies in such a way that government
budget constraints
 hold only for  one particular exchange rate. But what would happen
if the sunspot realization signals case {\it i)} or case {\it
iii)} to the households so that they actually abandon one
currency, making it worthless? The fiscal theory formulated by
Sims and Woodford contains no answer to this
question.
%\NFootnote{See Bassetto (2002) for a reformulation of the
%fiscal theory  that is careful to keep track of the way past
%actions of private agents influence the actions available to the
%government at
%each time $t$.} \auth{Bassetto, Marco}%
 Critics of the fiscal
theory of the price level instead prefer  to specify government
policies so that a government's budget constraint is satisfied for
{\it all\/} hypothetical outcomes, including $e\in\{0,\infty\}$.
For example, a government that finds itself issuing a worthless
currency could surrender its aspiration to make lump-sum transfers
with strictly positive value
 to its citizens, while  the other government
 would accept that the value of the  transfer of newly printed money to its citizens
has doubled in real terms. But of course, this remedy to the
puzzle would refute the fiscal theory of the price level and once
again render the  exchange rate indeterminate. \auth{Kocherlakota,
Narayana R.} \auth{Phelan, Christopher} %\auth{Buiter, Willem H.}
%\auth{McCallum, Bennett T.}



\subsection{A game theoretic view of the fiscal theory of the price level}\label{sec:BassettoFTPL}%
Bassetto (2002) agrees with criticisms of the fiscal theory of
the price level that question how the government can adopt a fiscal
policy  without being concerned about
outcomes that could make the policy infeasible. Bassetto  reformulates
the fiscal theory of the price level in terms of a game. The essence of his argument
is that in order to select an equilibrium, a government must specify
strategies for all arbitrary outcomes so that its desired outcome is the
only one that can be supported as an equilibrium outcome, merely on the
basis of individual rationality of private actors.
\auth{Bassetto, Marco}%

Bassetto (2002) studies  a government that seeks to
finance occasional deficits by issuing debt in a model with
`trading posts.' In such a trading environment it
might happen that not all government debt can be sold because private
agents fail to submit enough bids. What would  the equilibrium
outcome be then? The fiscal theory formulated by Sims and Woodford contains
no answer since it presupposes that the government budget constraint
will be satisfied for the specified fiscal policy. Bassetto provides
an answer by arguing that the government should formulate a strategy
for that and all other arbitrary outcomes. Specifically, the following
government strategy supports the desired fiscal policy as a unique
equilibrium outcome. If some debt cannot be sold,
the government responds by increasing taxes to make up for the present
shortfall, but without altering future taxes. Thus, ``the onset of a debt crisis
would be accompanied by an increase in the amount of resources that are
offered in repayment of debt and hence an increase in the rate of return
of government debt. As a consequence, any rational household would respond
to a debt crisis by lending the government {\it more}, rather than
less, which ensures that no such crisis can occur in an
equilibrium.''\NFootnote{A similar strategy would
establish Bassetto's version of the fiscal theory of the price level
in section \use{sec:FTPL_take1}. For example, suppose that each government
promises to increase taxation in order to purchase its currency if it turns
worthless, say, at the price level that would have prevailed in
case {\it ii)}. Such strategies can effectively rule out cases {\it i)}
and {\it iii)} as equilibrium outcomes, and make exchange rate $e=1$
the only possible equilibrium.}

\auth{Atkeson, Andrew}
\auth{Chari, V.V.}
\auth{Kehoe, Patrick J.}
Because Bassetto's argument works equally well in a real economy, the preceding paragraph did not mention money or nominal prices.
Moreover, our omission of money seems appropriate since Bassetto studies
a cashless economy where the relative price of goods and
nominal bonds merely determines the value of the unit of account
(the `dollar'). Atkeson et al. (2010) extend the analysis
to a monetary economy, and
%%%This  raises a modelling challenge when
%%%seeking to analyze incongruities that can arise between
%%%a government's desired outcome and  private agents  optimal actions.
%%% In Bassetto's cashless economy, it was straightforward
%%%to analyze how households might not submit enough bids at a trading
%%%post and therefore cause some government debt to be left unsold, because such
%%%`deviations' would not affect returns on past actions taken by private
%%%actors. In contrast, deviations in a monetary economy that affect
%%%the price level would in general overturn the optimality of past actions.
%%%To stay clear of that, Atkeson et al.\ (2009)
follow the same approach to multiplicity of equilibria that we took in the
cash-in-advance model in section \use{sec:SW_interestpeg}. %in chapter \use{optax}.
While theirs is a new-Keynesian model, they
analyze sunspot equilibria that satisfy a constraint similar to ours
when we imposed an unchanged value for the denominator of
equation \Ep{money_interest}, without any constraint on each individual
next-period price level. In the analysis of
Atkeson et al.\ (2010), the
corresponding restriction on sunspot equilibria is that the
expected inflation is unchanged when perturbing the sunspot-driven
uncertainty in next period's price level.

Note that  different versions of the fiscal theory of the price level
share the same key assumption that a government can fully commit to
its policy or strategy. In chapters \use{credible} and \use{chang},
we study credible government policies -- policies that a government
would like to enact under all circumstances.



%%%%%%%%%%%%%%%%%%%%%%%%%%%%%%%%%%%%%%%%%%%
%\subsection{Exchange rate (in)determinacy}
%Kareken and Wallace's (1981)
%exchange rate indeterminacy result provides a good laboratory for putting
%the fiscal theory of the price level to work.  First, we will describe a
%version of the Kareken-Wallace result.  Then we will show how it can be
%overturned by changing the assumptions about policy to ones like Woodford's.
%
%To describe the theory of exchange rate indeterminacy, we change
%the preceding model so that two governments each issue currency.
%The government of country $i$ has $M_{it+1}$ units of its currency
%outstanding at the end of period $t$.  The price level in
%terms of currency $i$ is $p_{it}$, and the exchange rate $e_t$ satisfies
%the purchasing power parity condition
%$p_{1t} = e_t p_{2t}$.   The household is indifferent
%about which currency to use so long as both currencies
%bear the same rate of return, and will not hold one with
%an inferior rate of return.  This fact implies
%that ${p_{1t} \over  p_{1t+1} } = { p_{2t} \over p_{2t+1} }$, which
%in turn implies that $e_{t+1} = e_t = e$.
%Thus, the exchange rate is constant in a nonstochastic  equilibrium with
%two currencies being valued.  We let
%$M_{t+1} = M_{1t+1} + e M_{2t+1}$.
%The equilibrium condition in the money market is
%now
% $$ {M_{t+1} \over p_{1t}}  = f(R_m).    \EQN KW1  $$
%
%We let $g_i, \tau_i$, and $B_i$ denote constant steady-state values
%for government purchases, lump-sum taxes, and real government
%indebtedness for government $i=1,2$. We assume that $g_i$, and
%$\tau_i$ are the expenditure and tax rates for government $i$ for
%all $t \geq 0$.   The budget constraint of government $i$ is
%$$ g_i = \tau_i  +  B_{it+1}/R - B_{it}
%   +{M_{it+1} - M_{it} \over p_{it} } .  \EQN Budgi $$
%We use variables without subscripts to denote worldwide totals.
%Then the sum of the government budget constraints for
%$t \geq 1$ can be written
%as
%$$ g  - \tau + B(R-1)/R = f(R_m) (1-R_m), \EQN KW2 $$
%and
%$$ g = \tau + B/R - B_0 + f(R_m) - {M_{10} +e M_{20} \over p_{10}}
%  , \EQN KW3 $$
%for $t=0$.
%
%  Here is a version of Kareken and Wallace's exchange rate
%indeterminacy result:  Assume
%that the governments of each country set $g_i, \tau_i$, and
%$B_i$, planning to adjust their rates of money creation
%to raise whatever revenues are needed to finance their
%budgets.   Then equation \Ep{KW2} determines
%$R_m$, and equation \Ep{KW3} is left to determine
%two variables, $p_{10}$ and $e$.   The system is underdetermined:
%if there is a solution of equation \Ep{KW3}
%with  positive $(p_{10}, e)$, then   for {\it any}
%positive $e$ there is another solution.  Kareken and Wallace
%conclude that under such settings for government
%policy variables, something more is needed to set the exchange
%rate.   With policy as specified here, the exchange rate
%is indeterminate.\NFootnote{See Sargent and Velde (1990)
%for an application of this theory to events surrounding
%German monetary unification.}
%\index{exchange rate!determinacy}
%\subsection{Determinacy of the exchange rate retrieved}
%A version of Woodford's assumptions about the variables
% that governments choose can render the exchange rate determinate.
%Thus,  suppose that rather than setting $B_i$, each
%government sets a constant rate of seigniorage
%$x_i = {M_{it+1} - M_{it} \over p_{it}}$ for all $t \geq 0$.
%Then $B_i / R$ is determined by the following version
%of equation \Ep{Beqn}:
%$$B_i / R =\sum_{\tau =1}^\infty R^{-t}\bigl[(\tau_i - g_i) + x_i\bigr]
%  \EQN Beqni $$
%and
%$$ x_1 +   x_2 = f(R_m)(1-R_m).  $$
%The time-$0$ worldwide budget constraint becomes
%$$ {M_0 \over p_{10}} + B_0 = \tau - g + B/R
%     + f(R_m). \EQN Budgworld $$
%The time-$0$ budget constraint for country $2$ can be
%written
%$$  e {M_{20} \over p_{10}} + B_{20} =
%  \tau_2 - g_2 + B_2 /R + \Bigl[f(R_m) - {M_{11} \over p_{10}}\Bigr],
%$$
%or
%$$  e {M_{20} \over p_{10}} + B_{20} =
%  \tau_2 - g_2 + B_2 /R + \Bigl[f(R_m) - \Bigl(x_1+ {M_{10} \over p_{10}}
%\Bigr)\Bigr] ,
%   \EQN Budgcount2 $$
%Equations \Ep{Budgworld} and \Ep{Budgcount2} are two independent equations

%in $e, p_{10}$ that can determine these two variables.  Thus,
%with Sims--Woodford's structure of government commitments (i.e., setting
%of exogenous variables), the exchange rate is determinate.
%%%%%%%%%%%%%%%%%%%%%%%%%%%%%%%%%%%%%%%%%%%%%%%%%%%%%%%%%%%%%%%%%%%%%%%%

 \index{optimal inflation tax} \index{optimal quantity
of money!Friedman rule}
%%\index{optimal quantity of money!|see{Friedman rule}}
\index{Friedman rule}
\section{Optimal inflation tax:  the Friedman rule}
Given lump-sum taxation, the sixth monetary doctrine (about
the ``optimum quantity'' of money) establishes the optimality of the Friedman
rule. The optimal policy is to satiate the economy with real balances by
generating a deflation that drives the net nominal interest rate to zero.
In a stationary economy, there can be deflation only if the government
retires currency with a
government surplus. We now ask if such a costly scheme remains
optimal when all government revenues must be raised through distortionary
taxation. Or would the Ramsey plan then include an inflation tax on
money holdings whose rate depends on the interest elasticity of money demand?


\auth{Correia, Isabel H.} \auth{Teles, Pedro}
Following Correia and Teles (1996),
we show that even with distortionary
taxation the Friedman rule is the optimal policy under a transaction
technology \Ep{doc_shop} that satisfies a  homogeneity condition.


Earlier analyses of the optimal tax on money in models with
transaction technologies include Kimbrough (1986), Faig (1988),
and Guidotti and Vegh (1993).
Chari, Christiano, and Kehoe (1996) also develop conditions for the optimality
of the Friedman rule in models with cash and credit goods (see
section \use{sec:nom_into_real}), and money in the
utility function.
\auth{Kimbrough, Kent P.} \auth{Faig, Miguel} \auth{Guidotti, Pablo E.} \auth{Vegh, Carlos A.}
\auth{Chari, V.V.} \auth{Christiano, Lawrence J.} \auth{Kehoe, Patrick J.}


\subsection{Economic environment}
We convert our
shopping time monetary economy into a production economy
with labor $n_t$ as the only input in a linear technology:
$$ c_{t} + g_t = n_t.                   \EQN doc_resource2$$
The household's time constraint becomes
$$
1\,=\, \ell_t + s_t + n_t.      \EQN doc_time2
$$


The shopping technology
is now assumed to be homogeneous of degree $\nu\geq 0$
in consumption $c_t$ and real money balances $\hat m_{t+1}\equiv m_{t+1}/p_t$;
$$
s_t = H(c_t,\, \hat m_{t+1})
    = c_t^{\nu} H\!\left(1, {\hat m_{t+1} \over c_t}\right),
                                         \hskip1cm \hbox{\rm for\ \ }c_t>0.
                                                             \EQN doc_shop2
$$
By Euler's theorem we have
$$
H_c(c, \hat m) c + H_{\hat m}(c, \hat m) \hat m \,=\, \nu H(c, \hat m). \EQN doc_shop2b
$$
For any consumption level $c$, we also assume a point of satiation in real
money balances $\psi c$ such that
$$
H_{\hat m}(c, \hat m) = H(c, \hat m) =  0, \hskip1.5cm
                       \hbox{\rm for\ \ }\hat m \geq \psi c.     \EQN doc_shop2c
$$




\subsection{Household's optimization problem}
After replacing net income $(y-\tau_t)$ in equation \Ep{doc_bc2}
by $(1-\tau_t) (1 - \ell_t -s_t)$, consolidation of budget
constraints yields the household's present-value budget constraint

$$
\sum_{t=0}^{\infty} q^0_t \left( c_t +
                          {i_{t} \over 1+i_{t}} \hat m_{t+1} \right) =
\sum_{t=0}^{\infty} q^0_t (1-\tau_t) (1 - \ell_t -s_t)
   + b_0 + {m_0 \over p_0},
                                                              \EQN doc_optbc
$$
where we have used equation \Ep{doc_arbitrage}, and $q^0_t$ is
the Arrow-Debreu price
$$
q^0_t = \prod_{i=0}^{t-1} R_i^{-1}
$$
with the numeraire $q^0_0=1$. We have also imposed the transversality
conditions,
$$\EQNalign{
&\lim_{T\to\infty} q^0_{T}
                   {b_{T+1} \over R_{T}} \,=\, 0,    \EQN doc_TVC;a \cr
&\lim_{T\to\infty} q^0_{T} \hat m_{T+1} \,=\, 0.
                                                       \EQN doc_TVC;b \cr
}$$
Given the satiation point in equation \Ep{doc_shop2c}, real money
balances held for
transaction purposes are bounded from above by $\psi$. Real balances may also
be held purely for savings purposes if money is not dominated in rate
of return by bonds, but an agent would never find it optimal to accumulate
balances that violate the transversality condition.
Thus, for whatever reason money is being held, condition \Ep{doc_TVC;b} must
hold in an equilibrium.


Substitute $s_t = H(c_t, \hat m_{t+1})$ into equation \Ep{doc_optbc},
and let $\lambda$ be the Lagrange multiplier on this present-value
budget constraint.  At an interior solution, the first-order conditions
of the household's optimization problem become
$$\EQNalign{
c_t\rm{:}&\quad \beta^t u_c(t) - \lambda q^0_t \bigl[(1-\tau_t) H_c(t) +
1 \bigr] =0,
                 \EQN shop_foc;a \cr
\ell_t\rm{:}&\quad \beta^t u_{\ell}(t) - \lambda q^0_t (1-\tau_t) =0,
                 \EQN shop_foc;b \cr
\hat m_{t+1}\rm{:}&\quad - \lambda  q^0_t \Bigl[(1-\tau_t) H_{\hat m}(t) +
{i_{t} \over 1+i_{t}} \Bigr] = 0. \EQN shop_foc;c \cr}
$$
From conditions \Ep{shop_foc;a} and \Ep{shop_foc;b}, we obtain
$$
{ u_\ell(t) \over 1-\tau_t } = u_c(t) - u_{\ell}(t) H_c(t).    \EQN shop_focX1
$$
The left side of equation \Ep{shop_focX1} is the utility of extra
leisure obtained from
giving up one unit of disposable labor income, which at the optimum should
equal the marginal utility of consumption reduced by the disutility of
shopping for the marginal unit of consumption, given by the right side
of equation \Ep{shop_focX1}.  Using condition
\Ep{shop_foc;b} and the corresponding expression for $t=0$ with the
numeraire $q^0_0=1$, the Arrow-Debreu price $q^0_t$ can be expressed as
$$
q^0_t = \beta^t {u_\ell(t) \over u_\ell (0) }\,{1-\tau_0 \over 1-\tau_t};
                                                          \EQN shop_focX2
$$
and by condition \Ep{shop_foc;c},
$$
{i_{t} \over 1+i_{t}} = - (1-\tau_t) H_{\hat m}(t).    \EQN shop_focX3
$$
This last condition equalizes the cost of holding one unit of real balances
(the left side) with  the opportunity value
of the shopping time that is released by an additional unit of real balances,
measured on the right side by the extra after-tax labor income that
can be generated.
\index{Ramsey plan}
\subsection{Ramsey plan}
Following the method for solving a Ramsey problem in chapter
\use{optax}, we use the household's first-order conditions to
eliminate prices and taxes from its present-value budget
constraint. Specifically, we substitute equations \Ep{shop_focX2}
and \Ep{shop_focX3} into equation \Ep{doc_optbc}, and then
multiply by $u_\ell (0)/(1-\tau_0)$. After also using equation
\Ep{shop_focX1}, the implementability condition becomes
%{\ninepoint
$$
\sum_{t=0}^\infty \beta^t \bigl\{ \bigl[ u_c(t) - u_{\ell}(t) H_c(t)\bigr]
c_t \,-\, u_\ell(t)  H_{\hat m}(t) \hat m_{t+1}
\,-\, u_\ell(t) (1 - \ell_t -s_t) \bigr\} = 0,
$$%}%endninepoint
where we have assumed zero initial assets, $b_0 = m_0 = 0$.
Finally, we substitute $s_t = H(c_t, \hat m_{t+1})$ into this
expression and invoke Euler's theorem \Ep{doc_shop2b}, to arrive at
$$
\sum_{t=0}^{\infty} \beta^t \left\{ u_c(t) c_t
\,-\, u_{\ell}(t) \left[ 1 - \ell_t - (1-\nu) H(c_t,\, \hat m_{t+1})  \right]
                                                    \right\} = 0.  \EQN shop_adjbc
$$


The \idx{Ramsey problem} is to maximize expression \Ep{doc_pref} subject to
equation \Ep{shop_adjbc} and a feasibility constraint that combines
equations \Ep{doc_resource2} through \Ep{doc_shop2}:
$$
1 - \ell_t - H(c_t,\, \hat m_{t+1}) - c_{t} - g_t = 0.    \EQN shop_feasible
$$
Let $\Phi$ and $\{\theta_t\}_{t=0}^\infty$ be a Lagrange multiplier
on equation \Ep{shop_adjbc} and a sequence of Lagrange multipliers on
equation \Ep{shop_feasible}, respectively. First-order conditions for
this problem are
%{\ninepoint
$$\EQNalign{
c_t\rm{:}& \ \  u_c(t) + \Phi \left\{ u_{cc}(t) c_t + u_c(t) \right.  \cr
&
- u_{\ell c}(t)\left[ 1 - \ell_t - (1-\nu) H(c_t,\, \hat m_{t+1})\right]   \cr
& + \left. (1-\nu) u_{\ell}(t) H_c(t) \right\}
                  - \theta_t \left[ H_c(t) +1 \right] = 0,
                                                          \EQN shop_Rfoc;a \cr
\ell_t\rm{:}&\ \ \ u_{\ell}(t) + \Phi \left\{ u_{c \ell}(t) c_t + u_\ell (t)
\right. \cr
& \left.
-\, u_{\ell \ell}(t)\left[ 1 - \ell_t - (1-\nu) H(c_t,\, \hat m_{t+1})\right]
                                    \right\}=-\theta_t,   \EQN shop_Rfoc;b \cr
\hat m_{t+1}\rm{:}&\ \ \ H_{\hat m}(t) \left[ \Phi (1-\nu) u_\ell (t) -
\theta_t \right] =0.
                                              \EQN shop_Rfoc;c \cr}
$$
%}%endninepoint
The first-order condition for real money balances \Ep{shop_Rfoc;c} is satisfied
when either  $H_{\hat m}(t) = 0$ or
$$
\theta_t = \Phi (1-\nu) u_{\ell}(t).                           \EQN shop_false
$$
We now show that equation \Ep{shop_false} cannot be a solution
of the problem. Notice that when $\nu > 1$, equation \Ep{shop_false}
implies that the multipliers $\Phi$ and $\theta_t$ will either
be zero or have opposite signs. Such a solution is excluded because
$\Phi$ is nonnegative, while the insatiable utility function implies
that $\theta_t$ is strictly positive. When $\nu = 1$, a strictly
positive $\theta_t$ also excludes equation \Ep{shop_false} as a
solution. To reject equation \Ep{shop_false} for $\nu \in [0,1)$,
we substitute equation \Ep{shop_false} into equation \Ep{shop_Rfoc;b},
%{\ninepoint
$$
u_\ell (t) + \Phi \left\{ u_{c \ell}(t) c_t + \nu u_\ell (t) -
   u_{\ell \ell}(t)\left[ 1 - \ell_t - (1-\nu) H(c_t,\, \hat m_{t+1})\right]
                                                 \right\}=0,
$$ %}%endninepoint
which is a contradiction because the left side is strictly positive,
given our assumption that $u_{c \ell}(t) \geq 0$. We
conclude that equation \Ep{shop_false} cannot characterize the
solution of the Ramsey
problem when the transaction technology is homogeneous of degree $\nu \geq 0$,
so the solution has to be $H_{\hat m}(t) = 0$. In other words, the social
planner follows the Friedman rule and satiates the economy with real
balances. According to condition \Ep{shop_foc;c}, this aim can be
accomplished with a monetary policy that sustains a zero net nominal
interest rate.


As an illustration of how the Ramsey plan is implemented, suppose that
$g_t=g$ in all periods. Example 1 of chapter \use{optax} presents the Ramsey
plan for this case if there were no transaction technology and no money
in the model. The optimal outcome is characterized by a constant allocation
$(\hat c, \hat n)$ and a constant tax rate $\hat \tau$ that
supports a balanced government budget. We conjecture that the Ramsey
solution to the present monetary economy shares that real
allocation. But how can it do so in the present economy with its
additional constraint in the form of a transaction technology? First, notice
that the preceding Ramsey solution calls for satiating the economy
with real balances, so there will be no time allocated to shopping in
the Ramsey outcome. Second, the real balances needed to satiate the
economy are constant over time and equal to
$$
{M_{t+1} \over p_t} = \psi \hat c,  \hskip1.5cm \forall t \geq 0,  \EQN shop_case1
$$
and the real return on money is equal to the constant real interest rate,
$$
{ p_t \over p_{t+1} } = R,         \hskip1.5cm \forall t \geq 0.  \EQN shop_case2
$$
Third, the real balances in equation \Ep{shop_case1} also equal the
real value of assets acquired by the government in period $0$ from
selling the money supply $M_1$ to the households. These government
assets earn a net real return in each future period equal to
$$
(R-1) \psi \hat c = R \,{M_{t} \over p_{t-1} } - {M_{t+1} \over p_t}
= {p_{t-1} \over p_t } \,{M_{t} \over p_{t-1} } - {M_{t+1} \over p_t}
= {M_{t} - M_{t+1} \over p_t},
$$
where we have invoked equations \Ep{shop_case1} and \Ep{shop_case2}
to show that the interest earnings just equal the funds for retiring
currency from circulation in all future periods needed
 to sustain an equilibrium
in the money market with a zero net nominal interest rate.
It is straightforward to verify that households
would be happy to incur the indebtedness of the initial period. They
use the borrowed funds to acquire money balances and meet
future interest payments by surrendering some of these money balances.
Yet  their real money balances are unchanged over time because of the
falling price level. In this way, money holdings are costless to the
households, and their optimal decisions with respect to consumption
and labor are the same as in the nonmonetary version of this economy.
\index{time consistency}


\section{Time consistency of monetary policy}
The optimality of the Friedman rule was derived in the previous
section under the assumption that the government can commit to a
plan for its future actions. The Ramsey plan is not time
consistent and requires that the government have a technology to
bind itself to it. In each period along the Ramsey plan, the
government is tempted to levy an unannounced inflation tax in
order to reduce future distortionary labor taxes. Rather than
examine this time consistency problem due to distortionary
taxation, we now turn to another time consistency problem arising
from a situation where surprise inflation can reduce unemployment.


\auth{Kydland, Finn E.} \auth{Prescott, Edward C.} \auth{Barro, Robert J.}
\auth{Gordon, David B.}%
Kydland and
Prescott (1977) and Barro and Gordon (1983a, 1983b)
study the time consistency
problem and credible monetary policies in reduced-form models with a
trade-off between surprise inflation and unemployment. In their spirit,
Ireland (1997)
\auth{Ireland, Peter N.}%
proposes a model with microeconomic foundations that gives
rise to such a trade-off because monopolistically competitive firms
set nominal goods prices before the government sets
monetary policy.\NFootnote{Ireland's model takes most of its structure from
those developed by Svensson (1986) and Rotemberg (1987). See
Rotemberg and Woodford (1997) and King and Wolman (1999)
for empirical implementations of related models.
\auth{King, Robert G.}
\auth{Woodford, Michael}
\auth{Wolman, Alexander L.}
\auth{Svensson, Lars E.O.} \auth{Rotemberg, Julio J.}}
 The government is here tempted to create surprise inflation that erodes
firms' markups and stimulates employment above a suboptimally low level.
But any anticipated inflation has negative welfare effects that arise as
a result of a postulated cash-in-advance constraint. More specifically,
anticipated inflation reduces the real value of nominal labor income
that can be spent or invested first in the next period, thereby distorting
incentives to work.


The following setup modifies Ireland's model and assumes that each
household has some market power with respect to its labor supply while a
single good is produced by perfectly competitive firms.


\index{monopolistic competition}
\subsection{Model with monopolistically competitive wage setting}
There is a continuum of households indexed on the unit interval,
$i \in [0,1]$. At time $t$, household $i$ consumes $c_{it}$ of a
single consumption good and supplies labor $n_{it} \geq 0$.\NFootnote{For
analytical simplicity, we assume that the households can supply any
nonnegative amount of labor. When we imposed a finite
time endowment in the first edition of this book, we had to confront
the issue of labor rationing across firms along some equilibrium
paths.}
 The
preferences of the household are
$$
\sum_{t=0}^\infty \beta^t \left( {c_{it}^\gamma \over \gamma} -n_{it}
                                                          \right), \EQN m_pref
$$
where $\beta \in (0,1)$ and $\gamma \in (0,1)$. The parameter restriction on
$\gamma$ ensures that the household's utility is well defined at zero
consumption.


The technology for producing the single consumption good is
$$
y_t \,=\, \left( \int_0^1 n_{it}^{{1-\alpha \over 1+\alpha}} {\rm d}i
    \right)^{{1+\alpha \over 1-\alpha}},
%%%\;=\; \int_0^1 c_t(i) {\rm d}i \equiv C_t,
                                                     \EQN m_tech
$$
where $y_t$ is per capita output and $\alpha \in (0,1)$. The technology has
constant returns to scale in labor inputs, and if all types of labor
are supplied in the same quantity $n_t$, we have $y_t=n_t$. The
marginal product of labor of type $i$ is
$$
{\partial \, y_t \hfill \over \partial \, n_{it} } =
\left( \int_0^1 n_{it}^{{1-\alpha \over 1+\alpha}} {\rm d}i
    \right)^{{2\alpha \over 1-\alpha}} n_{it}^{{-2\alpha \over 1+\alpha}}
\;=\;
\left({ y_t \over n_{it} }\right)^{{2\alpha \over 1+\alpha}}
\equiv \hat w_(y_t, n_{it}).                                 \EQN m_MP
$$
The single good is produced by a large number of competitive firms
that are willing to pay a real wage to labor of type $i$ equal to the
marginal product in equation \Ep{m_MP}.


The definition of the function $\hat w(y_t, n_{it})$ with its two
arguments $y_t$ and $n_{it}$ is motivated by the first of the
following two assumptions on households' labor-supply
behavior.\NFootnote{Analogous assumptions are made implicitly by
Ireland (1997), who takes the aggregate price index as given in the
monopolistically competitive firms' profit maximization problem, and
 disregards
firms' profitability when computing the output effect of a monetary
policy deviation.}
\medskip
\item{1.} When maximizing the rent of its labor supply,
household $i$ perceives that it can affect the marginal product
$\hat w(y_t, n_{it})$ through the second argument, while $y_t$ is
taken as given.
\medskip
\item{2.} The nominal wage for labor of type $i$ at time $t$ is
chosen by household $i$ at the very beginning of period $t$.
Given the nominal wage $w_{it}$, household $i$ is
obliged to deliver any amount of labor $n_{it}$ that is demanded in
the economy.
%%%%%%%%% with feasibility as the sole constraint, $n_{it}\leq 1$.
\index{predetermined wage}


\vskip.3cm


The government's only task is to increase or decrease
the money supply by making lump-sum transfers $(x_t-1)M_t$ to the
households, where $M_t$ is the per capita money supply at the
beginning of period $t$ and $x_t$ is the gross growth rate of
money in period $t$:
$$ M_{t+1} = x_t M_t.                                     \EQN m_msupply
$$
Following Ireland (1997),
we assume that $x_t \in [\beta, \bar x]$. These bounds on money growth
ensure the existence of a monetary equilibrium. The lower bound will be
shown to yield a zero net nominal interest rate in a stationary
equilibrium, whereas the upper bound $\bar x < \infty$
guarantees that households never abandon the use of money altogether.


During each period $t$, events unfold as follows for household $i$:
The household starts
period $t$ with money $m_{it}$ and real private bonds $b_{it}$, and the
household sets the nominal wage $w_{it}$ for its type of labor.
After the wage is determined, the government chooses a nominal transfer
$(x_t-1)M_t$ to be handed over to the household. Thereafter,
the household enters the asset market to settle maturing bonds $b_{it}$
and to pick a new portfolio composition with money and real bonds $b_{i,t+1}$.
After the asset market has closed, the household splits into
a shopper and a worker.\NFootnote{The interpretation that the household splits
into a shopper and a worker follows Lucas's (1980b) cash-in-advance
framework.   It embodies the constraint on transactions
recommended by Clower (1967).
\auth{Clower, Robert}
\auth{Lucas, Robert E., Jr.}}
During period $t$, the shopper purchases $c_{it}$ units of the single good
subject to the \idx{cash-in-advance constraint},
$$
{m_{it} \over p_t} + {(x_t-1) M_t \over p_t} + b_{it} - {b_{i,t+1} \over R_t}
\geq c_{it},                                                \EQN m_cia
$$
where $p_t$ and $R_t$ are the price level and the real interest
rate, respectively. Given the household's predetermined nominal
wage $w_{it}$,
the worker supplies all the labor $n_{it}$ demanded by firms.
At the end of period $t$ when the goods market has closed,
the shopper and the worker reunite, and the household's money holdings
$m_{i,t+1}$ now equal the worker's labor income
$w_{it} n_{it}$ plus any unspent cash from the shopping round. Thus,
the budget constraint of the household becomes\NFootnote{%%As long as the
%%labor supply constraint $n_{it}\leq 1$ is not binding for any $i$,
The assumptions of constant returns
to scale and perfect competition in the goods market imply that profits of
firms are zero.}
%%If any labor supply constraints were strictly binding, labor
%%would have to be rationed among firms, and there would be strictly
%%positive profits.
%%But binding labor supply constraints cannot be part of a perfect foresight
%%equilibrium because households would not be maximizing labor rents.
%%When we later consider monetary surprises, we assume
%%that monetary deviations are never so expansive that labor supply constraints
%%become strictly binding.}
$$
{m_{it} \over p_t} + {(x_t-1) M_t \over p_t} + b_{it} +
{w_{it} \over p_t} n_{it}
= c_{it} + {b_{i,t+1} \over R_t} + {m_{i,t+1} \over p_t}.       \EQN m_bc
$$




\subsection{Perfect foresight equilibrium}
We first study household $i$'s optimization problem under perfect
foresight. Given initial assets $(m_{i0},b_{i0})$ and sequences of
prices $\{p_t\}_{t=0}^\infty$, real interest rates
$\{R_t\}_{t=0}^\infty$, output levels $\{y_t\}_{t=0}^\infty$, and
nominal transfers $\{(x_t-1)M_t\}_{t=0}^\infty$, the household
maximizes expression \Ep{m_pref} by choosing sequences of
consumption $\{c_{it}\}_{t=0}^\infty$, labor supply
$\{n_{it}\}_{t=0}^\infty$, money holdings
$\{m_{i,t+1}\}_{t=0}^\infty$, real bond holdings
$\{b_{i,t+1}\}_{t=0}^\infty$, and nominal wages
$\{w_{it}\}_{t=0}^\infty$ that satisfy cash-in-advance constraints
\Ep{m_cia} and budget constraints \Ep{m_bc}, with the real wage
equaling the marginal product of labor of type $i$ at each point
in time, $w_{it}/p_t = \hat w(y_t,n_{it})$. The last constraint
ensures that the household's choices of $n_{it}$ and $w_{it}$ are
consistent with competitive firms' demand for labor of type $i$.
Let us incorporate this constraint into budget constraint
\Ep{m_bc} by replacing the real wage $w_{it}/p_t$ by the marginal
product $\hat w(y_t, n_{it})$. With $\beta^t \mu_{it}$ and
$\beta^t \lambda_{it}$ as the Lagrange multipliers on the time $t$
cash-in-advance constraint and budget constraint, respectively,
the first-order conditions at an interior solution are
$$\EQNalign{
c_{it}\rm{:}&\ \ \ c_{it}^{\gamma -1} - \mu_{it} - \lambda_{it} = 0,
                                                               \EQN m_foc;a \cr
n_{it}\rm{:}&\ \ \ -1 + \lambda_{it}
\left[{\partial \, \hat w(y_t,n_{it}) \over \partial \, n_{it} \hfill} n_{it}
\,+\, \hat w(y_t,n_{it}) \right] = 0, \hskip.8cm \EQN m_foc;b \cr
m_{i,t+1}\rm{:}&\ \ \ -\lambda_{it} {1 \over p_t}
     + \beta \left(\lambda_{i,t+1} + \mu_{i,t+1} \right)
{1 \over p_{t+1}} =0,  \EQN m_foc;c \cr
b_{i,t+1}\rm{:}&\ \ \ -\left(\lambda_{it} + \mu_{it} \right){1 \over R_t}
                 + \beta \left(\lambda_{i,t+1} + \mu_{i,t+1} \right)=0.
                                                      \EQN m_foc;d \cr}
$$


The first-order condition \Ep{m_foc;b} for the rent-maximizing labor
supply $n_{it}$ can be rearranged to read
$$\EQNalign{
\hat w(y_t,n_{it}) &= { \lambda_{it}^{-1} \over 1+ \epsilon_{it}^{-1}}
                    = {1+ \alpha \over 1- \alpha} \lambda_{it}^{-1} ,
                                                          \EQN m_markup \cr
\noalign{\vskip.3cm} \hbox{\rm where} \hskip.5cm
\epsilon_{it} &= \Bigl[{\partial \, \hat w(y_t,n_{it}) \over \partial \,
n_{it} \hfill}
                {n_{it} \over \hat w(y_t,n_{it})} \Bigr]^{-1}
                 = -{ 1+\alpha \over 2 \alpha} <0.              \cr}
$$
The Lagrange multiplier $\lambda_{it}$ is the shadow value of
relaxing the budget constraint in period $t$ by one unit,
measured in ``utils'' at time $t$.
Since preferences \Ep{m_pref} are linear in the disutility of labor,
$\lambda_{it}^{-1}$ is the value of leisure in period $t$ in
terms of the units of the budget constraint at time $t$. Equation \Ep{m_markup}
is then the familiar expression that the monopoly price $\hat w(y_t,n_{it})$
should be set as a markup above marginal cost $\lambda_{it}^{-1}$,
and the markup is inversely related to the absolute value of the
demand elasticity of labor type $i$,
$|\epsilon_{it}|$. %
%(Compare to equation XXX in chapter 12.)


First-order conditions \Ep{m_foc;c} and \Ep{m_foc;d} for asset decisions can
be used to solve for rates of return,
$$\EQNalign{
{p_t \over p_{t+1}} &= {\lambda_{it} \over
                        \beta \left(\lambda_{i,t+1} + \mu_{i,t+1} \right)},
                                                  \EQN m_asset;a \cr
\noalign{\smallskip}
R_t &= { \lambda_{it} + \mu_{it} \over
                        \beta \left(\lambda_{i,t+1} + \mu_{i,t+1} \right)}.
                                                  \EQN m_asset;b \cr}
$$
Whenever the Lagrange multiplier $\mu_{it}$ on the cash-in-advance constraint
is strictly positive, money has a lower rate of return than bonds, or,
equivalently, the net nominal interest rate is strictly positive, as shown
in equation \Ep{doc_arbitrage}.


Given initial conditions $m_{i0}=M_0$ and $b_{i0}=0$, we now turn to
characterizing an equilibrium under the additional assumption that the
cash-in-advance constraint \Ep{m_cia} holds with equality, even when
it does not bind.
Since all households are perfectly symmetric, they will make
identical consumption and labor decisions,  $c_{it}=c_t$ and $n_{it}=n_t$,
so by goods market clearing and the constant-returns-to-scale technology
\Ep{m_tech}, we have
$$
c_t = y_t = n_t,                                               \EQN m_eq;a
$$
and from the expression for the marginal product of labor
in equation \Ep{m_MP},
$$
\hat w(y_t,n_t) = 1.                                         \EQN m_eq;b
$$
Equilibrium asset holdings
satisfy $m_{i,t+1}=M_{t+1}$ and $b_{i,t+1}=0$. The substitution of equilibrium
quantities into the cash-in-advance constraint \Ep{m_cia} at equality yields
$$
{M_{t+1} \over p_t} = c_{t},                                  \EQN m_eq;c
$$
where a version of the ``quantity theory of money''
\index{quantity theory of money}%
 determines the price level, $p_t=M_{t+1}/c_t$.  We now
substitute this expression and conditions \Ep{m_foc;a} and \Ep{m_markup}
into equation \Ep{m_asset;a}:
$$
{M_{t+1}/c_t \over M_{t+2}/c_{t+1}} =
  { \displaystyle \Bigl[{1- \alpha \over 1+ \alpha} \,
\hat w(y_t,n_{t})\Bigr]^{-1}
        \over \beta \, c_{t+1}^{\gamma -1} },
$$
which can be rearranged to read
$$
c_t = {1- \alpha \over 1+ \alpha}\, {\beta \over x_{t+1}}\, c_{t+1}^{\gamma},
$$
where we have used equations \Ep{m_msupply} and \Ep{m_eq;b}.
After taking the logarithm of this expression, we get

$$
\log(c_t) = \log\left({1- \alpha \over 1+ \alpha}\,\beta\right)
            + \gamma \log(c_{t+1}) - \log(x_{t+1}).
$$
Since $0<\gamma<1$ and $x_{t+1}$ is bounded, this linear difference equation
in $\log(c_t)$ can be solved forward to obtain
$$
\log(c_t) = { \log\left( {\displaystyle 1-\alpha \over
\displaystyle 1+\alpha}\,
                          \beta \right) \over 1-\gamma }
          - \sum_{j=0}^{\infty} \gamma^j \log(x_{t+1+j}),    \EQN m_difflog
$$
where equilibrium considerations have prompted us to choose the particular
solution that yields a bounded sequence.\NFootnote{See the appendix
to chapter \use{timeseries}
 for the solution of scalar linear difference equations.}
 \index{Ramsey plan}


\subsection{Ramsey plan}
The Ramsey problem is to choose a sequence of monetary growth rates
$\{x_t\}_{t=0}^\infty$
that supports the perfect foresight equilibrium with the highest possible
welfare; that is, the optimal choice of $\{x_t\}_{t=0}^\infty$
maximizes the representative household's utility in expression \Ep{m_pref}
subject to expression \Ep{m_difflog} and $n_t=c_t$.
From the expression \Ep{m_difflog} it is apparent that
the constraints on money growth, $x_t \in [\beta, \bar x]$, translate into
 lower and  upper bounds on consumption,
$c_t \in [\underline c, \bar c]$, where
$$
\underline c = \left({\beta \over \bar x} \, {1-\alpha \over 1+\alpha}
                            \right)^{1 \over 1-\gamma}, \hskip1cm
\hbox{\rm and} \hskip1cm
\bar c = \left( {1-\alpha \over 1+\alpha}
                          \right)^{1 \over 1-\gamma} \;<\;1.   \EQN m_bounds
$$
The Ramsey plan then follows directly from inspecting the one-period return
of the Ramsey optimization problem,
$$
 {c_{t}^{\gamma} \over \gamma } -c_{t},                        \EQN m_oneperiod
$$
which is strictly concave and reaches a maximum at $c=1$. Thus,
the Ramsey solution calls for $x_{t+1}=\beta$ for $t\geq 0$
in order to support $c_t=\bar c$ for $t\geq 0$. Notice that the Ramsey
outcome can be supported by any initial money growth $x_0$. It is only future
money growth rates that must be equal to $\beta$ in order to eliminate labor
supply distortions that would otherwise arise from the cash-in-advance
constraint if the return on money were to fall short of the return
on bonds. The Ramsey outcome equalizes the returns on money and
bonds; that is, it implements the \idx{Friedman rule} with a zero net
nominal interest rate.


It is instructive to highlight the inability of the Ramsey monetary policy
to remove the distortions coming from monopolistic wage setting. Using the
fact that the equilibrium real wage is unity, we solve for $\lambda_{it}$ from
equation \Ep{m_markup} and substitute into equation \Ep{m_foc;a},
$$
c_{it}^{\gamma -1} = \mu_{it} + {1+ \alpha \over 1- \alpha} > 1.   \EQN m_distort
$$
The left side of equation \Ep{m_distort} is the marginal utility of
consumption.
Since technology \Ep{m_tech} is linear in labor, the marginal utility of
consumption should equal the marginal utility of leisure in a first-best
allocation.  But the right side of equation \Ep{m_distort} exceeds unity,
which is the marginal utility of leisure given preferences \Ep{m_pref}.
While the Ramsey monetary policy succeeds in removing distortions
from the cash-in-advance constraint by setting the
Lagrange multiplier $\mu_{it}$ equal to zero, the policy cannot
undo the distortion of monopolistic wage setting manifested in the
``markup'' $(1+ \alpha)/(1- \alpha)$.\NFootnote{The government would need to use
fiscal instruments, that is, subsidies and taxation, to correct the distortion
from monopolistically competitive wage setting.} Notice that the Ramsey
solution converges to the first-best allocation when the parameter
$\alpha$ goes to zero, that is, when households' market power goes to zero.


To illustrate the time consistency problem, we now solve for the Ramsey plan
when the initial nominal wages are taken as given,
$w_{i0}=w_0 \in [\beta M_0, \bar x M_0]$.
First, setting the initial period $0$ aside,
it is straightforward to show that the solution for $t\geq 1$ is
the same as before. That is, the optimal policy calls for $x_{t+1}=\beta$ for
$t\geq 1$ in order to support $c_t=\bar c$ for $t\geq 1$. Second, given $w_0$,
the first-best outcome $c_0=1$ can be attained in the initial period by
choosing $x_0=w_0/M_0$. The resulting money supply $M_1=w_0$ will then serve to
transact $c_0=1$ at the equilibrium price $p_0=w_0$. Specifically,
firms are happy to hire any number of workers at the wage $w_0$ when
the price of the good is $p_0=w_0$. At the price $p_0=w_0$, the goods
market clears at full employment,
since shoppers seek to spend their real balances $M_1/p_0=1$. The labor
market also clears because workers are obliged to deliver the demanded
$n_0=1$. Finally, money growth $x_1$ can be chosen freely and
does not affect the real allocation of the Ramsey solution. The reason is
that, because of the preset wage $w_0$, there cannot be any labor
supply distortions at time $0$ arising from a low return
on money holdings between periods $0$ and $1$.
\index{Friedman rule!credibility}
\subsection{Credibility of the Friedman rule}
Our comparison of the Ramsey equilibria with or without a preset
initial wage $w_0$ hints at the government's temptation to create
positive monetary surprises that will increase employment. We now
ask if the Friedman rule is credible when the government lacks the
commitment technology implicit in the Ramsey optimization problem.
Can the Friedman rule be supported with a trigger strategy where a
government deviation causes the economy to revert to the worst
possible subgame perfect equilibrium?


Using the concepts and notation of chapter \use{credible},
we specify the objects of a strategy
profile and state the definition of a \idx{subgame perfect equilibrium} (SPE).
Even though households possess market power with respect to their
labor type, they remain atomistic vis-\`a-vis the government. We
therefore stay within the framework of chapter \use{credible}
 where the government
behaves strategically, and the households' behavior can now be summarized as
a ``monopolistically competitive equilibrium'' that responds nonstrategically
to the government's choices. At every date $t$ for all possible histories,
a strategy of the households $\sigma^h$ and a strategy of the government
$\sigma^g$ specify actions $\tilde w_t\in \tilde W$ and
$x_t \in X \equiv [\beta, \bar x]$, respectively, where
$$
\tilde w_t = {w_t \over M_t},  \hskip1cm \hbox{\rm and} \hskip1cm
x_t        = {M_{t+1} \over M_t}.
$$
That is, the actions multiplied by the beginning-of-period money supply $M_t$
produce a nominal wage and a nominal money supply. (This scaling of nominal
variables is used by Ireland, 1997, throughout his analysis, since the
size of the nominal money supply at the beginning of a period has no
significance
{\it per se}.)

\smallskip
\noindent{\sc Definition:}  A strategy profile $\sigma = (\sigma^h,\sigma^g)$
is a {\it subgame perfect equilibrium\/} if, for each
$t\geq 0$ and each history $(\tilde w^{t-1},\, x^{t-1})
\in \tilde W^t\times X^t$,\medskip
\noindent
\noindent{(1)}  Given the trajectory of money growth rates
$\{x_{t-1+j}= %%\hfil\break
 x(\sigma\vert_{(\tilde w^{t-1},x^{t-1})})_j \}_{j=1}^\infty$,
the wage-setting outcome $\tilde w_t =  %%\hfil\break
\sigma^h_t\, (\tilde w^{t-1},x^{t-1})$
constitutes a monopolistically competitive equilibrium.
\smallskip
\noindent{(2)}  The government cannot strictly improve the households'
welfare by deviating from $x_t = \sigma^g_t\, (\tilde w^{t-1},x^{t-1})$,
that is, by choosing some other money growth rate $\eta \in X$ with
the implied continuation strategy profile
$\sigma\vert_{(\tilde w^t; x^{t-1},\eta)}$.

\smallskip
\medskip\noindent
Besides changing to a ``monopolistically competitive equilibrium,'' the
main difference from Definition 6 of chapter \use{credible}
 lies in requirement (1).
The equilibrium in period $t$ can no longer be stated in terms of an isolated
government action at time $t$ but requires the trajectory of the current and
all future money growth rates, generated by the strategy profile
$\sigma\vert_{(\tilde w^{t-1},x^{t-1})}$. The monopolistically competitive
equilibrium in requirement (1) is understood to be the perfect
foresight equilibrium described previously.  When the government
is contemplating a deviation in requirement (2),
the equilibrium is constructed as follows: In period $t$
when the deviation takes place, equilibrium consumption $c_t$
is a function of $\eta$ and $\tilde w_t$
as implied by the cash-in-advance constraint at equality,
$$
c_t = {\eta M_t \over p_t} =  {\eta M_t \over w_t}
    = {\eta \over \tilde w_t},          \EQN m_deviate
$$
%%$$
%%c_t = {\eta M_t \over p_t} = \min\left\{ {\eta M_t \over w_t},\,1\right\}
%%    = \min\left\{ {\eta \over \tilde w_t},\,1\right\},         \EQN m_deviate
%%$$
where we use the equilibrium condition $p_t = w_t$.
%%$p_t\geq w_t$ that holds with strict
%%equality unless labor is rationed among firms at full employment.\NFootnote{Notice
%%that all $\eta \geq \tilde w_t$ yield full employment.
%%Under the assumption that firm profits are evenly distributed among households,
%%it also follows that all $\eta \geq \tilde w_t$ share the same welfare
%%implications.
%%Without loss of generality, we can therefore restrict attention to choices of
%%$\eta$ that are no larger than $\tilde w_t$, that is, the assumption
%%referred to previously stating that monetary deviations are never so expansive
%%that labor supply constraints become strictly binding.}
Starting in period $t+1$, the deviation has triggered a switch to a new
perfect foresight equilibrium with a trajectory of money growth rates given by
$\{x_{t+j}=x(\sigma\vert_{(\tilde w^t;x^{t-1},\eta)})_j\}_{j=1}^\infty$.


We conjecture that the worst SPE has $c_t=\underline c$ for all
periods, and the candidate strategy profile $\hat \sigma$ is
$$\eqalign{
\hat \sigma^h_t &= {\bar x \over \underline c}
              \qquad \forall\ t\ , \quad \forall\ (\tilde w^{t-1}, x^{t-1});\cr
\hat \sigma^g_t &= \bar x
              \qquad \forall\ t\ ,\quad \forall\ (\tilde w^{t-1}, x^{t-1}).\cr}$$
The strategy profile instructs the government to choose the highest
permissible money growth rate $\bar x$ for all periods and for all
histories. Similarly, the households are instructed to set the nominal
wages that would constitute a perfect foresight equilibrium when
money growth will always be at its maximum.
Thus, requirement (1) of an SPE is clearly satisfied. It remains to
show that the government has no incentive to deviate. Since
the continuation strategy profile is $\hat \sigma$ regardless
of the history, the government needs only to find the best
response in terms of the one-period return \Ep{m_oneperiod}.
After substituting the household's action
$\tilde w_t = \bar x/ \underline c$ into equation \Ep{m_deviate}, we get
$c_t=\underline c \eta / \bar x$, so the best response of the government
is to follow the proposed strategy $\bar x$. We conclude that the strategy
profile $\hat \sigma$ is indeed an SPE, and it is the worst,
since $\underline c$ is the lower bound on consumption in any
perfect foresight equilibrium.


We are now ready to address the credibility of the Friedman rule. The best
chance for the Friedman rule to be credible is if a deviation
triggers a reversion to the worst possible subgame perfect equilibrium
given by $\hat \sigma$. The condition for credibility becomes
$$
{ {\displaystyle \bar c^{\gamma} \over \displaystyle \gamma } - \bar c \over 1-\beta }
\;\geq \; \left( {1 \over \gamma} -1\right) + \beta \,
{ {\displaystyle \underline c^{\gamma} \over \displaystyle \gamma }
                         - \underline c \over 1-\beta } .          \EQN m_Ramseycred1
$$
By following the Friedman rule, the government removes the labor supply
distortion coming from a binding cash-in-advance constraint and keeps
output at $\bar c$. By deviating from the Friedman rule, the government
creates a positive monetary surprise that increases output to its efficient
level of unity, thereby eliminating the distortion caused by monopolistically
competitive wage setting as well. However, this deviation destroys the
government's reputation, and the economy reverts to an equilibrium
that induces the government to inflate at the highest possible rate
thereafter, and output falls to $\underline c$. Hence, the Friedman
rule is credible if and only if equation \Ep{m_Ramseycred1} holds.


The Friedman rule is the more likely to be credible,
the higher is the exogenous upper bound on money growth $\bar x$,
since $\underline c$ depends negatively on $\bar x$. In other words,
a higher $\bar x$ translates into a larger penalty for deviating,
so the government becomes more willing to adhere to the Friedman
rule to avoid this penalty. In the limit when $\bar x$ becomes
arbitrarily large, $\underline c$ approaches zero and condition
\Ep{m_Ramseycred1} reduces to
$$
\left({1-\alpha \over 1+\alpha}\right)^{\gamma \over 1-\gamma}
\left( {1 \over \gamma} - {1-\alpha \over 1+\alpha} \right)
\geq (1-\beta) \left( {1 \over \gamma} -1\right),
$$
where we have used the expression for $\bar c$ in equations \Ep{m_bounds}.
The Friedman rule can be sustained for a
sufficiently large value of $\beta$. The government has less incentive
to deviate when households are patient and put a high weight on future
outcomes. Moreover, the Friedman rule is credible for
a sufficiently small value of $\alpha$, which is equivalent to
households having little market power. The associated small distortion from
monopolistically competitive wage setting means that the potential welfare gain
of a monetary surprise is also small, so the government is less
tempted to deviate from the Friedman rule.




\section{Concluding remarks}
Besides shedding light on a number of monetary doctrines, this chapter
has brought out the special importance of the initial date $t=0$
in the analysis. This point is especially pronounced in Woodford's (1995)
model where the initial interest-bearing government debt $B_0$ is
not indexed but rather denominated in nominal terms. So, although
the construction
of a perfect foresight equilibrium ensures that all future issues of nominal
bonds will {\it ex post} yield the real rates of return that are needed to
entice the households to hold these bonds, the realized real return on
the initial nominal bonds can be anything, depending on the price level
$p_0$. Activities at the initial date were also
important when we considered dynamic optimal taxation in chapter
\use{optax}.
 %We will then revisit the question of an optimum quantity of money,
%by asking whether or
%not real money balances should be subject to an inflation tax when there
%are only distortionary taxes available for financing government
%expenditures. The question will be addressed in both a version of the
%shopping-time model of this chapter and a cash-in-advance model with
%cash and credit goods.


Monetary issues are also discussed in other chapters of the book.
Chapters \use{ogmodels} and \use{incomplete}
 study money in overlapping generations models and
Bewley models, respectively.
Chapters \use{townsend} and \use{search2}
 present other explicit environments that give
rise to a positive value of fiat money: Townsend's turnpike model and
the Kiyotaki-Wright search model.


%\section{Exercises}
\showchaptIDfalse
\showsectIDfalse
\section{Exercises}
\showchaptIDtrue
\showsectIDtrue
\medskip
\medskip\noindent
{\it Exercise \the\chapternum.1} \quad  {\bf Why deficits in Italy and Brazil
were once extraordinary proportions of GDP}
\index{Italy and Brazil!inflation and measured deficits}
\medskip\noindent
%Consider a version of our model with money in the utility
%function.
The government's budget constraint can be written as
$$ g_t - \tau_t +{b_t \over R_{t-1}} (R_{t-1} -1)
   = {b_{t+1} \over R_t} - {b_t \over R_{t-1}} +
      {M_{t+1} \over p_t } - {M_t \over p_{t}}  . \eqno(1)$$
The left side is the real gross-of-interest government
deficit; the right side is change in the real value of government
liabilities between $t-1$ and $t$.


  Government budgets often report the {\it nominal\/}
gross-of-interest government deficit, defined as
$$ p_t (g_t - \tau_t) + p_t b_t \left(1 - {1 \over R_{t-1} p_t /p_{t-1}}
\right),$$
and their ratio to nominal GNP, $p_t y_t$, namely,
$$ \left[(g_t - \tau_t) + b_t \Bigl(1 - {1 \over R_{t-1} p_t / p_{t-1}}\Bigr)
\right] \, / y_t.$$
For countries with a large $b_t$ (e.g., Italy),
this number can be very big even
with a moderate rate of inflation.   For countries
with a rapid inflation rate, like Brazil in 1993, this
number  sometimes comes in at 30 percent of GDP.
Fortunately, this number overstates the magnitude of the
government's ``deficit problem,'' and there is a simple
adjustment to the interest component of the deficit that
renders a more accurate picture of the problem.  In particular,
notice that the real values of the interest component of the
real and  nominal deficits are related by
$$ b_t \left(1 - {1 \over R_{t-1}}\right) = \alpha_t b_t \left(1 -
         {1 \over R_{t-1} p_t / p_{t-1}}\right),$$
where
$$ \alpha_t = {R_{t-1} -1 \over R_{t-1} - p_{t-1}/p_t}.$$
Thus, we should multiply the real value of nominal
interest payments $b_t [1 - p_{t-1} / (R_{t-1} p_t) ]$
by $\alpha_t$ to get the real interest component of the
debt that appears on the left side of equation (1).
\medskip
\noindent
{\bf a.}
   Compute $\alpha_t$ for a country that has a $b_t /y$ ratio
of .5, a gross real interest rate of 1.02, and a zero net inflation
rate.

\medskip\noindent
{\bf b.}
Compute $\alpha$ for a country that has a $b_t/y$ ratio of .5, a gross
real interest rate of 1.02, and a 100 percent per year net inflation rate.


\medskip\noindent
{\it Exercise \the\chapternum.2}\quad  {\bf A strange example of Brock (1974)}
\auth{Brock, William A.}


\medskip\noindent
  Consider an economy consisting of a government
and a representative household.  There is one consumption
good, which is not produced and not storable.  The exogenous
supply of the good at time $t \geq 0$ is $y_t = y > 0$.  The household
owns the good.  At time $t$ the representative household's
preferences are ordered by
$$ \sum_{t=0}^\infty \beta^t \{ \ln c_t + \gamma \ln (m_{t+1}/p_t) \},
\eqno(1)$$    %\EQN old1$$
where $c_t$ is the household's consumption at $t$, $p_t$ is the
price level at $t$, and $m_{t+1}/p_t$ is the real balances that
the household carries over from time $t$ to $t+1$.  Assume that
$\beta \in (0,1)$ and $\gamma > 0$.  The household maximizes
equation (1)  %\Ep{old1}
over choices of $\{c_t, m_{t+1} \}$ subject
to the sequence of budget constraints
$$ c_t + m_{t+1}/p_t = y_t
- \tau_t + m_t/p_t, \quad t \geq 0, \eqno(2)$$  %\EQN old2$$
where $\tau_t$ is a lump-sum tax due at $t$.  The household
faces the price sequence $\{p_t\}$ as a price taker and
has given initial value of nominal balances $m_0$.


   At time $t$ the government faces the budget constraint
$$ g_t = \tau_t + (M_{t+1} - M_t)/p_t,\quad t \geq 0, \eqno(3)$$  %\EQN old3  $$
where $M_t$ is the amount of currency that the government
has outstanding at the beginning of time $t$ and $g_t$ is
government expenditures at time $t$.  In equilibrium, we
require that $M_t = m_t$ for all $t \geq 0$.
  The government chooses sequences of $\{g_t, \tau_t,
   M_{t+1} \}_{t=0}^\infty $
subject to the budget constraints (3)  %\Ep{old3}
being satisfied for
all $t \geq 0$ and subject to the given initial value
$M_0 = m_0$.


\medskip
\noindent
{\bf a.}  Define a {\it competitive equilibrium}.


\medskip
For the remainder of this problem assume that $g_t = g < y$ for
all $t \geq 0$, and that $\tau_t = \tau$ for all $t \geq 0$.
Define a {\it stationary equilibrium\/} as an equilibrium
in which the rate of return on currency is constant for all
$t \geq 0$.
\medskip
\noindent{\bf b.}  Find conditions under which there exists
a  stationary equilibrium for which $p_t > 0$ for all
$t \geq 0$. Derive formulas for real balances and the
rate of return on currency in that equilibrium, given
that it exists.  Is the stationary equilibrium unique?
\medskip
\noindent{\bf c.}  Find a first-order difference equation
in the equilibrium level of real balances $h_t = M_{t+1}/p_t$
whose satisfaction ensures equilibrium (possibly nonstationary).
\medskip
\noindent{\bf d.}  Show that there is a fixed point of
this difference equation with positive real balances, provided
that the condition that you derived in part b is satisfied.
Show that this fixed point agrees with the level of real balances
that you computed in part b.
%\medskip
%\noindent{\bf e.}  Under what conditions is the
%following statement true:  If there exists a stationary equilibrium,
%then there also exist many other nonstationary equilibria.  Describe
%these other equilibria.  In particular, what is happening to real
%balances and the price level in these other equilibria?  Among
%these other equilibria, within which one(s) are consumers better off?
%
%
%\medskip
%\noindent{\bf f.}  Within which of the equilibria that you found in parts b
%and e is the following ``old-time religion'' true:  ``Larger
%sustained government deficits imply permanently larger inflation rates''?
%

\medskip\noindent
{\it Exercise \the\chapternum.3} \quad  {\bf Optimal inflation tax in a
cash-in-advance model}


\medskip\noindent
Consider the version of Ireland's (1997) model described in the text,
but assume perfect competition (i.e., $\alpha=0$) with flexible
market-clearing wages.  Suppose now that the government must finance
a constant amount of purchases $g$ in each period by levying flat-rate
labor taxes and raising seigniorage. Solve the optimal taxation problem
under commitment.

%\vfil\eject
\medskip


\noindent
{\it Exercise \the\chapternum.4} \quad {\bf Deficits, inflation, and
anticipated monetary shocks}, donated by Rodolfo Manuelli
\medskip\noindent
Consider an economy populated by a large number of identical individuals.
Preferences over consumption and leisure are given by
$$ \sum_{t=0}^\infty \beta^t c_t^\alpha \ell_t^{1-\alpha}, $$
where $0 < \alpha < 1$.  Assume that leisure is positively related --
this is just a reduced form of a shopping-time model -- to the
stock of real money balances, and negatively related to a measure
of transactions:
$$ \ell_t = A(m_{t+1}/p_t)/c_t^\eta ,\quad A > 0, $$
and $\alpha-\eta(1-\alpha) > 0$.  Each individual owns a tree that
drops $y$ units of consumption per period (dividends).  There is
a government that issues one-period real bonds, money, and collects
taxes (lump-sum) to finance spending.  Per capita spending is equal
to $g$.  Thus, consumption equals $c=y-g$.  The government's budget
constraint is:
$$ g_t + B_t = \tau_t + B_{t+1}/R_t + (M_{t+1}-M_t)/p_t.  $$
Let the rate of return on money be $R_{mt} = p_t/p_{t+1}$.  Let the
nominal interest rate at time $t$ be $1+i_t = R_t p_{t+1}/p_t = R_t
\pi_t$.
\medskip


\noindent{\bf a.}  Derive the demand for money, and show that it decreases
with the nominal interest rate.
\medskip


\noindent{\bf b.}  Suppose that the government policy is such that $g_t = g$, $B_t = B$
and $\tau_t = \tau$.  Prove that the real interest rate, $R$, is constant
and equal to the inverse of the discount factor.
\medskip


\noindent{\bf c.}  Define the deficit as $d$, where $d = g + (B/R)(R-1)-\tau$.
What is the highest possible deficit that can be financed in this economy?
An economist claims that increases in $d$, which leave
$g$ unchanged, will result in increases in the inflation rate.
Discuss this view.
\medskip


\noindent{\bf d.}  Suppose that the economy is open to international capital
flows and that the world interest rate is $R^* = \beta^{-1}$.  Assume
that $d = 0$, and that $M_t = M$.  At $t=T$, the government increases
the money supply to $M' = (1+\mu) M$.  This increase in the money supply
is used to purchase (government) bonds.  This, of course, results in
a smaller deficit at $t > T$.  (In this case, it will result in
a surplus.)  However, the government also announces its intention to
cut taxes (starting at $T+1$) to bring the deficit back to zero.  Argue
that this open market operation will have the effect of increasing prices
at $t=T$ by $\mu$; %.  That is, prices at $t=T, p'$, will satisfy
$p'=(1+\mu)p$, where $p$ is the price level from $t=0$ to $t=T -1$.
\medskip


\noindent{\bf e.}  Consider the same setting as in d.  Suppose now that
the open market operation is announced at $t=0$ (it still takes
place at $t=T$).  Argue that prices will increase at $t=0$ and,
in particular, that the rate of inflation between $T-1$ and $T$
will be less than $1+\mu$.

\vfil\eject
\medskip
\noindent
{\it Exercise \the\chapternum.5}\quad  {\bf Interest elasticity of the
 demand for money}, donated by Rodolfo Manuelli
\medskip\noindent
Consider an economy in which the demand for money satisfies
$$ m_{t+1}/p_t = F(c_t, R_{mt}/R_t), $$
where $R_{mt} = p_t/p_{t+1}$ and $R_t$ is the one-period interest rate.
Consider the following open market operation: At $t=0$, the government
sells bonds and ``destroys'' the money it receives in exchange for those bonds.
No other real variables, e.g., government spending or taxes, are changed.
Find conditions on the income elasticity of the demand for money such
that the decrease in money balances at $t=0$ results in an increase in
the price level at $t=0$.


\medskip\noindent
{\it Exercise \the\chapternum.6} \quad  {\bf Dollarization}, donated by Rodolfo
Manuelli
\medskip\noindent
In recent years, several countries, e.g., Argentina and countries
hit by the Asian crisis, have considered the possibility of
giving up their currencies in favor of the U.S. dollar.  Consider a country,
say $A$, with deficit $d$ and inflation rate $\pi = 1/R_m$.  Output and
consumption are constant, and hence the real interest rate is fixed,
with $R = \beta^{-1}$.  The (gross-of-interest-payments) deficit is $d$, with
$$ d = g - \tau + (B/R)(R-1).  $$
Let the demand for money be $m_{t+1}/p_t = F(c_t, R_{mt}/R_t)$, and assume
that $c_t = y - g$.  Thus, the steady-state government budget constraint is
$$ d = F(y-g, \beta R_m)(1-R_m) > 0.  $$
Assume that the country is considering, at $t=0$, the retirement of its money
in exchange for dollars.  The government promises to give to each person who
brings a ``peso''
 to the Central Bank $1/e$ dollars, where $e$ is the exchange
rate (in pesos per dollar) between the country's currency and the U.S. dollar.
Assume that the U.S. inflation rate (before and after the switch) is given
and equal to $\pi^* = 1/R_m^* < \pi$, and that the country is on the ``good''
part of the Laffer curve.
\medskip


\noindent{\bf a.} If you are advising the government of $A$, how much would you
say that it should demand from the U.S. government to make the switch?  Why?
\medskip


\noindent{\bf b.} After the dollarization takes place, the government understands
that it needs to raise taxes.  Economist 1 argues that the increase in taxes
(on a per period basis) will equal the loss of revenue from
inflation -- $F(y-g, \beta R_m)(1-R_m)$ -- while Economist 2 claims that this is an
overestimate.  More precisely, he or she claims that if the government is a
good negotiator vis-\`a-vis the U.S. government, taxes need only increase by
$F(y-g, \beta R_m)(1-R_m) - F(y-g, \beta R_m^*)(1 -R_m^*)$
per period.  Discuss
these two views.


\medskip\noindent
{\it Exercise \the\chapternum.7}\quad  {\bf Currency boards}, donated by Rodolfo Manuelli
\medskip\noindent
In the last few years, several countries, e.g., Argentina (1991),
Estonia (1992), Lithuania (1994), Bosnia (1997) and Bulgaria (1997),
  have adopted the currency board model of monetary policy.
  In a nutshell, a currency board is a
commitment on the part of the country to fully back its domestic
currency with foreign-denominated assets.  For simplicity,
assume that the foreign asset is the U.S. dollar.
\medskip
The government's budget constraint is given by
$$ g_t + B_t + B^*_{t+1}e/(Rp_t) = \tau_t + B_{t+1}/R + B^*_te/p_t +
(M_{t+1} - M_t)/p_t,  $$
where $B^*_t$ is the stock of one-period bonds, denominated in dollars,
held by this country, $e$ is the exchange rate (pesos per dollar), and
$1/R$ is the price of one-period bonds (both domestic and dollar denominated).
  Note that the budget constraint equates the real value of income and
liabilities in units of consumption goods.
\medskip
The currency board ``contract'' requires that the money supply be fully backed.
One interpretation of this rule is that the domestic money supply is
$$ M_t = eB^*_t.  $$
Thus, the right side is the local currency value of foreign
reserves (in bonds) held by the government, while the left side
is the stock of money.  Finally, let the law of one price hold:
$p_t = ep^*_t$, where $p^*_t$ is the foreign (U.S.) price level.
\medskip


\noindent{\bf a.} Assume that $B_t = B$, and that foreign inflation is zero, $p^*_t =
 p^*$.  Show that even in this case, the properties of the demand for
money -- which you may take to be given by $F(y-g, \beta R_m)$ --
 are important in determining total revenue.  In particular,
 explain how a permanent increase in $y$,
income per capita, allows the government to lower taxes (permanently).
\medskip


\noindent{\bf b.} Assume that $B_t = B$.  Let foreign inflation be positive,
that is,
 $\pi^* > 1$.  In this case, the price in dollars of a one-period
dollar-denominated bond is $1/(R\pi^*)$.  Go as far as you can describing
the impact of foreign inflation on domestic inflation, and on per
capita taxes, $\tau$.
\medskip


\noindent{\bf c.} Assume that $B_t = B$.  Go as far as you can describing the
effects of a once-and-for-all surprise devaluation, i.e., an unexpected and
permanent increase in $e$, on the level of per capita taxes.
\medskip


\noindent{\it Exercise \the\chapternum.8} \quad {\bf Growth and inflation}, donated
by Rodolfo Manuelli
\medskip\noindent
Consider an economy populated by identical individuals with instantaneous
utility function given  by
$$ u(c,\ell) = [c^\varphi \ell^{1-\varphi}]^{(1-\sigma)}/(1-\sigma).  $$
Assume that shopping time is given by  $s_t = \psi c_t/(m_{t+1}/p_t)$.
Assume that in this economy, income grows exogenously at the rate $\gamma > 1$.
Thus, at time $t$, $y_t = \gamma^t y$.  Assume that government spending
also grows at the same rate, %%.  This implies that
$g_t = \gamma^t g$.
Finally, $c_t = y_t - g_t$.
\medskip


\noindent{\bf a.} Show that for this specification, if the demand for money at
$t$ is $x = m_{t+1}/p_t$, then the demand at $t+1$ is $\gamma x$.  Thus,
the demand for money grows at the same rate as the economy.
\medskip


\noindent{\bf b.} Show that the real rate of interest depends on the growth rate.
(You may assume that $\ell$ is constant for this calculation.)
\medskip


\noindent{\bf c.} Argue that even for monetary policies that keep the price
level constant, that is, $p_t = p$ for all $t$, the government raises
positive amounts of revenue from printing money. Explain.

\medskip


\noindent{\bf d.} Use your finding in c to discuss why, following monetary
reforms that generate big growth spurts, many countries manage to
``monetize'' their economies (this is just jargon for increases in
the money supply) without generating inflation.
